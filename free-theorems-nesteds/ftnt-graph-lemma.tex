% For double-blind review submission, w/o CCS and ACM Reference (max
% submission space)
\documentclass[acmsmall,review,anonymous]{acmart}
\settopmatter{printfolios=true,printccs=false,printacmref=false}
%% For double-blind review submission, w/ CCS and ACM Reference
%\documentclass[acmsmall,review,anonymous]{acmart}\settopmatter{printfolios=true}
%% For single-blind review submission, w/o CCS and ACM Reference (max submission space)
%\documentclass[acmsmall,review]{acmart}\settopmatter{printfolios=true,printccs=false,printacmref=false}
%% For single-blind review submission, w/ CCS and ACM Reference
%\documentclass[acmsmall,review]{acmart}\settopmatter{printfolios=true}
%% For final camera-ready submission, w/ required CCS and ACM Reference
%\documentclass[acmsmall]{acmart}\settopmatter{}


%% Journal information
%% Supplied to authors by publisher for camera-ready submission;
%% use defaults for review submission.
\acmJournal{PACMPL}
\acmVolume{1}
\acmNumber{POPL} % CONF = POPL or ICFP or OOPSLA
\acmArticle{1}
\acmYear{2020}
\acmMonth{1}
\acmDOI{} % \acmDOI{10.1145/nnnnnnn.nnnnnnn}
\startPage{1}

%% Copyright information
%% Supplied to authors (based on authors' rights management selection;
%% see authors.acm.org) by publisher for camera-ready submission;
%% use 'none' for review submission.
\setcopyright{none}
%\setcopyright{acmcopyright}
%\setcopyright{acmlicensed}
%\setcopyright{rightsretained}
%\copyrightyear{2018}           %% If different from \acmYear

%% Bibliography style
\bibliographystyle{ACM-Reference-Format}
%% Citation style
%% Note: author/year citations are required for papers published as an
%% issue of PACMPL.
\citestyle{acmauthoryear}   %% For author/year citations
%\citestyle{acmnumeric}

%%%%%%%%%%%%%%%%%%%%%%%%%%%%%%%%%%%%%%%%%%%%%%%%%%%%%%%%%%%%%%%%%%%%%%
%% Note: Authors migrating a paper from PACMPL format to traditional
%% SIGPLAN proceedings format must update the '\documentclass' and
%% topmatter commands above; see 'acmart-sigplanproc-template.tex'.
%%%%%%%%%%%%%%%%%%%%%%%%%%%%%%%%%%%%%%%%%%%%%%%%%%%%%%%%%%%%%%%%%%%%%%



\usepackage[utf8]{inputenc}
\usepackage{ccicons}

\usepackage{amsmath}
\usepackage{amsthm}
\usepackage{amscd}
%\usepackage{MnSymbol}
\usepackage{xcolor}

\usepackage{bbold}
\usepackage{url}
\usepackage{upgreek}
%\usepackage{stmaryrd}

\usepackage{lipsum}
\usepackage{tikz-cd}
\usetikzlibrary{cd}
\usetikzlibrary{calc}
\usetikzlibrary{arrows}

\usepackage{bussproofs}
\EnableBpAbbreviations

\DeclareMathAlphabet{\mathpzc}{OT1}{pzc}{m}{it}

%\usepackage[amsmath]{ntheorem}

\newcommand{\lam}{\lambda}
\newcommand{\eps}{\varepsilon}
\newcommand{\ups}{\upsilon}
\newcommand{\mcB}{\mathcal{B}}
\newcommand{\mcD}{\mathcal{D}}
\newcommand{\mcE}{\mathcal{E}}
\newcommand{\mcF}{\mathcal{F}}
\newcommand{\mcP}{\mathcal{P}}
\newcommand{\mcI}{\mathcal{I}}
\newcommand{\mcJ}{\mathcal{J}}
\newcommand{\mcK}{\mathcal{K}}
\newcommand{\mcL}{\mathcal{L}}
\newcommand{\WW}{\mathcal{W}}

\newcommand{\ex}{\mcE_x}
\newcommand{\ey}{\mcE_y}
\newcommand{\bzero}{\boldsymbol{0}}
\newcommand{\bone}{{\boldsymbol{1}}}
\newcommand{\tB}{{\bone_\mcB}}
\newcommand{\tE}{{\bone_\mcE}}
\newcommand{\bt}{\mathbf{t}}
\newcommand{\bp}{\mathbf{p}}
\newcommand{\bsig}{\mathbf{\Sigma}}
\newcommand{\bpi}{\boldsymbol{\pi}}
\newcommand{\Empty}{\mathtt{Empty}}
\newcommand{\truthf}{\mathtt{t}}
\newcommand{\id}{id}
\newcommand{\coo}{\mathtt{coo\ }}
\newcommand{\mcC}{\mathcal{C}}
\newcommand{\Rec}{\mathpzc{Rec}}
\newcommand{\types}{\mathcal{T}}

%\newcommand{\Homrel}{\mathsf{Hom_{Rel}}}
\newcommand{\HomoCPOR}{\mathsf{Hom_{\oCPOR}}}

%\newcommand{\semof}[1]{\llbracket{#1}\rrbracket^\rel}
\newcommand{\sem}[1]{\llbracket{#1}\rrbracket}
\newcommand{\setsem}[1]{\llbracket{#1}\rrbracket^\set}
\newcommand{\relsem}[1]{\llbracket{#1}\rrbracket^\rel}
\newcommand{\dsem}[1]{\llbracket{#1}\rrbracket^{\mathsf D}}
\newcommand{\setenv}{\mathsf{SetEnv}}
\newcommand{\relenv}{\mathsf{RelEnv}}
\newcommand{\oCPOenv}{\mathsf{SetEnv}}
\newcommand{\oCPORenv}{\mathsf{RelEnv}}
\newcommand{\oCPOsem}[1]{\llbracket{#1}\rrbracket^{\set}}
\newcommand{\oCPORsem}[1]{\llbracket{#1}\rrbracket^{\rel}}
\newcommand{\denv}{\mathsf{DEnv}}

\newcommand{\rel}{\mathsf{Rel}}
\newcommand{\setof}[1]{\{#1\}}
\newcommand{\letin}[1]{\texttt{let }#1\texttt{ in }}
\newcommand{\comp}[1]{{\{#1\}}}
\newcommand{\bcomp}[1]{\{\![#1]\!\}}
\newcommand{\beps}{\boldsymbol{\eps}}
%\newcommand{\B}{\mcB}
%\newcommand{\Bo}{{|\mcB|}}

\newcommand{\lmt}{\longmapsto}
\newcommand{\RA}{\Rightarrow}
\newcommand{\LA}{\Leftarrow}
\newcommand{\rras}{\rightrightarrows}
\newcommand{\colim}[2]{{{\underrightarrow{\lim}}_{#1}{#2}}}
\newcommand{\lift}[1]{{#1}\,{\hat{} \; \hat{}}}
\newcommand{\graph}[1]{\langle {#1} \rangle}

\newcommand{\carAT}{\mathsf{car}({\mathcal A}^T)}
\newcommand{\isoAto}{\mathsf{Iso}({\mcA^\to})}
\newcommand{\falg}{\mathsf{Alg}_F}
\newcommand{\CC}{\mathsf{Pres}(\mathcal{A})}
\newcommand{\PP}{\mathcal{P}}
\newcommand{\DD}{D_{(A,B,f)}}
\newcommand{\from}{\leftarrow}
\newcommand{\upset}[1]{{#1}{\uparrow}}
\newcommand{\smupset}[1]{{#1}\!\uparrow}

\newcommand{\Coo}{\mathpzc{Coo}}
\newcommand{\code}{\#}
\newcommand{\nat}{\mathpzc{Nat}}

\newcommand{\eq}{\; = \;}
\newcommand{\of}{\; : \;}
\newcommand{\df}{\; := \;}
\newcommand{\bnf}{\; ::= \;}

\newcommand{\zmap}[1]{{\!{\between\!\!}_{#1}\!}}
\newcommand{\bSet}{\mathbf{Set}}

\newcommand{\dom}{\mathsf{dom}}
\newcommand{\cod}{\mathsf{cod}}
\newcommand{\adjoint}[2]{\mathrel{\mathop{\leftrightarrows}^{#1}_{#2}}}
\newcommand{\isofunc}{\mathpzc{Iso}}
\newcommand{\ebang}{{\eta_!}}
\newcommand{\lras}{\leftrightarrows}
\newcommand{\rlas}{\rightleftarrows}
\newcommand{\then}{\quad\Longrightarrow\quad}
\newcommand{\hookup}{\hookrightarrow}

\newcommand{\spanme}[5]{\begin{CD} #1 @<#2<< #3 @>#4>> #5 \end{CD}}
\newcommand{\spanm}[3]{\begin{CD} #1 @>#2>> #3\end{CD}}
\newcommand{\pushout}{\textsf{Pushout}}
\newcommand{\mospace}{\qquad\qquad\!\!\!\!}

\newcommand{\natur}[2]{#1 \propto #2}

\newcommand{\Tree}{\mathsf{Tree}\,}
\newcommand{\GRose}{\mathsf{GRose}\,}
\newcommand{\List}{\mathsf{List}\,}
\newcommand{\PTree}{\mathsf{PTree}\,}
\newcommand{\Bush}{\mathsf{Bush}\,}
\newcommand{\Forest}{\mathsf{Forest}\,}
\newcommand{\Lam}{\mathsf{Lam}\,}
\newcommand{\LamES}{\mathsf{Lam}^+}
\newcommand{\Expr}{\mathsf{Expr}\,}

\newcommand{\ListNil}{\mathsf{Nil}}
\newcommand{\ListCons}{\mathsf{Cons}}
\newcommand{\LamVar}{\mathsf{Var}}
\newcommand{\LamApp}{\mathsf{App}}
\newcommand{\LamAbs}{\mathsf{Abs}}
\newcommand{\LamSub}{\mathsf{Sub}}
\newcommand{\ExprConst}{\mathsf{Const}}
\newcommand{\ExprPair}{\mathsf{Pair}}
\newcommand{\ExprProj}{\mathsf{Proj}}
\newcommand{\ExprAbs}{\mathsf{Abs}}
\newcommand{\ExprApp}{\mathsf{App}}
\newcommand{\Ptree}{\mathsf{Ptree}}

\newcommand{\kinds}{\mathpzc{K}}
\newcommand{\tvars}{\mathbb{T}}
\newcommand{\fvars}{\mathbb{F}}
\newcommand{\consts}{\mathpzc{C}}
\newcommand{\Lan}{\mathsf{Lan}}
\newcommand{\zerot}{\mathbb{0}}
\newcommand{\onet}{\mathbb{1}}
\newcommand{\bool}{\mathbb{2}}
\renewcommand{\nat}{\mathbb{N}}
%\newcommand{\semof}[1]{[\![#1]\!]}
%\newcommand{\setsem}[1]{\llbracket{#1}\rrbracket^\set}
\newcommand{\predsem}[1]{\llbracket{#1}\rrbracket^\pred}
%\newcommand{\todot}{\stackrel{.}{\to}}
\newcommand{\todot}{\Rightarrow}
\newcommand{\bphi}{{\bm \phi}}

\newcommand{\bm}[1]{\boldsymbol{#1}}

\newcommand{\cL}{\mathcal{L}}
\newcommand{\T}{\mathcal{T}}
\newcommand{\Pos}{P\!}
%\newcommand{\Pos}{\mathcal{P}\!}
\newcommand{\Neg}{\mathcal{N}}
\newcommand{\Hf}{\mathcal{H}}
\newcommand{\V}{\mathbb{V}}
\newcommand{\I}{\mathcal{I}}
\newcommand{\Set}{\mathsf{Set}}
%\newcommand{\Nat}{\mathsf{Nat}}
\newcommand{\Homrel}{\mathsf{Hom_{Rel}}}
\newcommand{\CV}{\mathcal{CV}}
\newcommand{\lan}{\mathsf{Lan}}
\newcommand{\Id}{\mathit{Id}}
\newcommand{\mcA}{\mathcal{A}}
\newcommand{\inl}{\mathsf{inl}}
\newcommand{\inr}{\mathsf{inr}}
%\newcommand{\case}[3]{\mathsf{case}\,{#1}\,\mathsf{of}\,\{{#2};\,{#3}\}}
\newcommand{\tin}{\mathsf{in}}
\newcommand{\fold}{\mathsf{fold}}
\newcommand{\Eq}{\mathsf{Eq}}
\newcommand{\Hom}{\mathsf{Hom}}
\newcommand{\curry}{\mathsf{curry}}
\newcommand{\uncurry}{\mathsf{uncurry}}
\newcommand{\eval}{\mathsf{eval}}
\newcommand{\apply}{\mathsf{apply}}
\newcommand{\oCPO}{{\mathsf{Set}}}
\newcommand{\oCPOR}{{\mathsf{Rel}}}
\newcommand{\oCPORT}{{\mathsf{RT}}}

\newcommand{\ar}[1]{\##1}
\newcommand{\mcG}{\mathcal{G}}
\newcommand{\mcH}{\mathcal{H}}
\newcommand{\TV}{\mathpzc{V}}

\newcommand{\essim}[1]{\mathsf{EssIm}(#1)}
\newcommand{\hra}{\hookrightarrow}

\newcommand{\ol}[1]{\overline{#1}}
\newcommand{\ul}[1]{\underline{#1}}
\newcommand{\op}{\mathsf{op}}
\newcommand{\trige}{\trianglerighteq}
\newcommand{\trile}{\trianglelefteq}
\newcommand{\LFP}{\mathsf{LFP}}
\newcommand{\LAN}{\mathsf{Lan}}
%\newcommand{\Mu}{{\mu\hskip-4pt\mu}}
\newcommand{\Mu}{{\mu\hskip-5.5pt\mu}}
%\newcommand{\Mu}{\boldsymbol{\upmu}}
\newcommand{\Terms}{\mathpzc{Terms}}
\newcommand{\Ord}{\mathpzc{Ord}}
\newcommand{\Anote}[1]{{\color{blue} {#1}}}
\newcommand{\Pnote}[1]{{\color{red} {#1}}}

\newcommand{\greyout}[1]{{\color{gray} {#1}}}
\newcommand{\ora}[1]{\overrightarrow{#1}}

%\newcommand{\?}{{.\ }}
%\theoremheaderfont{\scshape}
%\theorembodyfont{\normalfont}
%\theoremseparator{.\ \ }
\newtheorem{thm}{Theorem}
\newtheorem{dfn}[thm]{Definition}
\newtheorem{prop}[thm]{Proposition}
\newtheorem{cor}[thm]{Corollary}
\newtheorem{lemma}[thm]{Lemma}
\newtheorem{rmk}[thm]{Remark}
\newtheorem{expl}[thm]{Example}
\newtheorem{notn}[thm]{Notation}
%\theoremstyle{nonumberplain}
%\theoremsymbol{\Box}


\theoremstyle{definition}
\newtheorem{exmpl}{Example}

\renewcommand{\greyout}[1]{} %{{\color{gray} {#1}}} -- toggle to remove greyed text

\newcommand{\emptyfun}{{[]}}
\newcommand{\cal}{\mathcal}
%\newcommand{\fold}{\mathit{fold}}
\newcommand{\F}{\mathcal{F}}
\renewcommand{\G}{\mathcal{G}}
\newcommand{\N}{\mathcal{N}}
\newcommand{\E}{\mathcal{E}}
\newcommand{\B}{\mathcal{B}}
\renewcommand{\P}{\mathcal{A}}
\newcommand{\pred}{\mathsf{Fam}}
\newcommand{\env}{\mathsf{Env}}
\newcommand{\set}{\mathsf{Set}}
\renewcommand{\S}{\mathcal S}
\renewcommand{\C}{\mathcal{C}}
\newcommand{\D}{\mathcal{D}}
\newcommand{\A}{\mathcal{A}}
\renewcommand{\id}{\mathit{id}}
\newcommand{\map}{\mathsf{map}}
\newcommand{\pid}{\underline{\mathit{id}}}
\newcommand{\pcirc}{\,\underline{\circ}\,}
\newcommand{\pzero}{\underline{0}}
\newcommand{\pone}{\underline{1}}
\newcommand{\psum}{\,\underline{+}\,}
%\newcommand{\inl}{\mathit{inL}\,}
%\newcommand{\inr}{\mathit{inR}\,}
\newcommand{\pinl}{\underline{\mathit{inL}}\,}
\newcommand{\pinr}{\underline{\mathit{inR}}\,}
\newcommand{\ptimes}{\,\underline{\times}\,}
\newcommand{\ppi}{\underline{\pi_1}}
\newcommand{\pppi}{\underline{\pi_2}}
\newcommand{\pmu}{\underline{\mu}}

\title[Free Theorems for Nested Types]{Free Theorems for 
Nested Types} %% [Short Title] is optional;
                                        %% when present, will be used in
                                        %% header instead of Full Title.
%\titlenote{with title note}             %% \titlenote is optional;
                                        %% can be repeated if necessary;
                                        %% contents suppressed with 'anonymous'
%\subtitle{Subtitle}                     %% \subtitle is optional
%\subtitlenote{with subtitle note}       %% \subtitlenote is optional;
                                        %% can be repeated if necessary;
                                        %% contents suppressed with 'anonymous'


%% Author information
%% Contents and number of authors suppressed with 'anonymous'.
%% Each author should be introduced by \author, followed by
%% \authornote (optional), \orcid (optional), \affiliation, and
%% \email.
%% An author may have multiple affiliations and/or emails; repeat the
%% appropriate command.
%% Many elements are not rendered, but should be provided for metadata
%% extraction tools.

%% Author with single affiliation.
\author{Patricia Johann and Andrew Polonsky}
%\authornote{with author1 note}          %% \authornote is optional;
%                                        %% can be repeated if necessary
%\orcid{nnnn-nnnn-nnnn-nnnn}             %% \orcid is optional
\affiliation{
%  \position{Position1}
%  \department{Department1}              %% \department is recommended
  \institution{Appalachian State University}            %% \institution is required
%  \streetaddress{Street1 Address1}
%  \city{City1}
%  \state{State1}
%  \postcode{Post-Code1}
%  \country{Country1}                    %% \country is recommended
}
\email{johannp@appstate.edu, andrew.polonsky@gmail.com}          %% \email is recommended


\begin{document}

\begin{abstract}
\end{abstract}

%\begin{CCSXML}
%<ccs2012>
%<concept>
%<concept_id>10011007.10011006.10011008</concept_id>
%<concept_desc>Software and its engineering~General programming languages</concept_desc>
%<concept_significance>500</concept_significance>
%</concept>
%<concept>
%<concept_id>10003456.10003457.10003521.10003525</concept_id>
%<concept_desc>Social and professional topics~History of programming languages</concept_desc>
%<concept_significance>300</concept_significance>
%</concept>
%</ccs2012>
%\end{CCSXML}
%
%\ccsdesc[500]{Software and its engineering~General programming languages}
%\ccsdesc[300]{Social and professional topics~History of programming languages}
%% End of generated code


%% Keywords
%% comma separated list
%\keywords{keyword1, keyword2, keyword3}  %% \keywords is optional


\maketitle

\section{Free Theorems for Nested Types}

Let us recall the standard definition of graph relation of a function.

\begin{dfn}
If $f : A \to B$ then the relation $\graph{f} : \rel(A,B)$ is defined
by $(x,y) \in \graph{f}$ iff $fx = y$.
\end{dfn}

Also, note that we are using an angle-bracket notation
for both the graph relation of a function
and for the pairing of functions with the same domain.
Such notation is in part justified by the relation between the two notions,
exposed in Remark~\ref{rmk:graph-fn}.

Any set has an associated equality relation over itself,
and any morphism in $\set$ has an associated graph relation in $\rel$.
% Analogously, a $k$-ary set-functor has an associated equality relation transformer
% (we know that already because of the Identity Extension Lemma, Section 3).
Likewise, for a natural transformation between $k$-ary set functors,
we shall define an associated $k$-ary relation transformer.

\begin{dfn}\label{dfn:graph-nat-transf}
    If $F, G: \Set^k \to \Set$ are $k$-ary set functors
    and $\alpha : F \to G$ is a natural transformation,
    we define the functor $\graph{\alpha}^*: \rel^k \to \rel$ as follows.
    Given
    $R_1 : \rel(A_1, B_1),...,R_k : \rel(A_k,B_k)$,
    let $\iota_{R_i} : R_i \hookrightarrow A_i \times B_i$,
    for $i = 1,...,k$, be the inclusion of $R_i$ as a
    subset of $A_i \times B_i$. By the universal property of the product,
    there exists a unique $h_{\overline{A \times B}}$ making the diagram
    \[
    \begin{tikzcd}[row sep = large]
        F\overline{A}
        &F(\overline{A \times B})
        \ar[l, "{F\overline{\pi_1}}"']
        \ar[r, "{F\overline{\pi_2}}"]
        \ar[d, dashed, "{h_{\overline{A \times B}}}"]
        &F\overline{B}
        \ar[r, "{\alpha_{B}}"]
        &G\overline{B} \\
        &F\overline{A} \times G\overline{B}
        \ar[ul, "{\pi_1}"] \ar[urr, "{\pi_2}"']
    \end{tikzcd}
    \]
    commute.
    Let $h_{\overline{R}} : F\overline{R} \to F\overline{A}
    \times G\overline{B}$ be $h_{\overline{A \times B}} \circ
    F\overline{\iota_R}$.
    Define $\alpha^\wedge\overline{R}$ to be the subobject
    through which $h_{\overline{R}}$ is factorized by the mono-epi
    factorization system in $\set$, as shown in the following diagram:
    \[
    \begin{tikzcd}
        F\overline{R}
        \ar[rr, "{h_{\overline{R}}}"]
        \ar[dr, twoheadrightarrow, "{q_{\alpha^\wedge\overline{R}}}"']
        &&F\overline{A} \times G\overline{B} \\
        &\alpha^\wedge\overline{R}
        \ar[ur, hookrightarrow, "{\iota_{\alpha^\wedge\overline{R}}}"']
    \end{tikzcd}
    \]
    %
    %NO! $Eq_F^* \overline{(A,B,R)} =
    %(F\overline{A},F\overline{B},Eq_F^*\overline{R})$
    %Choose a different name for third component above, i.e., for the thing
    %in Def 22. Maybe $F^\wedge$ on relations, i.e., subsets of domains and
    %codomains?
    Note that $\alpha^\wedge\overline{R} : \rel(F\overline{A}, G\overline{B})$
    by construction, so we can define $\langle \alpha \rangle^* \overline{(A,B,R)} =
    (F\overline{A}, G\overline{B}, \iota_{\alpha^\wedge \overline{R}}\alpha^\wedge\overline{R})$.
    Moreover, if
    $\overline{(\beta, \beta') : (A,B,R) \to (C,D,S)}$ are morphisms in
    $\rel$, then we define $\graph{\alpha}^*\overline{(\beta, \beta')}$ to be
    $(F\overline\beta, G\overline\beta')$.
\end{dfn}

We now show that the above data yield a relation transformer.

\begin{lemma}\label{lem:graph-reln-functors}
If $\alpha : F \to G$ is a morphism in $[\Set^k, \Set]$,
i.e., a natural transformation between $\omega$-cocontinuous functors,
then $\graph{\alpha} = (F, G, \graph{\alpha}^*)$ is in $RT_k$.
\end{lemma}
\begin{proof}
    Clearly, $\graph{\alpha}^*$ is $\omega$-cocontinuous,
    so $\graph{\alpha}^* : [\rel^k,\rel]$.

    Now, consider $\overline{(\beta, \beta') : R \to S}$, where
    $\overline{R : \rel(A, B)}$ and $\overline{S : \rel(C, D)}$.  We want
    to show that there exists a morphism $\epsilon : \alpha^\wedge\overline{R}
    \to \alpha^\wedge\overline{S}$ such that
    \[
    \begin{tikzcd}
        \alpha^\wedge\overline{R}
        \ar[r, hookrightarrow, "{\iota_{\alpha^\wedge\overline{R}}}"]
        \ar[d, "{\epsilon}"']
        & F\overline{A} \times G\overline{B}
        \ar[d, "{F\overline{\beta} \times G\overline{\beta'}}"] \\
        \alpha^\wedge\overline{S}
        \ar[r, hookrightarrow, "{\iota_{\alpha^\wedge\overline{S}}}"']
        & F\overline{C} \times G\overline{D}
    \end{tikzcd}
    \]
    commutes.
    Since $\ol{(\beta,\beta') : R \to S}$, there exist $\overline{\gamma : R \to S}$
    such that each diagram
    \[
    \begin{tikzcd}
        R_i
        \ar[d, "{\gamma_i}"']
        \ar[r, hookrightarrow, "{\iota_{R_i}}"]
        &A_i \times B_i
        \ar[d, "{\beta_i \times \beta'_i}"] \\
        S_i
        \ar[r, hookrightarrow, "{\iota_{S_i}}"]
        &C_i \times D_i
    \end{tikzcd}
    \]
    commutes. Now note that both
    $h_{\overline{C \times D}} \circ F(\overline{\beta \times \beta'})$
    and
    $(F\overline{\beta} \times G\overline{\beta'}) \circ h_{\overline{A \times B}}$
    make
      \[
      \begin{tikzcd}[row sep = large]
          F\overline{C}
          &F\overline{C} \times F\overline{D}
          \ar[l, "{\pi_1}"'] \ar[r, "{\pi_2}"]
          &F\overline{D}
          \ar[r, "{\alpha_{D}}"]
          &G\overline{D}\\
          &F(\overline{A \times B})
          \ar[u, dashed, "{\exists !}"]
          \ar[ul, "{F\pi_1 \circ F(\overline{\beta \times \beta'})}"]
          \ar[urr, "{\alpha_{D} \circ F\pi_2 \circ F(\overline{\beta \times \beta'})}"']
      \end{tikzcd}
      \]
      commute, so they must be equal. We therefore get that the right-hand
      square below commutes, and thus that the entire following diagram does
      as well:
      \[
      \begin{tikzcd}
          F\overline{R}
          \ar[d, "{F\overline{\gamma}}"']
          \ar[r, hookrightarrow, "{F\overline{\iota_R}}"]
          \ar[rr, bend left, "{h_{\overline{R}}}"]
          &F(\overline{A \times B})
          \ar[d, "{F(\overline{\beta \times \beta'})}"]
          \ar[r, "{h_{\overline{A \times B}}}"]
          &F\overline{A} \times G\overline{B}
          \ar[d, "{F\overline{\beta} \times F\overline{\beta'}}"] \\
          F\overline{S}
          \ar[r, hookrightarrow, "{F\overline{\iota_S}}"']
          \ar[rr, bend right, "{h_{\overline{S}}}"']
          &F(\overline{C \times D})
          \ar[r, "{h_{\overline{C \times D}}}"']
          &F\overline{C} \times G\overline{D}
      \end{tikzcd}
      \]
      Finally, by the left-lifting property of $q_{F^\wedge\overline{R}}$
      with respect to $\iota_{F^\wedge\overline{S}}$ given by the epi-mono
      factorization system, there exists an $\epsilon$ such that the diagram
      \[
      \begin{tikzcd}
          F\overline{R}
          \ar[d, "{F\overline{\gamma}}"']
          \ar[r, twoheadrightarrow, "{q_{\alpha^\wedge\overline{R}}}"]
          &\alpha^\wedge\overline{R}
          \ar[d, dashed, "{\epsilon}"]
          \ar[r, hookrightarrow, "{\iota_{\alpha^\wedge\overline{R}}}"]
          &F\overline{A} \times G\overline{B}
          \ar[d, "{F\overline{\beta} \times G\overline{\beta'}}"] \\
          F\overline{S}
          \ar[r, twoheadrightarrow, "{q_{\alpha^\wedge\overline{S}}}"']
          &\alpha^\wedge\overline{S}
          \ar[r, hookrightarrow, "{\iota_{\alpha^\wedge\overline{S}}}"']
          &F\overline{C} \times G\overline{D}
      \end{tikzcd}
      \]
      commutes.
\end{proof}

\begin{rmk}\label{rmk:graph-fn}
Let $f : A \to B$ be a function with graph relation $\graph{f} = (A, B, \graph{f}^*)$.
Consider the function $\langle \id_{A}, f \rangle : A \to A \times B$.
Then $\langle \id_{A}, f \rangle\, A = \graph{f}^*$.
Moreover, if $\iota_{\graph{f}} : \graph{f}^* \hookrightarrow A \times B$
is the inclusion of $\graph{f}^*$ into $A \times B$,
there is an isomorphism of subobjects
\[
\begin{tikzcd}
A \ar[rr, "{\cong}"] \ar[dr, "{\langle \id_{A}, f \rangle}"']
&&{\graph{f}^*} \ar[dl, "{\iota_{\graph{f}}}"]\\
&A \times B
\end{tikzcd}
\]
\end{rmk}

\begin{rmk}
	If $f : A \to B$ is a function
	seen as a natural transformation between 0-ary functors,
	then $\graph{f}$ is (the 0-ary relation transformer associated with) the graph relation of $f$.
	Indeed, we need to apply Definition~\ref{dfn:graph-nat-transf} with $k = 0$,
	i.e., to the degenerate relation $\ast : \rel(\ast, \ast)$.
	As degenerate $0$-ary functors, $A$ and $B$ are constant functors,
	i.e., $A\, \ast = A$ and $B\, \ast = B$.
	By the universal property of the product,
    there exists a unique $h$ making the diagram
    \[
    \begin{tikzcd}[row sep = large]
        A
        &A
        \ar[l, equal]
        \ar[r, equal]
        \ar[d, dashed, "{h}"]
        &A
        \ar[r, "{f}"]
        &B \\
        &A \times B
        \ar[ul, "{\pi_1}"] \ar[urr, "{\pi_2}"']
    \end{tikzcd}
    \]
    commute.
    Notice that $\iota_\ast : \ast \to \ast$ is the identity on $\ast$,
    and $A\, \id_{\ast} = \id_{A}$, so $h_{\ast} = h$.
%    Let $h_{\overline{R}} : F\overline{R} \to F\overline{A}
%    \times G\overline{B}$ be $h_{\overline{A \times B}} \circ
%    F\overline{\iota_R}$.
	Notice that $h_{\overline{A \times B}} = \langle \id_{A}, f \rangle$
	is a monomorphism in $\set$ because $\id_{A}$ is.
	Then, $\iota_{f^\wedge\ast} = \langle \id_{A}, f \rangle$
	and $f^\wedge\ast = A$,
	from which we deduce that
	$\iota_{f^\wedge\ast} f^\wedge\ast = \langle \id_{A}, f \rangle\, A = \graph{f}^*$,
	meaning that the graph of $f$ as a 0-ary natural transformation coincides with the graph of $f$ as a function.
%	Finally, observe that $\langle \id_{A}, f \rangle A$ is a subset of \(A \times B\)
%	and isomorphic to $\graph{f}^*$. 
\end{rmk}

Just as the equality relation $\Eq_B$ on a set $B$ coincides with $\graph{\id_B}$, the graph of the identity on the set,
%i.e., for any $B : \set$, we have that $\graph{\id_B} = \Eq_B$,
we can define the equality relation transformer
to be the graph of the identity natural transformation.

\begin{dfn}
Let $F : [\set^k, \set]$.
Define the equality relation transformer on $F$ as $\Eq_F = \graph{\id_{F}}$.
Then, $Eq_F = (F, F, \Eq_F^*)$ with $\Eq_F^* = \graph{\id_{F}}^*$.
\end{dfn}

% Notice that the equality relation transformer associated to a $k$-ary set functor
% corresponds to the relation transformer associated to the identity natural transformation on the set functor.
% Thus, the notion of graph relation transformer of a natural transformation of set functors
% is a generalization of the notion of equality relation of a set functor.

Next, we show a useful formula to compute the graph relation transformer on a graph relation.

\begin{lemma}\label{lem:eq-reln-equalities}
If $\alpha : F \to G$ is a morphism in $[\Set^k, \Set]$
and $f_1: A_1 \to B_1, ..., f_k : A_k \to B_k$,
then $\graph{\alpha}^* \graph{\overline{f}}
= \langle G f \circ \alpha_{\ol{A}} \rangle
= \langle \alpha_{\ol{B}} \circ F f \rangle$.
\end{lemma}
\begin{proof}
    Since $h_{\overline{A \times B}}$ is the unique
    morphism making the bottom triangle of the diagram
    \[
    \begin{tikzcd}[row sep = large]
        &F\overline{A}
        \ar[d, "{F \overline{\langle \id_A, f \rangle}}" description]
        \ar[dl, equal]
        \ar[dr, "{Ff}"]\\
        F\overline{A}
        &F(\overline{A \times B})
        \ar[l, "{F\overline{\pi_1}}"']
        \ar[r, "{F\overline{\pi_2}}"]
        \ar[d, "{h_{\overline{A \times B}}}"]
        &F\overline{B}
        \ar[r, "{\alpha_B}"]
        &G\overline{B}\\
        &F\overline{A} \times G\overline{B}
        \ar[ul, "{\pi_1}"] \ar[urr, "{\pi_2}"']
    \end{tikzcd}
    \]
    commute,
    and since $h_{\graph{\overline{f}}} = h_{\overline{A \times B}} \circ F \iota_{\graph{f}} = h_{\overline{A \times B}} \circ F \overline{\langle \id_A, f \rangle}$ (the last equality being by Remark~\ref{rmk:graph-fn}),
    the universal property of the product
      \[
      \begin{tikzcd}[row sep = large]
          F\overline{A}
          &F\overline{A} \times G\overline{B}
          \ar[l, "{\pi_1}"'] \ar[r, "{\pi_2}"]
          &G\overline{B}\\
          &F\overline{A}
          \ar[u, dashed, "{\exists !}"]
          \ar[ul, equal]
          \ar[r, "{F\ol{f}}"']
          &F{\ol{B}}
          \ar[u, "{\alpha_{\ol{B}}}"']
      \end{tikzcd}
      \]
      gives that $h_{\graph{\overline{f}}} = \langle \id_{F \ol{A}}, \alpha_{\ol{B}} \circ F\ol{f} \rangle : F \ol{A} \to F \ol{A} \times G \ol{B}$.
      Moreover, $\langle \id_{F \ol{A}}, \alpha_{\ol{B}} \circ F\ol{f} \rangle$ is a monomorphism in $\set$ because $\id_{F \ol{A}}$ is,
      so its epi-mono factorization gives that
      $\iota_{\alpha^\wedge \graph{\overline{f}}} = \langle \id_{F \ol{A}}, \alpha_{\ol{B}} \circ F\ol{f} \rangle$,
      and thus that $\alpha^\wedge \graph{\overline{f}}$, the domain of $\iota_{\alpha^\wedge \graph{\overline{f}}}$
      is equal to $F\overline{A}$.
%     the image of $\langle \id_{F \ol{A}}, \alpha_{\ol{B}} \circ F\ol{f} \rangle$ is equal to the image of $\iota_{\alpha^\wedge \graph{\overline{f}}}$,
      Then, $\iota_{\alpha^\wedge \graph{\overline{f}}} \alpha^\wedge \graph{\overline{f}} = \langle \id_{F \ol{A}}, \alpha_{\ol{B}} \circ F\ol{f} \rangle (F \ol{A}) = \graph{ \alpha_{\ol{B}} \circ F\ol{f} }^*$ (where the last equality is by Remark~\ref{rmk:graph-fn}).
      Therefore, we deduce that $\graph{ \alpha }^* \graph{ \overline{f} }
      = (F\overline{A}, G\overline{B}, \iota_{\alpha^\wedge \graph{\overline{f}}}\,
      \alpha^\wedge \graph{\overline{f}})
      = (F\overline{A}, G\overline{B}, \graph{ \alpha_{\ol{B}} \circ F\ol{f} }^*)
      = \graph{ \alpha_{\ol{B}} \circ F\ol{f} }$.

      Finally, notice that
      $\alpha_{\ol{B}} \circ F\ol{f} = G\ol{f} \circ \alpha_{\ol{A}}$
      by naturality of $\alpha$.
\end{proof}

We have an immediate corollary of the previous result.

\begin{cor}
If $F : [\set^k, \set]$ and $\ol{A : \set}$,
then $\Eq^*_F \ol{\Eq_A} = \Eq_{F\ol{A}}$.
\end{cor}
\begin{proof}
We have that
\[
\Eq^*_F \ol{\Eq_A}
= \graph{\id_F}^* \graph{\id_{\ol{A}}}
= \graph{F \id_{\ol{A}} \circ (\id_F)_{\ol{A}}}  %by your Lemma 5
= \graph{\id_{F\ol{A}} \circ \id_{F\ol{A}}}
%= \graph{\id_F \ol{A} \circ \id_{\ol{A}}}
= \graph{\id_{F\ol{A}}}
= \Eq_{F\ol{A}}
\]
where the second identity is by Lemma~\ref{lem:eq-reln-equalities}.
\end{proof}

We can extend the previous notions of graph of a natural transformation
and equality relation transformer to environments.

\begin{definition}
Let $f : \rho \to \rho'$ is a morphism of set environments.
Then, we define a graph relation environment $\graph{f}$ pointwise,
i.e., for any variable $\phi$, we define $\graph{f} \phi = \graph{f \phi}$.
Notice that $\pi_1 \graph{f} = \rho$ and $\pi_2 \graph{f} = \rho'$.

Likewise, if $\rho$ is a set environment,
we define the equality relation environment $\Eq_\rho$ as $\graph{\id_{\rho}}$.
\end{definition}

With the previous definitions, we can prove an Identity Extension Lemma
for our interpretations.

{\color{red} Insert Identity Extension Lemma here}

Moreover, by making use of the Identity Extension Lemma we can also prove a Graph Lemma.

\begin{lemma}[Graph Lemma]\label{lem:graph}
If $f : \rho \to \rho'$ is a morphism of set environments
and $\Gamma; \Phi \vdash F : \F$,
then $\graph{\setsem{\Gamma; \Phi \vdash F} f} = \relsem{\Gamma; \Phi \vdash F}\graph{f}$
\end{lemma}
\begin{proof}
First observe that $(f, \id_{\rho'}) : \graph{f} \to \Eq_{\rho'}$
and $(\id_{\rho}, f) : \Eq_{\rho} \to \graph{f}$
are morphisms of relation environments.
Applying Lemma~\ref{lem:rel-transf-morph} to each of these observations gives that
\begin{equation}\label{eq:graph-one}
(\setsem{\Gamma; \Phi \vdash F}f, \setsem{\Gamma; \Phi \vdash F}\id_{\rho'})
= \relsem{\Gamma; \Phi \vdash F} (f, \id_{\rho'})
: \relsem{\Gamma; \Phi \vdash F}\graph{f} \to \relsem{\Gamma; \Phi \vdash F}\Eq_{\rho'}
\end{equation}
and
\begin{equation}\label{eq:graph-two}
(\setsem{\Gamma; \Phi \vdash F}\id_{\rho}, \setsem{\Gamma; \Phi \vdash F}f)
= \relsem{\Gamma; \Phi \vdash F} (\id_{\rho}, f)
: \relsem{\Gamma; \Phi \vdash F}\Eq_{\rho} \to \relsem{\Gamma; \Phi \vdash F}\graph{f}
\end{equation}
Expanding Equation~\ref{eq:graph-one} gives that if
$(x,y) \in \relsem{\Gamma; \Phi \vdash F}\graph{f}$
then
\[
(\setsem{\Gamma; \Phi \vdash F} f\, x, \setsem{\Gamma; \Phi \vdash F}\id_{\rho'}\, y) \in \relsem{\Gamma; \Phi \vdash F}\Eq_{\rho'}
\]
Observe that
$\setsem{\Gamma; \Phi \vdash F}\id_{\rho'}\, y = \id_{\setsem{\Gamma; \Phi \vdash F}\rho'}\, y = y$
and
$\relsem{\Gamma; \Phi \vdash F}\Eq_{\rho'} = \Eq_{\relsem{\Gamma; \Phi \vdash F}\rho'}$
So, if
$(x,y) \in \relsem{\Gamma; \Phi \vdash F}\graph{f}$
then
$(\setsem{\Gamma; \Phi \vdash F} f\, x, y) \in \Eq_{\relsem{\Gamma; \Phi \vdash F}\rho'}$,
i.e.,
$\setsem{\Gamma; \Phi \vdash F} f\, x = y$,
i.e.,
$(x, y) \in \graph{\setsem{\Gamma; \Phi \vdash F} f}$.
So, we have that
$\relsem{\Gamma; \Phi \vdash F}\graph{f} \subseteq \graph{\setsem{\Gamma; \Phi \vdash F}f}$

Expanding Equation~\ref{eq:graph-two} gives that, for any
$x \in \setsem{\Gamma; \Phi \vdash F}\rho$,
then
\[
(\setsem{\Gamma; \Phi \vdash F}\id_{\rho}\, x, \setsem{\Gamma; \Phi \vdash F} f\, x) \in \relsem{\Gamma; \Phi \vdash F}\graph{f}
\]
Observe that
$\setsem{\Gamma; \Phi \vdash F}\id_{\rho}\, x = \id_{\setsem{\Gamma; \Phi \vdash F}\rho} x = x$
so, for any $x \in \setsem{\Gamma; \Phi \vdash F}\rho$,
we have that
$(x, \setsem{\Gamma; \Phi \vdash F}f\, x) \in \relsem{\Gamma; \Phi \vdash F}\graph{f}$.
Moreover, $x \in \setsem{\Gamma; \Phi \vdash F}\rho$
if and only if $(x, \setsem{\Gamma; \Phi \vdash F} f\, x) \in \graph{\setsem{\Gamma; \Phi \vdash F}f}$
and, if $x \in \setsem{\Gamma; \Phi \vdash F}\rho$
then $(x, \setsem{\Gamma; \Phi \vdash F} f\, x) \in \setsem{\Gamma; \Phi \vdash F} \graph{f}$,
so if $(x, \setsem{\Gamma; \Phi \vdash F} f\, x) \in \graph{\setsem{\Gamma; \Phi \vdash F}f}$
then $(x, \setsem{\Gamma; \Phi \vdash F} f\, x) \in \setsem{\Gamma; \Phi \vdash F} \graph{f}$,
i.e.,
$\graph{\setsem{\Gamma; \Phi \vdash F}f} \subseteq \setsem{\Gamma; \Phi \vdash F} \graph{f}$.

%For a generic $x \in \setsem{\Gamma; \Phi \vdash F}\rho$,
%the pair $(x, \setsem{\Gamma; \Phi \vdash F} f\, x)$
%is the generic element in $\graph{\setsem{\Gamma; \Phi \vdash F} f}$,
%so we have that
%$\graph{\setsem{\Gamma; \Phi \vdash F}f} \subseteq \relsem{\Gamma; \Phi \vdash F}\graph{f}$.

So, we conclude that
$\relsem{\Gamma; \Phi \vdash F}\graph{f} = \graph{\setsem{\Gamma; \Phi \vdash F}f}$.
\end{proof}

% \begin{thm}\label{thm:graph-lemma}
% If $f_i : A_i \to B_i$ for $i = 1,...,k$ then $F^* {\graph
%   f_1}...{\graph f_k} = \graph{F f_1 ... f_k}$.
% \end{thm}
% \begin{proof}
%   First observe that
%   \[((f_1,...,f_k),(\id_{B_1},...,\id_{B_k})) \in
%   \Hom_{\rel^k}(\bm {\graph f}, \bm {\Eq_{B_i}})\]
%   \noindent
%   and
%   \[((\id_{A_1},...,\id_{A_k}),(f_1,...,f_k)) \in \Hom_{\rel^k}(\bm
%   {\Eq_{A_i}},\bm {\graph f})\] Applying
%   Proposition~\ref{prop:factoid1} to each of these observations gives
%   that
%     \begin{equation}\label{eq:one}
%     (F \bm f, F \bm \id_{B_i}) \in \Hom_\rel(F^* \bm
%         {\graph{f}}, F^* \bm {\Eq_{B_i}})
%         \end{equation}
%       and
%       \begin{equation}\label{eq:two}
%         (F \bm \id_{A_i},F \bm f) \in \Hom_\rel(F^* \bm
%         {\Eq_{A_i}},F^* \bm {\graph{f}})
%         \end{equation}
%       \noindent
%       Expanding Equation~\ref{eq:one} gives that if $(x,y) \in F^* \bm
%       {\graph{f}}$ then $(F {\bm f} x, F {\bm \id_{B_i}} y) \in F^*
%       \bm \Eq_{B_i} = \relsem{E}[\bm \alpha := \bm \Eq_{B_i}] =
%       \Eq_{\setsem{E}[\bm \alpha := \bm B_i]} = \Eq_{F \bm B}$, where
%       the penultimate equality holds by Theorem~\ref{thm:iel}. That
%       is, if $(x,y) \in F^* \bm {\graph{f}}$ then $(F \bm f x, y) \in
%       \Eq_{F \bm B}$, i.e., if $(x,y) \in F^* \bm {\graph{f}}$ then
%       $F\bm f x = y$, i.e., if $(x,y) \in F^* \bm {\graph{f}}$ then
%       ($x, y) \in \graph{F \bm f}$. Thus $F^* \bm {\graph{f}}
%       \subseteq \graph{F \bm f}$.

%       Similar analysis of Equation~\ref{eq:two} gives that $\graph{F
%         \bm f} \subseteq F^* \bm {\graph{f}}$.
% \end{proof}

% Inlining the definitions of $F$ and $F^*$ in the statement of
% Theorem~\ref{thm:graph-lemma} gives

% \begin{equation}\label{eq:three}
% \relsem{E}[\bm \alpha := \bm {\graph{f}}] = \graph{\setsem{E}[\bm
%     \alpha:= \bm f]}
% \end{equation}
% \noindent
% We can use Equation~\ref{eq:three} to prove that the set
% interpretation of a closed term of (closed) type $\Nat^{\bm
%   \alpha}\,F\,G$ is a natural transformation.
% %To see this, write $\fmap\, \bm f$ for $\setsem{F} \bm
% %f = \setsem{F}[\bm \alpha := \bm f]$ and $\gmap\, \bm f$ for
% %$\setsem{G} \bm f = \setsem{G}[\bm \alpha := \bm f]$.

% \begin{thm}
% If $\vdash t : \Nat^{\bm \alpha}\,F\,G$ and $\bm f : \bm A \to \bm B$,
% then $\setsem{t}_{\bm B} \circ \setsem{F}[\bm \alpha := \bm f]
% %Fmap \bm f = Gmap \bm f
% = \setsem{G}[\bm \alpha := \bm f] \circ \setsem{t}_{\bm A}$.
% \end{thm}
% \begin{proof}
% Theorem~\ref{thm:abstraction} ensures that $(\setsem{t},\setsem{t})
% \in \relsem{\Nat^{\bm \alpha}\,F\,G}$, i.e., that for all $\bm R :
% \rel(\bm A,\bm B)$, $x : F\bm A$, and $x': F\bm B$, if $(x,x') \in
% \relsem{F}[\bm \alpha := \bm R]$ then $(\setsem{t}_{\bm A}
% x,\setsem{t}_{\bm B} x') \in \relsem{G}[\bm \alpha := \bm R]$. If $\bm
% f : \bm A \to \bm B$, then taking $\bm R = \bm {\graph{f}}$ and
% instantiating gives that if $(x,x') \in \relsem{F}[\bm \alpha := \bm
%   {\graph{f}}]$ then $(\setsem{t}_{\bf A} x,\setsem{t}_{\bf B} x') \in
% \relsem{G}[\bm \alpha := \bm {\graph{f}}]$. By Equation~\ref{eq:three}
% this is the same as the requirement that if $(x,x') \in
% \graph{\setsem{F}[\bm \alpha := \bm f]}$ then $(\setsem{t}_{\bm A}
% x,\setsem{t}_{\bm B} x') \in \graph{\setsem{G}[\bm \alpha := \bm f]}$
% i.e., that if $x' = \setsem{F}[\bm \alpha := \bm f]x$ then
% $\setsem{t}_{\bm B} x' = \setsem{G}[\bm \alpha := \bm
%   f](\setsem{t}_{\bm A})$, i.e., that $\setsem{t}_{\bm B}
% (\setsem{F}[\bm \alpha := \bm f]x) = \setsem{G}[\bm \alpha := \bm f]
% (\setsem{t}_{\bm A} x)$ for all $x : F\bm A$, i.e., that
% $\setsem{t}_{\bm B} \circ \setsem{F}[\bm \alpha := \bm f] =
% \setsem{G}[\bm \alpha := \bm f] \circ \setsem{t}_{\bm A}$.
% \end{proof}

\bibliography{bibfile}


\end{document}







First, some preliminaries. For $F \in \F$ define $\lift{F}$ as in Agda
code file. This is the syntactic reflection of $\relsem{F}$, we think.

\begin{lemma}
%For $F \in \F$, if $\lift{F} \,\graph{f} \subseteq \graph{Ff}$ then
%$(Ff, \id) : \rel(\lift{F} \,\graph{f}, \id)$.  
For $F \in \F$, if $\relsem{F} \,\graph{f} \subseteq \graph{\setsem{F}f}$ then
$(\setsem{F}f, \id) : \rel(\relsem{F} \,\graph{f}, \id)$.  
\end{lemma}
\begin{proof}
If $\relsem{F} \, \graph{f} \subseteq \graph{\setsem{F}f}$ then $(a,b) \in
    \relsem{F}\,\graph{f}$ implies $(a,b) \in \graph{\setsem{F}f}$,
    i.e., $b = \setsem{F}fa$, i.e., $(\setsem{F}fa, b) \in \id$, i.e.,
    $(\setsem{F}f,\id) \in \rel(\relsem{F}\,\graph{f}, \id)$.
\end{proof}


\begin{lemma}\label{lem:props}
If $\bm \alpha = \{\alpha_1,...,\alpha_k\}$, $t : \Nat^{\bm
  \alpha}\,F\,G$, and $f_i : A_i \to B_i$ for $i = 1,...,k$, then
%\begin{itemize}
%\item
$(t_{A_1,...A_k},t_{B_1...B_k}) : \lift{F} \, \graph{f} \to
  \lift{G}\,\graph{f}$ in $\rel$.
%\item $(t_{A_1,...A_k},t_{B_1...B_k}) : \graph{\setsem{F}map \, f} \to
%  \graph{\setsem{G}map \, f}$ in $\rel$.
%\end{itemize}
\end{lemma}
\begin{proof}
  %The first
This is proved by noting that $\graph{f_i} : \rel(A_i,B_i)$ and
$t = \Lambda \alpha. \lambda x : F\alpha.\, v[x]$, and inducting on
the structure of $v$. {\color{red} Check!}
\end{proof}

If we knew that {\color{red} for more than one $\alpha$?}
\begin{prop}
  If $\emptyset; \alpha~|~\emptyset \vdash F \in \F$,
  %$\gamma$ is a set
%environment, $\ol{\gamma}$ is the relation environment mapping each
%type variable $\beta$ to the identity relation on $\beta \gamma$, each
%type constructor variable $\beta$ of arity $0$ to the identity
%relation on $\beta \gamma$, and each type constructor variable to the
  %identity relation transformer,
  and $f : A \to B$ then
  $\relsem{F}%\ol{\gamma}
  [\alpha := \graph{f}] = \graph{(\setsem{F}map)
    %\gamma map)
  \,f}$. 
\end{prop}
\noindent
then we'd get as a corollary the following useful free theorem:

\begin{cor}
Let $\bm \alpha = \{\alpha_1,...,\alpha_k\}$, $\vdash t : \Nat^{\bm
  \alpha}\,F\,G$, and $f_i : A_i \to B_i$ for $i = 1,...,k$. If
$\emptyset; \bm \alpha~|~\emptyset \vdash s : \setsem{F}[\alpha_1 :=
  A_1]...[\alpha_k := A_k]$ then $t_{B_1...B_k}
\,((\setsem{F}map)\,f_1\,...\,f_k\,s) = (\setsem{G}map) \,
f_1\,...\,f_k\,(t_{A_1...A_k} s)$.
\end{cor}
\begin{proof}
By part 2 of Lemma~\ref{lem:props} we have that for all $(s,s') \in
\relsem{F}[\bm \alpha := \bm {\graph{f}}]$, $(t_{A_1...A_k} s,
t_{B_1...B_k}s') \in \relsem{G}[\bm \alpha := \bm {\graph{f}}]$, i.e.,
for all $(s,s') \in \graph{\setsem{F}map \,\bm f}$, $(t_{A_1...A_k} s,
t_{B_1...B_k}s') \in \graph{\setsem{G}map\,\bm f}$ {\color{red} by a
  suitable proposition above}, i.e., $t_{B_1...B_k}
\,((\setsem{F}map)\,f_1\,...\,f_k\,s) = (\setsem{G}map) \,
f_1\,...\,f_k\,(t_{A_1...A_k} s)$.


An alternative proof might be: See p.1,15 july
\end{proof}






  






\section{Conclusion}

Can do everything in abstract locally presentable cartesian closed
category. 

Give definitions for arb lpccc, but compute free theorems in Set/Rel.



\bibliography{references}

\end{document}



\begin{lemma}\label{lem:rel-transf}
For any \,$\Gamma;\Phi \vdash \tau : \F$ and any relation environment
$\rho$,
\[\relsem{\Gamma;\Phi \vdash \tau}\rho :
\rel(\setsem{\Gamma;\Phi \vdash \tau} (\pi_1 \rho),
\setsem{\Gamma;\Phi \vdash \tau} (\pi_2 \rho))\]
\end{lemma}
\begin{proof}
By induction on the structure of $\tau$. The only interesting case is
when $\tau = (\mu \phi. \lambda
\overline{\alpha}. H)\overline{\tau}$. We first observe that $T_\rho$
is an endofunctor on $RT$, i.e., that, for any relation transformer $F
= (F^0, F^1, F^*)$, the triple $T_{\rho} F = (T^\set_{\pi_1 \rho}F^0,
T^\set_{\pi_2 \rho}F^1, T^\rel_{\rho}F)$ is also a relation
transformer.  Indeed, for every $R_j : \rel(A_j, B_j)$, $j = 1, \dots,
k$, and for $i = 1, 2$, we have
\[\begin{split}
\pi_i(T^\rel_{\rho}\,F\,\overline{R})
&= \pi_i(\relsem{\Gamma;\Phi,\phi,\overline{\alpha} \vdash H}\rho[\phi := F]\overline{[\alpha := R]}) \\
&= \setsem{\Gamma;\Phi,\phi,\overline{\alpha} \vdash H} (\pi_i (\rho[\phi := F]\overline{[\alpha := R]})) \\
&= \setsem{\Gamma;\Phi,\phi,\overline{\alpha} \vdash H} (\pi_i \rho)[\phi := \pi_i F]\overline{[\alpha := \pi_i R]}) \\
&= T^\set_{\pi_i \rho} (\pi_i F) (\overline{\pi_i R})
\end{split}\]
and 
\[\begin{split}
\pi_i(T^\rel_{\rho}\,F\,(\overline{\gamma_1}, \overline{\gamma_2}))
&= \pi_i(\relsem{\Gamma;\Phi,\phi,\overline{\alpha} \vdash H}\rho[\phi
  := F]\overline{[\alpha := (\overline{\gamma_1}, \overline{\gamma_2})]}) \\
&= \setsem{\Gamma;\Phi,\phi,\overline{\alpha} \vdash H} (\pi_i (\rho[\phi := F]\overline{[\alpha := (\gamma_1,\gamma_2)]})) \\
&= \setsem{\Gamma;\Phi,\phi,\overline{\alpha} \vdash H} (\pi_i \rho)[\phi := \pi_i F]\overline{[\alpha := \pi_i \gamma_i]}) \\
&= T^\set_{\pi_i \rho} (\pi_i F) (\overline{\pi_i \gamma_i})
\end{split}\]
Here, the second equality in each of the above chains of equalities is
by the induction hypothesis. The upshot is that
$T^\rel_{\rho}F^*\overline{R} : \rel( T^\set_{\pi_1\rho} F^0
\overline{A}, T^\set_{\pi_2\rho} F^1 \overline{B} )$, and, in
particular, that
\begin{equation}\label{eq:rel-transf-T}
\pi_i (T_\rho F) = T^\set_{\pi_i \rho} (\pi_i F)
\end{equation}
We now show that, for every morphism $\delta = (\delta^0, \delta^1) :
F \to G$ in $RT$, $T_\rho \delta : T_\rho F \to T_\rho G$ is a
morphism in $RT$. That is, we show that for all $\overline{R : \rel(A,
  B)}$ and $(x, y) \in (T_\rho F)^* \overline{R}$, we have that
$((T_\rho \delta)^0_{\overline A} \,x, (T_\rho \delta)^1_{\overline
  B}\, y) \in (T_\rho G)^* \overline{R}$, i.e., that
$((T^\set_{\pi_1\rho} \delta^0)_{\overline A}\, x, (T^\set_{\pi_2
  \rho} \delta^1)_{\overline B}\, y) \in (T_\rho G)^*
\overline{R}$. Let $\overline{R : \rel(A, B)}$ and $(x, y) \in (T_\rho
F)^* \overline{R} = \relsem{\Gamma;\Phi,\phi,\overline{\alpha} \vdash
  H}\rho[\phi := F]\overline{[\alpha := R]}$.  Since
$\relsem{\Gamma;\Phi,\phi,\overline{\alpha} \vdash H}\rho[\phi :=
  \text{--}]\overline{[\alpha := R]}$ is a functor from $\relenv$ to
$RT$ by the induction hypothesis, we have that
\[\begin{array}{ll}
 & \relsem{\Gamma;\Phi,\phi,\overline{\alpha} \vdash H}\rho[\phi :=
  \delta]\overline{[\alpha := R]}\\
: & \relsem{\Gamma;\Phi,\phi,\overline{\alpha} \vdash H}\rho[\phi :=
  F]\overline{[\alpha := R]} \to
\relsem{\Gamma;\Phi,\phi,\overline{\alpha} \vdash H}\rho[\phi :=
  G]\overline{[\alpha := R]}
\end{array}\]
is a morphism in $\rel$. Therefore,
\[\begin{array}{ll}
 &\relsem{\Gamma;\Phi,\phi,\overline{\alpha} \vdash H}\rho[\phi :=
  \delta]\overline{[\alpha := R]} (x, y) \\
=& ( (\relsem{\Gamma;\Phi,\phi,\overline{\alpha} \vdash H}\rho[\phi :=
  \delta]\overline{[\alpha := R]})^0 x,
(\relsem{\Gamma;\Phi,\phi,\overline{\alpha} \vdash H}\rho[\phi :=
  \delta]\overline{[\alpha := R]})^1 y)\\
= &( \setsem{\Gamma;\Phi,\phi,\overline{\alpha} \vdash H}(\pi_1
\rho)[\phi := \delta^0]\overline{[\alpha := A]} x, 
\setsem{\Gamma;\Phi,\phi,\overline{\alpha} \vdash H}(\pi_2 \rho)[\phi
  := \delta^1]\overline{[\alpha := B]} y) \\ 
= & ((T^\set_{\pi_1\rho} \delta^0)_{\overline A} x, (T^\set_{\pi_2
  \rho} \delta^1)_{\overline B} y) 
\end{array}\]
is in $(T_\rho G)^* \overline{R}$.  Here, the second equality is by
Lemma~\ref{lem:rel-transf-morph}.  This concludes the proof that
$T_\rho$ is an endofunctor on $RT$.

Next, we prove by induction on natural numbers that, for every $n :
\nat$ and for every relation transformer $F = (F^0, F^1, F^*)$, the
triple $T_{\rho}^n F = ({(T^\set_{\pi_1 \rho})}^n F^0, {(T^\set_{\pi_2
    \rho})}^n F^1, {(T^\rel_{\rho})}^n F^*)$ is a relation transformer
and, in particular, that
\begin{equation}\label{eq:rel-transf-T-iter}
\pi_i(T_{\rho}^n F) = {(T^\set_{\pi_i \rho})}^{n} (\pi_i F)
\end{equation}
for $i = 1, 2$.
Indeed, the base case $n = 0$ is obvious, and the inductive step is proven by
\[\begin{split}
\pi_i(T_{\rho}^{n+1} F)
&= \pi_i(T_{\rho}^{n} (T_{\rho} F))\\
&= {(T^\set_{\pi_i \rho})}^{n} ( \pi_i(T_{\rho} F) )\\
&= {(T^\set_{\pi_i \rho})}^{n} ( T^\set_{\pi_i \rho} (\pi_i F) ) \\
&= {(T^\set_{\pi_i \rho})}^{n+1} (\pi_i F)
\end{split}\]
The second equality is by the induction hypothesis, and the third is
by Equation~\ref{eq:rel-transf-T}. We can now prove that the triple
$\mu T_{\rho} = (\mu T^\set_{\pi_1 \rho}, \mu T^\set_{\pi_2 \rho}, \mu
T^\rel_\rho)$ is a relation transformer and, in particular, that
\begin{equation}\label{eq:rel-transf-T-limit}
\pi_i ( \mu T_\rho ) = \mu T^\set_{\pi_i \rho}
\end{equation}
for $i = 1, 2$. Moreover,
\[\begin{split}
\pi_i(\mu T_{\rho})
&= \pi_i( \lim_{\xrightarrow[n]{}} \, T_{\rho}^n K_0 )  \\
&= \lim_{\xrightarrow[n]{}} \, \pi_i ( T_{\rho}^n K_0 )  \\
&= \lim_{\xrightarrow[n]{}} \, {(T^\set_{\pi_i \rho})}^n (\pi_i K_0)  \\
&= \lim_{\xrightarrow[n]{}} \, {(T^\set_{\pi_i \rho})}^n K^\set_0  \\
&= \mu T^\set_{\pi_i \rho}
\end{split}\]
The third equality above is by
Equation~\ref{eq:rel-transf-T-iter}. Finally, we can prove the lemma
we seek for the $\mu$ case.  For $i = 1, 2$ we have
\[
\begin{split}
\pi_i(\relsem{\Gamma;\Phi \vdash (\mu \phi. \lambda \overline{\alpha}. H) \overline{\tau}}\rho)
&= \pi_i( \mu T_\rho (\overline{\relsem{\Gamma;\Phi \vdash \tau}\rho})) \\
&= \pi_i(\mu T_{\rho}) (\overline{\pi_i(\relsem{\Gamma;\Phi \vdash \tau}\rho})) \\
&= \mu T^\set_{\pi_i\rho} (\overline{\setsem{\Gamma;\Phi \vdash \tau}(\pi_i\rho)}) \\
&= \setsem{\Gamma;\Phi \vdash (\mu \phi. \lambda \overline\alpha. H) \overline{\tau}}(\pi_i\rho)
\end{split}
\]
The third equality is by Equation ~\ref{eq:rel-transf-T-limit} and
induction on $\tau$.
\end{proof}

\noindent

%%% We don't currently need this, but it should be true.
%\begin{lemma}
%\label{lem:rel-transf-uniqueness}
%In a relation transformer $(H^0, H^1, H^*)$, the $\set$-functors $H^0$ and $H^1$ are uniquely determined by the $\rel$-functor $H^*$.
%%A functor \(H^* : \rel^k \to \rel\) carries at most one relation transformer structure \((H^0, H^1, H^*)\).
%\end{lemma}
%\begin{proof}
%Notice that, for any $\bar{A} \in \set^k$, there exists a $\bar{R} \in \rel^k$ such that $\pi_i(\bar{R}) = \bar{A}\) for \(i = 1, 2$.
%In particular, $\bar{R}$ can be taken to be the equality relation $\delta_{\bar{A}}$ on $\bar{A}$.
%Then, $H^i$ is uniquely determined by $H^i(\bar{A}) = H^i(\pi_i(\delta_{\bar{A}})) = \pi_i(H^*(\delta_{\bar{A}}))$.
%\end{proof}

Since a well-formed type involves only finitely many (free) variables,
when convenient we identify relation environments with finite tuples
containing the relations to which the environment maps those
variables.  With this convention we also see by induction on the
structure of $\tau$ that in fact the following morphism counterpart to
Lemma~\ref{lem:rel-transf} also holds:


{\color{red} This makes the ``morphism counterpart'' comment clear. We
  actually need more than what is proved in
  Lemma~\ref{lem:rel-transf-morph} to conclude that the black version
  of this theorem holds, no?
\begin{lemma}
  For any \,$\Gamma;\Phi \vdash \tau : \F$ and any morphism $f : \rho
  \to \rho'$ between relation environments $\rho$ and $\rho'$,
\[\relsem{\Gamma;\Phi \vdash \tau}f :
\rel(\setsem{\Gamma;\Phi \vdash \tau} (\pi_1 f),
\setsem{\Gamma;\Phi \vdash \tau} (\pi_2 f))\]
\end{lemma}}

\begin{lemma}\label{lem:rel-transf-morph}
$\sem{\Gamma;\Phi \vdash \tau} = (\setsem{\Gamma;\Phi \vdash
    \tau},\setsem{\Gamma;\Phi \vdash \tau},\relsem{\Gamma;\Phi \vdash
    \tau})$ is a relation transformer.
\end{lemma}
\begin{proof}
By induction on the structure of $\tau$. Let $ i = 1,2$.
\begin{itemize}
\item When $\tau = v$, $\tau = \sigma_1 \to \sigma_2$, or $\tau =
  \Nat^{\bm \alpha}\,F\,G$, then the result follows from the fact
  that
\[\begin{array}{lll}
       \pi_i(\relsem{\Gamma;\emptyset \vdash \tau}f)
 & = & \id_{\pi_i(\setsem{\Gamma;\emptyset \vdash \tau}\rho)}\\
 & = & \id_{\setsem{\Gamma;\emptyset \vdash \tau}(\pi_i\rho)}\\
 & = & \setsem{\Gamma;\emptyset \vdash \tau}(\pi_i f)
\end{array}\]
The third equality is by Lemma~\ref{lem:rel-transf}.
\item When $\tau = 0$, $\tau = 1$, $\tau = \sigma_1 + \sigma_2$, or
  $\tau = \sigma \times \sigma_2$, the result follows from the
  definitions of the initial and terminal objects, and the definitions
  of sums and products, in $\rel$.
\item When $\tau = \Gamma; \Phi \vdash \phi^k\tau_1...\tau_k$, we have    
\[\begin{array}{lll}
&   & \pi_i \relsem{\Gamma; \Phi \vdash \phi^k\tau_1...\tau_k}f\\
& = & \pi_i ((f\phi)_{\overline{\relsem{\Gamma; \Phi \vdash \tau}\rho'}})
\circ \pi_i((\rho\phi)({\overline{\relsem{\Gamma; \Phi \vdash \tau}f}}))\\
& = & (\pi_i (f\phi))_{\overline{\pi_i (\relsem{\Gamma; \Phi \vdash \tau}\rho')}}
\circ \pi_i(\rho\phi)({\overline{\pi_i (\relsem{\Gamma; \Phi \vdash \tau}f}}))\\
& = & ((\pi_i f)\phi)_{\overline{\setsem{\Gamma; \Phi \vdash \tau}(\pi_i\rho')}}
\circ ((\pi_i\rho)\phi)({\overline{\setsem{\Gamma; \Phi \vdash \tau}(\pi_i f)}})\\
& = & \setsem{\Gamma; \Phi \vdash \phi^k\tau_1...\tau_k}(\pi_i f)
\end{array}\]
\item When $\tau = (\mu \phi. \lambda \overline\alpha. H)
  \overline{\tau}$, we first observe that if $F = (F^0,F^1, F^*)$ is a
  relation transformer, then so is the triple $\sigma_f F =
  (\sigma^\set_{\pi_i f} F^0,\sigma^\set_{\pi_2 f} F^1,\sigma^\rel_f
  F^*)$. Moreover, for any relation transformer $F = (F^0,F^1,F^*)$,
  we have that
\begin{equation}\label{eq:rel-transf-sigma}
\pi_i(\sigma_f F) = \sigma^\set_{\pi_i f} (\pi_i F)
\end{equation}
Indeed, for every $R_j : \rel(A_j, B_j)$ and $j = 1, \dots, k$
\[
\begin{split}
\pi_i (\sigma_f \,F \,\overline{R})
&= \pi_i (\relsem{\Gamma;\Phi,\phi,\overline{\alpha} \vdash H}f[\phi
  := \id_F]\overline{[\alpha := \id_R]}) \\ 
&= \setsem{\Gamma;\Phi,\phi,\overline{\alpha} \vdash H}(\pi_i(f[\phi := \id_F]\overline{[\alpha := \id_R]})) \\
&= \setsem{\Gamma;\Phi,\phi,\overline{\alpha} \vdash H}(\pi_i f)[\phi := \pi_i\id_F]\overline{[\alpha := \pi_i\id_R]} \\
&= \setsem{\Gamma;\Phi,\phi,\overline{\alpha} \vdash H}(\pi_i f)[\phi := \id_{\pi_i F}]\overline{[\alpha := \id_{\pi_i R}]} \\
&= \sigma^\set_{\pi_i f} (\pi_i F) (\overline{\pi_i R})
\end{split}
\]
Next, we prove by induction on natural numbers, for every $n : \nat$
and for every relation transformer $F = (F^0,F^1,F^*)$, the triple
$\sigma_f^n F = ((\sigma^\set_{\pi_1 f})^n F^0,(\sigma^\set_{\pi_2
  f})^n F^1, (\sigma^\rel_f)^n F^*)$ is a relation transformer. In
particular,
\begin{equation}\label{eq:rel-transf-sigma-iter}
\pi_i(\sigma_f^n F) = {(\sigma^\set_{\pi_i f})}^n (\pi_i F)
\end{equation}
for $i = 1, 2$.
Indeed, the base case $n = 0$ is obvious, and the inductive step is proven by
\[
\begin{split}
\pi_i(\sigma_f^{n+1} F)
&= \pi_i(\sigma_f^n (\sigma_f F) ) \\
&= {(\sigma^\set_{\pi_i f})}^n (\pi_i(\sigma_f F)) \\
&= {(\sigma^\set_{\pi_i f})}^n (\sigma^\set_{\pi_i f} (\pi_i F)) \\
&= {(\sigma^\set_{\pi_i f})}^{n+1} (\pi_i F)
\end{split}
\]
The third equality is by Equation~\ref{eq:rel-transf-sigma}.
Now we prove that
\begin{equation}\label{eq:rel-transf-sigma-limit}
\pi_i ( \mu \sigma_f ) = \mu \sigma^\set_{\pi_i f}
\end{equation}
for $i = 1, 2$.
Indeed,
\[
\begin{split}
\pi_i ( \mu \sigma_f )
&= \pi_i( \lim_{\xrightarrow[n]{}} \,\sigma_{f}^n K_0 )  \\
&= \lim_{\xrightarrow[n]{}} \, \pi_i( \sigma_{f}^n K_0 )  \\
&= \lim_{\xrightarrow[n]{}} \, (\sigma^\set_{\pi_i f})^n (\pi_i K_0)  \\
&= \lim_{\xrightarrow[n]{}} \, (\sigma^\set_{\pi_i f})^n K^\set_0  \\
&= \mu \sigma^\set_{\pi_i f}
\end{split}
\]
The third equality is by Equation~\ref{eq:rel-transf-sigma-iter}.
Finally, we can prove the lemma we seek for the $\mu$ case. We have
\[
\begin{split}
\pi_i(\relsem{\Gamma;\Phi \vdash (\mu \phi. \lambda
  \overline\alpha. H)\overline{\tau}}f) 
&= \pi_i(\mu T_{\rho'}(\overline{\relsem{\Gamma;\Phi \vdash
    \tau}f}) \circ (\mu \sigma_f)_{\overline{\relsem{\Gamma;\Phi
      \vdash \tau}\rho}}) \\ 
&= \pi_i(\mu T_{\rho'}(\overline{\relsem{\Gamma;\Phi \vdash
    \tau}f})) \circ \pi_i((\mu
\sigma_f)_{\overline{\relsem{\Gamma;\Phi \vdash \tau}\rho}}) \\  
&= \pi_i(\mu T_{\rho'})(\pi_i(\overline{\relsem{\Gamma;\Phi
    \vdash \tau}f})) \circ \pi_i(\mu
\sigma_f)_{\pi_i(\overline{\relsem{\Gamma;\Phi \vdash 
      \tau}\rho})} \\ 
&= (\mu T^\set_{\pi_i \rho'})(\overline{\setsem{\Gamma;\Phi \vdash \tau}(\pi_i f)}) \circ (\mu \sigma^\set_{\pi_i f})_{\overline{\setsem{\Gamma;\Phi \vdash \tau}\pi_i(\rho)}} \\
&= \setsem{\Gamma;\Phi \vdash (\mu \phi. \lambda \overline\alpha. H)\overline{\tau}}(\pi_i f).
\end{split}
\]
for $i = 1, 2$.
The fourth equality is by~\ref{eq:rel-transf-T-limit},
\ref{eq:rel-transf-sigma-limit}, and induction on $\tau$.
\end{itemize}
\end{proof}




\begin{thm}\label{thm:at}
Every well-formed term $\Gamma;\Phi~|~\Delta \vdash t : \tau$ induces
a natural transformation from $\sem{\Gamma; \Phi \vdash \Delta}$ to
$\sem{\Gamma; \Phi \vdash \tau}$, i.e., a triple of natural
transformations 
\[(\setsem{\Gamma;\Phi~|~\Delta \vdash t : \tau},
\setsem{\Gamma;\Phi~|~\Delta \vdash t : \tau},
\relsem{\Gamma;\Phi~|~\Delta \vdash t : \tau})\] such that, for all
$\rho : \relenv$,
\[\relsem{\Gamma;\Phi~|~\Delta \vdash t : \tau}\rho =
(\setsem{\Gamma;\Phi~|~\Delta \vdash t : \tau}(\pi_1 \rho),
\setsem{\Gamma;\Phi~|~\Delta \vdash t : \tau}(\pi_2 \rho))\]
\end{thm}
\begin{proof}
We proceed by induction on the judgement
$\Gamma;\Phi~|~\Delta \vdash t : \tau$, proving that
\[
\pi_i(\relsem{\Gamma;\Phi~|~\Delta \vdash t : \tau}\rho)
= \setsem{\Gamma;\Phi~|~\Delta \vdash t : \tau}(\pi_i\rho)
\]
for $i = 1, 2$.
We will use Definitions~\ref{def:set-interp} and~\ref{def:rel-interp}, together
with the facts that the cartesian structure of $\rel$ is derived from
that of $\set$ and that initial algebras in $\rel$ are computed in
terms of initial algebras in $\set$.
\begin{itemize}
\item for $\Gamma;\emptyset \,|\, \Delta,x :\tau \vdash x : \tau$ we have
  \[
  \begin{split}
    &\hspace{-0.3in} \pi_i(\relsem{\Gamma;\emptyset \,|\, \Delta,x :\tau \vdash x : \tau} \rho) \\
    &= \pi_i(
      \pi_{|\Delta|+1}: \relsem{\Gamma;\emptyset \vdash \Delta}\rho
      \times \relsem{\Gamma;\emptyset \vdash \tau}\rho
      \to \relsem{\Gamma;\emptyset \vdash \tau}\rho
    ) \\
    &= \pi_{|\Delta|+1}: \pi_i(\relsem{\Gamma;\emptyset \vdash \Delta}\rho)
      \times \pi_i(\relsem{\Gamma;\emptyset \vdash \tau}\rho)
      \to \pi_i(\relsem{\Gamma;\emptyset \vdash \tau}\rho) \\
    &= \pi_{|\Delta|+1}: \setsem{\Gamma;\emptyset \vdash \Delta}(\pi_i\rho)
      \times \setsem{\Gamma;\emptyset \vdash \tau}(\pi_i\rho)
      \to \setsem{\Gamma;\emptyset \vdash \tau}(\pi_i\rho) \\
    &= \setsem{\Gamma;\emptyset \,|\, \Delta,x :\tau \vdash x : \tau} (\pi_i\rho)
  \end{split}
  \]
\item for $\Gamma;\emptyset \,|\, \Delta \vdash \lambda x.t : \sigma \to \tau$ we have
  \[
  \begin{split}
    &\hspace{-0.3in} \pi_i(\relsem{\Gamma;\emptyset \,|\, \Delta \vdash \lambda x.t : \sigma \to \tau}\rho) \\
    &= \pi_i(\curry (\relsem{\Gamma;\emptyset \,|\, \Delta, x : \sigma \vdash t : \tau}\rho)) \\
    &= \curry(\pi_i (\relsem{\Gamma;\emptyset \,|\, \Delta, x : \sigma \vdash t : \tau}\rho)) \\
    &= \curry(\setsem{\Gamma;\emptyset \,|\, \Delta, x : \sigma \vdash t : \tau}(\pi_i \rho)) \\
    &= \setsem{\Gamma;\emptyset \,|\, \Delta \vdash \lambda x.t : \sigma \to \tau}(\pi_i \rho)
  \end{split}
  \]
\item for $\Gamma;\emptyset \,|\, \Delta \vdash st: \tau$ we have
  \[
  \begin{split}
    &\hspace{-0.3in} \pi_i(\relsem{\Gamma;\emptyset \,|\, \Delta \vdash st: \tau} \rho) \\
    &= \pi_i(\eval \circ \langle \relsem{\Gamma;\emptyset \,|\, \Delta \vdash s: \sigma \to
    \tau}\rho, \relsem{\Gamma;\emptyset \,|\, \Delta \vdash t: \sigma}\rho) \\
    &= \eval \circ \langle
       \pi_i(\relsem{\Gamma;\emptyset \,|\, \Delta \vdash s: \sigma \to \tau}\rho),
       \pi_i(\relsem{\Gamma;\emptyset \,|\, \Delta \vdash t: \sigma}\rho)
     \rangle \\
    &= \eval \circ \langle
       \setsem{\Gamma;\emptyset \,|\, \Delta \vdash s: \sigma \to \tau}(\pi_i\rho),
       \setsem{\Gamma;\emptyset \,|\, \Delta \vdash t: \sigma}(\pi_i\rho)
     \rangle \\
    &= \setsem{\Gamma;\emptyset \,|\, \Delta \vdash st: \tau} (\pi_i \rho)
  \end{split}
  \]
\item for $\Gamma;\emptyset \,|\, \Delta \vdash
  L_{\overline \alpha} x.t: \Nat^{\overline\alpha} \,F \,G$ we have
  \[
  \begin{split}
    &\hspace{-0.3in} \pi_i(\relsem{\Gamma;\emptyset \,|\, \Delta \vdash L_{\overline \alpha} x.t
      : \Nat^{\overline\alpha} \,F \,G}\rho) \\
    &= \pi_i(\curry (\relsem{\Gamma;\overline \alpha\,|\, \Delta, x : F \vdash t: G}
      \rho[\overline{\alpha := \_}])) \\
    &= \curry (\pi_i(\relsem{\Gamma;\overline \alpha\,|\, \Delta, x : F \vdash t: G}
      \rho[\overline{\alpha := \_}])) \\
    &= \curry (\setsem{\Gamma;\overline \alpha\,|\, \Delta, x : F \vdash t: G}
      (\pi_i(\rho[\overline{\alpha := \_}]))) \\
    &= \curry (\setsem{\Gamma;\overline \alpha\,|\, \Delta, x : F \vdash t: G}
      (\pi_i\rho)[\overline{\alpha := \_}]) \\
    &= \setsem{\Gamma;\emptyset \,|\, \Delta \vdash L_{\overline \alpha} x.t
      : \Nat^{\overline\alpha} \,F \,G}(\pi_i\rho)
  \end{split}
  \]
\item for $\Gamma;\Phi \,|\, \Delta \vdash
  t_{\bf \tau} s: G [\overline{\alpha := \tau}]$ we have
  \[
  \begin{split}
    &\hspace{-0.3in} \pi_i(\relsem{\Gamma;\Phi \,|\, \Delta \vdash t_{\bf \tau} s:
      G [\overline{\alpha := \tau}]}\rho) \\
    &= \pi_i( \eval \circ \langle
        (\relsem{\Gamma;\emptyset \,|\, \Delta \vdash t :
          \Nat^{\overline{\alpha}} \,F \,G}\rho\; \_)_{\overline{\relsem{\Gamma;\Phi
          \vdash \tau}\rho}},
        \relsem{\Gamma;\Phi \,|\, \Delta \vdash s: F [\overline{\alpha := \tau}]}\rho
      \rangle) \\ 
    &= \eval \circ \langle
        \pi_i((\relsem{\Gamma;\emptyset \,|\, \Delta \vdash t :
          \Nat^{\overline{\alpha}} \,F \,G}\rho\; \_)_{\overline{\relsem{\Gamma;\Phi
          \vdash \tau}\rho}}),
        \pi_i(\relsem{\Gamma;\Phi \,|\, \Delta \vdash s: F [\overline{\alpha := \tau}]}\rho)
      \rangle \\ 
    &= \eval \circ \langle
        (\pi_i(\relsem{\Gamma;\emptyset \,|\, \Delta \vdash t :
          \Nat^{\overline{\alpha}} \,F \,G}\rho)\; \_)_{\overline{\pi_i(\relsem{\Gamma;\Phi
          \vdash \tau}\rho)}},
        \pi_i(\relsem{\Gamma;\Phi \,|\, \Delta \vdash s: F [\overline{\alpha := \tau}]}\rho)
      \rangle \\ 
    &= \eval \circ \langle
        (\setsem{\Gamma;\emptyset \,|\, \Delta \vdash t :
          \Nat^{\overline{\alpha}} \,F \,G}(\pi_i\rho)\; \_)_{\overline{\setsem{\Gamma;\Phi
          \vdash \tau}(\pi_i\rho)}},
        \setsem{\Gamma;\Phi \,|\, \Delta \vdash s: F [\overline{\alpha := \tau}]}(\pi_i\rho)
      \rangle \\
    &= \setsem{\Gamma;\Phi \,|\, \Delta \vdash t_{\bf \tau} s:
      G [\overline{\alpha := \tau}]}(\pi_i\rho)
  \end{split}
  \]
\item for $\Gamma;\Phi \,|\, \Delta,x :\tau \vdash x : \tau$ is analogous to
  $\Gamma;\emptyset \,|\, \Delta,x :\tau \vdash x : \tau$.
\item for $\Gamma;\Phi \,|\, \Delta \vdash \bot_\tau t : \tau$ we have
  \[
  \begin{split}
    \pi_i(\relsem{\Gamma;\Phi \,|\, \Delta \vdash \bot_\tau t : \tau} \rho)
    &= \pi_i(!^0_{\relsem{\Gamma;\Phi \vdash \tau}\rho}
      \circ \relsem{\Gamma;\Phi~|~\Delta \vdash t : \zerot}\rho) \\
    &= !^0_{\pi_i(\relsem{\Gamma;\Phi \vdash \tau}\rho)}
      \circ \pi_i(\relsem{\Gamma;\Phi~|~\Delta \vdash t : \zerot}\rho) \\
    &= !^0_{\setsem{\Gamma;\Phi \vdash \tau}(\pi_i\rho)}
      \circ \setsem{\Gamma;\Phi~|~\Delta \vdash t : \zerot}(\pi_i\rho) \\
    &= \setsem{\Gamma;\Phi \,|\, \Delta \vdash \bot_\tau t : \tau}(\pi_i\rho) 
  \end{split}
  \]
\item for $\Gamma;\Phi \,|\, \Delta \vdash \top : \onet$ is analogous to
  $\Gamma;\Phi \,|\, \Delta \vdash \bot_\tau t : \tau$
\item for $\Gamma;\Phi \,|\, \Delta \vdash (s,t) : \sigma \times \tau$ we have
  \[
  \begin{split}
    \pi_i(\relsem{\Gamma;\Phi \,|\, \Delta \vdash (s,t) : \sigma \times \tau} \rho)
    &= \pi_i(\relsem{\Gamma;\Phi \,|\, \Delta \vdash s: \sigma} \rho
      \times \relsem{\Gamma;\Phi \,|\, \Delta \vdash t: \tau} \rho) \\
    &= \pi_i(\relsem{\Gamma;\Phi \,|\, \Delta \vdash s: \sigma} \rho)
      \times \pi_i(\relsem{\Gamma;\Phi \,|\, \Delta \vdash t: \tau} \rho) \\
    &= \setsem{\Gamma;\Phi \,|\, \Delta \vdash s: \sigma} (\pi_i\rho)
      \times \setsem{\Gamma;\Phi \,|\, \Delta \vdash t: \tau} (\pi_i\rho) \\
    &= \setsem{\Gamma;\Phi \,|\, \Delta \vdash (s,t) : \sigma \times \tau} (\pi_i\rho)
  \end{split}
  \]
\item for $\Gamma;\Phi \,|\, \Delta \vdash \pi_1 t : \sigma$ we have
  \[
  \begin{split}
    \pi_i(\relsem{\Gamma;\Phi \,|\, \Delta \vdash \pi_1 t : \sigma} \rho)
    &= \pi_i(\pi_1 \circ \relsem{\Gamma;\Phi \,|\, \Delta \vdash t : \sigma \times \tau}\rho) \\
    &= \pi_1 \circ \pi_i(\relsem{\Gamma;\Phi \,|\, \Delta \vdash t : \sigma \times \tau}\rho) \\
    &= \pi_1 \circ \setsem{\Gamma;\Phi \,|\, \Delta \vdash t : \sigma \times \tau}(\pi_i\rho) \\
    &= \setsem{\Gamma;\Phi \,|\, \Delta \vdash \pi_1 t : \sigma} (\pi_i\rho)
  \end{split}
  \]
\item for $\Gamma;\Phi \,|\, \Delta \vdash \pi_2 t : \sigma$
  is analogous to $\Gamma;\Phi \,|\, \Delta \vdash \pi_1 t : \sigma$.
\item for $\Gamma;\Phi~|~\Delta \vdash \case{t}{x \mapsto l}{y \mapsto r}: \gamma$ we have
  \[
  \begin{split}
    &\hspace{-0.3in} \pi_i(\relsem{\Gamma;\Phi~|~\Delta \vdash \case{t}{x \mapsto l}{y \mapsto r}: \gamma}\rho) \\
    &= \pi_i(\eval \circ \langle
        \curry \,[
          \relsem{\Gamma;\Phi \,|\, \Delta, x : \sigma \vdash l : \gamma}\rho,
          \relsem{\Gamma;\Phi \,|\, \Delta, y: \tau \vdash r : \gamma}\rho
        ],
        \relsem{\Gamma;\Phi \,|\, \Delta \vdash t :\sigma + \tau} \rho
      \rangle) \\
    &= \eval \circ \langle
        \curry \,[
          \pi_i(\relsem{\Gamma;\Phi \,|\, \Delta, x : \sigma \vdash l : \gamma}\rho),
          \pi_i(\relsem{\Gamma;\Phi \,|\, \Delta, y: \tau \vdash r : \gamma}\rho)
        ],
        \pi_i(\relsem{\Gamma;\Phi \,|\, \Delta \vdash t :\sigma + \tau} \rho)
      \rangle \\
    &= \eval \circ \langle
        \curry \,[
          \setsem{\Gamma;\Phi \,|\, \Delta, x : \sigma \vdash l : \gamma}(\pi_i\rho),
          \setsem{\Gamma;\Phi \,|\, \Delta, y: \tau \vdash r : \gamma}(\pi_i\rho)
        ],
        \setsem{\Gamma;\Phi \,|\, \Delta \vdash t :\sigma + \tau}(\pi_i\rho)
      \rangle \\
    &= \setsem{\Gamma;\Phi~|~\Delta \vdash \case{t}{x \mapsto l}{y \mapsto r}: \gamma}(\pi_i\rho)
  \end{split}
  \]
\item for $\Gamma;\Phi \,|\, \Delta \vdash \inl \,s: \sigma + \tau$ we have
  \[
  \begin{split}
    \pi_i(\relsem{\Gamma;\Phi \,|\, \Delta \vdash \inl \,s: \sigma + \tau} \rho)
    &=\pi_i(\inl \circ \relsem{\Gamma;\Phi \,|\, \Delta \vdash s: \sigma}\rho) \\
    &=\inl \circ \pi_i(\relsem{\Gamma;\Phi \,|\, \Delta \vdash s: \sigma}\rho) \\
    &=\inl \circ \setsem{\Gamma;\Phi \,|\, \Delta \vdash s: \sigma}(\pi_i\rho) \\
    &= \setsem{\Gamma;\Phi \,|\, \Delta \vdash \inl \,s: \sigma + \tau} (\pi_i\rho)
  \end{split}
  \]
\item for $\Gamma;\Phi \,|\, \Delta \vdash \inr \,t: \sigma + \tau$
  is analogous to $\Gamma;\Phi \,|\, \Delta \vdash \inl \,t: \sigma + \tau$.
\item for $\Gamma;\Phi \,|\, \Delta
      \vdash \tin\,t : (\mu \phi.\lambda {\overline \alpha}.H){\overline \tau}$ we have
  \[
  \begin{split}
    &\hspace{-0.3in} \pi_i(\relsem{\Gamma;\Phi \,|\, \Delta
      \vdash \tin\,t : (\mu \phi.\lambda {\overline \alpha}.H){\overline \tau}} \rho) \\
    &= \pi_i(\mathit{in} \circ
      \relsem{\Gamma;\Phi \,|\, \Delta \vdash t : H[\phi := \mu
      \phi.\lambda {\overline \alpha}.H][\overline{\alpha := \tau}]} \rho) \\
    &= \mathit{in} \circ
      \pi_i(\relsem{\Gamma;\Phi \,|\, \Delta \vdash t : H[\phi := \mu
      \phi.\lambda {\overline \alpha}.H][\overline{\alpha := \tau}]} \rho) \\
    &= \mathit{in} \circ
      \setsem{\Gamma;\Phi \,|\, \Delta \vdash t : H[\phi := \mu
      \phi.\lambda {\overline \alpha}.H][\overline{\alpha := \tau}]} (\pi_i\rho) \\
    &= \setsem{\Gamma;\Phi \,|\, \Delta
      \vdash \tin_H\,t : (\mu \phi.\lambda {\overline \alpha}.H){\overline \tau}} (\pi_i\rho)
  \end{split}
  \]
\item for $\Gamma;\emptyset \,|\, \Delta \vdash \fold^F_H\, t
  : \Nat^{\overline \alpha}\, ((\mu\phi.\lambda \overline \beta.H)\overline \alpha)\,F$ we have
  \[
  \begin{split}
    &\hspace{-0.3in} \pi_i(\relsem{\Gamma;\emptyset \,|\, \Delta \vdash
      \fold^F_H\, t : \Nat^{\overline \alpha}\,((\mu
      \phi.\lambda \overline \beta.H)\overline \alpha)\,F}\rho) \\
    &= \pi_i(\mathit{fold} \circ \relsem{\Gamma;\emptyset \,|\, \Delta \vdash t : 
      \Nat^{\overline \alpha}\,(H[\phi := F][\overline{\beta := \alpha}])\,F}\rho) \\
    &= \mathit{fold} \circ \pi_i(\relsem{\Gamma;\emptyset \,|\, \Delta \vdash t : 
      \Nat^{\overline \alpha}\,(H[\phi := F][\overline{\beta := \alpha}])\,F}\rho) \\
    &= \mathit{fold} \circ \setsem{\Gamma;\emptyset \,|\, \Delta \vdash t : 
      \Nat^{\overline \alpha}\,(H[\phi := F][\overline{\beta := \alpha}])\,F}(\pi_i\rho) \\
    &= \setsem{\Gamma;\emptyset \,|\, \Delta \vdash
      \fold^F_H\, t : \Nat^{\overline \alpha}\,((\mu
      \phi.\lambda \overline \beta.H)\overline \alpha)\,F}(\pi_i\rho) \qedhere
  \end{split}
  \]
\end{itemize}
\end{proof}
