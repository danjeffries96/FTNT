% For double-blind review submission, w/o CCS and ACM Reference (max
% submission space)
\documentclass[acmsmall,review,anonymous]{acmart}
\settopmatter{printfolios=true,printccs=false,printacmref=false}
%% For double-blind review submission, w/ CCS and ACM Reference
%\documentclass[acmsmall,review,anonymous]{acmart}\settopmatter{printfolios=true}
%% For single-blind review submission, w/o CCS and ACM Reference (max submission space)
%\documentclass[acmsmall,review]{acmart}\settopmatter{printfolios=true,printccs=false,printacmref=false}
%% For single-blind review submission, w/ CCS and ACM Reference
%\documentclass[acmsmall,review]{acmart}\settopmatter{printfolios=true}
%% For final camera-ready submission, w/ required CCS and ACM Reference
%\documentclass[acmsmall]{acmart}\settopmatter{}


%% Journal information
%% Supplied to authors by publisher for camera-ready submission;
%% use defaults for review submission.
\acmJournal{PACMPL}
\acmVolume{1}
\acmNumber{POPL} % CONF = POPL or ICFP or OOPSLA
\acmArticle{1}
\acmYear{2020}
\acmMonth{1}
\acmDOI{} % \acmDOI{10.1145/nnnnnnn.nnnnnnn}
\startPage{1}

%% Copyright information
%% Supplied to authors (based on authors' rights management selection;
%% see authors.acm.org) by publisher for camera-ready submission;
%% use 'none' for review submission.
\setcopyright{none}
%\setcopyright{acmcopyright}
%\setcopyright{acmlicensed}
%\setcopyright{rightsretained}
%\copyrightyear{2018}           %% If different from \acmYear

%% Bibliography style
\bibliographystyle{ACM-Reference-Format}
%% Citation style
%% Note: author/year citations are required for papers published as an
%% issue of PACMPL.
\citestyle{acmauthoryear}   %% For author/year citations
%\citestyle{acmnumeric}

%%%%%%%%%%%%%%%%%%%%%%%%%%%%%%%%%%%%%%%%%%%%%%%%%%%%%%%%%%%%%%%%%%%%%%
%% Note: Authors migrating a paper from PACMPL format to traditional
%% SIGPLAN proceedings format must update the '\documentclass' and
%% topmatter commands above; see 'acmart-sigplanproc-template.tex'.
%%%%%%%%%%%%%%%%%%%%%%%%%%%%%%%%%%%%%%%%%%%%%%%%%%%%%%%%%%%%%%%%%%%%%%



\usepackage[utf8]{inputenc}
\usepackage{ccicons}
\usepackage{verbatim}

\usepackage{amsmath}
\usepackage{amsthm}
\usepackage{amscd}
%\usepackage{MnSymbol}
\usepackage{xcolor}

\usepackage{bbold}
\usepackage{url}
\usepackage{upgreek}
%\usepackage{stmaryrd}

\usepackage{lipsum}
\usepackage{tikz-cd}
\usetikzlibrary{cd}
\usetikzlibrary{calc}
\usetikzlibrary{arrows}

\usepackage{bussproofs}
\EnableBpAbbreviations

\DeclareMathAlphabet{\mathpzc}{OT1}{pzc}{m}{it}

%\usepackage[amsmath]{ntheorem}

\newcommand{\lam}{\lambda}
\newcommand{\eps}{\varepsilon}
\newcommand{\ups}{\upsilon}
\newcommand{\mcB}{\mathcal{B}}
\newcommand{\mcD}{\mathcal{D}}
\newcommand{\mcE}{\mathcal{E}}
\newcommand{\mcF}{\mathcal{F}}
\newcommand{\mcP}{\mathcal{P}}
\newcommand{\mcI}{\mathcal{I}}
\newcommand{\mcJ}{\mathcal{J}}
\newcommand{\mcK}{\mathcal{K}}
\newcommand{\mcL}{\mathcal{L}}
\newcommand{\WW}{\mathcal{W}}

\newcommand{\ex}{\mcE_x}
\newcommand{\ey}{\mcE_y}
\newcommand{\bzero}{\boldsymbol{0}}
\newcommand{\bone}{{\boldsymbol{1}}}
\newcommand{\tB}{{\bone_\mcB}}
\newcommand{\tE}{{\bone_\mcE}}
\newcommand{\bt}{\mathbf{t}}
\newcommand{\bp}{\mathbf{p}}
\newcommand{\bsig}{\mathbf{\Sigma}}
\newcommand{\bpi}{\boldsymbol{\pi}}
\newcommand{\Empty}{\mathtt{Empty}}
\newcommand{\truthf}{\mathtt{t}}
\newcommand{\id}{id}
\newcommand{\coo}{\mathtt{coo\ }}
\newcommand{\mcC}{\mathcal{C}}
\newcommand{\Rec}{\mathpzc{Rec}}
\newcommand{\types}{\mathcal{T}}

%\newcommand{\Homrel}{\mathsf{Hom_{Rel}}}
\newcommand{\HomoCPOR}{\mathsf{Hom_{\oCPOR}}}

%\newcommand{\semof}[1]{\llbracket{#1}\rrbracket^\rel}
\newcommand{\sem}[1]{\llbracket{#1}\rrbracket}
\newcommand{\setsem}[1]{\llbracket{#1}\rrbracket^\set}
\newcommand{\relsem}[1]{\llbracket{#1}\rrbracket^\rel}
\newcommand{\dsem}[1]{\llbracket{#1}\rrbracket^{\mathsf D}}
\newcommand{\setenv}{\mathsf{SetEnv}}
\newcommand{\relenv}{\mathsf{RelEnv}}
\newcommand{\oCPOenv}{\mathsf{SetEnv}}
\newcommand{\oCPORenv}{\mathsf{RelEnv}}
\newcommand{\oCPOsem}[1]{\llbracket{#1}\rrbracket^{\set}}
\newcommand{\oCPORsem}[1]{\llbracket{#1}\rrbracket^{\rel}}
\newcommand{\denv}{\mathsf{DEnv}}

\newcommand{\rel}{\mathsf{Rel}}
\newcommand{\setof}[1]{\{#1\}}
\newcommand{\letin}[1]{\texttt{let }#1\texttt{ in }}
\newcommand{\comp}[1]{{\{#1\}}}
\newcommand{\bcomp}[1]{\{\![#1]\!\}}
\newcommand{\beps}{\boldsymbol{\eps}}
%\newcommand{\B}{\mcB}
%\newcommand{\Bo}{{|\mcB|}}

\newcommand{\lmt}{\longmapsto}
\newcommand{\RA}{\Rightarrow}
\newcommand{\LA}{\Leftarrow}
\newcommand{\rras}{\rightrightarrows}
\newcommand{\colim}[2]{{{\underrightarrow{\lim}}_{#1}{#2}}}
\newcommand{\lift}[1]{{#1}\,{\hat{} \; \hat{}}}
\newcommand{\graph}[1]{\langle {#1} \rangle}

\newcommand{\carAT}{\mathsf{car}({\mathcal A}^T)}
\newcommand{\isoAto}{\mathsf{Iso}({\mcA^\to})}
\newcommand{\falg}{\mathsf{Alg}_F}
\newcommand{\CC}{\mathsf{Pres}(\mathcal{A})}
\newcommand{\PP}{\mathcal{P}}
\newcommand{\DD}{D_{(A,B,f)}}
\newcommand{\from}{\leftarrow}
\newcommand{\upset}[1]{{#1}{\uparrow}}
\newcommand{\smupset}[1]{{#1}\!\uparrow}

\newcommand{\Coo}{\mathpzc{Coo}}
\newcommand{\code}{\#}
\newcommand{\nat}{\mathpzc{Nat}}

\newcommand{\eq}{\; = \;}
\newcommand{\of}{\; : \;}
\newcommand{\df}{\; := \;}
\newcommand{\bnf}{\; ::= \;}

\newcommand{\zmap}[1]{{\!{\between\!\!}_{#1}\!}}
\newcommand{\bSet}{\mathbf{Set}}

\newcommand{\dom}{\mathsf{dom}}
\newcommand{\cod}{\mathsf{cod}}
\newcommand{\adjoint}[2]{\mathrel{\mathop{\leftrightarrows}^{#1}_{#2}}}
\newcommand{\isofunc}{\mathpzc{Iso}}
\newcommand{\ebang}{{\eta_!}}
\newcommand{\lras}{\leftrightarrows}
\newcommand{\rlas}{\rightleftarrows}
\newcommand{\then}{\quad\Longrightarrow\quad}
\newcommand{\hookup}{\hookrightarrow}

\newcommand{\spanme}[5]{\begin{CD} #1 @<#2<< #3 @>#4>> #5 \end{CD}}
\newcommand{\spanm}[3]{\begin{CD} #1 @>#2>> #3\end{CD}}
\newcommand{\pushout}{\textsf{Pushout}}
\newcommand{\mospace}{\qquad\qquad\!\!\!\!}

\newcommand{\natur}[2]{#1 \propto #2}

\newcommand{\Tree}{\mathsf{Tree}\,}
\newcommand{\GRose}{\mathsf{GRose}\,}
\newcommand{\List}{\mathsf{List}\,}
\newcommand{\PTree}{\mathsf{PTree}\,}
\newcommand{\Bush}{\mathsf{Bush}\,}
\newcommand{\Forest}{\mathsf{Forest}\,}
\newcommand{\Lam}{\mathsf{Lam}\,}
\newcommand{\LamES}{\mathsf{Lam}^+}
\newcommand{\Expr}{\mathsf{Expr}\,}

\newcommand{\ListNil}{\mathsf{Nil}}
\newcommand{\ListCons}{\mathsf{Cons}}
\newcommand{\LamVar}{\mathsf{Var}}
\newcommand{\LamApp}{\mathsf{App}}
\newcommand{\LamAbs}{\mathsf{Abs}}
\newcommand{\LamSub}{\mathsf{Sub}}
\newcommand{\ExprConst}{\mathsf{Const}}
\newcommand{\ExprPair}{\mathsf{Pair}}
\newcommand{\ExprProj}{\mathsf{Proj}}
\newcommand{\ExprAbs}{\mathsf{Abs}}
\newcommand{\ExprApp}{\mathsf{App}}
\newcommand{\Ptree}{\mathsf{Ptree}}

\newcommand{\kinds}{\mathpzc{K}}
\newcommand{\tvars}{\mathbb{T}}
\newcommand{\fvars}{\mathbb{F}}
\newcommand{\consts}{\mathpzc{C}}
\newcommand{\Lan}{\mathsf{Lan}}
\newcommand{\zerot}{\mathbb{0}}
\newcommand{\onet}{\mathbb{1}}
\newcommand{\bool}{\mathbb{2}}
\renewcommand{\nat}{\mathbb{N}}
%\newcommand{\semof}[1]{[\![#1]\!]}
%\newcommand{\setsem}[1]{\llbracket{#1}\rrbracket^\set}
\newcommand{\predsem}[1]{\llbracket{#1}\rrbracket^\pred}
%\newcommand{\todot}{\stackrel{.}{\to}}
\newcommand{\todot}{\Rightarrow}
\newcommand{\bphi}{{\bm \phi}}

\newcommand{\bm}[1]{\boldsymbol{#1}}

\newcommand{\cL}{\mathcal{L}}
\newcommand{\T}{\mathcal{T}}
\newcommand{\Pos}{P\!}
%\newcommand{\Pos}{\mathcal{P}\!}
\newcommand{\Neg}{\mathcal{N}}
\newcommand{\Hf}{\mathcal{H}}
\newcommand{\V}{\mathbb{V}}
\newcommand{\I}{\mathcal{I}}
\newcommand{\Set}{\mathsf{Set}}
%\newcommand{\Nat}{\mathsf{Nat}}
\newcommand{\Homrel}{\mathsf{Hom_{Rel}}}
\newcommand{\CV}{\mathcal{CV}}
\newcommand{\lan}{\mathsf{Lan}}
\newcommand{\Id}{\mathit{Id}}
\newcommand{\mcA}{\mathcal{A}}
\newcommand{\inl}{\mathsf{inl}}
\newcommand{\inr}{\mathsf{inr}}
%\newcommand{\case}[3]{\mathsf{case}\,{#1}\,\mathsf{of}\,\{{#2};\,{#3}\}}
\newcommand{\tin}{\mathsf{in}}
\newcommand{\fold}{\mathsf{fold}}
\newcommand{\Eq}{\mathsf{Eq}}
\newcommand{\Hom}{\mathsf{Hom}}
\newcommand{\curry}{\mathsf{curry}}
\newcommand{\uncurry}{\mathsf{uncurry}}
\newcommand{\eval}{\mathsf{eval}}
\newcommand{\apply}{\mathsf{apply}}
\newcommand{\oCPO}{{\mathsf{Set}}}
\newcommand{\oCPOR}{{\mathsf{Rel}}}
\newcommand{\oCPORT}{{\mathsf{RT}}}

\newcommand{\ar}[1]{\##1}
\newcommand{\mcG}{\mathcal{G}}
\newcommand{\mcH}{\mathcal{H}}
\newcommand{\TV}{\mathpzc{V}}

\newcommand{\essim}[1]{\mathsf{EssIm}(#1)}
\newcommand{\hra}{\hookrightarrow}

\newcommand{\ol}[1]{\overline{#1}}
\newcommand{\ul}[1]{\underline{#1}}
\newcommand{\op}{\mathsf{op}}
\newcommand{\trige}{\trianglerighteq}
\newcommand{\trile}{\trianglelefteq}
\newcommand{\LFP}{\mathsf{LFP}}
\newcommand{\LAN}{\mathsf{Lan}}
%\newcommand{\Mu}{{\mu\hskip-4pt\mu}}
\newcommand{\Mu}{{\mu\hskip-5.5pt\mu}}
%\newcommand{\Mu}{\boldsymbol{\upmu}}
\newcommand{\Terms}{\mathpzc{Terms}}
\newcommand{\Ord}{\mathpzc{Ord}}
\newcommand{\Anote}[1]{{\color{blue} {#1}}}
\newcommand{\Pnote}[1]{{\color{red} {#1}}}

\newcommand{\greyout}[1]{{\color{gray} {#1}}}
\newcommand{\ora}[1]{\overrightarrow{#1}}

%\newcommand{\?}{{.\ }}
%\theoremheaderfont{\scshape}
%\theorembodyfont{\normalfont}
%\theoremseparator{.\ \ }
\newtheorem{thm}{Theorem}
\newtheorem{dfn}[thm]{Definition}
\newtheorem{prop}[thm]{Proposition}
\newtheorem{cor}[thm]{Corollary}
\newtheorem{lemma}[thm]{Lemma}
\newtheorem{rmk}[thm]{Remark}
\newtheorem{expl}[thm]{Example}
\newtheorem{notn}[thm]{Notation}
%\theoremstyle{nonumberplain}
%\theoremsymbol{\Box}


\theoremstyle{definition}
\newtheorem{exmpl}{Example}

\renewcommand{\greyout}[1]{} %{{\color{gray} {#1}}} -- toggle to remove greyed text

\newcommand{\emptyfun}{{[]}}
\newcommand{\cal}{\mathcal}
%\newcommand{\fold}{\mathit{fold}}
\newcommand{\F}{\mathcal{F}}
\renewcommand{\G}{\mathcal{G}}
\newcommand{\N}{\mathcal{N}}
\newcommand{\E}{\mathcal{E}}
\newcommand{\B}{\mathcal{B}}
\renewcommand{\P}{\mathcal{A}}
\newcommand{\pred}{\mathsf{Fam}}
\newcommand{\env}{\mathsf{Env}}
\newcommand{\set}{\mathsf{Set}}
\renewcommand{\S}{\mathcal S}
\renewcommand{\C}{\mathcal{C}}
\newcommand{\D}{\mathcal{D}}
\newcommand{\A}{\mathcal{A}}
\renewcommand{\id}{\mathit{id}}
\newcommand{\map}{\mathsf{map}}
\newcommand{\pid}{\underline{\mathit{id}}}
\newcommand{\pcirc}{\,\underline{\circ}\,}
\newcommand{\pzero}{\underline{0}}
\newcommand{\pone}{\underline{1}}
\newcommand{\psum}{\,\underline{+}\,}
\newcommand{\pinl}{\underline{\mathit{inL}}\,}
\newcommand{\pinr}{\underline{\mathit{inR}}\,}
\newcommand{\ptimes}{\,\underline{\times}\,}
\newcommand{\ppi}{\underline{\pi_1}}
\newcommand{\pppi}{\underline{\pi_2}}
\newcommand{\pmu}{\underline{\mu}}
\newcommand{\semmap}{\mathit{map}}
\newcommand{\subst}{\mathit{subst}}

\newcommand{\tb}[1]{~~ \mbox{#1} ~~}
\newcommand{\listt}[1]{(\mu \phi. \lambda \beta . \onet + \beta \times
  \phi \beta) #1} 
\newcommand{\filtype}{\Nat^\emptyset 
  \, (\Nat^\emptyset \, \alpha \, \mathit{Bool}) (\Nat^\emptyset \,
  (List \, \alpha) \, (List \, \alpha))} 
\newcommand{\maplist}{\map_{\lambda A. \setsem{\emptyset; \alpha
      \vdash List \, \alpha} \rho[\alpha := A]}} 
\newcommand{\PLeaves}{\mathsf{PLeaves}}
\newcommand{\swap}{\mathsf{swap}}
\newcommand{\reverse}{\mathsf{reverse}}
\newcommand{\Bcons}{\mathit{Bcons}}
\newcommand{\Bnil}{\mathit{Bnil}}

\title[Free Theorems for Nested Types]{Parametricity and Free Theorems for 
Nested Types} %% [Short Title] is optional;
                                        %% when present, will be used in
                                        %% header instead of Full Title.
%\titlenote{with title note}             %% \titlenote is optional;
                                        %% can be repeated if necessary;
                                        %% contents suppressed with 'anonymous'
%\subtitle{Subtitle}                     %% \subtitle is optional
%\subtitlenote{with subtitle note}       %% \subtitlenote is optional;
                                        %% can be repeated if necessary;
                                        %% contents suppressed with 'anonymous'


%% Author information
%% Contents and number of authors suppressed with 'anonymous'.
%% Each author should be introduced by \author, followed by
%% \authornote (optional), \orcid (optional), \affiliation, and
%% \email.
%% An author may have multiple affiliations and/or emails; repeat the
%% appropriate command.
%% Many elements are not rendered, but should be provided for metadata
%% extraction tools.

%% Author with single affiliation.
\author{Patricia Johann, Enrico Ghiorzi, and Daniel Jeffries}
%\authornote{with author1 note}          %% \authornote is optional;
%                                        %% can be repeated if necessary
%\orcid{nnnn-nnnn-nnnn-nnnn}             %% \orcid is optional
\affiliation{
%  \position{Position1}
%  \department{Department1}              %% \department is recommended
  \institution{Appalachian State University}            %% \institution is required
%  \streetaddress{Street1 Address1}
%  \city{City1}
%  \state{State1}
%  \postcode{Post-Code1}
%  \country{Country1}                    %% \country is recommended
}
\email{johannp@appstate.edu, ghiorzie@appstate.edu, jeffriesd@appstate.edu}          %% \email is recommended


\begin{document}

\begin{abstract}
Abstract goes here
\end{abstract}

%\begin{CCSXML}
%<ccs2012>
%<concept>
%<concept_id>10011007.10011006.10011008</concept_id>
%<concept_desc>Software and its engineering~General programming languages</concept_desc>
%<concept_significance>500</concept_significance>
%</concept>
%<concept>
%<concept_id>10003456.10003457.10003521.10003525</concept_id>
%<concept_desc>Social and professional topics~History of programming languages</concept_desc>
%<concept_significance>300</concept_significance>
%</concept>
%</ccs2012>
%\end{CCSXML}
%
%\ccsdesc[500]{Software and its engineering~General programming languages}
%\ccsdesc[300]{Social and professional topics~History of programming languages}
%% End of generated code


%% Keywords
%% comma separated list
%\keywords{keyword1, keyword2, keyword3}  %% \keywords is optional


\maketitle

\section{Introduction}\label{sec:intro}

Suppose we wanted to prove some property of programs over an algebraic
data type (ADT) such as that of lists, coded in Agda
%\footnote{All code appearing in this paper are written in Agda.}
as
{\small
\[\begin{array}{l}
\mathtt{data\; List \;(A : Set)\;:\;Set\;where}\\
\hspace*{0.4in}\mathtt{nil\;:\; List\;A}\\
\hspace*{0.4in}\mathtt{Cons\;:\;A \rightarrow List\;A \rightarrow List\;A}
\end{array}\]}
A natural approach to the problem uses structural induction on the
input data structure in question. This requires knowing not just the
definition of the ADT of which the input data structure is an
instance, but also the program text for the functions involved in the
properties to be proved. For example, to prove by induction that
mapping a polymorphic function over a list and then reversing the
resulting list is the same as reversing the original list and then
mapping the function over the result, we unwind the (recursive)
definitions of the $\mathtt{reverse}$ and $\mathtt{map}$ functions
over lists to according to the inductive structure of the input
list. Such data-driven induction proofs over ADTs are so routine that
they are often included in, say, undergraduate functional programming
courses.

An alternative technique for proving results like the above
$\mathtt{map}$-$\mathtt{reverse}$ property for lists is to use
parametricity, a formalization of extensional type-uniformity in
polymorphic languages. Parametricity captures the intuition that a
polymorphic program must act uniformly on all of its possible type
instantiations; it is formalized as the requirement that every
polymorphic program preserves all relations between any pair of types
that it is instantiated with.  Parametricity was originally put forth
by Reynolds~\cite{rey83} for System F~\cite{gr89}, the formal calculus
at the core of all polymorphic functional languages. It was later
popularized as Wadler's ``theorems for free''~\cite{wad89} because it
allows the deduction of many properties of programs in such languages
solely from their types, i.e., with no knowledge whatsoever of the
text of the programs involved. To get interesting free theorems,
Wadler's calculus included, implicitly, built-in list types; indeed,
most of the free theorems in~\cite{wad89} are consequences of
naturality for polymorphic list-processing functions. However,
parametricity can also be used to prove naturality properties for
non-list ADTs, as well as results, like correctness of program
optimizations like {\em short cut fusion}~\cite{glp93,joh02,joh03},
that go beyond simple naturality.
%In addition, parametricity in the presence of ADTs
%{\color{blue} check!}  has further been used to prove i) {\em
%  properties of programs}, e.g., that they perform no illegal
%operations, satisfy certain security criteria~\cite{ep03,rp10}, or are
%observationally equivalent to their compiled forms~\cite{bh09,hd11};
%ii) {\em whole-language properties}, e.g., that they support data
%abstraction and modularity via representation independence,
%\cite{bm05,dr04,jac99,kat11,mr92}, enforce information flow
%policies~\cite{ss01,ts04}, or guarantee privacy~\cite{rp10}; and iii)
%{\em properties of implementations}, e.g., compiler
%correctness~\cite{ab08,bh09,hd11,glpj93}. Parametricity is thus an
%important tool for reasoning about programs in polymorphic languages
%that natively support ADTs. {\color{blue} Be clear about what this
%  means. Re-word?}

This paper is about parametricity and free theorems for a polymorphic
calculus with explicit syntax not just for ADTs, but for nested types
as well. An ADT defines a {\em family of inductive data types}, one
for each input type. For example, the $\mathtt{List}$ data type
definition above defines a collection of data types $\mathtt{List\;
  A}$, $\mathtt{List\; B}$, $\mathtt{List\; (A \times B)}$,
$\mathtt{List\; (\List\;A)}$, etc., each independent of all the
others. By contrast, a nested type~\cite{bm98} is an {\em inductive
  family of data types} that is defined over, or is defined mutually
recursively with, (other) such data types. Since the structures of the
data type at one type can depend on those at other types, the entire
family of types must be defined at once. Examples of nested types
include, trivially, ordinary ADTs, such as list and tree types; simple
nested types, such as the data type {\small
\[\begin{array}{l}
\mathtt{data\; PTree\;(A : Set)\;:\;Set\;where}\\
\hspace*{0.4in}\mathtt{pleaf\;:\;A \rightarrow PTree\;A}\\
\hspace*{0.4in}\mathtt{pnode\;:\;PTree\;(A \times A) \rightarrow PTree\;A}
\end{array}\]}
\hspace{-0.04in}of perfect trees, whose recursive occurrences never
appear below other type constructors; ``deep'' nested
types~\cite{jp20}, such as the data type {\small
\[\begin{array}{l}
\mathtt{data\; Forest\;(A : Set)\;:\;Set\;where}\\
\hspace*{0.4in}\mathtt{fempty\;:\;Forest\;A}\\
\hspace*{0.4in}\mathtt{fnode\;:\; A \rightarrow PTree\;(Forest\;A) \to
Forest\;A}
\end{array}\]}
\hspace{-0.04in}of perfect forests, whose recursive occurrences appear
below type constructors for other nested types; and truly nested
types\footnote{Nested types that are defined over themselves are known
  as {\em truly nested types}.}, such as the data type {\small
\[\begin{array}{l}
\mathtt{data\; Bush\;(A : Set)\;:\;Set\;where}\\
\hspace*{0.4in}\mathtt{bnil\;:\; Bush\;A}\\
\hspace*{0.4in}\mathtt{bcons\;:\;A \rightarrow Bush\;(Bush \; A)
  \rightarrow Bush\;A} 
\end{array}\]}
\hspace{-0.04in}of bushes (also called {\em bootstrapped heaps}
in~\cite{oka99}), whose recursive occurrences appear below their own
type constructors.

Suppose we now want to prove properties of functions over nested
types. We might, for example, want to prove a
$\mathtt{map}$-$\mathtt{reverse}$ property for the functions on
perfect trees in Figure~\ref{fig:funs}, or for those on
bushes\footnote{To define the $\mathtt{foldBush}$ and
  $\mathtt{mapBush}$ functions in Figure~\ref{fig:funs2} it is
  necessary to turn off Agda's termination checker.} in
Figure~\ref{fig:funs2}. A few well-chosen examples quickly convince us
that such a property should indeed hold for perfect trees, and,
drawing inspiration from the situation for ADTs, we easily construct a
proof by induction on the input perfect tree. To formally establish
this result, we could even prove it in Coq or Agda: each of these
provers actually generates an induction rule for perfect trees and the
generated rule gives the expected result because proving properties of
perfect trees requires only that we induct over the top-level perfect
tree in the recursive position, leaving any data internal to the input
tree untouched.
\begin{figure*}
\hspace*{-0.5in}
\resizebox{0.35\linewidth}{!}{
\begin{minipage}[t]{0.5\textwidth}
\[\begin{array}{l}
\mathtt{reversePTree : \forall \{A : Set\}  \rightarrow PTree\; A
  \rightarrow PTree \;A}\\  
\mathtt{reversePTree \; \{A\} = foldPTree \; \{A\}\; \{PTree\}}\\
\hspace*{1.3in} \mathtt{pleaf}\\
\hspace*{1.3in} \mathtt{( \lambda p \rightarrow pnode\; (mapPTree\;
  swap\; p) )}\\  
\\
\mathtt{foldPTree : \forall \{A : Set\} \rightarrow \{F : Set \rightarrow Set \}
  \rightarrow}\\
\hspace*{0.8in} \mathtt{( \{B : Set\} \rightarrow B \rightarrow F B) \rightarrow}\\
\hspace*{0.8in} \mathtt{(\{B : Set\} \rightarrow F (B \times B)
  \rightarrow F B) \rightarrow}\\
\hspace*{0.8in}\mathtt{PTree \;A \rightarrow F\;A}\\
\mathtt{foldPTree\; n\; c\; (pleaf\; x) = n \;x}\\
\mathtt{foldPTree\; n\; c \;(pnode\; p) = c\; (foldPTree \;n \;c \;p)}\\
\\
\mathtt{mapPTree : \forall \{A\, B : Set\} \rightarrow (A \rightarrow B)
  \rightarrow PLeaves\; A \rightarrow PLeaves\; B}\\ 
\mathtt{mapPTree \;f\; (pleaf \;x) = pleaf\; (f\; x)}\\
\mathtt{mapPTree\; f \;(pnode\; p) = pnode\; (mapPTree \;(\lambda p
  \rightarrow (f (\pi_1 \,p), f (\pi_2\,p)))\; p)}\\ 
\\
\mathtt{swap : \forall \{A : Set\} \rightarrow (A \times A)
  \rightarrow (A \times A)}\\ 
\mathtt{swap \;(x,y) = (y,x)}
\end{array}\]
\caption{$\mathtt{reversePTree}$ and auxiliary functions in Agda}\label{fig:funs} 
\end{minipage}}
\quad\quad\quad\quad\quad\quad\quad\quad
\resizebox{0.35\linewidth}{!}{
\begin{minipage}[t]{0.5\textwidth}
\[\begin{array}{l}
  \mathtt{reverseBush : \forall \{A : Set\} \rightarrow Bush\; A
  \rightarrow Bush\; A}\\ 
\mathtt{reverseBush \;\{A\} = foldBush\; \{A\}\; \{Bush\}\; bnil \;balg}\\
\\
\mathtt{foldBush : \forall \{A : Set\} \rightarrow \{F : Set \rightarrow
  Set\} \rightarrow}\\
\hspace*{0.5in} \mathtt{(\{B : Set\} \rightarrow F B) \rightarrow}\\
\hspace*{0.5in} \mathtt{(\{B : Set\} \rightarrow B \rightarrow F\;
  (F\; B) \rightarrow F\; B) \rightarrow}\\ 
\hspace*{0.5in} \mathtt{Bush\; A \rightarrow F \;A}\\
\mathtt{foldBush\; bn\; bc\; bnil = bn}\\
\mathtt{foldBush \;bn \;bc \;(bcons\; x \;bb) =}\\
\hspace*{0.5in} \mathtt{bc\; x \;(foldBush \;bn\; bc
  \;(mapBush \;(foldBush\; bn\; bc) \;bb))}\\
\\
\mathtt{mapBush : \forall \{A \,B : Set\} \to (A \to B) \to (Bush\, A) \to
  (Bush\, B)}\\ 
\mathtt{mapBush\; \_ \; bnil = bnil}\\
\mathtt{mapBush\; f\; (bcons\; x \;bb) = bcons \;(f\, x) \; (mapBush\;
  (mapBush\, f) \; bb)}\\ 
\\
\mathtt{balg : \forall \{B : Set\} \rightarrow B \rightarrow Bush
  \;(Bush\; B) \rightarrow Bush\; B}\\
\mathtt{balg \;x \;bnil = bcons\; x \;bnil}\\
\mathtt{balg \;x\; (bcons \;bnil \;bbbx) = bcons\; x \;(bcons \;bnil\; bbbx)}\\
\mathtt{balg\; x \;(bcons \;(bcons\; y \;bx)\; bbbx) =}\\
\hspace*{0.5in} \mathtt{bcons\; y\; (bcons\; (bcons \;x \;bx) \;bbbx)}\\
\end{array}\]
\caption{$\mathtt{reverseBush}$ and auxiliary
  functions in Agda}\label{fig:funs2} 
\end{minipage}}
\end{figure*}
%want to prove that $\mathtt{reversePTree}$ commutes with
%$\mathtt{mapPTree}$ or that $\mathtt{reverseBush}$ commutes with
%$\mathtt{mapBush}$. It is not hard to convince oneself of the former:
%a few well-chosen examples show clearly why mapping a polymorphic
%%function over a perfect tree and then reversing the data at the leaves
%of the resulting perfect tree is indeed the same result as mapping
%that same function over the result of reversing the data at the leaves
%of the original perfect tree.

Unfortunately, it is nowhere near as clear that analogous intuitive or
formal inductive arguments can be made for the
$\mathtt{map}$-$\mathtt{reverse}$ property for bushes. Indeed, a proof
by induction on the input bush must recursively induct over the bushes
that are internal to the top-level bush in the recursive position.
This is sufficiently delicate that no induction rule for bushes or
other truly nested types was known until very recently, when {\em deep
  induction}~\cite{jp20} was developed as a way to induct over {\em
  all} of the structured data present in an input. Deep induction thus
not only gave the first principled and practically useful structural
induction rules for bushes and other truly nested types, and has also
opened the way for incorporating automatic generation of such rules
for (truly) nested data types --- and, eventually, even GADTs --- into
modern proof assistants.

Of course it is great to know that we {\em can}, at last, prove
properties of programs over (truly) nested types by induction.  But
recalling that inductive proofs over ADTs can sometimes be
circumvented in the presence of parametricity, we might naturally ask:
\begin{verse}
{\em Can we derive properties of functions over (truly) nested types
  from paramertricity?}
\end{verse}
This paper answers the above question in the affirmative, by showing
how to construct a parametric model for a polymorphic calculus with
{\em explicit syntax} for nested types. We introduce our calculus in
Section~\ref{sec:calculus}.
%To achieve this, we first introduce in
%Section~\ref{sec:calculus} a polymorphic calculus supporting nested
%types.
At the type level, it is
%(because it also has $\Nat$-types),
the level-2-truncation of the higher-kinded calculus from~\cite{jp19},
augmented with a primitive type of natural transformations. To
represent nested types, it constructs type expressions not just from
type variables, as for expressions representing ADTs, but from
variables representing type constructors of various arities as
well. It also includes an explicit $\mu$-construct to represent
type-level recursion with respect to these type constructor
variables. The class of nested types thus represented is very robust
and includes all (truly) nested types known from the literature. In
Section~\ref{sec:type-interp} we construct set and relational
interpretations of the types in our calculus. As usual in parametric
models, types are interpreted as functors from environments
interpreting their type variable contexts to set or relations, as
appropriate. But in order to ensure that these functors satisfy the
cocontinuity properties needed to know that the fixpoints interpreting
$\mu$-types exist, set environments must map each $k$-ary type
constructor variable to an appropriately cocontinuous $k$-ary functor
on sets, and relation environments must map each $k$-ary type
constructor variable to an appropriately cocontinuous $k$-ary relation
transformer, and the cocontinuity conditions must be threaded
throughout our construction in such a way that the resulting model
still satisfies an appropriate Identity Extension Lemma
(Theorem~\ref{thm:iel}) can be proved. This turns out to be both
subtle and challenging, and Section~\ref{sec:iel}, where it is done,
is where the bulk of the work in our model construction lies.
%
%When constructing categorical models of parametricity of System F, it
%is standard to interpret types (judgments) as functors and terms as
%natural transformations. To accommodate $\mu$-types, these functors
%must be appropriately continuous to ensure existence of fixpoints. The
%required continuity properties must be propagated throughout the
%construction of the model. This must also be the case for the
%higher-order functors that underlie the interpretations of nested
%types. Ensuring sufficient cocontinuity is where the delicacy in our
%construction lies.
%
At the term level, our calculus includes constructs representing the
actions on morphisms of the functors interpreting types, initial
algebras of these functors, and structured recursion over elements of
these initial algebras (i.e., $\map$, $\tin$, and $\fold$ constructs,
respectively). While our calculus does not support general recursion
at the term level, it is strongly normalizing, so does provide strong
termination guarantees, and thus edges us toward the kind of practical
programming language supporting only restricted forms of recursion
proposed at the end of~\cite{wad89}. In Section~\ref{sec:term-interp},
we construct set and relational interpretations of the terms of our
calculus. As usual in parametric models, terms are interpreted as
natural transformations, from interpretations of the term contexts in
which they are formed to the interpretations of their types, that
cohere in what is essentially a fibrational way~\cite{gjfor15}.
Immediately from the definition of our interpretation we prove in
Section~\ref{sec:ft-nat} a scheme for deriving free theorems that are
consequences of naturality for functions polymorphic over nested
types. This scheme is very general, is parameterized over both the
data type and the polymorphic function in question, and has the
$\mathtt{map}$-$\mathtt{reverse}$ results described above as
instances. The relationship between naturality and parametricity has
long been of interest, and our inclusion of a primitive type of
natural transformations lets us clearly delineate free theorems that
are consequences of naturality, and thus would hold even in a
non-parametric model of such types, from those that use the full power
of the Abstraction Theorem to go beyond naturality. In
Section~\ref{sec:thms} we prove that our model satisfies an
Abstraction Theorem (Theorem~\ref{thm:abstraction}), which is the
basis for deriving more general free theorems. Finally, in
Section~\ref{sec:ftnt} we formulate and prove in our calculus a
variety of free theorems for nested types. We prove (non-)inhabitation
results in Sections~\ref{sec:bottom} and~\ref{sec:identity}, prove a
free theorem for the type of a filter function on generalized rose
trees in Section~\ref{sec:filter-grose}, and state and prove the
correctness of short cut fusion for nested types in
Section~\ref{sec:short-cut-nested}.

We are not the first to consider parametricity at higher
kinds. Atkey~\cite{atk12} constructs a parametric model for full
System F$\omega$, but within the impredicative Calculus of Inductive
Constructions (iCIC) rather than in a semantic category. Although
Atkey's construction is similar to ours, his focus is on higher kinds
rather than on modeling advanced data types directly. Specifically,
Atkey constructs a syntactic model and is not concerned with modeling
$\mu$-types, so he need not be concerned with functors or the
existence of their fixpoints. As a result, his relation transformers
include no cocontinuity conditions, and no such conditions need be
propagated throughout his model construction. But while Atkey does not
consider explicit syntax for nested types, or even ADTs, his model
does verify the existence of initial algebras for (syntactic
representations of) functors, provided they are {\em given}, together
with an identity- and composition-preserving $\mathit{fmap}_F$
function (which is not required to act like a true map
function). Atkey's ``functors'' are therefore postulated rather than
constructed, and need not have the expected functorial actions on
morphisms, and is it not at all clear which data types can be
represented in this way in System F$\omega$ or what free theorems for
such an extended system would look like, although we suspect that
making this explicit would result in a higher-kinded extension of our
calculus. By contrast, our calculus gives a specific syntax
delineating the nested types it is guaranteed to support, and all such
are functorial {\em by construction}.

Pitts~\cite{pit98,pit00} extends parametricity from pure System F to
accommodate fixpoint recursion at the level of types. Only list types
are added in~\cite{pit00}, although other ADTs are easily accommodated
as in~\cite{pit98}. Pitts considers only polynomial data types, all of
which can all be modeled as fixpoints of {\em first-order}
functors. It would be interesting to see how his operational model
would need to be extended to accommodate nested types and other data
types that are modeled as fixpoints of higher-order functors ---
especially to discern the analogues of the required functoriality and
cocontinuity are in his operational setting.

There is, of course, a long line of work on categorical models of
parametricity for System F; see, for
example,~\cite{bfss90,bm05,dr04,gjfor15,has94,jac99,mr92,rr94}. Some
of these even extend System F with fixpoints for certain classes of
functors {\color{blue} MORE??, usually those with (universal)
  strength}.  Although the present paper draws on this rich tradition,
we emphasize that our calculus does not have the impedicative
polymorphism of full System F. On the other hand, as far as we know
all of the free theorems derived in practice, even for ADTs, need only
the polymorphism supported by our system.

\section{The Calculus}\label{sec:calculus}

\subsection{Types}

For each $k \ge 0$, we assume countable sets $\tvars^k$ of \emph{type
  constructor variables of arity $k$} and $\fvars^k$ of
\emph{functorial variables of arity $k$}, all mutually disjoint.
%disjoint for distinct $k$ and disjoint from each other.
The sets of all type constructor variables and functorial variables
are $\tvars = \bigcup_{k \ge 0} \tvars^k$ and $\fvars = \bigcup_{k \ge
  0} \fvars^k$, respectively, and a \emph{type variable} is any
element of $\tvars \cup \fvars$.  We use lower case Greek letters for
type variables, writing $\phi^k$ to indicate that $\phi \in \tvars^k
\cup \fvars^k$, and omitting the arity indicator $k$ when convenient,
unimportant, or clear from context. We reserve letters from the
beginning of the alphabet to denote type variables of arity $0$, i.e.,
elements of $\tvars^0 \cup \fvars^0$. We write $\overline{\zeta}$ for
either a set $\{\zeta_1,...,\zeta_n\}$ of type constructor variables
or a set of functorial variables when the cardinality $n$ of the set
is unimportant or clear from context. If $\Pos\,$ is a set of type
variables we write $\Pos, \overline{\phi}$ for $\Pos\, \cup
\overline{\phi}$ when $\Pos\, \cap \overline{\phi} = \emptyset$.  We
omit the vector notation for a singleton set, thus writing $\phi$,
instead of $\overline{\phi}$, for $\{\phi\}$.

\begin{dfn}
Let $V$ be a finite subset of\, $\tvars$, let $\Pos$ be a finite
subset of\, $\fvars$, let $\overline{\alpha}$ be a finite subset of\,
$\fvars^0$ disjoint from $\Pos$, and let $\phi^k \in \fvars^k
\setminus \Pos$.  The sets $\T(V)$ of {\em type constructor
  expressions} over $V$ and $\mcF^\Pos(V)$ of {\em functorial
  expressions} over $\Pos$ and $V$ are given by
\[\T(V) \; ::= \; V \mid 
   \Nat^{\ol{\alpha}} \, \mcF^{\overline{\alpha}}(V) \;
   \mcF^{\ol{\alpha}}(V) \mid V \ol{\T(V)}
\]
\noindent
and 
\begin{align*}
\mcF^\Pos(V) \; ::= \; \T(V) &\mid
\zerot \mid \onet 
\mid \Pos\; \ol{\mcF^\Pos(V)}  \,
\mid V\, \ol{\mcF^\Pos(V)}  
\mid \mcF^{\Pos}(V) + \mcF^\Pos(V)
\mid \mcF^{\Pos}(V) \times \mcF^\Pos(V)\\
&\mid \left(\mu \phi^{~k}. \lambda \alpha_1...\alpha_k. 
\mcF^{\Pos,\alpha_1,...,\alpha_k,\phi}(V)\right)
  \ol{\mcF^{\Pos}(V)}
\end{align*}
\end{dfn}
\noindent
A \emph{type} over $\Pos$ and $V$ is any element of $\T(V) \cup
\F^\Pos(V)$.

The notation for types entails that an application
$\tau\tau_1...\tau_k$ is allowed only when $\tau$ is a type variable
of arity $k$, or $\tau$ is a subexpression of the form $\mu
\phi^{k}.\lambda \alpha_1...\alpha_k.\tau'$. Moreover, if $\tau$ has
arity $k$ then $\tau$ must be applied to exactly $k$ arguments.
Accordingly, an overbar indicates a sequence of subexpressions whose
length matches the arity of the type applied to it.  The fact that
types are always in \emph{$\eta$-long normal form} avoids having to
consider $\beta$-conversion at the level of types. In a subexpression
$\Nat^{\ol{\alpha}}\sigma\,\tau$, the $\Nat$ operator binds all
occurrences of the variables in $\ol{\alpha}$ in $\sigma$ and
$\tau$. Similarly, in a subexpression $\mu \phi^k.\lambda
\ol{\alpha}.\tau$, the $\mu$ operator binds all occurrences of the
variable $\phi$, and the $\lambda$ operator binds all occurrences of
the variables in $\ol{\alpha}$, in the body $\tau$.

A {\em type constructor context} is a finite set $\Gamma$ of type
constructor variables, and a {\em functorial context} is a finite set
$\Phi$ of functorial variables. In Definition~\ref{def:wftypes}, a
judgment of the form $\Gamma;\Phi \vdash \tau : \T$ or $\Gamma;\Phi
\vdash \tau : \F$ indicates that the type $\tau$ is intended to be
functorial in the variables in $\Phi$ but not necessarily in the
variables in $\Gamma$.

\begin{dfn}\label{def:wftypes}
The formation rules for the set $\T \subseteq \bigcup_{V \subseteq
  \tvars}\T(V)$ of\, {\em well-formed type constructor expressions}
are
\[\begin{array}{cc}
\AXC{$\phantom{\Gamma;\Phi}$}
\UIC{$\Gamma,\alpha^0;\emptyset \vdash \alpha^0 : \T$}
\DisplayProof
\\
\\
\AXC{$\Gamma;\ol{\alpha^0} \vdash \sigma :
  \mcF$}
\AXC{$\Gamma;\ol{\alpha^0}  \vdash \tau : \mcF$}
\BIC{$\Gamma;\emptyset \vdash \Nat^{\ol{\alpha^0}}\sigma \,\tau : \T$}
\DisplayProof
\end{array}\]

\vspace*{0.1in}

\noindent
The formation rules for the set $\F \subseteq \bigcup_{V \subseteq
  \tvars, \Pos \subseteq \fvars}\F^\Pos(V)$ of\, {\em well-formed
  functorial expressions} are
\[\begin{array}{cccc}
\AXC{$\Gamma;\emptyset \vdash \tau : \T$}
\UIC{$\Gamma; \emptyset \vdash \tau : \F$}
\DisplayProof
&
\AXC{$\phantom{\Gamma;\Phi}$}
\UIC{$\Gamma;\Phi, \alpha^0 \vdash \alpha^0 : \F$}
\DisplayProof
&
\AXC{\phantom{$\Gamma,\Phi$}}
\UIC{$\Gamma;\Phi \vdash \zerot : \mcF$}
\DisplayProof
&
\AXC{\phantom{$\Gamma,\Phi$}}
\UIC{$\Gamma;\Phi \vdash \onet : \mcF$}
\DisplayProof
\end{array}\]
\[\begin{array}{c}
\AXC{$\phi^k \in \Gamma \cup \Phi$}
\AXC{$\quad\quad\ol{\Gamma;\Phi \vdash \tau : \mcF}$}
\BIC{$\Gamma;\Phi \vdash \phi^k \ol{\tau} : \mcF $}
\DisplayProof
\\
\AXC{$\Gamma;\Phi,\ol{\alpha},\phi^k \vdash \tau : \mcF$}
\AXC{$\quad\quad\ol{\Gamma;\Phi \vdash \tau : \mcF}$}
\BIC{$\Gamma;\Phi \vdash (\mu \phi^k.\lambda \ol{\alpha}. \,\tau)\ol{\tau} : \mcF$}
\DisplayProof
\end{array}\]
\[\begin{array}{cc}
\AXC{$\Gamma;\Phi \vdash \sigma : \mcF$}
\AXC{$\Gamma;\Phi \vdash \tau : \mcF$}
\BIC{$\Gamma; \Phi \vdash \sigma + \tau : \F$}
\DisplayProof
&
\AXC{$\Gamma;\Phi \vdash \sigma : \mcF$}
\AXC{$\Gamma;\Phi \vdash \tau : \mcF$}
\BIC{$\Gamma; \Phi \vdash \sigma \times \tau : \F$}
\DisplayProof
\end{array}\]

\vspace*{0.1in}

\noindent
A type $\tau$ is {\em well-formed} if it is either a well-formed type
constructor expression or a well-formed functorial expression.
\end{dfn}

If $\tau$ is a closed type we may write $\vdash \tau$, rather than
$\emptyset;\emptyset \vdash \tau$, for the judgment that it is
well-formed.  Definition~\ref{def:wftypes} ensures that the expected
weakening rules for well-formed types hold --- although weakening does
not change the contexts in which $\Nat$-types can be formed. If
$\Gamma;\emptyset \vdash \sigma : \T$ and $\Gamma;\emptyset \vdash
\tau : \T$, then our rules allow formation of the type
$\Gamma;\emptyset \vdash \Nat^\emptyset\sigma \,\tau$. Since a type
$\Gamma; \emptyset \vdash \Nat^{\ol \alpha}\sigma\,\tau$ represents a
natural transformation in $\ol \alpha$ from $\sigma$ to $\tau$, the
type $\Gamma;\emptyset \vdash \Nat^\emptyset\sigma \,\tau$ represents
the standard arrow type $\Gamma \vdash \sigma \to \tau$ in our
calculus. We similarly represent a standard $\forall$-type $\Gamma;
\emptyset \vdash \forall \ol\alpha . \tau$ as $\Gamma; \emptyset
\vdash \Nat^{\ol\alpha} \,\onet \,\tau :\F$ in our calculus.  However,
if $\ol\alpha$ is non-empty then $\tau$ cannot be of the form
$\Nat^{\ol\beta} H \, K$ since $\Gamma; \ol\alpha \vdash
\Nat^{\ol\beta} H \, K$ is not a valid type judgment in our calculus
(except by weakening).

Definition~\ref{def:wftypes} allows the formation of all of the
(closed) nested types from the introduction:
\[\begin{array}{lll}
\mathit{List}\, \alpha & = & \mu \beta. \,\onet + \alpha \times
\beta\; = \; (\mu \phi. \lambda \beta.\,\onet + \beta \times \phi
\beta)\,\alpha\\ 
\mathit{PTree}\,\alpha & = & (\mu \phi. \lambda \beta.\,\beta +
\phi\,(\beta \times \beta))\,\alpha\\
\mathit{Forest}\,\alpha & = & (\mu \phi. \lambda \beta. \,\onet +
\beta \times PTree(\phi \, \beta))\,\alpha\\ 
\mathit{Bush}\,\alpha & = & (\mu \phi.\lambda \beta. \,\onet + \beta
\times \phi\,(\phi\,\beta))\,\alpha 
\end{array}\]
Each of these types can either be natural in $\alpha$ or not,
according to whether $\alpha \in \Gamma$ or $\alpha \in \Phi$. 
%it is well-formed in the context $\emptyset;
%\alpha$ or $\alpha;\emptyset$.
For example, if $\emptyset;\alpha \vdash \mathit{List} \,\alpha$, then
the type $\vdash \Nat^\alpha \onet\, (\mathit{List}\,\alpha) : \T$ is
well-formed; If $\alpha;\emptyset \vdash \List \,\alpha$, then it is
not. Definition~\ref{def:wftypes} also allows the derivation of, e.g.,
the type $\alpha; \emptyset \vdash \Nat^\alpha (\mathit{List}\,
\alpha) \, ({\color{blue} \mathit{Tree}}\, \alpha\, \gamma)$
representing a natural transformation from lists to trees that is
natural in $\alpha$ but not necessarily in $\gamma$. We emphasize that
types can be functorial in variables of arity greater than $0$. For
example, the type $\mathit{GRose}\, \phi \, \alpha = \mu\beta . \alpha
\times \phi \beta$ can be functorial in $\phi$ if $\phi \in \Phi$. As
usual, whether $\phi \in \Gamma$ or $\phi \in \Phi$ determines whether
types such as $\Nat^\alpha ({\color{blue} \mathit{GRose}} \, \phi \,
\alpha)\, (List \alpha)$ are well-formed. But even if $\mathit{GRose}$
is functorial in $\phi$, it still cannot be the (co)domain of a $\Nat$
type representing a natural transformation in $\phi$. This is because
our calculus does not allow naturality in variables of arity greater
than $0$.

Definition~\ref{def:wftypes} explicitly considers types in $\T$ to be
types in $\F$ that are functorial in no variables. It is not hard to
see that this definition also supports the demotion of functorial
variables in a well-formed type $\tau$ to non-functorial status. The
proof is by induction on the structure of $\tau$.

\begin{lemma}
If\, $\Gamma; \Phi, \phi^k \vdash \tau : \F$, then $\Gamma, \psi^k;
\Phi \vdash \tau[\phi^k :== \psi^k]$ is also derivable. Here,
$\tau[\phi :== \psi]$ is the textual replacement of $\phi$ in $\tau$,
meaning that all occurences of $\phi\ol\sigma$ in $\tau$ become
$\psi\ol\sigma$.
\end{lemma}

In addition to textual replacement, we also have a proper substitution
operation on types. If $\tau$ is a type over $P$ and $V$, if $\Pos$
and $V$ contain only type variables of arity $0$, and if $k=0$ for
every occurrence of $\phi^k$ bound by $\mu$ in $\tau$, then we say
that $\tau$ is {\em first-order}; otherwise we say that $\tau$ is {\em
  second-order}.  Substitution for first-order types is the usual
capture-avoiding textual substitution. We write $\tau[\alpha :=
  \sigma]$ for the result of substituting $\sigma$ for $\alpha$ in
$\tau$, and $\tau[\alpha_1 := \tau_1,...,\alpha_k := \tau_k]$, or
$\tau[\ol{\alpha := \tau}]$ when convenient, for $\tau[\alpha_1 :=
  \tau_1][\alpha_2 := \tau_2,...,\alpha_k := \tau_k]$. Substitution
for second-order types is defined below, where we adopt a similar
notational convention for vectors of types.

\begin{dfn}\label{def:second-order-subst}
  If $\phi^k \in \Gamma \cup \Phi$ with $k \geq 1$, if \,$\Gamma; \Phi
  \vdash F : \F$, and if\, $\Gamma, \ol\beta;\Phi, \ol{\alpha} \vdash
  H : \F$ with $|\ol\alpha| + |\ol\beta| = k$, then $\Gamma \setminus
  \phi^k;\Phi \setminus \phi^k\vdash F[\phi :=_{\ol\beta,\ol{\alpha}}
    H] : \F$, where the operation $(\cdot)[\phi := H]$ of {\em
    second-order type substitution} is defined by:
\[\begin{array}{lll}
(\Nat^{\ol\gamma} G \,K)[\phi :=_{\ol\beta,\ol{\alpha}} H]
& = & \Nat^{\ol\gamma}\, (G[\phi :=_{\ol\beta,\ol{\alpha}} H]) \,(K[\phi
  :=_{\ol\beta,\ol{\alpha}} H]) \\
\onet[\phi :=_{\ol\beta,\ol{\alpha}} H] & = & \onet\\[0.5ex]
\zerot[\phi :=_{\ol\beta,\ol{\alpha}} H] & = & \zerot\\[0.25ex]
(\psi\ol{\sigma}\ol{\tau})[\phi :=_{\ol\beta,\ol{\alpha}} H] & = &
\left\{\begin{array}{ll}
\psi \,\ol{\tau[\phi :=_{\ol\beta,\ol{\alpha}} H]} & \mbox{if } \psi \not = \phi\\
  H[\ol{\alpha  := \tau[\phi :=_{\ol\beta,\ol{\alpha}} H]}] 
  [\ol{\beta := \sigma[\phi :=_{\ol\beta,\ol{\alpha}} H]}]
  & \mbox{if } \psi = \phi
\end{array}\right.\\
\\[-0.25ex]
(\sigma + \tau)[\phi :=_{\ol\beta,\ol{\alpha}} H] & = & \sigma[\phi
  :=_{\ol\beta,\ol{\alpha}} H] + \tau[\phi :=_{\ol\beta,\ol{\alpha}} H]\\[0.5ex] 
(\sigma \times \tau)[\phi :=_{\ol\beta,\ol{\alpha}} H] & = &
\sigma[\phi :=_{\ol\beta,\ol{\alpha}} H] \times \tau[\phi
  :=_{\ol\beta,\ol{\alpha}} H]\\[0.5ex]   
((\mu \psi. \lambda \ol{\gamma}.\, G)\ol{\tau})[\phi :=_{\ol\beta,\ol{\alpha}}
  H] & = & (\mu \psi. \lambda \ol{\gamma}. \,G[\phi :=_{\ol\beta,\ol{\alpha}}
  H])\, \ol{\tau[\phi :=_{\ol\beta,\ol{\alpha}} H]}
\end{array}\]
\end{dfn}
\noindent
We omit the variable subscripts in second-order type constructor
substitution when convenient.

\subsection{Terms}
We assume an infinite set $\cal V$ of term variables disjoint from
$\tvars$ and $\fvars$. If $\Gamma$ be a type constructor context and
$\Phi$ is a functorial context, then a {\em term context for $\Gamma$
  and $\Phi$} is a finite set of bindings of the form $x : \tau$,
where $x \in {\cal V}$ and $\Gamma; \Phi \vdash \tau : \F$. We adopt
the same conventions for denoting disjoint unions and for vectors in
term contexts as for type constructor contexts and functorial
contexts.

\begin{dfn}\label{def:well-formed-terms}
Let $\Delta$ be a term context for $\Gamma$ and $\Phi$.  The formation
rules for the set of\, {\em well-formed terms over $\Delta$} are
\[\begin{array}{cc}
\AXC{$\Gamma;\emptyset \vdash \tau : \T$}
\UIC{$\Gamma;\emptyset\,|\, \Delta,x :\tau \vdash x : \tau$}
\DisplayProof
&
\AXC{$\Gamma;\Phi \vdash \tau : \mcF$}
\UIC{$\Gamma;\Phi \,|\, \Delta,x :\tau \vdash x : \tau$}
\DisplayProof\\\\
\AXC{$\phantom{\Gamma;\Phi}$}
\UIC{$\Gamma;\Phi \,|\, \Delta \vdash \top : \onet$}
\DisplayProof
&
\AXC{$\Gamma;\Phi \,|\, \Delta \vdash t : \zerot$}
\AXC{$\Gamma;\Phi \vdash \tau: \F$}
\BIC{$\Gamma;\Phi \,|\, \Delta \vdash \bot_\tau t  : \tau$}
\DisplayProof
\\\\
\AXC{$\Gamma;\Phi \,|\, \Delta \vdash s: \sigma$}
\UIC{$\Gamma;\Phi \,|\, \Delta \vdash \inl \,s: \sigma + \tau$}
\DisplayProof
&
\AXC{$\Gamma;\Phi \,|\, \Delta \vdash t : \tau$}
\UIC{$\Gamma;\Phi \,|\, \Delta \vdash \inr \,t: \sigma + \tau$}
\DisplayProof\\\\
\end{array}\]
\[\begin{array}{c}
\AXC{$\Gamma; \Phi \vdash \tau,\sigma : \F$}
\AXC{$\Gamma;\Phi \,|\, \Delta \vdash t : \sigma + \tau$}
\AXC{$\Gamma;\Phi \,|\, \Delta, x : \sigma \vdash l : \gamma \hspace{0.3in} \Gamma;\Phi \,|\, \Delta, y : \tau \vdash r : \gamma$}
\TIC{$\Gamma;\Phi~|~\Delta \vdash \case{t}{{\color{red} \mathsf{inl
        \,}} x \mapsto l}{{\color{red} \mathsf{inr \,}} y \mapsto r} : \gamma$}
\DisplayProof
\end{array}\]

\vspace*{0.05in}

\[\begin{array}{lll}
\AXC{$\Gamma;\Phi \,|\, \Delta \vdash s: \sigma$}
\AXC{$\Gamma;\Phi \,|\, \Delta \vdash t : \tau$}
\BIC{$\Gamma;\Phi \,|\, \Delta \vdash (s,t) : \sigma \times \tau$}
\DisplayProof
&
\AXC{$\Gamma;\Phi \,|\, \Delta \vdash t : \sigma \times \tau$}
\UIC{$\Gamma;\Phi \,|\, \Delta \vdash \pi_1 t : \sigma$}
\DisplayProof
&
\AXC{$\Gamma;\Phi \,|\, \Delta \vdash t : \sigma \times \tau$}
\UIC{$\Gamma;\Phi \,|\, \Delta \vdash \pi_2 t : \tau$}
\DisplayProof
\end{array}\]

\vspace*{0.05in}

\[\begin{array}{c}
%\AXC{$\Gamma; \emptyset \vdash \Nat^{\ol{\alpha}} \,F\,G : \T$}
\AXC{$\Gamma; \ol{\alpha} \vdash F : \F$}
\AXC{$\Gamma; \ol{\alpha} \vdash G : \F$}
\AXC{$\Gamma; \ol{\alpha} \,|\, \Delta, x : F
  \vdash t: G  $} 
\TIC{$\Gamma;\emptyset \,|\, \Delta \vdash L_{\ol{\alpha}} x.t : \Nat^{\ol{\alpha}} \,F \,G$}
\DisplayProof
\\\\
\AXC{$\Gamma;\emptyset \,|\, \Delta \vdash t : \Nat^{\ol{\alpha}} \,F \,G$}
\AXC{$\ol{\Gamma;\Phi \vdash \tau : \F}$}
\AXC{$\Gamma;\Phi \,|\, \Delta \vdash s: F[\overline{\alpha := \tau}]$}
\TIC{$\Gamma;\Phi \,|\, \Delta \vdash t_{\ol{\tau}} s:
  G[\overline{\alpha := \tau}]$}
\DisplayProof
\\\\
\AXC{$\Gamma; \ol{\phi},\ol{\gamma} \vdash H : \F$}
\AXC{$\ol{\Gamma; \ol{\beta},\ol{\gamma} \vdash F : \F}$}
\AXC{$\ol{\Gamma; \ol{\beta},\ol{\gamma} \vdash G : \F}$}
\TIC{$\Gamma;\emptyset~|~\emptyset \vdash \map^{\ol{F},\ol{G}}_H :
  \Nat^\emptyset\;(\ol{\Nat^{\ol{\beta},\ol{\gamma}}\,F\,G})\;
  (\Nat^{\ol{\gamma}}\,H[\ol{\phi :=_{\ol{\beta}} F}]\,H[\ol{\phi
      :=_{\ol{\beta}} G}])$} 
\DisplayProof
\\\\
%\AXC{$\Gamma;\Phi \,|\, \Delta \vdash t : H[\phi := (\mu \phi.\lambda 
%  \ol{\alpha}.H)\ol{\beta}][\ol{\alpha := \tau}]$}
%\AXC{$\ol{\Gamma;\Phi \vdash \tau : \F}$}
\AXC{$\Gamma; \phi, \ol{\alpha},\ol{\gamma} \vdash H : \F$}
\UIC{$\Gamma;\emptyset \,|\, \emptyset \vdash \tin_H :
  \Nat^{\ol{\beta},\ol{\gamma}} H[\phi :=_{\ol{\beta}} (\mu
    \phi.\lambda \ol{\alpha}.H)\ol{\beta}][\ol{\alpha := \beta}]\,(\mu
  \phi.\lambda \ol{\alpha}.H)\ol{\beta}$}
\DisplayProof
\\\\
\AXC{$\Gamma; \phi,\ol{\alpha}, \ol{\gamma} \vdash H : \mcF$}
\AXC{$\Gamma; \ol{\beta}, \ol{\gamma} \vdash F : \F$}
%\AXC{$\Gamma;\emptyset \,|\, \Delta \vdash t :
%  \Nat^{\ol{\beta}, \ol{\gamma}}\,H[\phi :=_{\ol{\beta}} F][\ol{\alpha
%      := \beta}]\,F$} 
%\TIC{$\Gamma;\emptyset \,|\, \Delta \vdash
%  \fold_{H,F}\, t : \Nat^{\ol{\beta}, \ol{\gamma}}\,((\mu
%  \phi.\lambda \ol{\alpha}.H)\ol{\beta})\,F$} 
\BIC{$\Gamma;\emptyset \,|\, \emptyset \vdash \fold^F_H :
  \Nat^\emptyset\; (\Nat^{\ol{\beta}, \ol{\gamma}}\,H[\phi
    :=_{\ol{\beta}} F][\ol{\alpha := \beta}]\,F)\; (\Nat^{\ol{\beta},
    \ol{\gamma}}\,(\mu \phi.\lambda \ol{\alpha}.H)\ol{\beta})\,F)$}
\DisplayProof
\end{array}\]
\end{dfn}

In the rule for $L_{\ol{\alpha}}x.t$, the $L$ operator binds all
occurrences of the type variables in $\ol{\alpha}$ in the type of the
term variable $x$ and in the body $t$, as well as all occurrences of
$x$ in $t$. In the rule for $t_{\ol\tau} s$ there is one functorial
expression $\tau$ for every functorial variable $\alpha$. In the rule
for $\map^{\ol{F},\ol{G}}_H$ there is one functorial expression $F$
and one functorial expression $G$ for each functorial variable in
$\ol\phi$. Moreover, for each $\phi^k \in \ol\phi$ the number of
functorial variables $\beta$ in the judgments for its corresponding
functorial expresssions $F$ and $G$ is $k$. In the rules for $\tin_H$
and $\fold^F_H$, the functorial variables in $\ol{\beta}$ are fresh
with respect to $H$, and there is one $\beta$ for every
$\alpha$. (Recall from above that, in order for the types of $\tin_H$
and $\fold^F_H$ to be well-formed, the length of $\alpha$ must equal
the arity of $\phi$.) Substitution for terms is the obvious extension
of the usual capture-avoiding textual substitution, and
Definition~\ref{def:well-formed-terms} ensures that the expected
weakening rules for well-formed terms hold.

Using Definition~\ref{def:well-formed-terms} we can represent the
$\mathtt{reversePTree}$ function from Figure~\ref{fig:funs} in our
calculus as
\[\vdash \fold_{\beta + \phi(\beta \times \beta)}^{\mathit{PTree}\,
  \alpha} \, (\tin_{\beta + \phi(\beta \times \beta)} \circ s) :
\Nat^{\alpha} (\mathit{PTree}\,\alpha)\,(\mathit{PTree}\,\alpha)\] 
where
\[\begin{array}{lll}
\vdash \fold_{\beta + \phi(\beta \times \beta)}^{\mathit{PTree}\,\alpha} &
: & \Nat^{\emptyset} (\Nat^{\alpha} (\alpha + \mathit{PTree}\,(\alpha \times
\alpha)) \; (\mathit{PTree}\,\alpha))\; (\Nat^{\alpha}
(\mathit{PTree}\,\alpha) \; (\mathit{PTree}\,\alpha))\\
\vdash \tin_{\beta + \phi(\beta \times \beta)} & : & \Nat^{\alpha}
(\alpha + \mathit{PTree}\,(\alpha \times \alpha)) \; (\mathit{PTree}\,
\alpha)\\ 
\vdash \map_{\mathit{PTree}\,\alpha}^{\alpha \times \alpha, \alpha \times
  \alpha} & : & \Nat^{\emptyset} (\Nat^{\alpha} (\alpha \times \alpha)\,
(\alpha \times \alpha))\; (\Nat^{\alpha} (\mathit{PTree}\,(\alpha \times
\alpha))\, (\mathit{PTree}\,(\alpha \times \alpha)))
\end{array}\]
and 
$\mathit{swap}$ and $s$ are the terms
\[ \vdash L_{\alpha} p.\, (\pi_2 p, \pi_1 p) :
\Nat^{\alpha} (\alpha \times \alpha)\, (\alpha \times \alpha)\]
and
\[\begin{array}{l}
\vdash L_{\alpha} t. \,\case{t}{b \mapsto \inl\, b}{t' \mapsto \inr\,(
  \map_{\mathit{PTree}\,\alpha}^{\alpha \times \alpha, \alpha \times \alpha}\,
  \mathit{swap}\, t' )} : \Nat^{\alpha} (\alpha + \mathit{PTree}\,(\alpha \times
\alpha))\; (\alpha + \mathit{PTree}\,(\alpha \times \alpha))
\end{array}\]
respectively. We can similarly represent the $\mathtt{reverseBush}$
function from Figure~\ref{fig:funs2} as
\[\vdash \fold_{\onet + \beta \times
\phi (\phi\beta)}^{\mathit{Bush}\,\alpha}\, ( \tin_{\onet + \beta \times
  \phi (\phi\beta)} \circ (\onet + t \circ i \circ i') ) :
\Nat^{\alpha} (\mathit{Bush}\,\alpha)\,(\mathit{Bush}\,\alpha)\] 
where
\[\begin{array}{lll}
\vdash \fold_{\onet + \beta \times \phi
  (\phi\beta)}^{\mathit{Bush}\,\alpha} : \Nat^{\emptyset}\, (\Nat^{\alpha}\, 
(\onet + \alpha \times \mathit{Bush}\, (\mathit{Bush}\, \alpha))) \;
(\mathit{Bush}\,\alpha))\; (\Nat^{\alpha} \,(\mathit{Bush}\,\alpha) \;
(\mathit{Bush}\,\alpha))\\ 
\vdash \tin_{\onet + \beta \times \phi (\phi\beta)} : \Nat^{\alpha}\,
(\onet + \alpha \times \mathit{Bush}\, (\mathit{Bush} \, \alpha))
\;(\mathit{Bush}\, \alpha)\\
\end{array}\]
and $\mathit{bnil}$, $\mathit{bcons}$, $\tin^{-1}_{\onet + \beta
  \times \phi (\phi\beta)}$, $t$, $i$, and 
$i'$ are the terms
\[\begin{array}{l}
\vdash \tin_{\onet + \beta \times \phi (\phi\beta)} \circ (
L_{\alpha}\, x.\, \inl\, x) \; : \; \Nat^{\alpha}\, \onet\;
(\mathit{Bush}\,\alpha)\\ 
\vdash \tin_{\onet + \beta \times \phi (\phi\beta)} \circ (
L_{\alpha}\, x.\, \inr\, x ) \; : \; \Nat^{\alpha}\, (\alpha \times
\mathit{Bush}\,(\mathit{Bush}\,\alpha))\; (\mathit{Bush}\,\alpha)\\
\vdash \fold_{\onet + \beta \times \phi (\phi\beta)}^{(\onet + \beta
  \times \phi (\phi\beta))[\phi := \mathit{Bush}\,\alpha]}\,
(\map_{\onet + \beta \times \phi (\phi\beta)}^{(\onet + \beta \times
  \phi (\phi\beta))[\phi := \mathit{Bush}\,\alpha][\beta := \alpha],
  \mathit{Bush}\,\alpha}\, \tin_{\onet + \beta \times \phi
  (\phi\beta)})\\
\hspace*{0.2in} : \Nat^{\alpha}\, (\mathit{Bush}\,\alpha)\; (\onet +
\alpha \times \mathit{Bush}\,(\mathit{Bush}\,\alpha))\\ 
\vdash L_{\alpha}\, (b, s). \,\mathsf{case}\;s\, \{ \hspace*{0.21in}\ast \mapsto
\mathit{bcons}_{\alpha}\, b\, (\mathit{bnil}_{\alpha}\, \ast); \\
\hspace*{1in}(s', u) \mapsto \mathsf{case}\;s' \{\hspace*{0.35in}\ast
\mapsto \mathit{bcons}_{\alpha}\, b \,(\mathit{bcons}_{\mathit{Bush}\,
  \alpha}\, (\mathit{bnil}_{\alpha}\, \ast)\, u); \\
\hspace*{2in} (b', u') \mapsto \mathit{bcons}_{\alpha}\, b'
(\mathit{bcons}_{\mathit{Bush}\, \alpha}\, (\mathit{bcons}_{\alpha}\,
b\, u )\, u')\}\}\\
\hspace{0.2in} : \Nat^{\alpha} (\alpha \times (\onet + (\onet + \alpha
\times \mathit{Bush}\, (\mathit{Bush} \, \alpha))) \times
\mathit{Bush}\, (\mathit{Bush} \, (\mathit{Bush}\, \alpha))) \;
(\alpha \times \mathit{Bush}\, (\mathit{Bush} \, \alpha))\\
\vdash \alpha \times (\onet + \tin^{-1}_{\onet + \beta \times \phi
  (\phi\beta)} \times \mathit{Bush}\, (\mathit{Bush} \,
(\mathit{Bush}\, \alpha)))\\
\hspace{0.2in} : \Nat^{\alpha} (\alpha \times (\onet + \mathit{Bush}
\, \alpha \times \mathit{Bush}\,( \mathit{Bush} \, (\mathit{Bush}\,
\alpha))))\\
\hspace*{0.57in}(\alpha \times (\onet + (\onet + \alpha \times
\mathit{Bush}\, (\mathit{Bush} \, \alpha)) \times \mathit{Bush}\,
(\mathit{Bush} \, (\mathit{Bush}\, \alpha))))\\
\vdash \alpha \times (L_{\alpha}\,x.\, (\tin^{-1}_{\onet +
  \beta \times \phi (\phi\beta)})_{\mathit{Bush}\,\alpha}\, x)\\
\hspace{0.2in} : \Nat^{\alpha} (\alpha
\times \mathit{Bush} \, (\mathit{Bush}\, \alpha)) \; (\alpha \times (\onet
+ \mathit{Bush} \, (\alpha \times \mathit{Bush}\, (\mathit{Bush} \,
(\mathit{Bush}\, \alpha)))))
\end{array}\]
respectively. Here, $\Gamma; \emptyset \,|\, \Delta \vdash \sigma +
\eta : \Nat^{\ol{\alpha}} (\sigma + F)\; (\sigma + G)$ and $\Gamma;
\emptyset \,|\, \Delta \vdash \sigma \times \eta : \Nat^{\ol{\alpha}}
(\sigma \times F)\; (\sigma \times G)$ for $\sigma + \eta :=
L_{\ol{\alpha}}\, x.\, \case{x}{s \mapsto \inl\, s}{t \mapsto \inr\,
  (\eta_{\ol{\alpha}} t)}$ and $\sigma \times \eta := L_{\ol{\alpha}}\,
x.\, (\pi_1 x, \eta_{\ol{\alpha}} (\pi_2 x))$ for $\Gamma; \emptyset
\,|\, \Delta \vdash \eta : \Nat^{\ol{\alpha}} F\; G$ and $\Gamma;
\ol{\alpha} \vdash \sigma : \F$.

Unfortunately, we cannot write functions, such as $\mathit{concat} :
\mathit{PTree}\, \alpha \to \mathit{PTree} \,\alpha \to
\mathit{PTree}\, \alpha$, that take as input more than one
non-algebraic nested type. This is because $\Nat$-types must be formed
in empty functorial contexts, and this conflicts with the need to feed
$\fold$s algebras. (Can't fold over pairs to get the right functions;
can't get them by using continuation style because of the
aforementioned typing conflict.) {\color{blue} Add Daniel's commentary
  about ``real'' involution reverse for bushes, too. Massage
  paragraph. Not a good restriction. Also, generalized folds don't
  help much.}


The presence of the ``extra'' functorial variables in $\ol{\gamma}$ in
the rules for $\map^{\ol{F},\ol{G}}_H$, $\tin_H$, and $\fold^F_H$
merit special mention. They allows us to map or fold polymorphic
functions over nested types. Consider, for example, the function
$\mathit{flatten} : \Nat^\beta
(\mathit{PTree}\,\beta)\,(\mathit{List}\,\beta)$ that maps perfect
trees to lists. Even in the absence of extra variables the instance of
$\map$ required to map each non-functorial monomorphic instantiation
of $\mathit{flatten}$ over a list of perfect trees is well-typed:
\[\begin{array}{l}
\AXC{$\Gamma;\alpha \vdash \mathit{List} \, \alpha$}
\AXC{$\Gamma;\emptyset \vdash \sigma \hspace*{0.3in} \Gamma;\emptyset
  \vdash \tau$}
\AXC{$\Gamma;\emptyset \vdash \mathit{PTree}\,\sigma \hspace*{0.3in}
  \Gamma;\emptyset \vdash \mathit{List}\,\tau$}
\TIC{$\Gamma;\emptyset \,|\, \emptyset \vdash
  \map^{\mathit{PTree}\,\sigma, \,\mathit{List}\,\tau}_{\mathit{List}\,\alpha} :
  \Nat^\emptyset\;(\Nat^\emptyset\,(\mathit{PTree}\,
  \sigma)\,(\mathit{List}\, \tau))\; (\Nat^\emptyset\,(\mathit{List}\,
  (\mathit{PTree}\, \sigma))\,(\mathit{List}\, (\mathit{List}\,
  \tau)))$}
\DisplayProof
\end{array}\]
But in the absence of $\ol \gamma$, the instance
\[\Gamma;\emptyset \,|\, \emptyset \vdash
\map^{\mathit{PTree}\,\beta,\mathit{List}\,\beta}_{\mathit{List}\,\alpha}
: \Nat^\emptyset\;(\Nat^\beta(\mathit{PTree}\,
\beta)\,(\mathit{List}\, \beta))\; (\Nat^\beta\,(\mathit{List}\,
(\mathit{PTree}\, \beta))\,(\mathit{List}\, (\mathit{List}\,
\beta)))\] of $\mathtt{map}$ required to map the {\em polymorphic}
$\mathit{flatten}$ function over a list of perfect trees is not: in
that setting the functorial contexts for $F$ and $G$ in the rule for
$\map^{F,G}_H$ would have to be empty, but the fact that the
polymorphic $\mathit{flatten}$ function is functorial in some
variable, say $\delta$, means that it cannot possibly have a type of
the form $\Nat^\emptyset F\, G$ that would be required for it to be
the function input to $\map$. Since untypeability of this instance of
$\map$ is unsatisfactory in a polymorphic calculus, where we naturally
expect to be able to manipulate entire polymorphic functions rather
than just their monomorphic instances, we use the ``extra'' variables
in $\ol \gamma$ to remedy the situation. Specifrically, the rules from
Definition~\ref{def:well-formed-terms} ensure that the instance of
$\map$ needed to map the polymorphic $\mathit{flatten}$ function is
typeable as follows:
\[\begin{array}{l}
\AXC{$\Gamma;\alpha,\beta \vdash \mathit{List} \, \alpha$}
\AXC{$\Gamma;\beta \vdash \mathit{PTree}\,\beta \hspace*{0.3in}
  \Gamma;\beta \vdash \mathit{List}\,\beta$}
\BIC{$\Gamma;\emptyset \,|\, \emptyset \vdash
  \map^{F,G}_{\mathit{List}} :
  \Nat^\emptyset\;(\Nat^\beta(\mathit{PTree}\,
  \beta)\,(\mathit{List}\, \beta))\; (\Nat^\beta\,(\mathit{List}\,
  (\mathit{PTree}\, \beta))\,(\mathit{List}\, (\mathit{List}\,
  \beta)))$}
\DisplayProof
\end{array}\]
Similar remarks explain the appearance of $\ol \gamma$ in the typing
rules for $\tin$ and $\fold$.


\vspace*{0.05in}

\section{Interpreting Types}\label{sec:type-interp}

We denote the category of sets and functions by $\set$. The category
$\rel$ has as its objects triples $(A,B,R)$ where $R$ is a relation
between the objects $A$ and $B$ in $\set$, i.e., a subset of $A \times
B$, and has as its morphisms from $(A,B,R)$ to $(A',B',R')$ pairs $(f
: A \to A',g : B \to B')$ of morphisms in $\set$ such that $(f a,g\,b)
\in R'$ whenever $(a,b) \in R$. We write $R : \rel(A,B)$ in place of
$(A,B,R)$ when convenient.  If $R : \rel(A,B)$ we write $\pi_1 R$ and
$\pi_2 R$ for the {\em domain} $A$ of $R$ and the {\em codomain} $B$
of $R$, respectively.  If $A : \set$, then we write $\Eq_A =
(A,A,\{(x,x)~|~ x \in A\})$ for the {\em equality relation} on $A$.

The key idea underlying Reynolds' parametricity is to give each type
$\tau(\alpha)$ with one free variable $\alpha$ both an {\em object
  interpretation} $\tau_0$ taking sets to sets and a \emph{relational
  interpretation} $\tau_1$ taking relations $R : \rel(A,B)$ to
relations $\tau_1 (R) : \rel(\tau_0 (A), \tau_0 (B))$, and to
interpret each term $t(\alpha,x) : \tau(\alpha)$ with one free term
variable $x : \sigma(\alpha)$ as a map $t_0$ associating to each set
$A$ a function $t_0(A) : \sigma_0(A) \to \tau_0(A)$. These
interpretations are to be given inductively on the structures of
$\tau$ and $t$ in such a way that they imply two fundamental
theorems. The first is an \emph{Identity Extension Lemma}, which
states that $\tau_1(\Eq_A) = \Eq_{\tau_0(A)}$, {\color{blue} and is
  the essential property that makes a model relationally parametric
  rather than just induced by a logical relation.} The second is an
\emph{Abstraction Theorem}, which states that, for any $R :\rel(A,
B)$, $(t_0(A),t_0(B))$ is a morphism in $\rel$ from
$(\sigma_0(A),\sigma_0(B),\sigma_1(R))$ to
$(\tau_0(A),\tau_0(B),\tau_1(R))$. The Identity Extension Lemma is
similar to the Abstraction Theorem except that it holds for {\em all}
elements of a type's interpretation, not just those that are
interpretations of terms.
%i.e., $t_0(A)$ and $t_0(B)$ map related arguments to
%related results.
Similar results are expected to hold for types and terms with any
number of free variables.

The key to proving the Identity Extension Lemma
(Theorem~\ref{thm:iel}) in our setting is a familiar ``cutting down''
of the interpretations of universally quantified types, such as our
$\Nat$-types, to include only the ``parametric'' elements.  (See,
e.g.,~\cite{agj14,atk??,bfss90,rey83}). This requires that set
interpretations of types are defined simultaneously with their
relational interpretations. We give set interpretations for our types
in Section~\ref{sec:set-interp} and give their relational
interpretations in Section~\ref{sec:rel-interp}.  While the set
interpretations are relatively straightforward, their relation
interpretations are less so, mainly because of the cocontinuity
conditions we must impose to ensure that they are well-defined. We
take some effort to develop conditions in
Section~\ref{sec:rel-interp}, which separates
Definitions~\ref{def:set-sem} and~\ref{def:rel-sem} in space, but
otherwise has no impact on the fact that they are given by mutual
induction.

\subsection{Interpreting Types as Sets}\label{sec:set-interp}

%A poset $\mcD = (D,\leq)$ is \emph{$\omega$-directed} if every finite
%subset of $D$ has an upper bound. When $\mcD$ is considered as a
%category, we write $d \in \mcD$ to indicate that $d$ is an object of
%$\mcD$ (i.e., $d \in D$). An {\em $\omega$-directed colimit} in a
%category $\mcC$ is a colimit of a diagram $F : {\mathcal D} \to \mcC$,
%where $\mathcal D$ is an $\omega$-directed poset. A category $\mcC$ is
%{\em $\omega$-cocomplete} if it has all $\omega$-directed colimits. A
%cocomplete category is one that has all colimits.
%
%If $\mcA$ and $\mcC$ are $\omega$-cocomplete, then the functor $F :
%\mcA \to \mcC$ is {\em $\omega$-cocontinuous} if it preserves
%$\omega$-directed colimits.  If $\mcA$ is locally small, then an
%object $A$ of $\mcA$ is {\em finitely presentable} if the functor
%$\mathsf{Hom}_\mcA(A, -) : \mcA \to \Set$ preserves $\omega$-directed
%colimits, i.e., if for every $\omega$-directed poset $\mathcal D$ and
%every functor $F : {\mathcal D} \to \mcC$, there is a canonical
%isomorphism $\colim{d \in \mcD}{\mathsf{Hom}_\mcA(A,Fd)} \simeq
%\mathsf{Hom}_{\mcA}(A, \colim{d \in \mcD}{Fd})$. A category $\mcA$ is
%       {\em finitely accessible} if it is $\omega$-cocomplete and has
%       a set $\mcA_0$ of finitely presentable objects such that every
%%       object is an $omega$-directed colimit of objects in $\mcA_0$;
%       it is {\em locally finitely presentable} if it is finitely
%       accessible and cocomplete.
We will interpret the types in our calculus as $\omega$-cocontinuous
functors on locally finitely presentable categories~\cite{ar94}. Since
functor categories of locally finitely presentable categories are
again locally finitely presentable, this will ensure, in particular,
that the fixed points interpreting $\mu$-types in $\set$ and $\rel$
exist, and thus that both the set and relational interpretations of
all of the types in Definition~\ref{def:wftypes} are
well-defined~\cite{jp19}. To bootstrap this process, we interpret type
variables themselves as $\omega$-cocontinuous functors in
Definitions~\ref{def:set-env} and~\ref{def:reln-env}. If $\C$ and $\D$
are locally finitely presentable categories, we write $[\C,\D]$ for
the set of $\omega$-cocontinuous functors from $\C$ to $\D$.
%Note that the categories $\set$ and $\rel$ are both locally finitely
%presentable.

\begin{dfn}\label{def:set-env}
A {\em set environment} maps each type variable in $\tvars^k \cup
\fvars^k$ to an element of $[\set^k,\set]$.  A morphism $f : \rho \to
\rho'$ for set environments $\rho$ and $\rho'$ with $\rho|_\tvars =
\rho'|_\tvars$ maps each type constructor variable $\psi^k \in \tvars$
to the identity natural transformation on $\rho \psi^k = \rho'\psi^k$
and each functorial variable $\phi^k \in \fvars$ to a natural
transformation from the $k$-ary functor $\rho \phi^k$ on $\set$ to the
$k$-ary functor $\rho' \phi^k$ on $\set$.  Composition of morphisms on
set environments is given componentwise, with the identity morphism
mapping each set environment to itself. This gives a category of set
environments and morphisms between them, which we denote $\setenv$.
\end{dfn}
When convenient we identify a functor $F : [\set^0, \set]$ with the
set that is its codomain and consider a set environment to map
a type variable of arity $0$ to
%an $\omega$-cocontinuous functor from $\set^0$ to $\set$, i.e., to
a set.  If $\ol{\alpha} = \{\alpha_1,...,\alpha_k\}$ and $\ol{A} =
\{A_1,...,A_k\}$, then we write $\rho[\ol{\alpha := A}]$ for the set
environment $\rho'$ such that $\rho' \alpha_i = A_i$ for $i = 1,...,k$
and $\rho' \alpha = \rho \alpha$ if $\alpha \not \in
\{\alpha_1,...,\alpha_k\}$.  If $\rho$ is a set environment we write
$\Eq_\rho$ for the relation environment (see
Definition~\ref{def:reln-env}) such that $\Eq_\rho v = \Eq_{\rho v}$
for every type variable $v$.
%; see Definition~\ref{def:reln-env} for the complete definition of a
%relation environment.
The relational interpretations appearing in the second clause of
Definition~\ref{def:set-sem} are given in full in
Definition~\ref{def:rel-sem}.

\begin{dfn}\label{def:set-sem}
%Let $\rho$ be a set environment.
The {\em set interpretation} $\setsem{\cdot} : \F \to [\setenv, \set]$
is defined by
\begin{align*}
  \setsem{\Gamma;\emptyset \vdash v} \rho &= \rho v \mbox{
    if } v \in \tvars^0 \\ 
  \setsem{\Gamma;\emptyset \vdash \Nat^{\ol{\alpha}}
    \,F\,G}\rho &= \{\eta : \lambda \ol{A}. \,\setsem{\Gamma; \ol{\alpha} \vdash
    F}\rho[\ol{\alpha := A}] 
      \Rightarrow \lambda \ol{A}.\,\setsem{\Gamma;\ol{\alpha} \vdash
        G}\rho[\ol{\alpha := A}] \\ 
      &\hspace{0.3in}|~\forall \overline{A}, \overline{B} :
      \set. \forall \overline{R : \rel(A, B)}.\\ 
      &\hspace{0.4in}(\eta_{\overline{A}}, \eta_{\overline{B}})
      : \relsem{\Gamma; \ol{\alpha} \vdash F}\Eq_{\rho}[\ol{\alpha := R}]
      \rightarrow \relsem{\Gamma;\ol{\alpha} \vdash
        G}\Eq_{\rho}[\ol{\alpha := R}] \} \\
  \setsem{\Gamma;\Phi \vdash \zerot}\rho &= 0\\
  \setsem{\Gamma;\Phi \vdash \onet}\rho &= 1\\
  \setsem{\Gamma;\Phi \vdash \phi\ol{\tau}}\rho &=
  (\rho\phi)\,\ol{\setsem{\Gamma;\Phi \vdash
    \tau}\rho}\\
  \setsem{\Gamma;\Phi \vdash \sigma+\tau}\rho &=
  \setsem{\Gamma;\Phi \vdash \sigma}\rho +
  \setsem{\Gamma;\Phi \vdash \tau}\rho\\
  \setsem{\Gamma;\Phi \vdash \sigma\times \tau}\rho &=
  \setsem{\Gamma;\Phi \vdash \sigma}\rho \times
  \setsem{\Gamma;\Phi \vdash \tau}\rho\\ 
  \setsem{\Gamma;\Phi \vdash (\mu \phi.\lambda
    \ol{\alpha}. H)\ol{\tau}}\rho &= (\mu
    T^\set_{H,\rho})\ol{\setsem{\Gamma;\Phi \vdash \tau}\rho}\\
    \text{where } T^\set_{H,\rho}\,F & = \lambda
  \ol{A}. \setsem{\Gamma;\Phi,\phi, \ol{\alpha} \vdash
    H}\rho[\phi :=  F][\ol{\alpha := A}]\\
  \text{and } T^\set_{H,\rho}\,\eta &= \lambda
  \ol{A}. \setsem{\Gamma;\Phi,\phi, \ol{\alpha} \vdash
    H}\id_\rho[\phi := \eta][\ol{\alpha := \id_{A}}]
\end{align*}
\end{dfn}
The interpretations in Definition~\ref{def:set-sem} respect weakening,
i.e., a type and its weakenings all have the same set interpretations.
The same holds for the actions of these interpretations on morphisms
in Definition~\ref{def:set-sem-funcs} below.  Moreover, the
interpretation of $\Nat$ types ensures that $\setsem{\Gamma \vdash
  \sigma \to \tau}\rho = \setsem{\Gamma \vdash \sigma} \rho \to
\setsem{\Gamma \vdash \tau} \rho$, as expected.  If $\rho$ is a set
environment and $\vdash \tau : \F$ then we may write $\setsem{\vdash
  \tau}$ instead of $\setsem{\vdash \tau}\rho$ since the environment
is immaterial. We note that the second clause of
Definition~\ref{def:set-sem} does indeed define a set: local finite
presentability of $\set$ and $\omega$-cocontinuity of
$\setsem{\Gamma;\ol{\alpha} \vdash F}\rho$ ensure that $\{\eta :
\setsem{\Gamma;\ol{\alpha} \vdash F}\rho \Rightarrow
\setsem{\Gamma;\ol{\alpha} \vdash G}\rho\}$ (which contains
$\setsem{\Gamma;\emptyset \vdash \Nat^{\ol{\alpha}}\,F\,G}\rho$) is a
subset of
$\big\{({\setsem{\Gamma;\ol{\alpha} \vdash G}\rho[\ol{\alpha :=
    S}]})^{(\setsem{\Gamma;\ol{\alpha} \vdash F}\rho[\ol{\alpha :=
      S}])}~\big|~ \ol{S} = (S_1,...,S_{|\ol{\alpha}|}), \mbox{ and }$
$S_i \mbox{ is a finite set for } i = 1,...,|\ol{\alpha}|\big\}$. There
are countably many choices for tuples $\ol{S}$, and each of these
gives rise to a morphism from ${\setsem{\Gamma;\ol{\alpha} \vdash
    F}\rho[ \ol{\alpha := S}]}$ to ${\setsem{\Gamma;\ol{\alpha} \vdash
    G}\rho[\ol{\alpha := S}]}$. But there are only $\set$-many choices
of morphisms between these (or any) two objects because $\set$ is
locally small.

In order to make sense of the last clause in
Definition~\ref{def:set-sem}, we need to know that, for each $\rho \in
\setenv$, $T^\set_{H,\rho}$ is an $\omega$-cocontinuous endofunctor on
$[\set^k, \set]$, and thus admits a fixed point.  Since
$T_{H,\rho}^\set$ is defined in terms of $\setsem{\Gamma;\Phi,\phi,
  \ol{\alpha} \vdash H}$, this means that interpretations of types
must be such functors, which in turn means that the actions of set
interpretations of types on objects and on morphisms in $\setenv$ are
intertwined. Fortunately, we know from~\cite{jp19} that, for every
$\Gamma; \ol{\alpha} \vdash \tau : \F$, $\setsem{\Gamma; \ol{\alpha}
  \vdash \tau}$ is actually in $[\set^k,\set]$ where $k = |\ol
\alpha|$. This means that for each $\setsem{\Gamma; \Phi, \phi^k,
  \ol{\alpha} \vdash H}$, the corresponding operator $T^\set_{H}$ can
be extended to a {\em functor} from $\setenv$ to
$[[\set^k,\set],[\set^k,\set]]$. The action of $T^\set_H$ on an object
$\rho \in \setenv$ is given by the higher-order functor
$T_{H,\rho}^\set$, whose actions on objects (functors in $[\set^k,
  \set]$) and morphisms (natural transformations between such
functors) are given in Definition~\ref{def:set-sem}. Its action on a
morphism $f : \rho \to \rho'$ is the higher-order natural
transformation $T^\set_{H,f} : T^\set_{H,\rho} \to T^\set_{H,\rho'}$
whose action on $F : [\set^k,\set]$ is the natural transformation
$T^\set_{H,f}\, F : T^\set_{H,\rho}\,F \to T^\set_{H,\rho'}\,F$ whose
component at $\ol{A}$ is $(T^\set_{H,f}\, F)_{\ol{A}} =
\setsem{\Gamma; \Phi,\phi,\ol{\alpha} \vdash H}f[\phi :=
  \id_F][\ol{\alpha := \id_A}]$. The next definition uses the functor
$T^\set_H$ to define the actions of functors interpreting types on
morphisms between set environments.

\begin{dfn}\label{def:set-sem-funcs}
Let $f: \rho \to \rho'$ for set environments $\rho$ and $\rho'$ (so
that $\rho|_\tvars = \rho'|_\tvars$). The action $\setsem{\Gamma;\Phi
  \vdash \tau}f$ of\, $\setsem{\Gamma;\Phi \vdash \tau}$ on the
morphism $f$ is given as follows:
\begin{itemize}
\item If \,$\Gamma,v;\emptyset \vdash v$ then
  $\setsem{\Gamma,v;\emptyset \vdash v}f = \id_{\rho v}$
\item If \,$\Gamma; \emptyset \vdash \Nat^{\ol{\alpha}}\,F\,G$ then
  $\setsem{\Gamma;\emptyset \vdash \Nat^{\ol{\alpha}}\,F\,G} f =
  \id_{\setsem{\Gamma;\emptyset \vdash \Nat^{\ol{\alpha}}\,F\,G}\rho}$
\item If \,$\Gamma;\Phi \vdash \zerot$ then $\setsem{\Gamma;\Phi \vdash
  \zerot}f = \id_0$
\item If \,$\Gamma;\Phi \vdash \onet$ then $\setsem{\Gamma;\Phi \vdash
  \onet}f = \id_1$
\item If \,$\Gamma;\Phi \vdash \phi \ol{\tau}$ then
  $\setsem{\Gamma;\Phi \vdash \phi \ol{\tau}} f : \setsem{\Gamma;\Phi
  \vdash \phi \ol{\tau}}\rho \to \setsem{\Gamma;\Phi \vdash
  \phi\ol{\tau}}\rho' = (\rho\phi) \ol{\setsem{\Gamma;\Phi \vdash
    \tau}\rho} \to (\rho'\phi) \ol{\setsem{\Gamma;\Phi \vdash
    \tau}\rho'}$ is defined by $\setsem{\Gamma;\Phi \vdash \phi
  \ol{\tau}} f = (f\phi)_{\ol{\setsem{\Gamma;\Phi \vdash
      \tau}\rho'}}\, \circ\, (\rho\phi) {\ol{\setsem{\Gamma;\Phi
      \vdash \tau}f}} = (\rho'\phi) {\ol{\setsem{\Gamma;\Phi \vdash
      \tau}f}}\, \circ\, (f \phi)_{\ol{\setsem{\Gamma;\Phi \vdash
      \tau}\rho}}$.  The latter equality holds because $\rho\phi$ and
  $\rho'\phi$ are functors and $f\phi : \rho\phi \to \rho'\phi$ is a
  natural transformation, so the following naturality square commutes:
\begin{equation}\label{eq:cd2}
\begin{CD}
  (\rho\phi) \ol{\setsem{\Gamma;\Phi \vdash \tau}\rho} @> (f\phi)_{
    \ol{\setsem{\Gamma;\Phi \vdash \tau}\rho}} >> (\rho'\phi)
  \ol{\setsem{\Gamma;\Phi \vdash \tau}\rho} \\ @V(\rho\phi)
  \ol{\setsem{\Gamma;\Phi \vdash \tau}f}VV @V (\rho'\phi)
  \ol{\setsem{\Gamma;\Phi \vdash \tau}f} VV \\ (\rho\phi)
  \ol{\setsem{\Gamma;\Phi \vdash \tau}\rho'} @>(f\phi)_{
    \ol{\setsem{\Gamma;\Phi \vdash \tau}\rho'}}>> (\rho'\phi)
  \ol{\setsem{\Gamma;\Phi \vdash \tau}\rho'}
\end{CD}
\end{equation}
\item If\, $\Gamma;\Phi \vdash \sigma + \tau$ then $\setsem{\Gamma;\Phi
  \vdash \sigma + \tau}f$ is defined by $\setsem{\Gamma;\Phi \vdash
  \sigma + \tau}f(\inl\,x) = \inl\,(\setsem{\Gamma;\Phi \vdash
  \sigma}f x)$ and $\setsem{\Gamma;\Phi \vdash \sigma +
  \tau}f(\inr\,y) = \inr\,(\setsem{\Gamma;\Phi \vdash \tau}f y)$.
\item If \,$\Gamma;\Phi\vdash \sigma \times \tau$ then
  $\setsem{\Gamma;\Phi \vdash \sigma \times \tau}f = 
  \setsem{\Gamma;\Phi \vdash \sigma}f \times \setsem{\Gamma;\Phi \vdash
    \tau}f$.
\item If \,$\Gamma;\Phi \vdash (\mu \phi.\lambda
  \ol{\alpha}. H)\ol{\tau}$ then $ \setsem{\Gamma;\Phi \vdash (\mu
    \phi.\lambda \ol{\alpha}. H)\ol{\tau}} f : \setsem{\Gamma;\Phi
    \vdash (\mu \phi.\lambda \ol{\alpha}. H)\ol{\tau}} \rho \to
  \setsem{\Gamma;\Phi \vdash (\mu
    \phi.\lambda\ol{\alpha}. H)\ol{\tau}} \rho'$ $ = (\mu
  T^\set_{H,\rho})\ol{\setsem{\Gamma;\Phi \vdash \tau}\rho} \to (\mu
  T^\set_{H,\rho'})\ol{\setsem{\Gamma;\Phi \vdash \tau}\rho'}$ is
  defined by $ (\mu T^\set_{H,f})\ol{\setsem{\Gamma;\Phi \vdash
      \tau}\rho'} \circ (\mu T^\set_{H,\rho})\ol{\setsem{\Gamma;\Phi
      \vdash \tau}f}$ $= (\mu T^\set_{H,\rho'})\ol{\setsem{\Gamma;\Phi
      \vdash \tau}f} \circ (\mu T^\set_{H,f})\ol{\setsem{\Gamma;\Phi
      \vdash \tau}\rho}$.  The latter equality holds because $\mu
  T^\set_{H,\rho}$ and $\mu T^\set_{H,\rho'}$ are functors and $\mu
  T_{H,f}^\set : \mu T_{H,\rho}^\set \to \mu T_{H,\rho'}^\set$ is a
  natural transformation, so the following naturality square commutes:
\begin{equation}\label{eq:cd3}
\begin{CD}
 (\mu T^\set_{H,\rho}) \ol{\setsem{\Gamma;\Phi \vdash \tau}\rho} @> (\mu
  T^\set_{H,f})_{\ol{\setsem{\Gamma;\Phi \vdash \tau}\rho}} >> (\mu
  T^\set_{H,\rho'}) \ol{\setsem{\Gamma;\Phi \vdash \tau}\rho} \\ 
 @V(\mu T^\set_{H,\rho}) \ol{\setsem{\Gamma;\Phi \vdash \tau}f}VV @V  (\mu
 T^\set_{H,\rho'}) \ol{\setsem{\Gamma;\Phi \vdash \tau}f} VV \\ 
(\mu T^\set_{H,\rho}) \ol{\setsem{\Gamma;\Phi \vdash \tau}\rho'} @>(\mu
 T^\set_{H,f})_{\ol{\setsem{\Gamma;\Phi \vdash \tau}\rho'}}>> (\mu
 T^\set_{H,\rho'}) \ol{\setsem{\Gamma;\Phi \vdash \tau}\rho'}
\end{CD}
\end{equation}
\end{itemize}
\end{dfn}

\subsection{Interpreting Types as Relations}\label{sec:rel-interp}

\begin{dfn}\label{def:rel-transf}
A {\em $k$-ary relation transformer} $F$ is a triple $(F^1, F^2,F^*)$,
where $F^1,F^2 : [\set^k,\set]$ are functors, $F^* : [\rel^k, \rel]$
is a functor, if $R_1:\rel(A_1,B_1),...,R_k:\rel(A_k,B_k)$, then $F^*
\ol{R} : \rel(F^1 \ol{A}, F^2 \ol{B})$, and if $(\alpha_1, \beta_1)
\in \Homrel(R_1,S_1),..., (\alpha_k, \beta_k) \in \Homrel(R_k,S_k)$
then $F^* \ol{(\alpha, \beta)} = (F^1 \ol{\alpha}, F^2 \ol{\beta})$.
We define $F\ol{R}$ to be $F^*\overline{R}$ and
$F\overline{(\alpha,\beta)}$ to be $F^*\overline{(\alpha,\beta)}$.
\end{dfn}
The last clause of Definition~\ref{def:rel-transf} expands to: if
$\ol{(a,b) \in R}$ implies $\ol{(\alpha\,a,\beta\,b) \in S}$ then
$(c,d) \in F^*\ol{R}$ implies $(F^1 \ol{\alpha}\,c,F^2 \ol{\beta}\,d)
\in F^*\ol{S}$. When convenient we identify a $0$-ary relation
transformer $(A,B,R)$ with $R : \rel(A,B)$. We may also write $\pi_1
F$ for $F^1$ and $\pi_2 F$ for $F^2$. We extend these conventions to
relation environments, introduced in Definition~\ref{def:reln-env}
below, in the obvious way.

\begin{dfn}
The category $RT_k$ of $k$-ary relation transformers is given by the
following data:
\begin{itemize}
\item An object of $RT_k$ is a relation transformer.
\item A morphism $\delta : (G^1,G^2,G^*) \to (H^1,H^2,H^*)$ in $RT_k$
  is a pair of natural transformations $(\delta^1, \delta^2)$ where
  $\delta^1 : G^1 \to H^1$, $\delta^2 : G^2 \to H^2$ such that, for
  all $\ol{R : \rel(A, B)}$, if $(x, y) \in G^*\ol{R}$ then
  $(\delta^1_{\ol{A}}x, \delta^2_{\ol{B}}y) \in H^*\ol{R}$.
%  {\color{blue} Comment that this is basically a fibred natural
%    transformation, but for heterogeneous relations.}
\item Identity morphisms and composition are inherited from the
  category of functors on $\set$.
\end{itemize}
\end{dfn}

\begin{dfn}\label{def:RT-functor}
An endofunctor $H$ on $RT_k$ is a triple $H = (H^1,H^2,H^*)$, where
\begin{itemize}
\item $H^1$ and $H^2$ are functors from $[\set^k,\set]$ to $[\set^k,\set]$
\item $H^*$ is a functor from $RT_k$ to $[\rel^k,\rel]$
\item for all $\overline{R : \rel(A,B)}$,
  $\pi_1((H^*(\delta^1,\delta^2))_{\overline{R}}) = (H^1
  \delta^1)_{\overline{A}}$ and
  $\pi_2((H^*(\delta^1,\delta^2))_{\overline{R}}) = (H^2
  \delta^2)_{\overline{B}}$
\item The action of $H$ on objects is given by $H\,(F^1,F^2,F^*) =
  (H^1F^1,\,H^2F^2,\,H^*(F^1,F^2,F^*))$
\item The action of $H$ on morphisms is given by
  $H\,(\delta^1,\delta^2) = (H^1\delta^1,H^2\delta^2)$ for
  $(\delta^1,\delta^2) : (F^1,F^2,F^*)\to (G^1,G^2,G^*)$
\end{itemize}
\end{dfn}
Since the results of applying an endofunctor $H$ to $k$-ary relation
transformers and morphisms between them must again be $k$-ary relation
transformers and morphisms between them, respectively,
Definition~\ref{def:RT-functor} implicitly requires that the following
three conditions hold:\,{\em i})
if $R_1:\rel(A_1,B_1),...,R_k:\rel(A_k,B_k)$, then
  $H^*(F^1,F^2,F^*) \ol{R} : \rel(H^1F^1 \ol{A}, H^2F^2 
  \ol{B})$;
  %In other words, $\pi_1 (H^*(F^1,F^2,F^*) \ol{R}) = H^1F^1
  %\ol{A}$ and $\pi_2 (H^*(F^1,F^2,F^*) \ol{R}) = H^2F^2 \ol{B}$.
{\em ii}) if $(\alpha_1, \beta_1) \in \Homrel(R_1,S_1),..., (\alpha_k,
  \beta_k) \in \Homrel(R_k,S_k)$, then
  $H^*(F^1,F^2,F^*)\, \ol{(\alpha, \beta)} = (H^1F^1\ol{\alpha}, H^2F^2
  \ol{\beta})$; and
  %In other words, $\pi_1 (H^*(F^1,F^2,F^*) \ol{(\alpha,
  %  \beta)}) = H^1F^1 \ol{\alpha}$ and $\pi_2 (H^*(F^1,F^2,F^*) 
  % \ol{(\alpha, \beta)}) = H^2F^2 \ol{\beta}$.  {\em iii}) if
  $(\delta^1,\delta^2) : (F^1,F^2,F^*)\to (G^1,G^2,G^*)$ and
  $R_1:\rel(A_1,B_1),...,R_k:\rel(A_k,B_k)$, then
  $((H^1\delta^1)_{\ol{A}}x, (H^2\delta^2)_{\ol{B}}y) \in
  H^*(G^1,G^2,G^*)\ol{R}$ whenever $(x, y) \in
  H^*(F^1,F^2,F^*)\ol{R}$. Note, however, that this last condition is
  automatically satisfied because it is implied by the third bullet
  point of Definition~\ref{def:RT-functor}.

\begin{dfn}\label{def:RT-nat-trans}
If $H$ and $K$ are endofunctors on $RT_k$, then a {\em natural
  transformation} $\sigma : H \to K$ is a pair $\sigma = (\sigma^1,
\sigma^2)$, where $\sigma^1 : H^1 \to K^1$ and $\sigma^2 : H^2 \to
K^2$ are natural transformations between endofunctors on
$[\set^k,\set]$ and the component of $\sigma$ at $F
%= (F^1,F^2,F^*)
\in RT_k$ is given by $\sigma_F = (\sigma^1_{F^1}, \sigma^2_{F^2})$.
\end{dfn}
Definition~\ref{def:RT-nat-trans} entails that $\sigma^i_{F^i}$ must
be natural in $F^i : [\set^k,\set]$, and, for every $F$, both
$(\sigma^1_{F^1})_{\overline{A}}$ and
$(\sigma^2_{F^2})_{\overline{A}}$ must be natural in $\overline{A}$.
Moreover, since the results of applying $\sigma$ to $k$-ary relation
transformers must be morphisms of $k$-ary relation transformers,
Definition~\ref{def:RT-nat-trans} implicitly requires that
$(\sigma_F)_{\overline{R}} = ( (\sigma^1_{F^1})_{\overline{A}},
(\sigma^2_{F^2})_{\overline{B}})$ is a morphism in $\rel$ for any
$k$-tuple of relations $\overline{R : \rel(A, B)}$, i.e., that if $(x,
y) \in H^*F\overline{R}$, then $((\sigma^1_{F^1})_{\overline{A}} x,
(\sigma^2_{F^2})_{\overline{B}} y) \in K^*F\overline{R}$.

\vspace*{0.1in}

{\color{blue} More context? Weave cocontinuity requirement for fixed
  points through text better.}  Critically, we can compute
$\omega$-directed colimits in $RT_k$: indeed, if $\cal D$ is an
$\omega$-directed set, then $\colim{d \in {\cal D}}{(F^1_d,
  F^2_d,F^*_d)} = (\colim{d \in {\cal D}}{F^1_d}, \colim{d \in {\cal
    D}}{F^2_d}, \colim{d \in {\cal D}}{F^*_d})$.  We then define an
endofunctor $T = (T^1,T^2,T^*)$ on $RT_k$ to be {\em
  $\omega$-cocontinuous} if $T^1$ and $T^2$ are $\omega$-cocontinuous
endofunctors on $[\set^k,\set]$ and $T^*$ is an $\omega$-cocontinuous
functor from $RT_k$ to $[\rel^k,\rel]$, i.e., is in
$[RT_k,[\rel^k,\rel]]$.
%\begin{lemma}\label{lem:colimits}
%$\colim{d \in {\cal D}}{(F^1_d, F^2_d,F^*_d)} = (\colim{d \in {\cal
%      D}}{F^1_d}, \colim{d \in {\cal D}}{F^2_d}, \colim{d \in {\cal
%      D}}{F^*_d})$
%\end{lemma}
%\begin{proof}
%We first observe that $(\colim{d \in {\cal D}}{F^1_d}, \colim{d \in
%  {\cal D}}{F^2_d}, \colim{d \in {\cal D}}{F^*_d})$ is in $RT_k$.  If
%$R_1:\rel(A_1,B_1),...,R_k:\rel(A_k,B_k)$, then $\colim{d \in \cal
%  D}{F^*_d\ol{R}} : \rel(\colim{d \in {\cal D}}{F^1_d\ol{A}},
%\,\colim{d \in {\cal D}}{F^2_d\ol{B}})$ because of how colimits are
%computed in $\rel$. Moreover, if $(\alpha_1, \beta_1) \in
%\Homrel(R_1,S_1),..., (\alpha_k, \beta_k) \in \Homrel(R_k,S_k)$, then
%\[\begin{array}{ll}
%  & (\colim{d \in \cal D}{F_d^*})\ol{(\alpha,\beta)}\\
%= & \colim{d \in \cal D}{F_d^*\ol{(\alpha,\beta)}}\\
%= & \colim{d \in \cal D}{(F_d^1\ol{\alpha},\, F_d^2\ol{\beta})}\\
%= & (\colim{d \in \cal D}{F^1_d \ol{\alpha}}, \,\colim{d \in \cal D}{
%  F^2_d \ol{\beta}})
%\end{array}\]
%so $(\colim{d \in {\cal D}}{F^1_d}, \colim{d \in {\cal D}}{F^2_d},
%\colim{d \in {\cal D}}{F^*_d})$ actually is in $RT_k$.
%
%Now to see that $\colim{d \in {\cal D}}{(F^1_d, F^2_d,F^*_d)} =
%(\colim{d \in {\cal D}}{F^1_d}, \colim{d \in {\cal D}}{F^2_d},
%\colim{d \in {\cal D}}{F^*_d})$, let $\gamma^1_d : F^1_d \to \colim{d
%  \in {\cal D}}{F^1_d}$ and $\gamma^2_d : F^2_d \to \colim{d \in {\cal
%    D}}{F^2_d}$ be the injections for the colimits $\colim{d \in {\cal
%    D}}{F^1_d}$ and $\colim{d \in {\cal D}}{F^2_d}$,
%respectively. Then $(\gamma^1_d, \gamma^2_d) : (F^1_d, F^2_d,F^*_d)
%\to \colim{d \in {\cal D}}{(F^1_d, F^2_d,F^*_d)}$ is a morphism in
%$RT_k$ because, for all $\ol{R : \rel(A, B)}$,
%$((\gamma^1_d)_{\ol{A}}, (\gamma^2_d)_{\ol{B}}) : F^*_d \ol{R} \to
%\colim{d \in {\cal D}}{F^*_d \ol{R}}$ is a morphism in $\rel$. So
%$\{(\gamma^1_d, \gamma^2_d)\}_{d \in {\cal D}}$ are the mediating
%morphisms of a cocone in $RT_k$ with vertex $\colim{d \in {\cal
%    D}}{(F^1_d, F^2_d,F^*_d)}$. To see that this cocone is a
%colimiting cocone, let $C = (C^1,C^2,C^*)$ be the vertex of a cocone
%for $\{(F^1_d,F^2_d,F^*_d)\}_{d \in {\cal D}}$ with injections
%$(\delta^1_d,\delta^2_d) : (F^1_d,F^2_d,F^*_d) \to C$. If $\eta^1 :
%\colim{d \in {\cal D}}{F^1_d} \to C^1$ and $\eta^2 : \colim{d \in
%  {\cal D}}{F^2_d} \to C^2$ are the mediating morphisms in
%$[\set^k,\set]$, then $\eta^1$ and $\eta^2$ are unique such that
%$\delta^1_d = \eta^1 \circ \gamma^1_d$ and $\delta^2_d = \eta^2 \circ
%\gamma^2_d$.  We therefore have that $(\eta^1,\eta^2) : \colim{d \in
%  {\cal D}}{(F^1_d, F^2_d,F^*_d)} \to C$ is the mediating morphism in
%$RT_k$. Indeed, for all $\ol{R : \rel(A,B)}$ and $(x,y) \in \colim{d
%  \in {\cal D}}{F^*_d \ol{R}}$, there exist $d$ and $(x',y') \in
%F^*_d\ol{R}$ such that $(\gamma^1_d)_{\ol{A}}x' = x$ and
%$(\gamma^2_d)_{\ol{B}}y' = y$. But then $(\eta^1_{\ol{A}}x,
%\eta^2_{\ol{B}}y) = (\eta^1_{\ol{A}}((\gamma^1_d)_{\ol{A}}x'),
%\eta^2_{\ol{B}}((\gamma^2_d)_{\ol{B}}y')) = ((\delta^1_d)_{\ol{A}}x',
%(\delta^2_d)_{\ol{B}}y')$, and this pair is in $C^*\ol{R}$ because
%$(\delta^1_d, \delta^2_d)$ is a morphism from $(F^1_d,F^2_d,F^*_d)$ to
%$C$ in $RT_k$.
%\end{proof}
\begin{comment}
\begin{dfn}\label{def:omega-cocont}
An endofunctor $T = (T^1,T^2,T^*)$ on $RT_k$ is {\em
$\omega$-cocontinuous} if $T^1$ and $T^2$ are $\omega$-cocontinuous
endofunctors on $[\set^k,\set]$ and $T^*$ is an $\omega$-cocontinuous
functor from $RT_k$ to $[\rel^k,\rel]$, i.e., is in
$[RT_k,[\rel^k,\rel]]$.
\end{dfn}
\end{comment}
Now, for any $k$ and $R : \rel(A, B)$, let $K^\rel_R$ be the
constantly $R$-valued functor from $\rel^k$ to $\rel$, and for any $k$
and set $A$, let $K^\set_A$ be the constantly $A$-valued functor from
$\set^k$ to $\set$, and let $0$ denote either the initial object of
$\set$ or the initial object of $\rel$, as appropriate.  Observing
that, for every $k$, $K^\set_0$ is initial in $[\set^k,\set]$, and
$K^\rel_0$ is initial in $[\rel^k,\rel]$, we have that, for each $k$,
$K_0 = (K^\set_0,K^\set_0,K^\rel_0)$ is initial in $RT_k$. Thus, if $T
= (T^1,T^2,T^*) : RT_k \to RT_k$ is an endofunctor on $RT_k$ then we
can define the relation transformer $\mu T$ to be $\colim{n \in
  \nat}{T^n K_0}$.
%Then Lemma~\ref{lem:colimits} shows $\mu T$ is indeed a relation
%transformer, and that it is given explicitly by
It is not hard to see that $\mu T$ is given explicitly as 
\begin{equation}\label{eq:mu}
  %\colim{n \in \nat}{T^n K_0}
  \mu T = (\mu T^1,\mu T^2,
\colim{n \in \nat}{(T^nK_0)^*})
\end{equation}
and that, as our notation suggests, it really is a fixpoint for $T$ if
$T$ is $\omega$-cocontinuous:
\begin{lemma}\label{lem:fp}
For any $T : [RT_k,RT_k]$, $\mu T \cong T(\mu T)$.
\end{lemma}
\noindent
The isomorphism is given by the morphisms $(in_1, in_2) : T(\mu T)
\to \mu T$ and $(in_1^{-1}, in_2^{-1}) : \mu T \to T(\mu T)$ in
$RT_k$. The latter is always a morphism in $RT_k$, but the
former need not be if $T$ is not $\omega$-cocontinuous.

\begin{comment}
\begin{proof}
We have $T(\mu T) = T(\colim{n \in \nat}{(T^nK_0)}) \cong \colim{n \in
\nat}{T(T^nK_0)} =
\mu T$.
\end{proof}
In fact, the isomorphism in Lemma~\ref{lem:fp} is given by the
morphisms $(in_1, in_2) : T(\mu T) \to \mu T$ and $(in_1^{-1},
in_2^{-1}) : \mu T \to T(\mu T)$ in $RT_k$. It is worth noting that the
latter is always a morphism in $RT_k$, but the former isn't necessarily
a morphism in $RT_k$ unless $T$ is $\omega$-cocontinuous.
\end{comment}

{\color{blue} CHECK!}
It is worth noting that the third component in Equation~(\ref{eq:mu})
is the colimit in $[\rel^k,\rel]$ of third components of relation
transformers, rather than a fixpoint of an endofunctor on
$[\rel^k,\rel]$. That there is an asymmetry between the first two
components of $\mu T$ and its third is an important conceptual
observation, and reflects the fact that the third component of an
endofunctor on $RT_k$ need not be a functor on all of
$[\rel^k,\rel]$. For example, although we can define $T_{H,\rho}\, F$
for a relation transformer $F$ in Definition~\ref{def:rel-sem} below,
it is not clear how we could define it for $F : [\rel^k,\rel]$.

\begin{dfn}\label{def:reln-env}
A {\em relation environment} maps each each type variable in $\tvars^k
\cup \fvars^k$ to a $k$-ary relation transformer.  A morphism $f :
\rho \to \rho'$ for relation environments $\rho$ and $\rho'$ with
$\rho|_\tvars = \rho'|_\tvars$ maps each type constructor variable
$\psi^k \in \tvars$ to the identity morphism on $\rho \psi^k = \rho'
\psi^k$ and each functorial variable $\phi^k \in \fvars$ to a morphism
from the $k$-ary relation transformer $\rho \phi$ to the $k$-ary
relation transformer $\rho' \phi$. Composition of morphisms on
relation environments is given componentwise, with the identity
morphism mapping each relation environment to itself. This gives a
category of relation environments and morphisms between them, which we
denote $\relenv$.
\end{dfn}
When convenient we identify a $0$-ary relation transformer with the
relation (transformer) that is its codomain and consider
% With this convention,
a relation environment to map a type variable of arity $0$
%to a $0$-ary relation transformer, i.e.,
to a relation.  We write $\rho[\ol{\alpha := R}]$ for the relation
environment $\rho'$ such that $\rho' \alpha_i \, = R_i$ for $i =
1,...,k$ and $\rho' \alpha = \rho\alpha$ if $\alpha \not \in
\{\alpha_1,...,\alpha_k\}$.  If $\rho$ is a relation environment, we
write $\pi_1 \rho$ and $\pi_2 \rho$ for the set environments mapping
each type variable $\phi$ to the functors $(\rho\phi)^1$ and
$(\rho\phi)^2$, respectively.

We define, for each $k$, the notion of an $\omega$-cocontinuous
functor from $\relenv$ to $RT_k$:
\begin{dfn}\label{def:relenv-functor}
A functor $H : [\relenv, RT_k]$ is a triple $H = (H^1,H^2,H^*)$,
where
\begin{itemize}
\item $H^1$ and $H^2$ are objects in $[\setenv,[\set^k,\set]]$
\item $H^*$ is a an object in $[\relenv,[\rel^k,\rel]]$
\item for all $\overline{R : \rel(A,B)}$ and morphisms $f$ in
  $\relenv$, $\pi_1(H^*f \,{\overline{R}}) = H^1 (\pi_1
  f)\,{\overline{A}}$ and $\pi_2(H^*f \,{\overline{R}}) = H^2 (\pi_2
  f)\,{\overline{B}}$
\item The action of $H$ on $\rho$ in $\relenv$ is given by $H \rho = (H^1
  (\pi_1 \rho),\,H^2 (\pi_2 \rho),\,H^*\rho)$
\item The action of $H$ on morphisms $f : \rho \to \rho'$ in $\relenv$
  is given by $Hf = (H^1 (\pi_1 f),H^2 (\pi_2 f))$
\end{itemize}
\end{dfn}
\noindent Spelling out the last two bullet points above gives the
following analogues of the three conditions immediately following
Definition~\ref{def:RT-functor}: {\em i}) if $R_1 :
\rel(A_1,B_1),...,R_k : \rel(A_k,B_k)$, then $H^*\rho\, \ol{R} :
\rel(H^1(\pi_1 \rho)\, \ol{A}, H^2(\pi_2 \rho)\, \ol{B})$;
%In other words, $\pi_1 (H^*\rho \ol{R}) = H^1(\pi_1 \rho)
%  \,\ol{A}$ and $\pi_2 (H^*\rho \ol{R}) = H^2 (\pi_2 \rho) \,\ol{B}$;
{\em ii}) if $(\alpha_1, \beta_1) \in \Homrel(R_1,S_1),..., (\alpha_k,
\beta_k) \in \Homrel(R_k,S_k)$, then $H^*\rho\, \ol{(\alpha, \beta)} =
(H^1(\pi_1 \rho)\,\ol{\alpha}, H^2(\pi_2 \rho)\, \ol{\beta})$;
%  In other words, $\pi_1 (H^*\rho\, \ol{(\alpha, \beta)}) = H^1(\pi_1
%  \rho)\,\ol{\alpha}$ and $\pi_2 (H^*\rho\, \ol{(\alpha, \beta)}) =
%  H^2(\pi_2 \rho)\,\ol{\beta}$;
and {\em iii}) if $f : \rho \to \rho'$ and
$R_1:\rel(A_1,B_1),...,R_k:\rel(A_k,B_k)$, then $(H^1(\pi_1
f)\,{\ol{A}}\,x, H^2(\pi_2 f)\,{\ol{B}}\,y) \in H^*\rho'\,\ol{R}$
whenever $(x, y) \in H^*\rho\,\ol{R}$. As before, the last
condition is automatically satisfied because it is implied by the
third bullet point of Definition~\ref{def:relenv-functor}.

Considering $\relenv$ as a product $\Pi_{\phi^k \in \tvars \cup
  \fvars} RT_k$, we extend the computation of $\omega$-directed
colimits in $RT_k$ to compute colimits in $\relenv$ componentwise. We
similarly extend the notion of an $\omega$-cocontinuous endofunctor on
$RT_k$ componentwise to give a notion of $\omega$-cocontinuity for
functors from $\relenv$ to $RT_k$.  Recalling from the start of this
subsection that Definition~\ref{def:rel-sem} is given mutually
inductively with Definition~\ref{def:set-sem} we can, at last, define:

\begin{dfn}\label{def:rel-sem}
%Let $\rho$ be a relation environment.
The {\em relational interpretation} $\relsem{\cdot} : \F \to [\relenv,
  \rel]$ is defined by
\begin{align*}
  \relsem{\Gamma;\emptyset \vdash v}\rho &= \rho v \mbox{ if
  } v \in \tvars^0 \\ 
  \relsem{\Gamma;\emptyset \vdash \Nat^{\ol{\alpha}}
    \,F\,G}\rho &= \{\eta : \lambda \ol{R}.\,\relsem{\Gamma;
      \ol{\alpha} \vdash F}\rho[\ol{\alpha := R}] \Rightarrow \lambda
  \ol{R}. \,\relsem{
      \Gamma;\ol{\alpha} \vdash G}\rho[\ol{\alpha := R}]\}\\
  &=
  \{(t,t') \in \setsem{\Gamma;\emptyset \vdash \Nat^{\ol{\alpha}}
    \,F\,G} (\pi_1 \rho) \times \setsem{ 
      \Gamma;\emptyset \vdash \Nat^{\ol{\alpha}} \,F\,G} (\pi_2
  \rho)~|~\\ 
  & \hspace{0.3in} \forall {R_1 : \rel(A_1,B_1)}\,...\,{R_k : \rel(A_k,B_k)}.\\
  & \hspace{0.4in} (t_{\ol{A}},t'_{\ol{B}}) \in
  (\relsem{\Gamma;\ol{\alpha} \vdash G}\rho[\ol{\alpha :=
      R}])^{\relsem{\Gamma;\ol{\alpha}\vdash F}\rho[\ol{\alpha := R}]} \}\\  
  % exponential in $\rel$
%  &= \{(t,t') \in \setsem{\Gamma;\emptyset \vdash \Nat^{
%      \ol{\alpha}} \,F\,G} (\pi_1 \rho) \times 
%  \setsem{\Gamma;\emptyset \vdash \Nat^{\ol{\alpha}} \,F\,G}
%  (\pi_2 \rho)~|~\\ 
%  & \hspace{0.3in} \forall {R_1 : \rel(A_1,B_1)}\,...\,{R_k : \rel(A_k,B_k)}.\\
%  & \hspace{0.4in} \forall (a,b) \in \relsem{\Gamma;
%      \ol{\alpha} \vdash F}\rho[\ol{\alpha := R}].\\
%  & \hspace*{0.5in}  (t_{\ol{A}}a,t'_{\ol{B}}b) \in \relsem{
%      \Gamma; \ol{\alpha} \vdash G}\rho[\ol{\alpha := R}] \}\\
  \relsem{\Gamma;\Phi \vdash \zerot}\rho &= 0\\
  \relsem{\Gamma;\Phi \vdash \onet}\rho &= 1\\
  \relsem{\Gamma;\Phi \vdash \phi \ol{\tau}}\rho &=
  (\rho\phi)\ol{\relsem{\Gamma;\Phi \vdash 
    \tau}\rho}\\
  \relsem{\Gamma;\Phi \vdash \sigma+\tau}\rho &=
  \relsem{\Gamma;\Phi \vdash \sigma}\rho +
  \relsem{\Gamma;\Phi \vdash \tau}\rho\\
  \relsem{\Gamma;\Phi \vdash \sigma\times \tau}\rho &=
  \relsem{\Gamma;\Phi \vdash \sigma}\rho \times
  \relsem{\Gamma;\Phi \vdash \tau}\rho\\  
   \relsem{\Gamma;\Phi \vdash (\mu \phi.\lambda
    \ol{\alpha}. H)\ol{\tau}}\rho
  &= (\mu T_{H,\rho})\ol{\relsem{\Gamma;\Phi \vdash
     \tau}\rho}\\
  \text{where }	T_{H,\rho}
    &= (T^\set_{H,\pi_1\rho}, T^\set_{H,\pi_2\rho}, T^\rel_{H,\rho}) \\
  \text{and } T^\rel_{H,\rho}\,F
    &= \lambda \ol{R}. \relsem{
      \Gamma;\Phi,\phi,\ol{\alpha} \vdash H}\rho[\phi :=
    F][\ol{\alpha := R}]\\
  \text{and } T^\rel_{H,\rho}\,\delta
    &= \lambda \ol{R}. \relsem{
      \Gamma;\Phi,\phi,\ol{\alpha} \vdash H}\id_\rho[\phi :=
    \delta][\ol{\alpha := \id_{\ol{R}}}]
\end{align*}
\end{dfn}

The interpretations in Definition~\ref{def:rel-sem}, as well as in
Definition~\ref{def:rel-sem-funcs} below, respect weakening.
Definition~\ref{def:rel-sem} also ensures that $\relsem{\Gamma \vdash
  \sigma \to \tau}\rho = \relsem{\Gamma \vdash \sigma} \rho \to
\relsem{\Gamma \vdash \tau} \rho$. If $\rho$ is a relational
environment and $\vdash \tau : \F$, then we write $\relsem{\vdash \tau
}$ instead of $\relsem{\vdash \tau }\rho$ as for set interpretations.
For the last clause in Definition~\ref{def:rel-sem} to be
well-defined, we need to know that $T_{\rho}$ is an
$\omega$-cocontinuous endofunctor on $RT$ so that, by
Lemma~\ref{lem:fp}, it admits a fixed point. Since $T_\rho$ is defined
in terms of $\relsem{\Gamma;\Phi,\phi^k, \ol{\alpha} \vdash H}$, this
means that relational interpretations of types must be
$\omega$-cocontinuous functors from $\relenv$ to $RT_0$, which in turn
entails that the actions of relational interpretations of types on
objects and on morphisms in $\relenv$ are intertwined. As for set
interpretations, we know from~\cite{jp19} that, for every $\Gamma;
\ol{\alpha} \vdash \tau : \F$, $\setsem{\Gamma; \ol{\alpha} \vdash
  \tau}$ is actually in $[\rel^k,\rel]$ where $k = |\ol \alpha|$.
%In fact, we already know from~\cite{jp19} that, for every $\Gamma;
%\ol{\alpha} \vdash \tau : \F$, $\relsem{\Gamma; \ol{\alpha} \vdash
%  \tau}$ is actually functorial in $\ol{\alpha}$ and
%$\omega$-cocontinuous.
We first define the actions of each of these functors on morphisms
between environments in Definition~\ref{def:rel-sem-funcs}, and then
argue that the functors given by Definitions~\ref{def:rel-sem}
and~\ref{def:rel-sem-funcs} are well-defined and have the required
properties. To do this, we extend $T_H$ to a {\em functor} from
$\relenv$ to $[[\rel^k,\rel],[\rel^k,\rel]]$. Its action on an object
$\rho \in \relenv$ is given by the higher-order functor
$T^\rel_{H,\rho}$ whose actions on objects and morphisms are given in
Definition~\ref{def:rel-sem-funcs}. Its action on a morphism $f : \rho
\to \rho'$ is the higher-order natural transformation $T_{H,f} :
T_{H,\rho} \to T_{H,\rho'}$ whose action on any $F : [\rel^k,\rel]$ is
the natural transformation $T_{H,f}\, F : T_{H,\rho}\, F \to
T_{H,\rho'}\, F$ whose component at $\ol{R}$ is $(T_{H,f}\,
F)_{\ol{R}} = \relsem{\Gamma; \Phi,\phi,\ol{\alpha} \vdash H}f[\phi :=
  \id_F][\ol{\alpha := \id_R}]$.  The next definition uses the functor
$T_H$ to define the actions of functors interpreting types on
morphisms between relation environments.
    
\begin{dfn}\label{def:rel-sem-funcs}
Let $f: \rho \to \rho'$ for relation environments $\rho$ and $\rho'$
(so that $\rho|_\tvars = \rho'|_\tvars$). The action
$\relsem{\Gamma;\Phi \vdash \tau}f$ of $\relsem{\Gamma;\Phi \vdash
  \tau}$ on the morphism $f$ is given as follows:
\begin{itemize}
\item If $\Gamma,v;\emptyset \vdash v$ then
  $\relsem{\Gamma,v;\emptyset \vdash v}f = \id_{\rho v}$
\item If $\Gamma; \emptyset \vdash \Nat^{\ol{\alpha}}\,F\,G$, then
  $\relsem{\Gamma;\emptyset \vdash \Nat^{\ol{\alpha}}\,F\,G} f =
  \id_{\relsem{\Gamma;\emptyset \vdash \Nat^{\ol{\alpha}}\,F\,G}\rho}$
\item If $\Gamma;\Phi \vdash \zerot$ then $\relsem{\Gamma;\Phi \vdash
  \zerot}f = \id_0$
\item If $\Gamma;\Phi \vdash \onet$ then $\relsem{\Gamma;\Phi \vdash
  \onet}f = \id_1$
\item If $\Gamma;\Phi \vdash \phi \ol{\tau}$, then
  $\relsem{\Gamma;\Phi \vdash \phi \ol{\tau}} f : \relsem{\Gamma;\Phi
  \vdash \phi \ol{\tau}}\rho \to \relsem{\Gamma;\Phi \vdash \phi
  \ol{\tau}}\rho' = (\rho\phi) \ol{\relsem{\Gamma;\Phi \vdash
    \tau}\rho} \to (\rho'\phi) \ol{\relsem{\Gamma;\Phi \vdash
    \tau}\rho'}$ is defined by $\relsem{\Gamma;\Phi \vdash \phi
  \tau{A}} f = (f\phi)_{\ol{\relsem{\Gamma;\Phi \vdash \tau}\rho'}}
  \,\circ\, (\rho\phi) \ol{\relsem{\Gamma;\Phi \vdash \tau}f} =
  (\rho'\phi) \ol{\relsem{\Gamma;\Phi \vdash \tau}f} \,\circ\, (f
  \phi)_{\ol{\relsem{\Gamma;\Phi \vdash \tau}\rho}}$
\item If $\Gamma;\Phi\vdash \sigma + \tau$ then $\relsem{\Gamma;\Phi
  \vdash \sigma + \tau}f$ is defined by $\relsem{\Gamma;\Phi \vdash
  \sigma + \tau}f(\inl\,x) = \inl\,(\relsem{\Gamma;\Phi \vdash
  \sigma}f x)$ and $\relsem{\Gamma;\Phi \vdash \sigma +
  \tau}f(\inr\,y) = \inr\,(\relsem{\Gamma;\Phi \vdash \tau}f y)$
\item If $\Gamma;\Phi\vdash \sigma \times \tau$ then
  $\relsem{\Gamma;\Phi \vdash \sigma \times \tau}f =
  \relsem{\Gamma;\Phi \vdash \sigma}f \times \relsem{\Gamma;\Phi
    \vdash \tau}f$
\item If $\Gamma;\Phi \vdash (\mu \phi^k.\lambda
  \ol{\alpha}. H)\ol{\tau}$ then $\relsem{\Gamma;\Phi \vdash (\mu
    \phi.\lambda \ol{\alpha}. H)\ol{\tau}} f = (\mu
  T_{H,f})\ol{\relsem{\Gamma;\Phi \vdash \tau}\rho'} \circ (\mu
  T_{H,\rho})\ol{\relsem{\Gamma;\Phi \vdash \tau}f} = (\mu
  T_{H,\rho'})\ol{\relsem{\Gamma;\Phi \vdash \tau}f} \circ (\mu
  T_{H,f})\ol{\relsem{\Gamma;\Phi \vdash \tau}\rho}$
\end{itemize}
\end{dfn}

To see that the functors given by Definitions~\ref{def:rel-sem}
and~\ref{def:rel-sem-funcs} are well-defined we must show that, for
every $H$, $T_{H,\rho}\,F$ is a relation transformer for any relation
transformer $F$, and that $T_{H,f}\, F : T_{H,\rho}\, F \to
T_{H,\rho'}\, F$ is a morphism of relation transformers for every
relation transformer $F$ and every morphism $f : \rho \to \rho'$ in
$\relenv$. This is an immediate consequence of 
\begin{lemma}\label{lem:rel-transf-morph}
%The interpretations in Definitions~\ref{def:rel-sem}
%and~\ref{def:rel-sem-funcs} are well-defined and,
For every $\Gamma;\Phi \vdash \tau$, $\sem{\Gamma;\Phi \vdash \tau} =
(\setsem{\Gamma;\Phi \vdash \tau}, \setsem{\Gamma;\Phi \vdash
  \tau},\relsem{\Gamma;\Phi \vdash \tau}) \in
%is an $\omega$-cocontinuous functor from $\relenv$ to $RT_0$, i.e., is an
%element of
[\relenv,RT_0]$.
\end{lemma}
\noindent
The proof is a straightforward induction on the structure of $\tau$,
%%each case of which first shows that $(\setsem{\Gamma;\Phi \vdash
%  \tau}, \setsem{\Gamma;\Phi \vdash \tau},\relsem{\Gamma;\Phi \vdash
%  \tau})$ is indeed a functor from $\relenv$ to $RT_0$, and then uses
using an appropriate result from~\cite{jp19} to deduce
$\omega$-cocontinuity of $\sem{\Gamma;\Phi \vdash \tau}$ in each case.

\begin{comment}
\begin{proof}
By induction on the structure of $\tau$. The only interesting cases
are when $\tau = \phi\ol{\tau}$ and when $\tau = (\mu \phi^k. \lambda
\overline{\alpha}. H)\overline{\tau}$. We consider each in turn.

\begin{itemize}
\item When $\tau = \Gamma; \Phi \vdash \phi\ol{\tau}$, we have    
\[\begin{array}{ll}
  & \pi_i (\relsem{\Gamma; \Phi \vdash \phi\ol{\tau}}\rho)\\
= & \pi_i ((\rho \phi)\overline{\relsem{\Gamma; \Phi \vdash \tau}\rho})\\
= & (\pi_i (\rho \phi)) (\pi_i(\overline{\relsem{\Gamma; \Phi \vdash \tau}\rho}))\\
= & ((\pi_i \rho) \phi) (\overline{\setsem{\Gamma; \Phi \vdash
    \tau}(\pi_i \rho)})\\
= & \setsem{\Gamma; \Phi \vdash \phi\ol{\tau}}(\pi_i \rho)
\end{array}\]
and, for $f : \rho \to \rho'$ in $\relenv$,
\[\begin{array}{ll}
  & \pi_i (\relsem{\Gamma; \Phi \vdash \phi\ol{\tau}}f)\\
= & \pi_i ((f\phi)_{\overline{\relsem{\Gamma; \Phi \vdash \tau}\rho'}})
\circ \pi_i((\rho\phi)({\overline{\relsem{\Gamma; \Phi \vdash \tau}f}}))\\
= & (\pi_i (f\phi))_{\overline{\pi_i (\relsem{\Gamma; \Phi \vdash \tau}\rho')}}
\circ (\pi_i(\rho\phi))({\overline{\pi_i (\relsem{\Gamma; \Phi \vdash \tau}f}}))\\
= & ((\pi_i f)\phi)_{\overline{\setsem{\Gamma; \Phi \vdash \tau}(\pi_i\rho')}}
\circ ((\pi_i\rho)\phi)({\overline{\setsem{\Gamma; \Phi \vdash \tau}(\pi_i f)}})\\
= & \setsem{\Gamma; \Phi \vdash \phi\ol{\tau}}(\pi_i f)
\end{array}\]
The third equalities of each of the above derivations are by the
induction hypothesis. That $\sem{\Gamma; \Phi \vdash
  \phi\ol{\tau}}$ is $\omega$-cocontinuous is an immediate
consequence of the facts that $\set$ and $\rel$ are locally finitely
presentable, together with Corollary~12 of~\cite{jp19}.
\item When $\tau = (\mu \phi. \lambda
  \overline{\alpha}. H)\overline{\tau}$ we first show that $\sem{ (\mu
    \phi. \lambda \overline{\alpha}. H)\overline{\tau}}$ is
  well-defined.
\begin{itemize}
\item \underline{$T_\rho : [RT_k,RT_k]$}:\/ We must show that, for any
  relation transformer $F = 
  (F^1, F^2, F^*)$, the triple $T_{\rho} F = (T^\set_{\pi_1 \rho}F^1,
  T^\set_{\pi_2 \rho}F^2, T^\rel_{\rho}F)$ is also a relation
  transformer.  Let $\overline{R : \rel(A, B)}$. Then for
  $i = 1, 2$, we have
\[\begin{split}
\pi_i(T^\rel_{\rho}\,F\,\overline{R})
&= \pi_i(\relsem{\Gamma;\Phi,\phi,\overline{\alpha} \vdash H}\rho[\phi := F]\overline{[\alpha := R]}) \\
&= \setsem{\Gamma;\Phi,\phi,\overline{\alpha} \vdash H} (\pi_i (\rho[\phi := F]\overline{[\alpha := R]})) \\
&= \setsem{\Gamma;\Phi,\phi,\overline{\alpha} \vdash H} (\pi_i \rho)[\phi := \pi_i F]\overline{[\alpha := \pi_i R]}) \\
&= T^\set_{\pi_i \rho} (\pi_i F) (\overline{\pi_i R})
\end{split}\]
and 
\[\begin{split}
\pi_i(T^\rel_{\rho}\,F\,\overline{\gamma})
&= \pi_i(\relsem{\Gamma;\Phi,\phi,\overline{\alpha} \vdash H}\id_\rho[\phi
  := \id_F]\overline{[\alpha := \gamma]}) \\
&= \setsem{\Gamma;\Phi,\phi,\overline{\alpha} \vdash H} (\pi_i (\id_\rho[\phi := \id_F]\overline{[\alpha := \gamma]})) \\
&= \setsem{\Gamma;\Phi,\phi,\overline{\alpha} \vdash H} \id_{\pi_i \rho}[\phi := \id_{\pi_i F}]\overline{[\alpha := \pi_i \gamma]} \\
&= T^\set_{\pi_i \rho} (\pi_i F) (\overline{\pi_i \gamma})
\end{split}\]
Here, the second equality in each of the above chains of equalities is
by the induction hypothesis.

We also have that, for every morphism $\delta = (\delta^1, \delta^2) :
F \to G$ in $RT_k$ and all $\overline{R : \rel(A, B)}$,
\[\begin{array}{ll}
  & \pi_i((T^\rel_\rho \delta)_{\overline{R}})\\
= & \pi_i(\relsem{\Gamma;\Phi,\phi,\overline{\alpha} \vdash H}\id_\rho[\phi :=
  \delta]\overline{[\alpha := \id_R]})\\
= & \setsem{\Gamma;\Phi,\phi,\overline{\alpha} \vdash H}\id_{\pi_i\rho}[\phi :=
  \pi_i \delta]\overline{[\alpha := \id_{\pi_i R}]}\\
= & (T^\set_{\pi_i \rho} (\pi_i \delta))_{\overline{\pi_i R}}
\end{array}\]

Here, the second equality is by the induction hypothesis.  That
$T_\rho$ is $\omega$-cocontinuous follows immediately from the
induction hypothesis on $\sem{\Gamma;\Phi,\phi,\ol{\alpha} \vdash H}$
and the fact that colimts are computed componentwise in $RT$.
\item \underline{$\sigma_f = (\sigma^\set_{\pi_1 f},
  \sigma^\set_{\pi_2 f})$ is a natural transformation from $T_{\rho}$
  to $T_{\rho'}$}:\/ We must show that $(\sigma_f)_F =
  ((\sigma^\set_{\pi_1 f})_{F^1}, (\sigma^\set_{\pi_2 f})_{F^2})$ is a
  morphism in $RT_k$ for all relation transformers $F = (F^1, F^2,
  F^*)$, i.e., that $((\sigma_f)_F)_{\overline{R}}
  =(((\sigma^\set_{\pi_1 f})_{F^1})_{\overline{A}},
  ((\sigma^\set_{\pi_2 f})_{F_2})_{\overline{B}})$ is a morphism in
  $\rel$ for all relations $\overline{R : \rel(A, B)}$. Indeed, we
  have that
\[
((\sigma_f)_F)_{\overline{R}} =
\relsem{\Gamma;\Phi,\phi,\overline{\alpha} \vdash H}f[\phi :=
  \id_F]\overline{[\alpha := \id_R]}
\]
is a morphism in $RT_0$ (and thus in $\rel$) by the induction hypothesis.
\end{itemize}

\vspace*{0.05in}

The relation transformer $\mu T_\rho$ is therefore a fixed point of
$T_\rho$ by Lemma~\ref{lem:fp}, and $\mu \sigma_f$ is a morphism in
$RT_k$ from $\mu T_\rho$ to $\mu T_{\rho'}$. ($\mu$ is shown to be a
functor in~\cite{jp19}.)  So $\relsem{\Gamma; \Phi \vdash (\mu
  \phi. \lambda \overline{\alpha}. H)\overline{\tau}}$, and thus
$\sem{\Gamma; \Phi \vdash (\mu \phi. \lambda
  \overline{\alpha}. H)\overline{\tau}}$, is well-defined.

\vspace*{0.1in}

To see that $\sem{\Gamma; \Phi \vdash (\mu \phi. \lambda
  \overline{\alpha}. H)\overline{\tau}} : [\relenv,RT_0]$, we must
verify three conditions:
\begin{itemize}
\item Condition (1) after Definition~\ref{def:relenv-functor} is
  satisfied since
\[\begin{split}
\pi_i(\relsem{\Gamma;\Phi \vdash (\mu \phi. \lambda
  \overline{\alpha}. H) \overline{\tau}}\rho) 
&= \pi_i( (\mu T_\rho) (\overline{\relsem{\Gamma;\Phi \vdash
    \tau}\rho})) \\ 
&= \pi_i(\mu T_{\rho}) (\overline{\pi_i(\relsem{\Gamma;\Phi \vdash
    \tau}\rho})) \\ 
&= \mu T^\set_{\pi_i\rho} (\overline{\setsem{\Gamma;\Phi \vdash
    \tau}(\pi_i\rho)}) \\ 
&= \setsem{\Gamma;\Phi \vdash (\mu \phi. \lambda \overline\alpha. H)
  \overline{\tau}}(\pi_i\rho) 
\end{split}\]
The third equality is by Equation ~\ref{eq:mu} and the induction
hypothesis.
\item Condition (2) after Definition~\ref{def:relenv-functor} is
  satisfied since it is subsumed by the previous condition because $k
  = 0$.
\item The third bullet point of Definition~\ref{def:relenv-functor} is
  satisfied because
\[\begin{split}
& \pi_i(\relsem{\Gamma;\Phi \vdash (\mu \phi. \lambda
  \overline\alpha. H)\overline{\tau}}f)\\
&= \pi_i((\mu T_{\rho'})(\overline{\relsem{\Gamma;\Phi \vdash
    \tau}f}) \circ (\mu \sigma_f)_{\overline{\relsem{\Gamma;\Phi
      \vdash \tau}\rho}}) \\ 
&= \pi_i((\mu T_{\rho'})(\overline{\relsem{\Gamma;\Phi \vdash
    \tau}f})) \circ \pi_i((\mu
\sigma_f)_{\overline{\relsem{\Gamma;\Phi \vdash \tau}\rho}}) \\  
&= \pi_i(\mu T_{\rho'})(\overline{\pi_i(\relsem{\Gamma;\Phi
    \vdash \tau}f})) \circ \pi_i(\mu
\sigma_f)_{\overline{\pi_i(\relsem{\Gamma;\Phi \vdash 
      \tau}\rho})} \\ 
&= (\mu T^\set_{\pi_i \rho'})(\overline{\setsem{\Gamma;\Phi \vdash
    \tau}(\pi_i f)}) \circ (\mu \sigma^\set_{\pi_i
  f})_{\overline{\setsem{\Gamma;\Phi \vdash \tau}(\pi_i\rho)}} \\ 
&= \setsem{\Gamma;\Phi \vdash (\mu \phi. \lambda
  \overline\alpha. H)\overline{\tau}}(\pi_i f). 
\end{split}\]
The fourth equality is by~\ref{eq:mu} and the induction hypothesis.
\end{itemize}
As before, that $\sem{\Gamma; \Phi \vdash (\mu \phi. \lambda
  \overline{\alpha}. H)\overline{\tau}}$ is $\omega$-concontinuous
follows from the facts that $\set$ and $\rel$ are locally finitely
presentable, and that colimits in $\relenv$ are computed
componentwise, together with Corollary~12 of~\cite{jp19}.
\end{itemize}
\end{proof}
\end{comment}

We can also prove by induction that our interpretations of types
interact well with demotion of functorial variables and
substitution. Indeed, we have 
\begin{lemma}\label{thm:demotion-objects}
Let $\rho, \rho' : \setenv$ be such that $\rho \phi = \rho \psi =
\rho' \phi = \rho' \psi$, and let $f : \rho \to \rho'$ be a morphism
of set environments such that $f \phi = f \psi = \id_{\rho \phi}$.  If
\, $\Gamma; \Phi, \phi^k \vdash F : \F$, $\Gamma;\Phi,\ol{\alpha}
\vdash G$ $\Gamma;\Phi,\alpha_1...\alpha_k \vdash H$, and
$\Gamma;\Phi \vdash \tau$, then
\begin{gather}
\label{thm:demotion-objects}
\setsem{\Gamma; \Phi, \phi \vdash F} \rho = \setsem{\Gamma, \psi; \Phi
  \vdash F[\phi :== \psi] } \rho\\
\label{thm:demotion-morphisms}
\setsem{\Gamma; \Phi, \phi \vdash F} f = \setsem{\Gamma, \psi; \Phi
  \vdash F[\phi :== \psi]} f\\
\label{eq:subs-var}
\setsem{\Gamma;\Phi \vdash G[\ol{\alpha := \tau}]}\rho =
\setsem{\Gamma;\Phi,\ol{\alpha} \vdash G}\rho[\ol{\alpha :=
\setsem{\Gamma;\Phi \vdash \tau}\rho}]\\
\label{eq:subs-var-morph}
\setsem{\Gamma;\Phi \vdash G[\ol{\alpha := \tau}]}f =
\setsem{\Gamma;\Phi,\ol{\alpha} \vdash G}f[\ol{\alpha :=
\setsem{\Gamma;\Phi \vdash \tau}f}]\\
\label{eq:subs-const}
\setsem{\Gamma; \Phi \vdash F[\phi := H]}\rho
= \setsem{\Gamma; \Phi, \phi \vdash F}\rho
[\phi := \lambda \ol{A}.\, \setsem{\Gamma;\Phi,\overline{\alpha}\vdash
    H}\rho[\overline{\alpha := A}]] \\ 
\label{eq:subs-const-morph}
\setsem{\Gamma; \Phi \vdash F[\phi := H]}f
= \setsem{\Gamma; \Phi, \phi \vdash F}f
[\phi := \lambda \ol{A}.\,\setsem{\Gamma;\Phi,\overline{\alpha}\vdash
    H}f[\overline{\alpha := \id_{\ol{A}}}]] 
\end{gather}
Analogous identities hold for relational interpretations.
\end{lemma}


\begin{comment}
\begin{lemma}\label{lem:substitution}
Let $\rho$ and $\rho'$ be set environments and let $f : \rho \to \rho'$ be a
morphism of set environments. 
\begin{itemize}
\item If $\Gamma;\Phi,\ol{\alpha} \vdash F$ and $\Gamma;\Phi \vdash \tau$,
  then  
\begin{gather}\label{eq:subs-var}
\setsem{\Gamma;\Phi \vdash F[\ol{\alpha := \tau}]}\rho =
\setsem{\Gamma;\Phi,\ol{\alpha} \vdash F}\rho[\ol{\alpha :=
\setsem{\Gamma;\Phi \vdash \tau}\rho}]\\
\intertext{and}\label{eq:subs-var-morph}
\setsem{\Gamma;\Phi \vdash F[\ol{\alpha := \tau}]}f =
\setsem{\Gamma;\Phi,\ol{\alpha} \vdash F}f[\ol{\alpha :=
\setsem{\Gamma;\Phi \vdash \tau}f}]
\end{gather}
\item If $\Gamma;\Phi,\phi^k\vdash F$ and
  $\Gamma;\Phi,\alpha_1...\alpha_k \vdash H$, then 
\begin{gather}\label{eq:subs-const}
\setsem{\Gamma; \Phi \vdash F[\phi := H]}\rho
= \setsem{\Gamma; \Phi, \phi \vdash F}\rho
[\phi := \lambda \ol{A}.\, \setsem{\Gamma;\Phi,\overline{\alpha}\vdash
    H}\rho[\overline{\alpha := A}]] \\ 
\intertext{and}\label{eq:subs-const-morph}
\setsem{\Gamma; \Phi \vdash F[\phi := H]}f
= \setsem{\Gamma; \Phi, \phi \vdash F}f
[\phi := \lambda \ol{A}.\,\setsem{\Gamma;\Phi,\overline{\alpha}\vdash
    H}f[\overline{\alpha := \id_{\ol{A}}}]] 
\end{gather}
\end{itemize}
Analogous identities hold for relation environments and morphisms
between them.
\end{lemma}

\end{comment}






\begin{comment}

\begin{thm}\label{thm:demotion-objects}
Let \, $\Gamma; \Phi, \phi \vdash \tau : \F$. If $\rho, \rho' : \setenv$
are such that $\rho \phi = \rho \psi = \rho' \phi = \rho' \psi$, and
if $f : \rho \to \rho'$ is a morphism of set environments such that $f
\phi = f \psi = \id_{\rho \phi}$, then
\[\setsem{\Gamma; \Phi, \phi \vdash \tau} \rho = \setsem{\Gamma, \psi;
  \Phi \vdash \tau[\phi :== \psi] } \rho\]
and
\[\setsem{\Gamma; \Phi, \phi \vdash \tau} f = \setsem{\Gamma, \psi;
  \Phi \vdash \tau[\phi :== \psi]} f\]

\vspace*{0.1in}

\noindent
Analogously, if $\rho, \rho' : \relenv$ are such that $\rho \phi =
\rho \psi = \rho' \phi = \rho' \psi$, and if $f : \rho \to \rho'$ is a
morphism of relation environments such that $f \phi = f \psi =
\id_{\rho \phi}$, then
\[\relsem{\Gamma; \Phi, \phi \vdash \tau} \rho = \relsem{\Gamma, \psi;
  \Phi \vdash \tau[\phi :== \psi] } \rho\]
and
\[\relsem{\Gamma; \Phi, \phi \vdash \tau} f = \relsem{\Gamma, \psi;
  \Phi \vdash \tau[\phi :== \psi]} f\]
\end{thm}

\end{comment}

  
\begin{comment}
---
The next two theorems are proven by simultaneous induction. We are
actually only interested in using Theorem~\ref{thm:demotion-objects},
but in order to prove the $\mu$ case for this theorem, we need
Theorem~\ref{thm:demotion-morph} to show that two functors have equal
actions on morphisms.

\begin{thm}\label{thm:demotion-objects}
Let $\Gamma; \Phi, \phi \vdash \tau : \F$. If $\rho, \rho' : \setenv$
are such that $\rho \phi = \rho \psi = \rho' \phi = \rho' \psi$, and
if $f : \rho \to \rho'$ is a morphism of set environments such that $f
\phi = f \psi = \id_{\rho \phi}$, then
\[\setsem{\Gamma; \Phi, \phi \vdash \tau} \rho = \setsem{\Gamma, \psi;
  \Phi \vdash \tau[\phi :== \psi] } \rho\]
and
\[\setsem{\Gamma; \Phi, \phi \vdash \tau} f = \setsem{\Gamma, \psi;
  \Phi \vdash \tau[\phi :== \psi]} f\]

\vspace*{0.1in}

\noindent
Analogously, if $\rho, \rho' : \relenv$ are such that $\rho \phi =
\rho \psi = \rho' \phi = \rho' \psi$, and if $f : \rho \to \rho'$ is a
morphism of relation environments such that $f \phi = f \psi =
\id_{\rho \phi}$, then
\[\relsem{\Gamma; \Phi, \phi \vdash \tau} \rho = \relsem{\Gamma, \psi;
  \Phi \vdash \tau[\phi :== \psi] } \rho\]
and
\[\relsem{\Gamma; \Phi, \phi \vdash \tau} f = \relsem{\Gamma, \psi;
  \Phi \vdash \tau[\phi :== \psi]} f\]
\end{thm}
\begin{proof}
We prove the result for set interpretations by induction on the
structure of $\tau$.  The case for relational interpretations proceeds
analogously. Since the only interesting cases are the application case
and the $\mu$-case, we elide the others.
\begin{itemize}
\item If $\Gamma; \Phi, \phi \vdash \phi \ol\tau : \F$, then the
  induction hypothesis gives that  
\[\setsem{\Gamma; \Phi, \phi \vdash \tau } \rho = \setsem{\Gamma,
  \psi; \Phi \vdash \tau[\phi :== \psi] } \rho\] 
and
\[\setsem{\Gamma; \Phi, \phi \vdash \tau} f = \setsem{\Gamma, \psi;
  \Phi \vdash \tau[\phi :== \psi]} f\]
for each $\tau$. Then
\begin{align*}
 & \setsem{\Gamma; \Phi, \phi \vdash \phi \ol\tau } \rho \\
= \; & (\rho \phi) \ol{\setsem{\Gamma; \Phi, \phi \vdash \tau } \rho} \\
= \; & (\rho \phi) \ol{\setsem{\Gamma, \psi; \Phi \vdash \tau[\phi :==
      \psi] } \rho} \\ 
= \; & (\rho \psi) \ol{\setsem{\Gamma, \psi; \Phi \vdash \tau[\phi :==
      \psi] } \rho} \\ 
= \; & \setsem{\Gamma, \psi; \Phi \vdash \psi \ol{\tau[\phi :== \psi]}
} \rho \\ 
= \; & \setsem{\Gamma, \psi; \Phi \vdash (\phi \ol\tau)[\phi :== \psi]
} \rho 
\end{align*}
Here, the first and fifth equalities are by
Definition~\ref{def:set-sem}, and the fourth equality is by equality
of the functors $\rho \phi$ and $\rho \psi$. We also have that
\begin{align*}
& \setsem{\Gamma; \Phi, \phi \vdash \phi \ol\tau} f \\
= \; & (f \phi)_{\ol{\setsem{\Gamma; \Phi, \phi \vdash \tau} \rho'}} 
    \circ (\rho \phi) \ol{\setsem{\Gamma; \Phi, \phi \vdash \tau} f} \\
= \; & (\id_{\rho \phi})_{\ol{\setsem{\Gamma; \Phi, \phi \vdash \tau} \rho'}} 
    \circ (\rho \phi) \ol{\setsem{\Gamma; \Phi, \phi \vdash \tau} f} \\
= \; & (\rho \phi) \ol{\setsem{\Gamma; \Phi, \phi \vdash \tau} f} \\
= \; & (\rho \psi) \ol{\setsem{\Gamma, \psi; \Phi \vdash \tau[\phi :==
      \psi]} f} \\ 
= \; & (\id_{\rho \psi})_{\ol{\setsem{\Gamma, \psi; \Phi \vdash
      \tau[\phi :== \psi]} \rho'}} \circ (\rho \psi)
\ol{\setsem{\Gamma, \psi; \Phi \vdash \tau[\phi :== \psi]} f} \\ 
= \; & (f \psi)_{\ol{\setsem{\Gamma, \psi; \Phi \vdash \tau[\phi :==
        \psi]} \rho'}} \circ (\rho \psi) \ol{\setsem{\Gamma, \psi;
    \Phi \vdash \tau[\phi :== \psi]} f} \\ 
= \; & \setsem{\Gamma, \psi; \Phi \vdash \psi \ol{\tau[\phi :== \psi]}} f \\
= \; & \setsem{\Gamma, \psi; \Phi \vdash (\phi \ol\tau) [\phi :== \psi]} f
\end{align*}
\item If $\Gamma; \Phi, \phi \vdash (\mu \phi'. \lambda
  \ol\alpha. H)\ol\tau : \F$, then the induction hypothesis gives that
\[\setsem{\Gamma; \Phi, \phi', \ol{\alpha}, \phi \vdash H } \rho =
\setsem{\Gamma, \psi; \Phi, \phi', \ol{\alpha} \vdash H[\phi :== \psi]
} \rho\]
and 
\[\setsem{\Gamma; \Phi, \phi \vdash \tau } \rho = \setsem{\Gamma,
  \psi; \Phi \vdash \tau[\phi :== \psi] } \rho\]
as well as that
\[\setsem{\Gamma; \Phi, \phi', \ol{\alpha}, \phi \vdash H } f =
\setsem{\Gamma, \psi; \Phi, \phi', \ol{\alpha} \vdash H[\phi :== \psi]
} f\]
and 
\[\setsem{\Gamma; \Phi, \phi \vdash \tau } f = \setsem{\Gamma, \psi;
  \Phi \vdash \tau[\phi :== \psi] } f\]
for each $\tau$. Then
\begin{align*}
& \setsem{\Gamma; \Phi, \phi \vdash (\mu \phi'. \lambda
    \ol\alpha. H)\ol\tau } \rho \\ 
= \; & (\mu (\lambda F. \lambda \ol{A}. \setsem{\Gamma; \Phi, \phi',
  \ol{\alpha}, \phi \vdash H} \rho[\phi' := F][\ol{\alpha := A}]))
\ol{\setsem{ \Gamma; \Phi, \phi \vdash \tau} \rho} \\ 
= \; & (\mu (\lambda F. \lambda \ol{A}. \setsem{\Gamma, \psi; \Phi,
  \phi', \ol{\alpha} \vdash H[\phi :== \psi]} \rho[\phi' :=
  F][\ol{\alpha := A}])) \ol{\setsem{ \Gamma; \Phi, \phi \vdash \tau}
  \rho} \\ 
= \; &  (\mu (\lambda F. \lambda \ol{A}. \setsem{\Gamma, \psi; \Phi,
  \phi', \ol{\alpha} \vdash H[\phi :== \psi]} \rho[\phi' :=
  F][\ol{\alpha := A}])\ol{\setsem{ \Gamma, \psi; \Phi \vdash
    \tau[\phi :== \psi]} \rho} \\ 
= \; & \setsem{\Gamma, \psi; \Phi \vdash (\mu \phi'. \lambda
  \ol\alpha. H[\phi :== \psi]) \ol{\tau[\phi :== \psi]} } \rho \\ 
= \; & \setsem{\Gamma, \psi; \Phi \vdash ((\mu \phi'. \lambda
  \ol\alpha. H) \ol\tau)[\phi :== \psi] } \rho   
\end{align*} 
The first and fifth equalities are by Definition~\ref{def:set-sem}.
The second equality follows from the following equality:
\begin{align*}
&\lambda F. \lambda \ol{A}. \setsem{\Gamma; \Phi, \phi', \ol{\alpha},
    \phi \vdash H} \rho[\phi' := F][\ol{\alpha := A}] \\ 
& = \lambda F. \lambda \ol{A}. \setsem{\Gamma, \psi; \Phi, \phi',
    \ol{\alpha} \vdash H[\phi :== \psi]} \rho[\phi' := F][\ol{\alpha
      := A}] 
\end{align*}
These two maps have the same actions on objects and morphisms by the
induction hypothesis on $H$, and the fact that the extended
environment $\rho[\phi' := F][\ol{\alpha := A}]$ satisfies the
required hypothesis. They are thus equal as functors and so have the
same fixed point. We also have that 
\begin{align*}
& \setsem{\Gamma; \Phi, \phi \vdash (\mu \phi'. \lambda
    \ol\alpha. H)\ol\tau } f \\ 
= \; & (\mu (\lambda F. \lambda \ol{A}. \setsem{\Gamma; \Phi, \phi',
  \ol{\alpha}, \phi \vdash H} f [\phi' := \id_F][\ol{\alpha :=
    \id_A}]))_{\ol{\setsem{\Gamma; \Phi, \phi \vdash \tau} \rho'}} \\ 
\; & \circ (\mu (\lambda F. \lambda \ol{A}. \setsem{\Gamma; \Phi,
  \phi', \ol{\alpha}, \phi \vdash H} \rho[\phi' := F][\ol{\alpha :=
    A}])) \ol{\setsem{ \Gamma; \Phi, \phi \vdash \tau} f} \\ 
= \; & (\mu (\lambda F. \lambda \ol{A}. \setsem{\Gamma; \Phi, \phi',
  \ol{\alpha}, \phi \vdash H} f [\phi' := \id_F][\ol{\alpha :=
    \id_A}]))_{\ol{\setsem{\Gamma, \psi; \Phi \vdash \tau[\phi :==
        \psi]} \rho'}} \\ 
\; & \circ (\mu (\lambda F. \lambda \ol{A}. \setsem{\Gamma; \Phi,
  \phi', \ol{\alpha}, \phi \vdash H} \rho[\phi' := F][\ol{\alpha :=
    A}])) \ol{\setsem{ \Gamma, \psi; \Phi \vdash \tau[\phi :== \psi]}
  f} \\ 
= \; & (\mu (\lambda F. \lambda \ol{A}. \setsem{\Gamma, \psi; \Phi,
  \phi', \ol{\alpha} \vdash H[\phi :== \psi]} f [\phi' :=
  \id_F][\ol{\alpha := \id_A}]))_{\ol{\setsem{\Gamma, \psi; \Phi
      \vdash \tau[\phi :== \psi]} \rho'}} \\ 
\; & \circ (\mu (\lambda F. \lambda \ol{A}. \setsem{\Gamma; \Phi,
  \phi', \ol{\alpha}, \phi \vdash H} \rho[\phi' := F][\ol{\alpha :=
    A}])) \ol{\setsem{ \Gamma, \psi; \Phi \vdash \tau[\phi :== \psi]}
  f} \\ 
= \; & (\mu (\lambda F. \lambda \ol{A}. \setsem{\Gamma, \psi; \Phi,
  \phi', \ol{\alpha} \vdash H[\phi :== \psi]} f [\phi' :=
  \id_F][\ol{\alpha := \id_A}]))_{\ol{\setsem{\Gamma, \psi; \Phi
      \vdash \tau[\phi :== \psi]} \rho'}} \\ 
\; & \circ (\mu (\lambda F. \lambda \ol{A}. \setsem{\Gamma, \psi;
  \Phi, \phi', \ol{\alpha} \vdash H[\phi :== \psi]} \rho[\phi' :=
  F][\ol{\alpha := A}])) \ol{\setsem{ \Gamma, \psi; \Phi \vdash
    \tau[\phi :== \psi]} f} \\ 
= \; & \setsem{\Gamma, \psi; \Phi \vdash (\mu \phi'. \lambda
  \ol\alpha. H[\phi :== \psi]) \ol{\tau[\phi :== \psi]} } f \\ 
= \; & \setsem{\Gamma, \psi; \Phi \vdash ((\mu \phi'. \lambda
  \ol\alpha. H) \ol\tau)[\phi :== \psi] } f 
\end{align*} 
The first and fifth equalities are by Definition~\ref{def:set-sem}.
The third equality is by the equality of the arguments to the first
$\mu$ operator:
\begin{align*}
& \lambda F. \lambda \ol{A}. \setsem{\Gamma; \Phi, \phi', \ol{\alpha},
    \phi \vdash H} f [\phi' := \id_F][\ol{\alpha := \id_A}] \\ 
& = \lambda F. \lambda \ol{A}. \setsem{\Gamma, \psi; \Phi, \phi',
    \ol{\alpha} \vdash H[\phi :== \psi]} f [\phi' := \id_F][\ol{\alpha
      := \id_A}] 
\end{align*}
By the induction hypothesis on $H$ and the fact that the morphism
$f[\phi' := \id_F][\ol{\alpha := \id_A}] : \rho[\phi' := F][\ol{\alpha
    := A}] \to \rho'[\phi' := F][\ol{\alpha := A}]$ still satisfies
the required hypotheses.  The fourth equality is by the equality of
the arguments to the second $\mu$ operator:
\begin{align*}
& \lambda F. \lambda \ol{A}. \setsem{\Gamma; \Phi, \phi', \ol{\alpha},
    \phi \vdash H} \rho[\phi' := F][\ol{\alpha := A}] \\ 
& = \lambda F. \lambda \ol{A}. \setsem{\Gamma, \psi; \Phi, \phi',
    \ol{\alpha} \vdash H[\phi :== \psi]} \rho[\phi' := F][\ol{\alpha
      := A}] 
\end{align*}
By the same reasoning as above, these two maps are equal as functors,
and thus have the same fixed point.
\end{itemize}
\end{proof}

---


The following lemma ensures that substitution interacts well with type
interpretations.  It is a consequence of
Definitions~\ref{def:second-order-subst},~\ref{def:set-interp},
and~\ref{def:rel-interp}.

\begin{lemma}\label{lem:substitution}
Let $\rho$ and $\rho'$ be set environments and let $f : \rho \to \rho'$ be a
morphism of set environments. 
\begin{itemize}
\item If $\Gamma;\Phi,\ol{\alpha} \vdash F$ and $\Gamma;\Phi \vdash \tau$,
  then  
\begin{gather}\label{eq:subs-var}
\setsem{\Gamma;\Phi \vdash F[\ol{\alpha := \tau}]}\rho =
\setsem{\Gamma;\Phi,\ol{\alpha} \vdash F}\rho[\ol{\alpha :=
\setsem{\Gamma;\Phi \vdash \tau}\rho}]\\
\intertext{and}\label{eq:subs-var-morph}
\setsem{\Gamma;\Phi \vdash F[\ol{\alpha := \tau}]}f =
\setsem{\Gamma;\Phi,\ol{\alpha} \vdash F}f[\ol{\alpha :=
\setsem{\Gamma;\Phi \vdash \tau}f}]
\end{gather}
\item If $\Gamma;\Phi,\phi^k\vdash F$ and
  $\Gamma;\Phi,\alpha_1...\alpha_k \vdash H$, then 
\begin{gather}\label{eq:subs-const}
\setsem{\Gamma; \Phi \vdash F[\phi := H]}\rho
= \setsem{\Gamma; \Phi, \phi \vdash F}\rho
[\phi := \lambda \ol{A}.\, \setsem{\Gamma;\Phi,\overline{\alpha}\vdash
    H}\rho[\overline{\alpha := A}]] \\ 
\intertext{and}\label{eq:subs-const-morph}
\setsem{\Gamma; \Phi \vdash F[\phi := H]}f
= \setsem{\Gamma; \Phi, \phi \vdash F}f
[\phi := \lambda \ol{A}.\,\setsem{\Gamma;\Phi,\overline{\alpha}\vdash
    H}f[\overline{\alpha := \id_{\ol{A}}}]] 
\end{gather}
\end{itemize}
Analogous identities hold for relation environments and morphisms
between them.
\end{lemma}
\end{comment}

\begin{comment}
\begin{proof}
The proofs for the set and relational interpretations are completely
analogous, so we just prove the former. Likewise, we only prove
Equations~\ref{eq:subs-var} and~\ref{eq:subs-const}, since the proofs
for Equations~\ref{eq:subs-var-morph} and~\ref{eq:subs-const-morph}
are again analogous. Finally, we prove Equation~\ref{eq:subs-var} for
substitution for just a single type variable since the proof for
multiple simultaneous substitutions proceeds similarly.

Although Equation~\ref{eq:subs-var} is a special case of
Equation~\ref{eq:subs-const}, it is convenient to prove
Equation~\ref{eq:subs-var} first, and then use it to prove
Equation~\ref{eq:subs-const}. We prove Equation~\ref{eq:subs-var} by
induction on the structure of $F$ as follows:
\begin{itemize}
\item If $\Gamma; \emptyset \vdash F : \T$, or if $F$ is $\onet$ or
  $\zerot$, then $F$ does not contain any functorial variables to
  replace, so there is nothing to prove.
\item If $F$ is $F_1 \times F_2$ or $F_1 + F_2$, then the substitution
  distributes over the product or coproduct as appropriate, so the
  result follows immediately from the induction hypothesis.
\item If $F = \beta$ with $\beta \neq \alpha$, then there is nothing to
  prove.
\item If $F = \alpha$, then
\[\setsem{\Gamma;\Phi\vdash\alpha[\alpha := \tau]}\rho \, = \,
\setsem{\Gamma;\Phi\vdash\tau}\rho \, = \,
\setsem{\Gamma;\Phi,\alpha\vdash\alpha}\rho[\alpha :=
  \setsem{\Gamma;\Phi\vdash\tau}\rho]\]
\item If $F = \phi \overline{\sigma}$ with $\phi \neq \alpha$, then
\[\begin{array}{ll}
 &\setsem{\Gamma; \Phi \vdash (\phi \overline{\sigma})[\alpha := \tau]}\rho\\
=&\setsem{\Gamma; \Phi \vdash \phi (\overline{\sigma[\alpha := \tau]})}\rho \\
=&(\rho\phi)\overline{\setsem{\Gamma;\Phi\vdash\sigma[\alpha := \tau]}\rho} \\
=&(\rho\phi)\overline{\setsem{\Gamma;\Phi,\alpha\vdash\sigma}
      \rho [\alpha := \setsem{\Gamma;\Phi\vdash\tau}] } \\
=&\setsem{\Gamma; \Phi, \alpha \vdash \phi \overline{\sigma}}
      \rho [\alpha := \setsem{\Gamma;\Phi\vdash\tau}]
\end{array}\]
Here, the third equality is by the induction hypothesis.
\item If $F = (\mu \phi. \lambda \overline{\beta}. G)
  \overline{\sigma}$, then
\[\begin{array}{ll}
  & \setsem{\Gamma; \Phi \vdash ((\mu \phi. \lambda
  \overline{\beta}. G) \overline{\sigma}) [\alpha := \tau]}\rho \\
=&\setsem{\Gamma; \Phi \vdash (\mu \phi. \lambda
  \overline{\beta}. G[\alpha := \tau]) (\overline{\sigma [\alpha :=
      \tau]})}\rho \\ 
=&\mu(\setsem{\Gamma; \Phi, \phi, \overline{\beta} \vdash G[\alpha := \tau]}
      \rho[\phi := \text{--}][\overline{\beta := \text{--}}])
      (\overline{\setsem{\Gamma;\Phi \vdash \sigma [\alpha := \tau]}\rho}) \\
=&\mu(\setsem{\Gamma; \Phi, \phi, \overline{\beta}, \alpha \vdash G}
\rho [\alpha := \setsem{\Gamma;\Phi \vdash \tau}\rho] [\phi := \text{--}]
[\overline{\beta := \text{--}}]) \\
 &\hspace{0.5in} (\overline{\setsem{\Gamma;\Phi,\alpha \vdash \sigma}
   \rho [\alpha := \setsem{\Gamma;\Phi\vdash \tau}\rho]}) \\
=&\setsem{\Gamma;\Phi,\alpha\vdash(\mu\phi.\lambda\overline{\beta}.G)
  \overline{\sigma}}\rho  [\alpha := \setsem{\Gamma;\Phi \vdash
   \tau}\rho] 
\end{array}  \]
Here, the third equality is by the induction hypothesis and weakening.
\end{itemize}

We now prove Equation~\ref{eq:subs-const}, again by induction on the
structure of $F$.
\begin{itemize}
\item If $\Gamma; \emptyset \vdash F : \T$, or if $F$ is $\onet$ or
  $\zerot$, then $F$ does not contain any functorial variables to
  replace, so there is nothing to prove.
\item If $F$ is $F_1 \times F_2$ or $F_1 + F_2$, then the substitution
  distributes over the product or coproduct as appropriate, so the
  result follows immediately from the induction hypothesis.
\item If $F = \phi \overline{\tau}$, then
\[\begin{array}{ll}
 &\setsem{\Gamma; \Phi \vdash (\phi \overline{\tau})[\phi := H]}\rho \\
=&\setsem{\Gamma; \Phi \vdash H[\overline{\alpha:= \tau[\phi := H]}]}\rho \\
=&\setsem{\Gamma; \Phi \vdash H}\rho [\overline{\alpha:=
    \setsem{\Gamma;\Phi\vdash\tau[\phi := H]}\rho}] \\ 
=&\setsem{\Gamma; \Phi \vdash H}\rho [\overline{\alpha:=
    \setsem{\Gamma;\Phi,\phi\vdash\tau} \rho [\phi :=
      \setsem{\Gamma;\Phi,\overline{\alpha}\vdash
        H}\rho[\overline{\alpha := \text{--}}]]}] \\
=&\setsem{\Gamma; \Phi, \phi \vdash \phi \overline{\tau}}\rho [\phi :=
  \setsem{\Gamma;\Phi,\overline{\alpha}\vdash H}\rho[\overline{\alpha
      := \text{--}}]] 
  \end{array}\]
Here, the first equality is by
Definition~\ref{def:second-order-subst}, the second is by
Equation~\ref{eq:subs-var}, the third is by the induction hypothesis,
and the fourth is by Definition~\ref{def:set-sem}.
\item If $F = \psi \overline{\tau}$ with $\psi \neq \phi$, then the
  proof is similar to that for the previous case, but simpler, because
  $\phi$ only needs to be substituted in the arguments $\ol{\tau}$ of
  $\psi$.
\item If $F = (\mu \psi. \lambda \overline{\beta}. G)
  \overline{\tau}$, then
\[\begin{array}{ll}
 &\setsem{\Gamma; \Phi \vdash ((\mu \psi. \lambda \overline{\beta}. G)
 \overline{\tau})[\phi := H]}\rho \\ 
=&\setsem{\Gamma; \Phi \vdash (\mu \psi. \lambda
  \overline{\beta}. G[\phi := H]) (\overline{\tau[\phi := H]})}\rho \\
=&\mu(\setsem{\Gamma; \Phi, \psi, \overline{\beta} \vdash G[\phi := H]}
    \rho[\psi := \text{--}][\overline{\beta := \text{--}}])
    (\overline{\setsem{\Gamma;\Phi \vdash \tau[\phi := H]}\rho}) \\
=&\mu(\setsem{\Gamma; \Phi, \psi, \overline{\beta}, \phi \vdash G}
\rho [\phi := \setsem{\Gamma;\Phi,\overline{\alpha}\vdash
    H}\rho[\overline{\alpha := \text{--}}]][\psi := \text{--}]
     [\overline{\beta := \text{--}}]) \\
&\hspace{0.5in} (\overline{\setsem{\Gamma;\Phi,\phi \vdash \tau} \rho
       [\phi := \setsem{\Gamma;\Phi,\overline{\alpha}\vdash H}
         \rho[\overline{\alpha := \text{--}}]]}) \\
=&\setsem{\Gamma;\Phi,\phi\vdash(\mu\psi.\lambda\overline{\beta}.G)
  \overline{\tau}}\rho [\phi :=
  \setsem{\Gamma;\Phi,\overline{\alpha}\vdash H}\rho[\overline{\alpha
      := \text{--}}]] 
\end{array}\]
Here, the first equality is by
Definition~\ref{def:second-order-subst}, the second and fourth are by
Definition~\ref{def:set-sem}, and the third is by the induction
hypothesis and weakening.
\end{itemize}
\end{proof}
\end{comment}

\section{The Identity Extension Lemma}\label{sec:iel}

{\color{blue} Compare against Sec 3.3, 3.4 of Bob. R\&R refl graphs
  are our elts of $RT_0$. Bob defines equality relations starting from
  refl graphs by induction on kind structure. We do it as a special
  case of graphs.}

The standard definition of the {\em graph} of a morphism $f : A \to B$
in $\set$ is the relation $\graph{f} : \rel(A,B)$ defined by $(x,y)
\in \graph{f}$ iff $fx = y$. This definition naturally generalizes to
associate to each natural transformation between $k$-ary functors on
$\set$ a $k$-ary relation transformer as follows:

\begin{dfn}\label{dfn:graph-nat-transf}
If $F, G: \Set^k \to \Set$
%are $k$-ary functors
and $\alpha : F \to G$ is a natural transformation, then the functor
$\graph{\alpha}^*: \rel^k \to \rel$ is defined as follows. Given $R_1
: \rel(A_1, B_1),...,R_k : \rel(A_k,B_k)$, let $\iota_{R_i} : R_i
\hookrightarrow A_i \times B_i$, for $i = 1,...,k$, be the inclusion
of $R_i$ as a subset of $A_i \times B_i$,
%. By the universal property of the product, there exists a
let $h_{\overline{A \times B}}$ be the unique morphism making the diagram
\[\begin{tikzcd}[row sep = large]
        F\overline{A}
        &F(\overline{A \times B})
        \ar[l, "{F\overline{\pi_1}}"']
        \ar[r, "{F\overline{\pi_2}}"]
        \ar[d, dashed, "{h_{\overline{A \times B}}}"]
        &F\overline{B}
        \ar[r, "{\alpha_{\ol{B}}}"]
        &G\overline{B} \\
        &F\overline{A} \times G\overline{B}
        \ar[ul, "{\pi_1}"] \ar[urr, "{\pi_2}"']
\end{tikzcd}\]
commute, and let $h_{\overline{R}} : F\overline{R} \to F\overline{A}
\times G\overline{B}$ be $h_{\overline{A \times B}} \circ
F\overline{\iota_R}$. Further, let $\alpha^\wedge\overline{R}$ be the
subobject through which $h_{\overline{R}}$ is factorized by the
mono-epi factorization system in $\set$, as shown in the
following diagram:
\[\begin{tikzcd}
        F\overline{R}
        \ar[rr, "{h_{\overline{R}}}"]
        \ar[dr, twoheadrightarrow, "{q_{\alpha^\wedge\overline{R}}}"']
        &&F\overline{A} \times G\overline{B} \\
        &\alpha^\wedge\overline{R}
        \ar[ur, hookrightarrow, "{\iota_{\alpha^\wedge\overline{R}}}"']
\end{tikzcd}\]
Then $\alpha^\wedge\overline{R} : \rel(F\overline{A}, G\overline{B})$
by construction, so the action of $\langle \alpha \rangle^*$ on
objects can be given by $\langle \alpha \rangle^* \overline{(A,B,R)} =
(F\overline{A}, G\overline{B}, \iota_{\alpha^\wedge
  \overline{R}}\alpha^\wedge\overline{R})$. Its action on morphisms is
given by $\graph{\alpha}^*\overline{(\beta, \beta')} =
(F\overline\beta, G\overline\beta')$.
\end{dfn}

The data in Definition~\ref{dfn:graph-nat-transf} yield a relation
transformer $\graph{\alpha} = (F, G, \graph{\alpha}^*)$, called the
{\em graph relation transformer for $\alpha$}.

\begin{lemma}\label{lem:graph-reln-functors}
If $F,G : [\set^k,\set]$ and $\alpha : F \to G$ is a natural
transformation, then $\graph{\alpha}$ is in $RT_k$.
\end{lemma}
\begin{proof}
Clearly, $\graph{\alpha}^*$ is $\omega$-cocontinuous, so
$\graph{\alpha}^* : [\rel^k,\rel]$. Now, suppose $\overline{R :
  \rel(A, B)}$, $\overline{S : \rel(C, D)}$, and $\overline{(\beta,
  \beta') : R \to S}$. We want to show that there exists a morphism
$\epsilon : \alpha^\wedge\overline{R} \to \alpha^\wedge\overline{S}$
such that
    \[
    \begin{tikzcd}
        \alpha^\wedge\overline{R}
        \ar[r, hookrightarrow, "{\iota_{\alpha^\wedge\overline{R}}}"]
        \ar[d, "{\epsilon}"']
        & F\overline{A} \times G\overline{B}
        \ar[d, "{F\overline{\beta} \times G\overline{\beta'}}"] \\
        \alpha^\wedge\overline{S}
        \ar[r, hookrightarrow, "{\iota_{\alpha^\wedge\overline{S}}}"']
        & F\overline{C} \times G\overline{D}
    \end{tikzcd}
    \]
    commutes.
    Since $\ol{(\beta,\beta') : R \to S}$, there exist $\overline{\gamma : R \to S}$
    such that each diagram
    \[
    \begin{tikzcd}
        R_i
        \ar[d, "{\gamma_i}"']
        \ar[r, hookrightarrow, "{\iota_{R_i}}"]
        &A_i \times B_i
        \ar[d, "{\beta_i \times \beta'_i}"] \\
        S_i
        \ar[r, hookrightarrow, "{\iota_{S_i}}"]
        &C_i \times D_i
    \end{tikzcd}
    \]
    commutes. Moreover, since both
    $h_{\overline{C \times D}} \circ F(\overline{\beta \times \beta'})$
    and
    $(F\overline{\beta} \times G\overline{\beta'}) \circ h_{\overline{A \times B}}$
    make
      \[
      \begin{tikzcd}[row sep = large]
          F\overline{C}
          &F\overline{C} \times F\overline{D}
          \ar[l, "{\pi_1}"'] \ar[r, "{\pi_2}"]
          &F\overline{D}
          \ar[r, "{\alpha_{\ol{D}}}"]
          &G\overline{D}\\
          &F(\overline{A \times B})
          \ar[u, dashed, "{\exists !}"]
          \ar[ul, "{F\ol{\pi_1} \circ F(\overline{\beta \times \beta'})}"]
          \ar[urr, "{\alpha_{\ol{D}} \circ F\ol{\pi_2} \circ F(\overline{\beta \times \beta'})}"']
      \end{tikzcd}
      \]
      commute, they must be equal. We therefore get that the
      right-hand square below commutes, and thus that the entire
      following diagram does as well:
      \[
      \begin{tikzcd}
          F\overline{R}
          \ar[d, "{F\overline{\gamma}}"']
          \ar[r, hookrightarrow, "{F\overline{\iota_R}}"]
          \ar[rr, bend left, "{h_{\overline{R}}}"]
          &F(\overline{A \times B})
          \ar[d, "{F(\overline{\beta \times \beta'})}"]
          \ar[r, "{h_{\overline{A \times B}}}"]
          &F\overline{A} \times G\overline{B}
          \ar[d, "{F\overline{\beta} \times F\overline{\beta'}}"] \\
          F\overline{S}
          \ar[r, hookrightarrow, "{F\overline{\iota_S}}"']
          \ar[rr, bend right, "{h_{\overline{S}}}"']
          &F(\overline{C \times D})
          \ar[r, "{h_{\overline{C \times D}}}"']
          &F\overline{C} \times G\overline{D}
      \end{tikzcd}
      \]
      Finally, by the left-lifting property of $q_{F^\wedge\overline{R}}$
      with respect to $\iota_{F^\wedge\overline{S}}$ given by the epi-mono
      factorization system, there exists an $\epsilon$ such that the
      following diagram commutes:
      \[
      \begin{tikzcd}
          F\overline{R}
          \ar[d, "{F\overline{\gamma}}"']
          \ar[r, twoheadrightarrow, "{q_{\alpha^\wedge\overline{R}}}"]
          &\alpha^\wedge\overline{R}
          \ar[d, dashed, "{\epsilon}"]
          \ar[r, hookrightarrow, "{\iota_{\alpha^\wedge\overline{R}}}"]
          &F\overline{A} \times G\overline{B}
          \ar[d, "{F\overline{\beta} \times G\overline{\beta'}}"] \\
          F\overline{S}
          \ar[r, twoheadrightarrow, "{q_{\alpha^\wedge\overline{S}}}"']
          &\alpha^\wedge\overline{S}
          \ar[r, hookrightarrow, "{\iota_{\alpha^\wedge\overline{S}}}"']
          &F\overline{C} \times G\overline{D}
      \end{tikzcd}
      \]
\end{proof}

It is not hard to see that if $f : A \to B$ is a morphism in $\set$
then the standard definition of its graph $\langle f \rangle$ in
$\rel$ coincides with the its definition as the graph relation
transformer for $f$ as a natural transformation between the $0$-ary
functors $A$ and $B$. Graph relation transformers are thus a
reasonable generalization of graphs in $\rel$.
\begin{comment}

If $f : A \to B$ is a function with graph relation $\graph{f} = (A, B,
\graph{f}^*)$, then $\langle \id_{A}, f \rangle : A \to A \times B$
and $\langle \id_{A}, f \rangle\, A = \graph{f}^*$.  Moreover, if
$\iota_{\graph{f}} : \graph{f}^* \hookrightarrow A \times B$ is the
inclusion of $\graph{f}^*$ into $A \times B$ then there is an
isomorphism of subobjects
\[\begin{tikzcd}
A \ar[rr, "{\cong}"] \ar[dr, "{\langle \id_{A}, f \rangle}"']
&&{\graph{f}^*} \ar[dl, "{\iota_{\graph{f}}}"]\\
&A \times B
\end{tikzcd}\]

We also note that if $f : A \to B$ is a function seen as a natural
transformation between 0-ary functors, then $\graph{f}$ is (the 0-ary
relation transformer associated with) the graph relation of $f$.
Indeed, we need to apply Definition~\ref{dfn:graph-nat-transf} with $k
= 0$, i.e., to the degenerate relation $\ast : \rel(\ast, \ast)$.  As
degenerate $0$-ary functors, $A$ and $B$ are constant functors, i.e.,
$A\, \ast = A$ and $B\, \ast = B$.  By the universal property of the
product, there exists a unique $h$ making the diagram
\[ \begin{tikzcd}[row sep = large]
        A
        &A
        \ar[l, equal]
        \ar[r, equal]
        \ar[d, dashed, "{h}"]
        &A
        \ar[r, "{f}"]
        &B \\
        &A \times B
        \ar[ul, "{\pi_1}"] \ar[urr, "{\pi_2}"']
\end{tikzcd}\]
commute. Since $\iota_\ast : \ast \to \ast$ is the identity on $\ast$,
and $A\, \id_{\ast} = \id_{A}$, we have $h_{\ast} = h$.  Moreover,
$h_{\overline{A \times B}} = \langle \id_{A}, f \rangle$ is a
monomorphism in $\set$ because $\id_{A}$ is.  Then,
$\iota_{f^\wedge\ast} = \langle \id_{A}, f \rangle$ and $f^\wedge\ast
= A$, from which we deduce that $\iota_{f^\wedge\ast} f^\wedge\ast =
\langle \id_{A}, f \rangle\, A = \graph{f}^*$. This ensures that the
graph of $f$ as a 0-ary natural transformation coincides with the
graph of $f$ as a morphism in $\set$, and so that
Definition~\ref{dfn:graph-nat-transf} is a reasonable generalization
of Definition~\ref{def:graph}.

Just as the equality relation $\Eq_B$ on a set $B$ coincides with
$\graph{\id_B}$, the graph of the identity on the set, so we can
define the equality relation transformer to be the graph of the
identity natural transformation. This gives

\begin{dfn}
Let $F : [\set^k, \set]$.  The equality relation transformer on $F$ is
defined to be $\Eq_F = \graph{\id_{F}}$. This entails that $Eq_F = (F,
F, \Eq_F^*)$ with $\Eq_F^* = \graph{\id_{F}}^*$.
\end{dfn}
\end{comment}

The action of a graph relation transformer on a graph relation is
easily computed:

\begin{lemma}\label{lem:eq-reln-equalities}
If $\alpha : F \to G$ is a morphism in $[\Set^k, \Set]$
and $f_1: A_1 \to B_1, ..., f_k : A_k \to B_k$,
then $\graph{\alpha}^* \graph{\overline{f}}
= \langle G \ol{f} \circ \alpha_{\ol{A}} \rangle
= \langle \alpha_{\ol{B}} \circ F \ol{f} \rangle$.
\end{lemma}
\begin{proof}
Since $h_{\overline{A \times B}}$ is the unique morphism making the
bottom triangle of this diagram commute
\[\begin{tikzcd}[row sep = large]
        &F\overline{A}
        \ar[d, "{F \overline{\langle \id_A, f \rangle}}" description]
        \ar[dl, equal]
        \ar[dr, "{F\ol{f}}"]\\
        F\overline{A}
        &F(\overline{A \times B})
        \ar[l, "{F\overline{\pi_1}}"']
        \ar[r, "{F\overline{\pi_2}}"]
        \ar[d, "{h_{\overline{A \times B}}}"]
        &F\overline{B}
        \ar[r, "{\alpha_{\ol{B}}}"]
        &G\overline{B}\\
        &F\overline{A} \times G\overline{B}
        \ar[ul, "{\pi_1}"] \ar[urr, "{\pi_2}"']
\end{tikzcd}\]
and since $h_{\graph{\overline{f}}} = h_{\overline{A \times B}} \circ
F \,\ol{\iota_{\graph{f}}} = h_{\overline{A \times B}} \circ F
\overline{\langle \id_A, f \rangle}$,
%(the last equality being by the observation after
%Remark~\ref{rmk:graph-fn}),
the universal property of the product
      \[
      \begin{tikzcd}[row sep = large]
          F\overline{A}
          &F\overline{A} \times G\overline{B}
          \ar[l, "{\pi_1}"'] \ar[r, "{\pi_2}"]
          &G\overline{B}\\
          &F\overline{A}
          \ar[u, dashed, "{\exists !}"]
          \ar[ul, equal]
          \ar[r, "{F\ol{f}}"']
          &F{\ol{B}}
          \ar[u, "{\alpha_{\ol{B}}}"']
      \end{tikzcd}
      \]
gives $h_{\graph{\overline{f}}} = \langle \id_{F \ol{A}},
\alpha_{\ol{B}} \circ F\ol{f} \rangle : F \ol{A} \to F \ol{A} \times G
\ol{B}$.  Moreover, $\langle \id_{F \ol{A}}, \alpha_{\ol{B}} \circ
F\ol{f} \rangle$ is a monomorphism in $\set$ because $\id_{F \ol{A}}$
is, so its epi-mono factorization gives $\iota_{\alpha^\wedge
  \graph{\overline{f}}} = \langle \id_{F \ol{A}}, \alpha_{\ol{B}}
\circ F\ol{f} \rangle$, and thus $\alpha^\wedge \graph{\overline{f}} =
%the domain of $\iota_{\alpha^\wedge \graph{\overline{f}}}$, is equal
%to
F\overline{A}$.  Then $\iota_{\alpha^\wedge
  \graph{\overline{f}}} \alpha^\wedge \graph{\overline{f}} = \langle
\id_{F \ol{A}}, \alpha_{\ol{B}} \circ F\ol{f} \rangle (F \ol{A}) =
\graph{ \alpha_{\ol{B}} \circ F\ol{f} }^*$,
%(where the last equality is by Remark~\ref{rmk:graph-fn}).
%We therefore conclude that
so that $\graph{ \alpha }^* \graph{ \overline{f} } = (F\overline{A},
G\overline{B}, \iota_{\alpha^\wedge \graph{\overline{f}}}\,
\alpha^\wedge \graph{\overline{f}}) = (F\overline{A}, G\overline{B},
\graph{ \alpha_{\ol{B}} \circ F\ol{f} }^*) = \graph{ \alpha_{\ol{B}}
  \circ F\ol{f} }$.  Finally, $\alpha_{\ol{B}} \circ F\ol{f} = G\ol{f}
\circ \alpha_{\ol{A}}$ by naturality of $\alpha$.
\end{proof}

The {\em equality relation transformer} on $F : [\set^k,\set]$ is
defined to be $\Eq_F = \graph{\id_{F}}$. Specifically, $Eq_F = (F, F,
\Eq_F^*)$ with $\Eq_F^* = \graph{\id_{F}}^*$, and
Lemma~\ref{lem:eq-reln-equalities} ensures that $\Eq^*_F \,\ol{\Eq_A}
= \Eq_{F\ol{A}}$ for all $\ol{A : \set}$.
\begin{comment}
\begin{proof}
We have that
\[
\Eq^*_F \ol{\Eq_A}
= \graph{\id_F}^* \graph{\id_{\ol{A}}}
= \graph{F \id_{\ol{A}} \circ (\id_F)_{\ol{A}}}  %by your Lemma 5
= \graph{\id_{F\ol{A}} \circ \id_{F\ol{A}}}
%= \graph{\id_F \ol{A} \circ \id_{\ol{A}}}
= \graph{\id_{F\ol{A}}}
= \Eq_{F\ol{A}}
\]
The second identity here is by Lemma~\ref{lem:eq-reln-equalities}.
\end{proof}
\end{comment}
Graph relation transformers in general, and equality relation
transformers in particular, extend to relation environments in the
obvious ways.
%\begin{definition}
Indeed, if $\rho, \rho' : \setenv$ and $f : \rho \to \rho'$, then the
{\em graph relation environment} $\graph{f}$ is defined pointwise by
$\graph{f} \phi = \graph{f \phi}$ for every $\phi$. This entails that
$\pi_1 \graph{f} = \rho$ and $\pi_2 \graph{f} = \rho'$. In particular,
the {\em equality relation environment} $\Eq_\rho$ is defined to be
$\graph{\id_{\rho}}$. This entails that $\Eq_\rho \phi = \Eq_{\rho
  \phi}$ for every $\phi$.
%\end{definition}
With these definitions in hand, we can state and prove both an
Identity Extension Lemma and a Graph Lemma for our calculus.
\begin{thm}[IEL]\label{thm:iel}
  If $\rho : \setenv$ and $\Gamma; \Phi \vdash \tau : \F$ then
  $\relsem{\Gamma; \Phi \vdash \tau} \Eq_\rho = \Eq_{\setsem{\Gamma;
      \Phi \vdash \tau}\rho}$.
\end{thm}
The proof is by induction on the structure of $\tau$. Only the
application and fixpoint cases are non-routine. Both use
Lemma~\ref{lem:eq-reln-equalities}. The latter also uses the
observation that, for every $n \in \nat$, the following intermediate
results can be proved by simultaneous induction {\color{blue} with
  Theorem~\ref{thm:iel}}:
%\begin{equation}\label{eq:iel-fix-point-intermediate1}
$T^n_{\Eq_{\rho}} K_0\, \ol{\Eq_{\setsem{\Gamma; \Phi \vdash
      \tau}\rho}} = (\Eq_{(T^\set_\rho)^n K_0})^*
\ol{\Eq_{\setsem{\Gamma; \Phi \vdash \tau}\rho}}$\;
%\end{equation}
and\;
%\begin{equation}\label{eq:iel-fix-point-intermediate2}
%\begin{split}
$ \relsem{\Gamma; \Phi, \phi, \ol{\alpha} \vdash H} \Eq_{\rho} [\phi
  := T^{n}_{\Eq_{\rho}} K_0] \overline{[\alpha := \Eq_{\setsem{\Gamma;
        \Phi \vdash \tau}\rho}]} =$\\
$\relsem{\Gamma; \Phi, \phi,
  \ol{\alpha} \vdash H} \Eq_{\rho} [\phi := \Eq_{(T^\set_\rho)^n K_0}]
\overline{[\alpha := \Eq_{\setsem{\Gamma; \Phi \vdash \tau}\rho}]}$.
%\end{split}
%\end{equation}

\begin{comment}
\begin{proof}
By induction on the structure of $\tau$.
\begin{itemize}
\item $\relsem{\Gamma; \emptyset \vdash v}\Eq_{\rho} = \Eq_{\rho} v =
  \Eq_{\rho v} = \Eq_{\setsem{\Gamma; \emptyset \vdash v}\rho}$ where
  $v \in \Gamma$.
\item By definition, $\relsem{\Gamma; \emptyset \vdash
  \Nat^{\overline\alpha} \,F\,G} \Eq_{\rho}$ is the relation on
  $\setsem{\Gamma; \emptyset \vdash \Nat^{\overline\alpha} \,F\,G}
  \rho$ relating $t$ and $t'$ if, for all ${R_1 :
    \rel(A_1,B_1)},...,{R_k : \rel(A_k,B_k)}$, $(t_{\overline{A}},
  t'_{\overline{B}})$ is a morphism from $\relsem{\Gamma;
    \overline\alpha \vdash F} \Eq_{\rho}\overline{[\alpha := R]}$ to
  $\relsem{\Gamma ; \overline\alpha \vdash G}
  \Eq_{\rho}\overline{[\alpha := R]}$ in $\rel$.  To prove that this
  is equal to $\Eq_{\setsem{\Gamma; \emptyset \vdash
      \Nat^{\overline\alpha} \,F\,G} \rho}$ we need to show that
  $(t_{\overline{A}}, t'_{\overline{B}})$ is a morphism from
  $\relsem{\Gamma; \overline\alpha \vdash F}
  \Eq_{\rho}\overline{[\alpha := R]}$ to $\relsem{\Gamma ;
    \overline\alpha \vdash G} \Eq_{\rho}\overline{[\alpha := R]}$ in
  $\rel$ for all ${R_1 : \rel(A_1,B_1)},...,{R_k : \rel(A_k,B_k)}$ if
  and only if $t = t'$ and $(t_{\overline{A}}, t_{\overline{B}})$ is a
  morphism from $\relsem{\Gamma; \overline\alpha \vdash F}
  \Eq_{\rho}\overline{[\alpha := R]}$ to $\relsem{\Gamma ; \overline
    \alpha \vdash G} \Eq_{\rho}\overline{[\alpha := R]}$ in $\rel$ for
  all ${R_1 : \rel(A_1,B_1)}, ...,$ ${R_k : \rel(A_k,B_k)}$. The only
  interesting part of this equivalence is to show that if
  $(t_{\overline{A}}, t'_{\overline{B}})$ is a morphism from
  $\relsem{\Gamma; \overline\alpha \vdash F}
  \Eq_{\rho}\overline{[\alpha := R]}$ to $\relsem{\Gamma ;
    \overline\alpha \vdash G} \Eq_{\rho}\overline{[\alpha := R]}$ in
  $\rel$ for all ${R_1 : \rel(A_1,B_1),}$ $...,{R_k : \rel(A_k,B_k)}$,
  then $t = t'$.  By hypothesis, $(t_{\overline{A}},
  t'_{\overline{A}})$ is a morphism from $\relsem{\Gamma;
    \overline\alpha \vdash F} \Eq_{\rho}\overline{[\alpha :=
      \Eq_{A}]}$ to $\relsem{\Gamma ; \overline\alpha \vdash G}
  \Eq_{\rho}\overline{[\alpha := \Eq_{A}]}$ in $\rel$ for all
  $A_1\,...\,A_k : \set$. By the induction hypothesis, it is therefore
  a morphism from $\Eq_{\setsem{\Gamma; \overline\alpha \vdash F}
    \rho\overline{[\alpha := A]}}$ to $\Eq_{\setsem{\Gamma ;
      \overline\alpha \vdash G} \rho\overline{[\alpha := A]}}$ in
  $\rel$. This means that, for every $x : \Eq_{\setsem{\Gamma;
      \overline\alpha \vdash F} \rho\overline{[\alpha := A]}}$,
  $t_{\overline{A}}x = t'_{\overline{A}}x$.  Then, by extensionality,
  $t = t'$.
\item $\relsem{\Gamma; \Phi \vdash \zerot} \Eq_{\rho} = 0_\rel =
  \Eq_{0_\set} = \Eq_{\setsem{\Gamma; \Phi \vdash \zerot}\rho}$
\item $\relsem{\Gamma; \Phi \vdash \onet} \Eq_{\rho} = 1_\rel =
  \Eq_{1_\set} = \Eq_{\setsem{\Gamma; \Phi \vdash \onet}\rho}$
\item The application case is proved by the following sequence of
  equalities, where the second equality is by the induction hypothesis
  and the definition of the relation environment $\Eq_\rho$, the third
  is by the definition of application of relation transformers, and
  the fourth is by Lemma~\ref{lem:eq-reln-equalities}:
\[
\begin{split}
\relsem{\Gamma; \Phi \vdash \phi\ol{\tau}}\Eq_{\rho} &=
(\Eq_{\rho}\phi)\ol{\relsem{\Gamma; \Phi \vdash \tau}
\Eq_{\rho}}\\
&= \Eq_{\rho \phi}\, \ol{\Eq_{\setsem{\Gamma; \Phi \vdash \tau}
  \rho}}\\
&= (\Eq_{\rho \phi})^* \,\ol{\Eq_{\setsem{\Gamma; \Phi \vdash \tau}
  \rho}}\\
&= \Eq_{(\rho \phi) \,\ol{\setsem{\Gamma; \Phi \vdash \tau} \rho}}\\
&= \Eq_{\setsem{\Gamma; \Phi \vdash \phi\ol{\tau}}\rho}
\end{split}
\]
\item The fixed point case is proven by the sequence of equalities
\[
\begin{split}
\relsem{\Gamma; \Phi \vdash (\mu \phi.\lambda
  \ol{\alpha}. H)\ol{\tau}}\Eq_{\rho} 
&=(\mu {T_{\Eq_{\rho}}}) \,\ol{\relsem{\Gamma; \Phi \vdash \tau}\Eq_{\rho}}\\ 
&= \colim{n \in \nat}{T^n_{\Eq_{\rho}} K_0}\, \ol{\relsem{\Gamma; \Phi
  \vdash \tau}\Eq_{\rho}}\\
&= \colim{n \in \nat}{ T^n_{\Eq_{\rho}} K_0 \,\ol{\Eq_{\setsem{\Gamma;
    \Phi \vdash \tau}\rho}}}\\
&= \colim{n \in \nat}{(\Eq_{(T^\set_\rho)^n K_0})^*
  \ol{\Eq_{\setsem{\Gamma; \Phi \vdash \tau}\rho}}}\\
&= \colim{n \in \nat}{\Eq_{(T^\set_\rho)^n K_0 \,\ol{\setsem{\Gamma;
        \Phi \vdash \tau}\rho}}}\\ 
&= \Eq_{\colim{n \in \nat}{ (T^\set_{\rho})^n K_0\,
    \ol{\setsem{\Gamma; \Phi \vdash \tau}\rho}}}\\
&= \Eq_{\setsem{\Gamma; \Phi \vdash (\mu \phi.\lambda
      \ol{\alpha}. H)\ol{\tau}}\rho}
\end{split}
\]
Here, the third equality is by induction hypothesis, the fifth is by
Lemma~\ref{lem:eq-reln-equalities} and the fourth equality is because,
for every $n \in \nat$, the following two statements can be proved by
simultaneous induction:
\begin{equation}\label{eq:iel-fix-point-intermediate1}
T^n_{\Eq_{\rho}} K_0\, \ol{\Eq_{\setsem{\Gamma; \Phi \vdash
      \tau}\rho}} = (\Eq_{(T^\set_\rho)^n K_0})^*
\ol{\Eq_{\setsem{\Gamma; \Phi \vdash \tau}\rho}}
\end{equation}
and
\begin{equation}\label{eq:iel-fix-point-intermediate2}
\begin{split}
  \relsem{\Gamma; \Phi, \phi, \ol{\alpha} \vdash H}
\Eq_{\rho} [\phi := 
 & T^{n}_{\Eq_{\rho}} K_0] \overline{[\alpha :=
    \Eq_{\setsem{\Gamma; \Phi \vdash \tau}\rho}]} \\
=\;\; & \relsem{\Gamma; \Phi, \phi, \ol{\alpha} \vdash H} \Eq_{\rho} [\phi
  := \Eq_{(T^\set_\rho)^n K_0}] \overline{[\alpha :=
    \Eq_{\setsem{\Gamma; \Phi \vdash \tau}\rho}]}
\end{split}
\end{equation}
We prove~\eqref{eq:iel-fix-point-intermediate1}.  The case $n=0$ is
trivial, because $T^0_{\Eq_{\rho}} K_0 = K_0$ and
$(T^\set_\rho)^0 K_0 = K_0$; the inductive step is
proved by the following sequence of equalities:
\[
\begin{split}
T^{n+1}_{\Eq_{\rho}} K_0\, \overline{\Eq_{\setsem{\Gamma; \Phi \vdash \tau}\rho}}
&= T^\rel_{\Eq_{\rho}} (T^{n}_{\Eq_{\rho}} K_0)
\overline{\Eq_{\setsem{\Gamma; \Phi \vdash \tau}\rho}} \\ 
&= \relsem{\Gamma; \Phi, \phi, \ol{\alpha} \vdash H} \Eq_{\rho} [\phi
  := T^{n}_{\Eq_{\rho}} K_0] \overline{[\alpha :=
    \Eq_{\setsem{\Gamma; \Phi \vdash \tau}\rho}]} \\ 
&= \relsem{\Gamma; \Phi, \phi, \ol{\alpha} \vdash H} \Eq_{\rho} [\phi
  := \Eq_{(T^\set_\rho)^n K_0}] \overline{[\alpha :=
    \Eq_{\setsem{\Gamma; \Phi \vdash \tau}\rho}]} \\ 
&= \relsem{\Gamma; \Phi, \phi, \ol{\alpha} \vdash H} \Eq_{\rho [\phi
    := (T^\set_\rho)^n K_0] \overline{[\alpha :=
      \setsem{\Gamma; \Phi \vdash \tau}\rho]}} \\ 
&= \Eq_{\setsem{\Gamma; \Phi, \phi, \ol{\alpha} \vdash H} \rho [\phi
    := (T^\set_\rho)^n K_0] \overline{[\alpha :=
      \setsem{\Gamma; \Phi \vdash \tau}\rho]}} \\ 
&= \Eq_{(T^\set_\rho)^{n+1} K_0 \overline{\setsem{\Gamma; \Phi
      \vdash \tau}\rho}} \\ 
&= (\Eq_{(T^\set_\rho)^{n+1} K_0})^*\, \overline{\Eq_{\setsem{\Gamma;
      \Phi \vdash \tau}\rho}} 
\end{split}
\]
Here, the third equality is by~\eqref{eq:iel-fix-point-intermediate2},
the fifth by the induction hypothesis on $H$, and the last by
Lemma~\ref{lem:eq-reln-equalities}.  We prove the induction step
of~\eqref{eq:iel-fix-point-intermediate2} by structural induction on
$H$: the only interesting case, though, is when $\phi$ is applied,
i.e., for $H = \phi \ol{\sigma}$, which is proved by the sequence of
equalities:
\[
\begin{split}
& \relsem{\Gamma; \Phi, \phi, \ol{\alpha} \vdash \phi
    \ol{\sigma}} \Eq_{\rho} [\phi := T^{n}_{\Eq_{\rho}} K_0]
  \overline{[\alpha := \Eq_{\setsem{\Gamma; \Phi \vdash \tau}\rho}]}
  \\
&= T^{n}_{\Eq_{\rho}} K_0\, \overline{\relsem{\Gamma; \Phi,
      \phi, \ol{\alpha} \vdash \sigma} \Eq_{\rho} [\phi :=
      T^{n}_{\Eq_{\rho}} K_0] \overline{[\alpha :=
        \Eq_{\setsem{\Gamma; \Phi \vdash \tau} \rho}]}} \\ 
&= T^{n}_{\Eq_{\rho}} K_0\, \overline{\relsem{\Gamma; \Phi,
      \phi, \ol{\alpha} \vdash \sigma} \Eq_{\rho} [\phi :=
      \Eq_{(T^\set_\rho)^{n} K_0}] \overline{[\alpha :=
        \Eq_{\setsem{\Gamma; \Phi \vdash \tau} \rho}]}} \\ 
&= T^{n}_{\Eq_{\rho}} K_0\, \overline{\relsem{\Gamma; \Phi,
      \phi, \ol{\alpha} \vdash \sigma} \Eq_{\rho [\phi := (T^\set_\rho)^{n}
        K_0] \overline{[\alpha := \setsem{\Gamma; \Phi \vdash
            \tau} \rho]}}} \\ 
&= T^{n}_{\Eq_{\rho}} K_0\, \overline{\Eq_{\setsem{\Gamma;
        \Phi, \phi, \ol{\alpha} \vdash \sigma} \rho [\phi :=
        (T^\set_\rho)^{n} K_0] \overline{[\alpha :=
          \setsem{\Gamma; \Phi \vdash \tau} \rho]}}} \\ 
&= (\Eq_{(T^\set_\rho)^{n} K_0})^* \,\overline{\Eq_{\setsem{\Gamma;
        \Phi, \phi, \ol{\alpha} \vdash \sigma} \rho [\phi :=
        (T^\set_\rho)^{n} K_0] \overline{[\alpha :=
          \setsem{\Gamma; \Phi \vdash \tau} \rho]}}} \\ 
&= (\Eq_{(T^\set_{\rho})^{n} K_0})^* \overline{\relsem{\Gamma;
      \Phi, \phi, \ol{\alpha} \vdash \sigma} \Eq_{\rho} [\phi :=
      \Eq_{(T^\set_{\rho})^{n} K_0}] \overline{[\alpha :=
        \Eq_{\setsem{\Gamma; \Phi \vdash \tau}\rho}]}} \\ 
&= \relsem{\Gamma; \Phi, \phi, \ol{\alpha} \vdash \phi \ol{\sigma}}
  \Eq_{\rho} [\phi := \Eq_{(T^\set_{\rho})^{n} K_0}]
  \overline{[\alpha := \Eq_{\setsem{\Gamma; \Phi \vdash \tau}\rho}]} 
\end{split}
\]
Here, the second equality is by the induction hypothesis
for~\eqref{eq:iel-fix-point-intermediate2} on the $\sigma$s, the
fourth is by the induction hypothesis for Theorem~\ref{thm:iel} on the
$\sigma$s, and the fifth is by the induction hypothesis on $n$
for~\eqref{eq:iel-fix-point-intermediate1}.
\item $\relsem{\Gamma; \Phi \vdash \sigma + \tau} \Eq_{\rho} =
  \relsem{\Gamma; \Phi \vdash \sigma} \Eq_{\rho} + \relsem{\Gamma;
    \Phi \vdash \tau} \Eq_{\rho} = \Eq_{\setsem{\Gamma; \Phi \vdash
      \sigma}\rho} + \Eq_{\setsem{\Gamma; \Phi \vdash \tau}\rho} =
  \Eq_{\setsem{\Gamma; \Phi \vdash \sigma}\rho + \setsem{\Gamma; \Phi
      \vdash \tau}\rho} = \Eq_{\setsem{\Gamma; \Phi \vdash \sigma +
      \tau}\rho}$
\item $\relsem{\Gamma; \Phi \vdash \sigma \times \tau} \Eq_{\rho} =
  \relsem{\Gamma; \Phi \vdash \sigma}\Eq_{\rho} \times \relsem{\Gamma;
    \Phi \vdash \tau}\Eq_{\rho} = \Eq_{\setsem{\Gamma; \Phi \vdash
      \sigma}\rho} \times \Eq_{\setsem{\Gamma; \Phi \vdash \tau}\rho}
  = \Eq_{\setsem{\Gamma; \Phi \vdash \sigma}\rho \times
    \setsem{\Gamma; \Phi \vdash \tau}\rho} = \Eq_{\setsem{\Gamma;
      \Phi \vdash \sigma \times \tau}\rho}$
\end{itemize}
\end{proof}
\end{comment}

%It follows from Theorem~\ref{thm:iel} that $\relsem{\Gamma \vdash
%  \sigma \to \tau} \Eq_{\rho} = \Eq_{\setsem{\Gamma \vdash \sigma \to
%    \tau}\rho}$, as expected.  Moreover,

%\vspace*{0.05in}
%
%The Graph Lemma appropriate to our setting is now easily obtained:
\begin{lemma}[Graph Lemma]\label{lem:graph}
If $\rho, \rho' : \setenv$, $f : \rho \to \rho'$, and $\Gamma; \Phi
\vdash F : \F$, then $\graph{\setsem{\Gamma; \Phi \vdash F} f} =
\relsem{\Gamma; \Phi \vdash F}\graph{f}$.
\end{lemma}
\begin{proof}
Applying Lemma~\ref{lem:rel-transf-morph} to the morphisms $(f,
\id_{\rho'}) : \graph{f} \to \Eq_{\rho'}$ and $(\id_{\rho}, f) :
\Eq_{\rho} \to \graph{f}$ of relation environments gives that
$(\setsem{\Gamma; \Phi \vdash F}f, \setsem{\Gamma; \Phi \vdash
  F}\id_{\rho'}) = \relsem{\Gamma; \Phi \vdash F} (f, \id_{\rho'}) :
\relsem{\Gamma; \Phi \vdash F}\graph{f} \to \relsem{\Gamma; \Phi
  \vdash F}\Eq_{\rho'}$ and $(\setsem{\Gamma; \Phi \vdash
  F}\id_{\rho}, \setsem{\Gamma; \Phi \vdash F}f) = \relsem{\Gamma;
  \Phi \vdash F} (\id_{\rho}, f) : \relsem{\Gamma; \Phi \vdash
  F}\Eq_{\rho} \to \relsem{\Gamma; \Phi \vdash F}\graph{f}$.
Expanding the first equation gives that if $(x,y) \in \relsem{\Gamma;
  \Phi \vdash F}\graph{f}$ then $(\setsem{\Gamma; \Phi \vdash F} f\,
x, \setsem{\Gamma; \Phi \vdash F}\id_{\rho'}\, y) \in \relsem{\Gamma;
  \Phi \vdash F}\Eq_{\rho'}$.  Then $\setsem{\Gamma; \Phi \vdash
  F}\id_{\rho'}\, y = \id_{\setsem{\Gamma; \Phi \vdash F}\rho'}\, y =
y$ and $\relsem{\Gamma; \Phi \vdash F}\Eq_{\rho'} =
\Eq_{\setsem{\Gamma; \Phi \vdash F}\rho'}$, so if $(x,y) \in
\relsem{\Gamma; \Phi \vdash F}\graph{f}$ then $(\setsem{\Gamma; \Phi
  \vdash F} f\, x, y) \in \Eq_{\setsem{\Gamma; \Phi \vdash F}\rho'}$,
i.e., $\setsem{\Gamma; \Phi \vdash F} f\, x = y$, i.e., $(x, y) \in
\graph{\setsem{\Gamma; \Phi \vdash F} f}$.  So, we have that
$\relsem{\Gamma; \Phi \vdash F}\graph{f} \subseteq
\graph{\setsem{\Gamma; \Phi \vdash F}f}$.  Expanding the second
equation gives that if $x \in \setsem{\Gamma; \Phi \vdash F}\rho$ then
$(\setsem{\Gamma; \Phi \vdash F}\id_{\rho}\, x, \setsem{\Gamma; \Phi
  \vdash F} f\, x) \in \relsem{\Gamma; \Phi \vdash F}\graph{f}$.  Then
$\setsem{\Gamma; \Phi \vdash F}\id_{\rho}\, x = \id_{\setsem{\Gamma;
    \Phi \vdash F}\rho} x = x$, so for any $x \in \setsem{\Gamma; \Phi
  \vdash F}\rho$ we have that $(x, \setsem{\Gamma; \Phi \vdash F}f\,
x) \in \relsem{\Gamma; \Phi \vdash F}\graph{f}$.  Moreover, $x \in
\setsem{\Gamma; \Phi \vdash F}\rho$ if and only if $(x,
\setsem{\Gamma; \Phi \vdash F} f\, x) \in \graph{\setsem{\Gamma; \Phi
    \vdash F}f}$ and, if $x \in \setsem{\Gamma; \Phi \vdash F}\rho$
then $(x, \setsem{\Gamma; \Phi \vdash F} f\, x) \in \relsem{\Gamma;
  \Phi \vdash F} \graph{f}$, so if $(x, \setsem{\Gamma; \Phi \vdash F}
f\, x) \in \graph{\setsem{\Gamma; \Phi \vdash F}f}$ then $(x,
\setsem{\Gamma; \Phi \vdash F} f\, x) \in \relsem{\Gamma; \Phi \vdash
  F} \graph{f}$, i.e., $\graph{\setsem{\Gamma; \Phi \vdash F}f}
\subseteq \relsem{\Gamma; \Phi \vdash F} \graph{f}$.


\end{proof}



\begin{comment}

\begin{proof}
First observe that $(f, \id_{\rho'}) : \graph{f} \to \Eq_{\rho'}$ and
$(\id_{\rho}, f) : \Eq_{\rho} \to \graph{f}$ are morphisms of relation
environments.  Applying Lemma~\ref{lem:rel-transf-morph} to each of
these observations gives that
\begin{equation}\label{eq:graph-one}
(\setsem{\Gamma; \Phi \vdash F}f, \setsem{\Gamma; \Phi \vdash
    F}\id_{\rho'}) = \relsem{\Gamma; \Phi \vdash F} (f, \id_{\rho'}) :
  \relsem{\Gamma; \Phi \vdash F}\graph{f} \to \relsem{\Gamma; \Phi
    \vdash F}\Eq_{\rho'}
\end{equation}
and
\begin{equation}\label{eq:graph-two}
(\setsem{\Gamma; \Phi \vdash F}\id_{\rho}, \setsem{\Gamma; \Phi \vdash F}f)
= \relsem{\Gamma; \Phi \vdash F} (\id_{\rho}, f)
: \relsem{\Gamma; \Phi \vdash F}\Eq_{\rho} \to \relsem{\Gamma; \Phi \vdash F}\graph{f}
\end{equation}
Expanding Equation~\ref{eq:graph-one} gives that if
$(x,y) \in \relsem{\Gamma; \Phi \vdash F}\graph{f}$
then
\[(\setsem{\Gamma; \Phi \vdash F} f\, x, \setsem{\Gamma; \Phi \vdash
  F}\id_{\rho'}\, y) \in \relsem{\Gamma; \Phi \vdash F}\Eq_{\rho'}\]
Observe that $\setsem{\Gamma; \Phi \vdash F}\id_{\rho'}\, y =
\id_{\setsem{\Gamma; \Phi \vdash F}\rho'}\, y = y$ and
$\relsem{\Gamma; \Phi \vdash F}\Eq_{\rho'} = \Eq_{\setsem{\Gamma; \Phi
    \vdash F}\rho'}$. So, if $(x,y) \in \relsem{\Gamma; \Phi \vdash
  F}\graph{f}$ then $(\setsem{\Gamma; \Phi \vdash F} f\, x, y) \in
\Eq_{\setsem{\Gamma; \Phi \vdash F}\rho'}$, i.e., $\setsem{\Gamma;
  \Phi \vdash F} f\, x = y$, i.e., $(x, y) \in \graph{\setsem{\Gamma;
    \Phi \vdash F} f}$.  So, we have that $\relsem{\Gamma; \Phi \vdash
  F}\graph{f} \subseteq \graph{\setsem{\Gamma; \Phi \vdash F}f}$

Expanding Equation~\ref{eq:graph-two} gives that, for any
$x \in \setsem{\Gamma; \Phi \vdash F}\rho$,
then
\[
(\setsem{\Gamma; \Phi \vdash F}\id_{\rho}\, x, \setsem{\Gamma; \Phi
  \vdash F} f\, x) \in \relsem{\Gamma; \Phi \vdash F}\graph{f} 
\]
Observe that $\setsem{\Gamma; \Phi \vdash F}\id_{\rho}\, x =
\id_{\setsem{\Gamma; \Phi \vdash F}\rho} x = x$ so, for any $x \in
\setsem{\Gamma; \Phi \vdash F}\rho$, we have that $(x, \setsem{\Gamma;
  \Phi \vdash F}f\, x) \in \relsem{\Gamma; \Phi \vdash F}\graph{f}$.
Moreover, $x \in \setsem{\Gamma; \Phi \vdash F}\rho$ if and only if
$(x, \setsem{\Gamma; \Phi \vdash F} f\, x) \in \graph{\setsem{\Gamma;
    \Phi \vdash F}f}$ and, if $x \in \setsem{\Gamma; \Phi \vdash
  F}\rho$ then $(x, \setsem{\Gamma; \Phi \vdash F} f\, x) \in
\relsem{\Gamma; \Phi \vdash F} \graph{f}$, so if $(x, \setsem{\Gamma;
  \Phi \vdash F} f\, x) \in \graph{\setsem{\Gamma; \Phi \vdash F}f}$
then $(x, \setsem{\Gamma; \Phi \vdash F} f\, x) \in \relsem{\Gamma;
  \Phi \vdash F} \graph{f}$, i.e., $\graph{\setsem{\Gamma; \Phi \vdash
    F}f} \subseteq \relsem{\Gamma; \Phi \vdash F} \graph{f}$. We
conclude that $\relsem{\Gamma; \Phi \vdash F}\graph{f} =
\graph{\setsem{\Gamma; \Phi \vdash F}f}$ as desired.
\end{proof}
\end{comment}

\section{Interpreting Terms}\label{sec:term-interp}

{\color{blue} Here, we are using angle bracket notation for both the
  graph relation of a function and for the pairing of functions with
  the same domain. This is justified by the relationship between the
  two notions observed immediately after
  Lemma~\ref{lem:graph-reln-functors}.}

If $\Delta = x_1 : \tau_1,...,x_n : \tau_n$ is a term context for $\Gamma$
and $\Phi$, then the interpretations $\setsem{\Gamma;\Phi \vdash \Delta}$ and
$\relsem{\Gamma;\Phi \vdash \Delta}$ are defined by
\[\begin{array}{lll}
\setsem{\Gamma;\Phi \vdash \Delta} & = & \setsem{\Gamma;\Phi \vdash
  \tau_1} \times ... \times \setsem{\Gamma;\Phi \vdash \tau_n}\\ 
\relsem{\Gamma;\Phi \vdash \Delta} & = & \relsem{\Gamma;\Phi \vdash
  \tau_1} \times ... \times \relsem{\Gamma;\Phi \vdash \tau_n}\\ 
\end{array}\]
Every well-formed term $\Gamma;\Phi~|~\Delta \vdash t : \tau$ then
has, for every $\rho \in \setenv$, set interpretations
$\setsem{\Gamma;\Phi~|~\Delta \vdash t : \tau}\rho$ as natural
transformations from $\setsem{\Gamma; \Phi \vdash \Delta}\rho$ to
$\setsem{\Gamma; \Phi \vdash \tau}\rho$, and, for every $\rho \in
\relenv$, relational interpretations $\relsem{\Gamma;\Phi~|~\Delta
  \vdash t : \tau}\rho$ as natural transformations from
$\relsem{\Gamma; \Phi \vdash \Delta}\rho$ to $\relsem{\Gamma; \Phi
  \vdash \tau}\rho$. These are given in the next two definitions.

\begin{dfn}\label{def:set-interp}
If $\rho$ is a set environment and $\Gamma;\Phi~|~\Delta \vdash t :
\tau$ then $\setsem{\Gamma;\Phi~|~\Delta \vdash t : \tau}\rho$ is
defined as follows:
\[\begin{array}{lll}
\setsem{\Gamma;\emptyset \,|\, \Delta,x :\tau \vdash x : \tau} \rho& = &
\pi_{|\Delta|+1}\\
%\setsem{\Gamma;\emptyset \,|\, \Delta \vdash \lambda x.t : \sigma \to \tau}\rho &
%= & \curry (\setsem{\Gamma;\emptyset \,|\, \Delta, x : \sigma \vdash t :
%  \tau}\rho)\\ 
%\setsem{\Gamma;\emptyset \,|\, \Delta \vdash st: \tau} \rho & = &
%\eval \circ
% \langle \setsem{\Gamma;\emptyset \,|\, \Delta \vdash s: \sigma \to
%  \tau}\rho, \setsem{\Gamma;\emptyset \,|\, \Delta \vdash t: \sigma}\rho
%\rangle\\
\setsem{\Gamma;\emptyset \,|\, \Delta \vdash L_{\overline \alpha} x.t : \Nat^{\overline
    \alpha} \,F \,G}\rho & = &  \curry (\setsem{\Gamma;\overline \alpha
  \,|\, \Delta, x : F \vdash t: G}\rho[\overline{\alpha := \_}])\\
\setsem{\Gamma;\Phi \,|\, \Delta \vdash t_{\overline \tau} s:
  G [\overline{\alpha := \tau}]}\rho & = & \eval \circ \langle
  \lambda d.\,(\setsem{\Gamma;\emptyset \,|\, \Delta \vdash t :
  \Nat^{\overline{\alpha}} \,F \,G}\rho\; d)_{\overline{\setsem{\Gamma;\Phi
      \vdash \tau}\rho}},\\ 
 & & \hspace*{0.5in} \setsem{\Gamma;\Phi \,|\,
    \Delta \vdash s: F [\overline{\alpha := \tau}]}\rho \rangle\\ 
& & \\
%% \color{red} \mbox{Add rules for } \forall \mbox{ if we include it} & & \\
\setsem{\Gamma;\Phi \,|\, \Delta,x :\tau \vdash x : \tau} \rho& = &
\pi_{|\Delta|+1}\\
\setsem{\Gamma;\Phi \,|\, \Delta \vdash \bot_\tau t : \tau} \rho& = &
!^0_{\setsem{\Gamma;\Phi \vdash \tau}\rho} \circ
  \setsem{\Gamma;\Phi~|~\Delta \vdash t : \zerot}\rho, \mbox{ where } \\
 & & \hspace*{0.1in} !^0_{\setsem{\Gamma;\Phi \vdash \tau}\rho}
\mbox{ is the unique morphism from } 0\\
 & & \hspace*{0.1in} \mbox{ to } \setsem{\Gamma;\Phi \vdash \tau}\rho\\
\setsem{\Gamma;\Phi \,|\, \Delta \vdash \top : \onet}\rho & = &
!^{\setsem{\Gamma;\Phi\vdash \Delta}\rho}_1, \mbox{ where }
!^{\setsem{\Gamma;\Phi\vdash \Delta}\rho}_1\\ 
& & \hspace*{0.1in} \mbox{ is the unique morphism from }
\setsem{\Gamma;\Phi\vdash \Delta}\rho \mbox{ to } 1\\ 
\setsem{\Gamma;\Phi \,|\, \Delta \vdash (s,t) : \sigma \times \tau} \rho& = &
\setsem{\Gamma;\Phi \,|\, \Delta \vdash s: \sigma} \rho\times
\setsem{\Gamma;\Phi \,|\, \Delta \vdash t : \tau} \rho\\
\setsem{\Gamma;\Phi \,|\, \Delta \vdash \pi_1 t : \sigma} \rho& = &
\pi_1 \circ \setsem{\Gamma;\Phi \,|\, \Delta \vdash t : \sigma \times \tau}\rho\\
\setsem{\Gamma;\Phi \,|\, \Delta \vdash \pi_2 t : \sigma}\rho & = &
\pi_2 \circ \setsem{\Gamma;\Phi \,|\, \Delta \vdash t : \sigma \times
  \tau} \rho\\
\setsem{\Gamma;\Phi~|~\Delta \vdash \case{t}{x \mapsto l}{y \mapsto r} :
  \gamma}\rho & = & \eval \circ \langle \curry \,[\setsem{\Gamma;\Phi
    \,|\, \Delta, x : \sigma \vdash l : \gamma}\rho,\\
   & & \hspace*{0.79in} \setsem{\Gamma;\Phi \,|\, \Delta, y
    : \tau \vdash r : \gamma}\rho],\\
   & &  \hspace*{0.5in} \setsem{\Gamma;\Phi \,|\, \Delta \vdash t :
  \sigma + \tau} \rho\rangle\\   
\setsem{\Gamma;\Phi \,|\, \Delta \vdash \inl \,s: \sigma + \tau} \rho& = &
\inl \circ \setsem{\Gamma;\Phi \,|\, \Delta \vdash s: \sigma}\rho\\
\setsem{\Gamma;\Phi \,|\, \Delta \vdash \inr \,t: \sigma + \tau}\rho & = & 
\inr \circ \setsem{\Gamma;\Phi \,|\, \Delta \vdash t : \tau}\rho\\
\llbracket \Gamma;\emptyset \,|\, \emptyset \vdash \map^{\ol{F},\ol{G}}_H :
\Nat^\emptyset\;(\ol{\Nat^{\ol{\beta},\ol{\gamma}}\,F\,G})\;&
= & \lambda d\, \ol{\eta}\,\ol{B}.\,
\setsem{\Gamma; \ol{\phi},\ol{\gamma}\vdash H}\id_{\rho[\ol{\gamma :=
      B}]}[\ol{\phi := \lambda \ol{A}.\eta_{\ol{A}\,\ol{B}}}]\\
\hspace*{0.79in} (\Nat^{\ol{\gamma}}\,H[\ol{\phi :=_{\ol{\beta}} F}]\,H[\ol{\phi
      :=_{\ol{\beta}} G}]) \rrbracket^\set \rho & & \\
\llbracket \Gamma;\emptyset \,|\, \emptyset \vdash \tin_H :
Nat^{\ol{\beta},\ol{\gamma}} \, H[\phi := (\mu \phi.\lambda {\overline
    \alpha}.H){\overline \beta}][\ol{\alpha := \beta}] & = &
\lambda d\,\ol{B}\, \ol{C}.\,(\mathit{in}_{T^\set_{\rho[\ol{\gamma := C}]}})_{\ol{B}}\\
\hspace*{0.79in}(\mu \phi.\lambda {\overline \alpha}.H){\overline
  \beta} \rrbracket^\set \rho & & \\
\llbracket \Gamma;\emptyset \,|\, \emptyset \vdash
  \fold^F_H : \Nat^\emptyset\;(\Nat^{\ol{\beta}, \ol{\gamma}}\,H[\phi
    :=_{\ol{\beta}} F][\ol{\alpha := \beta}]\,F) & = &  
\lambda d\,\eta\,\ol{B}\,\ol{C}.\,
(\mathit{fold}_{T^\set_{\rho[\ol{\gamma := C}]}} \, (\lambda
\ol{A}.\,\eta_{\ol{A}\,\ol{C}}))_{\ol{B}}\\ 
\hspace*{0.79in}(\Nat^{{\ol{\beta},\ol{\gamma}} }\,(\mu
  \phi.\lambda \overline \alpha.H)\overline \beta\;F)
\rrbracket^\set \rho & & 
\end{array}\]
\end{dfn}
{\color{blue} Add return type for fold in last clause? Should be
  $\setsem{\Gamma;\ol\beta,\ol\gamma \vdash F}\rho]\ol{\gamma := C}]$.}

This interpretation gives that $\setsem{\Gamma;\emptyset \,|\, \Delta
  \vdash \lambda x.t : \sigma \to \tau}\rho = \curry
(\setsem{\Gamma;\emptyset \,|\, \Delta, x : \sigma \vdash t :
  \tau}\rho)$ and $\setsem{\Gamma;\emptyset \,|\, \Delta \vdash st:
  \tau} \rho = \eval \circ \langle \setsem{\Gamma;\emptyset \,|\,
  \Delta \vdash s: \sigma \to \tau}\rho, \setsem{\Gamma;\emptyset
  \,|\, \Delta \vdash t: \sigma}\rho \rangle$, as expected.

\begin{dfn}\label{def:rel-interp}
If $\rho$ is a relation environment and $\Gamma;\Phi~|~\Delta \vdash t :
\tau$ then $\relsem{\Gamma;\Phi~|~\Delta \vdash t : \tau}\rho$ is
defined as follows:
\[\begin{array}{lll}
\relsem{\Gamma;\emptyset \,|\, \Delta,x :\tau \vdash x : \tau} \rho& = &
\pi_{|\Delta|+1}\\
%\relsem{\Gamma;\emptyset \,|\, \Delta \vdash \lambda x.t : \sigma \to \tau}\rho &
%= & \curry (\relsem{\Gamma;\emptyset \,|\, \Delta, x : \sigma \vdash t :
%  \tau}\rho)\\ 
%\relsem{\Gamma;\emptyset \,|\, \Delta \vdash st: \tau} \rho & = & \eval
%\circ \langle \relsem{\Gamma;\emptyset \,|\, \Delta \vdash s: \sigma \to
%  \tau}\rho, \relsem{\Gamma;\emptyset \,|\, \Delta \vdash t: \sigma}\rho
%\rangle\\
\relsem{\Gamma;\emptyset \,|\, \Delta \vdash L_{\overline \alpha} x.t : \Nat^{\overline
    \alpha} \,F \,G}\rho & = &  \curry (\relsem{\Gamma;\overline \alpha
  \,|\, \Delta, x : F \vdash t: G}\rho[\overline{\alpha := \_}])\\
\relsem{\Gamma;\Phi \,|\, \Delta \vdash t_{\overline{\tau}} s:
  G [\overline{\alpha := \tau}]}\rho & = & \eval \circ \langle
  \lambda e. \,(\relsem{\Gamma;\emptyset \,|\, \Delta \vdash t :
  \Nat^{\overline{\alpha}} \,F \,G}\rho\; e)_{\overline{\relsem{\Gamma;\Phi
      \vdash \tau}\rho}},\\ 
 & & \hspace*{0.5in} \relsem{\Gamma;\Phi \,|\,
    \Delta \vdash s: F [\overline{\alpha := \tau}]}\rho \rangle\\ 
& & \\
%% \color{red} \mbox{Add rules for } \forall \mbox{ if we include it} & & \\
\relsem{\Gamma;\Phi \,|\, \Delta,x :\tau \vdash x : \tau} \rho& = &
\pi_{|\Delta|+1}\\
\relsem{\Gamma;\Phi \,|\, \Delta \vdash \bot_\tau t : \tau} \rho& = &
!^0_{\relsem{\Gamma;\Phi \vdash \tau}\rho} \circ
  \relsem{\Gamma;\Phi~|~\Delta \vdash t : \zerot}\rho, \mbox{ where } \\
 & & \hspace*{0.1in} !^0_{\relsem{\Gamma;\Phi \vdash \tau}\rho}
\mbox{ is the unique morphism from } 0\\
 & & \hspace*{0.1in} \mbox{ to } \relsem{\Gamma;\Phi \vdash \tau}\rho\\
\relsem{\Gamma;\Phi \,|\, \Delta \vdash \top : \onet}\rho & = &
!^{\relsem{\Gamma;\Phi\vdash \Delta}\rho}_1, \mbox{ where }
!^{\relsem{\Gamma;\Phi\vdash \Delta}\rho}_1\\ 
& & \hspace*{0.1in} \mbox{ is the unique morphism from }
\relsem{\Gamma;\Phi\vdash \Delta}\rho \mbox{ to } 1\\ 
\relsem{\Gamma;\Phi \,|\, \Delta \vdash (s,t) : \sigma \times \tau} \rho& = &
\relsem{\Gamma;\Phi \,|\, \Delta \vdash s: \sigma} \rho\times
\relsem{\Gamma;\Phi \,|\, \Delta \vdash t: \tau} \rho\\
\relsem{\Gamma;\Phi \,|\, \Delta \vdash \pi_1 t : \sigma} \rho& = &
\pi_1 \circ \relsem{\Gamma;\Phi \,|\, \Delta \vdash t : \sigma \times \tau}\rho\\
\relsem{\Gamma;\Phi \,|\, \Delta \vdash \pi_2 t : \sigma}\rho & = &
\pi_2 \circ \relsem{\Gamma;\Phi \,|\, \Delta \vdash t : \sigma \times
  \tau} \rho\\
\relsem{\Gamma;\Phi~|~\Delta \vdash \case{t}{x \mapsto l}{y \mapsto r} :
  \gamma}\rho & = & \eval \circ \langle \curry \,[\relsem{\Gamma;\Phi
    \,|\, \Delta, x : \sigma \vdash l : \gamma}\rho,\\
   & & \hspace*{0.79in} \relsem{\Gamma;\Phi \,|\, \Delta, y
    : \tau \vdash r : \gamma}\rho],\\
   & &  \hspace*{0.5in} \relsem{\Gamma;\Phi \,|\, \Delta \vdash t :
  \sigma + \tau} \rho\rangle \\
\relsem{\Gamma;\Phi \,|\, \Delta \vdash \inl \,s: \sigma + \tau} \rho& = &
\inl \circ \relsem{\Gamma;\Phi \,|\, \Delta \vdash s: \sigma}\rho\\
\relsem{\Gamma;\Phi \,|\, \Delta \vdash \inr \,t: \sigma + \tau}\rho & = & 
\inr \circ \relsem{\Gamma;\Phi \,|\, \Delta \vdash t : \tau}\rho\\
\llbracket \Gamma;\emptyset \,|\, \emptyset \vdash \map^{\ol{F},\ol{G}}_H :
\Nat^\emptyset\;(\ol{\Nat^{\ol{\beta},\ol{\gamma}}\,F\,G})\;&
= & \lambda d\, \ol{\eta}\,\ol{R}.\,
\relsem{\Gamma; \ol{\phi},\ol{\gamma}\vdash H}\id_{\rho[\ol{\gamma :=
      R}]}[\ol{\phi := \lambda \ol{S}.\eta_{\ol{S}\,\ol{R}}}]\\
\hspace*{0.79in} (\Nat^{\ol{\gamma}}\,H[\ol{\phi :=_{\ol{\beta}} F}]\,H[\ol{\phi
      :=_{\ol{\beta}} G}]) \rrbracket^\rel \rho & & \\
\llbracket \Gamma;\emptyset \,|\, \emptyset \vdash \tin_H :
Nat^{\ol{\beta},\ol{\gamma}} \, H[\phi := (\mu \phi.\lambda {\overline
    \alpha}.H){\overline \beta}][\ol{\alpha := \beta}] & = &
\lambda d\,\ol{R}\, \ol{S}.\,(\mathit{in}_{T_{\rho[\ol{\gamma := S}]}})_{\ol{R}}\\
\hspace*{0.79in}(\mu \phi.\lambda {\overline \alpha}.H){\overline
  \beta} \rrbracket^\rel \rho & & \\
\llbracket \Gamma;\emptyset \,|\, \emptyset \vdash
  \fold^F_H : \Nat^\emptyset\;(\Nat^{\ol{\beta}, \ol{\gamma}}\,H[\phi
    :=_{\ol{\beta}} F][\ol{\alpha := \beta}]\,F) & = &  
\lambda d\,\eta\,\ol{R}\,\ol{S}.\, (\mathit{fold}_{T_{\rho[\ol{\gamma
        := S}]}} \, (\lambda
\ol{Z}.\,\eta_{\ol{Z}\,\ol{S}}))_{\ol{R}}\\
\hspace*{0.79in}(\Nat^{{\ol{\beta},\ol{\gamma}} }\,(\mu
  \phi.\lambda \overline \alpha.H)\overline \beta\;F)
\rrbracket^\rel \rho & & 
\end{array}\]
\end{dfn}
{\color{blue} Add return type for fold in last clause? Should be
  $\relsem{\Gamma;\ol\beta,\ol\gamma \vdash F}\rho]\ol{\gamma :=
  C}]$.}

If $t$ is closed, i.e., if $\emptyset; \emptyset~|~\emptyset \vdash t
: \tau$, then we write $\setsem{\vdash t : \tau}$ instead of
$\setsem{\emptyset; \emptyset~|~\emptyset \vdash t : \tau}$, and
similarly for $\relsem{\emptyset; \emptyset~|~\emptyset \vdash t :
  \tau}$.
    
\subsection{Basic Properties of Term Interpretations}
    
This interpretation gives that $\relsem{\Gamma;\emptyset \,|\, \Delta
  \vdash \lambda x.t : \sigma \to \tau}\rho = \curry
(\relsem{\Gamma;\emptyset \,|\, \Delta, x : \sigma \vdash t :
  \tau}\rho)$ and $\relsem{\Gamma;\emptyset \,|\, \Delta \vdash st:
  \tau} \rho = \eval \circ \langle \relsem{\Gamma;\emptyset \,|\,
  \Delta \vdash s: \sigma \to \tau}\rho, \relsem{\Gamma;\emptyset
  \,|\, \Delta \vdash t: \sigma}\rho \rangle$, as expected.

The interpretations in Definitions~\ref{def:set-interp}
and~\ref{def:rel-interp} respect weakening, i.e., a term and its
weakenings all have the same set and relational interpretations. In
particular, for any $\rho \in \setenv$,
\[ \setsem{\Gamma;\Phi \,|\, \Delta, x : \sigma \vdash t : \tau}\rho
  = (\setsem{\Gamma;\Phi \,|\, \Delta \vdash t : \tau}\rho) \circ
  \pi_{\Delta}\] where $\pi_{\Delta}$ is the projection
  $\setsem{\Gamma;\Phi \vdash \Delta, x : \sigma} \to
  \setsem{\Gamma;\Phi \vdash \Delta}$, and for any
  $\rho \in \relenv$, 
\[ \relsem{\Gamma;\Phi \,|\, \Delta, x : \sigma \vdash t : \tau}\rho
  = (\relsem{\Gamma;\Phi \,|\, \Delta \vdash t : \tau}\rho) \circ
  \pi_{\Delta}\] where $\pi_{\Delta}$ is the projection
  $\relsem{\Gamma;\Phi \vdash \Delta, x : \sigma} \to
  \relsem{\Gamma;\Phi \vdash \Delta}$. Moreover, if
  $\Gamma,\alpha;\Phi \,|\, \Delta \vdash t : \tau$ and
  $\Gamma;\Phi,\alpha \,|\, \Delta \vdash t' : \tau$ and $\Gamma;\Phi
  \vdash \sigma : \F$ then
\begin{itemize}
\item $\setsem{\Gamma;\Phi \,|\, \Delta[\alpha := \sigma] \vdash
  t[\alpha := \sigma] : \tau[\alpha := \sigma]}\rho =
  \setsem{\Gamma,\alpha;\Phi \,|\, \Delta \vdash t : \tau }\rho [
    \alpha := \setsem{\Gamma;\Phi\vdash\sigma}\rho ]$
\item $\relsem{\Gamma;\Phi \,|\, \Delta[\alpha := \sigma] \vdash
  t[\alpha := \sigma] : \tau[\alpha := \sigma]}\rho =
  \relsem{\Gamma,\alpha;\Phi \,|\, \Delta \vdash t : \tau }\rho [
    \alpha := \relsem{\Gamma;\Phi\vdash\sigma}\rho ]$
\item $\setsem{\Gamma;\Phi \,|\, \Delta[\alpha := \sigma] \vdash
  t'[\alpha := \sigma] : \tau[\alpha := \sigma]}\rho =
  \setsem{\Gamma;\Phi,\alpha \,|\, \Delta \vdash t' : \tau }\rho [
    \alpha := \setsem{\Gamma;\Phi\vdash\sigma}\rho ]$
\item $\relsem{\Gamma;\Phi \,|\, \Delta[\alpha := \sigma] \vdash
  t'[\alpha := \sigma] : \tau[\alpha := \sigma]}\rho =
  \relsem{\Gamma;\Phi,\alpha \,|\, \Delta \vdash t' : \tau }\rho [
    \alpha := \relsem{\Gamma;\Phi\vdash\sigma}\rho ]$
\end{itemize}
and if $\Gamma;\Phi \,|\, \Delta, x: \sigma \vdash t : \tau$ and
$\Gamma;\Phi \,|\, \Delta \vdash s : \sigma$ then
\begin{itemize}
\item $\lambda A.\, \setsem{\Gamma;\Phi \,|\, \Delta \vdash t[x := s]
  : \tau }\rho\, A = \lambda A.\,\setsem{\Gamma;\Phi \,|\, \Delta,
  x: \sigma \vdash t : \tau}\rho (A, \setsem{\Gamma;\Phi \,|\,
  \Delta\vdash s: \sigma}\rho\, A)$
\item $\lambda R.\,\relsem{\Gamma;\Phi \,|\, \Delta \vdash t[x := s] :
  \tau }\rho\,R = \lambda R.\,\relsem{\Gamma;\Phi \,|\, \Delta, x:
  \sigma \vdash t : \tau}\rho \,(R, \relsem{\Gamma;\Phi \,|\,
  \Delta\vdash s: \sigma}\rho\,R)$
\end{itemize}

Direct calculation reveals that the set interpretations of terms also
satisfy
\begin{itemize}
\item $\setsem{\Gamma; \Phi~|~\Delta \vdash
  (L_{\ol{\alpha}}x.t)_{\ol{\tau}}s} = \setsem{\Gamma; \Phi~|~\Delta
  \vdash t [\ol{\alpha := \tau}][x := s]}$
\end{itemize}
Standard type extensionality $\setsem{\Gamma;\Phi\vdash (L_\alpha
  x.t)_\alpha t} = \setsem{\Gamma;\Phi \vdash t}$ and
$\relsem{\Gamma;\Phi\vdash (L_\alpha x.t)_\alpha t} =
\relsem{\Gamma;\Phi \vdash t}$, as well as term extensionality
$\setsem{\Gamma;\Phi\vdash (L_\alpha x.t)_\alpha \top} =
\setsem{\Gamma;\Phi \vdash t}$ and $\relsem{\Gamma;\Phi\vdash
  (L_\alpha x.t)_\alpha \top} = \relsem{\Gamma;\Phi \vdash t}$, for
terms are immediate consequences.

\subsection{Properties of Terms of $\Nat$-Type}

If we define, for $\Gamma; \ol{\alpha} \vdash F$, the term
$\id_F$ to be $\Gamma;\emptyset~|~\emptyset \vdash L_{\ol{\alpha}}x.x
: \Nat^{\ol{\alpha}} F\,F$ and, for terms $\Gamma; \emptyset \,|\,
\Delta \vdash t: \Nat^{\overline{\alpha}} F\,G$ and $\Gamma; \emptyset
\,|\, \Delta \vdash s: \Nat^{\overline{\alpha}} G\,H$, the {\em
  composition} $s \circ t$ of $t$ and $s$ to be $\Gamma; \emptyset
\,|\, \Delta \vdash L_{\overline{\alpha}}
x. s_{\overline{\alpha}}(t_{\overline{\alpha}}x):
\Nat^{\overline{\alpha}} F\,H$, then
\begin{itemize}
\item $\setsem{\Gamma; \emptyset \,|\, \emptyset \vdash \id_{F} :
  \Nat^{\ol{\alpha}} F\,F} \rho\, \ast = \id_{\lambda
  \ol{A}. \setsem{\Gamma; \ol{\alpha} \vdash F} \rho [\ol{\alpha :=
      A}]}$ for any set environment $\rho$
\item $\setsem{\Gamma; \emptyset \,|\, \Delta \vdash s \circ t:
  \Nat^{\overline{\alpha}} F\,H} = \setsem{\Gamma; \emptyset
  \,|\, \Delta \vdash s: \Nat^{\overline{\alpha}} G\,H} 
    \circ \setsem{\Gamma; \emptyset \,|\, \Delta \vdash t:
      \Nat^{\overline{\alpha}} F\,G}$ 
\end{itemize}
Moreover, terms of $\Nat$ type behave as natural transformations with
respect to their source and target functorial types.
\begin{itemize}
\item $\hspace*{0.135in}\setsem{\Gamma; \emptyset \,|\, x :
  \Nat^{\overline{\alpha}, 
    \overline{\gamma}} F\,G, \overline{y : \Nat^{\overline{\gamma}}
    \sigma\, \tau} \vdash ((\map_G^{\overline{\sigma},
    \overline{\tau}})_{\emptyset} \overline{y}) \circ
  (L_{\overline{\gamma}} z. x_{\overline{\sigma}, \overline{\gamma}}
  z) : \Nat^{\overline{\gamma}} F[\overline{\alpha := \sigma}]\,
  G[\overline{\alpha := \tau}]}$ $= \setsem{ \Gamma; \emptyset \,|\, x
  : \Nat^{\overline{\alpha}, \overline{\gamma}} F\, G, \overline{y :
    \Nat^{\overline{\gamma}} \sigma\, \tau} \vdash
  (L_{\overline{\gamma}} z. x_{\overline{\tau}, \overline{\gamma}} z)
  \circ ((\map_F^{\overline{\sigma}, \overline{\tau}})_{\emptyset}
  \overline{y}) : \Nat^{\overline{\gamma}} F[\overline{\alpha :=
      \sigma}]\, G[\overline{\alpha := \tau}]}$
\end{itemize}
As the special case of the previous equality when $x = in_H$ we have

\begin{thm}\label{thm:subst}
\begin{itemize}
\item $\hspace*{0.135in}\setsem{\Gamma; \emptyset \,|\, \overline{y :
    \Nat^{\overline{\gamma}} 
    \sigma\, \tau} \vdash ((\map_{(\mu \phi. \lambda
    \overline{\alpha}. H) \overline{\beta}}^{\overline{\sigma},
    \overline{\tau}})_{\emptyset} \overline{y}) \circ
  (L_{\overline{\gamma}} z. (\tin_{H})_{\overline{\sigma},
    \overline{\gamma}} z) : \xi}$\\
$= \setsem{\Gamma; \emptyset \,|\, \overline{y :
  \Nat^{\overline{\gamma}} \sigma\, \tau} 
  \vdash (L_{\overline{\gamma}} z. (\tin_H)_{\overline{\tau}, \overline{\gamma}} z)
  \circ ((\map_{H[\phi := (\mu \phi. \lambda \overline{\alpha}. H)
      \overline{\beta}]}^{\overline{\sigma},
    \overline{\tau}})_{\emptyset} \overline{y}) 
  : \xi}$\\
at type $\xi = \Nat^{\overline{\gamma}} H[\phi := (\mu \phi. \lambda
  \overline{\alpha}. H) \overline{\beta}][\overline{\alpha :=
    \sigma}]\, (\mu \phi. \lambda \overline{\alpha}. H)\overline{\tau}$
\end{itemize}
\end{thm}

Analogous results hold for relational interpretations of terms and
relational environments.

As we observe in Section~\ref{sec:ft-nat}, Theorem~\ref{thm:subst}
gives a family of results that we normally derive as free theorems but
actually are consequences of naturality. Most of Wadler's fall into
this family, for example, but not the free theorem for filter (even
for lists) or short cut fusion.

\subsection{Properties of Initial Algebraic Constructs}

We first observe that $\map$-terms are interpreted as semantic
$\semmap$s:

Let \(\Gamma; \ol{\phi}, \ol{\gamma} \vdash H : \F\),
\(\ol{\Gamma; \ol{\beta}, \ol{\gamma} \vdash F : \F}\)
and \(\ol{\Gamma; \ol{\beta}, \ol{\gamma} \vdash G : \F}\).
By definition of the semantic interpretation of map terms, we have
\begin{multline}\label{eq:map-sem-def}
\setsem{
\Gamma;\emptyset \,|\, \emptyset
\vdash \map^{\ol{F},\ol{G}}_H
: \Nat^\emptyset\;(\ol{\Nat^{\ol{\beta},\ol{\gamma}}\,F\,G})\; {}
(\Nat^{\ol{\gamma}}\,H[\ol{\phi :=_{\ol{\beta}} F}]\,H[\ol{\phi :=_{\ol{\beta}} G}])
} \rho \\
= \lambda d\, \ol{\eta}\,\ol{B}.\,
\setsem{\Gamma; \ol{\phi},\ol{\gamma}\vdash H}
\id_{\rho[\ol{\gamma := B}]}[\ol{\phi := \lambda \ol{A}.\eta_{\ol{A}\,\ol{B}}}]
\end{multline}

Then let \(\Gamma; \ol{\alpha} \vdash F : \F\),
\(\ol{\Gamma; \emptyset \vdash \sigma : \F}\),
\(\ol{\Gamma; \emptyset \vdash \tau : \F}\)
and $\ast$ be the unique element of $\setsem{\Gamma; \emptyset \vdash \emptyset} \rho$.
As a special case of the above definition, we have
\[
\begin{array}{rl}
&\setsem{
\Gamma;\emptyset \,|\, \emptyset
\vdash \map^{\ol{\sigma},\ol{\tau}}_F
: \Nat^\emptyset\;(\ol{\Nat^{\emptyset}\,\sigma\,\tau})\; {}
(\Nat^{\emptyset}\,F[\ol{\alpha := \sigma}]\,F[\ol{\alpha := \tau}])
} \rho\, \ast \\
=& \lambda \ol{f : \setsem{\Gamma; \emptyset \vdash \sigma}\rho \to \setsem{\Gamma; \emptyset \vdash \tau}\rho}.\,
\setsem{\Gamma; \ol{\alpha}\vdash F}
\id_{\rho}[\ol{\alpha := f}] \\
=& \lambda \ol{f : \setsem{\Gamma; \emptyset \vdash \sigma}\rho \to \setsem{\Gamma; \emptyset \vdash \tau}\rho}.\,
\textit{map}_{\lambda \ol{A}.\,\setsem{\Gamma; \ol{\alpha}\vdash F} \rho [\ol{\alpha := A}]} \ol{f} \\
=& \textit{map}_{\lambda \ol{A}.\,\setsem{\Gamma; \ol{\alpha}\vdash F} \rho [\ol{\alpha := A}]}
\end{array}
\]
where the first equality is by Equation~\ref{eq:map-sem-def}, the
second equality is obtained by noting that $\lambda
\ol{A}.\,\setsem{\Gamma; \ol{\alpha}\vdash F} \rho [\ol{\alpha := A}]$
is a functor in $\alpha$, and $\textit{map}_G$ denotes the action of
the functor $G$ on morphisms.

\vspace*{0.1in}

We also have the expected relationships between interpetations of
terms involving $\map$, $\tin$, and $\fold$:
\begin{itemize}
\item If
$\Gamma; \ol{\psi}, \ol{\gamma} \vdash H$,\;
$\ol{\Gamma; \ol{\alpha}, \ol{\gamma}, \ol{\phi} \vdash K}$,\;
$\ol{\Gamma; \ol{\beta}, \ol{\gamma} \vdash F}$,\; and 
$\ol{\Gamma; \ol{\beta}, \ol{\gamma} \vdash G}$, then
\[\setsem{\Gamma; \emptyset \,|\, \emptyset \vdash
  \map_{H[\ol{\psi := K}]}^{\ol{F}, \ol{G}} : \xi} = \setsem{\Gamma;
  \emptyset \,|\, \emptyset \vdash \map_H^{\ol{K[\ol{\phi := F}]},
    \ol{K[\ol{\phi := G}]}} \circ \ol{\map_K^{\ol{F}, \ol{G}}} : \xi}\] at
type $\xi = \Nat^{\emptyset} (\ol{\Nat^{\ol{\alpha}, \ol{\beta},
    \ol{\gamma}} F G}) (\Nat^{\ol{\gamma}} H[\ol{\psi := K}][\ol{\phi
    := F}] H[\ol{\psi := K}][\ol{\phi := G}])$
\item If $\Gamma; \overline{\beta}, \overline{\gamma} \vdash H$,\;
  $\Gamma; \overline{\beta}, \overline{\gamma} \vdash K$,\;
  $\overline{\Gamma; \overline{\alpha}, \overline{\gamma} \vdash
  F}$,\; $\overline{\Gamma; \overline{\alpha}, \overline{\gamma}
  \vdash G}$,\; $\overline{\Gamma; \phi, \overline{\psi},
  \overline{\gamma} \vdash \tau}$,\; $\overline{I}$ is the sequence
  $\overline{F}, H$ and $\overline{J}$ is the sequence $\overline{G},
  K$ then
\[\hspace*{0.13in}\setsem{\Gamma; \emptyset \,|\, \emptyset
    \vdash L_{\emptyset} (x, \overline{y}). L_{\overline{\gamma}} z.
    x_{\overline{\tau [\overline{\psi := G}] [\phi := K]},
      \overline{\gamma}} \Big(\big( (\map_{H}^{\overline{\tau
        [\overline{\psi := F}] [\phi := H]}, \overline{\tau
        [\overline{\psi := G}] [\phi := K]}})_{\emptyset}
    (\overline{(\map_{\tau}^{\overline{I}, \overline{J}})_{\emptyset}
      (x, \overline{y})}) \big)_{\overline{\gamma}} z \Big) : \xi}\]
\[= \setsem{\Gamma; \emptyset \,|\, \emptyset
    \vdash L_{\emptyset} (x, \overline{y}). L_{\overline{\gamma}} z.
    \big( (\map_{K}^{\overline{\tau [\overline{\psi := F}] [\phi :=
          H]}, \overline{\tau [\overline{\psi := G}] [\phi :=
          K]}})_{\emptyset} (\overline{(\map_{\tau}^{\overline{I},
        \overline{J}})_{\emptyset} (x, \overline{y})})
    \big)_{\overline{\gamma}} \Big( x_{\overline{\tau [\overline{\psi
            := F}] [\phi := H]}, \overline{\gamma}} z \Big) : \xi}\]
at type
    $\xi = \Nat^{\emptyset} (\Nat^{\overline{\beta},
      \overline{\gamma}} H\, K \times
    \overline{\Nat^{\overline{\alpha}, \overline{\gamma}} F\, G})\,
    (\Nat^{\overline{\gamma}} H[\overline{\beta:= \tau}]
             [\overline{\psi := F}] [\phi := H]\, K[\overline{\beta :=
                 \tau}] [\overline{\psi := G}] [\phi := K])$.
\item $\hspace*{0.13in}\setsem{\Gamma; \emptyset \,|\, x: \Nat^{\overline{\beta},
    \overline{\gamma}} H[\phi := F][\overline{\alpha := \beta}]\, F
  \vdash ((\fold_{H, F})_{\emptyset} x) \circ \tin_{H} : \xi}$\\
  $=
  \setsem{ \Gamma; \emptyset \,|\, x: \Nat^{\overline{\beta},
      \overline{\gamma}} H[\phi := F][\overline{\alpha := \beta}]\, F
    \vdash x \circ \big( (\map_{H [\ol{\alpha := \beta}]}^{(\mu
      \phi. \lambda \overline{\alpha} H) \overline{\beta},
      F})_{\emptyset} ((\fold_{H, F})_{\emptyset} x) \big) : \xi}$\\
at type $\xi = \Nat^{\overline{\beta}, \overline{\gamma}} H[\phi :=
  (\mu \phi. \lambda
  \overline{\alpha}. H)\overline{\beta}][\overline{\alpha := \beta}]\,
F$
\item $\hspace*{0.13in}\setsem{\Gamma; \emptyset \,|\, \emptyset \vdash \tin_H 
\circ (\fold_{H, H[\phi := (\mu \phi. \lambda
    \overline{\alpha}. H)\overline{\beta}]})_{\emptyset} 
((\map_{H}^{H[\phi := (\mu \phi. \lambda
    \overline{\alpha}. H)\overline{\beta}][\overline{\alpha :=
      \beta}], (\mu \phi. \lambda
  \overline{\alpha}. H)\overline{\beta}})_{\emptyset} \tin_H) : \xi}$\\
$  = \setsem{\Gamma; \emptyset \,|\, \emptyset
\vdash \Id_{(\mu \phi. \lambda \overline{\alpha}. H)\overline{\beta}}
: \xi}$\\
  at type $\xi = \Nat^{\overline{\beta}, \overline{\gamma}}
(\mu \phi. \lambda \overline{\alpha}. H)\overline{\beta}\, (\mu
\phi. \lambda \overline{\alpha}. H)\overline{\beta}$
\item $\hspace*{0.13in}\setsem{\Gamma; \emptyset \,|\, \emptyset
  \vdash (\fold_{H, 
    H[\phi := (\mu \phi. \lambda
      \overline{\alpha}. H)\overline{\beta}]})_{\emptyset}
  ((\map_{H}^{H[\phi := (\mu \phi. \lambda
      \overline{\alpha}. H)\overline{\beta}][\overline{\alpha :=
        \beta}], (\mu \phi. \lambda
    \overline{\alpha}. H)\overline{\beta}})_{\emptyset} \tin_H) \circ
  \tin_H : \xi}$\\
$=\setsem{
\Gamma; \emptyset \,|\, \emptyset
\vdash \Id_{H[\phi := (\mu \phi. \lambda
    \overline{\alpha}. H)\overline{\beta}]} : \xi}$\\
at type $\xi = 
\Nat^{\overline{\beta}, \overline{\gamma}}
H[\phi := (\mu \phi. \lambda \overline{\alpha}. H)\overline{\beta}]\,
H[\phi := (\mu \phi. \lambda \overline{\alpha}. H)\overline{\beta}]$.
\end{itemize}
%The latter is equivalent to
%$(\fold^F_H\,k)_{\ol{\sigma}\,\ol{\tau}}\,((\tin_H)_{\ol{\sigma}\,\ol{\tau}}\,t)
%= k_{\ol{\sigma}\,\ol{\tau}}\, ( \map_H^{\mu H,
%F}\,(\fold^F_H\,k)_{\ol{\sigma}\,\ol{\tau}}\, t)$ for all
%$\ol{\sigma}$ and $\ol{\tau}$.
Analogous results hold for relational interpretations of terms and
relational environments. {\color{blue} The set and relational
  interpretations of terms therefore respect the congruence closed
  equational theory obtained by adding these judgments to those
  generating the usual congruence closed equational theory induced by
  the other term formers.}

\subsection{Free Theorems Derived from Naturality}\label{sec:ft-nat}

Foralls in $\Nat$-types are at the object level, whereas the foralls
in contexts are at the meta-level. So par results in subst theorem
internalize parametricity in the calculus, whereas those parametricity
results that do not follow from the interpretation of $\Nat$-types are
externalized at the meta-level.


{\color{red} Make this not about $\subst$}
Note that the free theorem for a type is always independent of the
particular term of that type, so the proof below is independent of the
choice of function $\subst$.  In addition, it is independent of the
particular data type --- in this case, $\Lam$ --- over which $\subst$
acts. Also independent of the functor arguments --- in this case
$+\onet$ and $\id$ --- to the data type. Indeed, the following result
is just a consequence of naturality.

We already know from Theorem~\ref{thm:subst} that 

\begin{multline}
\setsem{\Gamma; \emptyset \,|\, x :
  \Nat^{\overline{\alpha}, 
    \overline{\gamma}} F\,G, \overline{y : \Nat^{\overline{\gamma}}
    \sigma\, \tau} \vdash ((\map_G^{\overline{\sigma},
    \overline{\tau}})_{\emptyset} \overline{y}) \circ
  (L_{\overline{\gamma}} z. x_{\overline{\sigma}, \overline{\gamma}}
  z) : \Nat^{\overline{\gamma}} F[\overline{\alpha := \sigma}]\,
  G[\overline{\alpha := \tau}]} \\
= \setsem{ \Gamma; \emptyset \,|\, x
  : \Nat^{\overline{\alpha}, \overline{\gamma}} F\, G, \overline{y :
    \Nat^{\overline{\gamma}} \sigma\, \tau} \vdash
  (L_{\overline{\gamma}} z. x_{\overline{\tau}, \overline{\gamma}} z)
  \circ ((\map_F^{\overline{\sigma}, \overline{\tau}})_{\emptyset}
  \overline{y}) : \Nat^{\overline{\gamma}} F[\overline{\alpha :=
      \sigma}]\, G[\overline{\alpha := \tau}]}
\end{multline}
In particular, if we instantiate $x$ with any term $\subst$ of type
$\vdash \Nat^{\alpha} (\Lam(\alpha + \onet) \times \Lam\alpha)\, \Lam\alpha$
(and thus there is a single $\alpha$ and no $\gamma$'s)
we have
\begin{multline}
\setsem{\Gamma; \emptyset \,|\, y : \Nat^{\emptyset} \sigma\, \tau
  \vdash ((\map_{\Lam\alpha}^{\sigma, \tau})_{\emptyset} y)
  \circ (L_{\emptyset} z. \subst_{\sigma} z)
  : \Nat^{\emptyset} (\Lam(\sigma + \onet) \times \Lam\sigma)\, \Lam\tau } \\
= \setsem{ \Gamma; \emptyset \,|\, y : \Nat^{\emptyset} \sigma\, \tau
  \vdash
  (L_{\emptyset} z. \subst_{\tau} z)
  \circ ((\map_{\Lam(\alpha + \onet) \times \Lam\alpha}^{\sigma, \tau})_{\emptyset} y)
  : \Nat^{\emptyset} (\Lam(\sigma + \onet) \times \Lam\sigma)\, \Lam\tau }
\end{multline}
Thus, for any set environment $\rho$ and any function
$f : \setsem{\Gamma; \emptyset \vdash  \Nat^{\emptyset} \sigma\, \tau}\rho$,
we have that
\begin{multline}
\setsem{\Gamma; \emptyset \,|\, y : \Nat^{\emptyset} \sigma\, \tau
  \vdash ((\map_{\Lam\alpha}^{\sigma, \tau})_{\emptyset} y)
  \circ (L_{\emptyset} z. \subst_{\sigma} z)
  : \Nat^{\emptyset} (\Lam(\sigma + \onet) \times \Lam\sigma)\, \Lam\tau } \rho f \\
= \setsem{\Gamma; \emptyset \,|\, y : \Nat^{\emptyset} \sigma\, \tau
  \vdash ((\map_{\Lam\alpha}^{\sigma, \tau})_{\emptyset} y)} \rho f
  \circ
  \setsem{\Gamma; \emptyset \,|\, y : \Nat^{\emptyset} \sigma\, \tau
  \vdash L_{\emptyset} z. \subst_{\sigma} z} \rho f \\
= \setsem{\Gamma; \emptyset \,|\, \emptyset
  \vdash \map_{\Lam\alpha}^{\overline{\sigma}, \overline{\tau}}} \rho f
  \circ \setsem{\Gamma; \emptyset \,|\, \emptyset
  \vdash L_{\emptyset} z. \subst_{\sigma} z} \rho \\
= \map_{\setsem{\emptyset; \alpha \vdash \Lam\alpha} [\alpha := \_]} f
  \circ
  (\setsem{\vdash \subst})_{\setsem{\Gamma; \emptyset \vdash \sigma} \rho}
\end{multline}
and
\begin{multline}
\setsem{ \Gamma; \emptyset \,|\, y : \Nat^{\emptyset} \sigma\, \tau
  \vdash
  (L_{\emptyset} z. \subst_{\tau} z)
  \circ
  ((\map_{\Lam(\alpha + \onet) \times \Lam\alpha}^{\sigma, \tau})_{\emptyset} y)
  : \Nat^{\emptyset} (\Lam(\sigma + \onet) \times \Lam\sigma)\, \Lam\tau
} \rho f \\
= \setsem{ \Gamma; \emptyset \,|\, y : \Nat^{\emptyset} \sigma\, \tau
  \vdash L_{\emptyset} z. \subst_{\tau} z
} \rho f
\circ
\setsem{ \Gamma; \emptyset \,|\, y : \Nat^{\emptyset} \sigma\, \tau
  \vdash (\map_{\Lam(\alpha + \onet) \times \Lam\alpha}^{\sigma, \tau})_{\emptyset} y
} \rho f \\
= \setsem{ \Gamma; \emptyset \,|\, \emptyset
  \vdash L_{\emptyset} z. \subst_{\tau} z
} \rho
\circ
\setsem{ \Gamma; \emptyset \,|\, \emptyset
  \vdash \map_{\Lam(\alpha + \onet) \times \Lam\alpha}^{\sigma, \tau}
} \rho f \\
= (\setsem{\vdash \subst})_{\setsem{\Gamma; \emptyset \vdash \tau}\rho}
\circ
\map_{\setsem{\emptyset; \alpha \vdash \Lam(\alpha + \onet) \times \Lam\alpha} [\alpha := \_]} f \\
= (\setsem{\vdash \subst})_{\setsem{\Gamma; \emptyset \vdash \tau}\rho}
\circ
(
\map_{
  \setsem{\emptyset; \alpha \vdash \Lam\alpha} [\alpha := \_]} (f + \onet)
\times
\map_{
  \setsem{\emptyset; \alpha \vdash \Lam\alpha} [\alpha := \_]} f
)
\end{multline}
So, we can conclude that
\begin{multline}
\map_{\setsem{\emptyset; \alpha \vdash \Lam\alpha} [\alpha := \_]} f
  \circ
  (\setsem{\vdash \subst})_{\setsem{\Gamma; \emptyset \vdash \sigma} \rho} \\
= (\setsem{\vdash \subst})_{\setsem{\Gamma; \emptyset \vdash \tau}\rho}
\circ
(
\map_{
  \setsem{\emptyset; \alpha \vdash \Lam\alpha} [\alpha := \_]} (f + \onet)
\times
\map_{
  \setsem{\emptyset; \alpha \vdash \Lam\alpha} [\alpha := \_]} f
)
\end{multline}
Moreover, for any $A, B: \set$,
we can choose $\sigma = v$ and $\tau = w$ to be variables such that
$\rho v = A$ and $\rho w = B$.
Then for any function $f : A \to B$ we have that
\begin{multline}
\map_{\setsem{\emptyset; \alpha \vdash \Lam\alpha} [\alpha := \_]} f
  \circ
  (\setsem{\vdash \subst})_{A} \\
= (\setsem{\vdash \subst})_{B}
\circ
(
\map_{
  \setsem{\emptyset; \alpha \vdash \Lam\alpha} [\alpha := \_]} (f + \onet)
\times
\map_{
  \setsem{\emptyset; \alpha \vdash \Lam\alpha} [\alpha := \_]} f
)
\end{multline}

\subsection{The Abstraction Theorem}\label{sec:thms} 

To go beyond naturality and get {\em all} consequences of
parametricity, we prove an Abstraction Theorem for our calculus. In
fact, we actually prove a more general result in
Theorem~\ref{thm:at-gen} about possibly open terms. We then recover
the Abstraction Theorem as the special case of
Theorem~\ref{thm:at-gen} for closed terms of closed type.

\begin{thm}\label{thm:at-gen}
Every well-formed term $\Gamma;\Phi~|~\Delta \vdash t : \tau$ induces
a natural transformation from $\sem{\Gamma; \Phi \vdash \Delta}$ to
$\sem{\Gamma; \Phi \vdash \tau}$, i.e., a triple of natural
transformations 
\[(\setsem{\Gamma;\Phi~|~\Delta \vdash t : \tau},
\setsem{\Gamma;\Phi~|~\Delta \vdash t : \tau},
\relsem{\Gamma;\Phi~|~\Delta \vdash t : \tau})\]
where
\[\begin{array}{lll}
\setsem{\Gamma;\Phi~|~\Delta \vdash t : \tau} & : & \setsem{\Gamma;
  \Phi \vdash \Delta} \to \setsem{\Gamma; \Phi \vdash \tau}
\end{array}\]
has as its component at $\rho : \setenv$ a morphism
\[\begin{array}{lll}
\setsem{\Gamma;\Phi~|~\Delta \vdash t : \tau}\rho & : & \setsem{\Gamma;
  \Phi \vdash \Delta}\rho \to \setsem{\Gamma; \Phi \vdash \tau}\rho
\end{array}\]
in $\set$, and
\[\begin{array}{lll}
\relsem{\Gamma;\Phi~|~\Delta \vdash t : \tau} & : & \relsem{\Gamma;
  \Phi \vdash \Delta} \to \relsem{\Gamma; \Phi \vdash \tau}
\end{array}\]
has as its component at $\rho : \relenv$ a morphism
\[\begin{array}{lll}
\relsem{\Gamma;\Phi~|~\Delta \vdash t : \tau}\rho & : & \relsem{\Gamma;
  \Phi \vdash \Delta}\rho \to \relsem{\Gamma; \Phi \vdash \tau}\rho
\end{array}\]
in $\rel$, and for all $\rho : \relenv$,
\[\relsem{\Gamma;\Phi~|~\Delta \vdash t : \tau}\rho =
(\setsem{\Gamma;\Phi~|~\Delta \vdash t : \tau}(\pi_1 \rho),
\setsem{\Gamma;\Phi~|~\Delta \vdash t : \tau}(\pi_2 \rho))\]
\end{thm}

\begin{proof}
We proceed by structural induction, showing only the interesting
cases.
\begin{itemize}
\item We first consider $\Gamma;\emptyset \,|\, \Delta \vdash
  L_{\overline{\alpha}} x.t : \Nat^{\overline{\alpha}} \,F \,G$.
\begin{itemize}
\item To see that $\setsem{\Gamma;\emptyset \,|\, \Delta \vdash
  L_{\overline{\alpha}} x.t : \Nat^{\overline{\alpha}} \,F \,G}$ is a
  natural transformation from $\setsem{\Gamma;\emptyset \vdash
    \Delta}$ to $\setsem{\Gamma;\emptyset \vdash
    \Nat^{\overline{\alpha}} \,F \,G}$, since the functorial part
  $\Phi$ of the context is empty, we need only show that, for
  every $\rho : \setenv$, $\setsem{\Gamma;\emptyset \,|\, \Delta
    \vdash L_{\overline{\alpha}} x.t : \Nat^{\overline{\alpha}} \,F
    \,G}\rho$ is a morphism in $\set$ from $\setsem{\Gamma; \emptyset
    \vdash \Delta}\rho$ to $\setsem{\Gamma; \emptyset \vdash
    \Nat^{\overline{\alpha}} \,F \,G}\rho$. For this, recall that
\[\setsem{\Gamma;\emptyset \,|\, \Delta \vdash L_{\overline{\alpha}}
  x.t : \Nat^{\overline{\alpha}} \,F \,G}\rho = \curry\,
(\setsem{\Gamma;\overline{\alpha} \,|\, \Delta, x : F \vdash t:
  G}\rho[\overline{\alpha := \_}])\]
By the induction hypothesis, $\sem{\Gamma;\overline{\alpha} \,|\,
  \Delta, x : F \vdash t: G}\rho[\overline{\alpha := \_}]$ induces a
natural transformation
\[\begin{array}{ll}
& \setsem{\Gamma;\overline{\alpha} \,|\, \Delta, x : F \vdash t:
  G}\rho[\overline{\alpha := \_}]\\
 : & \setsem{\Gamma; \ol{\alpha} \vdash \Delta, x :
  F}\rho[\overline{\alpha := \_}] \to \setsem{\Gamma; \ol{\alpha}
  \vdash G}\rho[\overline{\alpha := \_}]\\   
 = & \setsem{\Gamma; \ol{\alpha} \vdash \Delta}\rho[\overline{\alpha
    := \_}] \times \setsem{\Gamma; \ol{\alpha} \vdash
  F}\rho[\overline{\alpha := \_}] \to \setsem{\Gamma; \ol{\alpha}
  \vdash G}\rho[\overline{\alpha := \_}]  
\end{array}\]
and thus a family of morphisms
\[\begin{array}{ll}
& \curry\, (\sem{\Gamma;\overline{\alpha} \,|\,
  \Delta, x : F \vdash t: G}\rho[\overline{\alpha := \_}])\\
: & \setsem{\Gamma; \ol{\alpha} \vdash \Delta}\rho[\overline{\alpha
     := \_}] \to (\setsem{\Gamma; \ol{\alpha} \vdash
   F}\rho[\overline{\alpha := \_}] \to \setsem{\Gamma; \ol{\alpha} 
   \vdash G}\rho[\overline{\alpha := \_}]) 
\end{array}\]
That is, for each $\ol{A : \set}$ and each $d :
\setsem{\Gamma;\emptyset \vdash \Delta}\rho = \setsem{\Gamma;
  \ol{\alpha} \vdash \Delta}\rho[\overline{\alpha := A}]$ by
weakening, we have
\[\begin{array}{ll}
  & (\setsem{\Gamma;\emptyset \,|\, \Delta \vdash L_{\overline{\alpha}}
  x.t : \Nat^{\overline{\alpha}} \,F \,G}\rho\,d)_{\ol{A}}\\
= & \curry\, (\setsem{\Gamma;\overline{\alpha} \,|\, \Delta, x : F
  \vdash t: G}\rho[\overline{\alpha := A}])\,d\\
: & \setsem{\Gamma; \ol{\alpha} \vdash F}\rho[\overline{\alpha := A}]
\to \setsem{\Gamma; \ol{\alpha} \vdash G}\rho[\overline{\alpha := 
    A}]
\end{array}\]
Moreover, these maps actually form a natural transformation $\eta :
\setsem{\Gamma; \ol{\alpha} \vdash F}\rho[\overline{\alpha := \_}] \to
\setsem{\Gamma; \ol{\alpha} \vdash G}\rho[\overline{\alpha := \_}]$
because each
\[ \eta_{\ol{A}} \, = \, \curry\, (\setsem{\Gamma;\overline{\alpha}
  \,|\, \Delta, x : F \vdash t: G}\rho[\overline{\alpha := A}])\,d\]
is the component at $\ol{A}$ of the partial specialization to $d$ of
the natural transformation $\setsem{\Gamma;\overline{\alpha} \,|\,
  \Delta, x : F \vdash t: G}\rho[\overline{\alpha := \_}]$.

To see that the components of $\eta$ also satisfy the additional
condition necessary for $\eta$ to be in $\setsem{\Gamma;\emptyset
  \vdash \Nat^{\overline{\alpha}} \,F \,G}\rho$, let $\overline{R :
  \rel(A, B)}$ and \[(u, v) \in \relsem{\Gamma;\overline{\alpha} \vdash
  F} \Eq_{\rho}[\overline{\alpha := R}] =
(\setsem{\Gamma;\overline{\alpha} \vdash F} \rho[\overline{\alpha :=
    A}], \setsem{\Gamma;\overline{\alpha} \vdash F}
\rho[\overline{\alpha := B}])\] Then the induction hypothesis on the
term $t$ ensures that
\[\begin{array}{ll}
  & \relsem{\Gamma;\overline{\alpha} \,|\, \Delta, x : F \vdash t: G}
\Eq_{\rho}[\overline{\alpha := R}]\\
: & \relsem{\Gamma;\overline{\alpha} \vdash \Delta, x : F}
\Eq_{\rho}[\overline{\alpha := R}] \to
\relsem{\Gamma;\overline{\alpha} \vdash G} \Eq_{\rho}[\overline{\alpha
    := R}]
\end{array}\] and
\[\begin{array}{ll}
  &  \relsem{\Gamma;\overline{\alpha} \,|\, \Delta, x : F \vdash t: G}
\Eq_{\rho}[\overline{\alpha := R}] \hfill(*)\\
= & (\setsem{\Gamma;\overline{\alpha} \,|\, \Delta, x : F \vdash t: G}
\rho[\overline{\alpha := A}], \setsem{\Gamma;\overline{\alpha} \,|\,
  \Delta, x : F \vdash t: G} \rho[\overline{\alpha := B}])
\end{array}\] Since $(d,d) \in \relsem{\Gamma;\emptyset \vdash \Delta}
\Eq_\rho =  \relsem{\Gamma;\emptyset \vdash \Delta}
\Eq_\rho[\ol{\alpha := R}]$ we therefore have that
\[\begin{array}{ll}
& (\eta_{\ol{A}}u,\eta_{\ol{B}}v)\\
= & (\curry\, (\setsem{\Gamma;\overline{\alpha} \,|\, \Delta, x : F
  \vdash t: G}\rho[\overline{\alpha := A}])\,d\,u, \curry\,
(\setsem{\Gamma;\overline{\alpha} \,|\, \Delta, x : F \vdash t:
  G}\rho[\overline{\alpha := B}])\,d\,v)\\
= & \curry\, (\relsem{\Gamma;\overline{\alpha} \,|\, \Delta, x : F
  \vdash t: G}\Eq_\rho[\overline{\alpha := R}])\,(d,d)\,(u,v)\\
: & \relsem{\Gamma;\overline{\alpha} \vdash G}
\Eq_{\rho}[\overline{\alpha := R}]  
\end{array}\]
Here, the second equality is by $(*)$.  
\item The proofs that $\relsem{\Gamma;\emptyset \,|\, \Delta \vdash
  L_{\overline{\alpha}} x.t : \Nat^{\overline{\alpha}} \,F \,G}$ is a
  natural transformation from $\relsem{\Gamma;\emptyset \vdash
    \Delta}$ to $\relsem{\Gamma;\emptyset \vdash
    \Nat^{\overline{\alpha}} \,F \,G}$ and that, for all $\rho :
  \relenv$ and $d : \relsem{\Gamma;\emptyset \vdash \Delta}$,
  \[\relsem{\Gamma;\emptyset \,|\, \Delta \vdash L_{\overline{\alpha}}
    x.t : \Nat^{\overline{\alpha}} \,F \,G}\,\rho\,d\] is a natural
  transformation from $\relsem{\Gamma;\ol{\alpha} \vdash
    F}\rho[\ol{\alpha := \_}]$ to $\relsem{\Gamma;\ol{\alpha} \vdash
    G}\rho[\ol{\alpha := \_}]$, are analogous.
  %To see that
  %$\relsem{\Gamma;\emptyset \,|\, \Delta \vdash L_{\overline{\alpha}}
  %  x.t : \Nat^{\overline{\alpha}} \,F \,G}$ is a natural
  %transformation from $\relsem{\Gamma;\emptyset \vdash \Delta}$ to
  %$\relsem{\Gamma;\emptyset \vdash \Nat^{\overline{\alpha}} \,F \,G}$,
  %since the functorial part $\Phi$ of the type context is empty, we
  %need only show that, for every $\rho : \relenv$,
  %$\setsem{\Gamma;\emptyset \,|\, \Delta \vdash L_{\overline{\alpha}}
  %  x.t : \Nat^{\overline{\alpha}} \,F \,G}\rho$ is a morphism in
  %$\rel$ from $\setsem{\Gamma; \emptyset \vdash \Delta}\rho$ to
  %$\setsem{\Gamma; \emptyset \vdash \Nat^{\overline{\alpha}} \,F
  %  \,G}\rho$. The proof is similar to that in the previous bullet
  %point for the set semantics. The only difference is in the
  %additional condition on the components of the natural transformation
%\[ \eta \, = \, \curry\, (\relsem{\Gamma;\overline{\alpha}
%  \,|\, \Delta, x : F \vdash t: G}\rho[\overline{\alpha := \_}])\,d\]
%for $d : \relsem{\Gamma;\emptyset \vdash \Delta}\rho$ that must be
%verified to know that $\eta$ is in $\relsem{\Gamma;\emptyset \vdash
%  \Nat^{\overline{\alpha}} \,F \,G}\rho$. For that, let $\overline{R :
%  \rel(A,B)}$ and
%\[(u,v) \in \relsem{\Gamma;\overline{\alpha} \vdash
%  F} \rho[\overline{\alpha := R}] = (\setsem{\Gamma;\overline{\alpha}
%  \vdash F} (\pi_1\rho)[\overline{\alpha := A}],
%\setsem{\Gamma;\overline{\alpha} \vdash F}
%(\pi_2\rho)[\overline{\alpha := B}])\]
%Then the induction hypothesis on the term $t$ ensures that
%\[ \relsem{\Gamma;\overline{\alpha} \,|\, \Delta, x : F \vdash t: G}
%\rho[\overline{\alpha := R}] \,: \, \relsem{\Gamma;\overline{\alpha}
%  \vdash \Delta, x : F} \rho[\overline{\alpha := R}] \to
%\relsem{\Gamma;\overline{\alpha} \vdash G} \rho[\overline{\alpha :=
%    R}]\]
%and
%\[\eta_{\ol{R}} = ((\pi_1\eta)_{\ol{A}}, (\pi_1\eta)_{\ol{B}})\]
%so that
%\[\begin{array}{ll}
%  & ((\pi_1\eta)_{\ol{A}} u,(\pi_2\eta)_{\ol{B}} v)\\
%= & \eta_{\ol{R}}(u,v)\\
%= & \curry\,\relsem{\Gamma;\overline{\alpha} \,|\, \Delta, x : F
%  \vdash t: G} \rho[\overline{\alpha := R}]\,d\,(u,v)\\
%: & \relsem{\Gamma;\overline{\alpha} \vdash G} \rho[\overline{\alpha
%    := R}]
%\end{array}\]
\item Finally, to see that $\pi_i(\relsem{\Gamma;\emptyset \,|\,
  \Delta \vdash L_{\overline \alpha} x.t : \Nat^{\overline\alpha} \,F
  \,G}\rho) = \setsem{\Gamma;\emptyset \,|\, \Delta \vdash
  L_{\overline \alpha} x.t : \Nat^{\overline\alpha} \,F
  \,G}(\pi_i\rho)$ we observe that $\pi_1$ and $\pi_2$ are surjective
  and compute
\[\begin{split}
&~\pi_i(\relsem{\Gamma;\emptyset \,|\, \Delta \vdash L_{\overline
    \alpha} x.t : \Nat^{\overline\alpha} \,F \,G}\rho) \\
=&~\pi_i(\curry (\relsem{\Gamma;\overline \alpha\,|\, \Delta, x : F
  \vdash t: G} \rho[\overline{\alpha := \_}])) \\
= &~\curry (\pi_i(\relsem{\Gamma;\overline \alpha\,|\, \Delta, x : F
  \vdash t: G} \rho[\overline{\alpha := \_}])) \\
= &~\curry (\setsem{\Gamma;\overline \alpha\,|\, \Delta, x : F \vdash
  t: G} (\pi_i(\rho[\overline{\alpha := \_}]))) \\
= &~\curry (\setsem{\Gamma;\overline \alpha\,|\, \Delta, x : F \vdash
  t: G} (\pi_i\rho)[\overline{\alpha := \_}]) \\
= &~\setsem{\Gamma;\emptyset \,|\, \Delta \vdash L_{\overline \alpha}
  x.t : \Nat^{\overline\alpha} \,F \,G}(\pi_i\rho)
\end{split}\]
\end{itemize}

\item We now consider $\Gamma;\Phi \,|\, \Delta \vdash t_{\overline
  \tau} s: G [\overline{\alpha := \tau}]$.
\begin{itemize}
\item To see that $\setsem{\Gamma;\Phi \,|\, \Delta \vdash
  t_{\overline \tau} s: G [\overline{\alpha := \tau}]}$ is a natural
  transformation from $\setsem{\Gamma; \Phi \vdash \Delta}$ to
  $\setsem{\Gamma;\Phi \vdash G [\overline{\alpha := \tau}]}$ we need
  to show that, for every $\rho : \setenv$, $\setsem{\Gamma;\Phi \,|\,
    \Delta \vdash t_{\overline \tau} s: G [\overline{\alpha :=
        \tau}]}\rho$ is a morphism from $\setsem{\Gamma; \Phi \vdash
    \Delta}\rho$ to $\setsem{\Gamma;\Phi \vdash G [\overline{\alpha :=
        \tau}]}\rho$, and that this family of morphisms is natural in
  $\rho$. Let $d : \setsem{\Gamma; \Phi \vdash \Delta}\rho$. Then
\[\begin{array}{ll}
  & \setsem{\Gamma;\Phi \,|\, \Delta \vdash t_{\overline \tau} s: G
  [\overline{\alpha := \tau}]}\,\rho\,d\\
= & (\eval \circ \langle (\setsem{\Gamma;\emptyset \,|\, \Delta \vdash
  t : \Nat^{\overline{\alpha}} \,F \,G}\rho\;
\_)_{\overline{\setsem{\Gamma;\Phi \vdash \tau}\rho}},\,
\setsem{\Gamma;\Phi \,|\, \Delta \vdash s: F [\overline{\alpha :=
      \tau}]}\rho \rangle)\,d\\
= & \eval ((\setsem{\Gamma;\emptyset \,|\, \Delta \vdash t :
  \Nat^{\overline{\alpha}} \,F \,G}\rho\;
\_)_{\overline{\setsem{\Gamma;\Phi \vdash \tau}\rho}} \,d,\,
\setsem{\Gamma;\Phi \,|\, \Delta \vdash s: F [\overline{\alpha :=
      \tau}]}\rho\, d)\\
= & \eval ((\setsem{\Gamma;\emptyset \,|\, \Delta \vdash t :
  \Nat^{\overline{\alpha}} \,F \,G}\rho\;
d)_{\overline{\setsem{\Gamma;\Phi \vdash \tau}\rho}},\,
\setsem{\Gamma;\Phi \,|\, \Delta \vdash s: F [\overline{\alpha :=
      \tau}]}\rho\, d)\\
\end{array}\]
By the induction hypothesis, $(\setsem{\Gamma;\emptyset \,|\, \Delta
  \vdash t : \Nat^{\overline{\alpha}} \,F \,G}\rho\;
d)_{\overline{\setsem{\Gamma;\Phi \vdash \tau}\rho}}$ has type
\[\setsem{\Gamma;
  \ol{\alpha} \vdash F}\rho[\ol{\alpha := \setsem{\Gamma;\Phi \vdash
      \tau}\rho}] \to \setsem{\Gamma; \ol{\alpha} \vdash
  G}\rho[\ol{\alpha := \setsem{\Gamma;\Phi \vdash \tau}\rho}]\]
\noindent
and $\setsem{\Gamma;\Phi \,|\, \Delta \vdash s: F [\overline{\alpha :=
      \tau}]}\rho\, d$ has type
\[\begin{array}{ll}
  & \setsem{\Gamma; \Phi \vdash F[\ol{\alpha := \tau}]}\rho\\
= & \setsem{\Gamma; \Phi, \ol{\alpha} \vdash F}\rho[\ol{\alpha :=
    \setsem{\Gamma;\Phi \vdash \tau}\rho}]\\
= & \setsem{\Gamma; \ol{\alpha} \vdash F}\rho[\ol{\alpha :=
    \setsem{\Gamma;\Phi \vdash \tau}\rho}]
\end{array}\]
by Equation~\ref{eq:subs-var}, and by weakening in the last step,
since the type $\Gamma; \emptyset \vdash \Nat^{\ol{\alpha}} F\,G$ is
only well-formed if $\Gamma; \ol{\alpha} \vdash F : \F$ and $\Gamma;
\ol{\alpha} \vdash G : \F$. Thus, $\setsem{\Gamma;\Phi \,|\, \Delta
  \vdash t_{\overline \tau} s: G [\overline{\alpha :=
      \tau}]}\,\rho\,d$ has type $\setsem{\Gamma; \ol{\alpha} \vdash
  G}\rho[\ol{\alpha := \setsem{\Gamma;\Phi \vdash \tau}\rho}] =
\setsem{\Gamma; \Phi \vdash G[\ol{\alpha := \tau}]}\rho$, as desired.

\vspace*{0.1in}

To see that the family of maps comprising $\setsem{\Gamma;\Phi \,|\,
  \Delta \vdash t_{\overline \tau} s: G [\overline{\alpha := \tau}]}$
form a natural transformation, i.e., are natural in their set
environment argument, we need to show that the following diagram
commutes:
\[\begin{tikzcd}
\setsem{\Gamma;\Phi \vdash \Delta}\rho \ar[r, "{\setsem{\Gamma;\Phi
      \vdash \Delta}f}"] \ar[d, "{\langle \setsem{\Gamma;\emptyset
      \,|\, \Delta \vdash t : \Nat^{\overline{\alpha}} \,F \,G}\rho,
    \setsem{\Gamma;\Phi \,|\, \Delta \vdash s: F [\overline{\alpha :=
          \tau}]}\rho \rangle}"'] & \setsem{\Gamma;\Phi \vdash
  \Delta}\rho' \ar[d, "{\langle \setsem{\Gamma;\emptyset \,|\,
      \Delta \vdash t : \Nat^{\overline{\alpha}} \,F \,G}\rho',
    \setsem{\Gamma;\Phi \,|\, \Delta \vdash s: F [\overline{\alpha :=
          \tau}]}\rho' \rangle}"]\\
\setsem{\Gamma;\emptyset \vdash \Nat^{\overline{\alpha}} \,F \,G}\rho
\times \setsem{\Gamma;\Phi \vdash F [\overline{\alpha := \tau}]}\rho
\ar[d, "{\eval \circ ((-)_{\overline{\sem{\Gamma;\Phi\vdash \tau}\rho}} \times
    \id)}"']
\ar[r, bend left = 5, "{\setsem{\Gamma;\emptyset \vdash
      \Nat^{\overline{\alpha}} \,F \,G}f \times \setsem{\Gamma;\Phi
      \vdash F [\overline{\alpha := \tau}]}f}"] &
\setsem{\Gamma;\emptyset \vdash \Nat^{\overline{\alpha}} \,F \,G}\rho'
\times \setsem{\Gamma;\Phi \vdash F [\overline{\alpha := \tau}]}\rho'
\ar[d, "{\eval \circ ((-)_{\overline{\sem{\Gamma;\Phi\vdash
          \tau}\rho'}} \times \id)}"] \\
  %(\setsem{\Gamma;\overline{\alpha} \vdash F}\rho[\overline{\alpha :=
%    \setsem{\Gamma;\Phi\vdash\tau}\rho}] \to
%%\setsem{\Gamma;\overline{\alpha} \vdash G}\rho[\overline{\alpha :=
%    \setsem{\Gamma;\Phi\vdash\tau}\rho}]) \times \setsem{\Gamma;\Phi
%       \vdash F [\overline{\alpha := \tau}]}\rho \ar[d, equal] \\
%(\setsem{\Gamma;\Phi \vdash F [\overline{\alpha := \tau}]}\rho \to
%\setsem{\Gamma;\Phi \vdash G [\overline{\alpha := \tau}]}\rho) \times
%\setsem{\Gamma;\Phi \vdash F [\overline{\alpha := \tau}]}\rho \ar[d,
%  "{\eval}"]\\
%
\setsem{\Gamma;\Phi \vdash G [\overline{\alpha := \tau}]}\rho
\ar[r, "{\setsem{\Gamma;\Phi \vdash G [\overline{\alpha := \tau}]}f}"']
&
\setsem{\Gamma;\Phi \vdash G [\overline{\alpha := \tau}]}\rho'
\end{tikzcd}\]
The top diagram commutes because the induction hypothesis ensures
$\setsem{\Gamma;\emptyset \,|\, \Delta \vdash t :
  \Nat^{\overline{\alpha}} \,F \,G}$ and $\setsem{\Gamma;\Phi \,|\,
  \Delta \vdash s: F [\overline{\alpha := \tau}]}$ are natural in
$\rho$. To see that the bottom diagram commutes, we first note that
since $\rho|_\Gamma = \rho'|_\Gamma$, $\Gamma; \ol{\alpha} \vdash F :
\F$, and $\Gamma; \ol{\alpha} \vdash G : \F$ we can replace the
instance of $f$ in $\setsem{\Gamma;\emptyset \vdash
  \Nat^{\overline{\alpha}} \,F \,G}f$ with $\id$. Then, using the fact
that $\setsem{\Gamma;\emptyset \vdash \Nat^{\overline{\alpha}} \,F
  \,G}$ is a functor, we have that $\setsem{\Gamma;\emptyset \vdash
  \Nat^{\overline{\alpha}} \,F \,G}\id = \id$. To see that the bottom
diagram commutes we must therefore prove that, for every $\eta \in
\setsem{\Gamma;\emptyset \vdash \Nat^{\overline{\alpha}} \,F \,G}\rho$
and $x \in \setsem{\Gamma;\Phi \vdash F [\overline{\alpha :=
      \tau}]}\rho$, we have
\[\setsem{\Gamma;\Phi \vdash G [\overline{\alpha := \tau}]}f
  (\eta_{\overline{\sem{\Gamma;\Phi\vdash \tau}\rho}} x) =
\eta_{\overline{\sem{\Gamma;\Phi\vdash \tau}\rho'}}
(\setsem{\Gamma;\Phi \vdash F [\overline{\alpha := \tau}]}f x)\] i.e.,
that for every $\eta \in \setsem{\Gamma;\emptyset \vdash
  \Nat^{\overline{\alpha}} \,F \,G}\rho$,
\[\setsem{\Gamma;\Phi \vdash G [\overline{\alpha := \tau}]}f
 \circ \eta_{\overline{\sem{\Gamma;\Phi\vdash \tau}\rho}} =
 \eta_{\overline{\sem{\Gamma;\Phi\vdash \tau}\rho'}} \circ
 \setsem{\Gamma;\Phi \vdash F [\overline{\alpha := \tau}]}f\] But this
 follows from the naturality of $\eta$. Indeed, $\eta \in
 \setsem{\Gamma;\emptyset \vdash \Nat^{\overline{\alpha}} \,F
   \,G}\rho$ implies that $\eta$ is a natural transformation from
 $\setsem{\Gamma; \overline{\alpha} \vdash F}\rho[\overline{\alpha :=
     \_}]$ to $\setsem{\Gamma; \overline{\alpha} \vdash
   G}\rho[\overline{\alpha := \_}]$.
For each $\tau$, consider the morphism $\setsem{\Gamma;\Phi \vdash
  \tau}f : \setsem{\Gamma;\Phi \vdash \tau}\rho \to
\setsem{\Gamma;\Phi \vdash \tau}\rho'$. The following diagram commutes
by naturality of $\eta$:
\[\begin{tikzcd}
\setsem{\Gamma;\Phi \vdash F[\ol{\alpha := \tau}]}\rho \ar[d, equal] &
\setsem{\Gamma;\Phi \vdash G[\ol{\alpha := \tau}]}\rho \ar[d,
  equal]\\
\setsem{\Gamma; \ol{\alpha} \vdash F}\rho[\ol{\alpha :=
    \setsem{\Gamma;\Phi \vdash \tau}\rho}] \ar[r,
  "{\;\;\;\eta_{\ol{\setsem{\Gamma;\Phi \vdash \tau}\rho}}\;\;\; }"]
\ar[d, "{\setsem{\Gamma; \ol{\alpha} \vdash F}\id_\rho[\ol{\alpha := 
        \setsem{\Gamma;\Phi \vdash \tau}f}]}"]
& \setsem{\Gamma;
  \ol{\alpha} \vdash G}\rho[\ol{\alpha := \setsem{\Gamma;\Phi \vdash
      \tau}\rho}]
\ar[d, "{\setsem{\Gamma; \ol{\alpha} \vdash G}\id_\rho[\ol{\alpha := 
        \setsem{\Gamma;\Phi \vdash \tau}f}]}"]\\
\setsem{\Gamma; \ol{\alpha} \vdash F}\rho[\ol{\alpha :=
    \setsem{\Gamma;\Phi \vdash \tau}\rho'}] \ar[r,
  "{\eta_{\ol{\setsem{\Gamma;\Phi \vdash \tau}\rho'}} }"]
& \setsem{\Gamma; \ol{\alpha} \vdash G}\rho[\ol{\alpha :=
    \setsem{\Gamma;\Phi \vdash \tau}\rho'}]
\end{tikzcd}\]
That is,
\[\begin{array}{ll}
 & \setsem{\Gamma;\ol{\alpha} \vdash G}\id_\rho[\ol{\alpha :=
    \setsem{\Gamma;\Phi\vdash \tau}f}] \,\circ\,
\eta_{\ol{\setsem{\Gamma;\Phi \vdash \tau}\rho}}\\
=& \eta_{\setsem{\Gamma;\Phi \vdash \tau}\rho'} \,\circ\,
\setsem{\Gamma;\ol{\alpha} \vdash F}\id_\rho[\ol{\alpha :=
    \setsem{\Gamma;\Phi\vdash \tau}f}]
\end{array}\]
But since the only variables in the functorial contexts for $F$ and
$G$ are $\ol{\alpha}$, we have that
\[\begin{array}{ll}
 & \setsem{\Gamma;\ol{\alpha} \vdash
  F}\id_\rho[\ol{\alpha := \setsem{\Gamma;\Phi\vdash \tau}f}]\\
=& \setsem{\Gamma;\ol{\alpha}\vdash F}f[\ol{\alpha :=
    \setsem{\Gamma;\Phi \vdash \tau}f}]\\
=& \setsem{\Gamma;\Phi\vdash F[\ol{\alpha := \tau}]}f
\end{array}\]
and similarly for $G$.  Commutativity of this last diagram thus gives
that $\setsem{\Gamma;\Phi \vdash G[\ol{\alpha := \tau}]}f \,\circ\,
\eta_{\setsem{\Gamma;\Phi \vdash \tau}\rho} =
\eta_{\setsem{\Gamma;\Phi \vdash \tau}\rho'} \,\circ\,
\setsem{\Gamma;\Phi \vdash F[\ol{\alpha := \tau}]}f$, as desired.

\item The proof that $\relsem{\Gamma;\Phi \,|\, \Delta \vdash t_{\bf
    \tau} s: G [\overline{\alpha := \tau}]}$ is a natural
  transformation from $\relsem{\Gamma;\Phi \vdash \Delta}$ to
  $\relsem{\Gamma;\Phi \vdash G [\overline{\alpha := \tau}]}$ is
  analogous.
\item Finally, to see that $\pi_i(\relsem{\Gamma;\Phi \,|\, \Delta
  \vdash t_{\bf \tau} s: G [\overline{\alpha := \tau}]}\rho) =
  \setsem{\Gamma;\Phi \,|\, \Delta \vdash t_{\bf \tau} s: G
    [\overline{\alpha := \tau}]}(\pi_i\rho)$ we compute
\[\begin{array}{ll}
  & \pi_i(\relsem{\Gamma;\Phi \,|\, \Delta \vdash t_{\bf \tau} s: G
  [\overline{\alpha := \tau}]}\rho)\\
= & \pi_i( \eval \circ \langle (\relsem{\Gamma;\emptyset \,|\, \Delta
  \vdash t : \Nat^{\overline{\alpha}} \,F \,G}\rho\;
\_)_{\overline{\relsem{\Gamma;\Phi \vdash \tau}\rho}},
\relsem{\Gamma;\Phi \,|\, \Delta \vdash s: F [\overline{\alpha :=
      \tau}]}\rho \rangle)\\
= & \eval \circ \langle \pi_i((\relsem{\Gamma;\emptyset \,|\, \Delta
  \vdash t : \Nat^{\overline{\alpha}} \,F \,G}\rho\;
\_)_{\overline{\relsem{\Gamma;\Phi \vdash \tau}\rho}}),
\pi_i(\relsem{\Gamma;\Phi \,|\, \Delta \vdash s: F [\overline{\alpha
      := \tau}]}\rho) \rangle\\
= & \eval \circ \langle \pi_i(\relsem{\Gamma;\emptyset \,|\, \Delta
  \vdash t : \Nat^{\overline{\alpha}} \,F \,G}\rho\;
\_)_{\overline{\pi_i(\relsem{\Gamma;\Phi \vdash \tau}\rho)}},
\pi_i(\relsem{\Gamma;\Phi \,|\, \Delta \vdash s: F [\overline{\alpha
      := \tau}]}\rho) \rangle\\
= & \eval \circ \langle (\setsem{\Gamma;\emptyset \,|\, \Delta \vdash
  t : \Nat^{\overline{\alpha}} \,F \,G}(\pi_i\rho)\;
\_)_{\overline{\setsem{\Gamma;\Phi \vdash \tau}(\pi_i\rho)}},
\setsem{\Gamma;\Phi \,|\, \Delta \vdash s: F [\overline{\alpha :=
      \tau}]}(\pi_i\rho) \rangle\\
= & \setsem{\Gamma;\Phi \,|\, \Delta \vdash t_{\bf \tau} s: G
  [\overline{\alpha := \tau}]}(\pi_i\rho)
\end{array}\]
\end{itemize}
\item We now consider
  $\Gamma;\emptyset~|~\emptyset \vdash 
  \map^{\ol{F},\ol{G}}_H :
  \Nat^\emptyset\;(\ol{\Nat^{\ol{\beta},\ol{\gamma}}\,F\,G})\;
  (\Nat^{\ol{\gamma}}\,H[\ol{\phi :=_{\ol{\beta}} F}]\,H[\ol{\phi
      :=_{\ol{\beta}} G}])$.
\begin{itemize}
\item To see that $\setsem{\Gamma; \emptyset~|~\emptyset \vdash
  \map^{\ol{F},\ol{G}}_H :
  \Nat^\emptyset\;(\ol{\Nat^{\ol{\beta},\ol{\gamma}}\,F\,G})\;
  (\Nat^{\ol{\gamma}}\,H[\ol{\phi :=_{\ol{\beta}} F}]\,H[\ol{\phi
      :=_{\ol{\beta}} G}])}$ is a natural transformation from
  $\setsem{\Gamma;\emptyset \vdash \emptyset}$ to
  $\setsem{\Nat^\emptyset\;(\ol{\Nat^{\ol{\beta},\ol{\gamma}}\,F\,G})\;
  (\Nat^{\ol{\gamma}}\,H[\ol{\phi :=_{\ol{\beta}} F}]\,H[\ol{\phi
      :=_{\ol{\beta}} G}])}$, since the functorial part $\Phi$ of the
  context is empty, we need only show that, for every $\rho :
  \setenv$,
  \[\setsem{\Gamma; \emptyset~|~\emptyset \vdash
  \map^{\ol{F},\ol{G}}_H :
  \Nat^\emptyset\;(\ol{\Nat^{\ol{\beta},\ol{\gamma}}\,F\,G})\;
  (\Nat^{\ol{\gamma}}\,H[\ol{\phi :=_{\ol{\beta}} F}]\,H[\ol{\phi
      :=_{\ol{\beta}} G}])}\,\rho\]
is a morphism in $\set$
  from $\setsem{\Gamma;\emptyset \vdash \emptyset}\rho$ to
\[\setsem{\Gamma; \emptyset \vdash \Nat^\emptyset\;(\ol{\Nat^{\ol{\beta},\ol{\gamma}}\,F\,G})\;
  (\Nat^{\ol{\gamma}}\,H[\ol{\phi :=_{\ol{\beta}} F}]\,H[\ol{\phi
      :=_{\ol{\beta}} G}])}\,\rho\] i.e., that, for the unique $d :
\setsem{\Gamma;\emptyset \vdash \emptyset} \rho$,
\[\setsem{\Gamma; \emptyset~|~\emptyset \vdash
\map^{\ol{F},\ol{G}}_H :
\Nat^\emptyset\;(\ol{\Nat^{\ol{\beta},\ol{\gamma}}\,F\,G})\;
(\Nat^{\ol{\gamma}}\,H[\ol{\phi :=_{\ol{\beta}} F}]\,H[\ol{\phi
    :=_{\ol{\beta}} G}])}\,\rho\,d\] is a morphism from
$\ol{\setsem{\Gamma; \emptyset
    \vdash\Nat^{\ol{\beta},\ol{\gamma}}\,F\,G} \rho}$ to
$\setsem{\Gamma; \emptyset \vdash \Nat^{\ol{\gamma}}\,H[\ol{\phi
      :=_{\ol{\beta}} F}]\,H[\ol{\phi :=_{\ol{\beta}} G}]} \rho$.  For
this we show that for all $\ol{\eta : \setsem{\Gamma; \emptyset
    \vdash\Nat^{\ol{\beta},\ol{\gamma}}\,F\,G} \rho}$ we have
\[\begin{array}{ll}
  & \setsem{\Gamma; \emptyset~|~\emptyset \vdash
\map^{\ol{F},\ol{G}}_H :
\Nat^\emptyset\;(\ol{\Nat^{\ol{\beta},\ol{\gamma}}\,F\,G})\;
(\Nat^{\ol{\gamma}}\,H[\ol{\phi :=_{\ol{\beta}} F}]\,H[\ol{\phi
    :=_{\ol{\beta}} G}])}\,\rho\,d\, \ol{\eta}\\
: & \setsem{\Gamma; \emptyset \vdash \Nat^{\ol{\gamma}}\,H[\ol{\phi
      :=_{\ol{\beta}} F}]\,H[\ol{\phi :=_{\ol{\beta}} G}]} \rho
\end{array}\]
To this end, we note that, for any $\ol{B}$, 
\[\begin{array}{ll}
  & (\setsem{\Gamma; \emptyset~|~\emptyset \vdash
\map^{\ol{F},\ol{G}}_H :
\Nat^\emptyset\;(\ol{\Nat^{\ol{\beta},\ol{\gamma}}\,F\,G})\;
(\Nat^{\ol{\gamma}}\,H[\ol{\phi :=_{\ol{\beta}} F}]\,H[\ol{\phi
    :=_{\ol{\beta}} G}])}\,\rho\,d\, \ol{\eta})_{\ol{B}}\\
= & \setsem{\Gamma; \ol{\phi},\ol{\gamma} \vdash H}\id_{\rho[\ol{\gamma
      := B}]}[\ol{\phi := \lambda \ol{A}. \eta_{\ol{A}\,\ol{B}}}]
\end{array}\]
is indeed a morphism from
\[\begin{array}{ll}
  & \setsem{\Gamma ;\ol{\gamma} \vdash H[\ol{\phi := F}]}\rho[\ol{\gamma
      := B}]\\
= & \setsem{\Gamma ;\ol{\gamma},\ol{\phi} \vdash H}\rho[\ol{\gamma
      := B}][\ol{\phi := \lambda \ol{A}.\setsem{\Gamma;
        \ol{\gamma},\ol{\beta} \vdash F}\rho[\ol{\gamma :=
          B}][\ol{\beta := A}]}]
\end{array}\]
to 
\[\begin{array}{ll}
  & \setsem{\Gamma ;\ol{\gamma} \vdash H[\ol{\phi := G}]}\rho[\ol{\gamma
      := B}]\\
= & \setsem{\Gamma ;\ol{\gamma},\ol{\phi} \vdash H}\rho[\ol{\gamma
      := B}][\ol{\phi := \lambda \ol{A}.\setsem{\Gamma;
        \ol{\gamma},\ol{\beta} \vdash G}\rho[\ol{\gamma :=
          B}][\ol{\beta := A}]}]
\end{array}\]
since $\setsem{\Gamma ;\ol{\phi}, \ol{\gamma} \vdash H}$ is a functor
from $\setenv$ to $\set$ and $\id_{\rho[\ol{\gamma := B}]}[\ol{\phi :=
    \lambda \ol{A}. \eta_{\ol{A}\,\ol{B}}}]$ is a morphism in
$\setenv$ from \[\rho[\ol{\gamma := B}][\ol{\phi := \lambda
    \ol{A}.\setsem{\Gamma; \ol{\gamma},\ol{\beta} \vdash
      F}\rho[\ol{\gamma := B}][\ol{\beta := A}]}]\] to \[\rho[\ol{\gamma
    := B}][\ol{\phi := \lambda \ol{A}.\setsem{\Gamma;
      \ol{\gamma},\ol{\beta} \vdash G}\rho[\ol{\gamma := B}][\ol{\beta
        := A}]}]\]

\vspace*{0.1in}

To see that this family of morphisms is natural in $\ol{B}$ we first
observe that if $\ol{f : B \to B'}$ then, writing $t$ for
$\setsem{\Gamma; \emptyset~|~\emptyset \vdash \map^{\ol{F},\ol{G}}_H :
  \Nat^\emptyset\;(\ol{\Nat^{\ol{\beta},\ol{\gamma}}\,F\,G})\;
  (\Nat^{\ol{\gamma}}\,H[\ol{\phi :=_{\ol{\beta}} F}]\,H[\ol{\phi
      :=_{\ol{\beta}} G}])}\,\rho\,d\, \ol{\eta}$, we have
\[\begin{tikzcd}
\setsem{\Gamma ;\ol{\gamma} \vdash H[\ol{\phi := F}]}\rho[\ol{\gamma
      := B}] \ar[r,
  "{\;\;\;t_{\ol{B}}\;\;\; }"]
\ar[d, "{\setsem{\Gamma ;\ol{\gamma} \vdash H[\ol{\phi :=
          F}]}\id_{\rho}[\ol{\gamma := f}]}"']
& \setsem{\Gamma ;\ol{\gamma} \vdash H[\ol{\phi := G}]}\rho[\ol{\gamma
      := B}]
\ar[d, "{\setsem{\Gamma ;\ol{\gamma} \vdash H[\ol{\phi :=
          G}]}\id_{\rho}[\ol{\gamma := f}]}"]\\
\setsem{\Gamma ;\ol{\gamma} \vdash H[\ol{\phi := F}]}\rho[\ol{\gamma
      := B'}] \ar[r,
  "{t_{\ol{B'}}}"]
& \setsem{\Gamma ;\ol{\gamma} \vdash H[\ol{\phi := G}]}\rho[\ol{\gamma
      := B'}] 
\end{tikzcd}\]
This diagram commutes because $\setsem{\Gamma; \ol{\phi},\ol{\gamma}
  \vdash H}$ is a functor from $\setenv$ to $\set$ and because,
letting
\[ E_{F,B} = \rho[\ol{\gamma := B}][\ol{\phi := \lambda \ol{A}.\, \setsem{\Gamma;
      \ol{\gamma},\ol{\beta} \vdash F}\rho[\ol{\gamma := B}][\ol{\beta
        := A}]}]\] and
\[ e_{F,f} = \id_\rho[\ol{\gamma := f}][\ol{\phi := \lambda \ol{A}.\, \setsem{\Gamma;
      \ol{\gamma},\ol{\beta} \vdash F}\rho[\ol{\gamma := f}][\ol{\beta
        := \id_A}]}]\] for all $F$ and $B$ and $\ol{f : B \to B'}$,
the following diagram commutes by the fact that composition of
environments is componentwise together with the naturality of $\eta$:
\[\begin{tikzcd}[column sep=2in, row sep=0.75in]
E_{F,B}
\ar[r,  "{\hspace*{0.5in}\id_\rho[{\gamma :=
        \id_B}][\ol{\phi := \lambda 
        \ol{A}.\,\eta_{\ol{A}\,\ol{B}}}]\hspace*{0.5in}}"] 
\ar[d, "{e_{F,f}}"'] 
& E_{G,B}
\ar[d, "{e_{G,f}}"]\\
E_{F,B'} \ar[r, "{\hspace*{0.5in}\id_\rho[{\gamma :=
        \id_{B'}}][\ol{\phi := \lambda 
        \ol{A}.\,\eta_{\ol{A}\,\ol{B'}}}]\hspace*{0.5in}}"']
& E_{G,B'}
\end{tikzcd}\]
We therefore have that
\[\lambda \ol{B}.\, \setsem{\Gamma; \emptyset~|~\emptyset \vdash
  \map^{\ol{F},\ol{G}}_H :
  \Nat^\emptyset\;(\ol{\Nat^{\ol{\beta},\ol{\gamma}}\,F\,G})\; 
  (\Nat^{\ol{\gamma}}\,H[\ol{\phi :=_{\ol{\beta}} F}]\,H[\ol{\phi
      :=_{\ol{\beta}} G}])}\,\rho\,d\, \ol{\eta})_{\ol{B}}\]
  is natural in $\ol{B}$ as desired.
\item To see that, for every $\rho : \setenv$ and $d :
  \setsem{\Gamma;\emptyset \vdash \emptyset}\rho$, and all $\ol{\eta :
    \setsem{\Gamma; \emptyset \vdash
      \Nat^{\ol{\beta},\ol{\gamma}}\,F\,G}\rho}$,
\[ \setsem{\Gamma; \emptyset~|~\emptyset \vdash
  \map^{\ol{F},\ol{G}}_H :
  \Nat^\emptyset\;(\ol{\Nat^{\ol{\beta},\ol{\gamma}}\,F\,G})\;
  (\Nat^{\ol{\gamma}}\,H[\ol{\phi :=_{\ol{\beta}} F}]\,H[\ol{\phi
      :=_{\ol{\beta}} G}])}\,\rho\,d\, \ol{\eta}\] satisfies the
additional condition necessary for it to be in $\setsem{\Gamma;
  \emptyset \vdash \Nat^{\ol{\gamma}}\,H[\ol{\phi :=_{\ol{\beta}}
      F}]\,H[\ol{\phi :=_{\ol{\beta}} G}]}\rho$, let $\ol{R :
  \rel(B,B')}$ and $\ol{S : \rel(C,C')}$. Since each map in
$\ol{\eta}$ satisfies the extra condition necessary for it to be in
its corresponding $\setsem{\Gamma; \emptyset \vdash
  \Nat^{\ol{\beta},\ol{\gamma}}\,F\,G}\rho$ --- i.e., since
\[(\eta_{\ol{B}\,\ol{C}},\eta_{\ol{B'}\,\ol{C'}}) \in \relsem{\Gamma;
  \ol{\beta},\ol{\gamma} \vdash F}\Eq_\rho[\ol{\beta := R}][\ol{\gamma
    :=S}] \to \relsem{\Gamma; \ol{\beta},\ol{\gamma} \vdash
  G}\Eq_\rho[\ol{\beta := R}][\ol{\gamma :=S}]\]
--- we have that
\[\begin{array}{ll}
  & (\, (\lambda e\,\nu\,\ol{Z}.\,\setsem{\Gamma;\ol{\phi},\ol{\gamma}
  \vdash H}\id_{\rho[\ol{\gamma := Z}]}[\ol{\phi := \lambda
    \ol{A}.\nu_{\ol{A}\,\ol{Z}}}])\,d\,\ol{\eta}\,\ol{B},\\ 
 & \hspace*{0.5in}(\lambda e\,\nu\,\ol{Z}.\,\setsem{\Gamma;\ol{\phi},\ol{\gamma}
  \vdash H}\id_{\rho[\ol{\gamma := Z}]}[\ol{\phi := \lambda
    \ol{A}.\nu_{\ol{A}\,\ol{Z}}}])\,d\,\ol{\eta}\,\ol{B'}\;)\\
= & (\, \setsem{\Gamma;\ol{\phi},\ol{\gamma}
  \vdash H}\id_{\rho[\ol{\gamma := B}]}[\ol{\phi := \lambda
    \ol{A}.\eta_{\ol{A}\,\ol{B}}}]),\; \setsem{\Gamma;\ol{\phi},\ol{\gamma}
  \vdash H}\id_{\rho[\ol{\gamma := B'}]}[\ol{\phi := \lambda
    \ol{A}.\eta_{\ol{A}\,\ol{B'}}}])\,)
\end{array}\]
has type
\[\begin{array}{ll}
  & (\setsem{\Gamma;\ol{\gamma} \vdash H[\ol{\phi := F}]}\rho[\ol{\gamma
    := B}] \to \setsem{\Gamma;\ol{\gamma} \vdash H[\ol{\phi :=
      G}]}\rho[\ol{\gamma := B}],\\
  & \hspace*{0.5in}\setsem{\Gamma;\ol{\gamma} \vdash H[\ol{\phi := 
      F}]}\rho[\ol{\gamma := B'}] \to \setsem{\Gamma;\ol{\gamma} \vdash
  H[\ol{\phi := G}]}\rho[\ol{\gamma := B'}])\\
= & \relsem{\Gamma;\ol{\gamma} \vdash H[\ol{\phi := F}]}\Eq_\rho[\ol{\gamma
    := R}] \to \relsem{\Gamma;\ol{\gamma} \vdash H[\ol{\phi :=
      G}]}\Eq_\rho[\ol{\gamma := R}]
\end{array}\]
as desired.
\item The proofs that 
\[\relsem{\Gamma; \emptyset~|~\emptyset \vdash
  \map^{\ol{F},\ol{G}}_H :
  \Nat^\emptyset\;(\ol{\Nat^{\ol{\beta},\ol{\gamma}}\,F\,G})\;
  (\Nat^{\ol{\gamma}}\,H[\ol{\phi :=_{\ol{\beta}} F}]\,H[\ol{\phi
      :=_{\ol{\beta}} G}])}\]
is a natural transformation from $\relsem{\Gamma;\emptyset \vdash
  \emptyset}$ to 
\[\relsem{\Gamma; \emptyset \vdash
  \Nat^\emptyset\;(\ol{\Nat^{\ol{\beta},\ol{\gamma}}\,F\,G})\;
  (\Nat^{\ol{\gamma}}\,H[\ol{\phi :=_{\ol{\beta}} F}]\,H[\ol{\phi
      :=_{\ol{\beta}} G}])}\]
and that, for every $\rho : \relenv$ and the unique $d :
\relsem{\Gamma;\emptyset \vdash \emptyset}\rho$,
\[\relsem{\Gamma; \emptyset~|~\emptyset \vdash
  \map^{\ol{F},\ol{G}}_H :
  \Nat^\emptyset\;(\ol{\Nat^{\ol{\beta},\ol{\gamma}}\,F\,G})\;
  (\Nat^{\ol{\gamma}}\,H[\ol{\phi :=_{\ol{\beta}} F}]\,H[\ol{\phi
      :=_{\ol{\beta}} G}])}\,\rho\,d\] is a morphism from
$\ol{\relsem{\Gamma; \emptyset \vdash
    \Nat^{\ol{\beta},\ol{\gamma}}\,F\,G}}\,\rho$ to $\relsem{\Gamma;
  \emptyset \vdash \Nat^{\ol{\gamma}}\,H[\ol{\phi :=_{\ol{\beta}}
      F}]\,H[\ol{\phi :=_{\ol{\beta}} G}]}\,\rho$, are analogous.
\item Finally, to see that 
  \[\begin{array}{ll}
  & \pi_i (\relsem{\Gamma; \emptyset~|~\emptyset \vdash
    \map_H^{\ol{F}\,\ol{G}} : 
  \Nat^\emptyset\;(\ol{\Nat^{\ol{\beta},\ol{\gamma}}\,F\,G})\;
  (\Nat^{\ol{\gamma}}\,H[\ol{\phi :=_{\ol{\beta}} F}]\,H[\ol{\phi
      :=_{\ol{\beta}} G}])}\,\rho)\\
= & \setsem{\Gamma; \emptyset~|~\emptyset \vdash
    \map_H^{\ol{F}\,\ol{G}} : 
  \Nat^\emptyset\;(\ol{\Nat^{\ol{\beta},\ol{\gamma}}\,F\,G})\;
  (\Nat^{\ol{\gamma}}\,H[\ol{\phi :=_{\ol{\beta}} F}]\,H[\ol{\phi
      :=_{\ol{\beta}} G}])}\,(\pi_i \rho)
\end{array}\]
we compute
  \[\begin{array}{ll}
  & \pi_i (\relsem{\Gamma; \emptyset~|~\emptyset \vdash
    \map_H^{\ol{F}\,\ol{G}} : 
  \Nat^\emptyset\;(\ol{\Nat^{\ol{\beta},\ol{\gamma}}\,F\,G})\;
  (\Nat^{\ol{\gamma}}\,H[\ol{\phi :=_{\ol{\beta}} F}]\,H[\ol{\phi
      :=_{\ol{\beta}} G}])}\,\rho)\\
  = & \pi_i \, (\lambda e\, \ol{\nu}\,\ol{R}.\, \relsem{
    \Gamma; \ol{\phi},\ol{\gamma} \vdash H} \id_{\rho[\ol{\gamma :=
        R}]}[\ol{\phi := \lambda \ol{S}.\nu_{\ol{S}\,\ol{R}}}])\\
  = & \lambda e\, \ol{\nu}\,\ol{R}.\, \setsem{
    \Gamma; \ol{\phi},\ol{\gamma} \vdash H} \id_{(\pi_i \rho)[\ol{\gamma :=
        \pi_i R}]}[\ol{\phi := \lambda \ol{S}.(\pi_i \nu)_{\ol{\pi_i
          S}\,\ol{\pi_i R}}}]\\
  = & \lambda d\, \ol{\eta}\,\ol{B}.\, \setsem{
    \Gamma; \ol{\phi},\ol{\gamma} \vdash H} \id_{(\pi_i \rho)[\ol{\gamma :=
        B}]}[\ol{\phi := \lambda \ol{A}.\eta_{\ol{A}\,\ol{B}}}]\\
  = & \setsem{\Gamma; \emptyset~|~\emptyset \vdash
    \map_H^{\ol{F}\,\ol{G}} : 
  \Nat^\emptyset\;(\ol{\Nat^{\ol{\beta},\ol{\gamma}}\,F\,G})\;
  (\Nat^{\ol{\gamma}}\,H[\ol{\phi :=_{\ol{\beta}} F}]\,H[\ol{\phi
      :=_{\ol{\beta}} G}])}\,(\pi_i \rho)
  \end{array}\]
\end{itemize}
\item We now consider $\Gamma;\emptyset \,|\, \emptyset \vdash \tin_H :
  Nat^{\ol{\beta},\ol{\gamma}} \, H[\phi := (\mu \phi.\lambda
    {\overline \alpha}.H){\overline \beta}][\ol{\alpha := \beta}]
  \;(\mu \phi.\lambda {\overline \alpha}.H){\overline \beta}$.
\begin{itemize}
\item To see that if $d : \setsem{\Gamma;\emptyset \vdash \emptyset}
  \rho$ then \[\setsem{\Gamma;\emptyset \,|\, \emptyset \vdash \tin_H :
    Nat^{\ol{\beta},\ol{\gamma}} \, H[\phi := (\mu \phi.\lambda
      {\overline \alpha}.H){\overline \beta}][\ol{\alpha := \beta}]
    \;(\mu \phi.\lambda {\overline \alpha}.H){\overline \beta}}\,
  \rho\,d\] is in $\setsem{\Gamma;\emptyset \vdash
    Nat^{\ol{\beta},\ol{\gamma}} \, H[\phi := (\mu \phi.\lambda
      {\overline \alpha}.H){\overline \beta}][\ol{\alpha := \beta}]
    \;(\mu \phi.\lambda {\overline \alpha}.H){\overline \beta}}\,
  \rho$, we first note that, for all $\ol{B}$ and $\ol{C}$,
  \[\begin{array}{ll}
   &(\setsem{\Gamma;\emptyset \,|\, \emptyset \vdash \tin_H :
    Nat^{\ol{\beta},\ol{\gamma}} \, H[\phi := (\mu \phi.\lambda
      {\overline \alpha}.H){\overline \beta}][\ol{\alpha := \beta}]
    \;(\mu \phi.\lambda {\overline \alpha}.H){\overline \beta}}\,
  \rho\,d)_{\ol{B}\,\ol{C}}\\
  =& (\mathit{in}_{T_{\rho[\ol{\gamma := C}]}})_{\ol{B}}
\end{array}\] does indeed map
\[\begin{array}{ll}
  & \setsem{\Gamma;\ol{\beta},\ol{\gamma} \vdash H[\phi := (\mu
    \phi.\lambda {\overline \alpha}.H){\overline \beta}][\ol{\alpha :=
      \beta}]}\rho[\ol{\beta := B}][\ol{\gamma := C}]\\
\hspace{0.2in} = & \setsem{\Gamma;\ol{\beta},\ol{\gamma},\ol{\alpha} \vdash H[\phi :=
    (\mu \phi.\lambda {\overline \alpha}.H){\overline
      \beta}]}\rho[\ol{\beta := B}][\ol{\gamma := C}][\ol{\alpha :=
    B}]\\
\hspace{0.2in} = & \setsem{\Gamma;\phi,\ol{\beta},\ol{\gamma},\ol{\alpha} \vdash H}
\rho[\ol{\beta := B}][\ol{\gamma := C}][\ol{\alpha :=
    B}]\\ & \hspace{0.5in} [\phi := \lambda
  \ol{D}.\,\setsem{\Gamma; \ol{\beta},\ol{\gamma},\ol{\alpha}
    \vdash (\mu \phi.\lambda {\overline \alpha}.H){\overline
      \beta}}\rho[\ol{\beta := B}][\ol{\gamma := C}][\ol{\alpha :=
    B}][\ol{\beta := D}]]\\
\hspace{0.2in} = & \setsem{\Gamma;\phi,\ol{\gamma},\ol{\alpha} \vdash H}
\rho[\ol{\gamma := C}][\ol{\alpha := B}]\\
  & \hspace*{0.5in}[\phi := \lambda \ol{D}.\,
  \setsem{\Gamma; \ol{\beta},\ol{\gamma}\vdash (\mu 
    \phi.\lambda {\overline \alpha}.H){\overline \beta}}\rho[\ol{\beta
      := D}][\ol{\gamma := C}]]\\
\hspace{0.2in} = & T_{\rho[\ol{\gamma := C}]}\, (\lambda \ol{D}.\,\setsem{\Gamma;
  \ol{\beta},\ol{\gamma}\vdash (\mu 
  \phi.\lambda {\overline \alpha}.H){\overline \beta}}\rho[\ol{\beta
    := D}][\ol{\gamma := C}])\, \ol{B}\\
\hspace{0.2in} = & T_{\rho[\ol{\gamma := C}]}\, (\mu T_{\rho[\ol{\gamma := C}]}) \, \ol{B}
\end{array}\] to
\[\begin{array}{ll}
 & \setsem{\Gamma;\ol{\beta},\ol{\gamma} \vdash (\mu \phi.\lambda
  {\overline \alpha}.H){\overline \beta}} \rho[\ol{\beta :=
    B}][\ol{\gamma := C}]\\
= & (\lambda \ol{D}.\, \setsem{\Gamma;
  \ol{\beta},\ol{\gamma}\vdash (\mu \phi.\lambda {\overline
    \alpha}.H){\overline \beta}}\rho[\ol{\beta := D}][\ol{\gamma :=
    C}]) \ol{B}\\
= & (\mu T_{\rho[\ol{\gamma := C}]}) \, \ol{B}
\end{array}\]


\vspace*{0.1in}

To see that
\[\begin{array}{ll}
 & \setsem{\Gamma;\emptyset \,|\, \emptyset \vdash \tin_H :
  Nat^{\ol{\beta},\ol{\gamma}} \, H[\phi := (\mu \phi.\lambda
    {\overline \alpha}.H){\overline \beta}][\ol{\alpha := \beta}]
  \;(\mu \phi.\lambda {\overline \alpha}.H){\overline
    \beta}}\,\rho\,d\\
=& \lambda \ol{B}\,\ol{C}.\,(\mathit{in}_{T_{\rho[\ol{\gamma :=
        C}]}})_{\ol{B}}
\end{array}\]
is natural in $\ol{B}$ and $\ol{C}$, we observe that the following
diagram commutes for all $\ol{f : B \to B'}$ and $\ol{g : C \to C'}$:
\[\begin{tikzcd}[column sep=2.5in, row sep=0.75in]
T_{\rho[\ol{\gamma := C}]}\, (\mu T_{\rho[\ol{\gamma := C}]})\, \ol{B}
\ar[d, "{\sigma_{\id_\rho[\ol{\gamma := g}]}\,(\mu
    \sigma_{\id_\rho[\ol{\gamma := g}]}) \, \ol{B}}"] \ar[r,
  "{(\mathit{in}_{T_{\rho[\ol{\gamma := C}]}})_{\ol{B}}}" ]
& (\mu T_{\rho[\ol{\gamma := C}]})\, \ol{B} \ar[d, "{(\mu
    \sigma_{\id_\rho[\ol{\gamma := g}]}) \, \ol{B}}"]\\
T_{\rho[\ol{\gamma := C'}]}\, (\mu
T_{\rho[\ol{\gamma := C'}]})\, \ol{B} \ar[d, "{T_{\rho[\ol{\gamma :=
          C'}]}\, (\mu T_{\rho[\ol{\gamma := C'}]})\, \ol{f}}"]
\ar[r,"{(\mathit{in}_{T_{\rho[\ol{\gamma := C'}]}})_{\ol{B}}}" ] & 
  (\mu T_{\rho[\ol{\gamma := C'}]})\, \ol{B}
\ar[d,"{(\mathit{in}_{T_{\rho[\ol{\gamma := C'}]}})_{\ol{B}}}" ] 
\\
T_{\rho[\ol{\gamma := C'}]}\, (\mu T_{\rho[\ol{\gamma := C'}]})\,
\ol{B'} \ar[r, "{(\mathit{in}_{T_{\rho[\ol{\gamma :=
            C'}]}})_{\ol{B'}}}"] & (\mu T_{\rho[\ol{\gamma := C'}]})\,
\ol{B'}
%\ar[d, "{(\mathit{in}_{T_{\rho[\ol{\gamma := C}]}})_{\ol{B}}}"']
%\ar[r, "{T_{\rho[\ol{\gamma := C'}]}\, (\mu T_{\rho[\ol{\gamma :=
%          C'}]})\, \ol{f} \; \circ \; \sigma_{\id_\rho[\ol{\gamma :=
%          g}]}\,(\mu \sigma_{\id_\rho[\ol{\gamma := g}]}) \, \ol{B}}"] 
%& T_{\rho[\ol{\gamma := C'}]}\, (\mu T_{\rho[\ol{\gamma := C'}]})\,
%\ol{B'} \ar[d, "({\mathit{in}_{T_{\rho[\ol{\gamma := C'}]}}})_{\ol{B'}}"]\\
%(\mu T_{\rho[\ol{\gamma := C}]})\, \ol{B}  \ar[r, "{(\mu
%    T_{\rho[\ol{\gamma := C'}]})\, \ol{f} \; \circ \; \mu
%    \sigma_{\id_\rho[\ol{\gamma := g}]} \ol{B}}"] 
%    & (\mu T_{\rho[\ol{\gamma := C'}]})\, \ol{B'} 
\end{tikzcd}\]
Indeed, naturality of $\mathit{in}$ with respect to its functor
argument ensures that the top diagram commutes, and naturality of
$\mathit{in}_{T_{\rho[\ol{\gamma := C'}]}}$ ensures that the bottom
one commutes. 
\item To see that $\setsem{\Gamma;\emptyset \,|\, \emptyset \vdash
  \tin_H : Nat^{\ol{\beta},\ol{\gamma}} \, H[\phi := (\mu \phi.\lambda
    {\overline \alpha}.H){\overline \beta}][\ol{\alpha := \beta}]
  \;(\mu \phi.\lambda {\overline \alpha}.H){\overline
    \beta}}\,\rho\,d$ satisfies the additional property necessary for
  it to be in \[\setsem{\Gamma;\emptyset \vdash
    Nat^{\ol{\beta},\ol{\gamma}} \, H[\phi := (\mu \phi.\lambda
      {\overline \alpha}.H){\overline \beta}][\ol{\alpha := \beta}]
    \;(\mu \phi.\lambda {\overline \alpha}.H){\overline
      \beta}}\,\rho\]
  let $\ol{R : \rel(B,B')}$ and $\ol{S : \rel(C,C')}$. Then
\[\begin{array}{ll}
 & (\,(\setsem{\Gamma;\emptyset \,|\, \emptyset \vdash
  \tin_H : Nat^{\ol{\beta},\ol{\gamma}} \, H[\phi := (\mu \phi.\lambda
    {\overline \alpha}.H){\overline \beta}][\ol{\alpha := \beta}]
  \;(\mu \phi.\lambda {\overline \alpha}.H){\overline
    \beta}}\,\rho\,d)_{\ol{B},\ol{C}},\,\\
 & \hspace*{0.5in}(\setsem{\Gamma;\emptyset \,|\, \emptyset \vdash
  \tin_H : Nat^{\ol{\beta},\ol{\gamma}} \, H[\phi := (\mu \phi.\lambda
    {\overline \alpha}.H){\overline \beta}][\ol{\alpha := \beta}]
  \;(\mu \phi.\lambda {\overline \alpha}.H){\overline
    \beta}}\,\rho\,d)_{\ol{B'},\ol{C'}}\,)\\
=& (\, (\mathit{in}_{T_{\rho[\ol{\gamma := C}]}})_{\ol{B}},
(\mathit{in}_{T_{\rho[\ol{\gamma := C'}]}})_{\ol{B'}}\,)
\end{array}\]
has type
\[\begin{array}{ll}
& (\, \setsem{\Gamma;\ol{\beta},\ol{\gamma} \vdash H[\phi := (\mu
    \phi.\lambda {\overline \alpha}.H){\overline \beta}][\ol{\alpha :=
      \beta}]}\rho[\ol{\beta := B}][\ol{\gamma := C}] \to\\ 
& \hspace*{0.5in}\setsem{\Gamma;\ol{\beta},\ol{\gamma} \vdash (\mu \phi.\lambda
  {\overline \alpha}.H){\overline \beta}} \rho[\ol{\beta :=
    B}][\ol{\gamma := C}],\\
& \hspace*{0.05in}\setsem{\Gamma;\ol{\beta},\ol{\gamma} \vdash H[\phi := (\mu
    \phi.\lambda {\overline \alpha}.H){\overline \beta}][\ol{\alpha :=
      \beta}]}\rho[\ol{\beta := B'}][\ol{\gamma := C'}] \to\\ 
& \hspace*{0.5in}\setsem{\Gamma;\ol{\beta},\ol{\gamma} \vdash (\mu \phi.\lambda
  {\overline \alpha}.H){\overline \beta}} \rho[\ol{\beta :=
    B'}][\ol{\gamma := C'}]\,)\\
= & (\, T_{\rho[\ol{\gamma := C}]}\, (\mu T_{\rho[\ol{\gamma := C}]}) \,
\ol{B} \to (\mu T_{\rho[\ol{\gamma := C}]}) \, \ol{B}, \, 
T_{\rho[\ol{\gamma := C'}]}\, (\mu T_{\rho[\ol{\gamma := C'}]}) \,
\ol{B'} \to (\mu T_{\rho[\ol{\gamma := C'}]}) \,\ol{B'} \, )\\
= & \relsem{\Gamma;\ol{\beta},\ol{\gamma} \vdash H[\phi := (\mu
    \phi.\lambda {\overline \alpha}.H){\overline \beta}][\ol{\alpha :=
      \beta}]}\Eq_\rho[\ol{\beta := R}][\ol{\gamma := S}] \to \\
& \hspace*{0.5in} \relsem{\Gamma;\ol{\beta},\ol{\gamma} \vdash (\mu
  \phi.\lambda \ol{\alpha}.H)\ol{\beta}} \Eq_\rho[\ol{\beta:=
    R}][\ol{\gamma :=S}] 
\end{array}\]

\item The proofs that $\relsem{\Gamma;\emptyset \,|\,
    \emptyset \vdash \tin_H : Nat^{\ol{\beta},\ol{\gamma}} \, H[\phi
      := (\mu \phi.\lambda {\overline \alpha}.H){\overline
        \beta}][\ol{\alpha := \beta}] \;(\mu \phi.\lambda {\overline
      \alpha}.H){\overline \beta}}$ is a natural transformation from
  $\relsem{\Gamma;\emptyset \vdash \emptyset}$ to
  $\relsem{\Gamma;\emptyset \vdash Nat^{\ol{\beta},\ol{\gamma}} \,
    H[\phi := (\mu \phi.\lambda {\overline \alpha}.H){\overline
        \beta}][\ol{\alpha := \beta}] \;(\mu \phi.\lambda {\overline
      \alpha}.H){\overline \beta}}$ and that, for all $\rho : \relenv$
  and $d : \relsem{\Gamma;\emptyset \vdash \emptyset}$,
  \[\relsem{\Gamma;\emptyset \,|\, \emptyset \vdash \tin_H :
    Nat^{\ol{\beta},\ol{\gamma}} \, H[\phi := (\mu \phi.\lambda
      {\overline \alpha}.H){\overline \beta}][\ol{\alpha := \beta}]
    \;(\mu \phi.\lambda {\overline \alpha}.H){\overline
      \beta}}\,\rho\,d\] is a natural transformation from $\lambda
  \ol{R}\,\ol{S}.\,\relsem{\Gamma; \ol{\beta},\ol{\gamma} \vdash
    H[\phi := (\mu \phi.\lambda {\overline \alpha}.H){\overline
        \beta}][\ol{\alpha := \beta}]}\rho[\ol{\beta := R}][\ol{\gamma
      := S}]$ to $\lambda \ol{R}\,\ol{S}.\,\relsem{\Gamma;
    \ol{\beta},\ol{\gamma} \vdash (\mu \phi.\lambda {\overline
      \alpha}.H){\overline \beta}}\rho[\ol{\beta := R}][\ol{\gamma :=
      S}]$, are analogous.
\item Finally, to see that $\pi_i(\relsem{\Gamma;\emptyset \,|\,
  \emptyset \vdash \tin_H : Nat^{\ol{\beta},\ol{\gamma}} \, H[\phi :=
    (\mu \phi.\lambda {\overline \alpha}.H){\overline
      \beta}][\ol{\alpha := \beta}] \;(\mu \phi.\lambda {\overline
    \alpha}.H){\overline \beta}}\,\rho\,d) = \setsem{\Gamma;\emptyset
  \,|\, \emptyset \vdash \tin_H : Nat^{\ol{\beta},\ol{\gamma}} \,
  H[\phi := (\mu \phi.\lambda {\overline \alpha}.H){\overline
      \beta}][\ol{\alpha := \beta}] \;(\mu \phi.\lambda {\overline
    \alpha}.H){\overline \beta}}\,(\pi_i \rho)\,(\pi_i d)$ we first
  note that $d : \relsem{\Gamma; \emptyset \vdash \emptyset}\rho$ and
  $\pi_i d : \setsem{\Gamma; \emptyset \vdash \emptyset} (\pi_i \rho)$
  are uniquely determined. Further, the definition of natural
  transformations in $\rel$ ensures that, for any $\ol{R}$ and
  $\ol{S}$,
\[\begin{split}
 &~(\mathit{in}_{T_{\rho[\ol{\gamma := S}]}})_{\ol{R}}\\
=&~((\mathit{in}_{\pi_1 (T_{\rho[\ol{\gamma := S}]})})_{\ol{\pi_1 R}}, \,
(\mathit{in}_{\pi_2 (T_{\rho[\ol{\gamma := S}]})})_{\ol{\pi_2 R}})\\ 
=&~((\mathit{in}_{T^\set_{(\pi_1 \rho)[\ol{\gamma := \pi_1
        S}]}})_{\ol{\pi_1 R}}, \, (\mathit{in}_{T^\set_{(\pi_2
    \rho)[\ol{\gamma := \pi_2 S}]}})_{\ol{\pi_2 R}})\\
\end{split}\]
Observing that $\pi_1$ and $\pi_2$ are surjective, we therefore have
that
\[\begin{split}
 &~\pi_i(\relsem{\Gamma;\emptyset \,|\,
  \emptyset \vdash \tin_H : Nat^{\ol{\beta},\ol{\gamma}} \, H[\phi :=
    (\mu \phi.\lambda {\overline \alpha}.H){\overline
      \beta}][\ol{\alpha := \beta}] \;(\mu \phi.\lambda {\overline
    \alpha}.H){\overline \beta}}\,\rho\,d)\\
=&~\pi_i (\lambda \ol{R}\,\ol{S}.\, (\mathit{in}_{T_{\rho[\ol{\gamma :=
        S}]}})_{\ol{R}}) \\ 
=&~\lambda \ol{B}\,\ol{C}.\, (\mathit{in}_{T^\set_{(\pi_i
      \rho)[\ol{\gamma := C}]}})_{\ol{B}} \\
=&~\setsem{\Gamma;\emptyset \,|\,
  \emptyset \vdash \tin_H : Nat^{\ol{\beta},\ol{\gamma}} \, H[\phi :=
    (\mu \phi.\lambda {\overline \alpha}.H){\overline
      \beta}][\ol{\alpha := \beta}] \;(\mu \phi.\lambda {\overline
    \alpha}.H){\overline \beta}}\,(\pi_i \rho)\,(\pi_i d)
\end{split}\]
\end{itemize}
\item 
 We now consider $\Gamma; \emptyset~|~\emptyset \vdash \fold^F_H
  : \Nat^\emptyset\;(\Nat^{\ol{\beta}, \ol{\gamma}}\,H[\phi
  :=_{\ol{\beta}} F][\ol{\alpha := \beta}]\,F)\;
  (\Nat^{{\ol{\beta},\ol{\gamma}} }\,(\mu \phi.\lambda \overline
  \alpha.H)\overline \beta \;F)$.
\begin{itemize}
\item To see that $\setsem{\Gamma; \emptyset~|~\emptyset \vdash
  \fold^F_H : \Nat^\emptyset\;(\Nat^{\ol{\beta}, \ol{\gamma}}\,H[\phi
    :=_{\ol{\beta}} F][\ol{\alpha := \beta}]\,F)\;
  (\Nat^{{\ol{\beta},\ol{\gamma}} }\,(\mu \phi.\lambda \overline
  \alpha.H)\overline \beta\,F}$ is a natural transformation from
  $\setsem{\Gamma;\emptyset \vdash \emptyset}$ to
  \[\setsem{\Gamma; \emptyset \vdash
    \Nat^\emptyset\;(\Nat^{\ol{\beta}, \ol{\gamma}}\,H[\phi 
    :=_{\ol{\beta}} F][\ol{\alpha := \beta}]\,F)\;
    (\Nat^{{\ol{\beta},\ol{\gamma}} }\,(\mu \phi.\lambda \overline
    \alpha.H)\overline \beta\,F}\] since the functorial part $\Phi$ of
  the context is empty, we need only show that, for every $\rho :
  \setenv$,
\[\setsem{\Gamma; \emptyset~|~\emptyset \vdash
  \fold^F_H : \Nat^\emptyset\;(\Nat^{\ol{\beta}, \ol{\gamma}}\,H[\phi
    :=_{\ol{\beta}} F][\ol{\alpha := \beta}]\,F)\;
  (\Nat^{{\ol{\beta},\ol{\gamma}} }\,(\mu \phi.\lambda \overline
  \alpha.H)\overline \beta\,F}\,\rho\]
is a morphism in $\set$
  from $\setsem{\Gamma;\emptyset \vdash \emptyset}\rho$ to
\[\setsem{\Gamma; \emptyset \vdash \Nat^\emptyset\;(\Nat^{\ol{\beta},
    \ol{\gamma}} \,H[\phi :=_{\ol{\beta}} F][\ol{\alpha :=
      \beta}]\,F)\; (\Nat^{{\ol{\beta},\ol{\gamma}} }\,(\mu
  \phi.\lambda \overline \alpha.H)\overline \beta\,F}\,\rho\] i.e.,
that, for the unique $d : \setsem{\Gamma;\emptyset \vdash \emptyset}
\rho$,
\[\setsem{\Gamma; \emptyset~|~\emptyset \vdash
  \fold^F_H : \Nat^\emptyset\;(\Nat^{\ol{\beta}, \ol{\gamma}}\,H[\phi
    :=_{\ol{\beta}} F][\ol{\alpha := \beta}]\,F)\;
  (\Nat^{{\ol{\beta},\ol{\gamma}} }\,(\mu \phi.\lambda \overline
  \alpha.H)\overline \beta\,F}\,\rho\,d\] is a morphism from
$\setsem{\Gamma; \emptyset \vdash \Nat^{\ol{\beta},
    \ol{\gamma}}\,H[\phi :=_{\ol{\beta}} F][\ol{\alpha := \beta}]\,F}
\rho$ to $\setsem{\Gamma; \emptyset \vdash
  \Nat^{{\ol{\beta},\ol{\gamma}} }\,(\mu \phi.\lambda \overline
  \alpha.H)\overline \beta\,F}\rho$. For this we show that for every
$\eta : \setsem{\Gamma; \emptyset \vdash \Nat^{\ol{\beta},
    \ol{\gamma}}\,H[\phi :=_{\ol{\beta}} F][\ol{\alpha := \beta}]\,F}
\rho$ we have
\[\begin{array}{ll}
  & \setsem{\Gamma; \emptyset~|~\emptyset \vdash \fold^F_H :
  \Nat^\emptyset\;(\Nat^{\ol{\beta}, \ol{\gamma}}\,H[\phi
    :=_{\ol{\beta}} F][\ol{\alpha := \beta}]\,F)\;
  (\Nat^{{\ol{\beta},\ol{\gamma}} }\,(\mu \phi.\lambda \overline
  \alpha.H)\overline \beta\,F}\,\rho\,d\,\eta\\ : & \setsem{\Gamma;
  \emptyset \vdash \Nat^{{\ol{\beta},\ol{\gamma}} }\,(\mu \phi.\lambda
  \overline \alpha.H)\overline \beta\,F}\,\rho
\end{array}\]
To this end we show that, for any $\ol{B}$ and $\ol{C}$,
\[(\setsem{\Gamma; \emptyset~|~\emptyset \vdash \fold^F_H :
  \Nat^\emptyset\;(\Nat^{\ol{\beta}, \ol{\gamma}}\,H[\phi
    :=_{\ol{\beta}} F][\ol{\alpha := \beta}]\,F)\;
  (\Nat^{{\ol{\beta},\ol{\gamma}} }\,(\mu \phi.\lambda \overline
  \alpha.H)\overline \beta\,F}\,\rho\,d\,\eta)_{\ol{B}\,\ol{C}}\]
is a morphism from 
\[\setsem{\Gamma; \ol{\beta},\ol{\gamma} \vdash (\mu
    \phi.\lambda \overline \alpha.H)\overline \beta}\rho[\ol{\beta :=
    B}][\ol{\gamma := C}]\,=\,(\mu T^\set_{\rho[\ol{\gamma := C}]})
\ol{B}\] to \[\setsem{\Gamma; \ol{\beta},\ol{\gamma} \vdash
  F}\rho[\ol{\beta := B}][\ol{\gamma := C}]
%= (\lambda \ol{A}.\,
%\setsem{\Gamma; \ol{\beta},\ol{\gamma} \vdash
%  F}\rho[\ol{\beta := A}][\ol{\gamma := C}])\,\ol{B}
\] To see this, we use
Equations~\ref{eq:subs-var} and~\ref{eq:subs-const} for the first and
second equalities below, together with weakening, to see that $\eta$
is itself a natural transformation from 
\[\begin{array}{ll}
 & \lambda \ol{B}\,\ol{C}.\,\setsem{\Gamma; \ol{\beta},\ol{\gamma}
  \vdash H[\phi := F][\ol{\alpha := \beta}]}\rho[\ol{\beta :=
    B}][\ol{\gamma := C}]\\ 
= & \lambda
\ol{B}\,\ol{C}.\,\setsem{\Gamma;\ol{\beta},\ol{\gamma},\ol{\alpha}  
  \vdash H[\phi := F]}\rho[\ol{\beta := B}][\ol{\gamma :=
    C}][\ol{\alpha := B}]\\ 
= & \lambda \ol{B}\,\ol{C}.\,\setsem{\Gamma;
  \ol{\beta},\ol{\gamma},\ol{\alpha},\phi \vdash H}\rho
[\ol{\beta := B}][\ol{\gamma := C}][\ol{\alpha := B}]\\
 & \hspace*{0.5in}[\phi := \lambda \ol{A}.\, \setsem{\Gamma;\ol{\beta},
    \ol{\gamma},\ol{\alpha} \vdash F}\rho[\ol{\beta := B}][\ol{\gamma
      := C}][\ol{\alpha := B}][\ol{\beta := A}]]\\ 
= & \lambda \ol{B}\,\ol{C}.\, \setsem{\Gamma;
  \ol{\gamma},\ol{\alpha},\phi \vdash H}\rho[\ol{\gamma :=
    C}][\ol{\alpha := B}][\phi := \lambda \ol{A}.\,
  \setsem{\Gamma;\ol{\beta}, 
    \ol{\gamma}\vdash F}\rho[\ol{\gamma := C}][\ol{\beta := A}]]\\ 
=& \lambda \ol{B}\,\ol{C}.\,T^\set_{\rho[\ol{\gamma:=
      C}]}\,(\lambda \ol{A}. \, \setsem{\Gamma;\ol{\beta},\ol{\gamma} 
  \vdash F}\rho[\ol{\beta := A}][\ol{\gamma := C}]) \, \ol{B}\\
\end{array}\]
to 
\[\lambda \ol{B}\,\ol{C}.\,(\lambda
\ol{A}.\,\setsem{\Gamma;\ol{\beta},\ol{\gamma} 
  \vdash F}\rho[\ol{\beta := A}][\ol{\gamma := C}]) \ol{B}\;\;= \;\;
\lambda \ol{B}\,\ol{C}.\,\setsem{\Gamma;\ol{\beta},\ol{\gamma}
  \vdash F}\rho[\ol{\beta := B}][\ol{\gamma := C}]\] Thus, if $x :
\setsem{\Gamma; \ol{\beta},\ol{\gamma} \vdash (\mu \phi.\lambda
  \overline \alpha.H)\overline \beta}\rho[\ol{\beta := B}][\ol{\gamma
    := C}]\;=\;(\mu T^\set_{\rho[\ol{\gamma := C}]}) \ol{B}$, then
\[\begin{array}{ll}
  & (\setsem{\Gamma; \emptyset~|~\emptyset \vdash \fold^F_H :
  \Nat^\emptyset\;(\Nat^{\ol{\beta}, \ol{\gamma}}\,H[\phi
    :=_{\ol{\beta}} F][\ol{\alpha := \beta}]\,F)\;
  (\Nat^{{\ol{\beta},\ol{\gamma}} }\,(\mu \phi.\lambda \overline
  \alpha.H)\overline \beta\,F}\,\rho\,d\,\eta)_{\ol{B}\,\ol{C}}\,x\\
= & (\mathit{fold}_{T^\set_{\rho[{\gamma := C}]}}\,(\lambda
\ol{A}.\,\eta_{\ol{A}\,\ol{C}}))_{\ol{B}}\,x\\  
:& (\lambda \ol{A}.\,\setsem{\Gamma;\ol{\beta},\ol{\gamma} \vdash
  F}\rho[\ol{\beta := A}][\ol{\gamma := C}])\ol{B}
\end{array}\]
i.e., for each $\ol{B}$ and $\ol{C}$
\[(\setsem{\Gamma; \emptyset~|~\emptyset \vdash \fold^F_H :
  \Nat^\emptyset\;(\Nat^{\ol{\beta}, \ol{\gamma}}\,H[\phi
    :=_{\ol{\beta}} F][\ol{\alpha := \beta}]\,F)\;
  (\Nat^{{\ol{\beta},\ol{\gamma}} }\,(\mu \phi.\lambda \overline
  \alpha.H)\overline \beta\,F}\,\rho\,d\,\eta)_{\ol{B}\,\ol{C}}\] is a
morphism from $(\mu T^\set_{\rho[\ol{\gamma := C}]}) \ol{B}$ to
$\setsem{\Gamma;\ol{\beta},\ol{\gamma} \vdash F}\rho[\ol{\beta :=
    B}][\ol{\gamma := C}]$.

\vspace*{0.1in}

To see that this family of morphisms is natural in $\ol{B}$ and
$\ol{C}$, we observe that the following diagram commutes for all
$\ol{f : B \to B'}$ and $\ol{g : C \to C'}$:
\[\begin{tikzcd}[column sep=2.5in, row sep=0.75in]
(\mu T^\set_{\rho[\ol{\gamma := C}]})\, \ol{B}
\ar[d, "{(\mu \sigma_{\id_\rho[\ol{\gamma := g}]}) \, \ol{B}}"'] \ar[r, 
  "{(\mathit{fold}_{T^\set_{\rho[\ol{\gamma := C}]}}\,(\lambda
    \ol{A}.\,\eta_{\ol{A}\,\ol{C}}))_{\ol{B}}}"] 
& \setsem{\Gamma; \ol{\beta},\ol{\gamma} \vdash F}\rho[\ol{\gamma :=
    C}][\ol{\beta := B}]\ar[d, "{\setsem{\Gamma;
      \ol{\beta},\ol{\gamma} \vdash F}\id_\rho[\ol{\gamma := 
    g}][\ol{\beta := \id_B}]}"]\\
(\mu T^\set_{\rho[\ol{\gamma := C'}]})\, \ol{B} 
\ar[r,"{(\mathit{fold}_{T^\set_{\rho[\ol{\gamma :=
            C'}]}}\,(\lambda \ol{A}.\,\eta_{\ol{A}\,\ol{C'}}))_{\ol{B}}\,}" ] 
\ar[d, "{(\mu T^\set_{\rho[\ol{\gamma := C'}]}) \, \ol{f}}"'] & 
 \setsem{\Gamma; \ol{\beta},\ol{\gamma} \vdash F}\rho[\ol{\gamma :=
    C'}][\ol{\beta := B}]\ar[d, "{\setsem{\Gamma;
      \ol{\beta},\ol{\gamma} \vdash F}\id_\rho[\ol{\gamma := 
    \id_{C'}}][\ol{\beta := f}]}"]\\
 (\mu T^\set_{\rho[\ol{\gamma := C'}]})\, \ol{B'} 
\ar[r,"{(\mathit{fold}_{T^\set_{\rho[\ol{\gamma :=
            C'}]}}\,(\lambda \ol{A}.\,\eta_{\ol{A}\,\ol{C'}}))_{\ol{B'}}\,}"
] &  
 \setsem{\Gamma; \ol{\beta},\ol{\gamma} \vdash F}\rho[\ol{\gamma :=
    C'}][\ol{\beta := B'}]
\end{tikzcd}\]
Indeed, naturality of $\mathit{fold}_{T^\set_{\rho[\ol{\gamma :=
        C'}]}}\,(\lambda \ol{A}.\,\eta_{\ol{A}\,\ol{C'}})$ ensures that
the bottom diagram commutes. To see that the top one commutes is
considerably more delicate.

\vspace*{0.1in}

To see that the top diagram commutes we first observe that, given a
natural transformation $\Theta : H \to K : [\set^k, \set] \to [\set^k,
  \set]$, the fixpoint natural transformation $\mu \Theta : \mu H \to
\mu K : \set^k \to \set$ is defined to be
$\textit{fold}_{H}(\Theta\,(\mu K) \circ \textit{in}_{K})$, i.e., the
unique morphism making the following square commute:\label{page:dia1}
\[
\begin{tikzcd}[column sep = large]
H(\mu H)
	\ar[dd, "{\textit{in}_H}"']
	\ar[r, "{H(\mu \Theta)}"]
& H(\mu K)
	\ar[d, "{\Theta (\mu K)}"] \\
& K(\mu K)
	\ar[d, "{\textit{in}_K}"] \\
\mu H
	\ar[r, "{\mu \Theta}"']
& \mu K
\end{tikzcd}
\]
Taking $\Theta = \sigma^{\set}_{f}: T^{\set}_{\rho} \to
T^{\set}_{\rho'}$ gives that the following diagram commutes for any
morphism of set environments $f : \rho \to \rho'$:
\begin{equation}\label{eq:mu-sigma-def}
\begin{tikzcd}[column sep = large]
T^{\set}_{\rho}(\mu T^{\set}_{\rho})
	\ar[dd, "{\textit{in}_{T^{\set}_{\rho}}}"']
	\ar[r, "{T^{\set}_{\rho}(\mu \sigma^{\set}_{f})}"]
& T^{\set}_{\rho}(\mu T^{\set}_{\rho'})
	\ar[d, "{\sigma^{\set}_{f} (\mu T^{\set}_{\rho'})}"] \\
& T^{\set}_{\rho'}(\mu T^{\set}_{\rho'})
	\ar[d, "{\textit{in}_{T^{\set}_{\rho'}}}"] \\
\mu T^{\set}_{\rho}
	\ar[r, "{\mu \sigma^{\set}_{f}}"']
& \mu T^{\set}_{\rho'}
\end{tikzcd}
\end{equation}

We next observe that the action of the functor
\begin{equation*}
\lambda \ol{B}. \lambda \ol{C}. \setsem{\Gamma; \ol{\beta},
  \ol{\gamma} \vdash H[\phi := F][\ol{\alpha := \beta}]}\rho
        [\ol{\beta := B}] [\ol{\gamma := C}]
\end{equation*}
on the morphisms $\ol{f : B \to B'}, \ol{g : C \to C'}$ is given by
\[\begin{array}{ll}
 & \setsem{\Gamma; \ol{\beta}, \ol{\gamma} \vdash H[\phi :=
      F][\ol{\alpha := \beta}]} \id_{\rho} [\ol{\beta := f}]
           [\ol{\gamma := g}]\\
= & \setsem{\Gamma; \ol{\alpha}, \ol{\gamma} \vdash H[\phi :=
      F]} \id_{\rho} [\ol{\alpha := f}][\ol{\gamma := g}]\\
           = & \setsem{\Gamma; \phi, \ol{\alpha}, \ol{\gamma} \vdash H}
\id_{\rho}[\ol{\alpha := f}] [\ol{\gamma := g}][\phi := \lambda
  \ol{A}. \setsem{\Gamma; \ol{\beta}, \ol{\gamma} \vdash F} \id_{\rho
    [\ol{\beta := A}]} [\ol{\gamma := g}]] \\ 
=& \setsem{\Gamma; \phi, \ol{\alpha}, \ol{\gamma} \vdash H} (\id_{\rho
  [\ol{\gamma := C'}] [\phi := \lambda \ol{A}. \setsem{\Gamma;
      \ol{\beta}, \ol{\gamma} \vdash F} \rho [\ol{\beta := A}]
    [\ol{\gamma := C'}]]} [\ol{\alpha := f}] \\
&\hspace{3em} \circ \id_{\rho [\ol{\alpha := B}] [\phi := \lambda
    \ol{A}. \setsem{\Gamma; \ol{\beta}, \ol{\gamma} \vdash F} \rho
       [\ol{\beta := A}] [\ol{\gamma := C'}]]} [\ol{\gamma := g}] \\
&\hspace{3em} \circ \id_{\rho [\ol{\alpha := B}] [\ol{\gamma := C}]}
[\phi := \lambda \ol{A}. \setsem{\Gamma; \ol{\beta}, \ol{\gamma}
    \vdash F} \id_{\rho [\ol{\beta := A}]}[\ol{\gamma := g}]] ) \\
=& \setsem{\Gamma; \phi, \ol{\alpha}, \ol{\gamma} \vdash H} \id_{\rho
  [\ol{\gamma := C'}][\phi := \lambda \ol{A}. \setsem{\Gamma;
      \ol{\beta}, \ol{\gamma} \vdash F} \rho [\ol{\beta := A}]
    [\ol{\gamma := C'}]]} [\ol{\alpha := f}] \\
&\hspace{3em} \circ \setsem{\Gamma; \phi, \ol{\alpha}, \ol{\gamma}
  \vdash H} \id_{\rho [\ol{\alpha := B}][\phi := \lambda
    \ol{A}. \setsem{\Gamma; \ol{\beta}, \ol{\gamma} \vdash F} \rho
       [\ol{\beta := A}] [\ol{\gamma := C'}]]} [\ol{\gamma := g}] \\
&\hspace{3em} \circ \setsem{\Gamma; \phi, \ol{\alpha}, \ol{\gamma}
  \vdash H} \id_{\rho [\ol{\alpha := B}] [\ol{\gamma := C}]} [\phi :=
  \lambda \ol{A}. \setsem{\Gamma; \ol{\beta}, \ol{\gamma} \vdash F}
  \id_{\rho [\ol{\beta := A}]} [\ol{\gamma := g}]] \\
= & T^{\set}_{\rho [\ol{\gamma := C'}]} (\lambda
\ol{A}. \setsem{\Gamma; \ol{\beta}, \ol{\gamma} \vdash F} \rho
   [\ol{\beta := A}] [\ol{\gamma := C'}]) \ol{f} \\
&\hspace{3em} \circ \big( \sigma^{\set}_{\id_{\rho}[\ol{\gamma := g}]}
   (\lambda \ol{A}. \setsem{\Gamma; \ol{\beta}, \ol{\gamma} \vdash F}
   \rho [\ol{\beta := A}] [\ol{\gamma := C'}]) \big)_{\ol{B}} \\
&\hspace{3em} \circ \big( T^{\set}_{\rho [\ol{\gamma := C}]} (\lambda
   \ol{A}. \setsem{\Gamma; \ol{\beta}, \ol{\gamma} \vdash F} \id_{\rho
     [\ol{\beta := A}]} [\ol{\gamma := g}]) \big)_{\ol{B}}
\end{array}\]
So, if $\eta$ is a natural transformation from
\[\lambda \ol{B}\,\ol{C}. \setsem{\Gamma; \ol{\alpha},
  \ol{\gamma} \vdash H[\phi := F][\ol{\alpha := \beta}]}\rho
          [\ol{\beta := B}] [\ol{\gamma := C}]\]
          to
\[\lambda \ol{B}\, \ol{C}. \setsem{\Gamma; \ol{\beta},
   \ol{\gamma} \vdash F}\rho [\ol{\beta := B}] [\ol{\gamma := C}]\]
then, by naturality,
\[\begin{array}{ll}
 & \setsem{\Gamma; \ol{\beta}, \ol{\gamma} \vdash F} \id_{\rho}
       [\ol{\beta := f}] [\ol{\gamma := g}] \circ \eta_{\ol{B},
         \ol{C}} \\ 
= & \eta_{\ol{B'}, \ol{C'}} \circ \setsem{\Gamma; \ol{\alpha},
  \ol{\gamma} \vdash H[\phi := F][\ol{\alpha := \beta}]} \id_{\rho}
[\ol{\beta := f}] [\ol{\gamma := g}] \\ 
= & \eta_{\ol{B'}, \ol{C'}} \circ T^{\set}_{\rho [\ol{\gamma := C'}]}
(\lambda \ol{A}. \setsem{\Gamma; \ol{\beta}, \ol{\gamma} \vdash F}
\rho [\ol{\beta := A}] [\ol{\gamma := C'}]) \ol{f} \\
& \circ \big( \sigma^{\set}_{\id_{\rho}[\ol{\gamma := g}]} (\lambda
\ol{A}. \setsem{\Gamma; \ol{\beta}, \ol{\gamma} \vdash F} \rho
   [\ol{\beta := A}] [\ol{\gamma := C'}]) \big)_{\ol{B}} \\
& \circ \big( T^{\set}_{\rho [\ol{\gamma := C}]} (\lambda
   \ol{A}. \setsem{\Gamma; \ol{\beta}, \ol{\gamma} \vdash F} \id_{\rho
     [\ol{\beta := A}]} [\ol{\gamma := g}]) \big)_{\ol{B}}
\end{array}\]
As a special case when $\ol{f = \id_B}$ we have
\[\begin{array}{ll}
 & \setsem{\Gamma; \ol{\beta}, \ol{\gamma} \vdash F} \id_{\rho [\ol{\beta
      := B}]} [\ol{\gamma := g}] \circ \eta_{\ol{B}, \ol{C}} \\ 
= & \eta_{\ol{B}, \ol{C'}} \circ \big(
\sigma^{\set}_{\id_{\rho}[\ol{\gamma := g}]} (\lambda
\ol{A}. \setsem{\Gamma; \ol{\beta}, \ol{\gamma} \vdash F} \rho
   [\ol{\beta := A}] [\ol{\gamma := C'}]) \big)_{\ol{B}}\\
 & \hspace*{0.32in} \circ \big(
   T^{\set}_{\rho [\ol{\gamma := C}]} (\lambda \ol{A}. \setsem{\Gamma;
     \ol{\beta}, \ol{\gamma} \vdash F} \id_{\rho [\ol{\beta := A}]}
   [\ol{\gamma := g}]) \big)_{\ol{B}}
\end{array}\]
i.e., 
\begin{equation}\label{eq:T-sigma-functor}
\begin{split}
  & \lambda \ol{B}. \setsem{\Gamma; \ol{\beta}, \ol{\gamma} \vdash F}
\id_{\rho [\ol{\beta := B}]} [\ol{\gamma := g}] \circ \lambda
\ol{B}.\eta_{\ol{B}, \ol{C}} \\  
=\;\;\; & \lambda \ol{B}.\eta_{\ol{B}, \ol{C'}} \circ
\sigma^{\set}_{\id_{\rho}[\ol{\gamma := g}]} (\lambda
\ol{A}. \setsem{\Gamma; \ol{\beta}, \ol{\gamma} \vdash F} \rho
   [\ol{\beta := A}] [\ol{\gamma := C'}]) \\
 & \hspace*{0.5in} \circ
   T^{\set}_{\rho [\ol{\gamma := C}]} (\lambda \ol{A}. \setsem{\Gamma;
     \ol{\beta}, \ol{\gamma} \vdash F} \id_{\rho [\ol{\beta := A}]}
   [\ol{\gamma := g}]) 
\end{split}
\end{equation}
Now, to see that the top diagram in the diagram on
page~\pageref{page:dia1} commutes we first note that the diagram
\[\begin{tikzcd}[column sep = huge, row sep = huge]
T^{\set}_{\rho [\ol{\gamma := C}]} (\mu T^{\set}_{\rho [\ol{\gamma :=
      C}]}) \ar[rr, "{T^{\set}_{\rho [\ol{\gamma := C}]} (
    \fold_{T^{\set}_{\rho [\ol{\gamma := C'}]}} (\lambda
    \ol{A}. \eta_{\ol{A}, \ol{C'}}) \circ \mu
    \sigma^{\set}_{\id_{\rho}[\ol{\gamma := g}]} )}"] \ar[dd,
  "{\tin_{T^{\set}_{\rho [\ol{\gamma := C}]}}}"']
&& T^{\set}_{\rho [\ol{\gamma := C}]} (\lambda \ol{B}. \setsem{\Gamma;
  \ol{\beta}, \ol{\gamma} \vdash F} \rho [\ol{\beta := B}] [\ol{\gamma
    := C'}]) \ar[d, "{ \sigma^{\set}_{\id_{\rho}[\ol{\gamma := g}]}
    (\lambda \ol{B}. \setsem{\Gamma; \ol{\beta}, \ol{\gamma} \vdash F}
    \rho [\ol{\beta := B}] [\ol{\gamma := C'}]) }" description] \\
&& T^{\set}_{\rho [\ol{\gamma := C'}]} (\lambda
\ol{B}. \setsem{\Gamma; \ol{\beta}, \ol{\gamma} \vdash F} \rho
   [\ol{\beta := B}][\ol{\gamma := C'}]) \ar[d, "{ \lambda \ol{A}. \eta_{\ol{A}, \ol{C'}}
     }" description] \\
\mu T^{\set}_{\rho [\ol{\gamma := C}]}\ar[r, "{\mu
    \sigma^{\set}_{\id_{\rho}[\ol{\gamma := g}]}}"'] 
&\mu T^{\set}_{\rho [\ol{\gamma := C'}]} \ar[r,
  "{\fold_{T^{\set}_{\rho [\ol{\gamma := C'}]}} (\lambda
    \ol{A}. \eta_{\ol{A}, \ol{C'}})}"']
& \lambda \ol{B}. \setsem{\Gamma; \ol{\beta}, \ol{\gamma} \vdash F}
\rho [\ol{\beta := B}] [\ol{\gamma := C'}]
\end{tikzcd}\]
commutes because
\[\begin{array}{ll}
& \lambda \ol{A}. \eta_{\ol{A}, \ol{C'}} \circ
\sigma^{\set}_{\id_{\rho}[\ol{\gamma := g}]} (\lambda
\ol{B}. \setsem{\Gamma; \ol{\beta}, \ol{\gamma} \vdash F} \rho
   [\ol{\beta := B}] [\ol{\gamma := C'}])\\
& \hspace{3.6em}   \circ T^{\set}_{\rho
     [\ol{\gamma := C}]} ( \fold_{T^{\set}_{\rho [\ol{\gamma := C'}]}}
   (\lambda \ol{A}. \eta_{\ol{A}, \ol{C'}}) \circ \mu
   \sigma^{\set}_{\id_{\rho}[\ol{\gamma := g}]} ) \\
= & \lambda \ol{A}. \eta_{\ol{A}, \ol{C'}} \circ
\sigma^{\set}_{\id_{\rho}[\ol{\gamma := g}]} (\lambda
\ol{B}. \setsem{\Gamma; \ol{\beta}, \ol{\gamma} \vdash F} \rho
   [\ol{\beta := B}] [\ol{\gamma := C'}]) \\
&\hspace{3.6em} \circ T^{\set}_{\rho [\ol{\gamma := C}]} (
   \fold_{T^{\set}_{\rho [\ol{\gamma := C'}]}} (\lambda
   \ol{A}. \eta_{\ol{A}, \ol{C'}}) ) \circ T^{\set}_{\rho [\ol{\gamma
         := C}]} ( \mu \sigma^{\set}_{\id_{\rho}[\ol{\gamma := g}]} )  \\
= & \lambda \ol{A}. \eta_{\ol{A}, \ol{C'}} \circ T^{\set}_{\rho
     [\ol{\gamma := C'}]} ( \fold_{T^{\set}_{\rho [\ol{\gamma :=
           C'}]}} (\lambda \ol{A}. \eta_{\ol{A}, \ol{C'}}) ) \circ
   \sigma^{\set}_{\id_{\rho}[\ol{\gamma := g}]} (\mu T^{\set}_{\rho
     [\ol{\gamma := C'}]}) \circ T^{\set}_{\rho [\ol{\gamma := C}]} (
   \mu \sigma^{\set}_{\id_{\rho}[\ol{\gamma := g}]} ) \\
= & \fold_{T^{\set}_{\rho [\ol{\gamma := C'}]}} (\lambda
\ol{A}. \eta_{\ol{A}, \ol{C'}}) \circ \tin_{T^{\set}_{\rho [\ol{\gamma
        := C'}]}} \circ \sigma^{\set}_{\id_{\rho}[\ol{\gamma := g}]}
(\mu T^{\set}_{\rho [\ol{\gamma := C'}]}) \circ T^{\set}_{\rho
  [\ol{\gamma := C}]} ( \mu \sigma^{\set}_{\id_{\rho}[\ol{\gamma :=
      g}]} ) \\
= &\fold_{T^{\set}_{\rho [\ol{\gamma := C'}]}} (\lambda
\ol{A}. \eta_{\ol{A}, \ol{C'}}) \circ \mu
\sigma^{\set}_{\id_{\rho}[\ol{\gamma := g}]} \circ
\tin_{T^{\set}_{\rho [\ol{\gamma := C}]}}
\end{array}\]
Here, the first equality is by functoriality of $T^{\set}_{\rho
  [\ol{\gamma := C}]}$, the second equality is by naturality of
$\sigma^{\set}_{\id_{\rho}[\ol{\gamma := g}]}$, the third equality by
the universal property of $\fold_{T^{\set}_{\rho [\ol{\gamma := C'}]}}
(\lambda \ol{A}. \eta_{\ol{A}, \ol{C'}})$ and the last equality by
Equation~\ref{eq:mu-sigma-def}. That is, we have
\begin{equation}\label{eq:one-side}
\begin{split}
 & \fold_{T^{\set}_{\rho [\ol{\gamma := C'}]}} (\lambda
\ol{A}. \eta_{\ol{A}, \ol{C'}}) \circ \mu
\sigma^{\set}_{\id_{\rho}[\ol{\gamma := g}]}\\
=\;\;\; & \fold_{T^{\set}_{\rho [\ol{\gamma := C}]}} ( \lambda
\ol{A}. \eta_{\ol{A}, \ol{C'}} \circ
\sigma^{\set}_{\id_{\rho}[\ol{\gamma := g}]} (\lambda
\ol{B}. \setsem{\Gamma; \ol{\beta}, \ol{\gamma} \vdash F} \rho
   [\ol{\beta := B}] [\ol{\gamma := C'}]) )
\end{split}
\end{equation}
Next, we note that the diagram
\[\begin{tikzcd}[row sep = huge]
T^{\set}_{\rho [\ol{\gamma := C}]} (\mu T^{\set}_{\rho [\ol{\gamma :=
      C}]}) \ar[rr, "{T^{\set}_{\rho [\ol{\gamma := C}]} (
    \lambda \ol{B}.\setsem{\Gamma; \ol{\beta}, \ol{\gamma} \vdash F} \id_{\rho
      [\ol{\beta := B}]} [\ol{\gamma := g}] \circ
    \fold_{T^{\set}_{\rho [\ol{\gamma := C}]}} (\lambda
    \ol{A}. \eta_{\ol{A}, \ol{C}}) )}"] \ar[dd, "{\tin_{T^{\set}_{\rho
        [\ol{\gamma := C}]}}}"']
&& T^{\set}_{\rho [\ol{\gamma := C}]} (\lambda \ol{B}.\setsem{\Gamma; \ol{\beta},
  \ol{\gamma} \vdash F} \rho [\ol{\beta := B}] [\ol{\gamma := C'}])
\ar[d, "{ \sigma^{\set}_{\id_{\rho}[\ol{\gamma := g}]}
    (\lambda \ol{B}.\setsem{\Gamma; \ol{\beta}, \ol{\gamma} \vdash F} \rho [\ol{\beta
        := B}] [\ol{\gamma := C'}]) }" description]\\
&& T^{\set}_{\rho [\ol{\gamma := C'}]} (\lambda \ol{B}.\setsem{\Gamma; \ol{\beta},
  \ol{\gamma} \vdash F} \rho [\ol{\beta := B}][\ol{\gamma := C'}]) \ar[d, "{ \lambda
    \ol{A}. \eta_{\ol{A}, \ol{C'}} }" description] \\
\mu T^{\set}_{\rho [\ol{\gamma := C}]} \ar[r, bend right = 10,
  "{\fold_{T^{\set}_{\rho [\ol{\gamma := C}]}} (\lambda
    \ol{A}. \eta_{\ol{A}, \ol{C}})}"'] & \lambda \ol{B}.\setsem{\Gamma; \ol{\beta},
  \ol{\gamma} \vdash F} \rho [\ol{\beta := B}] [\ol{\gamma := C}]
\ar[r, bend right = 10, "{\lambda \ol{B}.\setsem{\Gamma; \ol{\beta}, \ol{\gamma}
      \vdash F} \id_{\rho [\ol{\beta := B}]} [\ol{\gamma := g}]}"'] &
\lambda \ol{B}.\setsem{\Gamma; \ol{\beta}, \ol{\gamma} \vdash F} \rho [\ol{\beta :=
    B}] [\ol{\gamma := C'}]
\end{tikzcd}
\]
commutes because
\begin{align*}
& \lambda \ol{A}. \eta_{\ol{A}, \ol{C'}} \circ
  \sigma^{\set}_{\id_{\rho}[\ol{\gamma := g}]} (\lambda
  \ol{B}.\setsem{\Gamma; \ol{\beta}, \ol{\gamma} \vdash F} \rho
     [\ol{\beta := B}] [\ol{\gamma := C'}]) \\
&\hspace{3em} \circ T^{\set}_{\rho [\ol{\gamma := C}]} ( \lambda
     \ol{B}. \setsem{\Gamma; \ol{\beta}, \ol{\gamma} \vdash F}
     \id_{\rho [\ol{\beta := B}]} [\ol{\gamma := g}] \circ
     \fold_{T^{\set}_{\rho [\ol{\gamma := C}]}} (\lambda
     \ol{A}. \eta_{\ol{A}, \ol{C}}) ) \\
&= \lambda \ol{A}. \eta_{\ol{A}, \ol{C'}} \circ
     \sigma^{\set}_{\id_{\rho}[\ol{\gamma := g}]} (\lambda
     \ol{B}. \setsem{\Gamma; \ol{\beta}, \ol{\gamma} \vdash F} \rho
        [\ol{\beta := B}] [\ol{\gamma := C'}]) \\
&\hspace{3em} \circ T^{\set}_{\rho [\ol{\gamma := C}]} ( \lambda
        \ol{B}. \setsem{\Gamma; \ol{\beta}, \ol{\gamma} \vdash F}
        \id_{\rho [\ol{\beta := B}]} [\ol{\gamma := g}] ) \circ
        T^{\set}_{\rho [\ol{\gamma := C}]} ( \fold_{T^{\set}_{\rho
            [\ol{\gamma := C}]}} (\lambda \ol{A}. \eta_{\ol{A},
          \ol{C}}) ) \\
 &= \lambda \ol{B}. \setsem{\Gamma; \ol{\beta}, \ol{\gamma} \vdash F}
        \id_{\rho [\ol{\beta := B}]} [\ol{\gamma := g}] \circ \lambda
        \ol{A}. \eta_{\ol{A}, \ol{C}} \circ T^{\set}_{\rho [\ol{\gamma
              := C}]} ( \fold_{T^{\set}_{\rho [\ol{\gamma := C}]}}
        (\lambda \ol{A}. \eta_{\ol{A}, \ol{C}}) ) \\
&= \lambda \ol{B}. \setsem{\Gamma; \ol{\beta}, \ol{\gamma} \vdash F}
        \id_{\rho [\ol{\beta := B}]} [\ol{\gamma := g}] \circ
        \fold_{T^{\set}_{\rho [\ol{\gamma := C}]}} (\lambda
        \ol{A}. \eta_{\ol{A}, \ol{C}}) \circ \tin_{T^{\set}_{\rho
            [\ol{\gamma := C}]}}
\end{align*}
Here, the first equality is by functoriality of $T^{\set}_{\rho
  [\ol{\gamma := C}]}$, the second equality is by
Equation~\ref{eq:T-sigma-functor}, and the last equality is by the
universal property of $\fold_{T^{\set}_{\rho [\ol{\gamma := C}]}}
(\lambda \ol{A}. \eta_{\ol{A}, \ol{C}})$. That is, we have
\begin{equation}\label{eq:other-side}
\begin{split}
  & \lambda \ol{B}. \setsem{\Gamma; \ol{\beta}, \ol{\gamma} \vdash F}
\id_{\rho [\ol{\beta := B}]} [\ol{\gamma := g}] \circ
\fold_{T^{\set}_{\rho [\ol{\gamma := C}]}} (\lambda
\ol{A}. \eta_{\ol{A}, \ol{C}}) \\ 
= \;\;\;& \fold_{T^{\set}_{\rho [\ol{\gamma := C}]}} (\lambda
\ol{A}. \eta_{\ol{A}, \ol{C'}} \circ
\sigma^{\set}_{\id_{\rho}[\ol{\gamma := g}]} (\lambda
\ol{B}. \setsem{\Gamma; \ol{\beta}, \ol{\gamma} \vdash F} \rho
   [\ol{\beta := B}] [\ol{\gamma := C'}]) ) 
\end{split}
\end{equation}
Combining Equations~\ref{eq:one-side} and~\ref{eq:other-side} we get
that
\[\fold_{T^{\set}_{\rho [\ol{\gamma := C'}]}} (\lambda \ol{A}. \eta_{\ol{A}, \ol{C'}})
  \circ \mu \sigma^{\set}_{\id_{\rho}[\ol{\gamma := g}]} = \lambda
  \ol{B}. \setsem{\Gamma; \ol{\beta}, \ol{\gamma} \vdash F} \id_{\rho
    [\ol{\beta := B}]} [\ol{\gamma := g}] \circ \fold_{T^{\set}_{\rho
      [\ol{\gamma := C}]}} (\lambda \ol{A}. \eta_{\ol{A}, \ol{C}})\]
  i.e., that the top diagram in the diagram on
  page~\pageref{page:dia1} commutes. We therefore have that
\[(\setsem{\Gamma; \emptyset~|~\emptyset
  \vdash \fold^F_H : \Nat^\emptyset\;(\Nat^{\ol{\beta},
    \ol{\gamma}}\,H[\phi :=_{\ol{\beta}} F][\ol{\alpha :=
      \beta}]\,F)\; (\Nat^{{\ol{\beta},\ol{\gamma}} }\,(\mu
  \phi.\lambda \overline \alpha.H)\overline
  \beta\,F}\,\rho\,d\,\eta)_{\ol{B}\,\ol{C}}\] is natural in $\ol{B}$
and $\ol{C}$ as desired.
\item To see that, for every $\rho : \setenv$, $d \in \setsem{\Gamma;
  \emptyset \vdash \emptyset}\rho$, and $\eta : \setsem{\Gamma; \emptyset
  \vdash \Nat^{\ol{\beta}, \ol{\gamma}}\,H[\phi :=_{\ol{\beta}}
    F][\ol{\alpha := \beta}]\,F} \rho$,
\[\setsem{\Gamma; \emptyset~|~\emptyset
  \vdash \fold^F_H : \Nat^\emptyset\;(\Nat^{\ol{\beta},
    \ol{\gamma}}\,H[\phi :=_{\ol{\beta}} F][\ol{\alpha :=
      \beta}]\,F)\; (\Nat^{{\ol{\beta},\ol{\gamma}} }\,(\mu
  \phi.\lambda \overline \alpha.H)\overline \beta\,F}\,\rho\,d\,\eta\]
%\[\setsem{\Gamma;\emptyset \,|\, \Delta \vdash \fold^F_H\, t :
%    \Nat^{{\ol{\beta},\ol{\gamma}} }\,(\mu \phi.\lambda \overline
%    \alpha.H)\overline \beta\;F}\,\rho\,d\]
satisfies the additional condition necessary for it to be in
$\setsem{\Gamma;\emptyset \vdash \Nat^{{\ol{\beta},\ol{\gamma}}
  }\,(\mu \phi.\lambda \overline \alpha.H)\overline \beta\;F}\,\rho$,
let $\ol{R : \rel(B,B')}$ and $\ol{S : \rel(C,C')}$.  Since $\eta$
satisfies the additional condition necessary for it to be in
$\setsem{\Gamma; \emptyset \vdash \Nat^{\ol{\beta},
    \ol{\gamma}}\,(H[\phi := F][\ol{\alpha := \beta}])\,F} \rho$ ---
i.e., since 
\[\begin{array}{lll}
 (\eta_{\ol{B}\,\ol{C}}\,,
\eta_{\ol{B'}\,\ol{C'}}) & \in & 
\relsem{\Gamma;\ol{\gamma},\ol{\beta} \vdash H[\phi := F][\ol{\alpha
      := \beta}]}\Eq_\rho[\ol{\gamma := S}][\ol{\beta := R}] \to\\
 &  & \hspace*{1in}
\relsem{\Gamma;\ol{\gamma},\ol{\beta} \vdash F}\Eq_\rho[\ol{\gamma
    := S}][\ol{\beta := R}]\\
& = & T_{\Eq_\rho[\ol{\gamma := S}]}\;
(\relsem{\Gamma;\ol{\gamma},\ol{\beta} \vdash F}\Eq_\rho[\ol{\gamma :=
    S}][\ol{\beta := R}]) \to\\ 
&  & \hspace*{1in}
\relsem{\Gamma;\ol{\gamma},\ol{\beta} \vdash F}\Eq_\rho[\ol{\gamma
    := S}][\ol{\beta := R}]
\end{array}\]
--- we have that
\[\begin{array}{ll}
 & (\,(\mathit{fold}_{T^\set_{\rho[\ol{\gamma :=
        C}]}}\,(\lambda \ol{A}.\,\eta_{\ol{A}\,\ol{C}}))_{\ol{B}},\,
(\mathit{fold}_{T^\set_{\rho[\ol{\gamma :=
        C'}]}}\,(\lambda \ol{A}.\eta_{\ol{A}\,\ol{C'}}))_{\ol{B'}}\,) 
\end{array}\]
has type
\[\begin{array}{ll}
  & (\mu T_{\Eq_\rho[\ol{\gamma := S}]}) \,\ol{R} \to
\relsem{\Gamma;\ol{\gamma},\ol{\beta} \vdash F}\Eq_\rho[\ol{\gamma := 
    S}][\ol{\beta:= R}]\\ 
= & (\mu T_{\Eq_\rho[\ol{\gamma :=
      S}]})\,\ol{\relsem{\Gamma;\ol{\gamma},\ol{\beta} 
  \vdash \beta}\Eq_\rho[\ol{\gamma := S}][\ol{\beta := R}]} \to
\relsem{\Gamma;\ol{\gamma},\ol{\beta} \vdash F}\Eq_\rho[\ol{\gamma := 
    S}][\ol{\beta:= R}]\\ 
= & \relsem{\Gamma; \ol{\gamma},\ol{\beta} \vdash (\mu \phi. \lambda
  \ol{\alpha}. H)\ol{\beta}}\Eq_\rho[\ol{\gamma := S}][\ol{\beta := R}] \to
\relsem{\Gamma;\ol{\gamma},\ol{\beta} \vdash F}\Eq_\rho[\ol{\gamma := 
    S}][\ol{\beta:= R}]
\end{array}\]
as desired.
\item The proofs that
\[\relsem{\Gamma; \emptyset~|~\emptyset \vdash
  \fold^F_H : \Nat^\emptyset\;(\Nat^{\ol{\beta}, \ol{\gamma}}\,H[\phi
    :=_{\ol{\beta}} F][\ol{\alpha := \beta}]\,F)\;
  (\Nat^{{\ol{\beta},\ol{\gamma}} }\,(\mu \phi.\lambda \overline
  \alpha.H)\overline \beta\,F)}\] is a natural transformation from
  $\relsem{\Gamma;\emptyset \vdash \emptyset}$ to
  \[\relsem{\Gamma ; \emptyset \vdash
    \Nat^\emptyset\;(\Nat^{\ol{\beta}, \ol{\gamma}}\,H[\phi 
    :=_{\ol{\beta}} F][\ol{\alpha := \beta}]\,F)\;
    (\Nat^{{\ol{\beta},\ol{\gamma}} }\,(\mu \phi.\lambda \overline
    \alpha.H)\overline \beta\,F)}\] and that, for all $\rho : \relenv$
  and the unique $d : \relsem{\Gamma;\emptyset \vdash \emptyset}\rho$,
\[\relsem{\Gamma; \emptyset~|~\emptyset \vdash
  \fold^F_H : \Nat^\emptyset\;(\Nat^{\ol{\beta}, \ol{\gamma}}\,H[\phi
    :=_{\ol{\beta}} F][\ol{\alpha := \beta}]\,F)\;
  (\Nat^{{\ol{\beta},\ol{\gamma}} }\,(\mu \phi.\lambda \overline
  \alpha.H)\overline \beta\,F)}\,\rho\,d\] is a morphism from
$\relsem{\Gamma; \emptyset \vdash \Nat^{\ol{\beta},
    \ol{\gamma}}\,H[\phi :=_{\ol{\beta}} F][\ol{\alpha := \beta}]\,F)}
\rho$ to $\relsem{\Gamma; \emptyset \vdash
  \Nat^{{\ol{\beta},\ol{\gamma}} }\,(\mu \phi.\lambda \overline
  \alpha.H)\overline \beta\,F}\rho$, are analogous.
\item Finally, to see that
\[\begin{array}{ll}
& \pi_i(\relsem{\Gamma; \emptyset~|~\emptyset \vdash
  \fold^F_H : \Nat^\emptyset\;(\Nat^{\ol{\beta}, \ol{\gamma}}\,H[\phi
    :=_{\ol{\beta}} F][\ol{\alpha := \beta}]\,F)\;
  (\Nat^{{\ol{\beta},\ol{\gamma}} }\,(\mu \phi.\lambda \overline
  \alpha.H)\overline \beta\,F)} \rho)\\
= & \setsem{\Gamma; \emptyset~|~\emptyset \vdash
  \fold^F_H : \Nat^\emptyset\;(\Nat^{\ol{\beta}, \ol{\gamma}}\,H[\phi
    :=_{\ol{\beta}} F][\ol{\alpha := \beta}]\,F)\;
  (\Nat^{{\ol{\beta},\ol{\gamma}} }\,(\mu \phi.\lambda \overline
  \alpha.H)\overline \beta\,F)}(\pi_i\rho)
\end{array}\]
we compute
\[\begin{array}{ll}
 & \pi_i(\relsem{\Gamma; \emptyset~|~\emptyset \vdash
  \fold^F_H : \Nat^\emptyset\;(\Nat^{\ol{\beta}, \ol{\gamma}}\,H[\phi
    :=_{\ol{\beta}} F][\ol{\alpha := \beta}]\,F)\;
  (\Nat^{{\ol{\beta},\ol{\gamma}} }\,(\mu \phi.\lambda \overline
  \alpha.H)\overline \beta\,F)})\\
=& \pi_i (\lambda
e\,\eta\,\ol{R}\,\ol{S}.\,(\mathit{fold}_{T_{\rho[\ol{\gamma 
        := S}]}}\,(\lambda \ol{Z}.\,\eta_{\ol{Z}\,\ol{S}}))_{\ol{R}})\\
=& \lambda
e\,\eta\,\ol{R}\,\ol{S}.\,(\mathit{fold}_{T_{(\pi_i\rho)[\ol{\gamma 
        := \pi_i S}]}}\,(\lambda \ol{Z}.\, (\pi_i
\eta)_{\ol{\pi_iZ}\,\ol{\pi_iS}}))_{\ol{\pi_i R}}\\ 
=& \lambda
d\,\eta\,\ol{B}\,\ol{C}.\,(\mathit{fold}_{T_{(\pi_i\rho)[\ol{\gamma 
        := C}]}} \,(\lambda \ol{A}.\eta_{\ol{A}\,\ol{C}}))_{\ol{B}}\\
=& \relsem{\Gamma; \emptyset~|~\emptyset \vdash
  \fold^F_H : \Nat^\emptyset\;(\Nat^{\ol{\beta}, \ol{\gamma}}\,H[\phi
    :=_{\ol{\beta}} F][\ol{\alpha := \beta}]\,F)\;
  (\Nat^{{\ol{\beta},\ol{\gamma}} }\,(\mu \phi.\lambda \overline
  \alpha.H)\overline \beta\,F)}(\pi_i \rho)\\

\end{array}\]
Here, we are again using the fact that $\pi_1$ and $\pi_2$ are
surjective. 
\end{itemize}  
\end{itemize}
\end{proof}

The Abstraction Theorem is now the special case of
Theorem~\ref{thm:at-gen} for closed terms of close type:

{\color{blue} State more generally as: If $(a,b) \in
  \relsem{\Gamma;\Phi\vdash \Delta}\rho$ then
  $(\setsem{\Gamma;\Phi\vdash\tau}(\pi_1\rho) a
  ,\setsem{\Gamma;\Phi\vdash\tau}(\pi_2\rho) b) \in
  \setsem{\Gamma;\Phi\vdash\tau}\rho$. Get the next theorem as a
  corollary for closed terms of closed type.}

\begin{thm}\label{thm:abstraction}
If $\,\vdash \tau : \F$ and $\vdash t : \tau$, then $(\setsem{\vdash t
  : \tau},\setsem{\vdash t : \tau}) \in \relsem{\vdash \tau}$.
\end{thm}

{\color{blue}
Our calculus does not support Church encodings of data types like pair
or sum or list types because all of the ``forall''s in our calculus
must be at the top level.  Nevertheless, our calculus does admit
actual sum and product and list types because they are coded by
$\mu$-terms in our calculus. We just don't have an equivalence of
these types and their Church encodings in our calculus, that's all.
}

\section{Free Theorems for Nested Types}\label{sec:ftnt}

\subsection{Free Theorem for Type of Polymorphic
  Bottom}\label{sec:bottom} 

Suppose $ \vdash g : \Nat^\alpha \,\onet\,\alpha$, let $G^\set =
\setsem{\vdash g : \Nat^\alpha \,\onet\,\alpha}$, and let $G^\rel =
\relsem{\vdash g : \Nat^\alpha \,\onet\,\alpha}$.  By
Theorem~\ref{thm:abstraction}, $(G^\set(\pi_1\rho),G^\set(\pi_2\rho))
= G^\rel\rho$. Thus, for all $\rho \in \relenv$ and any $(a, b) \in
\relsem{\vdash \emptyset}\rho = 1$, eliding the only possible
instantiations of $a$ and $b$ gives that
\[\begin{array}{lll}
(G^\set,G^\set) \;= \; (G^\set(\pi_1 \rho), G^\set (\pi_2 \rho))
& \in & \relsem{\vdash \Nat^\alpha \,\onet\,\alpha}\rho\\
& = & \{\eta : K_1 \Rightarrow \id\}\\
& = & \{(\eta_1 : K_1 \Rightarrow \id, \eta_2 : K_1 \Rightarrow
\id)\}\\ 
\end{array}\]
That is, $G^\set$ is a natural transformation from the constantly
$1$-valued functor to the identity functor in $\set$. In particular,
for every $S : \set$, $G^\set_S : 1 \to S$. Note, however, that if $S
= \emptyset$, then there can be no such morphism, so no such natural
transformation can exist in $\set$, and thus no term $\vdash g :
\Nat^\alpha \onet \,\alpha$ can exist in our calculus. That is, our
calculus does not admit any terms with the closed type $\Nat^\alpha
\onet \,\alpha$ of the polymorphic bottom.

\subsection{Free Theorem for Type of Polymorphic
  Identity}\label{sec:identity} 

Suppose $ \vdash g : \Nat^\alpha \,\alpha\,\alpha$, let $G^\set =
\setsem{\vdash g : \Nat^\alpha \,\alpha\,\alpha}$, and let $G^\rel =
\relsem{\vdash g : \Nat^\alpha \,\alpha\,\alpha}$.  By
Theorem~\ref{thm:abstraction}, $(G^\set(\pi_1\rho),G^\set(\pi_2\rho))
= G^\rel\rho$. Thus, for all $\rho \in \relenv$ and any $(a, b) \in
\relsem{\vdash \emptyset}\rho = 1$, eliding the only possible
instantiations of $a$ and $b$ gives that
\[\begin{array}{lll}
(G^\set, G^\set) \; = \; (G^\set(\pi_1 \rho), G^\set (\pi_2 \rho))
& \in & \relsem{\vdash \Nat^\alpha \,\alpha\,\alpha}\rho\\
& = & \{\eta : \id \Rightarrow \id\}\\
& = & \{(\eta_1 : \id \Rightarrow \id, \eta_2 : \id \Rightarrow
\id)\}\\ 
\end{array}\]
That is, $G^\set$ is a natural transformation from the identity
functor on $\set$ to itself.


Now let $S$ be any set.  If $S = \emptyset$, then there is exactly one
morphism $\id_S: S \to S$, so $G^\set_S : S \to S$ must be $\id_S$. If
$S \not = \emptyset$, then if $a$ is any element of $S$ and $K_a :S
\to S$ is the constantly $a$-valued morphism on $S$, then
instantiating the naturality square implied by the above equality
gives that $G^\set_S \circ K_a = K_a \circ G^\set_S$, i.e., $G^\set_S
\, a = a$, i.e., $G^\set_S = \id_S$.  Putting these two cases together
we have that for every $S : \set$, $G^\set_S = \id_S$, i.e., $G^\set$
is the identity natural transformation for the identity functor on
$\set$. So every closed term $g$ of closed type
$\Nat^\alpha\alpha\,\alpha$ always denotes the identity natural
transformation for the identity functor on $\set$, i.e., every closed
term $g$ of type $\Nat^\alpha\alpha\,\alpha$ denotes the polymorphic
identity function.

\subsection{Free Theorem for Type of $\mathit{filter}$ for Lists}

Let $List \, \alpha = \listt{\alpha}$, 
and let $map = \maplist$.

\begin{lemma}\label{lem:list-graph}
  If $g : A \to B$, $\rho : \relenv$, and $\rho\alpha = (A, B, \graph{g})$, then
    $\relsem{\alpha; \emptyset \vdash List \, \alpha} \rho = \graph{map \, g}$
\end{lemma}
\begin{proof}
% show by direct calculation
  \begin{align*}
    &\relsem{\alpha; \emptyset \vdash List \, \alpha} \rho  \\
    &= \mu T_{\rho} (\relsem{\alpha; \emptyset \vdash \alpha} \rho) \\    % def 19
    &= \mu T_{\rho} (A, B, \graph{g}) \\   
    &= (\mu T_{\pi_1\rho} A, \mu T_{\pi_2\rho} B, \colim{n \in \nat}{(T_{\rho}^n K_0)^*} (A, B, \graph{g})) \\ % eq 3, after dfn 15
    &= (\List \,A, \List \,B, \colim{n \in \nat}{\Sigma_{i = 0}^n (A, B, \graph{g})^i}) \\ % below (*)
    &= (\List \,A, \List \,B, \List (A, B, \graph{g})) \\ % definition of List
    &= (\List \,A, \List \,B, \graph{map \, g}) % below (**)
  \end{align*}

  The first equality is by Definition~\ref{def:rel-sem}, the third equality is by Equation~\ref{eq:mu}, 
  and the fourth and sixth equalities are by Equations~\ref{eq:T-list} and \ref{eq:List-graph-map} below.

  % (*)
  \noindent
  The following sequence of equalities shows 
  \begin{equation}\label{eq:T-list}
  (T_{\rho}^n K_0)^* \,R = \Sigma_{i=0}^n R^i
  \end{equation}
  by induction on $n$: \\
  \begin{align*}
    &  (T_{\rho}^n K_0)^* \,R \\
    &= T_{\rho}^\rel (T_{\rho}^{n-1} K_0)^* \,R \\
    &= \setsem{\alpha; \phi, \beta \vdash \onet + \beta \times \phi \beta} \rho 
          [\phi := (T_\rho^{n-1} K_0)^*] [\beta := R] \\
    &= \onet + R \times (T_\rho^{n-1} K_0)^* \, R \\
    &= \onet + R \times (\Sigma_{i=0}^{n-1} R^i) \\
    &= \Sigma_{i= 0}^n R^i
  \end{align*}

  \noindent
  The following reasoning shows 
  \begin{equation}\label{eq:List-graph-map}
    \List (A, B, \graph{g}) = \graph{map \,g}
  \end{equation}
  By showing that $(xs, xs') \in \List (A, B, \graph{g})$ if and only if $(xs, xs') \in \graph{map \,g}$:
  \begin{align*}
    &  (xs, xs') \in \List (A, B, \graph{g}) \\
    & \iff \forall i. (xs_i, xs'_i) \in \graph{g} \\
    & \iff \forall i. xs'_i = g (xs_i) \\
    & \iff xs'= map \,g \, xs \\
    & \iff (xs, xs') \in \graph{map \,g}
  \end{align*}
\end{proof}




\begin{thm}\label{thm:at-gen-rel}
If \,$\Gamma; \Phi \,|\, \Delta \vdash t : \tau$ and \,$\rho \in \relenv$,
  and if \,$(a, b) \in \relsem{\Gamma; \Phi \vdash \Delta} \rho$,
  then \\
  $(\setsem{\Gamma; \Phi \,|\, \Delta \vdash t : \tau} (\pi_1 \rho) \,a \, ,
      \setsem{\Gamma; \Phi \,|\, \Delta \vdash t : \tau } (\pi_2 \rho) \,b) \in 
    \relsem{\Gamma; \Phi \vdash \tau} \rho$
\end{thm}
\begin{proof}
  Immediate from Theorem~\ref{thm:at-gen} (at-gen).
\end{proof}


\begin{thm} 
  If $g : A \to B$, 
  $\rho : \relenv$, 
  $\rho \alpha = (A, B, \graph{g})$,
  $(a, b) \in \relsem{\alpha ;\emptyset \vdash \Delta} \rho$, 
  $(s \circ g, s) \in \relsem{\alpha; \emptyset \vdash \Nat^\emptyset \alpha \, \mathit{Bool}} \rho$,
  and, for some well-formed term $\mathit{filter}$, \\
  $$t = \setsem{\alpha; \emptyset \,|\, \Delta \vdash \mathit{filter} : \filtype} \emph{, then }$$
  \begin{align*}
  map \,g \circ t (\pi_1 \rho) \, a \, (s \circ g) = t (\pi_2\rho) \, b \, s \circ map \,g
  \end{align*}
\end{thm}
\begin{proof}
  By Theorem~\ref{thm:at-gen-rel},
  $(t (\pi_1 \rho) a, t (\pi_2 \rho) b) \in \relsem{\alpha; \emptyset \vdash \filtype} \rho$.
  Thus if 
  $(s, s') \in \relsem{\alpha; \emptyset \vdash \Nat^\emptyset \alpha \, \mathit{Bool}} \rho = \rho\alpha \to \Eq_{\mathit{Bool}}$,
  then 
  \begin{align*}
    (t (\pi_1 \rho) \,a \,s, t (\pi_2 \rho) \,b \,s') &\in \relsem{\alpha; \emptyset \vdash \Nat^\emptyset (List \, \alpha) \, (List \, \alpha)} \rho \\
   &= \relsem{\alpha; \emptyset \vdash List \, \alpha} \rho \to \relsem{\alpha; \emptyset \vdash List \, \alpha} \rho \\
  \end{align*}
  So if $(xs, xs') \in \relsem{\alpha; \emptyset \vdash List \, \alpha} \rho$ then,
  \begin{equation}\label{eq:filter-thm}
  (t (\pi_1\rho) \,a \,s \,xs, t (\pi_2\rho) \,b \,s' \,xs') \in \relsem{\alpha; \emptyset \vdash List \, \alpha} \rho
  \end{equation}

  Consider the case in which $\rho\alpha = (A, B, \graph{g})$. Then $\relsem{\alpha; \emptyset \vdash List \, \alpha} \rho
  = \graph{map \, g}$, by Lemma~\ref{lem:list-graph}, and $(xs, xs') \in \graph{map \,g}$ 
  implies $xs' = map \,g \,xs$. We also have that 
  $(s, s') \in \graph{g} \to \Eq_{\mathit{Bool}}$ implies 
  $\forall (x, g x) \in \graph{g}. \,\, s x = s' (g x)$ and thus
  $s = s' \circ g$ due to the definition of morphisms between relations.
  %% TODO show (s, s') \in <g> -> \mathit{Bool} implies s = s' \circ g %% 
  With these instantiations, Equation~\ref{eq:filter-thm} becomes
  \begin{align*}
    &(t (\pi_1\rho) \,a \,(s' \circ g) \,xs, t (\pi_2\rho) \,b \,s' \,(map \,g \,xs)) \in \graph{map \,g}, \\ 
    & i.e., \\ 
    &map \,g \, (t (\pi_1\rho) \,a \,(s' \circ g) \,xs) = t (\pi_2\rho) \,b \,s' \,(map \,g \,xs), \\ 
    & i.e., \\
    &map \,g \circ t (\pi_1 \rho) \, a \, (s' \circ g) = t (\pi_2\rho) \, b \, s' \circ map \,g
  \end{align*}
  as desired.


\end{proof}

\subsection{Free Theorem for Type of $\mathit{filter}$ for
  $\GRose$}\label{sec:filter-grose} 

\begin{thm} 
  Let $g : A \to B$ be a function,
  $\eta : F \to G$ a natural transformation of $\set$ functors,
  $\rho : \relenv$,
  $\rho \alpha = (A, B, \graph{g})$,
  $\rho \psi = (F, G, \graph{\eta})$,
  $(a, b) \in \relsem{\alpha, \psi ;\emptyset \vdash \Delta} \rho$, 
  and $(s \circ g, s) \in \relsem{\alpha; \emptyset \vdash \Nat^\emptyset \alpha \, \mathit{Bool}} \rho$.
  Then, for any well-formed term $\mathit{filter}$, if we call
  \[
    t = \setsem{\alpha, \psi; \emptyset \,|\, \Delta \vdash \mathit{filter} : \filtype}
  \]
  we have that
  \begin{align*}
  \semmap\, \eta\, (g + 1) \circ t (\pi_1 \rho) \, a \, (s \circ g) = t (\pi_2\rho) \, b \, s \circ \semmap\, \eta\, g
  \end{align*}
\end{thm}
\begin{proof}
  By Theorem~\ref{thm:at-gen-rel},
  \[
  (t (\pi_1 \rho) a, t (\pi_2 \rho) b) \in \relsem{\alpha, \psi; \emptyset \vdash \filtype} \rho
  \]
  Thus if 
  $(s, s') \in \relsem{\alpha; \emptyset \vdash \Nat^\emptyset \alpha \, \mathit{Bool}} \rho = \rho\alpha \to \Eq_{\mathit{Bool}}$,
  then 
  \begin{align*}
    (t (\pi_1 \rho) \,a \,s, t (\pi_2 \rho) \,b \,s')
    &\in \relsem{\alpha, \psi; \emptyset \vdash \Nat^\emptyset \, (\GRose\, \psi\, \alpha) \, (\GRose\, \psi\, (\alpha + \onet))} \rho \\
    &= \relsem{\alpha, \psi; \emptyset \vdash \GRose\, \psi\, \alpha} \rho \to \relsem{\alpha, \psi; \emptyset \vdash \GRose\, \psi\, (\alpha + \onet)} \rho
  \end{align*}
  So if $(xs, xs') \in \relsem{\alpha; \emptyset \vdash \GRose\, \psi\, \alpha} \rho$ then,
  \begin{equation}\label{eq:filter-thm}
    (t (\pi_1\rho) \,a \,s \,xs, t (\pi_2\rho) \,b \,s' \,xs')
    \in \relsem{\alpha, \psi; \emptyset \vdash \GRose\, \psi\, (\alpha + \onet)} \rho
  \end{equation}

  Since $\rho\alpha = (A, B, \graph{g})$ and $\rho\psi = (F, G, \graph{\psi})$,
  then $\relsem{\alpha, \psi; \emptyset \vdash \GRose\, \psi\, \alpha} \rho
  = \graph{\semmap \, \eta g}$
  and $\relsem{\alpha, \psi; \emptyset \vdash \GRose\, \psi\, (\alpha + \onet)} \rho
  = \graph{\semmap \, \eta (g + 1)}$,
  by Lemma~\ref{lem:list-graph}.
  Moreover, $(xs, xs') \in \graph{\semmap \, \eta\, g}$ implies $xs' = \semmap\, \eta\, g\, xs$.
  We also have that
  $(s, s') \in \graph{g} \to \Eq_{\mathit{Bool}}$ implies 
  $\forall (x, g x) \in \graph{g}. \,\, s x = s' (g x)$ and thus
  $s = s' \circ g$ due to the definition of morphisms between relations.
  %% TODO show (s, s') \in <g> -> \mathit{Bool} implies s = s' \circ g %% 
  With these instantiations, Equation~\ref{eq:filter-thm} becomes
  \begin{align*}
    &(t (\pi_1\rho) \,a \,(s' \circ g) \,xs, t (\pi_2\rho) \,b \,s' \,(\semmap \,\eta\, g \,xs)) \in \graph{\semmap \,\eta\, (g+1)}, \\ 
    & i.e., \\ 
    &\semmap \,\eta\,(g+1) \, (t (\pi_1\rho) \,a \,(s' \circ g) \,xs) = t (\pi_2\rho) \,b \,s' \,(\semmap \,\eta\, g \,xs), \\ 
    & i.e., \\
    &\semmap \,\eta\,(g+1) \circ t (\pi_1 \rho) \, a \, (s' \circ g) = t (\pi_2\rho) \, b \, s' \circ \semmap \,\eta\, g
  \end{align*}
  as desired.

\end{proof}

\subsection{Short Cut Fusion for Lists}\label{sec:short-cut}

\begin{thm}
Let $\vdash \tau : \F$, $\vdash \tau' : \F$, and 
$\beta; \emptyset \,|\, \emptyset \vdash g : \Nat^{\emptyset}
(\Nat^{\emptyset} (\onet + \tau \times \beta)\, \beta)\, \beta$. 
If \[G \, = \, \setsem{\beta; \emptyset \,|\, \emptyset \vdash g :
  \Nat^{\emptyset} (\Nat^{\emptyset} (\onet + \tau \times \beta)\,
  \beta)\, \beta}\] then
\[\textit{fold}_{1 + \tau \times \_}\, n\, c\; (G\; (\List
\setsem{\vdash \tau})\,\mathit{nil} \,\mathit{cons}) = G \,\setsem{\vdash
  \tau'}\, n\, c \]
\end{thm}

\vspace*{0.2in}

\begin{proof}
Let $\vdash \tau : \F$ and $\vdash \tau' : \F$, let
\[\beta; \emptyset \,|\, \emptyset \vdash g : \Nat^{\emptyset}
(\Nat^{\emptyset} (\onet + \tau \times \beta)\, \beta)\, \beta\]
and let
\[G = \setsem{\beta; \emptyset \,|\, \emptyset
  \vdash g : \Nat^{\emptyset} (\Nat^{\emptyset} (\onet + \tau \times
  \beta)\, \beta)\, \beta}\] Then Theorem~\ref{thm:abstraction} gives
that, for any relation environment $\rho$ and any $(a, b) \in
\relsem{\beta; \emptyset \vdash \emptyset}\rho = 1$, then (eliding the
only possible instantiations of $a$ and $b$) we have
\[(G \,(\pi_1 \rho), G\, (\pi_2 \rho))
\in \relsem{\beta; \emptyset \vdash \Nat^{\emptyset} (\Nat^{\emptyset}
  (\onet + \tau \times \beta)\, \beta)\, \beta} \rho\]
Since
\[\begin{array}{rl}
 & \relsem{\beta; \emptyset \vdash \Nat^{\emptyset}
  (\Nat^{\emptyset} (\onet + \tau \times \beta)\, \beta)\, \beta} \rho\\  
=& \relsem{\beta; \emptyset \vdash \Nat^{\emptyset} (\onet +
  \tau \times \beta)\, \beta} \rho \to \rho\beta \\ 
=& (\relsem{\beta; \emptyset \vdash \onet + \tau \times \beta}
\rho \to \rho\beta) \to \rho\beta \\ 
=& ((\onet + \relsem{\vdash \tau} \rho \times
\rho\beta) \to \rho\beta) \to \rho\beta \\ 
\cong& (((\relsem{\vdash \tau} \rho \times
\rho\beta) \to \rho\beta) \times \rho\beta) \to \rho\beta 
\end{array}\]
we have that if $(c', c) \in \relsem{\vdash \tau} \rho \times
\rho\beta \to \rho\beta$ and $(n', n) \in \rho\beta$, then
\[(G \,(\pi_1 \rho)\, n'\, c', G\, (\pi_2 \rho)\, n\, c)
\in \rho \beta\]

Now note that
\[\setsem{\vdash \fold_{\onet + \tau
    \times \beta}^{\tau'} : \Nat^{\emptyset} (\Nat^{\emptyset} (\onet
  + \tau \times \tau')\, \tau')\, (\Nat^{\emptyset} (\mu \alpha. \onet
  + \tau \times \alpha)\, \tau')} =
\textit{fold}_{1 + \tau \times \_}\] and observe that if $c \in
\setsem{\vdash \tau} \,\times\, \setsem{\vdash \tau'} \to
\setsem{\vdash \tau'}$ and $n \in \setsem{\vdash \tau'}$, then
\[(n, c) \in \setsem{\vdash \Nat^{\emptyset} (\onet + \tau \times \tau')\,
  \tau'}\]  Consider the instantiation:
\[\begin{array}{rcl}
\pi_1 \rho \beta &=& \setsem{\vdash \mu \alpha. \onet + \tau \times
  \alpha} \;=\;\List\,\setsem{\vdash \tau}\\   
\pi_2 \rho \beta &=& \setsem{\vdash \tau'}\\
\rho \beta &=& \graph{\textit{fold}_{1 + \tau \times \_}\, n\, c} :
\rel(\pi_1 \rho \beta, \pi_2 \rho \beta) \\ 
c' &=& \mathit{cons}\\
n' &=& \mathit{nil}
\end{array}\]
Clearly, $(\mathit{nil}, n) \in \rho\beta =
\graph{\textit{fold}_{1 + \tau \times \_}\, n\, c}$ because $
\textit{fold}_{1 + \tau \times \_}\, n\, c \,\mathit{nil} = n$.  Moreover,
$(\mathit{cons}, c) \in \relsem{\vdash \tau} \,\times\, \rho\beta \to
\rho\beta$ since if $(x, x') \in \relsem{\vdash \tau}$, i.e., $x =
x'$, and if $(y, y') \in \rho\beta =
\graph{\textit{fold}_{1 + \tau \times \_} n\, c}$, i.e., $y' =
\textit{fold}_{1 + \tau \times \_} n\, c\, y$, then
\[(\mathit{cons}\, x\, y, c\, x\, (\textit{fold}_{1 + \tau \times \_}\, n\,
c\, y)) \in \graph{\textit{fold}_{1 + \tau \times \_}\, n\, c}\]
i.e.,
\[c\, x\, (\textit{fold}_{1 + \tau \times \_}\, n\, c\, y)
= \textit{fold}_{1 + \tau \times \_}\, n\, c\, (\mathit{cons} \,x\, y)\]
holds by definition of $\textit{fold}_{1 + \tau \times \_}$.  We
therefore conclude that
\[(G \;(\List\,\setsem{\vdash \tau})\,\mathit{nil} \,\mathit{cons}, G \,\setsem{\vdash
  \tau'} \, n\, c) \in \graph{\textit{fold}_{1 + \tau \times \_}\, n\, c}\]
i.e., that
\[\textit{fold}_{1 + \tau \times \_}\, n\, c\; (G\;
(\List\,\setsem{\vdash \tau})\, \mathit{nil}\, \mathit{cons}) = G \,\setsem{\vdash
  \tau'}\, n\, c \] 
\end{proof}







\subsection{Short Cut Fusion for Arbitrary ADTs}\label{sec:short-cut-adt}

\begin{thm}
Let $\ol{\vdash \tau : \F}$, let $\vdash \tau' : \F$, let $\ol{\alpha}; \beta
\vdash F : \F$, and let $\beta; \emptyset  \,|\,
\emptyset \vdash g : \Nat^{\emptyset} (\Nat^{\emptyset} F[\ol{\alpha
    := \tau}]\, \beta)\, \beta$.  If we regard
\[\begin{array}{lll}
H & = & \setsem{\emptyset;\beta \vdash F[\ol{\alpha :=
      \tau}]}\\
G & = & \setsem{\beta; \emptyset \,|\, \emptyset \vdash g :
  \Nat^{\emptyset} (\Nat^{\emptyset}\, F[\ol{\alpha := \tau}]\,
  \beta)\, \beta}
\end{array}\]
as functors in $\beta$, then for every $B \in H \setsem{\vdash \tau'}
\rightarrow \setsem{\vdash \tau'}$ we have
\[\textit{fold}_H\, B \; (G\; \mu H \; in_H) = G \,\setsem{\vdash \tau'}\, B \]   
\end{thm}

\vspace*{0.2in}

\begin{proof}
  We first note that the type of $g$ is well-formed, since
  $\emptyset;\beta \vdash F[\ol{\alpha := \tau}] : \F$ so our
  promotion theorem gives that $\beta;\emptyset\vdash F[\ol{\alpha :=
      \tau}] : \F$, and $\emptyset;\beta\vdash\beta : \F$ so that our
  promotion theorem gives $\beta;\emptyset\vdash\beta : \F$. From
  these facts we deduce that $\beta;\emptyset \vdash
  \Nat^\emptyset\,F[\ol{\alpha := \tau}]\,\beta : \T$, and thus that
  $\beta; \emptyset \vdash \Nat^{\emptyset} (\Nat^{\emptyset}
  F[\ol{\alpha := \tau}]\, \beta)\, \beta : \T$.

  Theorem~\ref{thm:abstraction} gives that, for any relation environment
$\rho$ and any $(a, b) \in \relsem{\beta; \emptyset \vdash
  \emptyset}\rho = 1$, eliding the only possible instantiations
of $a$ and $b$ gives that
\[(G \,(\pi_1 \rho), G\, (\pi_2 \rho))
\in \relsem{\beta; \emptyset \vdash \Nat^{\emptyset} (\Nat^{\emptyset}
  F[\ol{\alpha := \tau}]\, \beta)\, \beta} \rho\]
Since
\[\begin{array}{rl}
 & \relsem{\beta; \emptyset \vdash \Nat^{\emptyset}
  (\Nat^{\emptyset} F[\ol{\alpha := \tau}]\, \beta)\, \beta} \rho\\  
=& \relsem{\beta; \emptyset \vdash \Nat^{\emptyset} F[\ol{\alpha :=
      \tau}] \, \beta} \rho \to \rho\beta \\ 
\end{array}\]
we have that if $(A, B) \in \relsem{\beta; \emptyset \vdash
  \Nat^{\emptyset} F[\ol{\alpha := \tau}] \, \beta} \rho$
then
\[(G \,(\pi_1 \rho)\, A, G\, (\pi_2 \rho)\, B)
\in \rho \beta\]
Now note that
\[\setsem{\vdash \fold_{F[\ol{\alpha := \tau}]}^{\tau'} :
  \Nat^{\emptyset}\, (\Nat^{\emptyset}\, F[\ol{\alpha := \tau}][\beta
    := \tau'] \, \tau')\, (\Nat^{\emptyset} \,(\mu
  \beta. F[\ol{\alpha := \tau}] \, \tau')} =
\textit{fold}_H\]
and consider the instantiation
\[\begin{array}{lll}
A &=& \mathit{in}_H :  H (\mu H) \to \mu H\\
B & : & H\setsem{\vdash \tau'} \to \setsem{\vdash \tau'}\\
\rho \beta &=& \graph{\textit{fold}_H\, B}
\end{array}\]
(Note that all the types here are well-formed.)  This gives
\[\begin{array}{lll}
\pi_1 \rho \beta &=& \setsem{\vdash \mu \beta. F[\ol{\alpha := \tau}]}
\; = \; \mu H\\
\pi_2 \rho \beta &=& \setsem{\vdash \tau'}\\
\rho \beta &:& \rel(\pi_1 \rho \beta, \pi_2 \rho \beta)\\
A & : & \setsem{\beta; \emptyset \vdash \Nat^{\emptyset}
  F[\ol{\alpha := \tau}] \, \beta} (\pi_1 \rho)\\
B & : & \setsem{\beta; \emptyset \vdash \Nat^{\emptyset} F[\ol{\alpha
      := \tau}] \, \beta}(\pi_2 \rho)\\
\end{array}\]
since
\[\begin{array}{lll}
A \; = \; \mathit{in}_H & : & H(\mu H) \to \mu H\\
 & = & \setsem{\emptyset;\beta \vdash F[\ol{\alpha := \tau}]}(\mu
\setsem{\emptyset;\beta \vdash F[\ol{\alpha := \tau}]}) \to \mu
\setsem{\emptyset;\beta \vdash F[\ol{\alpha := \tau}]}\\
& = & \setsem{\emptyset;\beta \vdash F[\ol{\alpha :=
      \tau}]}(\pi_1\rho) \to \setsem{\emptyset;\beta \vdash
  \beta}(\pi_1 \rho)\\
& = & \setsem{\beta;\emptyset \vdash F[\ol{\alpha :=
      \tau}]}(\pi_1\rho) \to \setsem{\beta; \emptyset \vdash
  \beta}(\pi_1 \rho) \hspace*{0.25in} {\color{red} \mbox{Daniel's
    trick; now a theorem}} \\ 
& = & \setsem{\beta;\emptyset \vdash \Nat^\emptyset F[\ol{\alpha :=
      \tau}]\,\beta}(\pi_1 \rho) 
\end{array}\]
{\color{red} where ``Daniel's trick'' is the observation that a
  functor can be seen as non-functorial when we only care about its
  action on objects. This is now a theorem.}  We also have
\[\begin{array}{lll}
(A,B) \,=\, (\mathit{in}_H,B) & \in & \relsem{\beta; \emptyset \vdash
  \Nat^{\emptyset} F[\ol{\alpha := \tau}] \, \beta} \rho \\
 & = & \relsem{\beta; \emptyset \vdash F[\ol{\alpha :=
      \tau}]}\rho[\beta := \graph{\textit{fold}_H\, B}] \to
\graph{\textit{fold}_H\, B}\\
 & = & \relsem{\beta; \emptyset \vdash F[\ol{\alpha :=
      \tau}]}\graph{\textit{fold}_H\, B} \to
\graph{\textit{fold}_H\, B}\\
 & = & \relsem{\emptyset; \beta\vdash F[\ol{\alpha :=
      \tau}]}\graph{\textit{fold}_H\, B} \to
\graph{\textit{fold}_H\, B} \hspace*{0.35in} {\color{red}
  \mbox{Daniel's trick; now a theorem}}\\  
& = & \graph{\setsem{\emptyset; \beta \vdash F[\ol{\alpha :=
      \tau}]}\,(\textit{fold}_H\, B)} \to
\graph{\textit{fold}_H\, B} \hspace*{0.25in} {\color{red} \mbox{by the
    graph lemma}}\\ 
& = & \graph{\mathit{map}_H \,(\mathit{fold}_H\,B)} \to
\graph{\textit{fold}_H\, B}\\
\end{array}\]
since if $(x,y) \in \graph{\mathit{map}_H \,(\mathit{fold}_H\,B)}$,
i.e., if $\mathit{map}_H \,(\mathit{fold}_H\,B) \, x = y$, then
$\textit{fold}_H\, B\, (\mathit{in}_H\,x) = B\,y = B\, (\mathit{map}_H
\,(\mathit{fold}_H\,B) \, x)$ by the definition of $\mathit{fold}_H$
as a (indeed, the unique) morphism from $\mathit{in}_H$ to $B$. Thus,
\[(G \,(\pi_1 \rho)\, A, G\, (\pi_2 \rho)\, B) \in
\graph{\textit{fold}_H\, B}\]
i.e.,
\[\textit{fold}_H \, B \, (G\, (\pi_1 \rho) \, \mathit{in}_H) =
G\,(\pi_2 \rho)\,B\] 

\vspace*{0.1in}

\noindent
Since $\beta$ is the only free variable in $G$, this simplifies to
\[\textit{fold}_H\, B \, (G\, \mu H\, \mathit{in}_H) =
G\,\setsem{\vdash \tau'}\,B\]  
\end{proof}


\subsection{Short Cut Fusion for Arbitrary Nested Types}\label{sec:short-cut-nested}

{\color{red} Can take $\emptyset;\alpha \vdash c$ with
  $\setsem{\emptyset;\alpha \vdash c}\rho = C$ for all $\rho$, i.e.,
  can take $c$ to denote a constant $C$. We then get a free theorem
  whose conclusion is $\textit{fold}_{H}\, B \; \circ \; G\; \mu H \;
  in_{H} = G \,\setsem{\emptyset;\alpha \vdash K}\, B$.

  Can do Hinze's bit-reversal protocol in our system with

  cat :: $\alpha; \emptyset \vdash \Nat^\emptyset (\Nat^\emptyset
  (\List \alpha) (\List \alpha)) (\List \alpha)$

  zip :: $\alpha; \emptyset \vdash \Nat^\emptyset (\Nat^\emptyset
  (\List \alpha) (\List \beta)) (\List (\alpha \times \beta))$

  ?}


\begin{thm}
Let $\emptyset;\phi,\alpha \vdash F : \F$, let $\emptyset; \alpha
\vdash K : \F$, and let $\phi ;\emptyset\,|\,\emptyset\vdash g :
\Nat^\emptyset\,(\Nat^\alpha\,F\,(\phi\alpha)) \,
(\Nat^\alpha\,\onet \, (\phi\alpha))$. If we let $H :
    [\set,\set] \to [\set,\set]$ be defined by
\[\begin{array}{lll}
H\,f\,x & = & \setsem{\emptyset; \phi, \alpha \vdash F}[\phi :=
  f][\alpha := x]\\
\end{array}\]
and let
\[G = \setsem{\phi;\emptyset\,|\,\emptyset \vdash g :
\Nat^\emptyset\,(\Nat^\alpha\,F\,(\phi\alpha))\,(\Nat^\alpha\,\onet 
\, (\phi\alpha))}\] 
then we have that, for every $B \in H
\setsem{\emptyset;\alpha \vdash K} \rightarrow \setsem{\emptyset;
  \alpha \vdash K}$, 
\[\textit{fold}_{H}\, B \, (G\; \mu H \; in_{H}) = G
\,\setsem{\emptyset;\alpha \vdash K}\, B \]
\end{thm}

\vspace*{0.2in}

\begin{proof}
We first note that the type of $g$ is well-formed since
$\emptyset;\phi,\alpha \vdash F : \F$ so our promotion theorem gives
that $\phi;\alpha \vdash F : \F$, and $\phi;\alpha \vdash \phi\alpha :
\F$, so that $\phi;\emptyset \vdash \Nat^\alpha F\,(\phi\alpha) : \T$
and $\phi;\emptyset \vdash \Nat^\alpha \onet \,(\phi\alpha) : \T$.
Then $\phi;\emptyset \vdash \Nat^\alpha F\,(\phi\alpha) : \F$ and
$\phi;\emptyset \vdash \Nat^\alpha \onet \,(\phi\alpha) : \F$ also
hold, and, finally, 
$\phi ;\emptyset\vdash 
\Nat^\emptyset\,(\Nat^\alpha\,F\,(\phi\alpha)) \,
(\Nat^\alpha\,\onet \, (\phi\alpha)) : \T$
  
Theorem~\ref{thm:abstraction} gives that, for any relation environment
$\rho$ and any $(a, b) \in \relsem{\phi,\alpha;\emptyset \vdash
  \emptyset}\rho = 1$, eliding the only possible instantiations
of $a$ and $b$ gives that
\[\begin{array}{lll}
(G \,(\pi_1 \rho), G\, (\pi_2 \rho))
& \in & \relsem{\phi;\emptyset\vdash \Nat^{\emptyset} (\Nat^\alpha
  F\, (\phi\alpha))\, (\Nat^\alpha\,\onet \, (\phi\alpha))}
\rho\\ 
& = & \relsem{\phi;\emptyset\vdash \Nat^\alpha F\,
  (\phi\alpha)}\rho \to \relsem{\phi;\emptyset\vdash
  \Nat^\alpha\,\onet \, (\phi\alpha)}\rho\\ 
& = & \relsem{\phi;\emptyset\vdash \Nat^\alpha F\,
  (\phi\alpha)}\rho \to (\,\lambda A. 1 \Rightarrow \lambda A.\,(\rho
\phi) A\,)\\   
& = & \relsem{\phi;\emptyset\vdash \Nat^\alpha F\,
  (\phi\alpha)}\rho \to (1 \Rightarrow \rho \phi)\\ 
& = & \relsem{\phi;\emptyset\vdash \Nat^\alpha F\,
  (\phi\alpha)}\rho \to \rho \phi
\end{array}\]
\noindent
So if $(A, B) \in \relsem{\phi;\emptyset\vdash \Nat^\alpha F\,
  (\phi\alpha)}\rho$ then
\[\begin{array}{lll}
(G \,(\pi_1 \rho)\, A, G\, (\pi_2 \rho)\, B) & \in & \rho \phi
\end{array}\]
Now note that
\[\setsem{\vdash \fold_F^K :
  \Nat^{\emptyset}\, (\Nat^\alpha F[\phi := K]\,K)\, (\Nat^\alpha ((\mu
  \phi.\lambda\alpha.F)\alpha)\,K)} = \textit{fold}_H\]
and consider the instantiation 
\[\begin{array}{lll}
A &=& \mathit{in}_H :  H (\mu H) \Rightarrow \mu H\\
B & : & H\setsem{\emptyset;\alpha\vdash K} \Rightarrow
\setsem{\emptyset;\alpha \vdash K}\\
\rho \phi &=& \graph{\textit{fold}_H\, B} \hspace*{0.2in} {
  \color{red} \mbox{a graph of a natural transformation, defined in
    Enrico's notes}} 
\end{array}\]
(Note that all the types here are well-formed.) This gives
\[\begin{array}{lll}
\pi_1 \rho \phi &=& \mu H\\
\pi_2 \rho \phi &=& \setsem{\emptyset;\alpha\vdash K}\\
\rho \phi &:& \rel(\pi_1 \rho \phi, \pi_2 \rho \phi)\\
A & : & \setsem{\phi; \emptyset \vdash \Nat^\alpha
  F \, (\phi \alpha)} (\pi_1 \rho)\\
B & : & \setsem{\phi; \emptyset \vdash \Nat^\alpha
  F \, (\phi \alpha)} (\pi_2 \rho)\\
\end{array}\]
since
\[\begin{array}{lll}
A \; = \; \mathit{in}_H & : & H(\mu H) \Rightarrow \mu H\\
& = & \setsem{\emptyset; \phi, \alpha \vdash F}[\phi := \mu
  \setsem{\emptyset; \phi, \alpha \vdash F}] \Rightarrow \mu
  \setsem{\emptyset; \phi, \alpha \vdash F}\\  
& = & \setsem{\emptyset; \phi, \alpha \vdash F}(\pi_1\rho)
  \Rightarrow \setsem{\emptyset; \phi,\alpha \vdash \phi\alpha}(\pi_1 \rho)\\
& = & \setsem{\phi; \alpha \vdash F}(\pi_1\rho)
  \Rightarrow \setsem{\phi;\alpha \vdash \phi\alpha}(\pi_1
  \rho) \hspace*{0.25in} {\color{red} \mbox{Daniel's trick; now a theorem}}\\
& = & \setsem{\phi;\emptyset \vdash \Nat^\alpha F
    \,(\phi\alpha)}(\pi_1\rho) 
\end{array}\]
We also have
\[\begin{array}{lll}
(A,B) \,=\, (\mathit{in}_H,B) & \in & \relsem{\phi;\emptyset\vdash
  \Nat^\alpha F\, (\phi\alpha)}\rho\\
 & = & \lambda A. \relsem{\phi;\alpha\vdash F}\rho[\alpha := A]
\Rightarrow \lambda A. (\rho\phi) A\\
& = & \lambda A. \relsem{\phi;\alpha\vdash F}[\phi :=
  \graph{\textit{fold}_H\, B}][\alpha := A] \Rightarrow 
\graph{\textit{fold}_H\, B} \\ 
& = & \lambda A. \relsem{\emptyset;\phi,\alpha\vdash F}[\phi :=
  \graph{\textit{fold}_H\, B}][\alpha := A] \Rightarrow 
\graph{\textit{fold}_H\, B}  \hspace*{0.25in} {\color{red}
  \mbox{Daniel's trick; now a theorem}}\\  
& = & \relsem{\emptyset;\phi,\alpha\vdash F}
  \graph{\textit{fold}_H\, B} \Rightarrow \graph{\textit{fold}_H\,
    B}\\
  & = & \graph{\setsem{\emptyset;\phi,\alpha\vdash F}
    \,(\mathit{fold}_H\,B)} \Rightarrow \graph{\textit{fold}_H\, B}
\hspace*{0.25in} {\color{red} \mbox{Graph Lemma}}\\
  & = & \graph{\mathit{map}_H \,(\mathit{fold}_H\,B)} \Rightarrow
\graph{\textit{fold}_H\, B}\\
\end{array}\]
since if $(x,y) \in \graph{\mathit{map}_H \,(\mathit{fold}_H\,B)}$,
i.e., if $\mathit{map}_H \,(\mathit{fold}_H\,B) \, x = y$, then
$\textit{fold}_H\, B\, (\mathit{in}_H\,x) = B\,y = B\, (\mathit{map}_H
\,(\mathit{fold}_H\,B) \, x)$ by the definition of $\mathit{fold}_H$
as a (indeed, the unique) morphism from $\mathit{in}_H$ to $B$. Thus,
\[(G \,(\pi_1 \rho)\, A, G\, (\pi_2 \rho)\, B) \in
\graph{\textit{fold}_H\, B}\] 
i.e.,
\[\textit{fold}_H \, B \, (G\, (\pi_1 \rho) \, \mathit{in}_H) =
G\,(\pi_2 \rho)\,B\] 

\vspace*{0.1in}

\noindent
Since $\phi$ is the only free variable in $G$, this simplifies to
\[\textit{fold}_H\, B \, (G\, \mu H\, \mathit{in}_H) =
G\,\setsem{\emptyset;\alpha\vdash K}\,B\]  
\end{proof}








\section{Conclusion and Directions for Future Work}\label{sec:conclusion}

Can do everything in abstract locally presentable cartesian closed
category. 

Give definitions for arb lpccc, but compute free theorems in Set/Rel.

Future Work (in progress): extend calculus to GADTs

Add more polymorphisms (all foralls), even though most free theorems
only use one level (or maybe two, like short cut).

Couldn't do this before~\cite{jp19} because we didn't know before that
nested types (and then some) always have well-defined interpretations
in locally finitely presentable categories like $\set$ and $\rel$. In
fact, could extend results here to ``locally presentable fibrations'',
where these are yet to be defined, but would at least have locally
presentable base and total categories with the locally presentable
structure preserved by the fibration and appropriate reflection of the
total category in the base (as in Alex's effects paper?).


fixed points at term level ala Pitts

\bibliography{references}

\end{document}


  
