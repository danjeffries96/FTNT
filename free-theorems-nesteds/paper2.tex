% For double-blind review submission, w/o CCS and ACM Reference (max
% submission space)
\documentclass[acmsmall,review,anonymous]{acmart}
\settopmatter{printfolios=true,printccs=false,printacmref=false}
%% For double-blind review submission, w/ CCS and ACM Reference
%\documentclass[acmsmall,review,anonymous]{acmart}\settopmatter{printfolios=true}
%% For single-blind review submission, w/o CCS and ACM Reference (max submission space)
%\documentclass[acmsmall,review]{acmart}\settopmatter{printfolios=true,printccs=false,printacmref=false}
%% For single-blind review submission, w/ CCS and ACM Reference
%\documentclass[acmsmall,review]{acmart}\settopmatter{printfolios=true}
%% For final camera-ready submission, w/ required CCS and ACM Reference
%\documentclass[acmsmall]{acmart}\settopmatter{}


%% Journal information
%% Supplied to authors by publisher for camera-ready submission;
%% use defaults for review submission.
\acmJournal{PACMPL}
\acmVolume{1}
\acmNumber{POPL} % CONF = POPL or ICFP or OOPSLA
\acmArticle{1}
\acmYear{2020}
\acmMonth{1}
\acmDOI{} % \acmDOI{10.1145/nnnnnnn.nnnnnnn}
\startPage{1}

%% Copyright information
%% Supplied to authors (based on authors' rights management selection;
%% see authors.acm.org) by publisher for camera-ready submission;
%% use 'none' for review submission.
\setcopyright{none}
%\setcopyright{acmcopyright}
%\setcopyright{acmlicensed}
%\setcopyright{rightsretained}
%\copyrightyear{2018}           %% If different from \acmYear

%% Bibliography style
\bibliographystyle{ACM-Reference-Format}
%% Citation style
%% Note: author/year citations are required for papers published as an
%% issue of PACMPL.
\citestyle{acmauthoryear}   %% For author/year citations
%\citestyle{acmnumeric}

%%%%%%%%%%%%%%%%%%%%%%%%%%%%%%%%%%%%%%%%%%%%%%%%%%%%%%%%%%%%%%%%%%%%%%
%% Note: Authors migrating a paper from PACMPL format to traditional
%% SIGPLAN proceedings format must update the '\documentclass' and
%% topmatter commands above; see 'acmart-sigplanproc-template.tex'.
%%%%%%%%%%%%%%%%%%%%%%%%%%%%%%%%%%%%%%%%%%%%%%%%%%%%%%%%%%%%%%%%%%%%%%



\usepackage[utf8]{inputenc}
\usepackage{ccicons}
\usepackage{verbatim}

\usepackage{amsmath}
\usepackage{amsthm}
\usepackage{amscd}
%\usepackage{MnSymbol}
\usepackage{xcolor}

\usepackage{bbold}
\usepackage{url}
\usepackage{upgreek}
%\usepackage{stmaryrd}

\usepackage{lipsum}
\usepackage{tikz-cd}
\usetikzlibrary{cd}
\usetikzlibrary{calc}
\usetikzlibrary{arrows}

\usepackage{bussproofs}
\EnableBpAbbreviations

\DeclareMathAlphabet{\mathpzc}{OT1}{pzc}{m}{it}

%\usepackage[amsmath]{ntheorem}

\newcommand{\lam}{\lambda}
\newcommand{\eps}{\varepsilon}
\newcommand{\ups}{\upsilon}
\newcommand{\mcB}{\mathcal{B}}
\newcommand{\mcD}{\mathcal{D}}
\newcommand{\mcE}{\mathcal{E}}
\newcommand{\mcF}{\mathcal{F}}
\newcommand{\mcP}{\mathcal{P}}
\newcommand{\mcI}{\mathcal{I}}
\newcommand{\mcJ}{\mathcal{J}}
\newcommand{\mcK}{\mathcal{K}}
\newcommand{\mcL}{\mathcal{L}}
\newcommand{\WW}{\mathcal{W}}

\newcommand{\ex}{\mcE_x}
\newcommand{\ey}{\mcE_y}
\newcommand{\bzero}{\boldsymbol{0}}
\newcommand{\bone}{{\boldsymbol{1}}}
\newcommand{\tB}{{\bone_\mcB}}
\newcommand{\tE}{{\bone_\mcE}}
\newcommand{\bt}{\mathbf{t}}
\newcommand{\bp}{\mathbf{p}}
\newcommand{\bsig}{\mathbf{\Sigma}}
\newcommand{\bpi}{\boldsymbol{\pi}}
\newcommand{\Empty}{\mathtt{Empty}}
\newcommand{\truthf}{\mathtt{t}}
\newcommand{\id}{id}
\newcommand{\coo}{\mathtt{coo\ }}
\newcommand{\mcC}{\mathcal{C}}
\newcommand{\Rec}{\mathpzc{Rec}}
\newcommand{\types}{\mathcal{T}}

%\newcommand{\Homrel}{\mathsf{Hom_{Rel}}}
\newcommand{\HomoCPOR}{\mathsf{Hom_{\oCPOR}}}

%\newcommand{\semof}[1]{\llbracket{#1}\rrbracket^\rel}
\newcommand{\sem}[1]{\llbracket{#1}\rrbracket}
\newcommand{\setsem}[1]{\llbracket{#1}\rrbracket^\set}
\newcommand{\relsem}[1]{\llbracket{#1}\rrbracket^\rel}
\newcommand{\dsem}[1]{\llbracket{#1}\rrbracket^{\mathsf D}}
\newcommand{\setenv}{\mathsf{SetEnv}}
\newcommand{\relenv}{\mathsf{RelEnv}}
\newcommand{\oCPOenv}{\mathsf{SetEnv}}
\newcommand{\oCPORenv}{\mathsf{RelEnv}}
\newcommand{\oCPOsem}[1]{\llbracket{#1}\rrbracket^{\set}}
\newcommand{\oCPORsem}[1]{\llbracket{#1}\rrbracket^{\rel}}
\newcommand{\denv}{\mathsf{DEnv}}

\newcommand{\rel}{\mathsf{Rel}}
\newcommand{\setof}[1]{\{#1\}}
\newcommand{\letin}[1]{\texttt{let }#1\texttt{ in }}
\newcommand{\comp}[1]{{\{#1\}}}
\newcommand{\bcomp}[1]{\{\![#1]\!\}}
\newcommand{\beps}{\boldsymbol{\eps}}
%\newcommand{\B}{\mcB}
%\newcommand{\Bo}{{|\mcB|}}

\newcommand{\lmt}{\longmapsto}
\newcommand{\RA}{\Rightarrow}
\newcommand{\LA}{\Leftarrow}
\newcommand{\rras}{\rightrightarrows}
\newcommand{\colim}[2]{{{\underrightarrow{\lim}}_{#1}{#2}}}
\newcommand{\lift}[1]{{#1}\,{\hat{} \; \hat{}}}
\newcommand{\graph}[1]{\langle {#1} \rangle}

\newcommand{\carAT}{\mathsf{car}({\mathcal A}^T)}
\newcommand{\isoAto}{\mathsf{Iso}({\mcA^\to})}
\newcommand{\falg}{\mathsf{Alg}_F}
\newcommand{\CC}{\mathsf{Pres}(\mathcal{A})}
\newcommand{\PP}{\mathcal{P}}
\newcommand{\DD}{D_{(A,B,f)}}
\newcommand{\from}{\leftarrow}
\newcommand{\upset}[1]{{#1}{\uparrow}}
\newcommand{\smupset}[1]{{#1}\!\uparrow}

\newcommand{\Coo}{\mathpzc{Coo}}
\newcommand{\code}{\#}
\newcommand{\nat}{\mathpzc{Nat}}

\newcommand{\eq}{\; = \;}
\newcommand{\of}{\; : \;}
\newcommand{\df}{\; := \;}
\newcommand{\bnf}{\; ::= \;}

\newcommand{\zmap}[1]{{\!{\between\!\!}_{#1}\!}}
\newcommand{\bSet}{\mathbf{Set}}

\newcommand{\dom}{\mathsf{dom}}
\newcommand{\cod}{\mathsf{cod}}
\newcommand{\adjoint}[2]{\mathrel{\mathop{\leftrightarrows}^{#1}_{#2}}}
\newcommand{\isofunc}{\mathpzc{Iso}}
\newcommand{\ebang}{{\eta_!}}
\newcommand{\lras}{\leftrightarrows}
\newcommand{\rlas}{\rightleftarrows}
\newcommand{\then}{\quad\Longrightarrow\quad}
\newcommand{\hookup}{\hookrightarrow}

\newcommand{\spanme}[5]{\begin{CD} #1 @<#2<< #3 @>#4>> #5 \end{CD}}
\newcommand{\spanm}[3]{\begin{CD} #1 @>#2>> #3\end{CD}}
\newcommand{\pushout}{\textsf{Pushout}}
\newcommand{\mospace}{\qquad\qquad\!\!\!\!}

\newcommand{\natur}[2]{#1 \propto #2}

\newcommand{\Tree}{\mathsf{Tree}\,}
\newcommand{\GRose}{\mathsf{GRose}\,}
\newcommand{\List}{\mathsf{List}\,}
\newcommand{\PTree}{\mathsf{PTree}\,}
\newcommand{\Bush}{\mathsf{Bush}\,}
\newcommand{\Forest}{\mathsf{Forest}\,}
\newcommand{\Lam}{\mathsf{Lam}\,}
\newcommand{\LamES}{\mathsf{Lam}^+}
\newcommand{\Expr}{\mathsf{Expr}\,}

\newcommand{\ListNil}{\mathsf{Nil}}
\newcommand{\ListCons}{\mathsf{Cons}}
\newcommand{\LamVar}{\mathsf{Var}}
\newcommand{\LamApp}{\mathsf{App}}
\newcommand{\LamAbs}{\mathsf{Abs}}
\newcommand{\LamSub}{\mathsf{Sub}}
\newcommand{\ExprConst}{\mathsf{Const}}
\newcommand{\ExprPair}{\mathsf{Pair}}
\newcommand{\ExprProj}{\mathsf{Proj}}
\newcommand{\ExprAbs}{\mathsf{Abs}}
\newcommand{\ExprApp}{\mathsf{App}}
\newcommand{\Ptree}{\mathsf{Ptree}}

\newcommand{\kinds}{\mathpzc{K}}
\newcommand{\tvars}{\mathbb{T}}
\newcommand{\fvars}{\mathbb{F}}
\newcommand{\consts}{\mathpzc{C}}
\newcommand{\Lan}{\mathsf{Lan}}
\newcommand{\zerot}{\mathbb{0}}
\newcommand{\onet}{\mathbb{1}}
\newcommand{\bool}{\mathbb{2}}
\renewcommand{\nat}{\mathbb{N}}
%\newcommand{\semof}[1]{[\![#1]\!]}
%\newcommand{\setsem}[1]{\llbracket{#1}\rrbracket^\set}
\newcommand{\predsem}[1]{\llbracket{#1}\rrbracket^\pred}
%\newcommand{\todot}{\stackrel{.}{\to}}
\newcommand{\todot}{\Rightarrow}
\newcommand{\bphi}{{\bm \phi}}

\newcommand{\bm}[1]{\boldsymbol{#1}}

\newcommand{\cL}{\mathcal{L}}
\newcommand{\T}{\mathcal{T}}
\newcommand{\Pos}{P\!}
%\newcommand{\Pos}{\mathcal{P}\!}
\newcommand{\Neg}{\mathcal{N}}
\newcommand{\Hf}{\mathcal{H}}
\newcommand{\V}{\mathbb{V}}
\newcommand{\I}{\mathcal{I}}
\newcommand{\Set}{\mathsf{Set}}
%\newcommand{\Nat}{\mathsf{Nat}}
\newcommand{\Homrel}{\mathsf{Hom_{Rel}}}
\newcommand{\CV}{\mathcal{CV}}
\newcommand{\lan}{\mathsf{Lan}}
\newcommand{\Id}{\mathit{Id}}
\newcommand{\mcA}{\mathcal{A}}
\newcommand{\inl}{\mathsf{inl}}
\newcommand{\inr}{\mathsf{inr}}
%\newcommand{\case}[3]{\mathsf{case}\,{#1}\,\mathsf{of}\,\{{#2};\,{#3}\}}
\newcommand{\tin}{\mathsf{in}}
\newcommand{\fold}{\mathsf{fold}}
\newcommand{\Eq}{\mathsf{Eq}}
\newcommand{\Hom}{\mathsf{Hom}}
\newcommand{\curry}{\mathsf{curry}}
\newcommand{\uncurry}{\mathsf{uncurry}}
\newcommand{\eval}{\mathsf{eval}}
\newcommand{\apply}{\mathsf{apply}}
\newcommand{\oCPO}{{\mathsf{Set}}}
\newcommand{\oCPOR}{{\mathsf{Rel}}}
\newcommand{\oCPORT}{{\mathsf{RT}}}

\newcommand{\ar}[1]{\##1}
\newcommand{\mcG}{\mathcal{G}}
\newcommand{\mcH}{\mathcal{H}}
\newcommand{\TV}{\mathpzc{V}}

\newcommand{\essim}[1]{\mathsf{EssIm}(#1)}
\newcommand{\hra}{\hookrightarrow}

\newcommand{\ol}[1]{\overline{#1}}
\newcommand{\ul}[1]{\underline{#1}}
\newcommand{\op}{\mathsf{op}}
\newcommand{\trige}{\trianglerighteq}
\newcommand{\trile}{\trianglelefteq}
\newcommand{\LFP}{\mathsf{LFP}}
\newcommand{\LAN}{\mathsf{Lan}}
%\newcommand{\Mu}{{\mu\hskip-4pt\mu}}
\newcommand{\Mu}{{\mu\hskip-5.5pt\mu}}
%\newcommand{\Mu}{\boldsymbol{\upmu}}
\newcommand{\Terms}{\mathpzc{Terms}}
\newcommand{\Ord}{\mathpzc{Ord}}
\newcommand{\Anote}[1]{{\color{blue} {#1}}}
\newcommand{\Pnote}[1]{{\color{red} {#1}}}

\newcommand{\greyout}[1]{{\color{gray} {#1}}}
\newcommand{\ora}[1]{\overrightarrow{#1}}

%\newcommand{\?}{{.\ }}
%\theoremheaderfont{\scshape}
%\theorembodyfont{\normalfont}
%\theoremseparator{.\ \ }
\newtheorem{thm}{Theorem}
\newtheorem{dfn}[thm]{Definition}
\newtheorem{prop}[thm]{Proposition}
\newtheorem{cor}[thm]{Corollary}
\newtheorem{lemma}[thm]{Lemma}
\newtheorem{rmk}[thm]{Remark}
\newtheorem{expl}[thm]{Example}
\newtheorem{notn}[thm]{Notation}
%\theoremstyle{nonumberplain}
%\theoremsymbol{\Box}


\theoremstyle{definition}
\newtheorem{exmpl}{Example}

\renewcommand{\greyout}[1]{} %{{\color{gray} {#1}}} -- toggle to remove greyed text

\newcommand{\emptyfun}{{[]}}
\newcommand{\cal}{\mathcal}
%\newcommand{\fold}{\mathit{fold}}
\newcommand{\F}{\mathcal{F}}
\renewcommand{\G}{\mathcal{G}}
\newcommand{\N}{\mathcal{N}}
\newcommand{\E}{\mathcal{E}}
\newcommand{\B}{\mathcal{B}}
\renewcommand{\P}{\mathcal{A}}
\newcommand{\pred}{\mathsf{Fam}}
\newcommand{\env}{\mathsf{Env}}
\newcommand{\set}{\mathsf{Set}}
\renewcommand{\S}{\mathcal S}
\renewcommand{\C}{\mathcal{C}}
\newcommand{\D}{\mathcal{D}}
\newcommand{\A}{\mathcal{A}}
\renewcommand{\id}{\mathit{id}}
\newcommand{\map}{\mathsf{map}}
\newcommand{\pid}{\underline{\mathit{id}}}
\newcommand{\pcirc}{\,\underline{\circ}\,}
\newcommand{\pzero}{\underline{0}}
\newcommand{\pone}{\underline{1}}
\newcommand{\psum}{\,\underline{+}\,}
\newcommand{\pinl}{\underline{\mathit{inL}}\,}
\newcommand{\pinr}{\underline{\mathit{inR}}\,}
\newcommand{\ptimes}{\,\underline{\times}\,}
\newcommand{\ppi}{\underline{\pi_1}}
\newcommand{\pppi}{\underline{\pi_2}}
\newcommand{\pmu}{\underline{\mu}}
\newcommand{\semmap}{\mathit{map}}
\newcommand{\subst}{\mathit{subst}}

\newcommand{\tb}[1]{~~ \mbox{#1} ~~}
\newcommand{\listt}[1]{(\mu \phi. \lambda \beta . \onet + \beta \times
  \phi \beta) #1} 
\newcommand{\filtype}{\Nat^\emptyset 
 (\Nat^\emptyset \, \alpha \, \mathit{Bool})\, (\Nat^\emptyset 
  (List \, \alpha) \, (List \, \alpha))} 
\newcommand{\filtypeGRose}{\Nat^\emptyset 
 (\Nat^\emptyset \, \alpha \, \mathit{Bool})\, (\Nat^\emptyset 
  (\mathit{GRose}\,\psi \, \alpha) \, (\mathit{GRose}\,\psi \, (\alpha
  + \onet)))} 
\newcommand{\maplist}{\mathit{map}_{\lambda A. \setsem{\emptyset; \alpha
      \vdash \mathit{List} \, \alpha} \rho[\alpha := A]}} 
\newcommand{\PLeaves}{\mathsf{PLeaves}}
\newcommand{\swap}{\mathsf{swap}}
\newcommand{\reverse}{\mathsf{reverse}}
\newcommand{\Bcons}{\mathit{Bcons}}
\newcommand{\Bnil}{\mathit{Bnil}}

\title[Free Theorems for Nested Types]{Free Theorems for
  %Primitive
  Nested Types} %% [Short Title] is optional; when present,
                         %% will be used in header instead of Full
                         %% Title.
%\titlenote{with title note}             %% \titlenote is optional;
                                        %% can be repeated if necessary;
                                        %% contents suppressed with 'anonymous'
%\subtitle{Subtitle}                     %% \subtitle is optional
%\subtitlenote{with subtitle note}       %% \subtitlenote is optional;
                                        %% can be repeated if necessary;
                                        %% contents suppressed with 'anonymous'


%% Author information
%% Contents and number of authors suppressed with 'anonymous'.
%% Each author should be introduced by \author, followed by
%% \authornote (optional), \orcid (optional), \affiliation, and
%% \email.
%% An author may have multiple affiliations and/or emails; repeat the
%% appropriate command.
%% Many elements are not rendered, but should be provided for metadata
%% extraction tools.

%% Author with single affiliation.
\author{Patricia Johann, Enrico Ghiorzi, and Daniel Jeffries}
%\authornote{with author1 note}          %% \authornote is optional;
%                                        %% can be repeated if necessary
%\orcid{nnnn-nnnn-nnnn-nnnn}             %% \orcid is optional
\affiliation{
%  \position{Position1}
%  \department{Department1}              %% \department is recommended
  \institution{Appalachian State University}            %% \institution is required
%  \streetaddress{Street1 Address1}
%  \city{City1}
%  \state{State1}
%  \postcode{Post-Code1}
%  \country{Country1}                    %% \country is recommended
}
\email{johannp@appstate.edu, ghiorzie@appstate.edu, jeffriesd@appstate.edu}          %% \email is recommended

\begin{document}

\begin{abstract}
This paper considers parametricity and its consequent free theorems
for nested data types.  Rather than representing nested types via
their Church encodings into a higher-kinded extension of System F, we
adopt a functional programming perspective and design a polymorphic
calculus that provides primitives for constructing a robust class of
nested types, including ADTs and all (truly) nested types appearing in
the literature, directly as fixpoints \`{a} la Pitts' PolyPCF.  At the
term level, our calculus supports the usual pattern matching, map
functions, and stylized recursion via folds for all such types. Using
a variation on Atkey's 2012 construction
%of a parametric model
for $F_\omega$,
%that explicitly constructs equality operators rather than merely
%postulating them,
we construct a parametric model for our calculus. However, to ensure
the existence of the fixpoints interpreting nested types, cocontinuity
conditions on the functors underlying these fixpoints must be threaded
throughout our model construction. This turns out to be both delicate
and challenging. Overall, this paper offers a blueprint for
deriving principled parametricity results 
%constructing parametric models
for practical programming languages,
such as Haskell and Agda, that directly construct nested types as
fixpoints rather than encoding them. It also shows how parametric
models can be used to derive useful free theorems for programs over
nested types. Like their counterparts for ADTs, the free theorems our
model derives provide, in particular, a principled alternative to
proving properties of programs over nested types using, e.g., the deep
induction rules for nested types recently developed by Johann and
Polonsky. This is illustrated with several examples. 
\end{abstract}

%\begin{CCSXML}
%<ccs2012>
%<concept>
%<concept_id>10011007.10011006.10011008</concept_id>
%<concept_desc>Software and its engineering~General programming languages</concept_desc>
%<concept_significance>500</concept_significance>
%</concept>
%<concept>
%<concept_id>10003456.10003457.10003521.10003525</concept_id>
%<concept_desc>Social and professional topics~History of programming languages</concept_desc>
%<concept_significance>300</concept_significance>
%</concept>
%</ccs2012>
%\end{CCSXML}
%
%\ccsdesc[500]{Software and its engineering~General programming languages}
%\ccsdesc[300]{Social and professional topics~History of programming languages}
%% End of generated code


%% Keywords
%% comma separated list
%\keywords{keyword1, keyword2, keyword3}  %% \keywords is optional


\maketitle

\section{Introduction}\label{sec:intro}

Suppose we wanted to prove some property of programs over an algebraic
data type (ADT) such as that of lists, coded in Agda
%\footnote{All code appearing in this paper are written in Agda.}
as 

\vspace*{-0.15in}

{\small
\[\begin{array}{l}
\mathtt{data\; List \;(A : Set)\;:\;Set\;where}\\
\hspace*{0.4in}\mathtt{nil\;:\; List\;A}\\
\hspace*{0.4in}\mathtt{Cons\;:\;A \rightarrow List\;A \rightarrow List\;A}
\end{array}\]}
A natural approach to the problem uses structural induction on the
input data structure in question. This requires knowing not just the
definition of the ADT of which the input data structure is an
instance, but also the program text for the functions involved in the
properties to be proved. For example, to prove by induction that
mapping a polymorphic function over a list and then reversing the
resulting list is the same as reversing the original list and then
mapping the function over the result, we unwind the (recursive)
definitions of the $\mathtt{reverse}$ and $\mathtt{map}$ functions
over lists to according to the inductive structure of the input
list. Such data-driven induction proofs over ADTs are so routine that
they are often included in, say, undergraduate functional programming
courses.

An alternative technique for proving results like the above
$\mathtt{map}$-$\mathtt{reverse}$ property for lists is to use
parametricity, a formalization of extensional type-uniformity in
polymorphic languages. Parametricity captures the intuition that a
polymorphic program must act uniformly on all of its possible type
instantiations; it is formalized as the requirement that every
polymorphic program preserves all relations between any pair of types
that it is instantiated with.  Parametricity was originally put forth
by Reynolds~\cite{rey83} for System F~\cite{gr89}, the formal calculus
at the core of all polymorphic functional languages. It was later
popularized as Wadler's ``theorems for free''~\cite{wad89} because it
allows the deduction of many properties of programs in such languages
solely from their types, i.e., with no knowledge whatsoever of the
text of the programs involved. To get interesting free theorems,
Wadler's calculus included, implicitly, built-in list types; indeed,
most of the free theorems in~\cite{wad89} are consequences of
naturality for polymorphic list-processing functions. However,
parametricity can also be used to prove naturality properties for
non-list ADTs, as well as results, like correctness of program
optimizations like {\em short cut fusion}~\cite{glp93,joh02,joh03},
that go beyond simple naturality.
%In addition, parametricity in the presence of ADTs
%{\color{blue} check!}  has further been used to prove i) {\em
%  properties of programs}, e.g., that they perform no illegal
%operations, satisfy certain security criteria~\cite{ep03,rp10}, or are
%observationally equivalent to their compiled forms~\cite{bh09,hd11};
%ii) {\em whole-language properties}, e.g., that they support data
%abstraction and modularity via representation independence,
%\cite{bm05,dr04,jac99,kat11,mr92}, enforce information flow
%policies~\cite{ss01,ts04}, or guarantee privacy~\cite{rp10}; and iii)
%{\em properties of implementations}, e.g., compiler
%correctness~\cite{ab08,bh09,hd11,glpj93}. Parametricity is thus an
%important tool for reasoning about programs in polymorphic languages
%that natively support ADTs. {\color{blue} Be clear about what this
%  means. Re-word?}

This paper is about parametricity and free theorems for a polymorphic
calculus with explicit syntax not just for ADTs, but for nested types
as well. An ADT defines a {\em family of inductive data types}, one
for each input type. For example, the $\mathtt{List}$ data type
definition above defines a collection of data types $\mathtt{List\;
  A}$, $\mathtt{List\; B}$, $\mathtt{List\; (A \times B)}$,
$\mathtt{List\; (\List\;A)}$, etc., each independent of all the
others. By contrast, a nested type~\cite{bm98} is an {\em inductive
  family of data types} that is defined over, or is defined mutually
recursively with, (other) such data types. Since the structures of the
data type at one type can depend on those at other types, the entire
family of types must be defined at once. Examples of nested types
include, trivially, ordinary ADTs, such as list and tree types; simple
nested types, such as the data type {\small
\[\begin{array}{l}
\mathtt{data\; PTree\;(A : Set)\;:\;Set\;where}\\
\hspace*{0.4in}\mathtt{pleaf\;:\;A \rightarrow PTree\;A}\\
\hspace*{0.4in}\mathtt{pnode\;:\;PTree\;(A \times A) \rightarrow PTree\;A}
\end{array}\]}
\hspace{-0.04in}of perfect trees, whose recursive occurrences never
appear below other type constructors; ``deep'' nested
types~\cite{jp20}, such as the data type {\small
\[\begin{array}{l}
\mathtt{data\; Forest\;(A : Set)\;:\;Set\;where}\\
\hspace*{0.4in}\mathtt{fempty\;:\;Forest\;A}\\
\hspace*{0.4in}\mathtt{fnode\;:\; A \rightarrow PTree\;(Forest\;A) \to
Forest\;A}
\end{array}\]}
\hspace{-0.04in}of perfect forests, whose recursive occurrences appear
below type constructors for other nested types; and truly nested
types\footnote{Nested types that are defined over themselves are known
  as {\em truly nested types}.}, such as the data type {\small
\[\begin{array}{l}
\mathtt{data\; Bush\;(A : Set)\;:\;Set\;where}\\
\hspace*{0.4in}\mathtt{bnil\;:\; Bush\;A}\\
\hspace*{0.4in}\mathtt{bcons\;:\;A \rightarrow Bush\;(Bush \; A)
  \rightarrow Bush\;A} 
\end{array}\]}
\hspace{-0.04in}of bushes (also called {\em bootstrapped heaps}
in~\cite{oka99}), whose recursive occurrences appear below their own
type constructors.

Suppose we now want to prove properties of functions over nested
types. We might, for example, want to prove a
$\mathtt{map}$-$\mathtt{reverse}$ property for the functions on
perfect trees in Figure~\ref{fig:funs}, or for those on
bushes\footnote{To define the $\mathtt{foldBush}$ and
  $\mathtt{mapBush}$ functions in Figure~\ref{fig:funs2} it is
  necessary to turn off Agda's termination checker.} in
Figure~\ref{fig:funs2}. A few well-chosen examples quickly convince us
that such a property should indeed hold for perfect trees, and,
drawing inspiration from the situation for ADTs, we easily construct a
proof by induction on the input perfect tree. To formally establish
this result, we could even prove it in Coq or Agda: each of these
provers actually generates an induction rule for perfect trees, and
the rule it generats gives the expected result because proving
properties of perfect trees requires only that we induct over the
top-level perfect tree in the recursive position, leaving any data
internal to the input tree untouched.

%want to prove that $\mathtt{reversePTree}$ commutes with
%$\mathtt{mapPTree}$ or that $\mathtt{reverseBush}$ commutes with
%$\mathtt{mapBush}$. It is not hard to convince oneself of the former:
%a few well-chosen examples show clearly why mapping a polymorphic
%%function over a perfect tree and then reversing the data at the leaves
%of the resulting perfect tree is indeed the same result as mapping
%that same function over the result of reversing the data at the leaves
%of the original perfect tree.

Unfortunately, it is nowhere near as clear that analogous intuitive or
formal inductive arguments can be made for the
$\mathtt{map}$-$\mathtt{reverse}$ property for bushes. Indeed, a proof
by induction on the input bush must recursively induct over the bushes
that are internal to the top-level bush in the recursive position.
This is sufficiently delicate that no induction rule for bushes or
other truly nested types was known until very recently, when {\em deep
  induction}~\cite{jp20} was developed as a way to induct over {\em
  all} of the structured data present in an input. Deep induction thus
not only gave the first principled and practically useful structural
induction rules for bushes and other truly nested types, and has also
opened the way for incorporating automatic generation of such rules
for (truly) nested data types --- and, eventually, even GADTs --- into
modern proof assistants.
\begin{figure*}
\hspace*{-0.5in}
\resizebox{0.35\linewidth}{!}{
\begin{minipage}[t]{0.5\textwidth}
\[\begin{array}{l}
\mathtt{reversePTree : \forall \{A : Set\}  \rightarrow PTree\; A
  \rightarrow PTree \;A}\\  
\mathtt{reversePTree \; \{A\} = foldPTree \; \{A\}\; \{PTree\}}\\
\hspace*{1.3in} \mathtt{pleaf}\\
\hspace*{1.3in} \mathtt{( \lambda p \rightarrow pnode\; (mapPTree\;
  swap\; p) )}\\  
\\
\mathtt{foldPTree : \forall \{A : Set\} \rightarrow \{F : Set \rightarrow Set \}
  \rightarrow}\\
\hspace*{0.8in} \mathtt{( \{B : Set\} \rightarrow B \rightarrow F B) \rightarrow}\\
\hspace*{0.8in} \mathtt{(\{B : Set\} \rightarrow F (B \times B)
  \rightarrow F B) \rightarrow}\\
\hspace*{0.8in}\mathtt{PTree \;A \rightarrow F\;A}\\
\mathtt{foldPTree\; n\; c\; (pleaf\; x) = n \;x}\\
\mathtt{foldPTree\; n\; c \;(pnode\; p) = c\; (foldPTree \;n \;c \;p)}\\
\\
\mathtt{mapPTree : \forall \{A\, B : Set\} \rightarrow (A \rightarrow B)
  \rightarrow PLeaves\; A \rightarrow PLeaves\; B}\\ 
\mathtt{mapPTree \;f\; (pleaf \;x) = pleaf\; (f\; x)}\\
\mathtt{mapPTree\; f \;(pnode\; p) = pnode\; (mapPTree \;(\lambda p
  \rightarrow (f (\pi_1 \,p), f (\pi_2\,p)))\; p)}\\ 
\\
\mathtt{swap : \forall \{A : Set\} \rightarrow (A \times A)
  \rightarrow (A \times A)}\\ 
\mathtt{swap \;(x,y) = (y,x)}
\end{array}\]
\caption{$\mathtt{reversePTree}$ and auxiliary functions in Agda}\label{fig:funs} 
\end{minipage}}
\quad\quad\quad\quad\quad\quad\quad\quad
\resizebox{0.35\linewidth}{!}{
\begin{minipage}[t]{0.5\textwidth}
\[\begin{array}{l}
  \mathtt{reverseBush : \forall \{A : Set\} \rightarrow Bush\; A
  \rightarrow Bush\; A}\\ 
\mathtt{reverseBush \;\{A\} = foldBush\; \{A\}\; \{Bush\}\; bnil \;balg}\\
\\
\mathtt{foldBush : \forall \{A : Set\} \rightarrow \{F : Set \rightarrow
  Set\} \rightarrow}\\
\hspace*{0.5in} \mathtt{(\{B : Set\} \rightarrow F B) \rightarrow}\\
\hspace*{0.5in} \mathtt{(\{B : Set\} \rightarrow B \rightarrow F\;
  (F\; B) \rightarrow F\; B) \rightarrow}\\ 
\hspace*{0.5in} \mathtt{Bush\; A \rightarrow F \;A}\\
\mathtt{foldBush\; bn\; bc\; bnil = bn}\\
\mathtt{foldBush \;bn \;bc \;(bcons\; x \;bb) =}\\
\hspace*{0.5in} \mathtt{bc\; x \;(foldBush \;bn\; bc
  \;(mapBush \;(foldBush\; bn\; bc) \;bb))}\\
\\
\mathtt{mapBush : \forall \{A \,B : Set\} \to (A \to B) \to (Bush\, A) \to
  (Bush\, B)}\\ 
\mathtt{mapBush\; \_ \; bnil = bnil}\\
\mathtt{mapBush\; f\; (bcons\; x \;bb) = bcons \;(f\, x) \; (mapBush\;
  (mapBush\, f) \; bb)}\\ 
\\
\mathtt{balg : \forall \{B : Set\} \rightarrow B \rightarrow Bush
  \;(Bush\; B) \rightarrow Bush\; B}\\
\mathtt{balg \;x \;bnil = bcons\; x \;bnil}\\
\mathtt{balg \;x\; (bcons \;bnil \;bbbx) = bcons\; x \;(bcons \;bnil\; bbbx)}\\
\mathtt{balg\; x \;(bcons \;(bcons\; y \;bx)\; bbbx) =}\\
\hspace*{0.5in} \mathtt{bcons\; y\; (bcons\; (bcons \;x \;bx) \;bbbx)}\\
\end{array}\]
\caption{$\mathtt{reverseBush}$ and auxiliary
  functions in Agda}\label{fig:funs2} 
\end{minipage}}
\end{figure*}

\vspace*{-0.2in}

Of course it is great to know that we {\em can}, at last, prove
properties of programs over (truly) nested types by induction.  But
recalling that inductive proofs over ADTs can sometimes be
circumvented in the presence of parametricity, we might naturally ask:
\begin{verse}
{\em Can we derive properties of functions over (truly) nested types
  from parametricity?}
\end{verse}
This paper answers the above question in the affirmative by
constructing a parametric model for a polymorphic calculus providing
primitives for {\em constructing} nested types
%(including ADTs)
directly via
%type-level
recursion --- rather than representing them indirectly by Church
encodings as in most polymorphic calculi.

We introduce our calculus in Section~\ref{sec:calculus}.  At the type
level, it is the level-2-truncation of the higher-kinded calculus
from~\cite{jp19}, augmented with a primitive type of natural
transformations. To construct nested types, it constructs type
expressions not just from standard type variables, but also from type
constructor variables of various arities, and includes an explicit
$\mu$-construct for type-level recursion with respect to these
variables. The class of nested types thus constructed is very robust
and includes all (truly) nested types known from the literature. In
Section~\ref{sec:type-interp} we give set and relational
interpretations for the types From Section~\ref{sec:calculus}. As is
usual when modeling parametricity, types are interpreted as functors
from environments interpreting their type variable contexts to set or
relations, as appropriate. But in order to ensure that these functors
satisfy the cocontinuity properties needed to know that the fixpoints
interpreting $\mu$-types exist, set environments must map each $k$-ary
type constructor variable to an appropriately cocontinuous $k$-ary
functor on sets and relation environments must map each $k$-ary type
constructor variable to an appropriately cocontinuous $k$-ary relation
transformer, and these cocontinuity conditions must be threaded
throughout the type interpretations in such a way that the resulting
model is guaranteed to satisfy an appropriate Identity Extension Lemma
(Theorem~\ref{thm:iel}). Properly progagating the cocontinuity
conditions turns out to be both subtle and challenging, and
Section~\ref{sec:iel}, where it is done, is where the bulk of the work
in constructing our model lies.  At the term level, our calculus
includes primitive constructs for the actions on morphisms of the
functors interpreting types, initial algebras for fixpoints of these
functors, and structured recursion over elements of these initial
algebras (i.e., $\map$, $\tin$, and $\fold$ constructs,
respectively). While our calculus does not support general recursion
at the term level, it is strongly normalizing, so does perhaps edge us
toward the kind of provably total practical programming language
proposed at the end of~\cite{wad89}. In Section~\ref{sec:term-interp},
we give set and relational interpretations for the terms of our
calculus. As usual in parametric models, terms are interpreted as
natural transformations from interpretations of the term contexts in
which they are formed to the interpretations of their types, and these
must cohere in what is essentially a fibrational way. Immediately from
the definitions of our interpretations we prove in
Section~\ref{sec:Nat-type-terms} a scheme deriving free theorems that
are consequences of naturality of functions that are polymorphic over
nested types. This scheme is very general, is parameterized over both
the data type and the polymorphic function at hand, and has each of
the above $\mathtt{map}$-$\mathtt{reverse}$ theorems as instances. The
relationship between naturality and parametricity has long been of
interest, and our inclusion of a primitive type of natural
transformations allows us to clearly delineate those free theorems
that are consequences of naturality, and thus hold even in
non-parametric models, from those that require the full power of
parametricity. In Section~\ref{sec:thms} we prove that our model
satisfies an Abstraction Theorem (Theorem~\ref{thm:abstraction}), and
we derive several of this latter kind of free theorem from it in
Section~\ref{sec:ftnt}. Specifically, we state and prove
(non-)inhabitation results in Sections~\ref{sec:bottom}
and~\ref{sec:identity}, a free theorem for the type of a filter
function on generalized rose trees in Section~\ref{sec:ft-adt}, the
correctness of short cut fusion for lists in
Section~\ref{sec:short-cut}, and its generalization to nested types in
Section~\ref{sec:short-cut-nested}.

There is a long line of work on categorical models of parametricity
for System F; see,
e.g.,~\cite{bfss90,bm05,dr04,gjfor15,has94,jac99,mr92,rr94}.  To our
knowledge, all categorical models that treat (algebraic) data types do
so via their Church encodings, verifying in the
%Much of this work does ultimately extend parametricity to data types
%that are modeled as fixpoints of first-order functors, but to our
%knowledge this has never been done by building the data types directly
%into the calculus whose parametricity is to be modeled
%categorically. Indeed, all extensions we know of use the categorical
%equivalent of Atkey's approach. That is, they verify in the
just-constructed parametric model that
%the Church encoding constructed
%from a type constructor is interpreted as the least fixpoint of the
%functor interpreting that type constructor.
the Church encoding of each such type is interpreted as the least
fixpoint of the (first-order) functor interpreting the type
constructor from which its Church encoding was constructed.
%This is the categorical analogue of the approach in~\cite{atk12}
%described above.
The present paper draws on this rich tradition of categorical models
of parametricity for System F, but treats nested types as well as
ADTs, and is the first to treat these types by direct construction via
primitives rather than by Church encodings. This requires that we
modify the type calculus to ensure that
%System F's full impredicative polymorphism interacts reasonably with
%functoriality,
functoriality is guaranteed syntactically, and that functoriality is
reflected in the standard model construction so that the existence of
the fixpoints by which nested types are to be interpreted is ensured.
%parametricity to primitive nested types {\em as the model is being
%constructed} by taking care to ensure that System F's full
%impredicative polymorphism does not interfere with the functoriality
%needed to support primitive nested types.

Like our calculus, Pitts' PolyPCF~\cite{pit98,pit00}
%extends parametricity from pure System F to System F augmented with
%to a calculus that
includes primitives for constructing data types directly. Only list
types are added in~\cite{pit00}, but other polynomial ADTs are easily
included as in~\cite{pit98}. Part of Pitts' motivation is to show that
ADTs and their Church encodings have the same operational behavior in
his calculus. We cannot even ask this question about our calculus,
which cannot express Church encodings of even simple ADTs such as
list, or pair, or sum types, since these are not functorial;
nevertheless, we do prove the correctness of short cut fusion for
nested types, whose operational analogue is valid in the parametric
model Pitts constructs and is the means by which he proves his
equivalence result. It would be interesting to know what operational
analogues of functoriality and cocontinuity are needed to extend
Pitts' parametricity results from a calculus with primitives for
constructing data types modeled as fixpoints of first-order functors
to one that also provides such primitives for data types modeled as
fixpoints of higher-order functors.

We are not the first to consider parametricity at higher
kinds. Atkey~\cite{atk12} constructs a parametric model for full
System $F_\omega$, but within the impredicative Calculus of Inductive
Constructions (iCIC) rather than in a semantic category. His
construction is in some ways similar to ours, but it represents data
types using Church encodings rather than constructing them via
primitives. Since his model is entirely syntactic, his syntactic
``functors'', whose associated $\mathit{fmap}$ functions representing
their functorial actions must be {\em given} together with their
underlying type constructors, cannot be interpreted as semantic
functors. The associated Church encodings therefore cannot be
interpreted as their fixpoints, so Atkey need not, and does not,
impose cocontinuity conditions on his model to ensure that these
fixpoints exist.  Nevertheless, Atkey does verify the existence of
initial algebras for ``functors'' in iCIC. He does not indicate which
type constructors support the kinds of $\mathit{fmap}$ functions
needed to ensure they are ``functors'', but we suspect spelling this
out explicitly would result in a full higher-kinded extension of the
calculus presented here.

\section{The Calculus}\label{sec:calculus}

\subsection{Types}
For each $k \ge 0$, we assume countable sets $\tvars^k$ of \emph{type
  constructor variables of arity $k$} and $\fvars^k$ of
\emph{functorial variables of arity $k$}, all mutually disjoint.
%disjoint for distinct $k$ and disjoint from each other.
The sets of all type constructor variables and functorial variables
are $\tvars = \bigcup_{k \ge 0} \tvars^k$ and $\fvars = \bigcup_{k \ge
  0} \fvars^k$, respectively, and a \emph{type variable} is any
element of $\tvars \cup \fvars$.  We use lower case Greek letters for
type variables, writing $\phi^k$ to indicate that $\phi \in \tvars^k
\cup \fvars^k$, and omitting the arity indicator $k$ when convenient,
unimportant, or clear from context. We reserve letters from the
beginning of the alphabet to denote type variables of arity $0$, i.e.,
elements of $\tvars^0 \cup \fvars^0$. We write $\overline{\zeta}$ for
either a set $\{\zeta_1,...,\zeta_n\}$ of type constructor variables
or a set of functorial variables when the cardinality $n$ of the set
is unimportant or clear from context. If $\Pos\,$ is a set of type
variables we write $\Pos, \overline{\phi}$ for $\Pos\, \cup
\overline{\phi}$ when $\Pos\, \cap \overline{\phi} = \emptyset$.  We
omit the vector notation for a singleton set, thus writing $\phi$,
instead of $\overline{\phi}$, for $\{\phi\}$.
\begin{dfn}
Let $V$ be a finite subset of\, $\tvars$, $\Pos$ be a finite
subset of\, $\fvars$, $\overline{\alpha}$ be a finite subset of\,
$\fvars^0$ disjoint from $\Pos$, and $\phi^k \in \fvars^k
\setminus \Pos$.  The set $\mcF^\Pos(V)$ of {\em functorial
  expressions} over $\Pos$ and $V$ are given by
\begin{align*}
  \mcF^\Pos(V) \; ::= \; %\T(V)
  &\mid
\zerot \mid \onet 
\mid \Nat^{\ol{\alpha}} \, \mcF^{\overline{\alpha}}(V) \;
\mcF^{P,\ol{\alpha}}(V) %\mid V \ol{\T(V)}
\mid \Pos\; \ol{\mcF^\Pos(V)}  \,
\mid V\, \ol{\mcF^\Pos(V)}  
\mid \mcF^{\Pos}(V) + \mcF^\Pos(V)\\
&\mid \mcF^{\Pos}(V) \times \mcF^\Pos(V) \mid \left(\mu
\phi^{~k}. \lambda \alpha_1...\alpha_k.  
\mcF^{\Pos,\alpha_1,...,\alpha_k,\phi}(V)\right)
  \ol{\mcF^{\Pos}(V)}
\end{align*}
\end{dfn}
\noindent
A \emph{type} over $\Pos$ and $V$ is any element of % $\T(V) \cup
$\F^\Pos(V)$.

The notation for types entails that an application
$\tau\tau_1...\tau_k$ is allowed only when $\tau$ is a type variable
of arity $k$, or $\tau$ is a subexpression of the form $\mu
\phi^{k}.\lambda \alpha_1...\alpha_k.\tau'$. Moreover, if $\tau$ has
arity $k$ then $\tau$ must be applied to exactly $k$ arguments.
Accordingly, an overbar indicates a sequence of subexpressions whose
length matches the arity of the type applied to it.  The fact that
types are always in \emph{$\eta$-long normal form} avoids having to
consider $\beta$-conversion at the level of types. In a subexpression
$\Nat^{\ol{\alpha}}\sigma\,\tau$, the $\Nat$ operator binds all
occurrences of the variables in $\ol{\alpha}$ in $\sigma$ and
$\tau$. Similarly, in a subexpression $\mu \phi^k.\lambda
\ol{\alpha}.\tau$, the $\mu$ operator binds all occurrences of the
variable $\phi$, and the $\lambda$ operator binds all occurrences of
the variables in $\ol{\alpha}$, in the body $\tau$.

A {\em type constructor context} is a finite set $\Gamma$ of type
constructor variables, and a {\em functorial context} is a finite set
$\Phi$ of functorial variables. In Definition~\ref{def:wftypes}, a
judgment of the form $\Gamma;\Phi \vdash \tau$ indicates that the
type $\tau$ is intended to be functorial in the variables in $\Phi$
but not necessarily in those in $\Gamma$.
\begin{dfn}\label{def:wftypes}
The formation rules for the set $\F \subseteq \bigcup_{V \subseteq
  \tvars, \Pos\, \subseteq \fvars}\F^\Pos(V)$ of\, {\em well-formed
  type expressions} are

\vspace*{-0.2in}

\[\begin{array}{cc}
\AXC{\phantom{$\Gamma,\Phi$}}
\UIC{$\Gamma;\Phi \vdash \zerot$}
\DisplayProof
&
\AXC{\phantom{$\Gamma,\Phi$}}
\UIC{$\Gamma;\Phi \vdash \onet$}
\DisplayProof
\end{array}\]
\[\begin{array}{c}
\AXC{$\Gamma;\ol{\alpha^0} \vdash \sigma$}
\AXC{$\Gamma;\Phi,\ol{\alpha^0}  \vdash \tau$}
\BIC{$\Gamma;\Phi \vdash \Nat^{\ol{\alpha^0}}\sigma \,\tau$}
\DisplayProof
\\[3ex]
\AXC{$\phi^k \in \Gamma \cup \Phi$}
\AXC{$\quad\quad\ol{\Gamma;\Phi \vdash \tau}$}
\BIC{$\Gamma;\Phi \vdash \phi^k \ol{\tau}$}
\DisplayProof
\\[3ex]
\AXC{$\Gamma;\Phi,\ol{\alpha},\phi^k \vdash \tau$}
\AXC{$\quad\quad\ol{\Gamma;\Phi \vdash \tau}$}
\BIC{$\Gamma;\Phi \vdash (\mu \phi^k.\lambda \ol{\alpha}. \,\tau)\ol{\tau}$}
\DisplayProof
\end{array}\]


\vspace*{0.08in}

\[\begin{array}{cc}
\AXC{$\Gamma;\Phi \vdash \sigma$}
\AXC{$\Gamma;\Phi \vdash \tau$}
\BIC{$\Gamma; \Phi \vdash \sigma + \tau$}
\DisplayProof
&
\AXC{$\Gamma;\Phi \vdash \sigma$}
\AXC{$\Gamma;\Phi \vdash \tau$}
\BIC{$\Gamma; \Phi \vdash \sigma \times \tau$}
\DisplayProof
\end{array}\]
%\noindent
%A type $\tau$ is {\em well-formed} if it is
%either a well-formed type constructor expression or a
%well-formed functorial expression.
\end{dfn}
If $\tau$ is a closed type we may write $\vdash \tau$, rather than
$\emptyset;\emptyset \vdash \tau$, for the judgment that it is
well-formed.  Definition~\ref{def:wftypes} ensures that the expected
weakening rules for well-formed types hold, although weakening does
not change the contexts in which $\Nat$-types can be formed. If
$\Gamma;\emptyset \vdash \sigma$ and $\Gamma;\emptyset \vdash \tau$,
then our rules allow formation of the type $\Gamma;\emptyset \vdash
\Nat^\emptyset\sigma \,\tau$. Since $\Gamma; \emptyset \vdash
\Nat^{\ol \alpha}\sigma\,\tau$ represents a natural transformation in
$\ol \alpha$ from $\sigma$ to $\tau$, the type $\Gamma;\emptyset
\vdash \Nat^\emptyset\sigma \,\tau$ represents the arrow type
$\Gamma \vdash \sigma \to \tau$ in our calculus. Our calculus
similarly represents the $\forall$-type $\Gamma; \emptyset
\vdash \forall \ol\alpha . \tau$ as $\Gamma; \emptyset \vdash
\Nat^{\ol\alpha} \,\onet \,\tau$.
%However, if $\ol\alpha$ is non-empty then $\tau$ cannot be of the
%form $\Nat^{\ol\beta} H \, K$ since $\Gamma; \ol\alpha \vdash
%\Nat^{\ol\beta} H \, K$ is not a valid type judgment in our calculus
%(except by weakening).

Definition~\ref{def:wftypes} allows the formation of all of the
(closed) nested types from the introduction:
\[\begin{array}{lll}
\mathit{List}\, \alpha & = & \mu \beta. \,\onet + \alpha \times
\beta\; = \; (\mu \phi. \lambda \beta.\,\onet + \beta \times \phi
\beta)\,\alpha\\ 
\mathit{PTree}\,\alpha & = & (\mu \phi. \lambda \beta.\,\beta +
\phi\,(\beta \times \beta))\,\alpha\\
\mathit{Forest}\,\alpha & = & (\mu \phi. \lambda \beta. \,\onet +
\beta \times \mathit{PTree}\,(\phi \beta))\,\alpha\\ 
\mathit{Bush}\,\alpha & = & (\mu \phi.\lambda \beta. \,\onet + \beta
\times \phi\,(\phi\beta))\,\alpha 
\end{array}\]
Each of these types can be considered either functorial in $\alpha$ or
not, according to whether $\alpha \in \Gamma$ or $\alpha \in \Phi$.
For example, if $\emptyset;\alpha \vdash \mathit{List} \,\alpha$, then
the type $\vdash \Nat^\alpha \onet\, (\mathit{List}\,\alpha)$ is
well-formed; if $\alpha;\emptyset \vdash \List \,\alpha$, then it is
not. If $\mathit{Tree}\,\alpha\,\gamma = \mu \,\beta.\, \alpha + \beta
\times \gamma \times \beta$, then Definition~\ref{def:wftypes} also
allows the derivation of, e.g., the type $\gamma; \emptyset \vdash
\Nat^\alpha (\mathit{List}\, \alpha) \, (\mathit{Tree} \, \alpha\,
\gamma)$ representing a natural transformation from lists to trees
that is natural in $\alpha$ but not necessarily in $\gamma$. We
emphasize that types can be functorial in variables of arity greater
than $0$. For example, the type $\mathit{GRose}\, \phi \, \alpha =
\mu\beta . \alpha \times \phi \beta$ can be functorial in $\phi$ if
$\phi \in \Phi$. As usual, whether $\phi \in \Gamma$ or $\phi \in
\Phi$ determines whether types such as $\Nat^\alpha ( \mathit{GRose}
\, \phi \, \alpha)\, (List\, \alpha)$ are well-formed. But even if
$\mathit{GRose}$ is functorial in $\phi$, it still cannot be the
(co)domain of a $\Nat$ type representing a natural transformation in
$\phi$. This is because our calculus does not allow naturality in
variables of arity greater than $0$.

Definition~\ref{def:wftypes} explicitly allows well-formed types to be 
%%considers types in $\T$ to be types in $\F$ that are
functorial in no variables.
%This allows the formation of types such as
%$\mathit{List}\,(\sigma \to \tau)$ and $\mathit{PTree}\, (\forall
%\alpha.\tau)$.
Functorial variables in a well-formed type $\tau$ can also be demoted
to non-functorial status. The proof is by induction on the structure
of $\tau$.

\begin{lemma}
If\, $\Gamma; \Phi, \phi^k \vdash \tau$, then $\Gamma, \psi^k; \Phi
\vdash \tau[\phi^k :== \psi^k]$ is also derivable. Here, $\tau[\phi
  :== \psi]$ is the textual replacement of $\phi$ in $\tau$, meaning
that all occurences of $\phi\ol\sigma$ in $\tau$ become
$\psi\ol\sigma$.
\end{lemma}

In addition to textual replacement, we also have a proper substitution
operation on types. If $\tau$ is a type over $P$ and $V$, if $\Pos$
and $V$ contain only type variables of arity $0$, and if $k=0$ for
every occurrence of $\phi^k$ bound by $\mu$ in $\tau$, then we say
that $\tau$ is {\em first-order}; otherwise we say that $\tau$ is {\em
  second-order}.  Substitution for first-order types is the usual
capture-avoiding textual substitution. We write $\tau[\alpha :=
  \sigma]$ for the result of substituting $\sigma$ for $\alpha$ in
$\tau$, and $\tau[\alpha_1 := \tau_1,...,\alpha_k := \tau_k]$, or
$\tau[\ol{\alpha := \tau}]$ when convenient, for $\tau[\alpha_1 :=
  \tau_1][\alpha_2 := \tau_2,...,\alpha_k := \tau_k]$. Substitution
for second-order types is defined below, where we adopt a similar
notational convention for vectors of types. We note that it is not
correct to substitute along non-functorial variables.

\begin{dfn}\label{def:second-order-subst}
If \,$\Gamma; \Phi,\phi^k \vdash F$ with $k \geq 1$, and if\,
$\Gamma;\Phi, \ol{\alpha} \vdash H$ with $|\ol\alpha| = k$, then
$\Gamma;\Phi \vdash F[\phi :=_{\ol{\alpha}} H]$.  Similarly, if
\,$\Gamma, \phi^k; \Phi \vdash F$ with $k \geq 1$, and if\, $\Gamma;
\Psi,\ol{\alpha} \vdash H$ with $|\ol\alpha| = k$ and $\Phi \cap \Psi
= \emptyset$, then $\Gamma,\Psi;\Phi \vdash F[\phi :=_{\ol{\alpha}}
  H]$. Here, the operation $(\cdot)[\phi := H]$ of {\em second-order
  type substitution} is defined by:
\[\begin{array}{lll}
\zerot[\phi :=_{\ol{\alpha}} H] & = & \zerot\\[0.5ex]
\onet[\phi :=_{\ol{\alpha}} H] & = & \onet\\[0.25ex]
(\Nat^{\ol\beta} G \,K)[\phi :=_{\ol{\alpha}} H]
& = & \Nat^{\ol\beta}\, (G[\phi :=_{\ol{\alpha}} H]) \,(K[\phi
  :=_{\ol{\alpha}} H])\\
(\psi\ol{\tau})[\phi :=_{\ol{\alpha}} H] & = &
\left\{\begin{array}{ll}
\psi \,\ol{\tau[\phi :=_{\ol{\alpha}} H]} & \mbox{if } \psi \not = \phi\\
  H[\ol{\alpha  := \tau[\phi :=_{\ol{\alpha}} H]}] 
  & \mbox{if } \psi = \phi
\end{array}\right.\\[2.8ex]
(\sigma + \tau)[\phi :=_{\ol{\alpha}} H] & = & \sigma[\phi
  :=_{\ol{\alpha}} H] + \tau[\phi :=_{\ol{\alpha}} H]\\[0.5ex] 
(\sigma \times \tau)[\phi :=_{\ol{\alpha}} H] & = &
\sigma[\phi :=_{\ol{\alpha}} H] \times \tau[\phi
  :=_{\ol{\alpha}} H]\\[0.5ex]   
((\mu \psi. \lambda \ol{\beta}.\, G)\ol{\tau})[\phi :=_{\ol{\alpha}}
  H] & = & (\mu \psi. \lambda \ol{\beta}. \,G[\phi :=_{\ol{\alpha}}
  H])\, \ol{\tau[\phi :=_{\ol{\alpha}} H]}
\end{array}\]
\end{dfn}
\noindent
We omit the variable subscripts in second-order type constructor
substitution when convenient.

\subsection{Terms}\label{sec:terms}
We assume an infinite set $\cal V$ of term variables disjoint from
$\tvars$ and $\fvars$. If $\Gamma$ be a type constructor context and
$\Phi$ is a functorial context, then a {\em term context for $\Gamma$
  and $\Phi$} is a finite set of bindings of the form $x : \tau$,
where $x \in {\cal V}$ and $\Gamma; \Phi \vdash \tau$. We adopt
the same conventions for denoting disjoint unions and for vectors in
term contexts as for type constructor contexts and functorial
contexts.

\begin{dfn}\label{def:well-formed-terms}
Let $\Delta$ be a term context for $\Gamma$ and $\Phi$.  The formation
rules for the set of\, {\em well-formed terms over $\Delta$} are
{\color{blue} Say term calc is new.}
\[\begin{array}{ccc}
%\AXC{$\Gamma;\emptyset \vdash \tau : \T$}
%\UIC{$\Gamma;\emptyset\,|\, \Delta,x :\tau \vdash x : \tau$}
%\DisplayProof
\AXC{$\Gamma;\Phi \vdash \tau$}
\UIC{$\Gamma;\Phi \,|\, \Delta,x :\tau \vdash x : \tau$}
\DisplayProof
&
\AXC{$\Gamma;\Phi \,|\, \Delta \vdash t : \zerot$}
\AXC{$\Gamma;\Phi \vdash \tau$}
\BIC{$\Gamma;\Phi \,|\, \Delta \vdash \bot_\tau t  : \tau$}
\DisplayProof
&
\AXC{$\phantom{\Gamma;\Phi}$}
\UIC{$\Gamma;\Phi \,|\, \Delta \vdash \top : \onet$}
\DisplayProof\\\\
\end{array}\]
\[\begin{array}{cc}
\AXC{$\Gamma;\Phi \,|\, \Delta \vdash s: \sigma$}
\UIC{$\Gamma;\Phi \,|\, \Delta \vdash \inl \,s: \sigma + \tau$}
\DisplayProof
&
\AXC{$\Gamma;\Phi \,|\, \Delta \vdash t : \tau$}
\UIC{$\Gamma;\Phi \,|\, \Delta \vdash \inr \,t: \sigma + \tau$}
\DisplayProof\\\\
\end{array}\]
\[\begin{array}{c}
\AXC{$\Gamma; \Phi \vdash \tau,\sigma$}
\AXC{$\Gamma;\Phi \,|\, \Delta \vdash t : \sigma + \tau$}
\AXC{$\Gamma;\Phi \,|\, \Delta, x : \sigma \vdash l : \gamma \hspace{0.3in} \Gamma;\Phi \,|\, \Delta, y : \tau \vdash r : \gamma$}
\TIC{$\Gamma;\Phi~|~\Delta \vdash \case{t}{{\color{blue} \mathsf{inl
        \,}} x \mapsto l}{{\color{blue} \mathsf{inr \,}} y \mapsto r} : \gamma$}
\DisplayProof
\end{array}\]

\vspace*{0.05in}

\[\begin{array}{lll}
\AXC{$\Gamma;\Phi \,|\, \Delta \vdash s: \sigma$}
\AXC{$\Gamma;\Phi \,|\, \Delta \vdash t : \tau$}
\BIC{$\Gamma;\Phi \,|\, \Delta \vdash (s,t) : \sigma \times \tau$}
\DisplayProof
&
\AXC{$\Gamma;\Phi \,|\, \Delta \vdash t : \sigma \times \tau$}
\UIC{$\Gamma;\Phi \,|\, \Delta \vdash \pi_1 t : \sigma$}
\DisplayProof
&
\AXC{$\Gamma;\Phi \,|\, \Delta \vdash t : \sigma \times \tau$}
\UIC{$\Gamma;\Phi \,|\, \Delta \vdash \pi_2 t : \tau$}
\DisplayProof
\end{array}\]

\vspace*{0.05in}

\[\begin{array}{c}
%\AXC{$\Gamma; \emptyset \vdash \Nat^{\ol{\alpha}} \,F\,G : \T$}
\AXC{$\Gamma; \ol{\alpha} \vdash F$}
\AXC{$\Gamma; \Phi, \ol{\alpha} \vdash G$}
\AXC{$\Gamma; \Phi, \ol{\alpha} \,|\, \Delta, x : F \vdash t: G$} 
\TIC{$\Gamma; \Phi
  %\emptyset
  \,|\, \Delta \vdash L_{\ol{\alpha}} x.t : \Nat^{\ol{\alpha}} \,F \,G$}
\DisplayProof
\\\\
\AXC{$\Gamma; \Phi
  %\emptyset
  \,|\, \Delta \vdash t : \Nat^{\ol{\alpha}} \,F \,G$}
\AXC{$\ol{\Gamma;\Phi \vdash \tau}$}
\AXC{$\Gamma;\Phi \,|\, \Delta \vdash s: F[\overline{\alpha := \tau}]$}
\TIC{$\Gamma;\Phi \,|\, \Delta \vdash t_{\ol{\tau}} s:
  G[\overline{\alpha := \tau}]$}
\DisplayProof
\\\\
\AXC{$\Gamma; \ol{\phi}, \ol{\gamma} \vdash H$}
\AXC{$\ol{\Gamma; \ol{\beta},\ol{\gamma} \vdash F}$}
\AXC{$\ol{\Gamma; \Phi, \ol{\beta},\ol{\gamma} \vdash
    G}$\hspace*{0.3in}$\ol{\Gamma;\Phi~|~\Delta \vdash t :
  \Nat^{\ol{\beta},\ol{\gamma}}\,F\,G}$}
\TIC{$\Gamma; \Phi
  %\emptyset
  ~|~\Delta
%  \ol{ x : \Nat^{\ol{\beta},\ol{\gamma}}\,F\,G}
  %  \emptyset
  \vdash \map^{\ol{F},\ol{G}}_H \, \ol t :
%  \Nat^\emptyset\;(\ol{\Nat^{\ol{\beta},\ol{\gamma}}\,F\,G})\;
  \Nat^{\ol{\gamma}}\,H[\ol{\phi :=_{\ol{\beta}} F}]\,H[\ol{\phi
      :=_{\ol{\beta}} G}]$} 
\DisplayProof
\\\\
%\AXC{$\Gamma;\Phi \,|\, \Delta \vdash t : H[\phi := (\mu \phi.\lambda 
%  \ol{\alpha}.H)\ol{\beta}][\ol{\alpha := \tau}]$}
%\AXC{$\ol{\Gamma;\Phi \vdash \tau : \F}$}
\AXC{$\Gamma; \phi, \ol{\alpha},\ol{\gamma} \vdash H$}
\UIC{$\Gamma; \emptyset  \,|\, \emptyset \vdash \tin_H :
  \Nat^{\ol{\beta},\ol{\gamma}} H[\phi :=_{\ol{\beta}} (\mu
    \phi.\lambda \ol{\alpha}.H)\ol{\beta}][\ol{\alpha := \beta}]\,(\mu
  \phi.\lambda \ol{\alpha}.H)\ol{\beta}$}
\DisplayProof
\\\\
\AXC{$\Gamma; \phi,\ol{\alpha}, \ol{\gamma} \vdash H$}
\AXC{$\Gamma; \ol{\beta}, \ol{\gamma} \vdash F$}
%\AXC{$\Gamma;\emptyset \,|\, \Delta \vdash t :
%  \Nat^{\ol{\beta}, \ol{\gamma}}\,H[\phi :=_{\ol{\beta}} F][\ol{\alpha
%      := \beta}]\,F$} 
%\TIC{$\Gamma;\emptyset \,|\, \Delta \vdash
%  \fold_{H,F}\, t : \Nat^{\ol{\beta}, \ol{\gamma}}\,((\mu
%  \phi.\lambda \ol{\alpha}.H)\ol{\beta})\,F$} 
\BIC{$\Gamma; \emptyset  \,|\, \emptyset \vdash \fold^F_H :
  \Nat^\emptyset\; (\Nat^{\ol{\beta}, \ol{\gamma}}\,H[\phi
    :=_{\ol{\beta}} F][\ol{\alpha := \beta}]\,F)\; (\Nat^{\ol{\beta},
    \ol{\gamma}}\,(\mu \phi.\lambda \ol{\alpha}.H)\ol{\beta})\,F)$}
\DisplayProof
\end{array}\]
\end{dfn}

In the rule for $L_{\ol{\alpha}}x.t$, the $L$ operator binds all
occurrences of the type variables in $\ol{\alpha}$ in the type of the
term variable $x$ and in the body $t$, as well as all occurrences of
$x$ in $t$. In the rule for $t_{\ol\tau} s$ there is one functorial
expression $\tau$ for every functorial variable $\alpha$. In the rule
for $\map^{\ol{F},\ol{G}}_H$ there is one functorial expression $F$
and one functorial expression $G$ for each functorial variable in
$\ol\phi$. Moreover, for each $\phi^k \in \ol\phi$ the number of
functorial variables $\beta$ in the judgments for its corresponding
functorial expresssions $F$ and $G$ is $k$. In the rules for $\tin_H$
and $\fold^F_H$, the functorial variables in $\ol{\beta}$ are fresh
with respect to $H$, and there is one $\beta$ for every
$\alpha$. (Recall from above that, in order for the types of $\tin_H$
and $\fold^F_H$ to be well-formed, the length of $\alpha$ must equal
the arity of $\phi$.) Substitution for terms is the obvious extension
of the usual capture-avoiding textual substitution, and
Definition~\ref{def:well-formed-terms} ensures that the expected
weakening rules for well-formed terms hold.

Using Definition~\ref{def:well-formed-terms} we can represent the
$\mathtt{reversePTree}$ function from Figure~\ref{fig:funs} in our
calculus as
\[\vdash \fold_{\beta + \phi(\beta \times \beta)}^{\mathit{PTree}\,
  \alpha} \, (\tin_{\beta + \phi(\beta \times \beta)} \circ s) :
\Nat^{\alpha} (\mathit{PTree}\,\alpha)\,(\mathit{PTree}\,\alpha)\] 
where
\[\begin{array}{lll}
\vdash \fold_{\beta + \phi(\beta \times \beta)}^{\mathit{PTree}\,\alpha} &
: & \Nat^{\emptyset} (\Nat^{\alpha} (\alpha + \mathit{PTree}\,(\alpha \times
\alpha)) \; (\mathit{PTree}\,\alpha))\; (\Nat^{\alpha}
(\mathit{PTree}\,\alpha) \; (\mathit{PTree}\,\alpha))\\
\vdash \tin_{\beta + \phi(\beta \times \beta)} & : & \Nat^{\alpha}
(\alpha + \mathit{PTree}\,(\alpha \times \alpha)) \; (\mathit{PTree}\,
\alpha)\\ 
{\color{red} \vdash \map_{\mathit{PTree}\,\alpha}^{\alpha \times \alpha, \alpha \times
  \alpha}} & : & \Nat^{\emptyset} (\Nat^{\alpha} (\alpha \times \alpha)\,
(\alpha \times \alpha))\; (\Nat^{\alpha} (\mathit{PTree}\,(\alpha \times
\alpha))\, (\mathit{PTree}\,(\alpha \times \alpha)))
\end{array}\]
and 
$\mathit{swap}$ and $s$ are the terms
\[ \vdash L_{\alpha} p.\, (\pi_2 p, \pi_1 p) :
\Nat^{\alpha} (\alpha \times \alpha)\, (\alpha \times \alpha)\]
and
\[\begin{array}{l}
\vdash L_{\alpha} t. \,\case{t}{b \mapsto {\color{blue} \inl}\, b}{t'
  \mapsto {\color{blue} \inr}\,( 
  {\color{red} \map_{\mathit{PTree}\,\alpha}^{\alpha \times \alpha, \alpha \times \alpha}\,
  \mathit{swap}\, t'} )} : \Nat^{\alpha} (\alpha + \mathit{PTree}\,(\alpha \times
\alpha))\; (\alpha + \mathit{PTree}\,(\alpha \times \alpha))
\end{array}\]
respectively. We can similarly represent the $\mathtt{reverseBush}$
function from Figure~\ref{fig:funs2} as
\[\vdash \fold_{\onet + \beta \times
\phi (\phi\beta)}^{\mathit{Bush}\,\alpha}\, ( \tin_{\onet + \beta \times
  \phi (\phi\beta)} \circ (\onet + t \circ i \circ i') ) :
\Nat^{\alpha} (\mathit{Bush}\,\alpha)\,(\mathit{Bush}\,\alpha)\] 
where
\[\begin{array}{lll}
\vdash \fold_{\onet + \beta \times \phi
  (\phi\beta)}^{\mathit{Bush}\,\alpha} : \Nat^{\emptyset}\, (\Nat^{\alpha}\, 
(\onet + \alpha \times \mathit{Bush}\, (\mathit{Bush}\, \alpha))) \;
(\mathit{Bush}\,\alpha))\; (\Nat^{\alpha} \,(\mathit{Bush}\,\alpha) \;
(\mathit{Bush}\,\alpha))\\ 
\vdash \tin_{\onet + \beta \times \phi (\phi\beta)} : \Nat^{\alpha}\,
(\onet + \alpha \times \mathit{Bush}\, (\mathit{Bush} \, \alpha))
\;(\mathit{Bush}\, \alpha)\\
\end{array}\]
and $\mathit{bnil}$, $\mathit{bcons}$, $\tin^{-1}_{\onet + \beta
  \times \phi (\phi\beta)}$, $t$, $i$, and 
$i'$ are the terms
\[\begin{array}{l}
\vdash \tin_{\onet + \beta \times \phi (\phi\beta)} \circ (
L_{\alpha}\, x.\, \inl\, x) \; : \; \Nat^{\alpha}\, \onet\;
(\mathit{Bush}\,\alpha)\\ 
\vdash \tin_{\onet + \beta \times \phi (\phi\beta)} \circ (
L_{\alpha}\, x.\, \inr\, x ) \; : \; \Nat^{\alpha}\, (\alpha \times
\mathit{Bush}\,(\mathit{Bush}\,\alpha))\; (\mathit{Bush}\,\alpha)\\
\vdash \fold_{\onet + \beta \times \phi (\phi\beta)}^{(\onet + \beta
  \times \phi (\phi\beta))[\phi := \mathit{Bush}\,\alpha]}\,
({\color{red} \map_{\onet + \beta \times \phi (\phi\beta)}^{(\onet + \beta \times
  \phi (\phi\beta))[\phi := \mathit{Bush}\,\alpha][\beta := \alpha],
  \mathit{Bush}\,\alpha}\, \tin_{\onet + \beta \times \phi
  (\phi\beta)}})\\
\hspace*{0.2in} : \Nat^{\alpha}\, (\mathit{Bush}\,\alpha)\; (\onet +
\alpha \times \mathit{Bush}\,(\mathit{Bush}\,\alpha))\\ 
\vdash L_{\alpha}\, (b, s). \,\mathsf{case}\;s\, \{ \hspace*{0.21in}
       {\color{blue} \inl}\,\ast \mapsto
\mathit{bcons}_{\alpha}\, b\, (\mathit{bnil}_{\alpha}\, \ast); \\
\hspace*{1in}{\color{blue} \inr}\,(s', u) \mapsto \mathsf{case}\;s'
\{\hspace*{0.35in}{\color{blue} \inl}\, \ast
\mapsto \mathit{bcons}_{\alpha}\, b \,(\mathit{bcons}_{\mathit{Bush}\,
  \alpha}\, (\mathit{bnil}_{\alpha}\, \ast)\, u); \\
\hspace*{2in} {\color{blue} \inr}\, (b', u') \mapsto \mathit{bcons}_{\alpha}\, b'
(\mathit{bcons}_{\mathit{Bush}\, \alpha}\, (\mathit{bcons}_{\alpha}\,
b\, u )\, u')\}\}\\
\hspace{0.2in} : \Nat^{\alpha} (\alpha \times (\onet + (\onet + \alpha
\times \mathit{Bush}\, (\mathit{Bush} \, \alpha))) \times
\mathit{Bush}\, (\mathit{Bush} \, (\mathit{Bush}\, \alpha))) \;
(\alpha \times \mathit{Bush}\, (\mathit{Bush} \, \alpha))\\
\vdash \alpha \times (\onet + \tin^{-1}_{\onet + \beta \times \phi
  (\phi\beta)} \times \mathit{Bush}\, (\mathit{Bush} \,
(\mathit{Bush}\, \alpha)))\\
\hspace{0.2in} : \Nat^{\alpha} (\alpha \times (\onet + \mathit{Bush}
\, \alpha \times \mathit{Bush}\,( \mathit{Bush} \, (\mathit{Bush}\,
\alpha))))\\
\hspace*{0.57in}(\alpha \times (\onet + (\onet + \alpha \times
\mathit{Bush}\, (\mathit{Bush} \, \alpha)) \times \mathit{Bush}\,
(\mathit{Bush} \, (\mathit{Bush}\, \alpha))))\\
\vdash \alpha \times (L_{\alpha}\,x.\, (\tin^{-1}_{\onet +
  \beta \times \phi (\phi\beta)})_{\mathit{Bush}\,\alpha}\, x)\\
\hspace{0.2in} : \Nat^{\alpha} (\alpha
\times \mathit{Bush} \, (\mathit{Bush}\, \alpha)) \; (\alpha \times (\onet
+ \mathit{Bush} \, (\alpha \times \mathit{Bush}\, (\mathit{Bush} \,
(\mathit{Bush}\, \alpha)))))
\end{array}\]
respectively. Here, $\Gamma; \emptyset \,|\, \Delta \vdash \sigma +
\eta : \Nat^{\ol{\alpha}} (\sigma + F)\; (\sigma + G)$ and $\Gamma;
\emptyset \,|\, \Delta \vdash \sigma \times \eta : \Nat^{\ol{\alpha}}
(\sigma \times F)\; (\sigma \times G)$ for $\sigma + \eta :=
L_{\ol{\alpha}}\, x.\, \case{x}{s \mapsto {\color{blue} \inl}\, s}{t
  \mapsto {\color{blue} \inr}\, 
  (\eta_{\ol{\alpha}} t)}$ and $\sigma \times \eta := L_{\ol{\alpha}}\,
x.\, (\pi_1 x, \eta_{\ol{\alpha}} (\pi_2 x))$ for $\Gamma; \emptyset
\,|\, \Delta \vdash \eta : \Nat^{\ol{\alpha}} F\; G$ and $\Gamma;
\ol{\alpha} \vdash \sigma$.

Because our system is designed to ensure functoriality, and
functoriality does not interact with polymorphism as well as might be
hoped, our system cannot express nontrivial recursive functions, such
as a concatenation function for perfect trees, that take not just one
nested type parameterized over a variable $\alpha$ as input, but other
inputs parameterized over $\alpha$ as well. The fundamental issue is
that recursion is expressible only via folds, and it is well-known
that a fold over a nested type must take a natural transformation as
an argument and return a natural transformation as its result.
%and, unfortunately, the polymorphism natural transformations entail
%leaves no nontrivial way to incorporate data from the second argument
%to a function with a type like that above into a fold over its first
%argument.
Even for ADTs there is a difference between which folds over them we
can write when ADTs are viewed as proper nested types --- i.e., as
fixpoints of types interpreted as first-order functors --- and which
ones we can write when ADTs are viewed as nested types --- i.e., as
fixpoints of types that are interpreted as higher-order functors. This
is because enforcing functoriality requires that the arity of the
recursion variable $\phi$ in the fixpoint in the return type of
$\fold$ must equal the number of variables bound by its $\Nat$
construct. For ADTs this number can be $0$, making it possible to
write as folds functions of types such as $\alpha; \emptyset \vdash
\Nat^\emptyset \ (\List\,\alpha)\,(\Nat^\emptyset \alpha\,(\List
    {\color{blue} \nat}))$, all of whose argument types are
    parameterized over the same type $\alpha$. This is not possible
    for nested types, even when those nested types are semantically
    equivalent to ADTs.
%Currying the type of the function does not help either, since the
%codomain of a fold applied to an algebra must be a functor, and this
%clearly is not the case for a function on nested types that takes
%more than one input involving the same type variable.
We can, however, represent types of functions that take as arguments
one proper nested type and as many other (possibly proper) nested
types as we wish, provided the types of those arguments are
parameterized over disjoint sets of type variables. The return type of
the function can be parameterized over any or all of the type
variables occurring in its argument types.
%A well-known deeper limitation on folds is that a fold over a nested
%type must take a natural transformation as an argument and return a
%natural transformation; this issue is discussed in more detail in
%Section~\ref{sec:conclusion},
%Unfortunately, the polymorphism natural transformations
%entail leaves no nontrivial way to incorporate data from the second
%argument to a function with a type like that above into a fold over
%its first argument.
Note, however, that even some recursive functions of a single proper
nested type --- e.g., a $\mathtt{reverseBush}$ function that is a true
involution --- cannot be expressed as folds because the algebra
arguments needed to define them are again recursive functions
%  cannot be defined. This happens, for
%  example, when the algebra argument to a fold must be a recursive
%  function
%, and thus must itself be defined as a fold
with type signatures of the same problematic forms.
%More sophisticated combinators for stylized recursion (e.g.,
%generalized folds)
%%  ~\cite{bp99})
%don't mitigate these difficulties, either. Indeed, they appear to be a
%fundamental consequence of the interaction of functoriality and
%polymorphism, not just some peculiarity of our particular calculus.
Adding combinators for more sophisticated stylized recursion may
ultimately enhance the expressivity of our calculus but, as discussed
in Section~\ref{sec:conclusion}, the expressivity issue for folds for
nested types long pre-dates, and is somewhat orthogonal to, the issue
of parametricity for languages that construct nested types as
fixpoints studied here.
%, and is orthogonal to, our focus on
%constructing parametric models for calculi in which nested types are
%constructed as fixpoints.

The presence of the ``extra'' functorial variables in $\ol{\gamma}$ in
the rules for $\map^{\ol{F},\ol{G}}_H$, $\tin_H$, and $\fold^F_H$
merit special mention. They allow us to map or fold polymorphic
functions over nested types. Consider, for example, the polymorphic
function $\mathit{flatten} : \Nat^\beta
(\mathit{PTree}\,\beta)\,(\mathit{List}\,\beta)$ that uniformly maps
perfect trees to lists. Even in the absence of extra variables the
instance of $\map$ required to map each non-functorial monomorphic
instantiation of $\mathit{flatten}$ over a list of perfect trees is
well-typed:
\[\begin{array}{l}
\AXC{$\Gamma;\alpha \vdash \mathit{List} \, \alpha$}
\AXC{$\Gamma;\emptyset \vdash \mathit{PTree}\,\sigma$}
\AXC{$\Gamma;\emptyset \vdash \mathit{List}\,\tau$}
\TIC{$\Gamma;\emptyset \,|\, x : \Nat^\emptyset\,(\mathit{PTree}\,
  \sigma)\,(\mathit{List}\, \tau)\; \vdash
  \map^{\mathit{PTree}\,\sigma, \,\mathit{List}\,\tau}_{\mathit{List}\,\alpha} \,x
 : \Nat^\emptyset\,(\mathit{List}\,
  (\mathit{PTree}\, \sigma))\,(\mathit{List}\, (\mathit{List}\,
  \tau))$}
\DisplayProof
\end{array}\]
But in the absence of $\ol \gamma$, the instance
\[\Gamma;\emptyset \,|\, x : \Nat^\beta(\mathit{PTree}\,
\beta)\,(\mathit{List}\, \beta) \vdash
\map^{\mathit{PTree}\,\beta,\mathit{List}\,\beta}_{\mathit{List}\,\alpha}\,x
: \Nat^\beta\,(\mathit{List}\,
(\mathit{PTree}\, \beta))\,(\mathit{List}\, (\mathit{List}\,
\beta))\] of $\mathtt{map}$ required to map the {\em polymorphic}
$\mathit{flatten}$ function over a list of perfect trees is not: in
that setting the functorial contexts for $F$ and $G$ in the rule for
$\map^{F,G}_H$ would have to be empty, but the fact that the
polymorphic $\mathit{flatten}$ function is functorial in some
variable, say $\delta$, means that it cannot possibly have a type of
the form $\Nat^\emptyset F\, G$ that would be required for it to be
the function input to $\map$. Since untypeability of this instance of
$\map$ is unsatisfactory in a polymorphic calculus, where we naturally
expect to be able to manipulate entire polymorphic functions rather
than just their monomorphic instances, we use the ``extra'' variables
in $\ol \gamma$ to remedy the situation. Specifically, the rules from
Definition~\ref{def:well-formed-terms} ensure that the instance of
$\map$ needed to map the polymorphic $\mathit{flatten}$ function is
typeable as follows:
\[\begin{array}{l}
\AXC{$\Gamma;\alpha,\beta \vdash \mathit{List} \, \alpha$}
\AXC{$\Gamma;\beta \vdash \mathit{PTree}\,\beta \hspace*{0.3in}
  \Gamma;\beta \vdash \mathit{List}\,\beta$}
\BIC{$\Gamma;\emptyset \,|\, x :   \Nat^\beta(\mathit{PTree}\,
  \beta)\,(\mathit{List}\, \beta) \vdash
  \map^{F,G}_{\mathit{List}\,\alpha} \, x :
 \Nat^\beta\,(\mathit{List}\,
  (\mathit{PTree}\, \beta))\,(\mathit{List}\, (\mathit{List}\,
  \beta))$}
\DisplayProof
\end{array}\]
Similar remarks explain the appearance of $\ol \gamma$ in the typing
rules for $\tin$ and $\fold$.

\vspace*{0.05in}

\section{Interpreting Types}\label{sec:type-interp}

We denote the category of sets and functions by $\set$. The category
$\rel$ has as its objects triples $(A,B,R)$ where $R$ is a relation
between the objects $A$ and $B$ in $\set$, i.e., a subset of $A \times
B$, and has as its morphisms from $(A,B,R)$ to $(A',B',R')$ pairs $(f
: A \to A',g : B \to B')$ of morphisms in $\set$ such that $(f a,g\,b)
\in R'$ whenever $(a,b) \in R$. We write $R : \rel(A,B)$ in place of
$(A,B,R)$ when convenient.  If $R : \rel(A,B)$ we write $\pi_1 R$ and
$\pi_2 R$ for the {\em domain} $A$ of $R$ and the {\em codomain} $B$
of $R$, respectively.  If $A : \set$, then we write $\Eq_A =
(A,A,\{(x,x)~|~ x \in A\})$ for the {\em equality relation} on $A$.

The key idea underlying Reynolds' parametricity is to give each type
$\tau(\alpha)$ with one free variable $\alpha$ both an {\em object
  interpretation} $\tau_0$ taking sets to sets and a \emph{relational
  interpretation} $\tau_1$ taking relations $R : \rel(A,B)$ to
relations $\tau_1 (R) : \rel(\tau_0 (A), \tau_0 (B))$, and to
interpret each term $t(\alpha,x) : \tau(\alpha)$ with one free term
variable $x : \sigma(\alpha)$ as a map $t_0$ associating to each set
$A$ a function $t_0(A) : \sigma_0(A) \to \tau_0(A)$. These
interpretations are to be given inductively on the structures of
$\tau$ and $t$ in such a way that they imply two fundamental
theorems. The first is an \emph{Identity Extension Lemma}, which
states that $\tau_1(\Eq_A) = \Eq_{\tau_0(A)}$, and is the essential
property that makes a model relationally parametric rather than just
induced by a logical relation. The second is an \emph{Abstraction
  Theorem}, which states that, for any $R :\rel(A, B)$,
$(t_0(A),t_0(B))$ is a morphism in $\rel$ from
$(\sigma_0(A),\sigma_0(B),\sigma_1(R))$ to
$(\tau_0(A),\tau_0(B),\tau_1(R))$. The Identity Extension Lemma is
similar to the Abstraction Theorem except that it holds for {\em all}
elements of a type's interpretation, not just those that are
interpretations of terms.
%i.e., $t_0(A)$ and $t_0(B)$ map related arguments to
%related results.
Similar results are expected to hold for types and terms with any
number of free variables.

The key to proving the Identity Extension Lemma
(Theorem~\ref{thm:iel}) in our setting is a familiar ``cutting down''
of the interpretations of universally quantified types, such as our
$\Nat$-types, to include only the ``parametric'' elements.
%(See, e.g.,~\cite{atk12,bfss90,gjfor15,rey83}).
This requires that set interpretations of types are defined
simultaneously with their relational interpretations. We give set
interpretations for our types in Section~\ref{sec:set-interp} and give
their relational interpretations in Section~\ref{sec:rel-interp}.
While the set interpretations are relatively straightforward, their
relation interpretations are less so, mainly because of the
cocontinuity conditions we must impose to ensure that they are
well-defined. We take some effort to develop conditions in
Section~\ref{sec:rel-interp}, which separates
Definitions~\ref{def:set-sem} and~\ref{def:rel-sem} in space, but
otherwise has no impact on the fact that they are given by mutual
induction.

\subsection{Interpreting Types as Sets}\label{sec:set-interp}

%A poset $\mcD = (D,\leq)$ is \emph{$\omega$-directed} if every finite
%subset of $D$ has an upper bound. When $\mcD$ is considered as a
%category, we write $d \in \mcD$ to indicate that $d$ is an object of
%$\mcD$ (i.e., $d \in D$). An {\em $\omega$-directed colimit} in a
%category $\mcC$ is a colimit of a diagram $F : {\mathcal D} \to \mcC$,
%where $\mathcal D$ is an $\omega$-directed poset. A category $\mcC$ is
%{\em $\omega$-cocomplete} if it has all $\omega$-directed colimits. A
%cocomplete category is one that has all colimits.
%
%If $\mcA$ and $\mcC$ are $\omega$-cocomplete, then the functor $F :
%\mcA \to \mcC$ is {\em $\omega$-cocontinuous} if it preserves
%$\omega$-directed colimits.  If $\mcA$ is locally small, then an
%object $A$ of $\mcA$ is {\em finitely presentable} if the functor
%$\mathsf{Hom}_\mcA(A, -) : \mcA \to \Set$ preserves $\omega$-directed
%colimits, i.e., if for every $\omega$-directed poset $\mathcal D$ and
%every functor $F : {\mathcal D} \to \mcC$, there is a canonical
%isomorphism $\colim{d \in \mcD}{\mathsf{Hom}_\mcA(A,Fd)} \simeq
%\mathsf{Hom}_{\mcA}(A, \colim{d \in \mcD}{Fd})$. A category $\mcA$ is
%       {\em finitely accessible} if it is $\omega$-cocomplete and has
%       a set $\mcA_0$ of finitely presentable objects such that every
%%       object is an $omega$-directed colimit of objects in $\mcA_0$;
%       it is {\em locally finitely presentable} if it is finitely
%       accessible and cocomplete.
We will interpret the types in our calculus as $\omega$-cocontinuous
functors on locally finitely presentable categories~\cite{ar94}. Since
functor categories of locally finitely presentable categories are
again locally finitely presentable, this will ensure, in particular,
that the fixpoints interpreting $\mu$-types in $\set$ and $\rel$
exist, and thus that both the set and relational interpretations of
all of the types in Definition~\ref{def:wftypes} are
well-defined~\cite{jp19}. To bootstrap this process, we interpret type
variables themselves as $\omega$-cocontinuous functors in
Definitions~\ref{def:set-env} and~\ref{def:reln-env}. If $\C$ and $\D$
are locally finitely presentable categories, we write $[\C,\D]$ for
the set of $\omega$-cocontinuous functors from $\C$ to $\D$.
%Note that the categories $\set$ and $\rel$ are both locally finitely
%presentable.

\begin{dfn}\label{def:set-env}
A {\em set environment} maps each type variable in $\tvars^k \cup
\fvars^k$ to an element of $[\set^k,\set]$.  A morphism $f : \rho \to
\rho'$ for set environments $\rho$ and $\rho'$ with $\rho|_\tvars =
\rho'|_\tvars$ maps each type constructor variable $\psi^k \in \tvars$
to the identity natural transformation on $\rho \psi^k = \rho'\psi^k$
and each functorial variable $\phi^k \in \fvars$ to a natural
transformation from the $k$-ary functor $\rho \phi^k$ on $\set$ to the
$k$-ary functor $\rho' \phi^k$ on $\set$.  Composition of morphisms on
set environments is given componentwise, with the identity morphism
mapping each set environment to itself. This gives a category of set
environments and morphisms between them, which we denote $\setenv$.
\end{dfn}
When convenient we identify a functor $F : [\set^0, \set]$ with the
set that is its codomain and consider a set environment to map
a type variable of arity $0$ to
%an $\omega$-cocontinuous functor from $\set^0$ to $\set$, i.e., to
a set.  If $\ol{\alpha} = \{\alpha_1,...,\alpha_k\}$ and $\ol{A} =
\{A_1,...,A_k\}$, then we write $\rho[\ol{\alpha := A}]$ for the set
environment $\rho'$ such that $\rho' \alpha_i = A_i$ for $i = 1,...,k$
and $\rho' \alpha = \rho \alpha$ if $\alpha \not \in
\{\alpha_1,...,\alpha_k\}$.  If $\rho$ is a set environment we write
$\Eq_\rho$ for the relation environment (see
Definition~\ref{def:reln-env}) such that $\Eq_\rho v = \Eq_{\rho v}$
for every type variable $v$.
%; see Definition~\ref{def:reln-env} for the complete definition of a
%relation environment.
The relational interpretations appearing in the second clause of
Definition~\ref{def:set-sem} are given in full in
Definition~\ref{def:rel-sem}.

\begin{dfn}\label{def:set-sem}
%Let $\rho$ be a set environment.
The {\em set interpretation} $\setsem{\cdot} : \F \to [\setenv, \set]$
is defined by
\begin{align*}
  \setsem{\Gamma;\Phi \vdash \zerot}\rho &= 0\\
  \setsem{\Gamma;\Phi \vdash \onet}\rho &= 1\\
%  \setsem{\Gamma;\emptyset \vdash v} \rho &= \rho v \mbox{
%    if } v \in \tvars^0 \\ 
  \setsem{\Gamma; \Phi
    %\emptyset
    \vdash \Nat^{\ol{\alpha}}
    \,F\,G}\rho &= \{\eta : \lambda \ol{A}. \,\setsem{\Gamma;
    \ol{\alpha} \vdash
    F}\rho[\ol{\alpha := A}] 
      \Rightarrow \lambda \ol{A}.\,\setsem{\Gamma; \Phi,
        \ol{\alpha} \vdash G}\rho[\ol{\alpha := A}] \\ 
      &\hspace{0.3in}|~\forall \overline{A}, \overline{B} :
      \set. \forall \overline{R : \rel(A, B)}.\\ 
      &\hspace{0.4in}(\eta_{\overline{A}}, \eta_{\overline{B}})
      : \relsem{\Gamma; \ol{\alpha} \vdash F}\Eq_{\rho}[\ol{\alpha := R}]
      \rightarrow \relsem{\Gamma;\Phi, \ol{\alpha} \vdash
        G}\Eq_{\rho}[\ol{\alpha := R}] \} \\
  \setsem{\Gamma;\Phi \vdash \phi\ol{\tau}}\rho &=
  (\rho\phi)\,\ol{\setsem{\Gamma;\Phi \vdash
    \tau}\rho}\\
  \setsem{\Gamma;\Phi \vdash \sigma+\tau}\rho &=
  \setsem{\Gamma;\Phi \vdash \sigma}\rho +
  \setsem{\Gamma;\Phi \vdash \tau}\rho\\
  \setsem{\Gamma;\Phi \vdash \sigma\times \tau}\rho &=
  \setsem{\Gamma;\Phi \vdash \sigma}\rho \times
  \setsem{\Gamma;\Phi \vdash \tau}\rho\\ 
  \setsem{\Gamma;\Phi \vdash (\mu \phi.\lambda
    \ol{\alpha}. H)\ol{\tau}}\rho &= (\mu
    T^\set_{H,\rho})\ol{\setsem{\Gamma;\Phi \vdash \tau}\rho}\\
    \text{where } T^\set_{H,\rho}\,F & = \lambda
  \ol{A}. \setsem{\Gamma;\Phi,\phi, \ol{\alpha} \vdash
    H}\rho[\phi :=  F][\ol{\alpha := A}]\\
  \text{and } T^\set_{H,\rho}\,\eta &= \lambda
  \ol{A}. \setsem{\Gamma;\Phi,\phi, \ol{\alpha} \vdash
    H}\id_\rho[\phi := \eta][\ol{\alpha := \id_{A}}]
\end{align*}
\end{dfn}
The interpretations in Definition~\ref{def:set-sem} respect weakening,
i.e., a type and its weakenings have the same set interpretations.
The same holds for the actions of these interpretations on morphisms
in Definition~\ref{def:set-sem-funcs} below.
%= \setsem{\Gamma \vdash \sigma} \rho \to
%\setsem{\Gamma \vdash \tau} \rho$, as expected {\color{blue} by IEL}.
If $\rho$ is a set environment and $\vdash \tau$ then we may
write $\setsem{\vdash \tau}$ instead of $\setsem{\vdash \tau}\rho$
since the environment is immaterial.
We note that the third clause of
Definition~\ref{def:set-sem} does indeed define a set: local finite
presentability of $\set$ and $\omega$-cocontinuity of
$\setsem{\Gamma;\ol{\alpha} \vdash F}\rho$ ensure that $\{\eta :
\setsem{\Gamma;\ol{\alpha} \vdash F}\rho \Rightarrow
\setsem{\Gamma;\Phi,\ol{\alpha} \vdash G}\rho\}$ (which contains
$\setsem{\Gamma;\Phi \vdash \Nat^{\ol{\alpha}}\,F\,G}\rho$) is a
subset of $\big\{({\setsem{\Gamma;\Phi,\ol{\alpha} \vdash G}\rho[\ol{\alpha
      := S}]})^{(\setsem{\Gamma;\ol{\alpha} \vdash F}\rho[\ol{\alpha
      := S}])}~\big|~ \ol{S} = (S_1,...,S_{|\ol{\alpha}|}), \mbox{ and
}$ $S_i \mbox{ is a finite set for } i =
1,...,|\ol{\alpha}|\big\}$. There are countably many choices for
tuples $\ol{S}$, and each of these gives rise to a morphism from
${\setsem{\Gamma;\ol{\alpha} \vdash F}\rho[ \ol{\alpha := S}]}$ to
${\setsem{\Gamma;\Phi,\ol{\alpha} \vdash G}\rho[\ol{\alpha := S}]}$. But
there are only $\set$-many choices of morphisms between any
%these (or any)
two objects since $\set$ is locally small. Finally, interpretations of
$\Nat$ types in Definition~\ref{def:set-sem} ensure $\setsem{\Gamma
  \vdash \sigma \to \tau}$ and $\setsem{\Gamma \vdash \forall
  \alpha. \tau}$ are as expected in any parametric model.

In order to make sense of the last clause in
Definition~\ref{def:set-sem}, we need to know that, for each $\rho \in
\setenv$, $T^\set_{H,\rho}$ is an $\omega$-cocontinuous endofunctor on
$[\set^k, \set]$, and thus admits a fixpoint.  Since
$T_{H,\rho}^\set$ is defined in terms of $\setsem{\Gamma;\Phi,\phi,
  \ol{\alpha} \vdash H}$, this means that interpretations of types
must be such functors, which in turn means that the actions of set
interpretations of types on objects and on morphisms in $\setenv$ are
intertwined. Fortunately, we know from~\cite{jp19} that, for every
$\Gamma; \ol{\alpha} \vdash \tau$, $\setsem{\Gamma; \ol{\alpha} \vdash
  \tau}$ is actually in $[\set^k,\set]$ where $k = |\ol \alpha|$. This
means that for each $\setsem{\Gamma; \Phi, \phi^k, \ol{\alpha} \vdash
  H}$, the corresponding operator $T^\set_{H}$ can be extended to a
{\em functor} from $\setenv$ to $[[\set^k,\set],[\set^k,\set]]$. The
action of $T^\set_H$ on an object $\rho \in \setenv$ is given by the
higher-order functor $T_{H,\rho}^\set$, whose actions on objects
(functors in $[\set^k, \set]$) and morphisms (natural transformations
between such functors) are given in Definition~\ref{def:set-sem}. Its
action on a morphism $f : \rho \to \rho'$ is the higher-order natural
transformation $T^\set_{H,f} : T^\set_{H,\rho} \to T^\set_{H,\rho'}$
whose action on $F : [\set^k,\set]$ is the natural transformation
$T^\set_{H,f}\, F : T^\set_{H,\rho}\,F \to T^\set_{H,\rho'}\,F$ whose
component at $\ol{A}$ is $(T^\set_{H,f}\, F)_{\ol{A}} =
\setsem{\Gamma; \Phi,\phi,\ol{\alpha} \vdash H}f[\phi :=
  \id_F][\ol{\alpha := \id_A}]$. The next definition uses the functor
$T^\set_H$ to define the actions of functors interpreting types on
morphisms between set environments.

\begin{dfn}\label{def:set-sem-funcs}
Let $f: \rho \to \rho'$ for set environments $\rho$ and $\rho'$ (so
that $\rho|_\tvars = \rho'|_\tvars$). The action $\setsem{\Gamma;\Phi
  \vdash \tau}f$ of\, $\setsem{\Gamma;\Phi \vdash \tau}$ on the
morphism $f$ is given as follows:
\begin{itemize}
\item If \,$\Gamma;\Phi \vdash \zerot$ then $\setsem{\Gamma;\Phi \vdash
  \zerot}f = \id_0$
\item If \,$\Gamma;\Phi \vdash \onet$ then $\setsem{\Gamma;\Phi \vdash
  \onet}f = \id_1$
%\item If \,$\Gamma,v;\emptyset \vdash v$ then
%  $\setsem{\Gamma,v;\emptyset \vdash v}f = \id_{\rho v}$
\item If \,$\Gamma; \Phi
  %\emptyset
  \vdash \Nat^{\ol{\alpha}}\,F\,G$ then
  $\setsem{\Gamma; \Phi
    %\emptyset
    \vdash \Nat^{\ol{\alpha}}\,F\,G} f =$\\
%  \id_{\setsem{\Gamma; \Phi
      %\emptyset
%      \vdash \Nat^{\ol{\alpha}}\,F\,G}\rho}$
\hspace*{0.2in} $\lambda t : \setsem{\Gamma;\Phi \vdash
    \Nat^{\ol{\alpha}}\,F\,G}\rho.\, 
  (\lambda \ol{A}. \setsem{\Gamma; \Phi,\ol\alpha \vdash G}f[\ol{\alpha
      := id_A}]) \circ t$  
\item If \,$\Gamma;\Phi \vdash \phi \ol{\tau}$ then
$\setsem{\Gamma;\Phi \vdash \phi \ol{\tau}} f : \setsem{\Gamma;\Phi
  \vdash \phi \ol{\tau}}\rho \to \setsem{\Gamma;\Phi \vdash
  \phi\ol{\tau}}\rho' = (\rho\phi) \ol{\setsem{\Gamma;\Phi \vdash
    \tau}\rho} \to (\rho'\phi) \ol{\setsem{\Gamma;\Phi \vdash
    \tau}\rho'}$ is defined by $\setsem{\Gamma;\Phi \vdash \phi
  \ol{\tau}} f = (f\phi)_{\ol{\setsem{\Gamma;\Phi \vdash
      \tau}\rho'}}\, \circ\, (\rho\phi) {\ol{\setsem{\Gamma;\Phi
      \vdash \tau}f}} = (\rho'\phi) {\ol{\setsem{\Gamma;\Phi \vdash
      \tau}f}}\, \circ\, (f \phi)_{\ol{\setsem{\Gamma;\Phi \vdash
      \tau}\rho}}$.  The latter equality holds because $\rho\phi$ and
  $\rho'\phi$ are functors and $f\phi : \rho\phi \to \rho'\phi$ is a
  natural transformation, so the following naturality square commutes:
{\footnotesize\begin{equation}\label{eq:cd2}
\begin{CD}
  (\rho\phi) \ol{\setsem{\Gamma;\Phi \vdash \tau}\rho} @> (f\phi)_{
    \ol{\setsem{\Gamma;\Phi \vdash \tau}\rho}} >> (\rho'\phi)
  \ol{\setsem{\Gamma;\Phi \vdash \tau}\rho} \\ @V(\rho\phi)
  \ol{\setsem{\Gamma;\Phi \vdash \tau}f}VV @V (\rho'\phi)
  \ol{\setsem{\Gamma;\Phi \vdash \tau}f} VV \\ (\rho\phi)
  \ol{\setsem{\Gamma;\Phi \vdash \tau}\rho'} @>(f\phi)_{
    \ol{\setsem{\Gamma;\Phi \vdash \tau}\rho'}}>> (\rho'\phi)
  \ol{\setsem{\Gamma;\Phi \vdash \tau}\rho'}
\end{CD}
\end{equation}}
\item If\, $\Gamma;\Phi \vdash \sigma + \tau$ then $\setsem{\Gamma;\Phi
  \vdash \sigma + \tau}f$ is defined by $\setsem{\Gamma;\Phi \vdash
  \sigma + \tau}f(\inl\,x) = \inl\,(\setsem{\Gamma;\Phi \vdash
  \sigma}f x)$ and $\setsem{\Gamma;\Phi \vdash \sigma +
  \tau}f(\inr\,y) = \inr\,(\setsem{\Gamma;\Phi \vdash \tau}f y)$
\item If \,$\Gamma;\Phi\vdash \sigma \times \tau$ then
  $\setsem{\Gamma;\Phi \vdash \sigma \times \tau}f = 
  \setsem{\Gamma;\Phi \vdash \sigma}f \times \setsem{\Gamma;\Phi \vdash
    \tau}f$
\item If \,$\Gamma;\Phi \vdash (\mu \phi.\lambda
  \ol{\alpha}. H)\ol{\tau}$ then
  \[\begin{array}{lll}
  \setsem{\Gamma;\Phi \vdash (\mu  \phi.\lambda
    \ol{\alpha}. H)\ol{\tau}} f &: &\setsem{\Gamma;\Phi 
    \vdash (\mu \phi.\lambda \ol{\alpha}. H)\ol{\tau}} \rho \to
  \setsem{\Gamma;\Phi \vdash (\mu
    \phi.\lambda\ol{\alpha}. H)\ol{\tau}} \rho'\\
  &= &(\mu
  T^\set_{H,\rho})\ol{\setsem{\Gamma;\Phi \vdash \tau}\rho} \to (\mu
  T^\set_{H,\rho'})\ol{\setsem{\Gamma;\Phi \vdash \tau}\rho'}
  \end{array}\]
  is
  defined by
  \[\begin{array}{ll} & (\mu T^\set_{H,f})\ol{\setsem{\Gamma;\Phi \vdash
      \tau}\rho'} \circ (\mu T^\set_{H,\rho})\ol{\setsem{\Gamma;\Phi
      \vdash \tau}f}\\ = & (\mu T^\set_{H,\rho'})\ol{\setsem{\Gamma;\Phi
      \vdash \tau}f} \circ (\mu T^\set_{H,f})\ol{\setsem{\Gamma;\Phi
      \vdash \tau}\rho}\end{array}\]  The latter equality holds because $\mu
  T^\set_{H,\rho}$ and $\mu T^\set_{H,\rho'}$ are functors and $\mu
  T_{H,f}^\set : \mu T_{H,\rho}^\set \to \mu T_{H,\rho'}^\set$ is a
  natural transformation, so the following naturality square commutes:
{\footnotesize\begin{equation}\label{eq:cd3}
\begin{CD}
 (\mu T^\set_{H,\rho}) \ol{\setsem{\Gamma;\Phi \vdash \tau}\rho} @> (\mu
  T^\set_{H,f})_{\ol{\setsem{\Gamma;\Phi \vdash \tau}\rho}} >> (\mu
  T^\set_{H,\rho'}) \ol{\setsem{\Gamma;\Phi \vdash \tau}\rho} \\ 
 @V(\mu T^\set_{H,\rho}) \ol{\setsem{\Gamma;\Phi \vdash \tau}f}VV @V  (\mu
 T^\set_{H,\rho'}) \ol{\setsem{\Gamma;\Phi \vdash \tau}f} VV \\ 
(\mu T^\set_{H,\rho}) \ol{\setsem{\Gamma;\Phi \vdash \tau}\rho'} @>(\mu
 T^\set_{H,f})_{\ol{\setsem{\Gamma;\Phi \vdash \tau}\rho'}}>> (\mu
 T^\set_{H,\rho'}) \ol{\setsem{\Gamma;\Phi \vdash \tau}\rho'}
\end{CD}
\end{equation}}
\end{itemize}
\end{dfn}

\subsection{Interpreting Types as Relations}\label{sec:rel-interp}

\begin{dfn}\label{def:rel-transf}
A {\em $k$-ary relation transformer} $F$ is a triple $(F^1, F^2,F^*)$,
where $F^1,F^2 : [\set^k,\set]$ are functors, $F^* : [\rel^k, \rel]$
is a functor, if $R_1:\rel(A_1,B_1),...,R_k:\rel(A_k,B_k)$, then $F^*
\ol{R} : \rel(F^1 \ol{A}, F^2 \ol{B})$, and if $(\alpha_1, \beta_1)
\in \Homrel(R_1,S_1),..., (\alpha_k, \beta_k) \in \Homrel(R_k,S_k)$
then $F^* \ol{(\alpha, \beta)} = (F^1 \ol{\alpha}, F^2 \ol{\beta})$.
We define $F\ol{R}$ to be $F^*\overline{R}$ and
$F\overline{(\alpha,\beta)}$ to be $F^*\overline{(\alpha,\beta)}$.
\end{dfn}
The last clause of Definition~\ref{def:rel-transf} expands to: if
$\ol{(a,b) \in R}$ implies $\ol{(\alpha\,a,\beta\,b) \in S}$ then
$(c,d) \in F^*\ol{R}$ implies $(F^1 \ol{\alpha}\,c,F^2 \ol{\beta}\,d)
\in F^*\ol{S}$. When convenient we identify a $0$-ary relation
transformer $(A,B,R)$ with $R : \rel(A,B)$. We may also write $\pi_1
F$ for $F^1$ and $\pi_2 F$ for $F^2$. We extend these conventions to
relation environments, introduced in Definition~\ref{def:reln-env}
below, in the obvious way.

\begin{dfn}
The category $RT_k$ of $k$-ary relation transformers is given by the
following data:
\begin{itemize}
\item An object of $RT_k$ is a relation transformer.
\item A morphism $\delta : (G^1,G^2,G^*) \to (H^1,H^2,H^*)$ in $RT_k$
  is a pair of natural transformations $(\delta^1, \delta^2)$ where
  $\delta^1 : G^1 \to H^1$, $\delta^2 : G^2 \to H^2$ such that, for
  all $\ol{R : \rel(A, B)}$, if $(x, y) \in G^*\ol{R}$ then
  $(\delta^1_{\ol{A}}x, \delta^2_{\ol{B}}y) \in H^*\ol{R}$.
\item Identity morphisms and composition are inherited from the
  category of functors on $\set$.
\end{itemize}
\end{dfn}

\begin{dfn}\label{def:RT-functor}
An endofunctor $H$ on $RT_k$ is a triple $H = (H^1,H^2,H^*)$, where
\begin{itemize}
\item $H^1$ and $H^2$ are functors from $[\set^k,\set]$ to $[\set^k,\set]$
\item $H^*$ is a functor from $RT_k$ to $[\rel^k,\rel]$
\item for all $\overline{R : \rel(A,B)}$,
  $\pi_1((H^*(\delta^1,\delta^2))_{\overline{R}}) = (H^1
  \delta^1)_{\overline{A}}$ and
  $\pi_2((H^*(\delta^1,\delta^2))_{\overline{R}}) = (H^2
  \delta^2)_{\overline{B}}$
\item The action of $H$ on objects is given by $H\,(F^1,F^2,F^*) =
  (H^1F^1,\,H^2F^2,\,H^*(F^1,F^2,F^*))$
\item The action of $H$ on morphisms is given by
  $H\,(\delta^1,\delta^2) = (H^1\delta^1,H^2\delta^2)$ for
  $(\delta^1,\delta^2) : (F^1,F^2,F^*)\to (G^1,G^2,G^*)$
\end{itemize}
\end{dfn}
Since the results of applying an endofunctor $H$ to $k$-ary relation
transformers and morphisms between them must again be $k$-ary relation
transformers and morphisms between them, respectively,
Definition~\ref{def:RT-functor} implicitly requires that the following
three conditions hold:\,{\em i})
if $R_1:\rel(A_1,B_1),...,R_k:\rel(A_k,B_k)$, then
  $H^*(F^1,F^2,F^*) \ol{R} : \rel(H^1F^1 \ol{A}, H^2F^2 
  \ol{B})$;
  %In other words, $\pi_1 (H^*(F^1,F^2,F^*) \ol{R}) = H^1F^1
  %\ol{A}$ and $\pi_2 (H^*(F^1,F^2,F^*) \ol{R}) = H^2F^2 \ol{B}$.
{\em ii}) if $(\alpha_1, \beta_1) \in \Homrel(R_1,S_1),..., (\alpha_k,
  \beta_k) \in \Homrel(R_k,S_k)$, then
  $H^*(F^1,F^2,F^*)\, \ol{(\alpha, \beta)} = (H^1F^1\ol{\alpha}, H^2F^2
  \ol{\beta})$; and
  %In other words, $\pi_1 (H^*(F^1,F^2,F^*) \ol{(\alpha,
  %  \beta)}) = H^1F^1 \ol{\alpha}$ and $\pi_2 (H^*(F^1,F^2,F^*) 
  % \ol{(\alpha, \beta)}) = H^2F^2 \ol{\beta}$.  {\em iii}) if
  $(\delta^1,\delta^2) : (F^1,F^2,F^*)\to (G^1,G^2,G^*)$ and
  $R_1:\rel(A_1,B_1),...,R_k:\rel(A_k,B_k)$, then
  $((H^1\delta^1)_{\ol{A}}x, (H^2\delta^2)_{\ol{B}}y) \in
  H^*(G^1,G^2,G^*)\ol{R}$ whenever $(x, y) \in
  H^*(F^1,F^2,F^*)\ol{R}$. Note, however, that this last condition is
  automatically satisfied because it is implied by the third bullet
  point of Definition~\ref{def:RT-functor}.

\begin{dfn}\label{def:RT-nat-trans}
If $H$ and $K$ are endofunctors on $RT_k$, then a {\em natural
  transformation} $\sigma : H \to K$ is a pair $\sigma = (\sigma^1,
\sigma^2)$, where $\sigma^1 : H^1 \to K^1$ and $\sigma^2 : H^2 \to
K^2$ are natural transformations between endofunctors on
$[\set^k,\set]$ and the component of $\sigma$ at $F
%= (F^1,F^2,F^*)
\in RT_k$ is given by $\sigma_F = (\sigma^1_{F^1}, \sigma^2_{F^2})$.
\end{dfn}
Definition~\ref{def:RT-nat-trans} entails that $\sigma^i_{F^i}$ must
be natural in $F^i : [\set^k,\set]$, and, for every $F$, both
$(\sigma^1_{F^1})_{\overline{A}}$ and
$(\sigma^2_{F^2})_{\overline{A}}$ must be natural in $\overline{A}$.
Moreover, since the results of applying $\sigma$ to $k$-ary relation
transformers must be morphisms of $k$-ary relation transformers,
Definition~\ref{def:RT-nat-trans} implicitly requires that
$(\sigma_F)_{\overline{R}} = ( (\sigma^1_{F^1})_{\overline{A}},
(\sigma^2_{F^2})_{\overline{B}})$ is a morphism in $\rel$ for any
$k$-tuple of relations $\overline{R : \rel(A, B)}$, i.e., that if $(x,
y) \in H^*F\overline{R}$, then $((\sigma^1_{F^1})_{\overline{A}} x,
(\sigma^2_{F^2})_{\overline{B}} y) \in K^*F\overline{R}$.

\vspace*{0.1in}

%{\color{blue} More context? Weave cocontinuity requirement for fixed
%  points through text better.}
Critically, we can compute $\omega$-directed colimits in $RT_k$: it is
not hard to see that if $\cal D$ is an $\omega$-directed set then
$\colim{d \in {\cal D}}{(F^1_d, F^2_d,F^*_d)} = (\colim{d \in {\cal
    D}}{F^1_d}, \colim{d \in {\cal D}}{F^2_d}, \colim{d \in {\cal
    D}}{F^*_d})$.  We define an endofunctor $T = (T^1,T^2,T^*)$ on
$RT_k$ to be {\em $\omega$-cocontinuous} if $T^1$ and $T^2$ are
$\omega$-cocontinuous endofunctors on $[\set^k,\set]$ and $T^*$ is an
$\omega$-cocontinuous functor from $RT_k$ to $[\rel^k,\rel]$, i.e., is
in $[RT_k,[\rel^k,\rel]]$.
%\begin{lemma}\label{lem:colimits}
%$\colim{d \in {\cal D}}{(F^1_d, F^2_d,F^*_d)} = (\colim{d \in {\cal
%      D}}{F^1_d}, \colim{d \in {\cal D}}{F^2_d}, \colim{d \in {\cal
%      D}}{F^*_d})$
%\end{lemma}
%\begin{proof}
%We first observe that $(\colim{d \in {\cal D}}{F^1_d}, \colim{d \in
%  {\cal D}}{F^2_d}, \colim{d \in {\cal D}}{F^*_d})$ is in $RT_k$.  If
%$R_1:\rel(A_1,B_1),...,R_k:\rel(A_k,B_k)$, then $\colim{d \in \cal
%  D}{F^*_d\ol{R}} : \rel(\colim{d \in {\cal D}}{F^1_d\ol{A}},
%\,\colim{d \in {\cal D}}{F^2_d\ol{B}})$ because of how colimits are
%computed in $\rel$. Moreover, if $(\alpha_1, \beta_1) \in
%\Homrel(R_1,S_1),..., (\alpha_k, \beta_k) \in \Homrel(R_k,S_k)$, then
%\[\begin{array}{ll}
%  & (\colim{d \in \cal D}{F_d^*})\ol{(\alpha,\beta)}\\
%= & \colim{d \in \cal D}{F_d^*\ol{(\alpha,\beta)}}\\
%= & \colim{d \in \cal D}{(F_d^1\ol{\alpha},\, F_d^2\ol{\beta})}\\
%= & (\colim{d \in \cal D}{F^1_d \ol{\alpha}}, \,\colim{d \in \cal D}{
%  F^2_d \ol{\beta}})
%\end{array}\]
%so $(\colim{d \in {\cal D}}{F^1_d}, \colim{d \in {\cal D}}{F^2_d},
%\colim{d \in {\cal D}}{F^*_d})$ actually is in $RT_k$.
%
%Now to see that $\colim{d \in {\cal D}}{(F^1_d, F^2_d,F^*_d)} =
%(\colim{d \in {\cal D}}{F^1_d}, \colim{d \in {\cal D}}{F^2_d},
%\colim{d \in {\cal D}}{F^*_d})$, let $\gamma^1_d : F^1_d \to \colim{d
%  \in {\cal D}}{F^1_d}$ and $\gamma^2_d : F^2_d \to \colim{d \in {\cal
%    D}}{F^2_d}$ be the injections for the colimits $\colim{d \in {\cal
%    D}}{F^1_d}$ and $\colim{d \in {\cal D}}{F^2_d}$,
%respectively. Then $(\gamma^1_d, \gamma^2_d) : (F^1_d, F^2_d,F^*_d)
%\to \colim{d \in {\cal D}}{(F^1_d, F^2_d,F^*_d)}$ is a morphism in
%$RT_k$ because, for all $\ol{R : \rel(A, B)}$,
%$((\gamma^1_d)_{\ol{A}}, (\gamma^2_d)_{\ol{B}}) : F^*_d \ol{R} \to
%\colim{d \in {\cal D}}{F^*_d \ol{R}}$ is a morphism in $\rel$. So
%$\{(\gamma^1_d, \gamma^2_d)\}_{d \in {\cal D}}$ are the mediating
%morphisms of a cocone in $RT_k$ with vertex $\colim{d \in {\cal
%    D}}{(F^1_d, F^2_d,F^*_d)}$. To see that this cocone is a
%colimiting cocone, let $C = (C^1,C^2,C^*)$ be the vertex of a cocone
%for $\{(F^1_d,F^2_d,F^*_d)\}_{d \in {\cal D}}$ with injections
%$(\delta^1_d,\delta^2_d) : (F^1_d,F^2_d,F^*_d) \to C$. If $\eta^1 :
%\colim{d \in {\cal D}}{F^1_d} \to C^1$ and $\eta^2 : \colim{d \in
%  {\cal D}}{F^2_d} \to C^2$ are the mediating morphisms in
%$[\set^k,\set]$, then $\eta^1$ and $\eta^2$ are unique such that
%$\delta^1_d = \eta^1 \circ \gamma^1_d$ and $\delta^2_d = \eta^2 \circ
%\gamma^2_d$.  We therefore have that $(\eta^1,\eta^2) : \colim{d \in
%  {\cal D}}{(F^1_d, F^2_d,F^*_d)} \to C$ is the mediating morphism in
%$RT_k$. Indeed, for all $\ol{R : \rel(A,B)}$ and $(x,y) \in \colim{d
%  \in {\cal D}}{F^*_d \ol{R}}$, there exist $d$ and $(x',y') \in
%F^*_d\ol{R}$ such that $(\gamma^1_d)_{\ol{A}}x' = x$ and
%$(\gamma^2_d)_{\ol{B}}y' = y$. But then $(\eta^1_{\ol{A}}x,
%\eta^2_{\ol{B}}y) = (\eta^1_{\ol{A}}((\gamma^1_d)_{\ol{A}}x'),
%\eta^2_{\ol{B}}((\gamma^2_d)_{\ol{B}}y')) = ((\delta^1_d)_{\ol{A}}x',
%(\delta^2_d)_{\ol{B}}y')$, and this pair is in $C^*\ol{R}$ because
%$(\delta^1_d, \delta^2_d)$ is a morphism from $(F^1_d,F^2_d,F^*_d)$ to
%$C$ in $RT_k$.
%\end{proof}
\begin{comment}
\begin{dfn}\label{def:omega-cocont}
An endofunctor $T = (T^1,T^2,T^*)$ on $RT_k$ is {\em
$\omega$-cocontinuous} if $T^1$ and $T^2$ are $\omega$-cocontinuous
endofunctors on $[\set^k,\set]$ and $T^*$ is an $\omega$-cocontinuous
functor from $RT_k$ to $[\rel^k,\rel]$, i.e., is in
$[RT_k,[\rel^k,\rel]]$.
\end{dfn}
\end{comment}
Now, for any $k$, any $A : \set$, and any $R : \rel(A, B)$, let
$K^\set_A$ be the constantly $A$-valued functor from $\set^k$ to
$\set$ and $K^\rel_R$ be the constantly $R$-valued functor from
$\rel^k$ to $\rel$.  Also let $0$ denote either the initial object of
$\set$ or the initial object of $\rel$, as appropriate.  Observing
that, for every $k$, $K^\set_0$ is initial in $[\set^k,\set]$, and
$K^\rel_0$ is initial in $[\rel^k,\rel]$, we have that, for each $k$,
$K_0 = (K^\set_0,K^\set_0,K^\rel_0)$ is initial in $RT_k$. Thus, if $T
= (T^1,T^2,T^*) : RT_k \to RT_k$ is an endofunctor on $RT_k$ then we
can define the relation transformer $\mu T$ to be $\colim{n \in
  \nat}{T^n K_0}$.
%Then Lemma~\ref{lem:colimits} shows $\mu T$ is indeed a relation
%transformer, and that it is given explicitly by
It is not hard to see that $\mu T$ is given explicitly as 
\begin{equation}\label{eq:mu}
  %\colim{n \in \nat}{T^n K_0}
  \mu T = (\mu T^1,\mu T^2,
\colim{n \in \nat}{(T^nK_0)^*})
\end{equation}
and that, as our notation suggests, it really is a fixpoint for $T$ if
$T$ is $\omega$-cocontinuous:
\begin{lemma}\label{lem:fp}
For any $T : [RT_k,RT_k]$, $\mu T \cong T(\mu T)$.
\end{lemma}
\noindent
The isomorphism is given by the morphisms $(in_1, in_2) : T(\mu T)
\to \mu T$ and $(in_1^{-1}, in_2^{-1}) : \mu T \to T(\mu T)$ in
$RT_k$. The latter is always a morphism in $RT_k$, but the
former need not be if $T$ is not $\omega$-cocontinuous.

\begin{comment}
\begin{proof}
We have $T(\mu T) = T(\colim{n \in \nat}{(T^nK_0)}) \cong \colim{n \in
\nat}{T(T^nK_0)} =
\mu T$.
\end{proof}
In fact, the isomorphism in Lemma~\ref{lem:fp} is given by the
morphisms $(in_1, in_2) : T(\mu T) \to \mu T$ and $(in_1^{-1},
in_2^{-1}) : \mu T \to T(\mu T)$ in $RT_k$. It is worth noting that the
latter is always a morphism in $RT_k$, but the former isn't necessarily
a morphism in $RT_k$ unless $T$ is $\omega$-cocontinuous.
\end{comment}

It is worth noting that the third component in Equation~(\ref{eq:mu})
is the colimit in $[\rel^k,\rel]$ of third components of relation
transformers, rather than a fixpoint of an endofunctor on
$[\rel^k,\rel]$. That there is an asymmetry between the first two
components of $\mu T$ and its third reflects the important conceptual
observation that the third component of an endofunctor on $RT_k$ need
not be a functor on all of $[\rel^k,\rel]$. In particular, although we
can define $T_{H,\rho}\, F$ for a relation transformer $F$ in
Definition~\ref{def:rel-sem} below, it is not clear how we could
define it for $F : [\rel^k,\rel]$.

\begin{dfn}\label{def:reln-env}
A {\em relation environment} maps each each type variable in $\tvars^k
\cup \fvars^k$ to a $k$-ary relation transformer.  A morphism $f :
\rho \to \rho'$ for relation environments $\rho$ and $\rho'$ with
$\rho|_\tvars = \rho'|_\tvars$ maps each type constructor variable
$\psi^k \in \tvars$ to the identity morphism on $\rho \psi^k = \rho'
\psi^k$ and each functorial variable $\phi^k \in \fvars$ to a morphism
from the $k$-ary relation transformer $\rho \phi$ to the $k$-ary
relation transformer $\rho' \phi$. Composition of morphisms on
relation environments is given componentwise, with the identity
morphism mapping each relation environment to itself. This gives a
category of relation environments and morphisms between them, which we
denote $\relenv$.
\end{dfn}
When convenient we identify a $0$-ary relation transformer with the
relation (transformer) that is its codomain and consider
% With this convention,
a relation environment to map a type variable of arity $0$
%to a $0$-ary relation transformer, i.e.,
to a relation.  We write $\rho[\ol{\alpha := R}]$ for the relation
environment $\rho'$ such that $\rho' \alpha_i \, = R_i$ for $i =
1,...,k$ and $\rho' \alpha = \rho\alpha$ if $\alpha \not \in
\{\alpha_1,...,\alpha_k\}$.  If $\rho$ is a relation environment, we
write $\pi_1 \rho$ and $\pi_2 \rho$ for the set environments mapping
each type variable $\phi$ to the functors $(\rho\phi)^1$ and
$(\rho\phi)^2$, respectively.

We define, for each $k$, the notion of an $\omega$-cocontinuous
functor from $\relenv$ to $RT_k$:
\begin{dfn}\label{def:relenv-functor}
A functor $H : [\relenv, RT_k]$ is a triple $H = (H^1,H^2,H^*)$,
where
\begin{itemize}
\item $H^1$ and $H^2$ are objects in $[\setenv,[\set^k,\set]]$
\item $H^*$ is a an object in $[\relenv,[\rel^k,\rel]]$
\item for all $\overline{R : \rel(A,B)}$ and morphisms $f$ in
  $\relenv$, $\pi_1(H^*f \,{\overline{R}}) = H^1 (\pi_1
  f)\,{\overline{A}}$ and $\pi_2(H^*f \,{\overline{R}}) = H^2 (\pi_2
  f)\,{\overline{B}}$
\item The action of $H$ on $\rho$ in $\relenv$ is given by $H \rho = (H^1
  (\pi_1 \rho),\,H^2 (\pi_2 \rho),\,H^*\rho)$
\item The action of $H$ on morphisms $f : \rho \to \rho'$ in $\relenv$
  is given by $Hf = (H^1 (\pi_1 f),H^2 (\pi_2 f))$
\end{itemize}
\end{dfn}
\noindent Spelling out the last two bullet points above gives the
following analogues of the three conditions immediately following
Definition~\ref{def:RT-functor}: {\em i}) if $R_1 :
\rel(A_1,B_1),...,R_k : \rel(A_k,B_k)$, then $H^*\rho\, \ol{R} :
\rel(H^1(\pi_1 \rho)\, \ol{A}, H^2(\pi_2 \rho)\, \ol{B})$;
%In other words, $\pi_1 (H^*\rho \ol{R}) = H^1(\pi_1 \rho)
%  \,\ol{A}$ and $\pi_2 (H^*\rho \ol{R}) = H^2 (\pi_2 \rho) \,\ol{B}$;
{\em ii}) if $(\alpha_1, \beta_1) \in \Homrel(R_1,S_1),..., (\alpha_k,
\beta_k) \in \Homrel(R_k,S_k)$, then $H^*\rho\, \ol{(\alpha, \beta)} =
(H^1(\pi_1 \rho)\,\ol{\alpha}, H^2(\pi_2 \rho)\, \ol{\beta})$;
%  In other words, $\pi_1 (H^*\rho\, \ol{(\alpha, \beta)}) = H^1(\pi_1
%  \rho)\,\ol{\alpha}$ and $\pi_2 (H^*\rho\, \ol{(\alpha, \beta)}) =
%  H^2(\pi_2 \rho)\,\ol{\beta}$;
and {\em iii}) if $f : \rho \to \rho'$ and
$R_1:\rel(A_1,B_1),...,R_k:\rel(A_k,B_k)$, then $(H^1(\pi_1
f)\,{\ol{A}}\,x, H^2(\pi_2 f)\,{\ol{B}}\,y) \in H^*\rho'\,\ol{R}$
whenever $(x, y) \in H^*\rho\,\ol{R}$. As before, the last
condition is automatically satisfied because it is implied by the
third bullet point of Definition~\ref{def:relenv-functor}.

Considering $\relenv$ as a product $\Pi_{\phi^k \in \tvars \cup
  \fvars} RT_k$, we extend the computation of $\omega$-directed
colimits in $RT_k$ to compute colimits in $\relenv$ componentwise. We
similarly extend the notion of an $\omega$-cocontinuous endofunctor on
$RT_k$ componentwise to give a notion of $\omega$-cocontinuity for
functors from $\relenv$ to $RT_k$.  Recalling from the start of this
subsection that Definition~\ref{def:rel-sem} is given mutually
inductively with Definition~\ref{def:set-sem} we can, at last, define:

\begin{dfn}\label{def:rel-sem}
The {\em relational interpretation} $\relsem{\cdot} : \F \to [\relenv,
 \rel]$ is defined by
\begin{align*}
  \relsem{\Gamma;\Phi \vdash \zerot}\rho &= 0\\
  \relsem{\Gamma;\Phi \vdash \onet}\rho &= 1\\
  \end{align*}
  \begin{align*}
  \relsem{\Gamma; \Phi \vdash \Nat^{\ol{\alpha}} \,F\,G}\rho &= \{\eta
  : \lambda \ol{R}.\,\relsem{\Gamma; \ol{\alpha} \vdash
    F}\rho[\ol{\alpha := R}] \Rightarrow \lambda \ol{R}. \,\relsem{
    \Gamma; \Phi,\ol{\alpha} \vdash G}\rho[\ol{\alpha := R}]\}\\
  &=
  \{(t,t') \in \setsem{\Gamma; \Phi
    \vdash \Nat^{\ol{\alpha}}
    \,F\,G} (\pi_1 \rho) \times \setsem{ 
    \Gamma;\Phi
    \vdash \Nat^{\ol{\alpha}} \,F\,G} (\pi_2
  \rho)~|~\\ 
  & \hspace{0.3in} \forall {R_1 : \rel(A_1,B_1)}\,...\,{R_k : \rel(A_k,B_k)}.\\
  & \hspace{0.4in} (t_{\ol{A}},t'_{\ol{B}}) \in
  (\relsem{\Gamma; \Phi,\ol{\alpha} \vdash G}\rho[\ol{\alpha :=
      R}])^{\relsem{\Gamma;\ol{\alpha}\vdash F}\rho[\ol{\alpha := R}]} \}\\  
  \relsem{\Gamma;\Phi \vdash \phi \ol{\tau}}\rho &=
  (\rho\phi)\ol{\relsem{\Gamma;\Phi \vdash 
    \tau}\rho}\\
  \relsem{\Gamma;\Phi \vdash \sigma+\tau}\rho &=
  \relsem{\Gamma;\Phi \vdash \sigma}\rho +
  \relsem{\Gamma;\Phi \vdash \tau}\rho\\
  \relsem{\Gamma;\Phi \vdash \sigma\times \tau}\rho &=
  \relsem{\Gamma;\Phi \vdash \sigma}\rho \times
  \relsem{\Gamma;\Phi \vdash \tau}\rho\\  
   \relsem{\Gamma;\Phi \vdash (\mu \phi.\lambda
    \ol{\alpha}. H)\ol{\tau}}\rho
  &= (\mu T_{H,\rho})\ol{\relsem{\Gamma;\Phi \vdash
     \tau}\rho}\\
  \text{where }	T_{H,\rho}
    &= (T^\set_{H,\pi_1\rho}, T^\set_{H,\pi_2\rho}, T^\rel_{H,\rho}) \\
  \text{and } T^\rel_{H,\rho}\,F
    &= \lambda \ol{R}. \relsem{
      \Gamma;\Phi,\phi,\ol{\alpha} \vdash H}\rho[\phi :=
    F][\ol{\alpha := R}]\\
  \text{and } T^\rel_{H,\rho}\,\delta
    &= \lambda \ol{R}. \relsem{
      \Gamma;\Phi,\phi,\ol{\alpha} \vdash H}\id_\rho[\phi :=
    \delta][\ol{\alpha := \id_{\ol{R}}}]
\end{align*}
\end{dfn}

The interpretations in Definition~\ref{def:rel-sem}, as well as in
Definition~\ref{def:rel-sem-funcs} below, respect weakening, and also
ensure that $\relsem{\Gamma \vdash \sigma \to \tau}$ and
$\relsem{\Gamma \vdash \forall \alpha. \tau}$ are as expected in any
parametric model.
%$\relsem{\Gamma \vdash
%  \sigma \to \tau}\rho = \relsem{\Gamma \vdash \sigma} \rho \to
%\relsem{\Gamma \vdash \tau} \rho$.
If $\rho$ is a relational environment and $\vdash \tau$, then we
write $\relsem{\vdash \tau }$ instead of $\relsem{\vdash \tau }\rho$
as for set interpretations.  For the last clause in
Definition~\ref{def:rel-sem} to be well-defined we need $T_{\rho}$ to
be an $\omega$-cocontinuous endofunctor on $RT$ so that, by
Lemma~\ref{lem:fp}, it admits a fixpoint. Since $T_\rho$ is defined
in terms of $\relsem{\Gamma;\Phi,\phi^k, \ol{\alpha} \vdash H}$, this
means that relational interpretations of types must be
$\omega$-cocontinuous functors from $\relenv$ to $RT_0$, which in turn
entails that the actions of relational interpretations of types on
objects and on morphisms in $\relenv$ are intertwined. As for set
interpretations, we know from~\cite{jp19} that, for every $\Gamma;
\ol{\alpha} \vdash \tau : \F$, $\setsem{\Gamma; \ol{\alpha} \vdash
  \tau}$ is actually in $[\rel^k,\rel]$ where $k = |\ol \alpha|$.
%In fact, we already know from~\cite{jp19} that, for every $\Gamma;
%\ol{\alpha} \vdash \tau : \F$, $\relsem{\Gamma; \ol{\alpha} \vdash
%  \tau}$ is actually functorial in $\ol{\alpha}$ and
%$\omega$-cocontinuous.
We first define the actions of each of these functors on morphisms
between environments in Definition~\ref{def:rel-sem-funcs}, and then
argue that the functors given by Definitions~\ref{def:rel-sem}
and~\ref{def:rel-sem-funcs} are well-defined and have the required
properties. To do this, we extend $T_H$ to a {\em functor} from
$\relenv$ to $[[\rel^k,\rel],[\rel^k,\rel]]$. Its action on an object
$\rho \in \relenv$ is given by the higher-order functor
$T^\rel_{H,\rho}$ whose actions on objects and morphisms are given in
Definition~\ref{def:rel-sem-funcs}. Its action on a morphism $f : \rho
\to \rho'$ is the higher-order natural transformation $T_{H,f} :
T_{H,\rho} \to T_{H,\rho'}$ whose action on any $F : [\rel^k,\rel]$ is
the natural transformation $T_{H,f}\, F : T_{H,\rho}\, F \to
T_{H,\rho'}\, F$ whose component at $\ol{R}$ is $(T_{H,f}\,
F)_{\ol{R}} = \relsem{\Gamma; \Phi,\phi,\ol{\alpha} \vdash H}f[\phi :=
  \id_F][\ol{\alpha := \id_R}]$.  The next definition uses the functor
$T_H$ to define the actions of functors interpreting types on
morphisms between relation environments.
    
\begin{dfn}\label{def:rel-sem-funcs}
Let $f: \rho \to \rho'$ for relation environments $\rho$ and $\rho'$
(so that $\rho|_\tvars = \rho'|_\tvars$). The action
$\relsem{\Gamma;\Phi \vdash \tau}f$ of $\relsem{\Gamma;\Phi \vdash
  \tau}$ on the morphism $f$ is given exactly as in
Definition~\ref{def:set-sem-funcs}, except that all interpretations
are relational interpretations and all occurrences of $T^\set_{H,f}$
are replaced by $T_{H,f}$. 
\begin{comment}
as follows:
\begin{itemize}
\item If \,$\Gamma;\Phi \vdash \zerot$ then $\relsem{\Gamma;\Phi \vdash
  \zerot}f = \id_0$
\item If\, $\Gamma;\Phi \vdash \onet$ then $\relsem{\Gamma;\Phi \vdash
  \onet}f = \id_1$
\item If \,$\Gamma; \Phi
  \vdash \Nat^{\ol{\alpha}}\,F\,G$ then
  $\relsem{\Gamma; \Phi
    \vdash \Nat^{\ol{\alpha}}\,F\,G} f =$\\
\hspace*{0.2in} $\lambda t : \relsem{\Gamma;\Phi \vdash
    \Nat^{\ol{\alpha}}\,F\,G}\rho.\, 
  (\lambda \ol{A}. \relsem{\Gamma; \Phi,\ol\alpha \vdash G}f[\ol{\alpha
      := id_A}]) \circ t$  
\item If \,$\Gamma;\Phi \vdash \phi \ol{\tau}$, then
  $\relsem{\Gamma;\Phi \vdash \phi \ol{\tau}} f : \relsem{\Gamma;\Phi
  \vdash \phi \ol{\tau}}\rho \to \relsem{\Gamma;\Phi \vdash \phi
  \ol{\tau}}\rho' = (\rho\phi) \ol{\relsem{\Gamma;\Phi \vdash
    \tau}\rho} \to (\rho'\phi) \ol{\relsem{\Gamma;\Phi \vdash
    \tau}\rho'}$ is defined by $\relsem{\Gamma;\Phi \vdash \phi
  \tau{A}} f = (f\phi)_{\ol{\relsem{\Gamma;\Phi \vdash \tau}\rho'}}
  \,\circ\, (\rho\phi) \ol{\relsem{\Gamma;\Phi \vdash \tau}f} =
  (\rho'\phi) \ol{\relsem{\Gamma;\Phi \vdash \tau}f} \,\circ\, (f
  \phi)_{\ol{\relsem{\Gamma;\Phi \vdash \tau}\rho}}$
\item If\, $\Gamma;\Phi\vdash \sigma + \tau$ then $\relsem{\Gamma;\Phi
  \vdash \sigma + \tau}f$ is defined by $\relsem{\Gamma;\Phi \vdash
  \sigma + \tau}f(\inl\,x) = \inl\,(\relsem{\Gamma;\Phi \vdash
  \sigma}f x)$ and $\relsem{\Gamma;\Phi \vdash \sigma +
  \tau}f(\inr\,y) = \inr\,(\relsem{\Gamma;\Phi \vdash \tau}f y)$
\item If\, $\Gamma;\Phi\vdash \sigma \times \tau$ then
  $\relsem{\Gamma;\Phi \vdash \sigma \times \tau}f =
  \relsem{\Gamma;\Phi \vdash \sigma}f \times \relsem{\Gamma;\Phi
    \vdash \tau}f$
\item If\, $\Gamma;\Phi \vdash (\mu \phi^k.\lambda
  \ol{\alpha}. H)\ol{\tau}$ then $\relsem{\Gamma;\Phi \vdash (\mu
    \phi.\lambda \ol{\alpha}. H)\ol{\tau}} f = (\mu
  T_{H,f})\ol{\relsem{\Gamma;\Phi \vdash \tau}\rho'} \circ (\mu
  T_{H,\rho})\ol{\relsem{\Gamma;\Phi \vdash \tau}f} = (\mu
  T_{H,\rho'})\ol{\relsem{\Gamma;\Phi \vdash \tau}f} \circ (\mu
  T_{H,f})\ol{\relsem{\Gamma;\Phi \vdash \tau}\rho}$
\end{itemize}
\end{comment}
\end{dfn}

To see that the functors given by Definitions~\ref{def:rel-sem}
and~\ref{def:rel-sem-funcs} are well-defined we must show that, for
every $H$, $T_{H,\rho}\,F$ is a relation transformer for any relation
transformer $F$, and that $T_{H,f}\, F : T_{H,\rho}\, F \to
T_{H,\rho'}\, F$ is a morphism of relation transformers for every
relation transformer $F$ and every morphism $f : \rho \to \rho'$ in
$\relenv$. This is an immediate consequence of 
\begin{lemma}\label{lem:rel-transf-morph}
%The interpretations in Definitions~\ref{def:rel-sem}
%and~\ref{def:rel-sem-funcs} are well-defined and,
For every $\Gamma;\Phi \vdash \tau$, $\sem{\Gamma;\Phi \vdash \tau} =
(\setsem{\Gamma;\Phi \vdash \tau}, \setsem{\Gamma;\Phi \vdash
  \tau},\relsem{\Gamma;\Phi \vdash \tau}) \in
%is an $\omega$-cocontinuous functor from $\relenv$ to $RT_0$, i.e., is an
%element of
[\relenv,RT_0]$.
\end{lemma}
\noindent
The proof is a straightforward induction on the structure of $\tau$,
%%each case of which first shows that $(\setsem{\Gamma;\Phi \vdash
%  \tau}, \setsem{\Gamma;\Phi \vdash \tau},\relsem{\Gamma;\Phi \vdash
%  \tau})$ is indeed a functor from $\relenv$ to $RT_0$, and then uses
using an appropriate result from~\cite{jp19} to deduce
$\omega$-cocontinuity of $\sem{\Gamma;\Phi \vdash \tau}$ in each case,
together with Lemma~\ref{lem:fp} and Equation~\ref{eq:mu} in the
$\mu$-case.

\begin{comment}
\begin{proof}
By induction on the structure of $\tau$. The only interesting cases
are when $\tau = \phi\ol{\tau}$ and when $\tau = (\mu \phi^k. \lambda
\overline{\alpha}. H)\overline{\tau}$. We consider each in turn.

\begin{itemize}
\item When $\tau = \Gamma; \Phi \vdash \phi\ol{\tau}$, we have    
\[\begin{array}{ll}
  & \pi_i (\relsem{\Gamma; \Phi \vdash \phi\ol{\tau}}\rho)\\
= & \pi_i ((\rho \phi)\overline{\relsem{\Gamma; \Phi \vdash \tau}\rho})\\
= & (\pi_i (\rho \phi)) (\pi_i(\overline{\relsem{\Gamma; \Phi \vdash \tau}\rho}))\\
= & ((\pi_i \rho) \phi) (\overline{\setsem{\Gamma; \Phi \vdash
    \tau}(\pi_i \rho)})\\
= & \setsem{\Gamma; \Phi \vdash \phi\ol{\tau}}(\pi_i \rho)
\end{array}\]
and, for $f : \rho \to \rho'$ in $\relenv$,
\[\begin{array}{ll}
  & \pi_i (\relsem{\Gamma; \Phi \vdash \phi\ol{\tau}}f)\\
= & \pi_i ((f\phi)_{\overline{\relsem{\Gamma; \Phi \vdash \tau}\rho'}})
\circ \pi_i((\rho\phi)({\overline{\relsem{\Gamma; \Phi \vdash \tau}f}}))\\
= & (\pi_i (f\phi))_{\overline{\pi_i (\relsem{\Gamma; \Phi \vdash \tau}\rho')}}
\circ (\pi_i(\rho\phi))({\overline{\pi_i (\relsem{\Gamma; \Phi \vdash \tau}f}}))\\
= & ((\pi_i f)\phi)_{\overline{\setsem{\Gamma; \Phi \vdash \tau}(\pi_i\rho')}}
\circ ((\pi_i\rho)\phi)({\overline{\setsem{\Gamma; \Phi \vdash \tau}(\pi_i f)}})\\
= & \setsem{\Gamma; \Phi \vdash \phi\ol{\tau}}(\pi_i f)
\end{array}\]
The third equalities of each of the above derivations are by the
induction hypothesis. That $\sem{\Gamma; \Phi \vdash
  \phi\ol{\tau}}$ is $\omega$-cocontinuous is an immediate
consequence of the facts that $\set$ and $\rel$ are locally finitely
presentable, together with Corollary~12 of~\cite{jp19}.
\item When $\tau = (\mu \phi. \lambda
  \overline{\alpha}. H)\overline{\tau}$ we first show that $\sem{ (\mu
    \phi. \lambda \overline{\alpha}. H)\overline{\tau}}$ is
  well-defined.
\begin{itemize}
\item \underline{$T_\rho : [RT_k,RT_k]$}:\/ We must show that, for any
  relation transformer $F = 
  (F^1, F^2, F^*)$, the triple $T_{\rho} F = (T^\set_{\pi_1 \rho}F^1,
  T^\set_{\pi_2 \rho}F^2, T^\rel_{\rho}F)$ is also a relation
  transformer.  Let $\overline{R : \rel(A, B)}$. Then for
  $i = 1, 2$, we have
\[\begin{split}
\pi_i(T^\rel_{\rho}\,F\,\overline{R})
&= \pi_i(\relsem{\Gamma;\Phi,\phi,\overline{\alpha} \vdash H}\rho[\phi := F]\overline{[\alpha := R]}) \\
&= \setsem{\Gamma;\Phi,\phi,\overline{\alpha} \vdash H} (\pi_i (\rho[\phi := F]\overline{[\alpha := R]})) \\
&= \setsem{\Gamma;\Phi,\phi,\overline{\alpha} \vdash H} (\pi_i \rho)[\phi := \pi_i F]\overline{[\alpha := \pi_i R]}) \\
&= T^\set_{\pi_i \rho} (\pi_i F) (\overline{\pi_i R})
\end{split}\]
and 
\[\begin{split}
\pi_i(T^\rel_{\rho}\,F\,\overline{\gamma})
&= \pi_i(\relsem{\Gamma;\Phi,\phi,\overline{\alpha} \vdash H}\id_\rho[\phi
  := \id_F]\overline{[\alpha := \gamma]}) \\
&= \setsem{\Gamma;\Phi,\phi,\overline{\alpha} \vdash H} (\pi_i (\id_\rho[\phi := \id_F]\overline{[\alpha := \gamma]})) \\
&= \setsem{\Gamma;\Phi,\phi,\overline{\alpha} \vdash H} \id_{\pi_i \rho}[\phi := \id_{\pi_i F}]\overline{[\alpha := \pi_i \gamma]} \\
&= T^\set_{\pi_i \rho} (\pi_i F) (\overline{\pi_i \gamma})
\end{split}\]
Here, the second equality in each of the above chains of equalities is
by the induction hypothesis.

We also have that, for every morphism $\delta = (\delta^1, \delta^2) :
F \to G$ in $RT_k$ and all $\overline{R : \rel(A, B)}$,
\[\begin{array}{ll}
  & \pi_i((T^\rel_\rho \delta)_{\overline{R}})\\
= & \pi_i(\relsem{\Gamma;\Phi,\phi,\overline{\alpha} \vdash H}\id_\rho[\phi :=
  \delta]\overline{[\alpha := \id_R]})\\
= & \setsem{\Gamma;\Phi,\phi,\overline{\alpha} \vdash H}\id_{\pi_i\rho}[\phi :=
  \pi_i \delta]\overline{[\alpha := \id_{\pi_i R}]}\\
= & (T^\set_{\pi_i \rho} (\pi_i \delta))_{\overline{\pi_i R}}
\end{array}\]

Here, the second equality is by the induction hypothesis.  That
$T_\rho$ is $\omega$-cocontinuous follows immediately from the
induction hypothesis on $\sem{\Gamma;\Phi,\phi,\ol{\alpha} \vdash H}$
and the fact that colimts are computed componentwise in $RT$.
\item \underline{$\sigma_f = (\sigma^\set_{\pi_1 f},
  \sigma^\set_{\pi_2 f})$ is a natural transformation from $T_{\rho}$
  to $T_{\rho'}$}:\/ We must show that $(\sigma_f)_F =
  ((\sigma^\set_{\pi_1 f})_{F^1}, (\sigma^\set_{\pi_2 f})_{F^2})$ is a
  morphism in $RT_k$ for all relation transformers $F = (F^1, F^2,
  F^*)$, i.e., that $((\sigma_f)_F)_{\overline{R}}
  =(((\sigma^\set_{\pi_1 f})_{F^1})_{\overline{A}},
  ((\sigma^\set_{\pi_2 f})_{F_2})_{\overline{B}})$ is a morphism in
  $\rel$ for all relations $\overline{R : \rel(A, B)}$. Indeed, we
  have that
\[
((\sigma_f)_F)_{\overline{R}} =
\relsem{\Gamma;\Phi,\phi,\overline{\alpha} \vdash H}f[\phi :=
  \id_F]\overline{[\alpha := \id_R]}
\]
is a morphism in $RT_0$ (and thus in $\rel$) by the induction hypothesis.
\end{itemize}

\vspace*{0.05in}

The relation transformer $\mu T_\rho$ is therefore a fixpoint of
$T_\rho$ by Lemma~\ref{lem:fp}, and $\mu \sigma_f$ is a morphism in
$RT_k$ from $\mu T_\rho$ to $\mu T_{\rho'}$. ($\mu$ is shown to be a
functor in~\cite{jp19}.)  So $\relsem{\Gamma; \Phi \vdash (\mu
  \phi. \lambda \overline{\alpha}. H)\overline{\tau}}$, and thus
$\sem{\Gamma; \Phi \vdash (\mu \phi. \lambda
  \overline{\alpha}. H)\overline{\tau}}$, is well-defined.

\vspace*{0.1in}

To see that $\sem{\Gamma; \Phi \vdash (\mu \phi. \lambda
  \overline{\alpha}. H)\overline{\tau}} : [\relenv,RT_0]$, we must
verify three conditions:
\begin{itemize}
\item Condition (1) after Definition~\ref{def:relenv-functor} is
  satisfied since
\[\begin{split}
\pi_i(\relsem{\Gamma;\Phi \vdash (\mu \phi. \lambda
  \overline{\alpha}. H) \overline{\tau}}\rho) 
&= \pi_i( (\mu T_\rho) (\overline{\relsem{\Gamma;\Phi \vdash
    \tau}\rho})) \\ 
&= \pi_i(\mu T_{\rho}) (\overline{\pi_i(\relsem{\Gamma;\Phi \vdash
    \tau}\rho})) \\ 
&= \mu T^\set_{\pi_i\rho} (\overline{\setsem{\Gamma;\Phi \vdash
    \tau}(\pi_i\rho)}) \\ 
&= \setsem{\Gamma;\Phi \vdash (\mu \phi. \lambda \overline\alpha. H)
  \overline{\tau}}(\pi_i\rho) 
\end{split}\]
The third equality is by Equation ~\ref{eq:mu} and the induction
hypothesis.
\item Condition (2) after Definition~\ref{def:relenv-functor} is
  satisfied since it is subsumed by the previous condition because $k
  = 0$.
\item The third bullet point of Definition~\ref{def:relenv-functor} is
  satisfied because
\[\begin{split}
& \pi_i(\relsem{\Gamma;\Phi \vdash (\mu \phi. \lambda
  \overline\alpha. H)\overline{\tau}}f)\\
&= \pi_i((\mu T_{\rho'})(\overline{\relsem{\Gamma;\Phi \vdash
    \tau}f}) \circ (\mu \sigma_f)_{\overline{\relsem{\Gamma;\Phi
      \vdash \tau}\rho}}) \\ 
&= \pi_i((\mu T_{\rho'})(\overline{\relsem{\Gamma;\Phi \vdash
    \tau}f})) \circ \pi_i((\mu
\sigma_f)_{\overline{\relsem{\Gamma;\Phi \vdash \tau}\rho}}) \\  
&= \pi_i(\mu T_{\rho'})(\overline{\pi_i(\relsem{\Gamma;\Phi
    \vdash \tau}f})) \circ \pi_i(\mu
\sigma_f)_{\overline{\pi_i(\relsem{\Gamma;\Phi \vdash 
      \tau}\rho})} \\ 
&= (\mu T^\set_{\pi_i \rho'})(\overline{\setsem{\Gamma;\Phi \vdash
    \tau}(\pi_i f)}) \circ (\mu \sigma^\set_{\pi_i
  f})_{\overline{\setsem{\Gamma;\Phi \vdash \tau}(\pi_i\rho)}} \\ 
&= \setsem{\Gamma;\Phi \vdash (\mu \phi. \lambda
  \overline\alpha. H)\overline{\tau}}(\pi_i f). 
\end{split}\]
The fourth equality is by~\ref{eq:mu} and the induction hypothesis.
\end{itemize}
As before, that $\sem{\Gamma; \Phi \vdash (\mu \phi. \lambda
  \overline{\alpha}. H)\overline{\tau}}$ is $\omega$-concontinuous
follows from the facts that $\set$ and $\rel$ are locally finitely
presentable, and that colimits in $\relenv$ are computed
componentwise, together with Corollary~12 of~\cite{jp19}.
\end{itemize}
\end{proof}
\end{comment}

%We need
%$\relsem{\Gamma; \Phi, \phi \vdash \tau}\rho$ $=
%\relsem{\Gamma, \psi; \Phi \vdash \tau[\phi :==
%    \psi]}\rho[\psi:=\rho\phi]$, and that $\relsem{\Gamma \vdash
%  \sigma \to \tau}$ and
%$\setsem{\Gamma; \Phi, \phi \vdash \tau}\rho$ $=
%\setsem{\Gamma, \psi; \Phi \vdash \tau[\phi :==
%    \psi]}\rho[\psi:=\rho\phi]$. 
We can also prove by simultaneous induction that our interpretations
of types interact well with demotion of functorial variables. Indeed,
we have that, for any $\rho,\rho'$ and $f: \rho \to \rho'$ in
$\setenv$, 
\begin{gather}
\label{thm:demotion-objects}
\setsem{\Gamma; \Phi, \phi \vdash F} \rho = \setsem{\Gamma, \psi; \Phi
  \vdash F[\phi :== \psi] } \rho[\psi := \rho \phi]\\
\label{thm:demotion-morphisms}
\setsem{\Gamma; \Phi, \phi \vdash F} f = \setsem{\Gamma, \psi; \Phi
  \vdash F[\phi :== \psi]} f[\psi \mapsto f\phi]
\end{gather}
Moreover,
%\begin{lemma}\label{lem:demotion-objects}
if $\rho, \rho' : \setenv$ are such that $\rho \phi = \rho \psi =
\rho' \phi = \rho' \psi$, \,$f : \rho \to \rho'$ in $\setenv$ is such
that $f \phi = f \psi = \id_{\rho \phi}$,\, $\Gamma; \Phi, \phi^k
\vdash F$,\, $\Gamma;\Phi,\ol{\alpha} \vdash G$
$\Gamma;\Phi,\alpha_1...\alpha_k \vdash H$, and $\Gamma;\Phi \vdash
\tau$, then
\begin{gather}
%\label{thm:demotion-objects}
%\setsem{\Gamma; \Phi, \phi \vdash F} \rho = \setsem{\Gamma, \psi; \Phi
%  \vdash F[\phi :== \psi] } \rho{\color{blue}[\psi := \rho \phi]}\\
%\label{thm:demotion-morphisms}
%\setsem{\Gamma; \Phi, \phi \vdash F} f = \setsem{\Gamma, \psi; \Phi
%  \vdash F[\phi :== \psi]} f{\color{blue}[\psi \mapsto f\phi]}\\
\label{eq:subs-var}
\setsem{\Gamma;\Phi \vdash G[\ol{\alpha := \tau}]}\rho =
\setsem{\Gamma;\Phi,\ol{\alpha} \vdash G}\rho[\ol{\alpha :=
\setsem{\Gamma;\Phi \vdash \tau}\rho}]\\
\label{eq:subs-var-morph}
\setsem{\Gamma;\Phi \vdash G[\ol{\alpha := \tau}]}f =
\setsem{\Gamma;\Phi,\ol{\alpha} \vdash G}f[\ol{\alpha :=
\setsem{\Gamma;\Phi \vdash \tau}f}]\\
\label{eq:subs-const}
\setsem{\Gamma; \Phi \vdash F[\phi := H]}\rho
= \setsem{\Gamma; \Phi, \phi \vdash F}\rho
[\phi := \lambda \ol{A}.\, \setsem{\Gamma;\Phi,\overline{\alpha}\vdash
    H}\rho[\overline{\alpha := A}]] \\ 
\label{eq:subs-const-morph}
\setsem{\Gamma; \Phi \vdash F[\phi := H]}f
= \setsem{\Gamma; \Phi, \phi \vdash F}f
[\phi := \lambda \ol{A}.\,\setsem{\Gamma;\Phi,\overline{\alpha}\vdash
    H}f[\overline{\alpha := \id_{\ol{A}}}]] 
\end{gather}
Identities analogous to (\ref{thm:demotion-objects}) through
(\ref{eq:subs-const-morph}) hold for relational interpretations as well.
%\end{lemma}


\begin{comment}
\begin{lemma}\label{lem:substitution}
Let $\rho$ and $\rho'$ be set environments and let $f : \rho \to \rho'$ be a
morphism of set environments. 
\begin{itemize}
\item If $\Gamma;\Phi,\ol{\alpha} \vdash F$ and $\Gamma;\Phi \vdash \tau$,
  then  
\begin{gather}\label{eq:subs-var}
\setsem{\Gamma;\Phi \vdash F[\ol{\alpha := \tau}]}\rho =
\setsem{\Gamma;\Phi,\ol{\alpha} \vdash F}\rho[\ol{\alpha :=
\setsem{\Gamma;\Phi \vdash \tau}\rho}]\\
\intertext{and}\label{eq:subs-var-morph}
\setsem{\Gamma;\Phi \vdash F[\ol{\alpha := \tau}]}f =
\setsem{\Gamma;\Phi,\ol{\alpha} \vdash F}f[\ol{\alpha :=
\setsem{\Gamma;\Phi \vdash \tau}f}]
\end{gather}
\item If $\Gamma;\Phi,\phi^k\vdash F$ and
  $\Gamma;\Phi,\alpha_1...\alpha_k \vdash H$, then 
\begin{gather}\label{eq:subs-const}
\setsem{\Gamma; \Phi \vdash F[\phi := H]}\rho
= \setsem{\Gamma; \Phi, \phi \vdash F}\rho
[\phi := \lambda \ol{A}.\, \setsem{\Gamma;\Phi,\overline{\alpha}\vdash
    H}\rho[\overline{\alpha := A}]] \\ 
\intertext{and}\label{eq:subs-const-morph}
\setsem{\Gamma; \Phi \vdash F[\phi := H]}f
= \setsem{\Gamma; \Phi, \phi \vdash F}f
[\phi := \lambda \ol{A}.\,\setsem{\Gamma;\Phi,\overline{\alpha}\vdash
    H}f[\overline{\alpha := \id_{\ol{A}}}]] 
\end{gather}
\end{itemize}
Analogous identities hold for relation environments and morphisms
between them.
\end{lemma}

\end{comment}






\begin{comment}

\begin{thm}\label{thm:demotion-objects}
Let \, $\Gamma; \Phi, \phi \vdash \tau : \F$. If $\rho, \rho' : \setenv$
are such that $\rho \phi = \rho \psi = \rho' \phi = \rho' \psi$, and
if $f : \rho \to \rho'$ is a morphism of set environments such that $f
\phi = f \psi = \id_{\rho \phi}$, then
\[\setsem{\Gamma; \Phi, \phi \vdash \tau} \rho = \setsem{\Gamma, \psi;
  \Phi \vdash \tau[\phi :== \psi] } \rho\]
and
\[\setsem{\Gamma; \Phi, \phi \vdash \tau} f = \setsem{\Gamma, \psi;
  \Phi \vdash \tau[\phi :== \psi]} f\]

\vspace*{0.1in}

\noindent
Analogously, if $\rho, \rho' : \relenv$ are such that $\rho \phi =
\rho \psi = \rho' \phi = \rho' \psi$, and if $f : \rho \to \rho'$ is a
morphism of relation environments such that $f \phi = f \psi =
\id_{\rho \phi}$, then
\[\relsem{\Gamma; \Phi, \phi \vdash \tau} \rho = \relsem{\Gamma, \psi;
  \Phi \vdash \tau[\phi :== \psi] } \rho\]
and
\[\relsem{\Gamma; \Phi, \phi \vdash \tau} f = \relsem{\Gamma, \psi;
  \Phi \vdash \tau[\phi :== \psi]} f\]
\end{thm}

\end{comment}

  
\begin{comment}
---
The next two theorems are proven by simultaneous induction. We are
actually only interested in using Theorem~\ref{thm:demotion-objects},
but in order to prove the $\mu$ case for this theorem, we need
Theorem~\ref{thm:demotion-morph} to show that two functors have equal
actions on morphisms.

\begin{thm}\label{thm:demotion-objects}
Let $\Gamma; \Phi, \phi \vdash \tau : \F$. If $\rho, \rho' : \setenv$
are such that $\rho \phi = \rho \psi = \rho' \phi = \rho' \psi$, and
if $f : \rho \to \rho'$ is a morphism of set environments such that $f
\phi = f \psi = \id_{\rho \phi}$, then
\[\setsem{\Gamma; \Phi, \phi \vdash \tau} \rho = \setsem{\Gamma, \psi;
  \Phi \vdash \tau[\phi :== \psi] } \rho\]
and
\[\setsem{\Gamma; \Phi, \phi \vdash \tau} f = \setsem{\Gamma, \psi;
  \Phi \vdash \tau[\phi :== \psi]} f\]

\vspace*{0.1in}

\noindent
Analogously, if $\rho, \rho' : \relenv$ are such that $\rho \phi =
\rho \psi = \rho' \phi = \rho' \psi$, and if $f : \rho \to \rho'$ is a
morphism of relation environments such that $f \phi = f \psi =
\id_{\rho \phi}$, then
\[\relsem{\Gamma; \Phi, \phi \vdash \tau} \rho = \relsem{\Gamma, \psi;
  \Phi \vdash \tau[\phi :== \psi] } \rho\]
and
\[\relsem{\Gamma; \Phi, \phi \vdash \tau} f = \relsem{\Gamma, \psi;
  \Phi \vdash \tau[\phi :== \psi]} f\]
\end{thm}
\begin{proof}
We prove the result for set interpretations by induction on the
structure of $\tau$.  The case for relational interpretations proceeds
analogously. Since the only interesting cases are the application case
and the $\mu$-case, we elide the others.
\begin{itemize}
\item If $\Gamma; \Phi, \phi \vdash \phi \ol\tau : \F$, then the
  induction hypothesis gives that  
\[\setsem{\Gamma; \Phi, \phi \vdash \tau } \rho = \setsem{\Gamma,
  \psi; \Phi \vdash \tau[\phi :== \psi] } \rho\] 
and
\[\setsem{\Gamma; \Phi, \phi \vdash \tau} f = \setsem{\Gamma, \psi;
  \Phi \vdash \tau[\phi :== \psi]} f\]
for each $\tau$. Then
\begin{align*}
 & \setsem{\Gamma; \Phi, \phi \vdash \phi \ol\tau } \rho \\
= \; & (\rho \phi) \ol{\setsem{\Gamma; \Phi, \phi \vdash \tau } \rho} \\
= \; & (\rho \phi) \ol{\setsem{\Gamma, \psi; \Phi \vdash \tau[\phi :==
      \psi] } \rho} \\ 
= \; & (\rho \psi) \ol{\setsem{\Gamma, \psi; \Phi \vdash \tau[\phi :==
      \psi] } \rho} \\ 
= \; & \setsem{\Gamma, \psi; \Phi \vdash \psi \ol{\tau[\phi :== \psi]}
} \rho \\ 
= \; & \setsem{\Gamma, \psi; \Phi \vdash (\phi \ol\tau)[\phi :== \psi]
} \rho 
\end{align*}
Here, the first and fifth equalities are by
Definition~\ref{def:set-sem}, and the fourth equality is by equality
of the functors $\rho \phi$ and $\rho \psi$. We also have that
\begin{align*}
& \setsem{\Gamma; \Phi, \phi \vdash \phi \ol\tau} f \\
= \; & (f \phi)_{\ol{\setsem{\Gamma; \Phi, \phi \vdash \tau} \rho'}} 
    \circ (\rho \phi) \ol{\setsem{\Gamma; \Phi, \phi \vdash \tau} f} \\
= \; & (\id_{\rho \phi})_{\ol{\setsem{\Gamma; \Phi, \phi \vdash \tau} \rho'}} 
    \circ (\rho \phi) \ol{\setsem{\Gamma; \Phi, \phi \vdash \tau} f} \\
= \; & (\rho \phi) \ol{\setsem{\Gamma; \Phi, \phi \vdash \tau} f} \\
= \; & (\rho \psi) \ol{\setsem{\Gamma, \psi; \Phi \vdash \tau[\phi :==
      \psi]} f} \\ 
= \; & (\id_{\rho \psi})_{\ol{\setsem{\Gamma, \psi; \Phi \vdash
      \tau[\phi :== \psi]} \rho'}} \circ (\rho \psi)
\ol{\setsem{\Gamma, \psi; \Phi \vdash \tau[\phi :== \psi]} f} \\ 
= \; & (f \psi)_{\ol{\setsem{\Gamma, \psi; \Phi \vdash \tau[\phi :==
        \psi]} \rho'}} \circ (\rho \psi) \ol{\setsem{\Gamma, \psi;
    \Phi \vdash \tau[\phi :== \psi]} f} \\ 
= \; & \setsem{\Gamma, \psi; \Phi \vdash \psi \ol{\tau[\phi :== \psi]}} f \\
= \; & \setsem{\Gamma, \psi; \Phi \vdash (\phi \ol\tau) [\phi :== \psi]} f
\end{align*}
\item If $\Gamma; \Phi, \phi \vdash (\mu \phi'. \lambda
  \ol\alpha. H)\ol\tau : \F$, then the induction hypothesis gives that
\[\setsem{\Gamma; \Phi, \phi', \ol{\alpha}, \phi \vdash H } \rho =
\setsem{\Gamma, \psi; \Phi, \phi', \ol{\alpha} \vdash H[\phi :== \psi]
} \rho\]
and 
\[\setsem{\Gamma; \Phi, \phi \vdash \tau } \rho = \setsem{\Gamma,
  \psi; \Phi \vdash \tau[\phi :== \psi] } \rho\]
as well as that
\[\setsem{\Gamma; \Phi, \phi', \ol{\alpha}, \phi \vdash H } f =
\setsem{\Gamma, \psi; \Phi, \phi', \ol{\alpha} \vdash H[\phi :== \psi]
} f\]
and 
\[\setsem{\Gamma; \Phi, \phi \vdash \tau } f = \setsem{\Gamma, \psi;
  \Phi \vdash \tau[\phi :== \psi] } f\]
for each $\tau$. Then
\begin{align*}
& \setsem{\Gamma; \Phi, \phi \vdash (\mu \phi'. \lambda
    \ol\alpha. H)\ol\tau } \rho \\ 
= \; & (\mu (\lambda F. \lambda \ol{A}. \setsem{\Gamma; \Phi, \phi',
  \ol{\alpha}, \phi \vdash H} \rho[\phi' := F][\ol{\alpha := A}]))
\ol{\setsem{ \Gamma; \Phi, \phi \vdash \tau} \rho} \\ 
= \; & (\mu (\lambda F. \lambda \ol{A}. \setsem{\Gamma, \psi; \Phi,
  \phi', \ol{\alpha} \vdash H[\phi :== \psi]} \rho[\phi' :=
  F][\ol{\alpha := A}])) \ol{\setsem{ \Gamma; \Phi, \phi \vdash \tau}
  \rho} \\ 
= \; &  (\mu (\lambda F. \lambda \ol{A}. \setsem{\Gamma, \psi; \Phi,
  \phi', \ol{\alpha} \vdash H[\phi :== \psi]} \rho[\phi' :=
  F][\ol{\alpha := A}])\ol{\setsem{ \Gamma, \psi; \Phi \vdash
    \tau[\phi :== \psi]} \rho} \\ 
= \; & \setsem{\Gamma, \psi; \Phi \vdash (\mu \phi'. \lambda
  \ol\alpha. H[\phi :== \psi]) \ol{\tau[\phi :== \psi]} } \rho \\ 
= \; & \setsem{\Gamma, \psi; \Phi \vdash ((\mu \phi'. \lambda
  \ol\alpha. H) \ol\tau)[\phi :== \psi] } \rho   
\end{align*} 
The first and fifth equalities are by Definition~\ref{def:set-sem}.
The second equality follows from the following equality:
\begin{align*}
&\lambda F. \lambda \ol{A}. \setsem{\Gamma; \Phi, \phi', \ol{\alpha},
    \phi \vdash H} \rho[\phi' := F][\ol{\alpha := A}] \\ 
& = \lambda F. \lambda \ol{A}. \setsem{\Gamma, \psi; \Phi, \phi',
    \ol{\alpha} \vdash H[\phi :== \psi]} \rho[\phi' := F][\ol{\alpha
      := A}] 
\end{align*}
These two maps have the same actions on objects and morphisms by the
induction hypothesis on $H$, and the fact that the extended
environment $\rho[\phi' := F][\ol{\alpha := A}]$ satisfies the
required hypothesis. They are thus equal as functors and so have the
same fixpoint. We also have that 
\begin{align*}
& \setsem{\Gamma; \Phi, \phi \vdash (\mu \phi'. \lambda
    \ol\alpha. H)\ol\tau } f \\ 
= \; & (\mu (\lambda F. \lambda \ol{A}. \setsem{\Gamma; \Phi, \phi',
  \ol{\alpha}, \phi \vdash H} f [\phi' := \id_F][\ol{\alpha :=
    \id_A}]))_{\ol{\setsem{\Gamma; \Phi, \phi \vdash \tau} \rho'}} \\ 
\; & \circ (\mu (\lambda F. \lambda \ol{A}. \setsem{\Gamma; \Phi,
  \phi', \ol{\alpha}, \phi \vdash H} \rho[\phi' := F][\ol{\alpha :=
    A}])) \ol{\setsem{ \Gamma; \Phi, \phi \vdash \tau} f} \\ 
= \; & (\mu (\lambda F. \lambda \ol{A}. \setsem{\Gamma; \Phi, \phi',
  \ol{\alpha}, \phi \vdash H} f [\phi' := \id_F][\ol{\alpha :=
    \id_A}]))_{\ol{\setsem{\Gamma, \psi; \Phi \vdash \tau[\phi :==
        \psi]} \rho'}} \\ 
\; & \circ (\mu (\lambda F. \lambda \ol{A}. \setsem{\Gamma; \Phi,
  \phi', \ol{\alpha}, \phi \vdash H} \rho[\phi' := F][\ol{\alpha :=
    A}])) \ol{\setsem{ \Gamma, \psi; \Phi \vdash \tau[\phi :== \psi]}
  f} \\ 
= \; & (\mu (\lambda F. \lambda \ol{A}. \setsem{\Gamma, \psi; \Phi,
  \phi', \ol{\alpha} \vdash H[\phi :== \psi]} f [\phi' :=
  \id_F][\ol{\alpha := \id_A}]))_{\ol{\setsem{\Gamma, \psi; \Phi
      \vdash \tau[\phi :== \psi]} \rho'}} \\ 
\; & \circ (\mu (\lambda F. \lambda \ol{A}. \setsem{\Gamma; \Phi,
  \phi', \ol{\alpha}, \phi \vdash H} \rho[\phi' := F][\ol{\alpha :=
    A}])) \ol{\setsem{ \Gamma, \psi; \Phi \vdash \tau[\phi :== \psi]}
  f} \\ 
= \; & (\mu (\lambda F. \lambda \ol{A}. \setsem{\Gamma, \psi; \Phi,
  \phi', \ol{\alpha} \vdash H[\phi :== \psi]} f [\phi' :=
  \id_F][\ol{\alpha := \id_A}]))_{\ol{\setsem{\Gamma, \psi; \Phi
      \vdash \tau[\phi :== \psi]} \rho'}} \\ 
\; & \circ (\mu (\lambda F. \lambda \ol{A}. \setsem{\Gamma, \psi;
  \Phi, \phi', \ol{\alpha} \vdash H[\phi :== \psi]} \rho[\phi' :=
  F][\ol{\alpha := A}])) \ol{\setsem{ \Gamma, \psi; \Phi \vdash
    \tau[\phi :== \psi]} f} \\ 
= \; & \setsem{\Gamma, \psi; \Phi \vdash (\mu \phi'. \lambda
  \ol\alpha. H[\phi :== \psi]) \ol{\tau[\phi :== \psi]} } f \\ 
= \; & \setsem{\Gamma, \psi; \Phi \vdash ((\mu \phi'. \lambda
  \ol\alpha. H) \ol\tau)[\phi :== \psi] } f 
\end{align*} 
The first and fifth equalities are by Definition~\ref{def:set-sem}.
The third equality is by the equality of the arguments to the first
$\mu$ operator:
\begin{align*}
& \lambda F. \lambda \ol{A}. \setsem{\Gamma; \Phi, \phi', \ol{\alpha},
    \phi \vdash H} f [\phi' := \id_F][\ol{\alpha := \id_A}] \\ 
& = \lambda F. \lambda \ol{A}. \setsem{\Gamma, \psi; \Phi, \phi',
    \ol{\alpha} \vdash H[\phi :== \psi]} f [\phi' := \id_F][\ol{\alpha
      := \id_A}] 
\end{align*}
By the induction hypothesis on $H$ and the fact that the morphism
$f[\phi' := \id_F][\ol{\alpha := \id_A}] : \rho[\phi' := F][\ol{\alpha
    := A}] \to \rho'[\phi' := F][\ol{\alpha := A}]$ still satisfies
the required hypotheses.  The fourth equality is by the equality of
the arguments to the second $\mu$ operator:
\begin{align*}
& \lambda F. \lambda \ol{A}. \setsem{\Gamma; \Phi, \phi', \ol{\alpha},
    \phi \vdash H} \rho[\phi' := F][\ol{\alpha := A}] \\ 
& = \lambda F. \lambda \ol{A}. \setsem{\Gamma, \psi; \Phi, \phi',
    \ol{\alpha} \vdash H[\phi :== \psi]} \rho[\phi' := F][\ol{\alpha
      := A}] 
\end{align*}
By the same reasoning as above, these two maps are equal as functors,
and thus have the same fixpoint.
\end{itemize}
\end{proof}

---


The following lemma ensures that substitution interacts well with type
interpretations.  It is a consequence of
Definitions~\ref{def:second-order-subst},~\ref{def:set-interp},
and~\ref{def:rel-interp}.

\begin{lemma}\label{lem:substitution}
Let $\rho$ and $\rho'$ be set environments and let $f : \rho \to \rho'$ be a
morphism of set environments. 
\begin{itemize}
\item If $\Gamma;\Phi,\ol{\alpha} \vdash F$ and $\Gamma;\Phi \vdash \tau$,
  then  
\begin{gather}\label{eq:subs-var}
\setsem{\Gamma;\Phi \vdash F[\ol{\alpha := \tau}]}\rho =
\setsem{\Gamma;\Phi,\ol{\alpha} \vdash F}\rho[\ol{\alpha :=
\setsem{\Gamma;\Phi \vdash \tau}\rho}]\\
\intertext{and}\label{eq:subs-var-morph}
\setsem{\Gamma;\Phi \vdash F[\ol{\alpha := \tau}]}f =
\setsem{\Gamma;\Phi,\ol{\alpha} \vdash F}f[\ol{\alpha :=
\setsem{\Gamma;\Phi \vdash \tau}f}]
\end{gather}
\item If $\Gamma;\Phi,\phi^k\vdash F$ and
  $\Gamma;\Phi,\alpha_1...\alpha_k \vdash H$, then 
\begin{gather}\label{eq:subs-const}
\setsem{\Gamma; \Phi \vdash F[\phi := H]}\rho
= \setsem{\Gamma; \Phi, \phi \vdash F}\rho
[\phi := \lambda \ol{A}.\, \setsem{\Gamma;\Phi,\overline{\alpha}\vdash
    H}\rho[\overline{\alpha := A}]] \\ 
\intertext{and}\label{eq:subs-const-morph}
\setsem{\Gamma; \Phi \vdash F[\phi := H]}f
= \setsem{\Gamma; \Phi, \phi \vdash F}f
[\phi := \lambda \ol{A}.\,\setsem{\Gamma;\Phi,\overline{\alpha}\vdash
    H}f[\overline{\alpha := \id_{\ol{A}}}]] 
\end{gather}
\end{itemize}
Analogous identities hold for relation environments and morphisms
between them.
\end{lemma}
\end{comment}

\begin{comment}
\begin{proof}
The proofs for the set and relational interpretations are completely
analogous, so we just prove the former. Likewise, we only prove
Equations~\ref{eq:subs-var} and~\ref{eq:subs-const}, since the proofs
for Equations~\ref{eq:subs-var-morph} and~\ref{eq:subs-const-morph}
are again analogous. Finally, we prove Equation~\ref{eq:subs-var} for
substitution for just a single type variable since the proof for
multiple simultaneous substitutions proceeds similarly.

Although Equation~\ref{eq:subs-var} is a special case of
Equation~\ref{eq:subs-const}, it is convenient to prove
Equation~\ref{eq:subs-var} first, and then use it to prove
Equation~\ref{eq:subs-const}. We prove Equation~\ref{eq:subs-var} by
induction on the structure of $F$ as follows:
\begin{itemize}
\item If $\Gamma; \emptyset \vdash F : \T$, or if $F$ is $\onet$ or
  $\zerot$, then $F$ does not contain any functorial variables to
  replace, so there is nothing to prove.
\item If $F$ is $F_1 \times F_2$ or $F_1 + F_2$, then the substitution
  distributes over the product or coproduct as appropriate, so the
  result follows immediately from the induction hypothesis.
\item If $F = \beta$ with $\beta \neq \alpha$, then there is nothing to
  prove.
\item If $F = \alpha$, then
\[\setsem{\Gamma;\Phi\vdash\alpha[\alpha := \tau]}\rho \, = \,
\setsem{\Gamma;\Phi\vdash\tau}\rho \, = \,
\setsem{\Gamma;\Phi,\alpha\vdash\alpha}\rho[\alpha :=
  \setsem{\Gamma;\Phi\vdash\tau}\rho]\]
\item If $F = \phi \overline{\sigma}$ with $\phi \neq \alpha$, then
\[\begin{array}{ll}
 &\setsem{\Gamma; \Phi \vdash (\phi \overline{\sigma})[\alpha := \tau]}\rho\\
=&\setsem{\Gamma; \Phi \vdash \phi (\overline{\sigma[\alpha := \tau]})}\rho \\
=&(\rho\phi)\overline{\setsem{\Gamma;\Phi\vdash\sigma[\alpha := \tau]}\rho} \\
=&(\rho\phi)\overline{\setsem{\Gamma;\Phi,\alpha\vdash\sigma}
      \rho [\alpha := \setsem{\Gamma;\Phi\vdash\tau}] } \\
=&\setsem{\Gamma; \Phi, \alpha \vdash \phi \overline{\sigma}}
      \rho [\alpha := \setsem{\Gamma;\Phi\vdash\tau}]
\end{array}\]
Here, the third equality is by the induction hypothesis.
\item If $F = (\mu \phi. \lambda \overline{\beta}. G)
  \overline{\sigma}$, then
\[\begin{array}{ll}
  & \setsem{\Gamma; \Phi \vdash ((\mu \phi. \lambda
  \overline{\beta}. G) \overline{\sigma}) [\alpha := \tau]}\rho \\
=&\setsem{\Gamma; \Phi \vdash (\mu \phi. \lambda
  \overline{\beta}. G[\alpha := \tau]) (\overline{\sigma [\alpha :=
      \tau]})}\rho \\ 
=&\mu(\setsem{\Gamma; \Phi, \phi, \overline{\beta} \vdash G[\alpha := \tau]}
      \rho[\phi := \text{--}][\overline{\beta := \text{--}}])
      (\overline{\setsem{\Gamma;\Phi \vdash \sigma [\alpha := \tau]}\rho}) \\
=&\mu(\setsem{\Gamma; \Phi, \phi, \overline{\beta}, \alpha \vdash G}
\rho [\alpha := \setsem{\Gamma;\Phi \vdash \tau}\rho] [\phi := \text{--}]
[\overline{\beta := \text{--}}]) \\
 &\hspace{0.5in} (\overline{\setsem{\Gamma;\Phi,\alpha \vdash \sigma}
   \rho [\alpha := \setsem{\Gamma;\Phi\vdash \tau}\rho]}) \\
=&\setsem{\Gamma;\Phi,\alpha\vdash(\mu\phi.\lambda\overline{\beta}.G)
  \overline{\sigma}}\rho  [\alpha := \setsem{\Gamma;\Phi \vdash
   \tau}\rho] 
\end{array}  \]
Here, the third equality is by the induction hypothesis and weakening.
\end{itemize}

We now prove Equation~\ref{eq:subs-const}, again by induction on the
structure of $F$.
\begin{itemize}
\item If $\Gamma; \emptyset \vdash F : \T$, or if $F$ is $\onet$ or
  $\zerot$, then $F$ does not contain any functorial variables to
  replace, so there is nothing to prove.
\item If $F$ is $F_1 \times F_2$ or $F_1 + F_2$, then the substitution
  distributes over the product or coproduct as appropriate, so the
  result follows immediately from the induction hypothesis.
\item If $F = \phi \overline{\tau}$, then
\[\begin{array}{ll}
 &\setsem{\Gamma; \Phi \vdash (\phi \overline{\tau})[\phi := H]}\rho \\
=&\setsem{\Gamma; \Phi \vdash H[\overline{\alpha:= \tau[\phi := H]}]}\rho \\
=&\setsem{\Gamma; \Phi \vdash H}\rho [\overline{\alpha:=
    \setsem{\Gamma;\Phi\vdash\tau[\phi := H]}\rho}] \\ 
=&\setsem{\Gamma; \Phi \vdash H}\rho [\overline{\alpha:=
    \setsem{\Gamma;\Phi,\phi\vdash\tau} \rho [\phi :=
      \setsem{\Gamma;\Phi,\overline{\alpha}\vdash
        H}\rho[\overline{\alpha := \text{--}}]]}] \\
=&\setsem{\Gamma; \Phi, \phi \vdash \phi \overline{\tau}}\rho [\phi :=
  \setsem{\Gamma;\Phi,\overline{\alpha}\vdash H}\rho[\overline{\alpha
      := \text{--}}]] 
  \end{array}\]
Here, the first equality is by
Definition~\ref{def:second-order-subst}, the second is by
Equation~\ref{eq:subs-var}, the third is by the induction hypothesis,
and the fourth is by Definition~\ref{def:set-sem}.
\item If $F = \psi \overline{\tau}$ with $\psi \neq \phi$, then the
  proof is similar to that for the previous case, but simpler, because
  $\phi$ only needs to be substituted in the arguments $\ol{\tau}$ of
  $\psi$.
\item If $F = (\mu \psi. \lambda \overline{\beta}. G)
  \overline{\tau}$, then
\[\begin{array}{ll}
 &\setsem{\Gamma; \Phi \vdash ((\mu \psi. \lambda \overline{\beta}. G)
 \overline{\tau})[\phi := H]}\rho \\ 
=&\setsem{\Gamma; \Phi \vdash (\mu \psi. \lambda
  \overline{\beta}. G[\phi := H]) (\overline{\tau[\phi := H]})}\rho \\
=&\mu(\setsem{\Gamma; \Phi, \psi, \overline{\beta} \vdash G[\phi := H]}
    \rho[\psi := \text{--}][\overline{\beta := \text{--}}])
    (\overline{\setsem{\Gamma;\Phi \vdash \tau[\phi := H]}\rho}) \\
=&\mu(\setsem{\Gamma; \Phi, \psi, \overline{\beta}, \phi \vdash G}
\rho [\phi := \setsem{\Gamma;\Phi,\overline{\alpha}\vdash
    H}\rho[\overline{\alpha := \text{--}}]][\psi := \text{--}]
     [\overline{\beta := \text{--}}]) \\
&\hspace{0.5in} (\overline{\setsem{\Gamma;\Phi,\phi \vdash \tau} \rho
       [\phi := \setsem{\Gamma;\Phi,\overline{\alpha}\vdash H}
         \rho[\overline{\alpha := \text{--}}]]}) \\
=&\setsem{\Gamma;\Phi,\phi\vdash(\mu\psi.\lambda\overline{\beta}.G)
  \overline{\tau}}\rho [\phi :=
  \setsem{\Gamma;\Phi,\overline{\alpha}\vdash H}\rho[\overline{\alpha
      := \text{--}}]] 
\end{array}\]
Here, the first equality is by
Definition~\ref{def:second-order-subst}, the second and fourth are by
Definition~\ref{def:set-sem}, and the third is by the induction
hypothesis and weakening.
\end{itemize}
\end{proof}
\end{comment}

\section{The Identity Extension Lemma}\label{sec:iel}

In most treatments of parametricity, equality relations on sets are
taken as {\em given} --- either directly as diagonal relations, or
perhaps via reflexive graphs if kinds are also being tracked --- and
the graph relations used to validate existence of initial algebras are
defined in terms of them. We take a different approach here, giving a
categorical definition of graph relations for morphisms (i.e., natural
transformations) between functors and {\em constructing} equality
relations as particular graph relations. Our definitions specialize to
the usual ones for the graph relation for a morphism between sets and
the equality relation for a set. In light of its novelty, we spell out
our construction in detail.

The standard definition of the graph for a morphism $f : A \to B$ in
$\set$ is the relation $\graph{f} : \rel(A,B)$ defined by $(x,y) \in
\graph{f}$ iff $fx = y$. This definition naturally generalizes to
associate to each natural transformation between $k$-ary functors on
$\set$ a $k$-ary relation transformer as follows:

\begin{dfn}\label{dfn:graph-nat-transf}
If $F, G: \Set^k \to \Set$
%are $k$-ary functors
and $\alpha : F \to G$ is a natural transformation, then the functor
$\graph{\alpha}^*: \rel^k \to \rel$ is defined as follows. Given $R_1
: \rel(A_1, B_1),...,R_k : \rel(A_k,B_k)$, let $\iota_{R_i} : R_i
\hookrightarrow A_i \times B_i$, for $i = 1,...,k$, be the inclusion
of $R_i$ as a subset of $A_i \times B_i$,
%. By the universal property of the product, there exists a
let $h_{\overline{A \times B}}$ be the unique morphism making the diagram
{\footnotesize\[\begin{tikzcd}[row sep = large]
        F\overline{A}
        &F(\overline{A \times B})
        \ar[l, "{F\overline{\pi_1}}"']
        \ar[r, "{F\overline{\pi_2}}"]
        \ar[d, dashed, "{h_{\overline{A \times B}}}"]
        &F\overline{B}
        \ar[r, "{\alpha_{\ol{B}}}"]
        &G\overline{B} \\
        &F\overline{A} \times G\overline{B}
        \ar[ul, "{\pi_1}"] \ar[urr, "{\pi_2}"']
\end{tikzcd}\]}

\noindent
commute, and let $h_{\overline{R}} : F\overline{R} \to F\overline{A}
\times G\overline{B}$ be $h_{\overline{A \times B}} \circ
F\overline{\iota_R}$. Further, let $\alpha^\wedge\overline{R}$ be the
subobject through which $h_{\overline{R}}$ is factorized by the
mono-epi factorization system in $\set$, as shown in the
following diagram:
{\footnotesize\[\begin{tikzcd}
        F\overline{R}
        \ar[rr, "{h_{\overline{R}}}"]
        \ar[dr, twoheadrightarrow, "{q_{\alpha^\wedge\overline{R}}}"']
        &&F\overline{A} \times G\overline{B} \\
        &\alpha^\wedge\overline{R}
        \ar[ur, hookrightarrow, "{\iota_{\alpha^\wedge\overline{R}}}"']
\end{tikzcd}\]}

\noindent
Then $\alpha^\wedge\overline{R} : \rel(F\overline{A}, G\overline{B})$
by construction, so the action of $\langle \alpha \rangle^*$ on
objects can be given by $\langle \alpha \rangle^* \overline{(A,B,R)} =
(F\overline{A}, G\overline{B}, \iota_{\alpha^\wedge
  \overline{R}}\alpha^\wedge\overline{R})$. Its action on morphisms is
given by $\graph{\alpha}^*\overline{(\beta, \beta')} =
(F\overline\beta, G\overline\beta')$.
\end{dfn}

The data in Definition~\ref{dfn:graph-nat-transf} yield the {\em graph relation
transformer for $\alpha$}, denoted $\graph{\alpha} = (F, G,
\graph{\alpha}^*)$.
% yield a relation transformer $\graph{\alpha} = (F, G,
%\graph{\alpha}^*)$, called the {\em graph relation transformer for
%$\alpha$}.

\begin{lemma}\label{lem:graph-reln-functors}
If $F,G : [\set^k,\set]$, and if $\alpha : F \to G$ is a natural
transformation, then $\graph{\alpha}$ is in $RT_k$.
\end{lemma}
\begin{proof}
Clearly, $\graph{\alpha}^*$ is $\omega$-cocontinuous, so
$\graph{\alpha}^* : [\rel^k,\rel]$. Now, suppose $\overline{R :
  \rel(A, B)}$, $\overline{S : \rel(C, D)}$, and $\overline{(\beta,
  \beta') : R \to S}$. We want to show that there exists a morphism
$\epsilon : \alpha^\wedge\overline{R} \to \alpha^\wedge\overline{S}$
such that the diagram on the left below commutes. Since
$\ol{(\beta,\beta') : R \to S}$, there exist $\overline{\gamma : R \to
  S}$ such that each diagram in the middle commutes.
Moreover, since both $h_{\overline{C \times D}} \circ
F(\overline{\beta \times \beta'})$ and $(F\overline{\beta} \times
G\overline{\beta'}) \circ h_{\overline{A \times B}}$ make the diagram
on the right commute, they must be equal.
\begin{figure*}[ht]
\vspace*{-0.1in}
  \hspace*{-0.85in}
  \begin{minipage}[b]{0.25\linewidth}
 {\small    \[
    \begin{tikzcd}
        \alpha^\wedge\overline{R}
        \ar[r, hookrightarrow, "{\iota_{\alpha^\wedge\overline{R}}}"]
        \ar[d, "{\epsilon}"']
        & F\overline{A} \times G\overline{B}
        \ar[d, "{F\overline{\beta} \times G\overline{\beta'}}"] \\
        \alpha^\wedge\overline{S}
        \ar[r, hookrightarrow, "{\iota_{\alpha^\wedge\overline{S}}}"']
        & F\overline{C} \times G\overline{D}
    \end{tikzcd}
    \]}
\end{minipage}
\begin{minipage}[b]{0.25\linewidth}
{\small    \[
    \begin{tikzcd}
        R_i
        \ar[d, "{\gamma_i}"']
        \ar[r, hookrightarrow, "{\iota_{R_i}}"]
        &A_i \times B_i
        \ar[d, "{\beta_i \times \beta'_i}"] \\
        S_i
        \ar[r, hookrightarrow, "{\iota_{S_i}}"]
        &C_i \times D_i
    \end{tikzcd}
    \]}
\end{minipage}
\begin{minipage}[b]{0.25\linewidth}
{\footnotesize \[
      \begin{tikzcd}[row sep = large]
          F\overline{C}
          &F\overline{C} \times F\overline{D}
          \ar[l, "{\pi_1}"'] \ar[r, "{\pi_2}"]
          &F\overline{D}
          \ar[r, "{\alpha_{\ol{D}}}"]
          &G\overline{D}\\
          &F(\overline{A \times B})
          \ar[u, dashed, "{\exists !}"]
          \ar[ul, "{F\ol{\pi_1} \circ F(\overline{\beta \times \beta'})}"]
          \ar[urr, "{\alpha_{\ol{D}} \circ F\ol{\pi_2} \circ F(\overline{\beta \times \beta'})}"']
      \end{tikzcd}
      \]}
\end{minipage}
\end{figure*}
%{\footnotesize    \[
%    \begin{tikzcd}
%        \alpha^\wedge\overline{R}
%        \ar[r, hookrightarrow, "{\iota_{\alpha^\wedge\overline{R}}}"]
%        \ar[d, "{\epsilon}"']
%        & F\overline{A} \times G\overline{B}
%        \ar[d, "{F\overline{\beta} \times G\overline{\beta'}}"] \\
%        \alpha^\wedge\overline{S}
%        \ar[r, hookrightarrow, "{\iota_{\alpha^\wedge\overline{S}}}"']
%        & F\overline{C} \times G\overline{D}
%    \end{tikzcd}
%    \]}
%    commutes.
%    Since $\ol{(\beta,\beta') : R \to S}$, there exist $\overline{\gamma : R \to S}$
%    such that each diagram
%{\footnotesize    \[
%    \begin{tikzcd}
%        R_i
%        \ar[d, "{\gamma_i}"']
%        \ar[r, hookrightarrow, "{\iota_{R_i}}"]
%        &A_i \times B_i
%        \ar[d, "{\beta_i \times \beta'_i}"] \\
%        S_i
%        \ar[r, hookrightarrow, "{\iota_{S_i}}"]
%        &C_i \times D_i
%    \end{tikzcd}
%    \]}
%    commutes.
%Moreover, since both $h_{\overline{C \times D}} \circ
%F(\overline{\beta \times \beta'})$ and $(F\overline{\beta} \times
%G\overline{\beta'}) \circ h_{\overline{A \times B}}$ make
%{\footnotesize \[
%      \begin{tikzcd}[row sep = large]
%          F\overline{C}
%          &F\overline{C} \times F\overline{D}
%          \ar[l, "{\pi_1}"'] \ar[r, "{\pi_2}"]
%          &F\overline{D}
%          \ar[r, "{\alpha_{\ol{D}}}"]
%          &G\overline{D}\\
%          &F(\overline{A \times B})
%          \ar[u, dashed, "{\exists !}"]
%          \ar[ul, "{F\ol{\pi_1} \circ F(\overline{\beta \times \beta'})}"]
%          \ar[urr, "{\alpha_{\ol{D}} \circ F\ol{\pi_2} \circ F(\overline{\beta \t%imes \beta'})}"']
%      \end{tikzcd}
%      \]}
%      commute, they must be equal.

\vspace*{-0.1in}

\noindent
We therefore get that the right-hand square in the diagram on the left
below commutes, and thus that the entire diagram does as well.
Finally, by the left-lifting property of $q_{F^\wedge\overline{R}}$
with respect to $\iota_{F^\wedge\overline{S}}$ given by the epi-mono
factorization system, there exists an $\epsilon$ such that the diagram
on the right below commutes.
\begin{figure*}[ht]
  \vspace*{-0.1in}
  \hspace*{-0.5in}
  \begin{minipage}[b]{0.45\linewidth}
{\footnotesize \[
      \begin{tikzcd}
          F\overline{R}
          \ar[d, "{F\overline{\gamma}}"']
          \ar[r, hookrightarrow, "{F\overline{\iota_R}}"]
          \ar[rr, bend left, "{h_{\overline{R}}}"]
          &F(\overline{A \times B})
          \ar[d, "{F(\overline{\beta \times \beta'})}"]
          \ar[r, "{h_{\overline{A \times B}}}"]
          &F\overline{A} \times G\overline{B}
          \ar[d, "{F\overline{\beta} \times F\overline{\beta'}}"] \\
          F\overline{S}
          \ar[r, hookrightarrow, "{F\overline{\iota_S}}"']
          \ar[rr, bend right, "{h_{\overline{S}}}"']
          &F(\overline{C \times D})
          \ar[r, "{h_{\overline{C \times D}}}"']
          &F\overline{C} \times G\overline{D}
      \end{tikzcd}
      \]}
\end{minipage}
  \vspace*{-0.5in}
  \begin{minipage}[b]{0.45\linewidth}
      {\footnotesize
        \[  \vspace*{0.4in}
      \begin{tikzcd}
          F\overline{R}
          \ar[d, "{F\overline{\gamma}}"']
          \ar[r, twoheadrightarrow, "{q_{\alpha^\wedge\overline{R}}}"]
          &\alpha^\wedge\overline{R}
          \ar[d, dashed, "{\epsilon}"]
          \ar[r, hookrightarrow, "{\iota_{\alpha^\wedge\overline{R}}}"]
          &F\overline{A} \times G\overline{B}
          \ar[d, "{F\overline{\beta} \times G\overline{\beta'}}"] \\
          F\overline{S}
          \ar[r, twoheadrightarrow, "{q_{\alpha^\wedge\overline{S}}}"']
          &\alpha^\wedge\overline{S}
          \ar[r, hookrightarrow, "{\iota_{\alpha^\wedge\overline{S}}}"']
          &F\overline{C} \times G\overline{D}
      \end{tikzcd}
      \]}
\end{minipage}
\end{figure*}


\begin{comment}

Finally, by the left-lifting property of $q_{F^\wedge\overline{R}}$
      with respect to $\iota_{F^\wedge\overline{S}}$ given by the epi-mono
      factorization system, there exists an $\epsilon$ such that the
      following diagram commutes:
      {\footnotesize
        \[
      \begin{tikzcd}
          F\overline{R}
          \ar[d, "{F\overline{\gamma}}"']
          \ar[r, twoheadrightarrow, "{q_{\alpha^\wedge\overline{R}}}"]
          &\alpha^\wedge\overline{R}
          \ar[d, dashed, "{\epsilon}"]
          \ar[r, hookrightarrow, "{\iota_{\alpha^\wedge\overline{R}}}"]
          &F\overline{A} \times G\overline{B}
          \ar[d, "{F\overline{\beta} \times G\overline{\beta'}}"] \\
          F\overline{S}
          \ar[r, twoheadrightarrow, "{q_{\alpha^\wedge\overline{S}}}"']
          &\alpha^\wedge\overline{S}
          \ar[r, hookrightarrow, "{\iota_{\alpha^\wedge\overline{S}}}"']
          &F\overline{C} \times G\overline{D}
      \end{tikzcd}
      \]}

      \vspace*{-0.1in}
\end{comment}
\end{proof}

\vspace*{0.1in}

If $f : A \to B$ is a morphism in $\set$ then the definition of the
graph relation transformer $\langle f \rangle$ for $f$ as a natural
transformation between $0$-ary functors $A$ and $B$ coincides with its
standard definition.  Graph relation transformers are thus a
reasonable extension of graph relations to functors.
\begin{comment}

\vspace*{0.2in}
  
If $f : A \to B$ is a function with graph relation $\graph{f} = (A, B,
\graph{f}^*)$, then $\langle \id_{A}, f \rangle : A \to A \times B$
and $\langle \id_{A}, f \rangle\, A = \graph{f}^*$.  Moreover, if
$\iota_{\graph{f}} : \graph{f}^* \hookrightarrow A \times B$ is the
inclusion of $\graph{f}^*$ into $A \times B$ then there is an
isomorphism of subobjects
\[\begin{tikzcd}
A \ar[rr, "{\cong}"] \ar[dr, "{\langle \id_{A}, f \rangle}"']
&&{\graph{f}^*} \ar[dl, "{\iota_{\graph{f}}}"]\\
&A \times B
\end{tikzcd}\]

We also note that if $f : A \to B$ is a function seen as a natural
transformation between 0-ary functors, then $\graph{f}$ is (the 0-ary
relation transformer associated with) the graph relation of $f$.
Indeed, we need to apply Definition~\ref{dfn:graph-nat-transf} with $k
= 0$, i.e., to the degenerate relation $\ast : \rel(\ast, \ast)$.  As
degenerate $0$-ary functors, $A$ and $B$ are constant functors, i.e.,
$A\, \ast = A$ and $B\, \ast = B$.  By the universal property of the
product, there exists a unique $h$ making the diagram
\[ \begin{tikzcd}[row sep = large]
        A
        &A
        \ar[l, equal]
        \ar[r, equal]
        \ar[d, dashed, "{h}"]
        &A
        \ar[r, "{f}"]
        &B \\
        &A \times B
        \ar[ul, "{\pi_1}"] \ar[urr, "{\pi_2}"']
\end{tikzcd}\]
commute. Since $\iota_\ast : \ast \to \ast$ is the identity on $\ast$,
and $A\, \id_{\ast} = \id_{A}$, we have $h_{\ast} = h$.  Moreover,
$h_{\overline{A \times B}} = \langle \id_{A}, f \rangle$ is a
monomorphism in $\set$ because $\id_{A}$ is.  Then,
$\iota_{f^\wedge\ast} = \langle \id_{A}, f \rangle$ and $f^\wedge\ast
= A$, from which we deduce that $\iota_{f^\wedge\ast} f^\wedge\ast =
\langle \id_{A}, f \rangle\, A = \graph{f}^*$. This ensures that the
graph of $f$ as a 0-ary natural transformation coincides with the
graph of $f$ as a morphism in $\set$, and so that
Definition~\ref{dfn:graph-nat-transf} is a reasonable generalization
of Definition~\ref{def:graph}.

Just as the equality relation $\Eq_B$ on a set $B$ coincides with
$\graph{\id_B}$, the graph of the identity on the set, so we can
define the equality relation transformer to be the graph of the
identity natural transformation. This gives

\begin{dfn}
Let $F : [\set^k, \set]$.  The equality relation transformer on $F$ is
defined to be $\Eq_F = \graph{\id_{F}}$. This entails that $Eq_F = (F,
F, \Eq_F^*)$ with $\Eq_F^* = \graph{\id_{F}}^*$.
\end{dfn}
\end{comment}

To prove the IEL, we will need to know that the equality relation
transformer preserves equality relations; see
Equation~\ref{eq:eq-pres-eq} below. This will follow from the next
lemma, which shows how to compute the action of a graph relation
transformer on any graph relation.

\begin{lemma}\label{lem:eq-reln-equalities}
If $\alpha : F \to G$ is a morphism in $[\Set^k, \Set]$
and $f_1: A_1 \to B_1, ..., f_k : A_k \to B_k$,
then $\graph{\alpha}^* \graph{\overline{f}}
= \langle G \ol{f} \circ \alpha_{\ol{A}} \rangle
= \langle \alpha_{\ol{B}} \circ F \ol{f} \rangle$.
\end{lemma}
\begin{proof}
Since $h_{\overline{A \times B}}$ is the unique morphism making the
bottom triangle of the diagram on the left below commute, and since
$h_{\graph{\overline{f}}} = h_{\overline{A \times B}} \circ F
\,\ol{\iota_{\graph{f}}} = h_{\overline{A \times B}} \circ F
\overline{\langle \id_A, f \rangle}$, the universal property of the
product depicted in the diagram on the right gives
$h_{\graph{\overline{f}}} = \langle \id_{F \ol{A}}, \alpha_{\ol{B}}
\circ F\ol{f} \rangle : F \ol{A} \to F \ol{A} \times G \ol{B}$.

\begin{figure*}[ht]
  \vspace*{-0.15in}
%  \hspace*{-0.5in}
  \begin{minipage}[b]{0.45\linewidth}
{\footnotesize
\[\begin{tikzcd}[row sep = large]
        &F\overline{A}
        \ar[d, "{F \overline{\langle \id_A, f \rangle}}" description]
        \ar[dl, equal]
        \ar[dr, "{F\ol{f}}"]\\
        F\overline{A}
        &F(\overline{A \times B})
        \ar[l, "{F\overline{\pi_1}}"']
        \ar[r, "{F\overline{\pi_2}}"]
        \ar[d, "{h_{\overline{A \times B}}}"]
        &F\overline{B}
        \ar[r, "{\alpha_{\ol{B}}}"]
        &G\overline{B}\\
        &F\overline{A} \times G\overline{B}
        \ar[ul, "{\pi_1}"] \ar[urr, "{\pi_2}"']
\end{tikzcd}\]}
\end{minipage}
%  \vspace*{-0.5in}
  \begin{minipage}[b]{0.45\linewidth}
{\footnotesize
\[
      \begin{tikzcd}[row sep = large]
          F\overline{A}
          &F\overline{A} \times G\overline{B}
          \ar[l, "{\pi_1}"'] \ar[r, "{\pi_2}"]
          &G\overline{B}\\
          &F\overline{A}
          \ar[u, dashed, "{\exists !}"]
          \ar[ul, equal]
          \ar[r, "{F\ol{f}}"']
          &F{\ol{B}}
          \ar[u, "{\alpha_{\ol{B}}}"']
      \end{tikzcd}
      \]}
\end{minipage}
\end{figure*}

\vspace*{-0.1in}

\noindent
Moreover, $\langle \id_{F \ol{A}}, \alpha_{\ol{B}} \circ
F\ol{f} \rangle$ is a monomorphism in $\set$ because $\id_{F \ol{A}}$
is, so its epi-mono factorization gives $\iota_{\alpha^\wedge
  \graph{\overline{f}}} = \langle \id_{F \ol{A}}, \alpha_{\ol{B}}
\circ F\ol{f} \rangle$, and thus $\alpha^\wedge \graph{\overline{f}} =
%the domain of $\iota_{\alpha^\wedge \graph{\overline{f}}}$, is equal
%to
F\overline{A}$.  Then $\iota_{\alpha^\wedge
  \graph{\overline{f}}} \alpha^\wedge \graph{\overline{f}} = \langle
\id_{F \ol{A}}, \alpha_{\ol{B}} \circ F\ol{f} \rangle (F \ol{A}) =
\graph{ \alpha_{\ol{B}} \circ F\ol{f} }^*$,
%(where the last equality is by Remark~\ref{rmk:graph-fn}).
%We therefore conclude that
so that $\graph{ \alpha }^* \graph{ \overline{f} } = (F\overline{A},
G\overline{B}, \iota_{\alpha^\wedge \graph{\overline{f}}}\,
\alpha^\wedge \graph{\overline{f}}) = (F\overline{A}, G\overline{B},
\graph{ \alpha_{\ol{B}} \circ F\ol{f} }^*) = \graph{ \alpha_{\ol{B}}
  \circ F\ol{f} }$.  Finally, $\alpha_{\ol{B}} \circ F\ol{f} = G\ol{f}
\circ \alpha_{\ol{A}}$ by naturality of $\alpha$.
\end{proof}

The {\em equality relation transformer} on $F : [\set^k,\set]$ is
defined to be $\Eq_F = \graph{\id_{F}}$. Specifically, $Eq_F = (F, F,
\Eq_F^*)$ with $\Eq_F^* = \graph{\id_{F}}^*$, and
Lemma~\ref{lem:eq-reln-equalities} indeed ensures that
\begin{equation}\label{eq:eq-pres-eq}
\Eq^*_F \ol{\Eq_A}
= \graph{\id_F}^* \graph{\id_{\ol{A}}}
= \graph{F \id_{\ol{A}} \circ (\id_F)_{\ol{A}}}
= \graph{\id_{F\ol{A}} \circ \id_{F\ol{A}}}
= \graph{\id_{F\ol{A}}}
= \Eq_{F\ol{A}}
\end{equation}
%$\Eq^*_F \,\ol{\Eq_A} = \Eq_{F\ol{A}}$
for all $\ol{A : \set}$.
\begin{comment}
  \begin{lemma}
$\Eq^*_F \,\ol{\Eq_A}
= \Eq_{F\ol{A}}$ for all $\ol{A : \set}$.
  \end{lemma}
\begin{proof}
We have that\
\begin{equation}\label{eq:eq-pres-eq}
\Eq^*_F \ol{\Eq_A}
= \graph{\id_F}^* \graph{\id_{\ol{A}}}
= \graph{F \id_{\ol{A}} \circ (\id_F)_{\ol{A}}}  %by your Lemma 5
= \graph{\id_{F\ol{A}} \circ \id_{F\ol{A}}}
%= \graph{\id_F \ol{A} \circ \id_{\ol{A}}}
= \graph{\id_{F\ol{A}}}
= \Eq_{F\ol{A}}
\end{equation}
The second identity here is by Lemma~\ref{lem:eq-reln-equalities}.
\end{proof}
\end{comment}
Graph relation transformers in general, and equality relation
transformers in particular, extend to relation environments in the
obvious ways.
%\begin{definition}
Indeed, if $\rho, \rho' : \setenv$ and $f : \rho \to \rho'$, then the
{\em graph relation environment} $\graph{f}$ is defined pointwise by
$\graph{f} \phi = \graph{f \phi}$ for every $\phi$. This entails that
$\pi_1 \graph{f} = \rho$ and $\pi_2 \graph{f} = \rho'$. In particular,
the {\em equality relation environment} $\Eq_\rho$ is defined to be
$\graph{\id_{\rho}}$. This entails that $\Eq_\rho \phi = \Eq_{\rho
  \phi}$ for every $\phi$.
%\end{definition}
With these definitions in hand, we can state and prove both an
Identity Extension Lemma and a Graph Lemma for our calculus.
\begin{thm}[IEL]\label{thm:iel}
  If $\rho : \setenv$ and $\Gamma; \Phi \vdash \tau$ then
  $\relsem{\Gamma; \Phi \vdash \tau} \Eq_\rho = \Eq_{\setsem{\Gamma;
      \Phi \vdash \tau}\rho}$.
\end{thm}

The proof is by induction on the structure of $\tau$. Only the
application and fixpoint cases are non-routine. Both use
Lemma~\ref{lem:eq-reln-equalities}. The latter also uses the
observation that, for every $n \in \nat$, the following intermediate
results can be proved {\color{blue} by simultaneous induction with
  Theorem~\ref{thm:iel}}:
%\begin{equation}\label{eq:iel-fix-point-intermediate1}
$T^n_{\Eq_{\rho}} K_0\, \ol{\Eq_{\setsem{\Gamma; \Phi \vdash
      \tau}\rho}} = (\Eq_{(T^\set_\rho)^n K_0})^*
\ol{\Eq_{\setsem{\Gamma; \Phi \vdash \tau}\rho}}$\;
%\end{equation}
and\;
%\begin{equation}\label{eq:iel-fix-point-intermediate2}
%\begin{split}
$ \relsem{\Gamma; \Phi, \phi, \ol{\alpha} \vdash H} \Eq_{\rho} [\phi
  := T^{n}_{\Eq_{\rho}} K_0] \overline{[\alpha := \Eq_{\setsem{\Gamma;
        \Phi \vdash \tau}\rho}]} =$\\
$\relsem{\Gamma; \Phi, \phi,
  \ol{\alpha} \vdash H} \Eq_{\rho} [\phi := \Eq_{(T^\set_\rho)^n K_0}]
\overline{[\alpha := \Eq_{\setsem{\Gamma; \Phi \vdash \tau}\rho}]}$.
%\end{split}
%\end{equation}

\begin{comment}
\begin{proof}
By induction on the structure of $\tau$.
\begin{itemize}
\item $\relsem{\Gamma; \emptyset \vdash v}\Eq_{\rho} = \Eq_{\rho} v =
  \Eq_{\rho v} = \Eq_{\setsem{\Gamma; \emptyset \vdash v}\rho}$ where
  $v \in \Gamma$.
\item By definition, $\relsem{\Gamma; \emptyset \vdash
  \Nat^{\overline\alpha} \,F\,G} \Eq_{\rho}$ is the relation on
  $\setsem{\Gamma; \emptyset \vdash \Nat^{\overline\alpha} \,F\,G}
  \rho$ relating $t$ and $t'$ if, for all ${R_1 :
    \rel(A_1,B_1)},...,{R_k : \rel(A_k,B_k)}$, $(t_{\overline{A}},
  t'_{\overline{B}})$ is a morphism from $\relsem{\Gamma;
    \overline\alpha \vdash F} \Eq_{\rho}\overline{[\alpha := R]}$ to
  $\relsem{\Gamma ; \overline\alpha \vdash G}
  \Eq_{\rho}\overline{[\alpha := R]}$ in $\rel$.  To prove that this
  is equal to $\Eq_{\setsem{\Gamma; \emptyset \vdash
      \Nat^{\overline\alpha} \,F\,G} \rho}$ we need to show that
  $(t_{\overline{A}}, t'_{\overline{B}})$ is a morphism from
  $\relsem{\Gamma; \overline\alpha \vdash F}
  \Eq_{\rho}\overline{[\alpha := R]}$ to $\relsem{\Gamma ;
    \overline\alpha \vdash G} \Eq_{\rho}\overline{[\alpha := R]}$ in
  $\rel$ for all ${R_1 : \rel(A_1,B_1)},...,{R_k : \rel(A_k,B_k)}$ if
  and only if $t = t'$ and $(t_{\overline{A}}, t_{\overline{B}})$ is a
  morphism from $\relsem{\Gamma; \overline\alpha \vdash F}
  \Eq_{\rho}\overline{[\alpha := R]}$ to $\relsem{\Gamma ; \overline
    \alpha \vdash G} \Eq_{\rho}\overline{[\alpha := R]}$ in $\rel$ for
  all ${R_1 : \rel(A_1,B_1)}, ...,$ ${R_k : \rel(A_k,B_k)}$. The only
  interesting part of this equivalence is to show that if
  $(t_{\overline{A}}, t'_{\overline{B}})$ is a morphism from
  $\relsem{\Gamma; \overline\alpha \vdash F}
  \Eq_{\rho}\overline{[\alpha := R]}$ to $\relsem{\Gamma ;
    \overline\alpha \vdash G} \Eq_{\rho}\overline{[\alpha := R]}$ in
  $\rel$ for all ${R_1 : \rel(A_1,B_1),}$ $...,{R_k : \rel(A_k,B_k)}$,
  then $t = t'$.  By hypothesis, $(t_{\overline{A}},
  t'_{\overline{A}})$ is a morphism from $\relsem{\Gamma;
    \overline\alpha \vdash F} \Eq_{\rho}\overline{[\alpha :=
      \Eq_{A}]}$ to $\relsem{\Gamma ; \overline\alpha \vdash G}
  \Eq_{\rho}\overline{[\alpha := \Eq_{A}]}$ in $\rel$ for all
  $A_1\,...\,A_k : \set$. By the induction hypothesis, it is therefore
  a morphism from $\Eq_{\setsem{\Gamma; \overline\alpha \vdash F}
    \rho\overline{[\alpha := A]}}$ to $\Eq_{\setsem{\Gamma ;
      \overline\alpha \vdash G} \rho\overline{[\alpha := A]}}$ in
  $\rel$. This means that, for every $x : \Eq_{\setsem{\Gamma;
      \overline\alpha \vdash F} \rho\overline{[\alpha := A]}}$,
  $t_{\overline{A}}x = t'_{\overline{A}}x$.  Then, by extensionality,
  $t = t'$.
\item $\relsem{\Gamma; \Phi \vdash \zerot} \Eq_{\rho} = 0_\rel =
  \Eq_{0_\set} = \Eq_{\setsem{\Gamma; \Phi \vdash \zerot}\rho}$
\item $\relsem{\Gamma; \Phi \vdash \onet} \Eq_{\rho} = 1_\rel =
  \Eq_{1_\set} = \Eq_{\setsem{\Gamma; \Phi \vdash \onet}\rho}$
\item The application case is proved by the following sequence of
  equalities, where the second equality is by the induction hypothesis
  and the definition of the relation environment $\Eq_\rho$, the third
  is by the definition of application of relation transformers, and
  the fourth is by Lemma~\ref{lem:eq-reln-equalities}:
\[
\begin{split}
\relsem{\Gamma; \Phi \vdash \phi\ol{\tau}}\Eq_{\rho} &=
(\Eq_{\rho}\phi)\ol{\relsem{\Gamma; \Phi \vdash \tau}
\Eq_{\rho}}\\
&= \Eq_{\rho \phi}\, \ol{\Eq_{\setsem{\Gamma; \Phi \vdash \tau}
  \rho}}\\
&= (\Eq_{\rho \phi})^* \,\ol{\Eq_{\setsem{\Gamma; \Phi \vdash \tau}
  \rho}}\\
&= \Eq_{(\rho \phi) \,\ol{\setsem{\Gamma; \Phi \vdash \tau} \rho}}\\
&= \Eq_{\setsem{\Gamma; \Phi \vdash \phi\ol{\tau}}\rho}
\end{split}
\]
\item The fixpoint case is proven by the sequence of equalities
\[
\begin{split}
\relsem{\Gamma; \Phi \vdash (\mu \phi.\lambda
  \ol{\alpha}. H)\ol{\tau}}\Eq_{\rho} 
&=(\mu {T_{\Eq_{\rho}}}) \,\ol{\relsem{\Gamma; \Phi \vdash \tau}\Eq_{\rho}}\\ 
&= \colim{n \in \nat}{T^n_{\Eq_{\rho}} K_0}\, \ol{\relsem{\Gamma; \Phi
  \vdash \tau}\Eq_{\rho}}\\
&= \colim{n \in \nat}{ T^n_{\Eq_{\rho}} K_0 \,\ol{\Eq_{\setsem{\Gamma;
    \Phi \vdash \tau}\rho}}}\\
&= \colim{n \in \nat}{(\Eq_{(T^\set_\rho)^n K_0})^*
  \ol{\Eq_{\setsem{\Gamma; \Phi \vdash \tau}\rho}}}\\
&= \colim{n \in \nat}{\Eq_{(T^\set_\rho)^n K_0 \,\ol{\setsem{\Gamma;
        \Phi \vdash \tau}\rho}}}\\ 
&= \Eq_{\colim{n \in \nat}{ (T^\set_{\rho})^n K_0\,
    \ol{\setsem{\Gamma; \Phi \vdash \tau}\rho}}}\\
&= \Eq_{\setsem{\Gamma; \Phi \vdash (\mu \phi.\lambda
      \ol{\alpha}. H)\ol{\tau}}\rho}
\end{split}
\]
Here, the third equality is by induction hypothesis, the fifth is by
Lemma~\ref{lem:eq-reln-equalities} and the fourth equality is because,
for every $n \in \nat$, the following two statements can be proved by
simultaneous induction:
\begin{equation}\label{eq:iel-fix-point-intermediate1}
T^n_{\Eq_{\rho}} K_0\, \ol{\Eq_{\setsem{\Gamma; \Phi \vdash
      \tau}\rho}} = (\Eq_{(T^\set_\rho)^n K_0})^*
\ol{\Eq_{\setsem{\Gamma; \Phi \vdash \tau}\rho}}
\end{equation}
and
\begin{equation}\label{eq:iel-fix-point-intermediate2}
\begin{split}
  \relsem{\Gamma; \Phi, \phi, \ol{\alpha} \vdash H}
\Eq_{\rho} [\phi := 
 & T^{n}_{\Eq_{\rho}} K_0] \overline{[\alpha :=
    \Eq_{\setsem{\Gamma; \Phi \vdash \tau}\rho}]} \\
=\;\; & \relsem{\Gamma; \Phi, \phi, \ol{\alpha} \vdash H} \Eq_{\rho} [\phi
  := \Eq_{(T^\set_\rho)^n K_0}] \overline{[\alpha :=
    \Eq_{\setsem{\Gamma; \Phi \vdash \tau}\rho}]}
\end{split}
\end{equation}
We prove~\eqref{eq:iel-fix-point-intermediate1}.  The case $n=0$ is
trivial, because $T^0_{\Eq_{\rho}} K_0 = K_0$ and
$(T^\set_\rho)^0 K_0 = K_0$; the inductive step is
proved by the following sequence of equalities:
\[
\begin{split}
T^{n+1}_{\Eq_{\rho}} K_0\, \overline{\Eq_{\setsem{\Gamma; \Phi \vdash \tau}\rho}}
&= T^\rel_{\Eq_{\rho}} (T^{n}_{\Eq_{\rho}} K_0)
\overline{\Eq_{\setsem{\Gamma; \Phi \vdash \tau}\rho}} \\ 
&= \relsem{\Gamma; \Phi, \phi, \ol{\alpha} \vdash H} \Eq_{\rho} [\phi
  := T^{n}_{\Eq_{\rho}} K_0] \overline{[\alpha :=
    \Eq_{\setsem{\Gamma; \Phi \vdash \tau}\rho}]} \\ 
&= \relsem{\Gamma; \Phi, \phi, \ol{\alpha} \vdash H} \Eq_{\rho} [\phi
  := \Eq_{(T^\set_\rho)^n K_0}] \overline{[\alpha :=
    \Eq_{\setsem{\Gamma; \Phi \vdash \tau}\rho}]} \\ 
&= \relsem{\Gamma; \Phi, \phi, \ol{\alpha} \vdash H} \Eq_{\rho [\phi
    := (T^\set_\rho)^n K_0] \overline{[\alpha :=
      \setsem{\Gamma; \Phi \vdash \tau}\rho]}} \\ 
&= \Eq_{\setsem{\Gamma; \Phi, \phi, \ol{\alpha} \vdash H} \rho [\phi
    := (T^\set_\rho)^n K_0] \overline{[\alpha :=
      \setsem{\Gamma; \Phi \vdash \tau}\rho]}} \\ 
&= \Eq_{(T^\set_\rho)^{n+1} K_0 \overline{\setsem{\Gamma; \Phi
      \vdash \tau}\rho}} \\ 
&= (\Eq_{(T^\set_\rho)^{n+1} K_0})^*\, \overline{\Eq_{\setsem{\Gamma;
      \Phi \vdash \tau}\rho}} 
\end{split}
\]
Here, the third equality is by~\eqref{eq:iel-fix-point-intermediate2},
the fifth by the induction hypothesis on $H$, and the last by
Lemma~\ref{lem:eq-reln-equalities}.  We prove the induction step
of~\eqref{eq:iel-fix-point-intermediate2} by structural induction on
$H$: the only interesting case, though, is when $\phi$ is applied,
i.e., for $H = \phi \ol{\sigma}$, which is proved by the sequence of
equalities:
\[
\begin{split}
& \relsem{\Gamma; \Phi, \phi, \ol{\alpha} \vdash \phi
    \ol{\sigma}} \Eq_{\rho} [\phi := T^{n}_{\Eq_{\rho}} K_0]
  \overline{[\alpha := \Eq_{\setsem{\Gamma; \Phi \vdash \tau}\rho}]}
  \\
&= T^{n}_{\Eq_{\rho}} K_0\, \overline{\relsem{\Gamma; \Phi,
      \phi, \ol{\alpha} \vdash \sigma} \Eq_{\rho} [\phi :=
      T^{n}_{\Eq_{\rho}} K_0] \overline{[\alpha :=
        \Eq_{\setsem{\Gamma; \Phi \vdash \tau} \rho}]}} \\ 
&= T^{n}_{\Eq_{\rho}} K_0\, \overline{\relsem{\Gamma; \Phi,
      \phi, \ol{\alpha} \vdash \sigma} \Eq_{\rho} [\phi :=
      \Eq_{(T^\set_\rho)^{n} K_0}] \overline{[\alpha :=
        \Eq_{\setsem{\Gamma; \Phi \vdash \tau} \rho}]}} \\ 
&= T^{n}_{\Eq_{\rho}} K_0\, \overline{\relsem{\Gamma; \Phi,
      \phi, \ol{\alpha} \vdash \sigma} \Eq_{\rho [\phi := (T^\set_\rho)^{n}
        K_0] \overline{[\alpha := \setsem{\Gamma; \Phi \vdash
            \tau} \rho]}}} \\ 
&= T^{n}_{\Eq_{\rho}} K_0\, \overline{\Eq_{\setsem{\Gamma;
        \Phi, \phi, \ol{\alpha} \vdash \sigma} \rho [\phi :=
        (T^\set_\rho)^{n} K_0] \overline{[\alpha :=
          \setsem{\Gamma; \Phi \vdash \tau} \rho]}}} \\ 
&= (\Eq_{(T^\set_\rho)^{n} K_0})^* \,\overline{\Eq_{\setsem{\Gamma;
        \Phi, \phi, \ol{\alpha} \vdash \sigma} \rho [\phi :=
        (T^\set_\rho)^{n} K_0] \overline{[\alpha :=
          \setsem{\Gamma; \Phi \vdash \tau} \rho]}}} \\ 
&= (\Eq_{(T^\set_{\rho})^{n} K_0})^* \overline{\relsem{\Gamma;
      \Phi, \phi, \ol{\alpha} \vdash \sigma} \Eq_{\rho} [\phi :=
      \Eq_{(T^\set_{\rho})^{n} K_0}] \overline{[\alpha :=
        \Eq_{\setsem{\Gamma; \Phi \vdash \tau}\rho}]}} \\ 
&= \relsem{\Gamma; \Phi, \phi, \ol{\alpha} \vdash \phi \ol{\sigma}}
  \Eq_{\rho} [\phi := \Eq_{(T^\set_{\rho})^{n} K_0}]
  \overline{[\alpha := \Eq_{\setsem{\Gamma; \Phi \vdash \tau}\rho}]} 
\end{split}
\]
Here, the second equality is by the induction hypothesis
for~\eqref{eq:iel-fix-point-intermediate2} on the $\sigma$s, the
fourth is by the induction hypothesis for Theorem~\ref{thm:iel} on the
$\sigma$s, and the fifth is by the induction hypothesis on $n$
for~\eqref{eq:iel-fix-point-intermediate1}.
\item $\relsem{\Gamma; \Phi \vdash \sigma + \tau} \Eq_{\rho} =
  \relsem{\Gamma; \Phi \vdash \sigma} \Eq_{\rho} + \relsem{\Gamma;
    \Phi \vdash \tau} \Eq_{\rho} = \Eq_{\setsem{\Gamma; \Phi \vdash
      \sigma}\rho} + \Eq_{\setsem{\Gamma; \Phi \vdash \tau}\rho} =
  \Eq_{\setsem{\Gamma; \Phi \vdash \sigma}\rho + \setsem{\Gamma; \Phi
      \vdash \tau}\rho} = \Eq_{\setsem{\Gamma; \Phi \vdash \sigma +
      \tau}\rho}$
\item $\relsem{\Gamma; \Phi \vdash \sigma \times \tau} \Eq_{\rho} =
  \relsem{\Gamma; \Phi \vdash \sigma}\Eq_{\rho} \times \relsem{\Gamma;
    \Phi \vdash \tau}\Eq_{\rho} = \Eq_{\setsem{\Gamma; \Phi \vdash
      \sigma}\rho} \times \Eq_{\setsem{\Gamma; \Phi \vdash \tau}\rho}
  = \Eq_{\setsem{\Gamma; \Phi \vdash \sigma}\rho \times
    \setsem{\Gamma; \Phi \vdash \tau}\rho} = \Eq_{\setsem{\Gamma;
      \Phi \vdash \sigma \times \tau}\rho}$
\end{itemize}
\end{proof}
\end{comment}

%It follows from Theorem~\ref{thm:iel} that $\relsem{\Gamma \vdash
%  \sigma \to \tau} \Eq_{\rho} = \Eq_{\setsem{\Gamma \vdash \sigma \to
%    \tau}\rho}$, as expected.  Moreover,

%\vspace*{0.05in}
%
%The Graph Lemma appropriate to our setting is now easily obtained:
\begin{lemma}[Graph Lemma]\label{lem:graph}
If $\rho, \rho' : \setenv$, $f : \rho \to \rho'$, and $\Gamma; \Phi
\vdash F$, then $\graph{\setsem{\Gamma; \Phi \vdash F} f} =
\relsem{\Gamma; \Phi \vdash F}\graph{f}$.
\end{lemma}
\begin{proof}
Applying Lemma~\ref{lem:rel-transf-morph} to the morphisms $(f,
\id_{\rho'}) : \graph{f} \to \Eq_{\rho'}$ and $(\id_{\rho}, f) :
\Eq_{\rho} \to \graph{f}$ of relation environments gives that
$(\setsem{\Gamma; \Phi \vdash F}f, \setsem{\Gamma; \Phi \vdash
  F}\id_{\rho'}) = \relsem{\Gamma; \Phi \vdash F} (f, \id_{\rho'}) :
\relsem{\Gamma; \Phi \vdash F}\graph{f} \to \relsem{\Gamma; \Phi
  \vdash F}\Eq_{\rho'}$ and $(\setsem{\Gamma; \Phi \vdash
  F}\id_{\rho}, \setsem{\Gamma; \Phi \vdash F}f) = \relsem{\Gamma;
  \Phi \vdash F} (\id_{\rho}, f) : \relsem{\Gamma; \Phi \vdash
  F}\Eq_{\rho} \to \relsem{\Gamma; \Phi \vdash F}\graph{f}$.
Expanding the first equation gives that if $(x,y) \in \relsem{\Gamma;
  \Phi \vdash F}\graph{f}$ then $(\setsem{\Gamma; \Phi \vdash F} f\,
x, \setsem{\Gamma; \Phi \vdash F}\id_{\rho'}\, y) \in \relsem{\Gamma;
  \Phi \vdash F}\Eq_{\rho'}$.  Then $\setsem{\Gamma; \Phi \vdash
  F}\id_{\rho'}\, y = \id_{\setsem{\Gamma; \Phi \vdash F}\rho'}\, y =
y$ and $\relsem{\Gamma; \Phi \vdash F}\Eq_{\rho'} =
\Eq_{\setsem{\Gamma; \Phi \vdash F}\rho'}$, so if $(x,y) \in
\relsem{\Gamma; \Phi \vdash F}\graph{f}$ then $(\setsem{\Gamma; \Phi
  \vdash F} f\, x, y) \in \Eq_{\setsem{\Gamma; \Phi \vdash F}\rho'}$,
i.e., $\setsem{\Gamma; \Phi \vdash F} f\, x = y$, i.e., $(x, y) \in
\graph{\setsem{\Gamma; \Phi \vdash F} f}$.  So, we have that
$\relsem{\Gamma; \Phi \vdash F}\graph{f} \subseteq
\graph{\setsem{\Gamma; \Phi \vdash F}f}$.  Expanding the second
equation gives that if $x \in \setsem{\Gamma; \Phi \vdash F}\rho$ then
$(\setsem{\Gamma; \Phi \vdash F}\id_{\rho}\, x, \setsem{\Gamma; \Phi
  \vdash F} f\, x) \in \relsem{\Gamma; \Phi \vdash F}\graph{f}$.  Then
$\setsem{\Gamma; \Phi \vdash F}\id_{\rho}\, x = \id_{\setsem{\Gamma;
    \Phi \vdash F}\rho} x = x$, so for any $x \in \setsem{\Gamma; \Phi
  \vdash F}\rho$ we have that $(x, \setsem{\Gamma; \Phi \vdash F}f\,
x) \in \relsem{\Gamma; \Phi \vdash F}\graph{f}$.  Moreover, $x \in
\setsem{\Gamma; \Phi \vdash F}\rho$ if and only if $(x,
\setsem{\Gamma; \Phi \vdash F} f\, x) \in \graph{\setsem{\Gamma; \Phi
    \vdash F}f}$ and, if $x \in \setsem{\Gamma; \Phi \vdash F}\rho$
then $(x, \setsem{\Gamma; \Phi \vdash F} f\, x) \in \relsem{\Gamma;
  \Phi \vdash F} \graph{f}$, so if $(x, \setsem{\Gamma; \Phi \vdash F}
f\, x) \in \graph{\setsem{\Gamma; \Phi \vdash F}f}$ then $(x,
\setsem{\Gamma; \Phi \vdash F} f\, x) \in \relsem{\Gamma; \Phi \vdash
  F} \graph{f}$, i.e., $\graph{\setsem{\Gamma; \Phi \vdash F}f}
\subseteq \relsem{\Gamma; \Phi \vdash F} \graph{f}$.


\end{proof}



\begin{comment}

\begin{proof}
First observe that $(f, \id_{\rho'}) : \graph{f} \to \Eq_{\rho'}$ and
$(\id_{\rho}, f) : \Eq_{\rho} \to \graph{f}$ are morphisms of relation
environments.  Applying Lemma~\ref{lem:rel-transf-morph} to each of
these observations gives that
\begin{equation}\label{eq:graph-one}
(\setsem{\Gamma; \Phi \vdash F}f, \setsem{\Gamma; \Phi \vdash
    F}\id_{\rho'}) = \relsem{\Gamma; \Phi \vdash F} (f, \id_{\rho'}) :
  \relsem{\Gamma; \Phi \vdash F}\graph{f} \to \relsem{\Gamma; \Phi
    \vdash F}\Eq_{\rho'}
\end{equation}
and
\begin{equation}\label{eq:graph-two}
(\setsem{\Gamma; \Phi \vdash F}\id_{\rho}, \setsem{\Gamma; \Phi \vdash F}f)
= \relsem{\Gamma; \Phi \vdash F} (\id_{\rho}, f)
: \relsem{\Gamma; \Phi \vdash F}\Eq_{\rho} \to \relsem{\Gamma; \Phi \vdash F}\graph{f}
\end{equation}
Expanding Equation~\ref{eq:graph-one} gives that if
$(x,y) \in \relsem{\Gamma; \Phi \vdash F}\graph{f}$
then
\[(\setsem{\Gamma; \Phi \vdash F} f\, x, \setsem{\Gamma; \Phi \vdash
  F}\id_{\rho'}\, y) \in \relsem{\Gamma; \Phi \vdash F}\Eq_{\rho'}\]
Observe that $\setsem{\Gamma; \Phi \vdash F}\id_{\rho'}\, y =
\id_{\setsem{\Gamma; \Phi \vdash F}\rho'}\, y = y$ and
$\relsem{\Gamma; \Phi \vdash F}\Eq_{\rho'} = \Eq_{\setsem{\Gamma; \Phi
    \vdash F}\rho'}$. So, if $(x,y) \in \relsem{\Gamma; \Phi \vdash
  F}\graph{f}$ then $(\setsem{\Gamma; \Phi \vdash F} f\, x, y) \in
\Eq_{\setsem{\Gamma; \Phi \vdash F}\rho'}$, i.e., $\setsem{\Gamma;
  \Phi \vdash F} f\, x = y$, i.e., $(x, y) \in \graph{\setsem{\Gamma;
    \Phi \vdash F} f}$.  So, we have that $\relsem{\Gamma; \Phi \vdash
  F}\graph{f} \subseteq \graph{\setsem{\Gamma; \Phi \vdash F}f}$

Expanding Equation~\ref{eq:graph-two} gives that, for any
$x \in \setsem{\Gamma; \Phi \vdash F}\rho$,
then
\[
(\setsem{\Gamma; \Phi \vdash F}\id_{\rho}\, x, \setsem{\Gamma; \Phi
  \vdash F} f\, x) \in \relsem{\Gamma; \Phi \vdash F}\graph{f} 
\]
Observe that $\setsem{\Gamma; \Phi \vdash F}\id_{\rho}\, x =
\id_{\setsem{\Gamma; \Phi \vdash F}\rho} x = x$ so, for any $x \in
\setsem{\Gamma; \Phi \vdash F}\rho$, we have that $(x, \setsem{\Gamma;
  \Phi \vdash F}f\, x) \in \relsem{\Gamma; \Phi \vdash F}\graph{f}$.
Moreover, $x \in \setsem{\Gamma; \Phi \vdash F}\rho$ if and only if
$(x, \setsem{\Gamma; \Phi \vdash F} f\, x) \in \graph{\setsem{\Gamma;
    \Phi \vdash F}f}$ and, if $x \in \setsem{\Gamma; \Phi \vdash
  F}\rho$ then $(x, \setsem{\Gamma; \Phi \vdash F} f\, x) \in
\relsem{\Gamma; \Phi \vdash F} \graph{f}$, so if $(x, \setsem{\Gamma;
  \Phi \vdash F} f\, x) \in \graph{\setsem{\Gamma; \Phi \vdash F}f}$
then $(x, \setsem{\Gamma; \Phi \vdash F} f\, x) \in \relsem{\Gamma;
  \Phi \vdash F} \graph{f}$, i.e., $\graph{\setsem{\Gamma; \Phi \vdash
    F}f} \subseteq \relsem{\Gamma; \Phi \vdash F} \graph{f}$. We
conclude that $\relsem{\Gamma; \Phi \vdash F}\graph{f} =
\graph{\setsem{\Gamma; \Phi \vdash F}f}$ as desired.
\end{proof}
\end{comment}

\vspace*{-.2in}

\section{Interpreting Terms}\label{sec:term-interp}

%{\color{blue} Double use of brackets confusing? (Graphs and pairing
%  morphisms)} 
%  Here, we are using angle bracket notation for both the
%  graph relation of a function and for the pairing of functions with
%  the same domain. This is justified by the relationship between the
%  two notions observed immediately after
%  Lemma~\ref{lem:graph-reln-functors}.}

If $\Delta = x_1 : \tau_1,...,x_n : \tau_n$ is a term context for $\Gamma$
and $\Phi$, then the interpretations $\setsem{\Gamma;\Phi \vdash \Delta}$ and
$\relsem{\Gamma;\Phi \vdash \Delta}$ are defined by
\[\begin{array}{lll}
\setsem{\Gamma;\Phi \vdash \Delta} & = & \setsem{\Gamma;\Phi \vdash
  \tau_1} \times ... \times \setsem{\Gamma;\Phi \vdash \tau_n}\\ 
\relsem{\Gamma;\Phi \vdash \Delta} & = & \relsem{\Gamma;\Phi \vdash
  \tau_1} \times ... \times \relsem{\Gamma;\Phi \vdash \tau_n}\\ 
\end{array}\]
Every well-formed term $\Gamma;\Phi~|~\Delta \vdash t : \tau$ then
has, for every $\rho \in \setenv$, set interpretations
$\setsem{\Gamma;\Phi~|~\Delta \vdash t : \tau}\rho$ as natural
transformations from $\setsem{\Gamma; \Phi \vdash \Delta}\rho$ to
$\setsem{\Gamma; \Phi \vdash \tau}\rho$, and, for every $\rho \in
\relenv$, relational interpretations $\relsem{\Gamma;\Phi~|~\Delta
  \vdash t : \tau}\rho$ as natural transformations from
$\relsem{\Gamma; \Phi \vdash \Delta}\rho$ to $\relsem{\Gamma; \Phi
  \vdash \tau}\rho$. These are given in the next (two) definitions.

\begin{dfn}\label{def:set-interp}
If $\rho$ is a set (resp., relation) environment and
$\Gamma;\Phi~|~\Delta \vdash t : \tau$ then
$\setsem{\Gamma;\Phi~|~\Delta \vdash t : \tau}\rho$ (resp.,
$\relsem{\Gamma;\Phi~|~\Delta \vdash t : \tau}\rho$) is defined as in
Figure~\ref{fig:term-sem}, where $\mathsf D$ is either $\set$ or
$\rel$ as appropriate.
\end{dfn}

\subsection{Basic Properties of Term Interpretations}

The interpretations in Definition~\ref{def:set-interp} respect
weakening, i.e., a term and its weakenings all have the same set and
relational interpretations. Specifically, for any $\rho \in \setenv$,
$\setsem{\Gamma;\Phi \,|\, \Delta, x : \sigma \vdash t : \tau}\rho =
(\setsem{\Gamma;\Phi \,|\, \Delta \vdash t : \tau}\rho) \circ
\pi_{\Delta}$, where $\pi_{\Delta}$ is the projection
$\setsem{\Gamma;\Phi \vdash \Delta, x : \sigma} \to
\setsem{\Gamma;\Phi \vdash \Delta}$. A similar result holds for
relational interpretations.

\begin{figure*}
\hspace*{-3.1in}
\resizebox{0.46\linewidth}{!}{
\begin{minipage}[t]{0.5\textwidth}
\[\begin{array}{lll}
\dsem{\Gamma;\emptyset \,|\, \Delta,x :\tau \vdash x : \tau} \rho& = &
\pi_{|\Delta|+1}\\
%\dsem{\Gamma;\emptyset \,|\, \Delta \vdash \lambda x.t : \sigma \to \tau}\rho &
%= & \curry (\dsem{\Gamma;\emptyset \,|\, \Delta, x : \sigma \vdash t :
%  \tau}\rho)\\ 
%\dsem{\Gamma;\emptyset \,|\, \Delta \vdash st: \tau} \rho & = &
%\eval \circ
% \langle \dsem{\Gamma;\emptyset \,|\, \Delta \vdash s: \sigma \to
%  \tau}\rho, \dsem{\Gamma;\emptyset \,|\, \Delta \vdash t: \sigma}\rho
%\rangle\\
\dsem{\Gamma;\emptyset \,|\, \Delta \vdash L_{\overline \alpha} x.t : \Nat^{\overline
    \alpha} \,F \,G}\rho & = &  \curry (\dsem{\Gamma;\overline \alpha
  \,|\, \Delta, x : F \vdash t: G}\rho[\overline{\alpha := \_}])\\
\dsem{\Gamma;\Phi \,|\, \Delta \vdash t_{\overline \tau} s:
  G [\overline{\alpha := \tau}]}\rho & = & \eval \circ \langle
  \lambda d.\,(\dsem{\Gamma;\emptyset \,|\, \Delta \vdash t :
  \Nat^{\overline{\alpha}} \,F \,G}\rho\; d)_{\overline{\dsem{\Gamma;\Phi
      \vdash \tau}\rho}},\\ 
 & & \hspace*{0.5in} \dsem{\Gamma;\Phi \,|\,
    \Delta \vdash s: F [\overline{\alpha := \tau}]}\rho \rangle\\ 
& & \\
%% \color{red} \mbox{Add rules for } \forall \mbox{ if we include it} & & \\
\dsem{\Gamma;\Phi \,|\, \Delta,x :\tau \vdash x : \tau} \rho& = &
\pi_{|\Delta|+1}\\
\dsem{\Gamma;\Phi \,|\, \Delta \vdash \bot_\tau t : \tau} \rho& = &
!^0_{\dsem{\Gamma;\Phi \vdash \tau}\rho} \circ
  \dsem{\Gamma;\Phi~|~\Delta \vdash t : \zerot}\rho, \mbox{ where } \\
 & & \hspace*{0.1in} !^0_{\dsem{\Gamma;\Phi \vdash \tau}\rho}
\mbox{ is the unique morphism from } 0\\
 & & \hspace*{0.1in} \mbox{ to } \dsem{\Gamma;\Phi \vdash \tau}\rho\\
\dsem{\Gamma;\Phi \,|\, \Delta \vdash \top : \onet}\rho & = &
!^{\dsem{\Gamma;\Phi\vdash \Delta}\rho}_1, \mbox{ where }
!^{\dsem{\Gamma;\Phi\vdash \Delta}\rho}_1\\ 
& & \hspace*{0.1in} \mbox{ is the unique morphism from }
\dsem{\Gamma;\Phi\vdash \Delta}\rho \mbox{ to } 1\\ 
\dsem{\Gamma;\Phi \,|\, \Delta \vdash (s,t) : \sigma \times \tau} \rho& = &
\dsem{\Gamma;\Phi \,|\, \Delta \vdash s: \sigma} \rho\times
\dsem{\Gamma;\Phi \,|\, \Delta \vdash t : \tau} \rho\\
\dsem{\Gamma;\Phi \,|\, \Delta \vdash \pi_1 t : \sigma} \rho& = &
\pi_1 \circ \dsem{\Gamma;\Phi \,|\, \Delta \vdash t : \sigma \times \tau}\rho\\
\dsem{\Gamma;\Phi \,|\, \Delta \vdash \pi_2 t : \sigma}\rho & = &
\pi_2 \circ \dsem{\Gamma;\Phi \,|\, \Delta \vdash t : \sigma \times
  \tau} \rho\\
\dsem{\Gamma;\Phi~|~\Delta \vdash \case{t}{x \mapsto l}{y \mapsto r} :
  \gamma}\rho & = & \eval \circ \langle \curry \,[\dsem{\Gamma;\Phi
    \,|\, \Delta, x : \sigma \vdash l : \gamma}\rho,\\
   & & \hspace*{0.79in} \dsem{\Gamma;\Phi \,|\, \Delta, y
    : \tau \vdash r : \gamma}\rho],\\
   & &  \hspace*{0.5in} \dsem{\Gamma;\Phi \,|\, \Delta \vdash t :
  \sigma + \tau} \rho\rangle\\   
\dsem{\Gamma;\Phi \,|\, \Delta \vdash \inl \,s: \sigma + \tau} \rho& = &
\inl \circ \dsem{\Gamma;\Phi \,|\, \Delta \vdash s: \sigma}\rho\\
\dsem{\Gamma;\Phi \,|\, \Delta \vdash \inr \,t: \sigma + \tau}\rho & = & 
\inr \circ \dsem{\Gamma;\Phi \,|\, \Delta \vdash t : \tau}\rho\\
\llbracket \Gamma;\emptyset \,|\, \emptyset \vdash {\color{blue} \map^{\ol{F},\ol{G}}_H :
\Nat^\emptyset\;(\ol{\Nat^{\ol{\beta},\ol{\gamma}}\,F\,G}})\;&
= & \lambda d\, \ol{\eta}\,\ol{B}.\,
\dsem{\Gamma; \ol{\phi},\ol{\gamma}\vdash H}\id_{\rho[\ol{\gamma :=
      B}]}[\ol{\phi := \lambda \ol{A}.\eta_{\ol{A}\,\ol{B}}}]\\
\hspace*{0.79in} (\Nat^{\ol{\gamma}}\,H[\ol{\phi :=_{\ol{\beta}} F}]\,H[\ol{\phi
      :=_{\ol{\beta}} G}]) \rrbracket^\set \rho & & \\
\llbracket \Gamma;\emptyset \,|\, \emptyset \vdash \tin_H :
Nat^{\ol{\beta},\ol{\gamma}} \, H[\phi := (\mu \phi.\lambda {\overline
    \alpha}.H){\overline \beta}][\ol{\alpha := \beta}] & = &
\lambda d\,\ol{B}\, \ol{C}.\,(\mathit{in}_{{T}^\set_{H,\rho[\ol{\gamma := C}]}})_{\ol{B}}\\
\hspace*{0.79in}(\mu \phi.\lambda {\overline \alpha}.H){\overline
  \beta} \rrbracket^\set \rho & & \\
\llbracket \Gamma;\emptyset \,|\, \emptyset \vdash
  \fold^F_H : \Nat^\emptyset\;(\Nat^{\ol{\beta}, \ol{\gamma}}\,H[\phi
    :=_{\ol{\beta}} F][\ol{\alpha := \beta}]\,F) & = &  
\lambda d\,\eta\,\ol{B}\,\ol{C}.\,
(\mathit{fold}_{T^\set_{H,\rho[\ol{\gamma := C}]}} \, (\lambda
\ol{A}.\,\eta_{\ol{A}\,\ol{C}}))_{\ol{B}}\\ 
\hspace*{0.79in}(\Nat^{{\ol{\beta},\ol{\gamma}} }\,(\mu
  \phi.\lambda \overline \alpha.H)\overline \beta\;F)
\rrbracket^\set \rho & & \vspace*{-0.1in}
\end{array}\]
\caption{Term semantics}\label{fig:term-sem} 
\end{minipage}}\vspace*{-0.05in}
\end{figure*}

\begin{comment}
\begin{dfn}\label{def:rel-interp}
If $\rho$ is a relation environment and $\Gamma;\Phi~|~\Delta \vdash t :
\tau$ then $\relsem{\Gamma;\Phi~|~\Delta \vdash t : \tau}\rho$ is
defined as follows:
\[\begin{array}{lll}
\relsem{\Gamma;\emptyset \,|\, \Delta,x :\tau \vdash x : \tau} \rho& = &
\pi_{|\Delta|+1}\\
%\relsem{\Gamma;\emptyset \,|\, \Delta \vdash \lambda x.t : \sigma \to \tau}\rho &
%= & \curry (\relsem{\Gamma;\emptyset \,|\, \Delta, x : \sigma \vdash t :
%  \tau}\rho)\\ 
%\relsem{\Gamma;\emptyset \,|\, \Delta \vdash st: \tau} \rho & = & \eval
%\circ \langle \relsem{\Gamma;\emptyset \,|\, \Delta \vdash s: \sigma \to
%  \tau}\rho, \relsem{\Gamma;\emptyset \,|\, \Delta \vdash t: \sigma}\rho
%\rangle\\
\relsem{\Gamma;\emptyset \,|\, \Delta \vdash L_{\overline \alpha} x.t : \Nat^{\overline
    \alpha} \,F \,G}\rho & = &  \curry (\relsem{\Gamma;\overline \alpha
  \,|\, \Delta, x : F \vdash t: G}\rho[\overline{\alpha := \_}])\\
\relsem{\Gamma;\Phi \,|\, \Delta \vdash t_{\overline{\tau}} s:
  G [\overline{\alpha := \tau}]}\rho & = & \eval \circ \langle
  \lambda e. \,(\relsem{\Gamma;\emptyset \,|\, \Delta \vdash t :
  \Nat^{\overline{\alpha}} \,F \,G}\rho\; e)_{\overline{\relsem{\Gamma;\Phi
      \vdash \tau}\rho}},\\ 
 & & \hspace*{0.5in} \relsem{\Gamma;\Phi \,|\,
    \Delta \vdash s: F [\overline{\alpha := \tau}]}\rho \rangle\\ 
& & \\
%% \color{red} \mbox{Add rules for } \forall \mbox{ if we include it} & & \\
\relsem{\Gamma;\Phi \,|\, \Delta,x :\tau \vdash x : \tau} \rho& = &
\pi_{|\Delta|+1}\\
\relsem{\Gamma;\Phi \,|\, \Delta \vdash \bot_\tau t : \tau} \rho& = &
!^0_{\relsem{\Gamma;\Phi \vdash \tau}\rho} \circ
  \relsem{\Gamma;\Phi~|~\Delta \vdash t : \zerot}\rho, \mbox{ where } \\
 & & \hspace*{0.1in} !^0_{\relsem{\Gamma;\Phi \vdash \tau}\rho}
\mbox{ is the unique morphism from } 0\\
 & & \hspace*{0.1in} \mbox{ to } \relsem{\Gamma;\Phi \vdash \tau}\rho\\
\relsem{\Gamma;\Phi \,|\, \Delta \vdash \top : \onet}\rho & = &
!^{\relsem{\Gamma;\Phi\vdash \Delta}\rho}_1, \mbox{ where }
!^{\relsem{\Gamma;\Phi\vdash \Delta}\rho}_1\\ 
& & \hspace*{0.1in} \mbox{ is the unique morphism from }
\relsem{\Gamma;\Phi\vdash \Delta}\rho \mbox{ to } 1\\ 
\relsem{\Gamma;\Phi \,|\, \Delta \vdash (s,t) : \sigma \times \tau} \rho& = &
\relsem{\Gamma;\Phi \,|\, \Delta \vdash s: \sigma} \rho\times
\relsem{\Gamma;\Phi \,|\, \Delta \vdash t: \tau} \rho\\
\relsem{\Gamma;\Phi \,|\, \Delta \vdash \pi_1 t : \sigma} \rho& = &
\pi_1 \circ \relsem{\Gamma;\Phi \,|\, \Delta \vdash t : \sigma \times \tau}\rho\\
\relsem{\Gamma;\Phi \,|\, \Delta \vdash \pi_2 t : \sigma}\rho & = &
\pi_2 \circ \relsem{\Gamma;\Phi \,|\, \Delta \vdash t : \sigma \times
  \tau} \rho\\
\relsem{\Gamma;\Phi~|~\Delta \vdash \case{t}{x \mapsto l}{y \mapsto r} :
  \gamma}\rho & = & \eval \circ \langle \curry \,[\relsem{\Gamma;\Phi
    \,|\, \Delta, x : \sigma \vdash l : \gamma}\rho,\\
   & & \hspace*{0.79in} \relsem{\Gamma;\Phi \,|\, \Delta, y
    : \tau \vdash r : \gamma}\rho],\\
   & &  \hspace*{0.5in} \relsem{\Gamma;\Phi \,|\, \Delta \vdash t :
  \sigma + \tau} \rho\rangle \\
\relsem{\Gamma;\Phi \,|\, \Delta \vdash \inl \,s: \sigma + \tau} \rho& = &
\inl \circ \relsem{\Gamma;\Phi \,|\, \Delta \vdash s: \sigma}\rho\\
\relsem{\Gamma;\Phi \,|\, \Delta \vdash \inr \,t: \sigma + \tau}\rho & = & 
\inr \circ \relsem{\Gamma;\Phi \,|\, \Delta \vdash t : \tau}\rho\\
\llbracket \Gamma;\emptyset \,|\, \emptyset \vdash \map^{\ol{F},\ol{G}}_H :
\Nat^\emptyset\;(\ol{\Nat^{\ol{\beta},\ol{\gamma}}\,F\,G})\;&
= & \lambda d\, \ol{\eta}\,\ol{R}.\,
\relsem{\Gamma; \ol{\phi},\ol{\gamma}\vdash H}\id_{\rho[\ol{\gamma :=
      R}]}[\ol{\phi := \lambda \ol{S}.\eta_{\ol{S}\,\ol{R}}}]\\
\hspace*{0.79in} (\Nat^{\ol{\gamma}}\,H[\ol{\phi :=_{\ol{\beta}} F}]\,H[\ol{\phi
      :=_{\ol{\beta}} G}]) \rrbracket^\rel \rho & & \\
\llbracket \Gamma;\emptyset \,|\, \emptyset \vdash \tin_H :
Nat^{\ol{\beta},\ol{\gamma}} \, H[\phi := (\mu \phi.\lambda {\overline
    \alpha}.H){\overline \beta}][\ol{\alpha := \beta}] & = &
\lambda d\,\ol{R}\, \ol{S}.\,(\mathit{in}_{T_{\rho[\ol{\gamma := S}]}})_{\ol{R}}\\
\hspace*{0.79in}(\mu \phi.\lambda {\overline \alpha}.H){\overline
  \beta} \rrbracket^\rel \rho & & \\
\llbracket \Gamma;\emptyset \,|\, \emptyset \vdash
  \fold^F_H : \Nat^\emptyset\;(\Nat^{\ol{\beta}, \ol{\gamma}}\,H[\phi
    :=_{\ol{\beta}} F][\ol{\alpha := \beta}]\,F) & = &  
\lambda d\,\eta\,\ol{R}\,\ol{S}.\, (\mathit{fold}_{T_{\rho[\ol{\gamma
        := S}]}} \, (\lambda
\ol{Z}.\,\eta_{\ol{Z}\,\ol{S}}))_{\ol{R}}\\
\hspace*{0.79in}(\Nat^{{\ol{\beta},\ol{\gamma}} }\,(\mu
  \phi.\lambda \overline \alpha.H)\overline \beta\;F)
\rrbracket^\rel \rho & & 
\end{array}\]
\end{dfn}
{\color{blue} Add return type for fold in last clause? Should be
  $\relsem{\Gamma;\ol\beta,\ol\gamma \vdash F}\rho[\ol{\gamma :=
  C}]$.}

This interpretation gives that $\relsem{\Gamma;\emptyset \,|\, \Delta
  \vdash \lambda x.t : \sigma \to \tau}\rho = \curry
(\relsem{\Gamma;\emptyset \,|\, \Delta, x : \sigma \vdash t :
  \tau}\rho)$ and $\relsem{\Gamma;\emptyset \,|\, \Delta \vdash st:
  \tau} \rho = \eval \circ \langle \relsem{\Gamma;\emptyset \,|\,
  \Delta \vdash s: \sigma \to \tau}\rho, \relsem{\Gamma;\emptyset
  \,|\, \Delta \vdash t: \sigma}\rho \rangle$, as expected.

If $t$ is closed, i.e., if $\emptyset; \emptyset~|~\emptyset \vdash t
: \tau$, then we write $\setsem{\vdash t : \tau}$ instead of
$\setsem{\emptyset; \emptyset~|~\emptyset \vdash t : \tau}$.
\end{comment}

\begin{comment}
  , and for any $\rho \in \relenv$, 
\[ \relsem{\Gamma;\Phi \,|\, \Delta, x : \sigma \vdash t : \tau}\rho
  = (\relsem{\Gamma;\Phi \,|\, \Delta \vdash t : \tau}\rho) \circ
  \pi_{\Delta}\] where $\pi_{\Delta}$ is the projection
  $\relsem{\Gamma;\Phi \vdash \Delta, x : \sigma} \to
  \relsem{\Gamma;\Phi \vdash \Delta}$.
\end{comment}

The return type for the semantic fold in the last clause is
  $\dsem{\Gamma;\ol\beta,\ol\gamma \vdash F}\rho[\ol{\gamma := C}]$.
  This interpretation gives $\dsem{\Gamma;\emptyset \,|\, \Delta
    \vdash \lambda x.t : \sigma \to \tau}\rho = \curry
  (\dsem{\Gamma;\emptyset \,|\, \Delta, x : \sigma \vdash t :
    \tau}\rho)$ and $\dsem{\Gamma;\emptyset \,|\, \Delta \vdash st:
    \tau} \rho = \eval \circ \langle \dsem{\Gamma;\emptyset \,|\,
    \Delta \vdash s: \sigma \to \tau}\rho, \dsem{\Gamma;\emptyset
    \,|\, \Delta \vdash t: \sigma}\rho \rangle$, so it specializes to
  the standard interpretations for System F terms.  If $t$ is closed,
  i.e., if $\emptyset; \emptyset~|~\emptyset \vdash t : \tau$, then we
  write $\dsem{\vdash t : \tau}$ instead of $\dsem{\emptyset;
    \emptyset~|~\emptyset \vdash t : \tau}$.
%, and similarly for $\relsem{\emptyset; \emptyset~|~\emptyset \vdash
%t : \tau}$.
Moreover, if
  $\Gamma,\alpha;\Phi \,|\, \Delta \vdash t : \tau$ and
  $\Gamma;\Phi,\alpha \,|\, \Delta \vdash t' : \tau$ and $\Gamma;\Phi
  \vdash \sigma$ then
\begin{itemize}
\item $\setsem{\Gamma;\Phi \,|\, \Delta[\alpha := \sigma] \vdash
  t[\alpha := \sigma] : \tau[\alpha := \sigma]}\rho =
  \setsem{\Gamma,\alpha;\Phi \,|\, \Delta \vdash t : \tau }\rho [
    \alpha := \setsem{\Gamma;\Phi\vdash\sigma}\rho ]$
%\item $\relsem{\Gamma;\Phi \,|\, \Delta[\alpha := \sigma] \vdash
%  t[\alpha := \sigma] : \tau[\alpha := \sigma]}\rho =
%  \relsem{\Gamma,\alpha;\Phi \,|\, \Delta \vdash t : \tau }\rho [
%    \alpha := \relsem{\Gamma;\Phi\vdash\sigma}\rho ]$
\item $\setsem{\Gamma;\Phi \,|\, \Delta[\alpha := \sigma] \vdash
  t'[\alpha := \sigma] : \tau[\alpha := \sigma]}\rho =
  \setsem{\Gamma;\Phi,\alpha \,|\, \Delta \vdash t' : \tau }\rho [
    \alpha := \setsem{\Gamma;\Phi\vdash\sigma}\rho ]$
%\item $\relsem{\Gamma;\Phi \,|\, \Delta[\alpha := \sigma] \vdash
%  t'[\alpha := \sigma] : \tau[\alpha := \sigma]}\rho =
%  \relsem{\Gamma;\Phi,\alpha \,|\, \Delta \vdash t' : \tau }\rho [
%    \alpha := \relsem{\Gamma;\Phi\vdash\sigma}\rho ]$
\end{itemize}
and if $\Gamma;\Phi \,|\, \Delta, x: \sigma \vdash t : \tau$ and
$\Gamma;\Phi \,|\, \Delta \vdash s : \sigma$ then
\begin{itemize}
\item $\lambda A.\, \setsem{\Gamma;\Phi \,|\, \Delta \vdash t[x := s]
  : \tau }\rho \, A = \lambda A.\,\setsem{\Gamma;\Phi \,|\, \Delta, x:
  \sigma \vdash t : \tau}\rho \,(A, \setsem{\Gamma;\Phi \,|\,
  \Delta\vdash s: \sigma}\rho\, A)$
%\item $\lambda R.\,\relsem{\Gamma;\Phi \,|\, \Delta \vdash t[x := s] :
%  \tau }\rho\,R = \lambda R.\,\relsem{\Gamma;\Phi \,|\, \Delta, x:
%  \sigma \vdash t : \tau}\rho \,(R, \relsem{\Gamma;\Phi \,|\,
%  \Delta\vdash s: \sigma}\rho\,R)$
\end{itemize}
Direct calculation reveals that the set interpretations of terms also
satisfy
\begin{itemize}
\item $\setsem{\Gamma; \Phi~|~\Delta \vdash
  (L_{\ol{\alpha}}x.t)_{\ol{\tau}}s} = \setsem{\Gamma; \Phi~|~\Delta
  \vdash t [\ol{\alpha := \tau}][x := s]}$
\end{itemize}
Term extensionality with respect to types and terms --- i.e.,
$\setsem{\Gamma;\Phi\vdash (L_\alpha x.t)_\alpha \top} =
\setsem{\Gamma;\Phi \vdash t}$ and $\setsem{\Gamma;\Phi\vdash
  (L_\alpha x.t)_\alpha x} = \setsem{\Gamma;\Phi \vdash t}$
%$\relsem{\Gamma;\Phi\vdash (L_\alpha x.t)_\alpha t} =
%\relsem{\Gamma;\Phi \vdash t}$, as well as
%term extensionality
%and $\relsem{\Gamma;\Phi\vdash
%  (L_\alpha x.t)_\alpha \top} = \relsem{\Gamma;\Phi \vdash t}$
--- follow. Similar properties hold for relational interpretations.

\subsection{Properties of Terms of $\Nat$-Type}\label{sec:Nat-type-terms}

Define, for $\Gamma; \ol{\alpha} \vdash F$, the term {\color{blue}
  $\id_F$} to be $\Gamma;\emptyset~|~\emptyset \vdash
L_{\ol{\alpha}}x.x : \Nat^{\ol{\alpha}} F\,F$ and, for terms $\Gamma;
\emptyset \,|\, \Delta \vdash t: \Nat^{\overline{\alpha}} F\,G$ and
$\Gamma; \emptyset \,|\, \Delta \vdash s: \Nat^{\overline{\alpha}}
G\,H$, the {\em composition} $s \circ t$ of $t$ and $s$ to be $\Gamma;
\emptyset \,|\, \Delta \vdash L_{\overline{\alpha}}
x. s_{\overline{\alpha}}(t_{\overline{\alpha}}x):
\Nat^{\overline{\alpha}} F\,H$. Then $\setsem{\Gamma; \emptyset \,|\,
  \emptyset \vdash \id_{F} : \Nat^{\ol{\alpha}} F\,F} \rho\, \ast =
\id_{\lambda \ol{A}. \setsem{\Gamma; \ol{\alpha} \vdash F} \rho
  [\ol{\alpha := A}]}$ for any set environment $\rho$, and
$\setsem{\Gamma; \emptyset \,|\, \Delta \vdash s \circ t:
  \Nat^{\overline{\alpha}} F\,H} = \setsem{\Gamma; \emptyset \,|\,
  \Delta \vdash s: \Nat^{\overline{\alpha}} G\,H} \circ
\setsem{\Gamma; \emptyset \,|\, \Delta \vdash t:
  \Nat^{\overline{\alpha}} F\,G}$.  Straightforward calculation using
these equalities shows that terms of $\Nat$ type behave as
natural transformations with respect to their source and target
functorial types, i.e., we have
\begin{thm}\label{eq:ft-from-nat}
\[\begin{array}{l}
\hspace*{0.135in}\setsem{\Gamma; \emptyset \,|\, x :
  \Nat^{\overline{\alpha}, 
    \overline{\gamma}} F\,G, \overline{y : \Nat^{\overline{\gamma}}
    \sigma\, \tau} \vdash ((\map_G^{\overline{\sigma},
    \overline{\tau}})_{\emptyset} \overline{y}) \circ
  (L_{\overline{\gamma}} z. x_{\overline{\sigma}, \overline{\gamma}}
  z) : \Nat^{\overline{\gamma}} F[\overline{\alpha := \sigma}]\,
  G[\overline{\alpha := \tau}]}\\
= \setsem{ \Gamma; \emptyset \,|\, x
  : \Nat^{\overline{\alpha}, \overline{\gamma}} F\, G, \overline{y :
    \Nat^{\overline{\gamma}} \sigma\, \tau} \vdash
  (L_{\overline{\gamma}} z. x_{\overline{\tau}, \overline{\gamma}} z)
  \circ ((\map_F^{\overline{\sigma}, \overline{\tau}})_{\emptyset}
  \overline{y}) : \Nat^{\overline{\gamma}} F[\overline{\alpha :=
      \sigma}]\, G[\overline{\alpha := \tau}]}
\end{array}\]
\end{thm}
When $x = in_H$ and $\xi = \Nat^{\overline{\gamma}} H[\phi := (\mu
  \phi. \lambda \overline{\alpha}. H)
  \overline{\beta}][\overline{\alpha := \sigma}]\, (\mu \phi. \lambda
\overline{\alpha}. H)\overline{\tau}$ Theorem~\ref{eq:ft-from-nat}
specializes to
\begin{equation}\label{thm:subst}
\begin{array}{l}
\hspace*{0.135in}\setsem{\Gamma; \emptyset \,|\, \overline{y :
    \Nat^{\overline{\gamma}} 
    \sigma\, \tau} \vdash ((\map_{(\mu \phi. \lambda
    \overline{\alpha}. H) \overline{\beta}}^{\overline{\sigma},
    \overline{\tau}})_{\emptyset} \overline{y}) \circ
  (L_{\overline{\gamma}} z. (\tin_{H})_{\overline{\sigma},
    \overline{\gamma}} z) : \xi}\\
= \setsem{\Gamma; \emptyset \,|\, \overline{y :
  \Nat^{\overline{\gamma}} \sigma\, \tau} 
  \vdash (L_{\overline{\gamma}} z. (\tin_H)_{\overline{\tau}, \overline{\gamma}} z)
  \circ ((\map_{H[\phi := (\mu \phi. \lambda \overline{\alpha}. H)
      \overline{\beta}]}^{\overline{\sigma},
    \overline{\tau}})_{\emptyset} \overline{y}) 
  : \xi}
\end{array}
\end{equation}
\noindent
Theorem~\ref{eq:ft-from-nat} gives a family of free theorems that are
consequences of naturality, and thus do not require the full power of
parametricity. Other specific instances include the three
$\mathtt{map}$/$\mathtt{reverse}$ theorems from
Section~\ref{sec:intro}. The $\mathtt{map}$/$\mathtt{reverse}$ theorem
for bushes, e.g., is
\begin{align*}
  &\setsem{\emptyset; \alpha~|~ y : \Nat^\emptyset \sigma \, \tau
    \vdash ((\map_{\mathit{Bush}\,\alpha}^{{\sigma},
      {\tau}})_{\emptyset} {y}) \circ (L_{{\emptyset}}
    z. \mathit{reverseBush}_{\sigma} z) : \Nat^{{\emptyset}}
    (\mathit{Bush} \, \sigma) \, (\mathit{Bush} \, \tau) }\\ =\;
  &\setsem{\emptyset; \alpha~|~ y : \Nat^\emptyset \sigma \, \tau
    \vdash (L_{\emptyset} z. \mathit{reverseBush}_{\tau} z) \circ
    ((\map_{\mathit{Bush} \, \alpha}^{{\sigma}, {\tau}})_{\emptyset}
           {y}) : \Nat^{\emptyset} (\mathit{Bush} \, \sigma ) \,
           (\mathit{Bush} \, \tau)}
\end{align*}
for any term $\vdash \mathit{reverseBush} : \Nat^\alpha (\mathit{Bush}
\,\alpha) \, (\mathit{Bush} \, \alpha)$. We can also prove a theorem
for the data type $\emptyset;\alpha,\gamma \vdash
\mathit{Tree}\,\alpha\,\gamma$ from Section~\ref{sec:calculus} that is
similar but only maps along some of its functorial variables:
%  \begin{align*}
%  &\setsem{\emptyset; \alpha~|~ y : \Nat^\emptyset \sigma \, \tau
%    \vdash ((\map_{\mathit{PTree}\,\alpha}^{{\sigma},
%      {\tau}})_{\emptyset} {y}) \circ (L_{{\emptyset}}
%    z. \mathit{reverse}_{\sigma} z) : \Nat^{{\emptyset}}
%    (\mathit{PTree} \, \sigma) \, (\mathit{PTree} \, \tau) }\\ =\;
%  &\setsem{\emptyset; \alpha~|~ y : \Nat^\emptyset \sigma \, \tau
%    \vdash (L_{\emptyset} z. \mathit{reverse}_{\tau} z) \circ
%    ((\map_{\mathit{PTree} \, \alpha}^{{\sigma}, {\tau}})_{\emptyset} {y}) :
%    \Nat^{\emptyset} (\mathit{PTree} \, \sigma ) \, (\mathit{PTree} \,
%    \tau)}
%\end{align*}
%for any term $\vdash \mathit{reverse} : \Nat^\alpha (\mathit{PTree}
%\,\alpha) \, (\mathit{PTree} \, \alpha)$,
if $\vdash \mathit{reverseTree} : \Nat^{\alpha, \gamma} (\mathit{Tree}
\,\alpha \, \gamma) \, (\mathit{Tree} \, \alpha \, \gamma)$ then
\begin{align*}
  &\setsem{\emptyset; \alpha,\gamma~|~y : \Nat^\emptyset \sigma \,
    \tau \vdash ((\map_{\mathit{Tree}\,\alpha \, \gamma}^{{\sigma},
      {\tau}})_{\emptyset} {y}) \circ (L_{{\gamma}}
    z. \mathit{reverseTree}_{\sigma, \gamma} z) : \Nat^{{\gamma}}
    (\mathit{Tree} \, \sigma \, \gamma) \, (\mathit{Tree} \, \tau \,
    \gamma) }\\ =\; &\setsem{\emptyset; \alpha,\gamma~|~y :
    \Nat^\emptyset \sigma \, \tau \vdash (L_{\gamma}
    z. \mathit{reverseTree}_{\tau, \gamma} z) \circ
    ((\map_{\mathit{Tree} \, \alpha\, \gamma}^{{\sigma},
      {\tau}})_{\emptyset} {y}) : \Nat^{\gamma} (\mathit{Tree} \,
    \sigma \, \gamma) \, (\mathit{Tree} \, \tau\, \gamma)}
\end{align*}
\noindent
%Note this instance uses the following version of map:
%$$\map_{\mathit{Tree} \,\alpha \, \gamma}^{\sigma, \tau} : 
%\Nat^{\emptyset} (\Nat^\gamma \, \sigma \,\tau)\,
%(\Nat^\gamma(\mathit{Tree}\,\sigma\,\gamma)\,(\mathit{Tree}\,\tau\,\gamma))$$
%as opposed to 
%$$\map_{\mathit{Tree} \,\alpha \, \gamma}^{\sigma, \sigma', \tau, \tau'} : 
%\Nat^{\emptyset} ((\Nat^\emptyset \, \sigma \,\tau) \times
%(\Nat^\emptyset\,\sigma'\,\tau')) \,
%(\Nat^\emptyset\,(\mathit{Tree}\,\sigma\,\sigma')\,(\mathit{Tree}\,\tau\,\tau'))$$
Finally, in Section~\ref{sec:calculus} we observed that $\emptyset;
\phi, \alpha \vdash \mathit{GRose}\,\phi\,\alpha$ cannot appear as the
(co)domain of a $\Nat$ type binding both $\phi$ and $\alpha$. As a
result, there are no instances of, say, a
$\mathtt{map}$/$\mathtt{reverse}$ free theorem for this data type that
map replacements for both $\phi$ and $\alpha$. On the other hand,
$\phi; \alpha \vdash \mathit{GRose}\,\phi\,\alpha$ does support a
$\mathtt{map}$/$\mathtt{reverse}$ theorem similar to those from
Section~\ref{sec:intro}.

As usual, analogous results hold for relation environments and
relational interpretations.

\subsection{Properties of Initial Algebraic Constructs}\label{sec:iaps}

Our semantics interprets $\map$ terms as semantic $\mathit{map}$
functions. Indeed, if \(\Gamma; \ol{\phi}, \ol{\gamma} \vdash H\),
\(\ol{\Gamma; \ol{\beta}, \ol{\gamma} \vdash F}\) and 
\(\ol{\Gamma; \ol{\beta}, \ol{\gamma} \vdash G}\), then
Definition~\ref{def:set-interp} gives
%\begin{multline}\label{eq:map-sem-def}
\begin{align}
& \setsem{
\Gamma;\emptyset \,|\, \emptyset
\vdash \map^{\ol{F},\ol{G}}_H
: \Nat^\emptyset\;(\ol{\Nat^{\ol{\beta},\ol{\gamma}}\,F\,G})\; {}
(\Nat^{\ol{\gamma}}\,H[\ol{\phi :=_{\ol{\beta}} F}]\,H[\ol{\phi :=_{\ol{\beta}} G}])
} \rho \nonumber \\ 
=\;\; & \lambda d\, \ol{\eta}\,\ol{B}.\,
\setsem{\Gamma; \ol{\phi},\ol{\gamma}\vdash H}
\id_{\rho[\ol{\gamma := B}]}[\ol{\phi := \lambda
    \ol{A}.\eta_{\ol{A}\,\ol{B}}}] \label{eq:map-sem-def}
\end{align}
%\end{multline}
In particular, if $\Gamma; \ol{\alpha} \vdash H$, $\ol{\Gamma;
  \emptyset \vdash \sigma}$, $\ol{\Gamma; \emptyset \vdash
  \tau}$, and $\ast$ is the unique element of $\setsem{\Gamma;
  \emptyset \vdash \emptyset} \rho$, then
%as a special case of the above definition we get that
\[\begin{array}{rl}
&\setsem{
\Gamma;\emptyset \,|\, \emptyset
\vdash \map^{\ol{\sigma},\ol{\tau}}_H
: \Nat^\emptyset\;(\ol{\Nat^{\emptyset}\,\sigma\,\tau})\; {}
(\Nat^{\emptyset}\,H[\ol{\alpha := \sigma}]\,H[\ol{\alpha := \tau}])
} \rho\, \ast \\
=& \lambda \ol{f : \setsem{\Gamma; \emptyset \vdash \sigma}\rho \to \setsem{\Gamma; \emptyset \vdash \tau}\rho}.\,
\setsem{\Gamma; \ol{\alpha}\vdash H}
\id_{\rho}[\ol{\alpha := f}] \\
=& \lambda \ol{f : \setsem{\Gamma; \emptyset \vdash \sigma}\rho \to \setsem{\Gamma; \emptyset \vdash \tau}\rho}.\,
\textit{map}_{\lambda \ol{A}.\,\setsem{\Gamma; \ol{\alpha}\vdash H} \rho [\ol{\alpha := A}]} \ol{f} \\
=& \textit{map}_{\lambda \ol{A}.\,\setsem{\Gamma; \ol{\alpha}\vdash H} \rho [\ol{\alpha := A}]}
\end{array}
\]
Here, the first equality is by Equation~\ref{eq:map-sem-def}, the
second is obtained by noting that $\lambda \ol{A}.\,\setsem{\Gamma;
  \ol{\alpha}\vdash H} \rho [\ol{\alpha := A}]$ is a functor in
$\alpha$, and $\textit{map}_G$ denotes the action of the semantic
functor $G$ on morphisms.

\vspace*{0.1in}

We also have the expected relationships between the interpetations of
$\map$, $\tin$, and $\fold$, namely

\vspace*{0.1in}

\noindent
$\bullet$\;  If
$\Gamma; \ol{\psi}, \ol{\gamma} \vdash H$,\;
$\ol{\Gamma; \ol{\alpha}, \ol{\gamma}, \ol{\phi} \vdash K}$,\;
$\ol{\Gamma; \ol{\beta}, \ol{\gamma} \vdash F}$, and
$\ol{\Gamma; \ol{\beta}, \ol{\gamma} \vdash G}$,
then
\[\setsem{\Gamma; \emptyset \,|\, \emptyset \vdash
\map_{H[\ol{\psi := K}]}^{\ol{F}, \ol{G}} : \xi} = \setsem{\Gamma;
  \emptyset \,|\, \emptyset \vdash \map_H^{\ol{K[\ol{\phi := F}]},
    \ol{K[\ol{\phi := G}]}} \circ \ol{\map_K^{\ol{F}, \ol{G}}} :
  \xi}\]
for $\xi = \Nat^{\emptyset} (\ol{\Nat^{\ol{\alpha}, \ol{\beta},
    \ol{\gamma}} F\, G}) (\Nat^{\ol{\gamma}} H[\ol{\psi :=
    K}][\ol{\phi := F}] H[\ol{\psi := K}][\ol{\phi := G}])$

\vspace*{0.18in}

\noindent
$\bullet$\; If $\Gamma; \overline{\beta}, \overline{\gamma} \vdash H$,\;
  $\Gamma; \overline{\beta}, \overline{\gamma} \vdash K$,\;
  $\overline{\Gamma; \overline{\alpha}, \overline{\gamma} \vdash
  F}$,\; $\overline{\Gamma; \overline{\alpha}, \overline{\gamma}
  \vdash G}$,\; $\overline{\Gamma; \phi, \overline{\psi},
  \overline{\gamma} \vdash \tau}$,\; $\overline{I}$ is the sequence
  $\overline{F}, H$ and $\overline{J}$ is the sequence $\overline{G},
  K$ then
\[\hspace*{0.13in}\setsem{\Gamma; \emptyset \,|\, \emptyset
    \vdash L_{\emptyset} (x, \overline{y}). L_{\overline{\gamma}} z.
    x_{\overline{\tau [\overline{\psi := G}] [\phi := K]},
      \overline{\gamma}} \Big(\big( (\map_{H}^{\overline{\tau
        [\overline{\psi := F}] [\phi := H]}, \overline{\tau
        [\overline{\psi := G}] [\phi := K]}})_{\emptyset}
    (\overline{(\map_{\tau}^{\overline{I}, \overline{J}})_{\emptyset}
      (x, \overline{y})}) \big)_{\overline{\gamma}} z \Big) : \xi}\]
\[= \setsem{\Gamma; \emptyset \,|\, \emptyset
    \vdash L_{\emptyset} (x, \overline{y}). L_{\overline{\gamma}} z.
    \big( (\map_{K}^{\overline{\tau [\overline{\psi := F}] [\phi :=
          H]}, \overline{\tau [\overline{\psi := G}] [\phi :=
          K]}})_{\emptyset} (\overline{(\map_{\tau}^{\overline{I},
        \overline{J}})_{\emptyset} (x, \overline{y})})
    \big)_{\overline{\gamma}} \Big( x_{\overline{\tau [\overline{\psi
            := F}] [\phi := H]}, \overline{\gamma}} z \Big) : \xi}\]
for $\xi = \Nat^{\emptyset} (\Nat^{\overline{\beta},
  \overline{\gamma}} H\, K \times \overline{\Nat^{\overline{\alpha},
    \overline{\gamma}} F\, G})\, (\Nat^{\overline{\gamma}}
H[\overline{\beta:= \tau}] [\overline{\psi := F}] [\phi := H]\,
K[\overline{\beta := \tau}] [\overline{\psi := G}] [\phi := K])$.

\vspace*{0.18in}

\noindent
$\bullet$\; If $\xi = \Nat^{\overline{\beta}, \overline{\gamma}}
H[\phi := (\mu \phi. \lambda
  \overline{\alpha}. H)\overline{\beta}][\overline{\alpha := \beta}]\,
F$, then
\[\hspace*{-0.88in}\setsem{\Gamma; \emptyset \,|\, x:
  \Nat^{\overline{\beta}, \overline{\gamma}} H[\phi :=
    F][\overline{\alpha := \beta}]\, F \vdash ((\fold_{H,
    F})_{\emptyset} x) \circ \tin_{H} : \xi}\]

\vspace*{-0.3in}

\[= \setsem{ \Gamma; \emptyset \,|\, x: \Nat^{\overline{\beta},
    \overline{\gamma}} H[\phi := F][\overline{\alpha := \beta}]\, F
  \vdash x \circ \big( (\map_{H [\ol{\alpha := \beta}]}^{(\mu
    \phi. \lambda \overline{\alpha} H) \overline{\beta},
    F})_{\emptyset} ((\fold_{H, F})_{\emptyset} x) \big) : \xi}\]

\vspace*{0.18in}

\noindent
$\bullet$\; If $\xi = \Nat^{\overline{\beta}, \overline{\gamma}} (\mu
\phi. \lambda \overline{\alpha}. H)\overline{\beta}\, (\mu
\phi. \lambda \overline{\alpha}. H)\overline{\beta}$, then
\[\setsem{\Gamma; \emptyset \,|\, \emptyset
  \vdash \tin_H \circ (\fold_{H, H[\phi := (\mu \phi. \lambda
      \overline{\alpha}. H)\overline{\beta}]})_{\emptyset}
  ((\map_{H}^{H[\phi := (\mu \phi. \lambda
      \overline{\alpha}. H)\overline{\beta}][\overline{\alpha :=
        \beta}], (\mu \phi. \lambda
    \overline{\alpha}. H)\overline{\beta}})_{\emptyset} \tin_H) :
  \xi}\]

\vspace*{-0.2in}

\[\hspace*{-3.6in} = \setsem{\Gamma; \emptyset \,|\, \emptyset \vdash \Id_{(\mu
    \phi. \lambda \overline{\alpha}. H)\overline{\beta}} : \xi}\]

\vspace*{0.18in}

\noindent
$\bullet$\; If $\xi = \Nat^{\overline{\beta}, \overline{\gamma}}
H[\phi := (\mu \phi. \lambda \overline{\alpha}. H)\overline{\beta}]\,
H[\phi := (\mu \phi. \lambda \overline{\alpha}. H)\overline{\beta}]$,
then
\[\setsem{\Gamma; \emptyset \,|\, \emptyset \vdash
(\fold_{H, H[\phi := (\mu \phi. \lambda
      \overline{\alpha}. H)\overline{\beta}]})_{\emptyset}
  ((\map_{H}^{H[\phi := (\mu \phi. \lambda
      \overline{\alpha}. H)\overline{\beta}][\overline{\alpha :=
        \beta}], (\mu \phi. \lambda
    \overline{\alpha}. H)\overline{\beta}})_{\emptyset} \tin_H) \circ
  \tin_H : \xi}\]

\vspace*{-0.2in}

\[\hspace*{-3.32in} =\setsem{ \Gamma; \emptyset \,|\, \emptyset \vdash
  \Id_{H[\phi := (\mu \phi. \lambda
      \overline{\alpha}. H)\overline{\beta}]} : \xi}\]

\vspace*{0.1in}

\noindent
Analogous results hold for relational interpretations of terms and
relational environments.

\begin{comment}
\subsection{Free Theorems Derived from Naturality}\label{sec:ft-nat}

In our setting, free theorems of a certain familiar form are derivable
directly from the semantic properties of $\Nat$-types. Indeed, the
following theorem is immediate from Equation~\ref{thm:subst}, by taking
$\gamma$ to be empty, expanding the interpretations on the left- and
right-hand sides, and using the fact from Section~\ref{sec:iaps} that
$\map$ terms are interpreted as semantic $\mathit{map}$ functions:
\begin{thm}\label{thm:ft-nat}
If\, $\Gamma; \ol\alpha \vdash F : \F$,\, $\Gamma; \ol\alpha \vdash G :
\F$, $\rho : \setenv$, $x \in \setsem{\Gamma; \emptyset \vdash
  \Nat^{\ol\alpha} F\, G } \rho$, and, for vectors of length
$|\ol\alpha|$, \, $\ol{\Gamma; \emptyset \vdash \sigma : \F}$,\,
$\ol{\Gamma; \emptyset \vdash \tau: \F}$, and $\ol{y \in
  \setsem{\Gamma; \emptyset \vdash \Nat^{\emptyset} \sigma\, \tau }
  \rho}$, then
\[\mathit{map}_{\setsem{\Gamma; \ol\alpha \vdash G} \rho[\ol{\alpha := \_}]} \ol{y}
  \circ x_{\ol{{\setsem{\Gamma; \emptyset \vdash \sigma} \rho}}} =
  x_{\ol{\setsem{\Gamma; \emptyset \vdash \tau} \rho} } \circ
  \mathit{map}_{\setsem{\Gamma; \ol\alpha \vdash F} \rho[\ol{\alpha :=
        \_}]} \, \ol{y}\]
\end{thm}
\noindent
%INCORRECT
%In particular, for $\ol{A, B: \set}$, variables $\ol{v}$ and $\ol{w}$
%with $\ol{\rho v = A}$ and $\ol{\rho w = B}$, and functions $\ol{f : A
%  \to B}$, the following holds: $\mathit{map}_{\setsem{\Gamma; \ol\alpha \vdash G}
%  \rho[\ol{\alpha := \_}]} \ol{f} \circ x_{\ol{A}} = x_{\ol{B}} \circ
%\mathit{map}_{\setsem{\Gamma; \ol\alpha \vdash F} \rho[\ol{\alpha :=
%      \_}]} \, \ol{f}$.
A similar result holds for relational interpretations.

Specific instances of Theorem~\ref{thm:ft-nat} include {\color{blue}
  Wadler free theorems, map-reverse theorems from intro, ???}
\end{comment}

\begin{comment}
We already know from Equation~\ref{thm:subst} that 
\begin{multline}
\setsem{\Gamma; \emptyset \,|\, x :
  \Nat^{\overline{\alpha}, 
    \overline{\gamma}} F\,G, \overline{y : \Nat^{\overline{\gamma}}
    \sigma\, \tau} \vdash ((\map_G^{\overline{\sigma},
    \overline{\tau}})_{\emptyset} \overline{y}) \circ
  (L_{\overline{\gamma}} z. x_{\overline{\sigma}, \overline{\gamma}}
  z) : \Nat^{\overline{\gamma}} F[\overline{\alpha := \sigma}]\,
  G[\overline{\alpha := \tau}]} \\
= \setsem{ \Gamma; \emptyset \,|\, x
  : \Nat^{\overline{\alpha}, \overline{\gamma}} F\, G, \overline{y :
    \Nat^{\overline{\gamma}} \sigma\, \tau} \vdash
  (L_{\overline{\gamma}} z. x_{\overline{\tau}, \overline{\gamma}} z)
  \circ ((\map_F^{\overline{\sigma}, \overline{\tau}})_{\emptyset}
  \overline{y}) : \Nat^{\overline{\gamma}} F[\overline{\alpha :=
      \sigma}]\, G[\overline{\alpha := \tau}]}
\end{multline}

In particular, for any set environment $\rho$, natural transformation 
$x \in \setsem{\Gamma; \emptyset \vdash \Nat^{\ol\alpha} F\, G } \rho$, and functions
$\ol{y \in \setsem{\Gamma; \emptyset \vdash \Nat^{\emptyset} \sigma\, \tau } \rho}$
(thus specializing to the case where there are no $\gamma$’s), we have
\begin{multline}
\setsem{\Gamma; \emptyset \,|\, x :
  \Nat^{\overline{\alpha}} F\,G, \overline{y : \Nat^{\emptyset}
    \sigma\, \tau} \vdash ((\map_G^{\overline{\sigma},
    \overline{\tau}})_{\emptyset} \overline{y}) \circ
  (L_{\emptyset} z. x_{\overline{\sigma}}
  z) : \Nat^{\emptyset} F[\overline{\alpha := \sigma}]\,
  G[\overline{\alpha := \tau}]} \rho (x, \ol{y}) \\
= \setsem{ \Gamma; \emptyset \,|\, x
  : \Nat^{\overline{\alpha}} F\, G, \overline{y :
    \Nat^{\emptyset} \sigma\, \tau} \vdash
  (L_{\emptyset} z. x_{\overline{\tau}} z)
  \circ ((\map_F^{\overline{\sigma}, \overline{\tau}})_{\emptyset}
  \overline{y}) : \Nat^{\emptyset} F[\overline{\alpha :=
      \sigma}]\, G[\overline{\alpha := \tau}]} \rho (x, \ol{y})
\end{multline}

Unwinding these interpretations yields
\begin{multline}
\setsem{\Gamma; \emptyset \,|\, x :
  \Nat^{\overline{\alpha}} F\,G, \overline{y : \Nat^{\emptyset}
    \sigma\, \tau} \vdash ((\map_G^{\overline{\sigma},
    \overline{\tau}})_{\emptyset} \overline{y}) \circ
  (L_\emptyset z. x_{\overline{\sigma}}
  z) : \Nat^{\emptyset} F[\overline{\alpha := \sigma}]\,
  G[\overline{\alpha := \tau}]} \rho \, (x , \ol{y}) \\
= 
\setsem{\Gamma; \emptyset \,|\, x : \Nat^{\overline{\alpha}} F\,G, 
  \overline{y : \Nat^{\emptyset}
    \sigma\, \tau} \vdash (\map_G^{\overline{\sigma},
    \overline{\tau}})_{\emptyset} \overline{y} } \rho \, (x, \ol{y})
   \circ
  \setsem{\Gamma; \emptyset \,|\, x : \Nat^{\overline{\alpha}} F\,G 
   , \overline{y : \Nat^{\emptyset}
      \sigma\, \tau} \vdash
   L_{\emptyset} z. x_{\overline{\sigma}}
  z } \rho \, (x, \ol{y}) \\
= 
\setsem{\Gamma; \emptyset \,|\, \overline{y : \Nat^{\emptyset}
    \sigma\, \tau} \vdash (\map_G^{\overline{\sigma},
    \overline{\tau}})_{\emptyset} \overline{y} } \rho \, \ol{y}
   \circ
  \setsem{\Gamma; \emptyset \,|\, x : \Nat^{\overline{\alpha}} F\,G \vdash
    L_{\emptyset} z. x_{\overline{\sigma}}
  z } \rho \, x \\
= \map_{\setsem{\Gamma; \ol\alpha \vdash G} \rho[\ol{\alpha := \_}]} \ol{y}
  \circ 
  x_{\ol{{\setsem{\Gamma; \emptyset \vdash \sigma} \rho}}}
\end{multline}


and 
\begin{multline}
\setsem{ \Gamma; \emptyset \,|\, x
  : \Nat^{\overline{\alpha}} F\, G, \overline{y :
    \Nat^{\emptyset} \sigma\, \tau} \vdash
  (L_{\emptyset} z. x_{\overline{\tau}} z)
  \circ ((\map_F^{\overline{\sigma}, \overline{\tau}})_{\emptyset}
  \overline{y}) : \Nat^{\emptyset} F[\overline{\alpha :=
      \sigma}]\, G[\overline{\alpha := \tau}]} \rho (x, \ol{y}) \\
=
\setsem{ \Gamma; \emptyset \,|\, x
  : \Nat^{\overline{\alpha}} F\, G, \overline{y :
    \Nat^{\emptyset} \sigma\, \tau} \vdash
  L_{\emptyset} z. x_{\overline{\tau}} z)} \rho (x, \ol{y})
  \circ 
  \setsem{\Gamma; \emptyset
  \,|\, x : \Nat^{\overline{\alpha}} F\, G, \overline{y :
    \Nat^{\emptyset} \sigma\, \tau} \vdash 
  (\map_F^{\overline{\sigma}, \overline{\tau}})_{\emptyset}
  \overline{y} } \rho (x, \ol{y}) \\
= \setsem{\Gamma; \emptyset \,|\, x
  : \Nat^{\overline{\alpha}} F\, G \vdash 
  L_{\emptyset} z. x_{\overline{\tau}} z} \rho \, x
  \circ 
  \setsem{\Gamma;\emptyset \,|\, \overline{y :
    \Nat^{\emptyset} \sigma\, \tau} \vdash
    (\map_F^{\overline{\sigma}, \overline{\tau}})_{\emptyset}
  \overline{y} } \rho \, \ol{y}  \\
  = x_{\ol{\setsem{\Gamma; \emptyset \vdash \tau} \rho} }
  \circ \map_{\setsem{\Gamma; \ol\alpha \vdash F} 
    \rho[\ol{\alpha := \_}]} \, \ol{y}
\end{multline}


So, we can conclude that
\begin{equation}
\map_{\setsem{\Gamma; \ol\alpha \vdash G} \rho[\ol{\alpha := \_}]} \ol{y}
  \circ 
  x_{\ol{{\setsem{\Gamma; \emptyset \vdash \sigma} \rho}}}
= x_{\ol{\setsem{\Gamma; \emptyset \vdash \tau} \rho} }
\circ \map_{\setsem{\Gamma; \ol\alpha \vdash F} 
  \rho[\ol{\alpha := \_}]} \, \ol{y}
\end{equation}

Moreover, for any $\ol{A, B: \set}$,
we can choose $\ol{\sigma = v}$ and $\ol{\tau = w}$ to be variables such that
$\ol{\rho v = A}$ and $\ol{\rho w = B}$.
Then for any functions $\ol{f : A \to B}$ we have that
\begin{equation}
\map_{\setsem{\Gamma; \ol\alpha \vdash G} \rho[\ol{\alpha := \_}]} \ol{f}
  \circ 
  x_{\ol{A}}
  = x_{\ol{B}}
\circ \map_{\setsem{\Gamma; \ol\alpha \vdash F} 
  \rho[\ol{\alpha := \_}]} \, \ol{f}
\end{equation}
\end{proof}
\end{comment}

\subsection{The Abstraction Theorem}\label{sec:thms} 

To go beyond naturality and get {\em all} consequences of
parametricity, we prove an Abstraction Theorem for our calculus. As
usual for such theorems, we prove a more general result in
Theorem~\ref{thm:at-gen} that handles possibly open terms. We then
recover the Abstraction Theorem (Theorem~\ref{thm:abstraction}) as the
special case Theorem~\ref{thm:at-gen} for closed terms of closed type.

\begin{thm}\label{thm:at-gen}
Every well-formed term $\Gamma;\Phi~|~\Delta \vdash t : \tau$ induces
a natural transformation from $\sem{\Gamma; \Phi \vdash \Delta}$ to
$\sem{\Gamma; \Phi \vdash \tau}$, i.e., a triple of natural
transformations 
\[(\setsem{\Gamma;\Phi~|~\Delta \vdash t : \tau},\,
\setsem{\Gamma;\Phi~|~\Delta \vdash t : \tau},\,
\relsem{\Gamma;\Phi~|~\Delta \vdash t : \tau})\]
where
\[\begin{array}{lll}
\setsem{\Gamma;\Phi~|~\Delta \vdash t : \tau} & : & \setsem{\Gamma;
  \Phi \vdash \Delta} \to \setsem{\Gamma; \Phi \vdash \tau}
\end{array}\]
has as its component at $\rho : \setenv$ a morphism
\[\begin{array}{lll}
\setsem{\Gamma;\Phi~|~\Delta \vdash t : \tau}\rho & : & \setsem{\Gamma;
  \Phi \vdash \Delta}\rho \to \setsem{\Gamma; \Phi \vdash \tau}\rho
\end{array}\]
in $\set$,
\[\begin{array}{lll}
\relsem{\Gamma;\Phi~|~\Delta \vdash t : \tau} & : & \relsem{\Gamma;
  \Phi \vdash \Delta} \to \relsem{\Gamma; \Phi \vdash \tau}
\end{array}\]
has as its component at $\rho : \relenv$ a morphism
\[\begin{array}{lll}
\relsem{\Gamma;\Phi~|~\Delta \vdash t : \tau}\rho & : & \relsem{\Gamma;
  \Phi \vdash \Delta}\rho \to \relsem{\Gamma; \Phi \vdash \tau}\rho
\end{array}\]
in $\rel$,
and, for all $\rho : \relenv$,
\begin{equation}\label{eq:projs}
\relsem{\Gamma;\Phi~|~\Delta \vdash t : \tau}\rho =
(\setsem{\Gamma;\Phi~|~\Delta \vdash t : \tau}(\pi_1 \rho),\,
\setsem{\Gamma;\Phi~|~\Delta \vdash t : \tau}(\pi_2 \rho))
\end{equation}
\end{thm}
\begin{proof}
By structural induction on the type judgment for $t$. The overall
challenge of our development manifests in this proof. It lies in
showing that the set and relational interpretations of each term
judgment are natural transformations, and that all set interpretations
of terms of $\Nat$-types satisfy the appropriate {\color{blue}
  equality preservation} conditions from Definition~\ref{def:set-sem}.
For the interesting cases of abstraction, application, $\map$, $\tin$,
and $\fold$ terms, propagating the naturality conditions is quite
involved; the latter two cases especially require some rather
sophisticated diagram chasing. That this is possible provides strong
evidence that our definitions and development are sensible,
natural, and at an appropriate level of abstraction. Detailed proofs
for the set interpretations for the five interesting forms of terms
are included in the accompanying anonymous supplementary
material. Even in the supplementary material we omit the other,
straightforward, cases.  The proofs for the relational interpretations
of all term judgments are entirely analogous to the proofs for their
set interpretations. The proof that Equation~\ref{eq:projs} holds in
each case is by direct calculation, using the facts that projections
are surjective and that the set and relational interpretations are
defined ``in parallel'', i.e., are fibred.
\end{proof}

The following theorem is an immediate consequence of
Theorem~\ref{thm:at-gen}:
\begin{thm}\label{thm:at-gen-rel}
If \,$\Gamma; \Phi \,|\, \Delta \vdash t : \tau$ and \,$\rho \in \relenv$,
  and if \,$(a, b) \in \relsem{\Gamma; \Phi \vdash \Delta} \rho$,
  then \\
  $(\setsem{\Gamma; \Phi \,|\, \Delta \vdash t : \tau} (\pi_1 \rho) \,a \, ,
      \setsem{\Gamma; \Phi \,|\, \Delta \vdash t : \tau } (\pi_2 \rho) \,b) \in 
    \relsem{\Gamma; \Phi \vdash \tau} \rho$
\end{thm}
\noindent
Finally, the Abstraction Theorem is the instantiation of
Theorem~\ref{thm:at-gen-rel} to closed terms of closed type:
\begin{thm}[Abstraction Theorem]\label{thm:abstraction}
If $\vdash t : \tau$, then $(\setsem{\vdash t : \tau},\setsem{\vdash t
  : \tau}) \in \relsem{\vdash \tau}$.
\end{thm}

\section{Free Theorems for Nested Types}\label{sec:ftnt}

In this section we show how Theorem~\ref{thm:abstraction} can be used
to prove more advanced free theorems than those of
Section~\ref{sec:Nat-type-terms} in the presence of directly
constructed nested types.

\subsection{Free Theorem for Type of Polymorphic
  Bottom}\label{sec:bottom} 

Suppose $ \vdash g : \Nat^\alpha \,\onet\,\alpha$, let $G^\set =
\setsem{\vdash g : \Nat^\alpha \,\onet\,\alpha}$, and let $G^\rel =
\relsem{\vdash g : \Nat^\alpha \,\onet\,\alpha}$.  By
Theorem~\ref{thm:abstraction}, $(G^\set(\pi_1\rho),G^\set(\pi_2\rho))
= G^\rel\rho$. Thus, for all $\rho \in \relenv$ and any $(a, b) \in
\relsem{\vdash \emptyset}\rho = 1$, eliding the only possible
instantiations of $a$ and $b$ gives that
%\[\begin{array}{lll}
$(G^\set,G^\set) \;= \; (G^\set(\pi_1 \rho), G^\set (\pi_2 \rho))$
$ \in  \relsem{\vdash \Nat^\alpha \,\onet\,\alpha}\rho$
$ =  \{\eta : K_1 \Rightarrow \id\}$
$ =  \{(\eta_1 : K_1 \Rightarrow \id, \eta_2 : K_1 \Rightarrow
\id)\}$ 
%\end{array}\]
That is, $G^\set$ is a natural transformation from the constantly
$1$-valued functor to the identity functor in $\set$. In particular,
for every $S : \set$, $G^\set_S : 1 \to S$. Note, however, that if $S
= \emptyset$, then there can be no such morphism, so no such natural
transformation can exist,
%in $\set$,
and thus no term $\vdash g : \Nat^\alpha \onet \,\alpha$ can exist.
%in our calculus.
That is, our calculus does not admit any terms with the closed type
$\Nat^\alpha \onet \,\alpha$ of the polymorphic bottom.

\subsection{Free Theorem for Type of Polymorphic
  Identity}\label{sec:identity} 

Suppose $ \vdash g : \Nat^\alpha \,\alpha\,\alpha$, let $G^\set =
\setsem{\vdash g : \Nat^\alpha \,\alpha\,\alpha}$, and let $G^\rel =
\relsem{\vdash g : \Nat^\alpha \,\alpha\,\alpha}$.  By
Theorem~\ref{thm:abstraction}, $(G^\set(\pi_1\rho),G^\set(\pi_2\rho))
= G^\rel\rho$. Thus, for all $\rho \in \relenv$ and any $(a, b) \in
\relsem{\vdash \emptyset}\rho = 1$, eliding the only possible
instantiations of $a$ and $b$ gives that
%\[\begin{array}{lll}
$(G^\set, G^\set) \; = \; (G^\set(\pi_1 \rho), G^\set (\pi_2 \rho))$
$ \in \relsem{\vdash \Nat^\alpha \,\alpha\,\alpha}\rho$
$ =  \{\eta : \id \Rightarrow \id\}$
$ =  \{(\eta_1 : \id \Rightarrow \id, \eta_2 : \id \Rightarrow
\id)\}$
%\end{array}\]
That is, $G^\set$ is a natural transformation from the identity
functor on $\set$ to itself. Now let $S$ be any set.  If $S =
\emptyset$, then there is exactly one morphism $\id_S: S \to S$, so
$G^\set_S : S \to S$ must be $\id_S$. If $S \not = \emptyset$, then if
$a$ is any element of $S$ and $K_a :S \to S$ is the constantly
$a$-valued morphism on $S$, then instantiating the naturality square
implied by the above equality gives that $G^\set_S \circ K_a = K_a
\circ G^\set_S$, i.e., $G^\set_S \, a = a$, i.e., $G^\set_S = \id_S$.
Putting these two cases together we have that for every $S : \set$,
$G^\set_S = \id_S$, i.e., $G^\set$ is the identity natural
transformation for the identity functor on $\set$. So every closed
term $g$ of closed type $\Nat^\alpha\alpha\,\alpha$ always denotes the
identity natural transformation for the identity functor on $\set$,
i.e., every closed term $g$ of type $\Nat^\alpha\alpha\,\alpha$
denotes the polymorphic identity function.

\subsection{Standard Free Theorems for ADTs and Their Analogues for
  Nested types}\label{sec:ft-adt}

We can derive in our system even those free theorems for polymorphic
functions over ADTs that are not consequences of naturality.  We can,
e.g., prove the free theorem for $\mathit{filter}$'s type as follows:

%\begin{lemma}\label{lem:list-graph}
%If $g : A \to B$, $\rho : \relenv$, and $\rho\alpha = (A, B,
%\graph{g})$, then $\relsem{\alpha; \emptyset \vdash List \, \alpha}
%\rho = \graph{map_{\mathit{List}} \, g}$.
%\end{lemma}
%\begin{proof}
%The result follows by straightforward calculation using
%Definition~\ref{def:rel-sem} and Equation~\ref{eq:mu}, together with
%the facts that $\mathit{List} \,(A, B, \graph{g}) =
%\graph{\mathit{map}_{\mathit{List}} \,g}$ and, if $H = \onet + \beta
%\times\phi \beta$ {\color{blue} Why not $H = \onet + \alpha \times
%  \beta$? Does it matter?}, then $(T_{H,\rho}^n K_0)^* R =
%\Sigma_{i=0}^n R^i$ for all $n \in \nat$ and $R \in \rel$ by induction
%on $n$.
%\end{proof}

\begin{thm} 
If $g : A \to B$, $\rho : \relenv$, $\rho \alpha = (A, B, \graph{g})$,
$(a, b) \in \relsem{\alpha ;\emptyset \vdash \Delta} \rho$, $(s \circ
g, s) \in \relsem{\alpha; \emptyset \vdash \Nat^\emptyset \alpha \,
  \mathit{Bool}} \rho$, and % for $s : A \to \mathit{Bool}$, and
%$\alpha; \emptyset \,|\, \Delta \vdash t :
%\filtype$, if
$\mathit{filter} = \setsem{\alpha; \emptyset \,|\, \Delta \vdash t :
  \filtype}$ for some $t$, then
\[  \mathit{map}_{\mathit{List}} \,g \circ \mathit{filter} \, (\pi_1
\rho) \, a \, (s \circ g) = \mathit{filter}\, (\pi_2\rho) \, b \, s
\circ \mathit{map}_{\mathit{List}} \,g\]
\end{thm}
\begin{proof}
By Theorem~\ref{thm:at-gen-rel}, \[(\mathit{filter}\, (\pi_1 \rho)\, a,
\mathit{filter}\, (\pi_2 \rho)\, b) \in \relsem{\alpha; \emptyset
  \vdash \filtype} \rho\] so if $(s, s') \in \relsem{\alpha; \emptyset
  \vdash \Nat^\emptyset \alpha \, \mathit{Bool}} \rho = \rho\alpha \to
\Eq_{\mathit{Bool}}$ and $(xs, xs') \in \relsem{\alpha; \emptyset
  \vdash List \, \alpha} \rho$ then
% then
%  \begin{align*}
%    (t (\pi_1 \rho) \,a \,s, t (\pi_2 \rho) \,b \,s') &\in \relsem{\alpha; \emptys%et \vdash \Nat^\emptyset (List \, \alpha) \, (List \, \alpha)} \rho \\
%   &= \relsem{\alpha; \emptyset \vdash List \, \alpha} \rho \to \relsem{\alpha; \e%mptyset \vdash List \, \alpha} \rho \\
%  \end{align*}
%  So if $(xs, xs') \in \relsem{\alpha; \emptyset \vdash List \, \alpha} \rho$ then,
\begin{equation}\label{eq:filter-thm-list}
  (\mathit{filter}\, (\pi_1\rho) \,a \,s \,xs, \mathit{filter} \,
  (\pi_2\rho) \,b \,s' \,xs') \in \relsem{\alpha; \emptyset \vdash
    \mathit{List} \, \alpha} \rho
\end{equation}
If $\rho\alpha = (A, B, \graph{g})$, then $\relsem{\alpha; \emptyset
  \vdash \mathit{List} \, \alpha} \rho =
\graph{\mathit{map}_{\mathit{List}} \, g}$ by Lemma~\ref{lem:graph}
and Equation~\ref{thm:demotion-objects}.  Moreover, $xs' =
\mathit{map}_{\mathit{List}} \,g \,xs$ and $(s', s) \in \graph{g} \to
\Eq_{\mathit{Bool}}$, so $s' = s \circ g$. The result
follows from % Instantiating
Equation~\ref{eq:filter-thm-list}.\end{proof}
%gives the result.
% \begin{align*}
%    &(t (\pi_1\rho) \,a \,(s' \circ g) \,xs, t (\pi_2\rho) \,b \,s' \,(\mathit{map%}_{\mathit{List}} \,g \,xs)) \in \graph{\mathit{map}_{\mathit{List}} \,g}, \\ 
%    & i.e., \\ 
%    &\mathit{map}_{\mathit{List}} \,g \, (t (\pi_1\rho) \,a \,(s' \circ g) \,xs) =% t (\pi_2\rho) \,b \,s' \,(\mathit{map}_{\mathit{List}} \,g \,xs), \\ 
%    & i.e., \\
%    &\mathit{map}_{\mathit{List}} \,g \circ t (\pi_1 \rho) \, a \, (s' \circ g) = %t (\pi_2\rho) \, b \, s' \circ \mathit{map}_{\mathit{List}} \,g
%  \end{align*}
%  as desired.
%\end{proof}

\begin{comment}
\begin{thm}\label{thm:at-gen-rel}
If \,$\Gamma; \Phi \,|\, \Delta \vdash t : \tau$ and \,$\rho \in \relenv$,
  and if \,$(a, b) \in \relsem{\Gamma; \Phi \vdash \Delta} \rho$,
  then \\
  $(\setsem{\Gamma; \Phi \,|\, \Delta \vdash t : \tau} (\pi_1 \rho) \,a \, ,
      \setsem{\Gamma; \Phi \,|\, \Delta \vdash t : \tau } (\pi_2 \rho) \,b) \in 
    \relsem{\Gamma; \Phi \vdash \tau} \rho$
\end{thm}
\begin{proof}
 Immediate from Theorem~\ref{thm:at-gen}.
\end{proof}
\end{comment}

\begin{comment}
\begin{thm} 
  If $g : A \to B$, 
  $\rho : \relenv$, 
  $\rho \alpha = (A, B, \graph{g})$,
  $(a, b) \in \relsem{\alpha ;\emptyset \vdash \Delta} \rho$, 
  $(s \circ g, s) \in \relsem{\alpha; \emptyset \vdash \Nat^\emptyset \alpha \, \mathit{Bool}} \rho$,
  and, for some well-formed term $\mathit{filter}$, \\
  $$t = \setsem{\alpha; \emptyset \,|\, \Delta \vdash \mathit{filter} : \filtype} \emph{, then }$$
  \begin{align*}
  \mathit{map}_{\mathit{List}} \,g \circ t (\pi_1 \rho) \, a \, (s \circ g) = t (\pi_2\rho) \, b \, s \circ \mathit{map}_{\mathit{List}} \,g
  \end{align*}
\end{thm}
\begin{proof}
  By Theorem~\ref{thm:at-gen-rel},
  $(t (\pi_1 \rho) a, t (\pi_2 \rho) b) \in \relsem{\alpha; \emptyset \vdash \filtype} \rho$.
  Thus if 
  $(s, s') \in \relsem{\alpha; \emptyset \vdash \Nat^\emptyset \alpha \, \mathit{Bool}} \rho = \rho\alpha \to \Eq_{\mathit{Bool}}$,
  then 
  \begin{align*}
    (t (\pi_1 \rho) \,a \,s, t (\pi_2 \rho) \,b \,s') &\in \relsem{\alpha; \emptyset \vdash \Nat^\emptyset (List \, \alpha) \, (List \, \alpha)} \rho \\
   &= \relsem{\alpha; \emptyset \vdash List \, \alpha} \rho \to \relsem{\alpha; \emptyset \vdash List \, \alpha} \rho \\
  \end{align*}
  So if $(xs, xs') \in \relsem{\alpha; \emptyset \vdash List \, \alpha} \rho$ then,
  \begin{equation}\label{eq:filter-thm}
  (t (\pi_1\rho) \,a \,s \,xs, t (\pi_2\rho) \,b \,s' \,xs') \in \relsem{\alpha; \emptyset \vdash List \, \alpha} \rho
  \end{equation}

  Consider the case in which $\rho\alpha = (A, B, \graph{g})$. Then $\relsem{\alpha; \emptyset \vdash List \, \alpha} \rho
  = \graph{\mathit{map}_{\mathit{List}} \, g}$, by Lemma~\ref{lem:list-graph}, and $(xs, xs') \in \graph{\mathit{map}_{\mathit{List}} \,g}$ 
  implies $xs' = \mathit{map}_{\mathit{List}} \,g \,xs$. We also have that 
  $(s, s') \in \graph{g} \to \Eq_{\mathit{Bool}}$ implies 
  $\forall (x, g x) \in \graph{g}. \,\, s x = s' (g x)$ and thus
  $s = s' \circ g$ due to the definition of morphisms between relations.
  %% TODO show (s, s') \in <g> -> \mathit{Bool} implies s = s' \circ g %% 
  With these instantiations, Equation~\ref{eq:filter-thm} becomes
  \begin{align*}
    &(t (\pi_1\rho) \,a \,(s' \circ g) \,xs, t (\pi_2\rho) \,b \,s' \,(\mathit{map}_{\mathit{List}} \,g \,xs)) \in \graph{\mathit{map}_{\mathit{List}} \,g}, \\ 
    & i.e., \\ 
    &\mathit{map}_{\mathit{List}} \,g \, (t (\pi_1\rho) \,a \,(s' \circ g) \,xs) = t (\pi_2\rho) \,b \,s' \,(\mathit{map}_{\mathit{List}} \,g \,xs), \\ 
    & i.e., \\
    &\mathit{map}_{\mathit{List}} \,g \circ t (\pi_1 \rho) \, a \, (s' \circ g) = t (\pi_2\rho) \, b \, s' \circ \mathit{map}_{\mathit{List}} \,g
  \end{align*}
  as desired.
\end{proof}
\end{comment}

%\subsection{Free Theorem for Type of $\mathit{filter}$ for
%  $\GRose$}\label{sec:filter-grose} 

A similar proof establishes the analogous result for, say, generalized
rose trees. 
\begin{thm} 
  If $g : A \to B$,
$F, G : \set \to \set$,
  $\eta : F \to G$ in $\set$, $\rho : \relenv$, $\rho \alpha =
 (A, B, \graph{g})$, $\rho \psi = (F, G, \graph{\eta})$, $(a, b) \in
 \relsem{\alpha, \psi ;\emptyset \vdash \Delta} \rho$, $(s \circ
 g, s) \in \relsem{\alpha; \emptyset \vdash \Nat^\emptyset \alpha \,
   \mathit{Bool}} \rho$, and
 \[ \mathit{filter} = \setsem{\alpha, \psi; \emptyset \,|\, \Delta
      \vdash t : \filtypeGRose}  \]
for some $t$, then
\[ \semmap_{\mathit{GRose}}\, \eta\, (g + 1) \circ \mathit{filter} \,
(\pi_1 \rho) \, a \, (s \circ g) = \mathit{filter} \, (\pi_2\rho) \, b
\, s \circ 
\semmap_{\mathit{GRose}}\, \eta\, g\]
\end{thm}

\noindent
This is not surprising since rose trees are essentially
ADT-like. However, as noted in Section~\ref{sec:terms}, our system
cannot express the type of a polymorphic filter function for a proper
nested type.

\subsection{Short Cut Fusion for Lists}\label{sec:short-cut}

We can recover standard short cut fusion for lists~\cite{glp93} in our
system: 
\begin{thm}
If $\vdash \tau$, $\vdash \tau'$, and 
%$\beta; \emptyset \,|\, \emptyset \vdash g : \Nat^{\emptyset}
%(\Nat^{\emptyset} (\onet + \tau \times \beta)\, \beta)\, \beta$. 
%If
$G \, = \, \setsem{\beta; \emptyset \,|\, \emptyset \vdash g :
  \Nat^{\emptyset} (\Nat^{\emptyset} (\onet + \tau \times \beta)\,
  \beta)\, \beta}$ for some $g$, then
\[\textit{fold}_{1 + \tau \times \_}\, n\, c\; (G\; (\mathit{List}\,
\setsem{\vdash \tau})\,\mathit{nil} \,\mathit{cons}) = G \,\setsem{\vdash
  \tau'}\, n\, c \]
\end{thm}
\begin{proof}
%Let $\vdash \tau : \F$ and $\vdash \tau' : \F$, let
%\[\beta; \emptyset \,|\, \emptyset \vdash g : \Nat^{\emptyset}
%(\Nat^{\emptyset} (\onet + \tau \times \beta)\, \beta)\, \beta\]
%and let
%\[G = \setsem{\beta; \emptyset \,|\, \emptyset
%  \vdash g : \Nat^{\emptyset} (\Nat^{\emptyset} (\onet + \tau \times
  %  \beta)\, \beta)\, \beta}\] Then
Theorem~\ref{thm:abstraction} gives
that, for any $\rho : \relenv$,
%and any $(a, b) \in
%\relsem{\beta; \emptyset \vdash \emptyset}\rho = 1$, then (eliding the
%only possible instantiations of $a$ and $b$) we have
\[(G \,(\pi_1 \rho), G\, (\pi_2 \rho))
 \in  \relsem{\beta; \emptyset \vdash \Nat^{\emptyset} (\Nat^{\emptyset}
  (\onet + \tau \times \beta)\, \beta)\, \beta} \rho\\
\cong (((\relsem{\vdash \tau} \rho \times
\rho\beta) \to \rho\beta) \times \rho\beta) \to \rho\beta\] 
%\[\begin{array}{lll}
%(G \,(\pi_1 \rho), G\, (\pi_2 \rho))
% & \in & \relsem{\beta; \emptyset \vdash \Nat^{\emptyset} (\Nat^{\emptyset}
%  (\onet + \tau \times \beta)\, \beta)\, \beta} \rho\\
%& \cong& (((\relsem{\vdash \tau} \rho \times
%\rho\beta) \to \rho\beta) \times \rho\beta) \to \rho\beta 
%\end{array}\]
%
%Since
%\[\begin{array}{rl}
% & \relsem{\beta; \emptyset \vdash \Nat^{\emptyset}
%  (\Nat^{\emptyset} (\onet + \tau \times \beta)\, \beta)\, \beta} \rho\\  
%=& \relsem{\beta; \emptyset \vdash \Nat^{\emptyset} (\onet +
%  \tau \times \beta)\, \beta} \rho \to \rho\beta \\ 
%=& (\relsem{\beta; \emptyset \vdash \onet + \tau \times \beta}
%\rho \to \rho\beta) \to \rho\beta \\ 
%=& ((\onet + \relsem{\vdash \tau} \rho \times
%\rho\beta) \to \rho\beta) \to \rho\beta \\ 
%\cong& (((\relsem{\vdash \tau} \rho \times
%\rho\beta) \to \rho\beta) \times \rho\beta) \to \rho\beta 
%\end{array}\]
%we have that
so if $(c', c) \in \relsem{\vdash \tau} \rho \times
\rho\beta \to \rho\beta$ and $(n', n) \in \rho\beta$ then
$(G \,(\pi_1 \rho)\, n'\, c', G\, (\pi_2 \rho)\, n\, c)
\in \rho \beta$.
In addition,
\[\setsem{\vdash \fold_{\onet + \tau
    \times \beta}^{\tau'} : \Nat^{\emptyset} (\Nat^{\emptyset} (\onet
  + \tau \times \tau')\, \tau')\, (\Nat^{\emptyset} (\mu \alpha. \onet
  + \tau \times \alpha)\, \tau')} =
\textit{fold}_{1 + \tau \times \_}\] so that if $c \in
\setsem{\vdash \tau} \,\times\, \setsem{\vdash \tau'} \to
\setsem{\vdash \tau'}$ and $n \in \setsem{\vdash \tau'}$, then
$(n, c) \in \setsem{\vdash \Nat^{\emptyset} (\onet + \tau \times \tau')\,
  \tau'}$. The instantiation
$\pi_1 \rho \beta = \setsem{\vdash \mu \alpha. \onet + \tau \times
  \alpha}=\mathit{List}\,\setsem{\vdash \tau}$,\, 
$\pi_2 \rho \beta = \setsem{\vdash \tau'}$,\,
$\rho \beta = \graph{\textit{fold}_{1 + \tau \times \_}\, n\, c} :
\rel(\pi_1 \rho \beta, \pi_2 \rho \beta)$,\,
$c' = \mathit{cons}$,\, and 
$n' = \mathit{nil}$
%Consider the instantiation:
%\[\begin{array}{rcl}
%\pi_1 \rho \beta &=& \setsem{\vdash \mu \alpha. \onet + \tau \times
%  \alpha} \;=\;\List\,\setsem{\vdash \tau}\\   
%\pi_2 \rho \beta &=& \setsem{\vdash \tau'}\\
%\rho \beta &=& \graph{\textit{fold}_{1 + \tau \times \_}\, n\, c} :
%\rel(\pi_1 \rho \beta, \pi_2 \rho \beta) \\ 
%c' &=& \mathit{cons}\\
%n' &=& \mathit{nil}
%\end{array}\]
%Clearly, $(\mathit{nil}, n) \in \rho\beta =
%\graph{\textit{fold}_{1 + \tau \times \_}\, n\, c}$ because $
%\textit{fold}_{1 + \tau \times \_}\, n\, c \,\mathit{nil} = n$.  Moreover,
%$(\mathit{cons}, c) \in \relsem{\vdash \tau} \,\times\, \rho\beta \to
%\rho\beta$ since if $(x, x') \in \relsem{\vdash \tau}$, i.e., $x =
%x'$, and if $(y, y') \in \rho\beta =
%\graph{\textit{fold}_{1 + \tau \times \_} n\, c}$, i.e., $y' =
%\textit{fold}_{1 + \tau \times \_} n\, c\, y$, then
%\[(\mathit{cons}\, x\, y, c\, x\, (\textit{fold}_{1 + \tau \times \_}\, n\,
%c\, y)) \in \graph{\textit{fold}_{1 + \tau \times \_}\, n\, c}\]
%i.e.,
%\[c\, x\, (\textit{fold}_{1 + \tau \times \_}\, n\, c\, y)
%= \textit{fold}_{1 + \tau \times \_}\, n\, c\, (\mathit{cons} \,x\, y)\]
%holds by definition of $\textit{fold}_{1 + \tau \times \_}$.  We
%therefore conclude that
thus gives that $(G \;(\mathit{List}\,\setsem{\vdash
  \tau})\,\mathit{nil} \,\mathit{cons}, G \,\setsem{\vdash \tau'} \,
n\, c) \in \graph{\textit{fold}_{1 + \tau \times \_}\, n\, c}$, i.e.,
that $\textit{fold}_{1 + \tau \times \_}\, n\, c\; (G\;
(\mathit{List}\,\setsem{\vdash \tau})\, \mathit{nil}\, \mathit{cons})
= G \,\setsem{\vdash \tau'}\, n\, c$.
\end{proof}

Using Equation~\ref{thm:demotion-objects} we can extend short
cut fusion results to arbitrary ADTs, as in~\cite{joh02}.

\begin{comment}
\subsection{Short Cut Fusion for Arbitrary ADTs}\label{sec:short-cut-adt}

\begin{thm}
Let $\ol{\vdash \tau}$, let $\vdash \tau'$, let $\ol{\alpha}; \beta
\vdash F$, and let $\beta; \emptyset  \,|\,
\emptyset \vdash g : \Nat^{\emptyset} (\Nat^{\emptyset} F[\ol{\alpha
    := \tau}]\, \beta)\, \beta$.  If we regard
\[\begin{array}{lll}
H & = & \setsem{\emptyset;\beta \vdash F[\ol{\alpha :=
      \tau}]}\\
G & = & \setsem{\beta; \emptyset \,|\, \emptyset \vdash g :
  \Nat^{\emptyset} (\Nat^{\emptyset}\, F[\ol{\alpha := \tau}]\,
  \beta)\, \beta}
\end{array}\]
as functors in $\beta$, then for every $B \in H \setsem{\vdash \tau'}
\rightarrow \setsem{\vdash \tau'}$ we have
\[\textit{fold}_H\, B \; (G\; \mu H \; in_H) = G \,\setsem{\vdash \tau'}\, B\]
\end{thm}

\vspace*{0.2in}

\begin{proof}
  We first note that the type of $g$ is well-formed, since
  $\emptyset;\beta \vdash F[\ol{\alpha := \tau}]$ so our
  promotion theorem gives that $\beta;\emptyset\vdash F[\ol{\alpha :=
      \tau}]$, and $\emptyset;\beta\vdash\beta$ so that our
  promotion theorem gives $\beta;\emptyset\vdash\beta$. From
  these facts we deduce that $\beta;\emptyset \vdash
  \Nat^\emptyset\,F[\ol{\alpha := \tau}]\,\beta : \T$, and thus that
  $\beta; \emptyset \vdash \Nat^{\emptyset} (\Nat^{\emptyset}
  F[\ol{\alpha := \tau}]\, \beta)\, \beta : \T$.

  Theorem~\ref{thm:abstraction} gives that, for any $\rho : \relenv$
  and any $(a, b) \in \relsem{\beta; \emptyset \vdash \emptyset}\rho =
  1$, eliding the only possible instantiations of $a$ and $b$ gives
  that
\[(G \,(\pi_1 \rho), G\, (\pi_2 \rho))
\in \relsem{\beta; \emptyset \vdash \Nat^{\emptyset} (\Nat^{\emptyset}
  F[\ol{\alpha := \tau}]\, \beta)\, \beta} \rho\]
Since
\[\begin{array}{rl}
 & \relsem{\beta; \emptyset \vdash \Nat^{\emptyset}
  (\Nat^{\emptyset} F[\ol{\alpha := \tau}]\, \beta)\, \beta} \rho\\  
=& \relsem{\beta; \emptyset \vdash \Nat^{\emptyset} F[\ol{\alpha :=
      \tau}] \, \beta} \rho \to \rho\beta \\ 
\end{array}\]
we have that if $(A, B) \in \relsem{\beta; \emptyset \vdash
  \Nat^{\emptyset} F[\ol{\alpha := \tau}] \, \beta} \rho$
then
\[(G \,(\pi_1 \rho)\, A, G\, (\pi_2 \rho)\, B)
\in \rho \beta\]
Now note that
\[\setsem{\vdash \fold_{F[\ol{\alpha := \tau}]}^{\tau'} :
  \Nat^{\emptyset}\, (\Nat^{\emptyset}\, F[\ol{\alpha := \tau}][\beta
    := \tau'] \, \tau')\, (\Nat^{\emptyset} \,(\mu
  \beta. F[\ol{\alpha := \tau}] \, \tau')} =
\textit{fold}_H\]
and consider the instantiation
\[\begin{array}{lll}
A &=& \mathit{in}_H :  H (\mu H) \to \mu H\\
B & : & H\setsem{\vdash \tau'} \to \setsem{\vdash \tau'}\\
\rho \beta &=& \graph{\textit{fold}_H\, B}
\end{array}\]
(Note that all the types here are well-formed.)  This gives
\[\begin{array}{lll}
\pi_1 \rho \beta &=& \setsem{\vdash \mu \beta. F[\ol{\alpha := \tau}]}
\; = \; \mu H\\
\pi_2 \rho \beta &=& \setsem{\vdash \tau'}\\
\rho \beta &:& \rel(\pi_1 \rho \beta, \pi_2 \rho \beta)\\
A & : & \setsem{\beta; \emptyset \vdash \Nat^{\emptyset}
  F[\ol{\alpha := \tau}] \, \beta} (\pi_1 \rho)\\
B & : & \setsem{\beta; \emptyset \vdash \Nat^{\emptyset} F[\ol{\alpha
      := \tau}] \, \beta}(\pi_2 \rho)\\
\end{array}\]
since
\[\begin{array}{lll}
A \; = \; \mathit{in}_H & : & H(\mu H) \to \mu H\\
 & = & \setsem{\emptyset;\beta \vdash F[\ol{\alpha := \tau}]}(\mu
\setsem{\emptyset;\beta \vdash F[\ol{\alpha := \tau}]}) \to \mu
\setsem{\emptyset;\beta \vdash F[\ol{\alpha := \tau}]}\\
& = & \setsem{\emptyset;\beta \vdash F[\ol{\alpha :=
      \tau}]}(\pi_1\rho) \to \setsem{\emptyset;\beta \vdash
  \beta}(\pi_1 \rho)\\
& = & \setsem{\beta;\emptyset \vdash F[\ol{\alpha :=
      \tau}]}(\pi_1\rho) \to \setsem{\beta; \emptyset \vdash
  \beta}(\pi_1 \rho) \hspace*{0.25in} {\color{red} \mbox{Daniel's
    trick; now a theorem}} \\ 
& = & \setsem{\beta;\emptyset \vdash \Nat^\emptyset F[\ol{\alpha :=
      \tau}]\,\beta}(\pi_1 \rho) 
\end{array}\]
{\color{red} where ``Daniel's trick'' is the observation that a
  functor can be seen as non-functorial when we only care about its
  action on objects. This is now a theorem.}  We also have
\[\begin{array}{lll}
(A,B) \,=\, (\mathit{in}_H,B) & \in & \relsem{\beta; \emptyset \vdash
  \Nat^{\emptyset} F[\ol{\alpha := \tau}] \, \beta} \rho \\
 & = & \relsem{\beta; \emptyset \vdash F[\ol{\alpha :=
      \tau}]}\rho[\beta := \graph{\textit{fold}_H\, B}] \to
\graph{\textit{fold}_H\, B}\\
 & = & \relsem{\beta; \emptyset \vdash F[\ol{\alpha :=
      \tau}]}\graph{\textit{fold}_H\, B} \to
\graph{\textit{fold}_H\, B}\\
 & = & \relsem{\emptyset; \beta\vdash F[\ol{\alpha :=
      \tau}]}\graph{\textit{fold}_H\, B} \to
\graph{\textit{fold}_H\, B} \hspace*{0.35in} {\color{red}
  \mbox{Daniel's trick; now a theorem}}\\  
& = & \graph{\setsem{\emptyset; \beta \vdash F[\ol{\alpha :=
      \tau}]}\,(\textit{fold}_H\, B)} \to
\graph{\textit{fold}_H\, B} \hspace*{0.25in} {\color{red} \mbox{by the
    graph lemma}}\\ 
& = & \graph{\mathit{map}_H \,(\mathit{fold}_H\,B)} \to
\graph{\textit{fold}_H\, B}\\
\end{array}\]
since if $(x,y) \in \graph{\mathit{map}_H \,(\mathit{fold}_H\,B)}$,
i.e., if $\mathit{map}_H \,(\mathit{fold}_H\,B) \, x = y$, then
$\textit{fold}_H\, B\, (\mathit{in}_H\,x) = B\,y = B\, (\mathit{map}_H
\,(\mathit{fold}_H\,B) \, x)$ by the definition of $\mathit{fold}_H$
as a (indeed, the unique) morphism from $\mathit{in}_H$ to $B$. Thus,
\[(G \,(\pi_1 \rho)\, A, G\, (\pi_2 \rho)\, B) \in
\graph{\textit{fold}_H\, B}\]
i.e.,
\[\textit{fold}_H \, B \, (G\, (\pi_1 \rho) \, \mathit{in}_H) =
G\,(\pi_2 \rho)\,B\] 

\vspace*{0.1in}

\noindent
Since $\beta$ is the only free variable in $G$, this simplifies to
\[\textit{fold}_H\, B \, (G\, \mu H\, \mathit{in}_H) =
G\,\setsem{\vdash \tau'}\,B\]  
\end{proof}
\end{comment}

\subsection{Short Cut Fusion for Arbitrary Nested
  Types}\label{sec:short-cut-nested} 

%fold/build rule shows that Church encodings for nested types are iso
%(have the same interp) as the nested types themselves. Just like Ex
%2.8 in parpoe, but categorical rather than operational semantics. But
%built-in data types are more versatile in implementations. They can be
%inducted on, e.g. And they are stored in the heap, not the run-time
%stack, and are therefore more efficient.
%``Most higher order type languages meant for human programmers eschew
%fully impredicative polymorphism'' says Andy on p322.

We can also, finally, formally prove correctness of the categorically
inspired short cut fusion for nested types from~\cite{jg10}. Indeed,
we have:
\begin{thm}\label{thm:short-cut-nested}
If $\emptyset;\phi,\alpha \vdash F$, \,$\emptyset; \alpha
\vdash K$, \,
%let $\phi ;\emptyset\,|\,\emptyset\vdash g :
%\Nat^\emptyset\,(\Nat^\alpha\,F\,(\phi\alpha)) \,
%(\Nat^\alpha\,\onet \, (\phi\alpha))$. If we let
$H : [\set,\set] \to [\set,\set]$ is defined by
\[\begin{array}{lll}
H\,f\,x & = & \setsem{\emptyset; \phi, \alpha \vdash F}[\phi :=
  f][\alpha := x]\\
\end{array}\]
and 
\[G = \setsem{\phi;\emptyset\,|\,\emptyset \vdash g :
\Nat^\emptyset\,(\Nat^\alpha\,F\,(\phi\alpha))\,(\Nat^\alpha\,\onet \,
(\phi\alpha))}\] for some $g$, then, for every $B \in H
\setsem{\emptyset;\alpha \vdash K} \rightarrow \setsem{\emptyset;
  \alpha \vdash K}$,
$\textit{fold}_{H}\, B \, (G\; \mu H \; in_{H}) = G
\,\setsem{\emptyset;\alpha \vdash K}\, B$.
\end{thm}

\begin{proof}
%We first note that the type of $g$ is well-formed since
%$\emptyset;\phi,\alpha \vdash F : \F$ so our promotion theorem gives
%that $\phi;\alpha \vdash F : \F$, and $\phi;\alpha \vdash \phi\alpha :
%\F$, so that $\phi;\emptyset \vdash \Nat^\alpha F\,(\phi\alpha) : \T$
%and $\phi;\emptyset \vdash \Nat^\alpha \onet \,(\phi\alpha) : \T$.
%Then $\phi;\emptyset \vdash \Nat^\alpha F\,(\phi\alpha) : \F$ and
%$\phi;\emptyset \vdash \Nat^\alpha \onet \,(\phi\alpha) : \F$ also
%%%hold, and, finally, 
%$\phi ;\emptyset\vdash 
%\Nat^\emptyset\,(\Nat^\alpha\,F\,(\phi\alpha)) \,
%(\Nat^\alpha\,\onet \, (\phi\alpha)) : \T$
%  
Theorem~\ref{thm:abstraction} gives that, for any 
$\rho : \relenv$,
%and any $(a, b) \in \relsem{\phi,\alpha;\emptyset \vdash
%  \emptyset}\rho = 1$, eliding the only possible instantiations
%of $a$ and $b$ gives that
\[\begin{array}{lll}
(G \,(\pi_1 \rho), G\, (\pi_2 \rho))
& \in & \relsem{\phi;\emptyset\vdash \Nat^{\emptyset} (\Nat^\alpha
  F\, (\phi\alpha))\, (\Nat^\alpha\,\onet \, (\phi\alpha))}
\rho\\ 
& = & \relsem{\phi;\emptyset\vdash \Nat^\alpha F\,
  (\phi\alpha)}\rho \to \relsem{\phi;\emptyset\vdash
  \Nat^\alpha\,\onet \, (\phi\alpha)}\rho\\ 
%& = & \relsem{\phi;\emptyset\vdash \Nat^\alpha F\,
%  (\phi\alpha)}\rho \to (\,\lambda A. 1 \Rightarrow \lambda A.\,(\rho
%\phi) A\,)\\   
%& = & \relsem{\phi;\emptyset\vdash \Nat^\alpha F\,
%  (\phi\alpha)}\rho \to (1 \Rightarrow \rho \phi)\\ 
& = & \relsem{\phi;\emptyset\vdash \Nat^\alpha F\,
  (\phi\alpha)}\rho \to \rho \phi
\end{array}\]
\noindent
so if $(A, B) \in \relsem{\phi;\emptyset\vdash \Nat^\alpha F\,
  (\phi\alpha)}\rho$ then $(G \,(\pi_1 \rho)\, A, G\, (\pi_2 \rho)\,
B) \in \rho \phi$.
In addition,
%$\textit{fold}_H = \setsem{\vdash \fold_F^K :
%  \Nat^{\emptyset}\, (\Nat^\alpha F[\phi := K]\,K)\, (\Nat^\alpha
%  ((\mu \phi.\lambda\alpha.F)\alpha)\,K)}$.
\[\setsem{\vdash \fold_F^K :
  \Nat^{\emptyset}\, (\Nat^\alpha F[\phi := K]\,K)\, (\Nat^\alpha
  ((\mu \phi.\lambda\alpha.F)\alpha)\,K)} = \textit{fold}_H\]
Consider the instantiation $A = \mathit{in}_H : H (\mu H) \Rightarrow
\mu H$,\, $B : H\setsem{\emptyset;\alpha\vdash K} \Rightarrow
\setsem{\emptyset;\alpha \vdash K}$,\, $\rho \phi =
\graph{\mathit{fold}_H\, B}$,\, $\pi_1 \rho \phi = \mu H$,\, $\pi_2
\rho \phi = \setsem{\emptyset;\alpha\vdash K}$,\, $\rho \phi :
\rel(\pi_1 \rho \phi, \pi_2 \rho \phi)$,\, $A : \setsem{\phi;
  \emptyset \vdash \Nat^\alpha F \, (\phi \alpha)} (\pi_1 \rho)$,\,
and $B : \setsem{\phi; \emptyset \vdash \Nat^\alpha F \, (\phi
  \alpha)} (\pi_2 \rho)$.
%\[\begin{array}{lll}
%A &=& \mathit{in}_H :  H (\mu H) \Rightarrow \mu H\\
%B & : & H\setsem{\emptyset;\alpha\vdash K} \Rightarrow
%\setsem{\emptyset;\alpha \vdash K}\\
%\rho \phi &=& \graph{\textit{fold}_H\, B} \hspace*{0.2in} {
%  \color{red} \mbox{a graph of a natural transformation, defined in
%    Enrico's notes}} 
%\end{array}\]
%(Note that all the types here are well-formed.) This gives
%\[\begin{array}{lll}
%\pi_1 \rho \phi &=& \mu H\\
%\pi_2 \rho \phi &=& \setsem{\emptyset;\alpha\vdash K}\\
%\rho \phi &:& \rel(\pi_1 \rho \phi, \pi_2 \rho \phi)\\
%A & : & \setsem{\phi; \emptyset \vdash \Nat^\alpha
%  F \, (\phi \alpha)} (\pi_1 \rho)\\
%B & : & \setsem{\phi; \emptyset \vdash \Nat^\alpha
%  F \, (\phi \alpha)} (\pi_2 \rho)\\
%\end{array}\]
%since
Equation~\ref{thm:demotion-objects} ensures that $A = \mathit{in}_H :
H(\mu H) \Rightarrow \mu H = \setsem{\phi;\emptyset \vdash \Nat^\alpha
  F \,(\phi\alpha)}(\pi_1\rho)$,
%\[\begin{array}{lll}
%A \; = \; \mathit{in}_H & : & H(\mu H) \Rightarrow \mu H\\
%& = & \setsem{\emptyset; \phi, \alpha \vdash F}[\phi := \mu
%  \setsem{\emptyset; \phi, \alpha \vdash F}] \Rightarrow \mu
%  \setsem{\emptyset; \phi, \alpha \vdash F}\\  
%& = & \setsem{\emptyset; \phi, \alpha \vdash F}(\pi_1\rho)
%  \Rightarrow \setsem{\emptyset; \phi,\alpha \vdash \phi\alpha}(\pi_1 \rho)\\
%& = & \setsem{\phi; \alpha \vdash F}(\pi_1\rho)
%  \Rightarrow \setsem{\phi;\alpha \vdash \phi\alpha}(\pi_1
%  \rho) \hspace*{0.25in} {\color{red} \mbox{Daniel's trick; now a theorem}}\\
%& = & \setsem{\phi;\emptyset \vdash \Nat^\alpha F
%    \,(\phi\alpha)}(\pi_1\rho) 
%\end{array}\]
and Equation~\ref{thm:demotion-objects} and Lemma~\ref{lem:graph}
together give that
\[\begin{array}{lll}
(A,B) \,=\, (\mathit{in}_H,B) & \in & \relsem{\phi;\emptyset\vdash
  \Nat^\alpha F\, (\phi\alpha)}\rho\\
% & = & \lambda A. \relsem{\phi;\alpha\vdash F}\rho[\alpha := A]
%\Rightarrow \lambda A. (\rho\phi) A\\
& = & \lambda A. \relsem{\phi;\alpha\vdash F}[\phi :=
  \graph{\textit{fold}_H\, B}][\alpha := A] \Rightarrow 
 \graph{\textit{fold}_H\, B} \\ 
%& = & \lambda A. \relsem{\emptyset;\phi,\alpha\vdash F}[\phi :=
%  \graph{\textit{fold}_H\, B}][\alpha := A] \Rightarrow 
%\graph{\textit{fold}_H\, B}  \hspace*{0.25in} {\color{red}
%  \mbox{Daniel's trick; now a theorem}}\\  
& = & \relsem{\emptyset;\phi,\alpha\vdash F}
  \graph{\textit{fold}_H\, B} \Rightarrow \graph{\textit{fold}_H\,
    B}\\
  & = & \graph{\setsem{\emptyset;\phi,\alpha\vdash F}
    \,(\mathit{fold}_H\,B)} \Rightarrow \graph{\textit{fold}_H\, B}\\
%\hspace*{0.25in} {\color{red} \mbox{Graph Lemma}}\\
  & = & \graph{\mathit{map}_H \,(\mathit{fold}_H\,B)} \Rightarrow
\graph{\textit{fold}_H\, B}\\
\end{array}\]
since if $(x,y) \in \graph{\mathit{map}_H \,(\mathit{fold}_H\,B)}$,
%i.e., if $\mathit{map}_H \,(\mathit{fold}_H\,B) \, x = y$,
then $\textit{fold}_H\, B\, (\mathit{in}_H\,x) = B\,y = B\,
(\mathit{map}_H \,(\mathit{fold}_H\,B) \, x)$ by the definition of
$\mathit{fold}_H$ as a (indeed, the unique) morphism from
$\mathit{in}_H$ to $B$.  Thus, $(G \,(\pi_1 \rho)\, A, G\, (\pi_2
\rho)\, B) \in \graph{\textit{fold}_H\, B}$, i.e., $\textit{fold}_H \,
B \, (G\, (\pi_1 \rho) \, \mathit{in}_H) = G\,(\pi_2 \rho)\,B$.  But
since $\phi$ is the only free variable in $G$, this simplifies to
$\textit{fold}_H\, B \, (G\, \mu H\, \mathit{in}_H) =
G\,\setsem{\emptyset;\alpha\vdash K}\,B$. 
\end{proof}

\vspace*{-0.1in}

As in~\cite{jg10}, replacing $\onet$ with any term $\emptyset;\alpha
\vdash c$ generalizes
%with $\setsem{\emptyset;\alpha \vdash c}\rho = C$
Theorem~\ref{thm:short-cut-nested} to deliver more general a free
theorem whose conclusion is $\textit{fold}_{H}\, B \; \circ \; G\; \mu
H \; in_{H} = G \,\setsem{\emptyset;\alpha \vdash K}\, B$.

\section{Conclusion and Directions for Future Work}\label{sec:conclusion}

In this paper we have constructed a parametric model for a calculus
providing primitives for constructing nested types directly as
fixpoints rather than representing them via their Chuch encodings. We
have also used the Abstraction Theorem for this model to prove free
theorems in the presence of nested types. This was not possible
before~\cite{jp19} because the nested types definable in our syntax
were not previously known to have well-defined interpretations in
locally finitely presentable categories like $\set$ and $\rel$. No
calculus for terms of these types existed, either. The key to
obtaining our parametric model is the careful threading of
functoriality and its accompanying naturality conditions throughout
its construction.

All recursion in our calculus must be expressed using
folds. Expressivity of standard folds for nested types has long been
known to be a vexing issue; indeed, such folds can only express
computations that return natural transformations. Of course our system
inherits the constraints standard folds for nested types impose, and
thus is perhaps not as expressive as we would ideally like. But since
this expressivity issue is orthogonal to constructing parametric
models in the presence of nested types built from primitives, and
since it would detreact from the conceptual clarity of our
presentation, we have deliberately chosen not to address it in this
paper. (We do discuss some ideas for increasing expressivity in the
next paragraph.) Despite the limitations of standard folds, our
results provide a blueprint for deriving principled parametricity
results for practical programming languages, such as Haskell and Agda,
that directly construct nested types as fixpoints.

Many efforts to strengthen standard folds have been made. Most
prominent among these are generalized folds~\cite{bp99},
%and Mendler iterators~\cite{amu05}.
whcih are used to express computations whose results are not natural
transformations. Generalized folds are shown in~\cite{jg10} to be
equivalent to standard folds in the presence of right Kan extensions,
which suggests extending our term calculus with an object-level right
Kan extension construct. It should be possible to interpret such a
construct in $\set$ and $\rel$, {\color{blue} since these categories
  have all(?) limits}. Dually, it was shown in~\cite{jg08} that GADTs
can be represented using and object-level {\em left} Kan extension
construct, and in~\cite{jp19} that augmenting the types in our
calculus with a carefully designed such construct preserves the
cocontinuity needed for GADTs to have well-defined interpretations in
$\set$ and $\rel$. These observations suggest natural directions for
future work. Another possibility is to add term-level fixpoints, as
in~\cite{pit00}, to our calculus. However, this would require the
categories interpreting types to be not just locally finitely
presentable categories, but to support some kind of domain structure
as well. Unfortunately, $\omega$-CPO, the most obvious category of
domains to replace $\set$, {\color{blue} is not locally finitely
  presentable.}

The constructions of the present paper should all be possible in any
locally presentable cartesian closed category (lpccc) $\cal C$ whose
category of (abstract) relations, obtained by pullback as
in~\cite{jac99}, is also a lpccc and is appropriately fibred over
$C$. More specifically, it should be possible to extend the results
presented here along the lines of~\cite{gjfor15}. This would give a
framework based on ``locally presentable fibrations'', for some
appropriate definition thereof, for constructing parametric models for
calculi supporting nested types built from primitives.

\vspace*{0.1in}

\noindent

%Won't print in anonymous mode.
\begin{acks}
We thank Andrew Polonsky for valuable feedback and fruitful
discussions. Supported by NSF awards CCR-1906388 and 1420175.
\end{acks}


\bibliography{references}

\end{document}


  
fixpoints at term level ala Pitts:

I'm not sure if we actually want to do this or not (because not all
fixpoints will necessarily exist, and this will have consequences, and
because of the restriction to strict functions it will entail and the
fact that I haven't yet thought about how those will thread through
our calculus), but another idea we might consider is adding fixpoints
at the term level. This is done for System F in Wadler's section
7. [It is also done in Pitts' Parametric Polymorphism and Operational
  Equivalence paper, where the term introduction rule is SEE SOURCE

%Gamma \vdash F : \tau \to \tau
%-------------------------------
%Gamma \vdash fix (F) : tau

But of course the model there is entirely operational rather than
categorical.]


On the other hand, check out the last paragraph of Wadler's paper, which
suggests that (at least in 1989) there was some interest in calculi like
ours:

"The desire to derive theorems from types therefore suggests that it would be
valuable to explore programming languages that prohibit recursion or only
allow its restricted use. In theory, this is well understood; we have already
noted that any computable function that is provably total in second-order
Peano arithmetic can be defined in the pure polymorphic lambda calculus,
without using the fixpoint as a primitive. However, practical languages based
on this notion remain terra incognita."

Of course we are not working in System F, but perhaps the same desire applies
here, and we do get the same kinds of theorems as Wadler, many just as a
consequence of our interpretation, without invoking parametricity.

In any case, I guess Wadler could be cited as justifying restricting
our attention to our calculus since it could be seen as a way to have
a practical language that ensures that all programs are
terminating. This has been considered in the past, e.g., in ADL,
Charity, PolyP.


TERMINATING RECURSION
Another possible solution to the expressivity issue is extend folds to
track of deep structure of nested types as in~\cite{jp19}, so that
specialization of fixpoint can be computed as fixpoint of
specialization.

Add more polymorphism (all foralls), even though most free theorems
only use one level (or maybe two, like short cut).
