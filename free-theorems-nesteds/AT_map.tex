% For double-blind review submission, w/o CCS and ACM Reference (max
% submission space)
\documentclass[acmsmall,review,anonymous]{acmart}
\settopmatter{printfolios=true,printccs=false,printacmref=false}
%% For double-blind review submission, w/ CCS and ACM Reference
%\documentclass[acmsmall,review,anonymous]{acmart}\settopmatter{printfolios=true}
%% For single-blind review submission, w/o CCS and ACM Reference (max submission space)
%\documentclass[acmsmall,review]{acmart}\settopmatter{printfolios=true,printccs=false,printacmref=false}
%% For single-blind review submission, w/ CCS and ACM Reference
%\documentclass[acmsmall,review]{acmart}\settopmatter{printfolios=true}
%% For final camera-ready submission, w/ required CCS and ACM Reference
%\documentclass[acmsmall]{acmart}\settopmatter{}


%% Journal information
%% Supplied to authors by publisher for camera-ready submission;
%% use defaults for review submission.
\acmJournal{PACMPL}
\acmVolume{1}
\acmNumber{POPL} % CONF = POPL or ICFP or OOPSLA
\acmArticle{1}
\acmYear{2020}
\acmMonth{1}
\acmDOI{} % \acmDOI{10.1145/nnnnnnn.nnnnnnn}
\startPage{1}

%% Copyright information
%% Supplied to authors (based on authors' rights management selection;
%% see authors.acm.org) by publisher for camera-ready submission;
%% use 'none' for review submission.
\setcopyright{none}
%\setcopyright{acmcopyright}
%\setcopyright{acmlicensed}
%\setcopyright{rightsretained}
%\copyrightyear{2018}           %% If different from \acmYear

%% Bibliography style
\bibliographystyle{ACM-Reference-Format}
%% Citation style
%% Note: author/year citations are required for papers published as an
%% issue of PACMPL.
\citestyle{acmauthoryear}   %% For author/year citations
%\citestyle{acmnumeric}

%%%%%%%%%%%%%%%%%%%%%%%%%%%%%%%%%%%%%%%%%%%%%%%%%%%%%%%%%%%%%%%%%%%%%%
%% Note: Authors migrating a paper from PACMPL format to traditional
%% SIGPLAN proceedings format must update the '\documentclass' and
%% topmatter commands above; see 'acmart-sigplanproc-template.tex'.
%%%%%%%%%%%%%%%%%%%%%%%%%%%%%%%%%%%%%%%%%%%%%%%%%%%%%%%%%%%%%%%%%%%%%%



\usepackage[utf8]{inputenc}
\usepackage{ccicons}
\usepackage{verbatim}

\usepackage{amsmath}
\usepackage{amsthm}
\usepackage{amscd}
%\usepackage{MnSymbol}
\usepackage{xcolor}

\usepackage{bbold}
\usepackage{url}
\usepackage{upgreek}
%\usepackage{stmaryrd}

\usepackage{lipsum}
\usepackage{tikz-cd}
\usetikzlibrary{cd}
\usetikzlibrary{calc}
\usetikzlibrary{arrows}

\usepackage{bussproofs}
\EnableBpAbbreviations

\DeclareMathAlphabet{\mathpzc}{OT1}{pzc}{m}{it}

%\usepackage[amsmath]{ntheorem}

\newcommand{\lam}{\lambda}
\newcommand{\eps}{\varepsilon}
\newcommand{\ups}{\upsilon}
\newcommand{\mcB}{\mathcal{B}}
\newcommand{\mcD}{\mathcal{D}}
\newcommand{\mcE}{\mathcal{E}}
\newcommand{\mcF}{\mathcal{F}}
\newcommand{\mcP}{\mathcal{P}}
\newcommand{\mcI}{\mathcal{I}}
\newcommand{\mcJ}{\mathcal{J}}
\newcommand{\mcK}{\mathcal{K}}
\newcommand{\mcL}{\mathcal{L}}
\newcommand{\WW}{\mathcal{W}}

\newcommand{\ex}{\mcE_x}
\newcommand{\ey}{\mcE_y}
\newcommand{\bzero}{\boldsymbol{0}}
\newcommand{\bone}{{\boldsymbol{1}}}
\newcommand{\tB}{{\bone_\mcB}}
\newcommand{\tE}{{\bone_\mcE}}
\newcommand{\bt}{\mathbf{t}}
\newcommand{\bp}{\mathbf{p}}
\newcommand{\bsig}{\mathbf{\Sigma}}
\newcommand{\bpi}{\boldsymbol{\pi}}
\newcommand{\Empty}{\mathtt{Empty}}
\newcommand{\truthf}{\mathtt{t}}
\newcommand{\id}{id}
\newcommand{\coo}{\mathtt{coo\ }}
\newcommand{\mcC}{\mathcal{C}}
\newcommand{\Rec}{\mathpzc{Rec}}
\newcommand{\types}{\mathcal{T}}

%\newcommand{\Homrel}{\mathsf{Hom_{Rel}}}
\newcommand{\HomoCPOR}{\mathsf{Hom_{\oCPOR}}}

%\newcommand{\semof}[1]{\llbracket{#1}\rrbracket^\rel}
\newcommand{\sem}[1]{\llbracket{#1}\rrbracket}
\newcommand{\setsem}[1]{\llbracket{#1}\rrbracket^\set}
\newcommand{\relsem}[1]{\llbracket{#1}\rrbracket^\rel}
\newcommand{\dsem}[1]{\llbracket{#1}\rrbracket^{\mathsf D}}
\newcommand{\setenv}{\mathsf{SetEnv}}
\newcommand{\relenv}{\mathsf{RelEnv}}
\newcommand{\oCPOenv}{\mathsf{SetEnv}}
\newcommand{\oCPORenv}{\mathsf{RelEnv}}
\newcommand{\oCPOsem}[1]{\llbracket{#1}\rrbracket^{\set}}
\newcommand{\oCPORsem}[1]{\llbracket{#1}\rrbracket^{\rel}}
\newcommand{\denv}{\mathsf{DEnv}}

\newcommand{\rel}{\mathsf{Rel}}
\newcommand{\setof}[1]{\{#1\}}
\newcommand{\letin}[1]{\texttt{let }#1\texttt{ in }}
\newcommand{\comp}[1]{{\{#1\}}}
\newcommand{\bcomp}[1]{\{\![#1]\!\}}
\newcommand{\beps}{\boldsymbol{\eps}}
%\newcommand{\B}{\mcB}
%\newcommand{\Bo}{{|\mcB|}}

\newcommand{\lmt}{\longmapsto}
\newcommand{\RA}{\Rightarrow}
\newcommand{\LA}{\Leftarrow}
\newcommand{\rras}{\rightrightarrows}
\newcommand{\colim}[2]{{{\underrightarrow{\lim}}_{#1}{#2}}}
\newcommand{\lift}[1]{{#1}\,{\hat{} \; \hat{}}}
\newcommand{\graph}[1]{\langle {#1} \rangle}

\newcommand{\carAT}{\mathsf{car}({\mathcal A}^T)}
\newcommand{\isoAto}{\mathsf{Iso}({\mcA^\to})}
\newcommand{\falg}{\mathsf{Alg}_F}
\newcommand{\CC}{\mathsf{Pres}(\mathcal{A})}
\newcommand{\PP}{\mathcal{P}}
\newcommand{\DD}{D_{(A,B,f)}}
\newcommand{\from}{\leftarrow}
\newcommand{\upset}[1]{{#1}{\uparrow}}
\newcommand{\smupset}[1]{{#1}\!\uparrow}

\newcommand{\Coo}{\mathpzc{Coo}}
\newcommand{\code}{\#}
\newcommand{\nat}{\mathpzc{Nat}}

\newcommand{\eq}{\; = \;}
\newcommand{\of}{\; : \;}
\newcommand{\df}{\; := \;}
\newcommand{\bnf}{\; ::= \;}

\newcommand{\zmap}[1]{{\!{\between\!\!}_{#1}\!}}
\newcommand{\bSet}{\mathbf{Set}}

\newcommand{\dom}{\mathsf{dom}}
\newcommand{\cod}{\mathsf{cod}}
\newcommand{\adjoint}[2]{\mathrel{\mathop{\leftrightarrows}^{#1}_{#2}}}
\newcommand{\isofunc}{\mathpzc{Iso}}
\newcommand{\ebang}{{\eta_!}}
\newcommand{\lras}{\leftrightarrows}
\newcommand{\rlas}{\rightleftarrows}
\newcommand{\then}{\quad\Longrightarrow\quad}
\newcommand{\hookup}{\hookrightarrow}

\newcommand{\spanme}[5]{\begin{CD} #1 @<#2<< #3 @>#4>> #5 \end{CD}}
\newcommand{\spanm}[3]{\begin{CD} #1 @>#2>> #3\end{CD}}
\newcommand{\pushout}{\textsf{Pushout}}
\newcommand{\mospace}{\qquad\qquad\!\!\!\!}

\newcommand{\natur}[2]{#1 \propto #2}

\newcommand{\Tree}{\mathsf{Tree}\,}
\newcommand{\GRose}{\mathsf{GRose}\,}
\newcommand{\List}{\mathsf{List}\,}
\newcommand{\PTree}{\mathsf{PTree}\,}
\newcommand{\Bush}{\mathsf{Bush}\,}
\newcommand{\Forest}{\mathsf{Forest}\,}
\newcommand{\Lam}{\mathsf{Lam}\,}
\newcommand{\LamES}{\mathsf{Lam}^+}
\newcommand{\Expr}{\mathsf{Expr}\,}

\newcommand{\ListNil}{\mathsf{Nil}}
\newcommand{\ListCons}{\mathsf{Cons}}
\newcommand{\LamVar}{\mathsf{Var}}
\newcommand{\LamApp}{\mathsf{App}}
\newcommand{\LamAbs}{\mathsf{Abs}}
\newcommand{\LamSub}{\mathsf{Sub}}
\newcommand{\ExprConst}{\mathsf{Const}}
\newcommand{\ExprPair}{\mathsf{Pair}}
\newcommand{\ExprProj}{\mathsf{Proj}}
\newcommand{\ExprAbs}{\mathsf{Abs}}
\newcommand{\ExprApp}{\mathsf{App}}
\newcommand{\Ptree}{\mathsf{Ptree}}

\newcommand{\kinds}{\mathpzc{K}}
\newcommand{\tvars}{\mathbb{T}}
\newcommand{\fvars}{\mathbb{F}}
\newcommand{\consts}{\mathpzc{C}}
\newcommand{\Lan}{\mathsf{Lan}}
\newcommand{\zerot}{\mathbb{0}}
\newcommand{\onet}{\mathbb{1}}
\newcommand{\bool}{\mathbb{2}}
\renewcommand{\nat}{\mathbb{N}}
%\newcommand{\semof}[1]{[\![#1]\!]}
%\newcommand{\setsem}[1]{\llbracket{#1}\rrbracket^\set}
\newcommand{\predsem}[1]{\llbracket{#1}\rrbracket^\pred}
%\newcommand{\todot}{\stackrel{.}{\to}}
\newcommand{\todot}{\Rightarrow}
\newcommand{\bphi}{{\bm \phi}}

\newcommand{\bm}[1]{\boldsymbol{#1}}

\newcommand{\cL}{\mathcal{L}}
\newcommand{\T}{\mathcal{T}}
\newcommand{\Pos}{P\!}
%\newcommand{\Pos}{\mathcal{P}\!}
\newcommand{\Neg}{\mathcal{N}}
\newcommand{\Hf}{\mathcal{H}}
\newcommand{\V}{\mathbb{V}}
\newcommand{\I}{\mathcal{I}}
\newcommand{\Set}{\mathsf{Set}}
%\newcommand{\Nat}{\mathsf{Nat}}
\newcommand{\Homrel}{\mathsf{Hom_{Rel}}}
\newcommand{\CV}{\mathcal{CV}}
\newcommand{\lan}{\mathsf{Lan}}
\newcommand{\Id}{\mathit{Id}}
\newcommand{\mcA}{\mathcal{A}}
\newcommand{\inl}{\mathsf{inl}}
\newcommand{\inr}{\mathsf{inr}}
%\newcommand{\case}[3]{\mathsf{case}\,{#1}\,\mathsf{of}\,\{{#2};\,{#3}\}}
\newcommand{\tin}{\mathsf{in}}
\newcommand{\fold}{\mathsf{fold}}
\newcommand{\Eq}{\mathsf{Eq}}
\newcommand{\Hom}{\mathsf{Hom}}
\newcommand{\curry}{\mathsf{curry}}
\newcommand{\uncurry}{\mathsf{uncurry}}
\newcommand{\eval}{\mathsf{eval}}
\newcommand{\apply}{\mathsf{apply}}
\newcommand{\oCPO}{{\mathsf{Set}}}
\newcommand{\oCPOR}{{\mathsf{Rel}}}
\newcommand{\oCPORT}{{\mathsf{RT}}}

\newcommand{\ar}[1]{\##1}
\newcommand{\mcG}{\mathcal{G}}
\newcommand{\mcH}{\mathcal{H}}
\newcommand{\TV}{\mathpzc{V}}

\newcommand{\essim}[1]{\mathsf{EssIm}(#1)}
\newcommand{\hra}{\hookrightarrow}

\newcommand{\ol}[1]{\overline{#1}}
\newcommand{\ul}[1]{\underline{#1}}
\newcommand{\op}{\mathsf{op}}
\newcommand{\trige}{\trianglerighteq}
\newcommand{\trile}{\trianglelefteq}
\newcommand{\LFP}{\mathsf{LFP}}
\newcommand{\LAN}{\mathsf{Lan}}
%\newcommand{\Mu}{{\mu\hskip-4pt\mu}}
\newcommand{\Mu}{{\mu\hskip-5.5pt\mu}}
%\newcommand{\Mu}{\boldsymbol{\upmu}}
\newcommand{\Terms}{\mathpzc{Terms}}
\newcommand{\Ord}{\mathpzc{Ord}}
\newcommand{\Anote}[1]{{\color{blue} {#1}}}
\newcommand{\Pnote}[1]{{\color{red} {#1}}}

\newcommand{\greyout}[1]{{\color{gray} {#1}}}
\newcommand{\ora}[1]{\overrightarrow{#1}}

%\newcommand{\?}{{.\ }}
%\theoremheaderfont{\scshape}
%\theorembodyfont{\normalfont}
%\theoremseparator{.\ \ }
\newtheorem{thm}{Theorem}
\newtheorem{dfn}[thm]{Definition}
\newtheorem{prop}[thm]{Proposition}
\newtheorem{cor}[thm]{Corollary}
\newtheorem{lemma}[thm]{Lemma}
\newtheorem{rmk}[thm]{Remark}
\newtheorem{expl}[thm]{Example}
\newtheorem{notn}[thm]{Notation}
%\theoremstyle{nonumberplain}
%\theoremsymbol{\Box}


\theoremstyle{definition}
\newtheorem{exmpl}{Example}

\renewcommand{\greyout}[1]{} %{{\color{gray} {#1}}} -- toggle to remove greyed text

\newcommand{\emptyfun}{{[]}}
\newcommand{\cal}{\mathcal}
%\newcommand{\fold}{\mathit{fold}}
\newcommand{\F}{\mathcal{F}}
\renewcommand{\G}{\mathcal{G}}
\newcommand{\N}{\mathcal{N}}
\newcommand{\E}{\mathcal{E}}
\newcommand{\B}{\mathcal{B}}
\renewcommand{\P}{\mathcal{A}}
\newcommand{\pred}{\mathsf{Fam}}
\newcommand{\env}{\mathsf{Env}}
\newcommand{\set}{\mathsf{Set}}
\renewcommand{\S}{\mathcal S}
\renewcommand{\C}{\mathcal{C}}
\newcommand{\D}{\mathcal{D}}
\newcommand{\A}{\mathcal{A}}
\renewcommand{\id}{\mathit{id}}
\newcommand{\map}{\mathsf{map}}
\newcommand{\pid}{\underline{\mathit{id}}}
\newcommand{\pcirc}{\,\underline{\circ}\,}
\newcommand{\pzero}{\underline{0}}
\newcommand{\pone}{\underline{1}}
\newcommand{\psum}{\,\underline{+}\,}
\newcommand{\pinl}{\underline{\mathit{inL}}\,}
\newcommand{\pinr}{\underline{\mathit{inR}}\,}
\newcommand{\ptimes}{\,\underline{\times}\,}
\newcommand{\ppi}{\underline{\pi_1}}
\newcommand{\pppi}{\underline{\pi_2}}
\newcommand{\pmu}{\underline{\mu}}
\newcommand{\semmap}{\mathit{map}}
\newcommand{\subst}{\mathit{subst}}

\newcommand{\tb}[1]{~~ \mbox{#1} ~~}
\newcommand{\listt}[1]{(\mu \phi. \lambda \beta . \onet + \beta \times
  \phi \beta) #1} 
\newcommand{\filtype}{\Nat^\emptyset 
 (\Nat^\emptyset \, \alpha \, \mathit{Bool})\, (\Nat^\emptyset 
  (List \, \alpha) \, (List \, \alpha))} 
\newcommand{\filtypeGRose}{\Nat^\emptyset 
 (\Nat^\emptyset \, \alpha \, \mathit{Bool})\, (\Nat^\emptyset 
  (\mathit{GRose}\,\psi \, \alpha) \, (\mathit{GRose}\,\psi \, (\alpha
  + \onet)))} 
\newcommand{\maplist}{\mathit{map}_{\lambda A. \setsem{\emptyset; \alpha
      \vdash \mathit{List} \, \alpha} \rho[\alpha := A]}} 
\newcommand{\PLeaves}{\mathsf{PLeaves}}
\newcommand{\swap}{\mathsf{swap}}
\newcommand{\reverse}{\mathsf{reverse}}
\newcommand{\Bcons}{\mathit{Bcons}}
\newcommand{\Bnil}{\mathit{Bnil}}

\title[Abstraction theorem (with new Nat)]{Abstraction theorem (with new Nat)}
 %% [Short Title] is optional; when present,
                         %% will be used in header instead of Full
                         %% Title.
%\titlenote{with title note}             %% \titlenote is optional;
                                        %% can be repeated if necessary;
                                        %% contents suppressed with 'anonymous'
%\subtitle{Subtitle}                     %% \subtitle is optional
%\subtitlenote{with subtitle note}       %% \subtitlenote is optional;
                                        %% can be repeated if necessary;
                                        %% contents suppressed with 'anonymous'


%% Author information
%% Contents and number of authors suppressed with 'anonymous'.
%% Each author should be introduced by \author, followed by
%% \authornote (optional), \orcid (optional), \affiliation, and
%% \email.
%% An author may have multiple affiliations and/or emails; repeat the
%% appropriate command.
%% Many elements are not rendered, but should be provided for metadata
%% extraction tools.

%% Author with single affiliation.
\author{Patricia Johann, Enrico Ghiorzi, and Daniel Jeffries}
%\authornote{with author1 note}          %% \authornote is optional;
%                                        %% can be repeated if necessary
%\orcid{nnnn-nnnn-nnnn-nnnn}             %% \orcid is optional
\affiliation{
%  \position{Position1}
%  \department{Department1}              %% \department is recommended
  \institution{Appalachian State University}            %% \institution is required
%  \streetaddress{Street1 Address1}
%  \city{City1}
%  \state{State1}
%  \postcode{Post-Code1}
%  \country{Country1}                    %% \country is recommended
}
\email{johannp@appstate.edu, ghiorzie@appstate.edu, jeffriesd@appstate.edu}          %% \email is recommended


\begin{document}

%\begin{CCSXML}
%<ccs2012>
%<concept>
%<concept_id>10011007.10011006.10011008</concept_id>
%<concept_desc>Software and its engineering~General programming languages</concept_desc>
%<concept_significance>500</concept_significance>
%</concept>
%<concept>
%<concept_id>10003456.10003457.10003521.10003525</concept_id>
%<concept_desc>Social and professional topics~History of programming languages</concept_desc>
%<concept_significance>300</concept_significance>
%</concept>
%</ccs2012>
%\end{CCSXML}
%
%\ccsdesc[500]{Software and its engineering~General programming languages}
%\ccsdesc[300]{Social and professional topics~History of programming languages}
%% End of generated code


%% Keywords
%% comma separated list
%\keywords{keyword1, keyword2, keyword3}  %% \keywords is optional


\maketitle

To go beyond naturality and get {\em all} consequences of
parametricity, we prove an Abstraction Theorem for our calculus. As
usual for such theorems, we prove a more general result in
Theorem~\ref{thm:at-gen} that handles possibly open terms. We then
recover the Abstraction Theorem (Theorem~\ref{thm:abstraction}) as the
special case Theorem~\ref{thm:at-gen} for closed terms of closed type.

\begin{thm}\label{thm:at-gen}
Every well-formed term $\Gamma;\Phi~|~\Delta \vdash t : \tau$ induces
a natural transformation from $\sem{\Gamma; \Phi \vdash \Delta}$ to
$\sem{\Gamma; \Phi \vdash \tau}$, i.e., a triple of natural
transformations 
\[(\setsem{\Gamma;\Phi~|~\Delta \vdash t : \tau},\,
\setsem{\Gamma;\Phi~|~\Delta \vdash t : \tau},\,
\relsem{\Gamma;\Phi~|~\Delta \vdash t : \tau})\]
where
\[\begin{array}{lll}
\setsem{\Gamma;\Phi~|~\Delta \vdash t : \tau} & : & \setsem{\Gamma;
  \Phi \vdash \Delta} \to \setsem{\Gamma; \Phi \vdash \tau}
\end{array}\]
has as its component at $\rho : \setenv$ a morphism
\[\begin{array}{lll}
\setsem{\Gamma;\Phi~|~\Delta \vdash t : \tau}\rho & : & \setsem{\Gamma;
  \Phi \vdash \Delta}\rho \to \setsem{\Gamma; \Phi \vdash \tau}\rho
\end{array}\]
in $\set$,
\[\begin{array}{lll}
\relsem{\Gamma;\Phi~|~\Delta \vdash t : \tau} & : & \relsem{\Gamma;
  \Phi \vdash \Delta} \to \relsem{\Gamma; \Phi \vdash \tau}
\end{array}\]
has as its component at $\rho : \relenv$ a morphism
\[\begin{array}{lll}
\relsem{\Gamma;\Phi~|~\Delta \vdash t : \tau}\rho & : & \relsem{\Gamma;
  \Phi \vdash \Delta}\rho \to \relsem{\Gamma; \Phi \vdash \tau}\rho
\end{array}\]
in $\rel$,
and, for all $\rho : \relenv$,
\begin{equation}\label{eq:projs}
\relsem{\Gamma;\Phi~|~\Delta \vdash t : \tau}\rho =
(\setsem{\Gamma;\Phi~|~\Delta \vdash t : \tau}(\pi_1 \rho),\,
\setsem{\Gamma;\Phi~|~\Delta \vdash t : \tau}(\pi_2 \rho))
\end{equation}
\end{thm}

\begin{proof}
By structural induction on the type judgment for $t$. The only
interesting cases are the cases for abstraction, application, $\map$,
$\tin$, and $\fold$ so we omit the others. The challenging lies in
showing that the set and relational interpretations of each type
judgment are natural transformations (also satisfying the appropriate
{\color{blue} equality preservation} condition in the case of the set
interpretations). We give proofs for the set interpretations only;
those for the relational interpretations are similar.  The proof that
Equation~\ref{eq:projs} holds in each case is by direct calculation,
using the facts that projections are surjective and that the set and
relational interpretations are defined ``in parallel'', i.e., are
fibred.
\begin{itemize}
\item
{\color{blue} CHANGED!}
  $\underline{\Gamma;\Phi \,|\, \Delta \vdash L_{\overline{\alpha}}
  x.t : \Nat^{\overline{\alpha}} \,F \,G}$ \;
  To see that $\setsem{\Gamma;\Phi
    \,|\, \Delta \vdash L_{\overline{\alpha}} x.t :
    \Nat^{\overline{\alpha}} \,F \,G}$ is a natural transformation
  from $\setsem{\Gamma;\Phi \vdash \Delta}$ to
  $\setsem{\Gamma;\Phi \vdash \Nat^{\overline{\alpha}} \,F \,G}$
  we need show that, for every $\rho : \setenv$,
  $\setsem{\Gamma;\Phi \,|\, \Delta \vdash L_{\overline{\alpha}}
    x.t : \Nat^{\overline{\alpha}} \,F \,G}\rho$ is a morphism in
  $\set$ from $\setsem{\Gamma; \Phi \vdash \Delta}\rho$ to
  $\setsem{\Gamma; \Phi \vdash \Nat^{\overline{\alpha}} \,F
    \,G}\rho$, and that such family of morphisms is natural.
  First, we need to show that, for each $\ol{A : \set}$ and each $d :
  \setsem{\Gamma;\Phi \vdash \Delta}\rho = \setsem{\Gamma;
    \Phi, \ol{\alpha} \vdash \Delta}\rho[\overline{\alpha := A}]$, we have
  $(\setsem{\Gamma;\Phi \,|\, \Delta \vdash L_{\overline{\alpha}}
    x.t : \Nat^{\overline{\alpha}} \,F \,G}\rho\,d)_{\ol{A}}$ $:
  \setsem{\Gamma; \ol{\alpha} \vdash F}\rho[\overline{\alpha := A}]$
  $\to \setsem{\Gamma; \Phi, \ol{\alpha} \vdash G}\rho[\overline{\alpha :=
      A}]$, but this follows easily from the induction hypothesis.
That these maps comprise a natural transformation $\eta :
\setsem{\Gamma; \ol{\alpha} \vdash F}\rho[\overline{\alpha := \_}] \to
\setsem{\Gamma; \Phi, \ol{\alpha} \vdash G}\rho[\overline{\alpha := \_}]$ is
clear since $\eta_{\ol{A}} \, = \, \curry\,
(\setsem{\Gamma;\Phi,\overline{\alpha} \,|\, \Delta, x : F \vdash t:
  G}\rho[\overline{\alpha := A}])\,d$ is the component at $\ol{A}$ of
the partial specialization to $d$ of the natural transformation
$\setsem{\Gamma;\Phi,\overline{\alpha} \,|\, \Delta, x : F \vdash t:
  G}\rho[\overline{\alpha := \_}]$.  To see that the components of
$\eta$ also satisfy the additional condition needed for $\eta$ to
be in $\setsem{\Gamma;\Phi \vdash \Nat^{\overline{\alpha}} \,F
  \,G}\rho$, let $\overline{R : \rel(A, B)}$ and suppose $(u, v) \in
\relsem{\Gamma;\overline{\alpha} \vdash F} \Eq_{\rho}[\overline{\alpha
    := R}] = (\setsem{\Gamma;\overline{\alpha} \vdash F}
\rho[\overline{\alpha := A}], \setsem{\Gamma;\overline{\alpha} \vdash
  F} \rho[\overline{\alpha := B}])$. Then the induction hypothesis and
$(d,d) \in \relsem{\Gamma;\Phi \vdash \Delta} \Eq_\rho =
\relsem{\Gamma;\Phi \vdash \Delta} \Eq_\rho[\ol{\alpha := R}]$
ensure that
\[\begin{array}{ll}
& (\eta_{\ol{A}}u,\eta_{\ol{B}}v)\\
= & (\curry\, (\setsem{\Gamma;\Phi,\overline{\alpha} \,|\, \Delta, x : F
  \vdash t: G}\rho[\overline{\alpha := A}])\,d\,u, \curry\,
(\setsem{\Gamma;\Phi,\overline{\alpha} \,|\, \Delta, x : F \vdash t:
  G}\rho[\overline{\alpha := B}])\,d\,v)\\
= & \curry\, (\relsem{\Gamma;\Phi,\overline{\alpha} \,|\, \Delta, x : F
  \vdash t: G}\Eq_\rho[\overline{\alpha := R}])\,(d,d)\,(u,v)\\
: & \relsem{\Gamma;\Phi,\overline{\alpha} \vdash G}
\Eq_{\rho}[\overline{\alpha := R}]  
\end{array}\]
Moreover, to see that
$\setsem{\Gamma;\Phi \,|\, \Delta \vdash L_{\overline{\alpha}} x.t
: \Nat^{\overline{\alpha}} \,F \,G} \rho$
is natural in $\rho$,
let $f : \rho \to \rho'$ and consider the following computation
  \[\begin{array}{ll}
  &\setsem{\Gamma;\Phi \vdash \Nat^{\overline{\alpha}} \,F \,G}f
  \circ
  \setsem{\Gamma;\Phi \,|\, \Delta \vdash L_{\overline{\alpha}}
    x.t : \Nat^{\overline{\alpha}} \,F \,G}\rho \\
  = & \lambda d.\,
  \setsem{\Gamma;\Phi \vdash \Nat^{\overline{\alpha}} \,F \,G}f
  (\setsem{\Gamma;\Phi \,|\, \Delta \vdash L_{\overline{\alpha}}
    x.t : \Nat^{\overline{\alpha}} \,F \,G}\rho\, d) \\
  = & \lambda d.\,
  (\lambda \ol{A}.\, \setsem{\Gamma;\Phi,\ol{\alpha} \vdash G}f[\ol{\alpha := \id_A}])
  \circ
  \setsem{\Gamma;\Phi \,|\, \Delta \vdash L_{\overline{\alpha}}
    x.t : \Nat^{\overline{\alpha}} \,F \,G}\rho\, d \\
  = & \lambda d\, \ol{A}.\,
  \setsem{\Gamma;\Phi,\ol{\alpha} \vdash G}f[\ol{\alpha := \id_A}]
  \circ
  (\setsem{\Gamma;\Phi \,|\, \Delta \vdash L_{\overline{\alpha}}
    x.t : \Nat^{\overline{\alpha}} \,F \,G}\rho\, d)_{\ol{A}} \\
  = & \lambda d\, \ol{A}.\,
  \setsem{\Gamma;\Phi,\ol{\alpha} \vdash G}f[\ol{\alpha := \id_A}]
  \circ
  \setsem{\Gamma;\Phi, \ol{\alpha} \,|\, \Delta, x : F \vdash t : G}\rho[\ol{\alpha := A}]\, d \\
  = & \lambda d\, \ol{A}\, x.\,
  \setsem{\Gamma;\Phi,\ol{\alpha} \vdash G}f[\ol{\alpha := \id_A}]
  (\setsem{\Gamma;\Phi, \ol{\alpha} \,|\, \Delta, x : F \vdash t : G}\rho[\ol{\alpha := A}]\, d\, x) \\
  = & \lambda d\, \ol{A}\, x.\,
  \setsem{\Gamma;\Phi, \ol{\alpha} \,|\, \Delta, x : F \vdash t : G}\rho'[\ol{\alpha := A}]
  (\setsem{\Gamma;\Phi,\ol{\alpha} \vdash \Delta, x : F}f[\ol{\alpha := \id_A}] \, d\, x) \\
  = & \lambda d\, \ol{A}\, x.\,
  \setsem{\Gamma;\Phi, \ol{\alpha} \,|\, \Delta, x : F \vdash t : G}\rho'[\ol{\alpha := A}]
  (\setsem{\Gamma;\Phi \vdash \Delta}f\, d)
  (\setsem{\Gamma;\ol{\alpha} \vdash F}f[\ol{\alpha := \id_A}] x) \\
  = & \lambda d\, \ol{A}\, x.\,
  \setsem{\Gamma;\Phi, \ol{\alpha} \,|\, \Delta, x : F \vdash t : G}\rho'[\ol{\alpha := A}]
  (\setsem{\Gamma;\Phi \vdash \Delta}f\, d)\, x \\
  = & \lambda d\, \ol{A}.\,
  (\setsem{\Gamma;\Phi \,|\, \Delta \vdash L_{\overline{\alpha}} x.t : \Nat^{\overline{\alpha}} \,F \,G}\rho'
  (\setsem{\Gamma;\Phi \vdash \Delta}f\, d))_{\ol{A}} \\
  = &
  \setsem{\Gamma;\Phi \,|\, \Delta \vdash L_{\overline{\alpha}} x.t : \Nat^{\overline{\alpha}} \,F \,G}\rho'
  \circ
  \setsem{\Gamma;\Phi \vdash \Delta}f
  \end{array}\]
  where the sixth equality is by the naturality of the interpretation of
  $\Gamma;\Phi, \ol{\alpha} \,|\, \Delta, x : F \vdash t : G$,
  which is given by the induction hypothesis,
  the seventh equality is by currying,
  and the eight equality uses the functoriality of $\setsem{\Gamma;\ol{\alpha} \vdash F}$
  and the fact that the only functorial variables in $F$ are the $\ol{\alpha}$'s.

\item {\color{blue} CHANGED!}
$\underline{\Gamma;\Phi \,|\, \Delta \vdash t_{\overline
    \tau} s: G [\overline{\alpha := \tau}]}$\; To see that
$\setsem{\Gamma;\Phi \,|\, \Delta \vdash t_{\overline \tau} s: G
  [\overline{\alpha := \tau}]}$ is a natural transformation from
$\setsem{\Gamma; \Phi \vdash \Delta}$ to $\setsem{\Gamma;\Phi \vdash G
  [\overline{\alpha := \tau}]}$ we must show that, for every $\rho
: \setenv$, $\setsem{\Gamma;\Phi \,|\, \Delta \vdash t_{\overline
    \tau} s: G [\overline{\alpha := \tau}]}\rho$ is a morphism from
$\setsem{\Gamma; \Phi \vdash \Delta}\rho$ to $\setsem{\Gamma;\Phi
  \vdash G [\overline{\alpha := \tau}]}\rho$, and that this family of
morphisms is natural in $\rho$. Let $d : \setsem{\Gamma; \Phi \vdash
  \Delta}\rho$. Then
\[\begin{array}{ll}
  & \setsem{\Gamma;\Phi \,|\, \Delta \vdash t_{\overline \tau} s: G
  [\overline{\alpha := \tau}]}\,\rho\,d\\
= & (\eval \circ \langle (\setsem{\Gamma;\Phi \,|\, \Delta \vdash
  t : \Nat^{\overline{\alpha}} \,F \,G}\rho\;
\_)_{\overline{\setsem{\Gamma;\Phi \vdash \tau}\rho}},\,
\setsem{\Gamma;\Phi \,|\, \Delta \vdash s: F [\overline{\alpha :=
      \tau}]}\rho \rangle)\,d\\
= & \eval ((\setsem{\Gamma;\Phi \,|\, \Delta \vdash t :
  \Nat^{\overline{\alpha}} \,F \,G}\rho\;
\_)_{\overline{\setsem{\Gamma;\Phi \vdash \tau}\rho}} \,d,\,
\setsem{\Gamma;\Phi \,|\, \Delta \vdash s: F [\overline{\alpha :=
      \tau}]}\rho\, d)\\
= & \eval ((\setsem{\Gamma;\Phi \,|\, \Delta \vdash t :
  \Nat^{\overline{\alpha}} \,F \,G}\rho\;
d)_{\overline{\setsem{\Gamma;\Phi \vdash \tau}\rho}},\,
\setsem{\Gamma;\Phi \,|\, \Delta \vdash s: F [\overline{\alpha :=
      \tau}]}\rho\, d)\\
\end{array}\]
The induction hypothesis ensures that $(\setsem{\Gamma;\Phi \,|\,
  \Delta \vdash t : \Nat^{\overline{\alpha}} \,F \,G}\rho\;
d)_{\overline{\setsem{\Gamma;\Phi \vdash \tau}\rho}}$ has type
$\setsem{\Gamma; \ol{\alpha} \vdash F}\rho[\ol{\alpha :=
    \setsem{\Gamma;\Phi \vdash \tau}\rho}] \to \setsem{\Gamma;
  \Phi,\ol{\alpha} \vdash G}\rho[\ol{\alpha := \setsem{\Gamma;\Phi \vdash
      \tau}\rho}]$.  Since, in addition, $\setsem{\Gamma;\Phi \,|\,
  \Delta \vdash s: F [\overline{\alpha := \tau}]}\rho\, d :
\setsem{\Gamma; \Phi \vdash F[\ol{\alpha := \tau}]}\rho =
\setsem{\Gamma; \Phi, \ol{\alpha} \vdash F}\rho[\ol{\alpha :=
    \setsem{\Gamma;\Phi \vdash \tau}\rho}] = \setsem{\Gamma;
  \ol{\alpha} \vdash F}\rho[\ol{\alpha := \setsem{\Gamma;\Phi \vdash
      \tau}\rho}]$ by Equation~\ref{eq:subs-var},
we have that $\setsem{\Gamma;\Phi \,|\, \Delta
  \vdash t_{\overline \tau} s: G [\overline{\alpha :=
      \tau}]}\,\rho\,d : \setsem{\Gamma; \Phi,\ol{\alpha} \vdash
  G}\rho[\ol{\alpha := \setsem{\Gamma;\Phi \vdash \tau}\rho}] =
\setsem{\Gamma; \Phi \vdash G[\ol{\alpha := \tau}]}\rho$, as desired.

\vspace*{0.1in}

To see that the family of maps comprising $\setsem{\Gamma;\Phi \,|\,
  \Delta \vdash t_{\overline \tau} s: G [\overline{\alpha := \tau}]}$
is natural in $\rho$
we need to show that, if $f : \rho \to \rho'$ in $\setenv$, then the
following diagram commutes, where $g = \setsem{\Gamma;\Phi \,|\,
  \Delta \vdash t : \Nat^{\overline{\alpha}} \,F \,G}$ and $h =
\setsem{\Gamma;\Phi \,|\, \Delta \vdash s: F [\overline{\alpha :=
      \tau}]}$:
{\footnotesize
\[\begin{tikzcd}
\setsem{\Gamma;\Phi \vdash \Delta}\rho \ar[r, "{\setsem{\Gamma;\Phi
      \vdash \Delta}f}"] \ar[d, "{\langle g \rho, h \rho\rangle}"']
& \setsem{\Gamma;\Phi \vdash 
  \Delta}\rho' \ar[d, "{\langle g \rho', h \rho' \rangle}"]\\
\setsem{\Gamma;\Phi \vdash \Nat^{\overline{\alpha}} \,F \,G}\rho
\times \setsem{\Gamma;\Phi \vdash F [\overline{\alpha := \tau}]}\rho
\ar[d, "{\eval \circ ((-)_{\overline{\sem{\Gamma;\Phi\vdash \tau}\rho}} \times
    \id)}"']
\ar[r, bend left = 5, "{\setsem{\Gamma;\Phi \vdash
      \Nat^{\overline{\alpha}} \,F \,G}f\, \times\, \setsem{\Gamma;\Phi
      \vdash F [\overline{\alpha := \tau}]}f}"] &
\setsem{\Gamma;\Phi \vdash \Nat^{\overline{\alpha}} \,F \,G}\rho'
\times \setsem{\Gamma;\Phi \vdash F [\overline{\alpha := \tau}]}\rho'
\ar[d, "{\eval \circ ((-)_{\overline{\sem{\Gamma;\Phi\vdash
          \tau}\rho'}} \times \id)}"] \\
\setsem{\Gamma;\Phi \vdash G [\overline{\alpha := \tau}]}\rho
\ar[r, "{\setsem{\Gamma;\Phi \vdash G [\overline{\alpha := \tau}]}f}"']
&
\setsem{\Gamma;\Phi \vdash G [\overline{\alpha := \tau}]}\rho'
\end{tikzcd}\]}

\noindent
The top diagram commutes because $g$ and $h$ are natural in $\rho$ by
the induction hypothesis.
To see that the bottom diagram commutes,
we need to show that
$\setsem{\Gamma;\Phi \vdash G [\overline{\alpha := \tau}]}f
(\eta_{\overline{\sem{\Gamma;\Phi\vdash \tau}\rho}} x) =
(\setsem{\Gamma;\Phi \vdash \Nat^{\overline{\alpha}} \,F \,G} f\, \eta
)_{\overline{\sem{\Gamma;\Phi\vdash \tau}\rho'}}
(\setsem{\Gamma;\Phi \vdash F [\overline{\alpha := \tau}]}f x)$
holds for all $\eta \in \setsem{\Gamma;\Phi \vdash
  \Nat^{\overline{\alpha}} \,F \,G}\rho$ and $x \in
\setsem{\Gamma;\Phi \vdash F [\overline{\alpha := \tau}]}\rho$,
i.e.,
by remembering the following facts,
\begin{align*}
\setsem{\Gamma;\Phi \vdash F[\ol{\alpha := \tau}]}\rho
&= \setsem{\Gamma; \ol{\alpha} \vdash F}\rho[\ol{\alpha :=
    \setsem{\Gamma;\Phi \vdash \tau}\rho}] \\
\setsem{\Gamma;\Phi
  \vdash F[\ol{\alpha := \tau}]}f
&= \setsem{\Gamma; \ol{\alpha} \vdash F}
  \id_{\rho} [\ol{\alpha := \setsem{\Gamma;\Phi \vdash \tau}f}] \\
\setsem{\Gamma;\Phi \vdash G[\ol{\alpha := \tau}]}\rho
&= \setsem{\Gamma; \Phi,\ol{\alpha} \vdash G}\rho[\ol{\alpha :=
    \setsem{\Gamma;\Phi \vdash \tau}\rho}] \\
\setsem{\Gamma;\Phi \vdash G[\ol{\alpha := \tau}]}f
&= \setsem{\Gamma; \Phi, \ol{\alpha} \vdash
  G} f [\ol{\alpha := \setsem{\Gamma;\Phi \vdash \tau}f}]
\end{align*}
we need to show that
\begin{multline*}
\setsem{\Gamma;\Phi, \ol{\alpha} \vdash G} f[\overline{\alpha := \setsem{\Gamma;\Phi\vdash \tau} f }]
  \circ
\eta_{\overline{\setsem{\Gamma;\Phi\vdash \tau}\rho}} \\
=
\setsem{\Gamma;\Phi, \ol{\alpha} \vdash G} f[\overline{\alpha := \id_{\setsem{\Gamma;\Phi\vdash \tau}\rho'}}]
\circ
\eta_{\overline{\setsem{\Gamma;\Phi\vdash \tau}\rho'}}
\circ
\setsem{\Gamma; \ol{\alpha} \vdash F} \id_{\rho} [\ol{\alpha := \setsem{\Gamma;\Phi\vdash \tau}f}]
\end{multline*}
for all $\eta \in \setsem{\Gamma;\Phi \vdash
  \Nat^{\overline{\alpha}} \,F \,G}\rho$.
But this follows from the naturality of $\eta$, which ensures the commutativity of
{\footnotesize
\[\begin{tikzcd}[column sep = large]
\setsem{\Gamma; \ol{\alpha} \vdash F}\rho[\ol{\alpha :=
    \setsem{\Gamma;\Phi \vdash \tau}\rho}] \ar[r,
  "{\;\;\;\eta_{\ol{\setsem{\Gamma;\Phi \vdash \tau}\rho}}\;\;\; }"]
\ar[d, "{\setsem{\Gamma; \ol{\alpha} \vdash F} \id_{\rho} [\ol{\alpha := 
        \setsem{\Gamma;\Phi \vdash \tau}f}]}"']
& \setsem{\Gamma;
  \Phi,\ol{\alpha} \vdash G}\rho[\ol{\alpha := \setsem{\Gamma;\Phi \vdash
      \tau}\rho}]
\ar[d, "{\setsem{\Gamma; \Phi,\ol{\alpha} \vdash G} \id_{\rho} [\ol{\alpha := 
        \setsem{\Gamma;\Phi \vdash \tau}f}]}"]\\
\setsem{\Gamma; \ol{\alpha} \vdash F}\rho[\ol{\alpha :=
    \setsem{\Gamma;\Phi \vdash \tau}\rho'}] \ar[r,
  "{\eta_{\ol{\setsem{\Gamma;\Phi \vdash \tau}\rho'}} }"]
& \setsem{\Gamma; \Phi,\ol{\alpha} \vdash G}\rho[\ol{\alpha :=
    \setsem{\Gamma;\Phi \vdash \tau}\rho'}]
\end{tikzcd}\]}
and the observation that
$\setsem{\Gamma;\Phi, \ol{\alpha} \vdash G} f[\overline{\alpha := \setsem{\Gamma;\Phi\vdash \tau} f }]$
is equal to
\[
\setsem{\Gamma;\Phi, \ol{\alpha} \vdash G} f[\overline{\alpha := \id_{\setsem{\Gamma;\Phi\vdash \tau}\rho'}}]
\circ \setsem{\Gamma;\Phi, \ol{\alpha} \vdash G} \id_{\rho} [\overline{\alpha := \setsem{\Gamma;\Phi\vdash \tau}f}]
\]

\item {\color{blue} CHANGED!}
  $\underline{\Gamma;\Phi~|~\Delta \vdash
  \map^{\ol{F},\ol{G}}_H\, \ol{t} :
  \Nat^{\ol{\gamma}}\,H[\ol{\phi :=_{\ol{\beta}} F}]\,H[\ol{\phi
      :=_{\ol{\beta}} G}]}$\;
  To see that
  $\setsem{\Gamma; \Phi~|~\Delta \vdash \map^{\ol{F},\ol{G}}_H\, \ol{t}
  : \Nat^{\ol{\gamma}}\,H[\ol{\phi :=_{\ol{\beta}} F}]\,H[\ol{\phi
      :=_{\ol{\beta}} G}]}$ is a natural transformation from
  $\setsem{\Gamma;\Phi \vdash \Delta}$ to
  $\setsem{\Gamma;\Phi \vdash \Nat^{\ol{\gamma}}\,H[\ol{\phi :=_{\ol{\beta}} F}]\,H[\ol{\phi
      :=_{\ol{\beta}} G}]}$ we need to show that
\[
\setsem{\Gamma; \Phi~|~\Delta \vdash \map^{\ol{F},\ol{G}}_H\, \ol{t}
    : \Nat^{\ol{\gamma}}\,H[\ol{\phi :=_{\ol{\beta}} F}]\,H[\ol{\phi
        :=_{\ol{\beta}} G}]}\,\rho\,d
  : \setsem{\Gamma; \Phi \vdash
    \Nat^{\ol{\gamma}}\,H[\ol{\phi :=_{\ol{\beta}} F}]\,H[\ol{\phi
        :=_{\ol{\beta}} G}]} \rho
\]
  for all $\rho : \setenv$ and $d : \setsem{\Gamma;\Phi \vdash \Delta} \rho$,
  and that this family of morphisms is natural in $\rho$.
  For this, we first note that
  $\setsem{\Gamma ;\ol{\phi}, \ol{\gamma} \vdash H}$ is a functor from
  $\setenv$ to $\set$ and, for any $\ol B$, $\id_{\rho[\ol{\gamma := B}]}
  [\ol{\phi := \lambda \ol{A}. {(\setsem{\Gamma; \Phi~|~\Delta \vdash t : \Nat^{\ol{\beta},\ol{\gamma}}\,F\,G} \rho d)}_{\ol{A}\,\ol{B}}}]$
  is a
  morphism in $\setenv$ from \[\rho[\ol{\gamma := B}][\ol{\phi :=
      \lambda \ol{A}.\setsem{\Gamma; \ol{\gamma},\ol{\beta} \vdash
        F}\rho[\ol{\gamma := B}][\ol{\beta := A}]}]\] to
  \[\rho[\ol{\gamma := B}][\ol{\phi := \lambda \ol{A}.\setsem{\Gamma;
        \Phi, \ol{\gamma},\ol{\beta} \vdash G}\rho[\ol{\gamma :=
          B}][\ol{\beta := A}]}]\]
so that
\begin{multline*}
(\setsem{\Gamma; \Phi~|~\Delta \vdash
\map^{\ol{F},\ol{G}}_H\, \ol{t} :
\Nat^{\ol{\gamma}}\,H[\ol{\phi :=_{\ol{\beta}} F}]\,H[\ol{\phi
    :=_{\ol{\beta}} G}]}\,\rho\,d)_{\ol{B}} \\
= \setsem{\Gamma; \ol{\phi},\ol{\gamma} \vdash H}\id_{\rho[\ol{\gamma := B}]}
[\ol{\phi := \lambda \ol{A}. {(\setsem{\Gamma; \Phi~|~\Delta \vdash t : \Nat^{\ol{\beta},\ol{\gamma}}\,F\,G} \rho d)}_{\ol{A}\,\ol{B}}}]
\end{multline*}
which is indeed a morphism from
$\setsem{\Gamma ;\ol{\gamma} \vdash H[\ol{\phi := F}]}\rho[\ol{\gamma := B}]$
to 
$\setsem{\Gamma ;\Phi, \ol{\gamma} \vdash H[\ol{\phi := G}]}\rho[\ol{\gamma := B}]$.
This family of morphisms is natural in $\ol B$: if $\ol{f : B \to B'}$
then, writing $\eta$ for
\[
\setsem{\Gamma; \Phi~|~\Delta \vdash
\map^{\ol{F},\ol{G}}_H\, \ol{t} :
\Nat^{\ol{\gamma}}\,H[\ol{\phi :=_{\ol{\beta}} F}]\,H[\ol{\phi
    :=_{\ol{\beta}} G}]}\,\rho\,d
\]
the naturality of
$\setsem{\Gamma; \Phi~|~\Delta \vdash t : \Nat^{\ol{\beta},\ol{\gamma}}\,F\,G} \rho\, d$,
together with the fact that composition of environments is
computed componentwise, ensure that the following naturality diagram
for $\eta$ commutes:
{\footnotesize
\[\begin{tikzcd}
\setsem{\Gamma ;\ol{\gamma} \vdash H[\ol{\phi := F}]}\rho[\ol{\gamma
      := B}] \ar[r,
  "{\;\;\;\eta_{\ol{B}}\;\;\; }"]
\ar[d, "{\setsem{\Gamma ;\ol{\gamma} \vdash H[\ol{\phi :=
          F}]}\id_{\rho}[\ol{\gamma := f}]}"']
& \setsem{\Gamma ; \Phi, \ol{\gamma} \vdash H[\ol{\phi := G}]}\rho[\ol{\gamma
      := B}]
\ar[d, "{\setsem{\Gamma ; \Phi, \ol{\gamma} \vdash H[\ol{\phi :=
          G}]}\id_{\rho}[\ol{\gamma := f}]}"]\\
\setsem{\Gamma ;\ol{\gamma} \vdash H[\ol{\phi := F}]}\rho[\ol{\gamma
      := B'}] \ar[r,
  "{\eta_{\ol{B'}}}"]
& \setsem{\Gamma ; \Phi, \ol{\gamma} \vdash H[\ol{\phi := G}]}\rho[\ol{\gamma
      := B'}] 
\end{tikzcd}\]}
That, for all $\rho : \setenv$ and $d :
  \setsem{\Gamma;\Phi \vdash \Delta}\rho$,
$\eta$ satisfies the
additional condition needed for it to be in $\setsem{\Gamma;
  \Phi \vdash \Nat^{\ol{\gamma}}\,H[\ol{\phi :=_{\ol{\beta}}
      F}]\,H[\ol{\phi :=_{\ol{\beta}} G}]}\rho$, follows from the fact
that $\setsem{\Gamma; \Phi~|~\Delta \vdash t : \Nat^{\ol{\beta},\ol{\gamma}}\,F\,G} \rho d$
satisfies the extra condition needed for it to be
in its corresponding $\setsem{\Gamma; \Phi \vdash
  \Nat^{\ol{\beta},\ol{\gamma}}\,F\,G}\rho$.

For the naturality of
$\setsem{\Gamma; \Phi~|~\Delta \vdash \map^{\ol{F},\ol{G}}_H\, \ol{t}
  : \Nat^{\ol{\gamma}}\,H[\ol{\phi :=_{\ol{\beta}} F}]\,H[\ol{\phi
      :=_{\ol{\beta}} G}]}$,
consider $f : \rho \to \rho'$.
We need to prove that
\begin{multline*}
\setsem{\Gamma; \Phi \vdash
 \Nat^{\ol{\gamma}}\,H[\ol{\phi :=_{\ol{\beta}} F}]\,H[\ol{\phi
      :=_{\ol{\beta}} G}]} f
\circ
\setsem{\Gamma; \Phi~|~\Delta \vdash
\map^{\ol{F},\ol{G}}_H\, \ol{t} :
\Nat^{\ol{\gamma}}\,H[\ol{\phi :=_{\ol{\beta}} F}]\,H[\ol{\phi
    :=_{\ol{\beta}} G}]}\,\rho \\
= 
\setsem{\Gamma; \Phi~|~\Delta \vdash
\map^{\ol{F},\ol{G}}_H\, \ol{t} :
\Nat^{\ol{\gamma}}\,H[\ol{\phi :=_{\ol{\beta}} F}]\,H[\ol{\phi
    :=_{\ol{\beta}} G}]}\,\rho'
\circ
\setsem{\Gamma; \Phi \vdash \Delta}\, f
\end{multline*}
i.e., that
\begin{multline*}
\setsem{\Gamma; \Phi \vdash 
\Nat^{\ol{\gamma}}\,H[\ol{\phi :=_{\ol{\beta}} F}]\,H[\ol{\phi
      :=_{\ol{\beta}} G}]} f
(\setsem{\Gamma; \Phi~|~\Delta \vdash
\map^{\ol{F},\ol{G}}_H\, \ol{t} :
\Nat^{\ol{\gamma}}\,H[\ol{\phi :=_{\ol{\beta}} F}]\,H[\ol{\phi
    :=_{\ol{\beta}} G}]}\,\rho\,d) \\
=
\setsem{\Gamma; \Phi~|~\Delta \vdash
\map^{\ol{F},\ol{G}}_H\, \ol{t} :
\Nat^{\ol{\gamma}}\,H[\ol{\phi :=_{\ol{\beta}} F}]\,H[\ol{\phi
    :=_{\ol{\beta}} G}]}\,\rho'
(\setsem{\Gamma; \Phi \vdash \Delta}\, f\, d )
\end{multline*}
for any $d : \setsem{\Gamma; \Phi \vdash \Delta} \rho$.
That is shown by the following calculations,
\[
\begin{array}{ll}
&\setsem{\Gamma; \Phi \vdash 
\Nat^{\ol{\gamma}}\,H[\ol{\phi :=_{\ol{\beta}} F}]\,H[\ol{\phi
      :=_{\ol{\beta}} G}]} f
(\setsem{\Gamma; \Phi~|~\Delta \vdash
\map^{\ol{F},\ol{G}}_H\, \ol{t} :
\Nat^{\ol{\gamma}}\,H[\ol{\phi :=_{\ol{\beta}} F}]\,H[\ol{\phi
    :=_{\ol{\beta}} G}]}\,\rho\,d) \\
=&
\lambda \ol{B}.\,
\setsem{\Gamma; \Phi, \ol{\gamma} \vdash 
H[\ol{\phi :=_{\ol{\beta}} G}]} f[\ol{\gamma := \id_B}] \\
&\hspace{2em}\circ
(\setsem{\Gamma; \Phi~|~\Delta \vdash
\map^{\ol{F},\ol{G}}_H\, \ol{t} :
\Nat^{\ol{\gamma}}\,H[\ol{\phi :=_{\ol{\beta}} F}]\,H[\ol{\phi
    :=_{\ol{\beta}} G}]}\,\rho\,d)_{\ol{B}} \\
=&
\lambda \ol{B}.\,
\setsem{\Gamma; \ol{\phi}, \ol{\gamma} \vdash 
H} f[\ol{\gamma := \id_B}] [\ol{\phi := \lambda \ol{A}.\, \setsem{\Gamma; \Phi, \ol{\beta}, \ol{\gamma} \vdash G}f[\ol{\beta := \id_A}][\ol{\gamma := \id_B}]}] \\
&\hspace{2em}\circ
\setsem{\Gamma; \ol{\phi},\ol{\gamma} \vdash H}\id_{\rho[\ol{\gamma := B}]}
[\ol{\phi := \lambda \ol{A}. {(\setsem{\Gamma; \Phi~|~\Delta \vdash t : \Nat^{\ol{\beta},\ol{\gamma}}\,F\,G} \rho\, d)}_{\ol{A}\,\ol{B}}}] \\
=&
\lambda \ol{B}.\,
\setsem{\Gamma; \ol{\phi}, \ol{\gamma} \vdash H}
f[\ol{\gamma := \id_B}] \\
&\hspace{2em}[\ol{\phi := \lambda \ol{A}.\, \setsem{\Gamma; \Phi, \ol{\beta}, \ol{\gamma} \vdash G}f[\ol{\beta := \id_A}][\ol{\gamma := \id_B}]
\circ {(\setsem{\Gamma; \Phi~|~\Delta \vdash t : \Nat^{\ol{\beta},\ol{\gamma}}\,F\,G} \rho\, d)}_{\ol{A}\,\ol{B}}}] \\
=&
\lambda \ol{B}.\, 
\setsem{\Gamma; \ol{\phi},\ol{\gamma} \vdash H}\id_{\rho'[\ol{\gamma := B}]}
[\ol{\phi := \lambda \ol{A}. {(\setsem{\Gamma; \Phi \vdash \Nat^{\ol{\beta},\ol{\gamma}}\,F\,G} f (\setsem{\Gamma; \Phi~|~\Delta \vdash t : \Nat^{\ol{\beta},\ol{\gamma}}\,F\,G}\, \rho\, d ))}_{\ol{A}\,\ol{B}}}] \\
=&
\lambda \ol{B}.\, 
\setsem{\Gamma; \ol{\phi},\ol{\gamma} \vdash H}\id_{\rho'[\ol{\gamma := B}]}
[\ol{\phi := \lambda \ol{A}. {(\setsem{\Gamma; \Phi~|~\Delta \vdash t : \Nat^{\ol{\beta},\ol{\gamma}}\,F\,G} \rho' (\setsem{\Gamma; \Phi \vdash \Delta}\, f\, d ))}_{\ol{A}\,\ol{B}}}] \\
=&
\setsem{\Gamma; \Phi~|~\Delta \vdash
\map^{\ol{F},\ol{G}}_H\, \ol{t} :
\Nat^{\ol{\gamma}}\,H[\ol{\phi :=_{\ol{\beta}} F}]\,H[\ol{\phi
    :=_{\ol{\beta}} G}]}\,\rho'
(\setsem{\Gamma; \Phi \vdash \Delta}\, f\, d )
\end{array}
\]
where the third equality is given by composition of morphisms of environments
and the fifth equality is given by the naturality of $\Gamma; \Phi~|~\Delta \vdash t : \Nat^{\ol{\beta},\ol{\gamma}}\,F\,G$,
which we have by the induction hypothesis.

\item
$\underline{\Gamma;\emptyset \,|\, \emptyset \vdash \tin_H :
  Nat^{\ol{\beta},\ol{\gamma}} \, H[\phi := (\mu \phi.\lambda
    {\overline \alpha}.H){\overline \beta}][\ol{\alpha := \beta}]
  \;(\mu \phi.\lambda {\overline \alpha}.H){\overline \beta}}$\; To
  see that if $d : \setsem{\Gamma;\emptyset \vdash \emptyset} \rho$
  then $\setsem{\Gamma;\emptyset \,|\, \emptyset \vdash \tin_H :
    Nat^{\ol{\beta},\ol{\gamma}} \, H[\phi := (\mu \phi.\lambda
      {\overline \alpha}.H){\overline \beta}][\ol{\alpha := \beta}]
    \;(\mu \phi.\lambda {\overline \alpha}.H){\overline \beta}}\,
  \rho\,d$ is in\\ $\setsem{\Gamma;\emptyset \vdash
    Nat^{\ol{\beta},\ol{\gamma}} \, H[\phi := (\mu \phi.\lambda
      {\overline \alpha}.H){\overline \beta}][\ol{\alpha := \beta}]
    \;(\mu \phi.\lambda {\overline \alpha}.H){\overline \beta}}\,
  \rho$, we first note that, for all $\ol{B}$ and $\ol{C}$,
  $(\setsem{\Gamma;\emptyset \,|\, \emptyset \vdash \tin_H :
    Nat^{\ol{\beta},\ol{\gamma}} \, H[\phi := (\mu \phi.\lambda
      {\overline \alpha}.H){\overline \beta}][\ol{\alpha := \beta}]
    \;(\mu \phi.\lambda {\overline \alpha}.H){\overline \beta}}\,
  \rho\,d)_{\ol{B}\,\ol{C}}\, =\,
  (\mathit{in}_{T^\set_{H,\rho[\ol{\gamma := C}]}})_{\ol{B}}$ maps
  $\setsem{\Gamma;\ol{\beta},\ol{\gamma} \vdash H[\phi := (\mu
      \phi.\lambda {\overline \alpha}.H){\overline \beta}][\ol{\alpha
        := \beta}]}\rho[\ol{\beta := B}][\ol{\gamma := C}] =
  T^\set_{H,\rho[\ol{\gamma := C}]}\, (\mu T^\set_{H,\rho[\ol{\gamma
        := C}]}) \, \ol{B}$ to $\setsem{\Gamma;\ol{\beta},\ol{\gamma}
    \vdash (\mu \phi.\lambda {\overline \alpha}.H){\overline \beta}}
  \rho[\ol{\beta := B}][\ol{\gamma := C}] = (\mu
  T^\set_{H,\rho[\ol{\gamma := C}]}) \, \ol{B}$. Secondly, we observe
  that\\ $\setsem{\Gamma;\emptyset \,|\, \emptyset \vdash \tin_H :
    Nat^{\ol{\beta},\ol{\gamma}} \, H[\phi := (\mu \phi.\lambda
      {\overline \alpha}.H){\overline \beta}][\ol{\alpha := \beta}]
    \;(\mu \phi.\lambda {\overline \alpha}.H){\overline
      \beta}}\,\rho\,d = \lambda
  \ol{B}\,\ol{C}.\,(\mathit{in}_{T^\set_{H,\rho[\ol{\gamma :=
          C}]}})_{\ol{B}}$ is natural in $\ol{B}$ and $\ol{C}$, since
  naturality of $\mathit{in}$ with respect to its functor argument and
  naturality of $\mathit{in}_{T^\set_{H,\rho[\ol{\gamma := C'}]}}$ ensure
  that the following diagram commutes for all $\ol{f : B \to B'}$ and
  $\ol{g : C \to C'}$:
{\tiny
  \[\begin{tikzcd}[column sep=2.5in, row sep=0.75in]
T^\set_{H,\rho[\ol{\gamma := C}]}\, (\mu T^\set_{H,\rho[\ol{\gamma := C}]})\, \ol{B}
\ar[d, "{T^\set_{H,\id_\rho[\ol{\gamma := g}]}\,(\mu
    T^\set_{H,\id_\rho[\ol{\gamma := g}]}) \, \ol{B}}"] \ar[r,
  "{(\mathit{in}_{T^\set_{H,\rho[\ol{\gamma := C}]}})_{\ol{B}}}" ]
& (\mu T^\set_{H,\rho[\ol{\gamma := C}]})\, \ol{B} \ar[d, "{(\mu
    T^\set_{H,\id_\rho[\ol{\gamma := g}]}) \, \ol{B}}"]\\
T^\set_{H,\rho[\ol{\gamma := C'}]}\, (\mu
T^\set_{H,\rho[\ol{\gamma := C'}]})\, \ol{B} \ar[d, "{T^\set_{H,\rho[\ol{\gamma :=
          C'}]}\, (\mu T^\set_{H,\rho[\ol{\gamma := C'}]})\, \ol{f}}"]
\ar[r,"{(\mathit{in}_{T^\set_{H,\rho[\ol{\gamma := C'}]}})_{\ol{B}}}" ] & 
  (\mu T^\set_{H,\rho[\ol{\gamma := C'}]})\, \ol{B}
\ar[d,"{(\mathit{in}_{T^\set_{H,\rho[\ol{\gamma := C'}]}})_{\ol{B}}}" ] 
\\
T^\set_{H,\rho[\ol{\gamma := C'}]}\, (\mu T^\set_{H,\rho[\ol{\gamma := C'}]})\,
\ol{B'} \ar[r, "{(\mathit{in}_{T^\set_{H,\rho[\ol{\gamma :=
            C'}]}})_{\ol{B'}}}"] & (\mu T^\set_{H,\rho[\ol{\gamma := C'}]})\,
\ol{B'}
\end{tikzcd}\]
}
  That, for all $\rho : \setenv$ and $d :
\setsem{\Gamma;\emptyset \vdash \emptyset}\rho$,
\[\setsem{\Gamma;\emptyset \,|\, \emptyset \vdash
  \tin_H : Nat^{\ol{\beta},\ol{\gamma}} \, H[\phi := (\mu \phi.\lambda
    {\overline \alpha}.H){\overline \beta}][\ol{\alpha := \beta}]
  \;(\mu \phi.\lambda {\overline \alpha}.H){\overline
    \beta}}\,\rho\,d\] satisfies the additional property needed for
it to be in \[\setsem{\Gamma;\emptyset \vdash
  Nat^{\ol{\beta},\ol{\gamma}} \, H[\phi := (\mu \phi.\lambda
    {\overline \alpha}.H){\overline \beta}][\ol{\alpha := \beta}]
  \;(\mu \phi.\lambda {\overline \alpha}.H){\overline \beta}}\,\rho\]
  let $\ol{R : \rel(B,B')}$ and $\ol{S : \rel(C,C')}$ follows from the
  fact that 
\[\begin{array}{ll}
 & (\,(\setsem{\Gamma;\emptyset \,|\, \emptyset \vdash
  \tin_H : Nat^{\ol{\beta},\ol{\gamma}} \, H[\phi := (\mu \phi.\lambda
    {\overline \alpha}.H){\overline \beta}][\ol{\alpha := \beta}]
  \;(\mu \phi.\lambda {\overline \alpha}.H){\overline
    \beta}}\,\rho\,d)_{\ol{B},\ol{C}},\,\\
 & \hspace*{0.5in}(\setsem{\Gamma;\emptyset \,|\, \emptyset \vdash
  \tin_H : Nat^{\ol{\beta},\ol{\gamma}} \, H[\phi := (\mu \phi.\lambda
    {\overline \alpha}.H){\overline \beta}][\ol{\alpha := \beta}]
  \;(\mu \phi.\lambda {\overline \alpha}.H){\overline
    \beta}}\,\rho\,d)_{\ol{B'},\ol{C'}}\,)\\
=& (\, (\mathit{in}_{T^\set_{H,\rho[\ol{\gamma := C}]}})_{\ol{B}},
(\mathit{in}_{T^\set_{H,\rho[\ol{\gamma := C'}]}})_{\ol{B'}}\,)
\end{array}\]
has type
\[\begin{array}{ll}
%& (\, \setsem{\Gamma;\ol{\beta},\ol{\gamma} \vdash H[\phi := (\mu
%    \phi.\lambda {\overline \alpha}.H){\overline \beta}][\ol{\alpha :=
%      \beta}]}\rho[\ol{\beta := B}][\ol{\gamma := C}] \to\\ 
%& \hspace*{0.5in}\setsem{\Gamma;\ol{\beta},\ol{\gamma} \vdash (\mu \phi.\lambda
%  {\overline \alpha}.H){\overline \beta}} \rho[\ol{\beta :=
%    B}][\ol{\gamma := C}],\\
%& \hspace*{0.05in}\setsem{\Gamma;\ol{\beta},\ol{\gamma} \vdash H[\phi := (\mu
%    \phi.\lambda {\overline \alpha}.H){\overline \beta}][\ol{\alpha :=
%      \beta}]}\rho[\ol{\beta := B'}][\ol{\gamma := C'}] \to\\ 
%& \hspace*{0.5in}\setsem{\Gamma;\ol{\beta},\ol{\gamma} \vdash (\mu \phi.\lambda
%  {\overline \alpha}.H){\overline \beta}} \rho[\ol{\beta :=
%    B'}][\ol{\gamma := C'}]\,)\\
& (\, T^\set_{H,\rho[\ol{\gamma := C}]}\, (\mu
T^\set_{H,\rho[\ol{\gamma := C}]}) \, 
\ol{B} \to (\mu T^\set_{H,\rho[\ol{\gamma := C}]}) \, \ol{B}, \, \\
& \hspace*{0.5in}T^\set_{H,\rho[\ol{\gamma := C'}]}\, (\mu
T^\set_{H,\rho[\ol{\gamma := C'}]}) \, 
\ol{B'} \to (\mu T^\set_{H,\rho[\ol{\gamma := C'}]}) \,\ol{B'} \, )\\
= &
 \relsem{\Gamma;\ol{\beta},\ol{\gamma} \vdash H[\phi := (\mu
    \phi.\lambda {\overline \alpha}.H){\overline \beta}][\ol{\alpha :=
      \beta}]}\Eq_\rho[\ol{\beta := R}][\ol{\gamma := S}] \to\\
 & \hspace*{0.5in} \relsem{\Gamma;\ol{\beta},\ol{\gamma} \vdash (\mu
  \phi.\lambda \ol{\alpha}.H)\ol{\beta}} \Eq_\rho[\ol{\beta:=
    R}][\ol{\gamma :=S}]
\end{array}\]

\begin{comment}  
\begin{itemize}
\item To see that if $d : \setsem{\Gamma;\emptyset \vdash \emptyset}
  \rho$ then \[\setsem{\Gamma;\emptyset \,|\, \emptyset \vdash \tin_H :
    Nat^{\ol{\beta},\ol{\gamma}} \, H[\phi := (\mu \phi.\lambda
      {\overline \alpha}.H){\overline \beta}][\ol{\alpha := \beta}]
    \;(\mu \phi.\lambda {\overline \alpha}.H){\overline \beta}}\,
  \rho\,d\] is in $\setsem{\Gamma;\emptyset \vdash
    Nat^{\ol{\beta},\ol{\gamma}} \, H[\phi := (\mu \phi.\lambda
      {\overline \alpha}.H){\overline \beta}][\ol{\alpha := \beta}]
    \;(\mu \phi.\lambda {\overline \alpha}.H){\overline \beta}}\,
  \rho$, we first note that, for all $\ol{B}$ and $\ol{C}$,
  \[\begin{array}{ll}
   &(\setsem{\Gamma;\emptyset \,|\, \emptyset \vdash \tin_H :
    Nat^{\ol{\beta},\ol{\gamma}} \, H[\phi := (\mu \phi.\lambda
      {\overline \alpha}.H){\overline \beta}][\ol{\alpha := \beta}]
    \;(\mu \phi.\lambda {\overline \alpha}.H){\overline \beta}}\,
  \rho\,d)_{\ol{B}\,\ol{C}}\\
  =& (\mathit{in}_{T_{\rho[\ol{\gamma := C}]}})_{\ol{B}}
\end{array}\] does indeed map
\[\begin{array}{ll}
  & \setsem{\Gamma;\ol{\beta},\ol{\gamma} \vdash H[\phi := (\mu
    \phi.\lambda {\overline \alpha}.H){\overline \beta}][\ol{\alpha :=
      \beta}]}\rho[\ol{\beta := B}][\ol{\gamma := C}]\\
\hspace{0.2in} = & \setsem{\Gamma;\ol{\beta},\ol{\gamma},\ol{\alpha} \vdash H[\phi :=
    (\mu \phi.\lambda {\overline \alpha}.H){\overline
      \beta}]}\rho[\ol{\beta := B}][\ol{\gamma := C}][\ol{\alpha :=
    B}]\\
\hspace{0.2in} = & \setsem{\Gamma;\phi,\ol{\beta},\ol{\gamma},\ol{\alpha} \vdash H}
\rho[\ol{\beta := B}][\ol{\gamma := C}][\ol{\alpha :=
    B}]\\ & \hspace{0.5in} [\phi := \lambda
  \ol{D}.\,\setsem{\Gamma; \ol{\beta},\ol{\gamma},\ol{\alpha}
    \vdash (\mu \phi.\lambda {\overline \alpha}.H){\overline
      \beta}}\rho[\ol{\beta := B}][\ol{\gamma := C}][\ol{\alpha :=
    B}][\ol{\beta := D}]]\\
\hspace{0.2in} = & \setsem{\Gamma;\phi,\ol{\gamma},\ol{\alpha} \vdash H}
\rho[\ol{\gamma := C}][\ol{\alpha := B}]\\
  & \hspace*{0.5in}[\phi := \lambda \ol{D}.\,
  \setsem{\Gamma; \ol{\beta},\ol{\gamma}\vdash (\mu 
    \phi.\lambda {\overline \alpha}.H){\overline \beta}}\rho[\ol{\beta
      := D}][\ol{\gamma := C}]]\\
\hspace{0.2in} = & T_{\rho[\ol{\gamma := C}]}\, (\lambda \ol{D}.\,\setsem{\Gamma;
  \ol{\beta},\ol{\gamma}\vdash (\mu 
  \phi.\lambda {\overline \alpha}.H){\overline \beta}}\rho[\ol{\beta
    := D}][\ol{\gamma := C}])\, \ol{B}\\
\hspace{0.2in} = & T_{\rho[\ol{\gamma := C}]}\, (\mu T_{\rho[\ol{\gamma := C}]}) \, \ol{B}
\end{array}\] to
\[\begin{array}{ll}
 & \setsem{\Gamma;\ol{\beta},\ol{\gamma} \vdash (\mu \phi.\lambda
  {\overline \alpha}.H){\overline \beta}} \rho[\ol{\beta :=
    B}][\ol{\gamma := C}]\\
= & (\lambda \ol{D}.\, \setsem{\Gamma;
  \ol{\beta},\ol{\gamma}\vdash (\mu \phi.\lambda {\overline
    \alpha}.H){\overline \beta}}\rho[\ol{\beta := D}][\ol{\gamma :=
    C}]) \ol{B}\\
= & (\mu T_{\rho[\ol{\gamma := C}]}) \, \ol{B}
\end{array}\]

\vspace*{0.1in}

To see that
\[\begin{array}{ll}
 & \setsem{\Gamma;\emptyset \,|\, \emptyset \vdash \tin_H :
  Nat^{\ol{\beta},\ol{\gamma}} \, H[\phi := (\mu \phi.\lambda
    {\overline \alpha}.H){\overline \beta}][\ol{\alpha := \beta}]
  \;(\mu \phi.\lambda {\overline \alpha}.H){\overline
    \beta}}\,\rho\,d\\
=& \lambda \ol{B}\,\ol{C}.\,(\mathit{in}_{T_{\rho[\ol{\gamma :=
        C}]}})_{\ol{B}}
\end{array}\]
is natural in $\ol{B}$ and $\ol{C}$, we observe that the following
diagram commutes for all $\ol{f : B \to B'}$ and $\ol{g : C \to C'}$:
\[\begin{tikzcd}[column sep=2.5in, row sep=0.75in]
T_{\rho[\ol{\gamma := C}]}\, (\mu T_{\rho[\ol{\gamma := C}]})\, \ol{B}
\ar[d, "{\sigma_{\id_\rho[\ol{\gamma := g}]}\,(\mu
    \sigma_{\id_\rho[\ol{\gamma := g}]}) \, \ol{B}}"] \ar[r,
  "{(\mathit{in}_{T_{\rho[\ol{\gamma := C}]}})_{\ol{B}}}" ]
& (\mu T_{\rho[\ol{\gamma := C}]})\, \ol{B} \ar[d, "{(\mu
    \sigma_{\id_\rho[\ol{\gamma := g}]}) \, \ol{B}}"]\\
T_{\rho[\ol{\gamma := C'}]}\, (\mu
T_{\rho[\ol{\gamma := C'}]})\, \ol{B} \ar[d, "{T_{\rho[\ol{\gamma :=
          C'}]}\, (\mu T_{\rho[\ol{\gamma := C'}]})\, \ol{f}}"]
\ar[r,"{(\mathit{in}_{T_{\rho[\ol{\gamma := C'}]}})_{\ol{B}}}" ] & 
  (\mu T_{\rho[\ol{\gamma := C'}]})\, \ol{B}
\ar[d,"{(\mathit{in}_{T_{\rho[\ol{\gamma := C'}]}})_{\ol{B}}}" ] 
\\
T_{\rho[\ol{\gamma := C'}]}\, (\mu T_{\rho[\ol{\gamma := C'}]})\,
\ol{B'} \ar[r, "{(\mathit{in}_{T_{\rho[\ol{\gamma :=
            C'}]}})_{\ol{B'}}}"] & (\mu T_{\rho[\ol{\gamma := C'}]})\,
\ol{B'}
%\ar[d, "{(\mathit{in}_{T_{\rho[\ol{\gamma := C}]}})_{\ol{B}}}"']
%\ar[r, "{T_{\rho[\ol{\gamma := C'}]}\, (\mu T_{\rho[\ol{\gamma :=
%          C'}]})\, \ol{f} \; \circ \; \sigma_{\id_\rho[\ol{\gamma :=
%          g}]}\,(\mu \sigma_{\id_\rho[\ol{\gamma := g}]}) \, \ol{B}}"] 
%& T_{\rho[\ol{\gamma := C'}]}\, (\mu T_{\rho[\ol{\gamma := C'}]})\,
%\ol{B'} \ar[d, "({\mathit{in}_{T_{\rho[\ol{\gamma := C'}]}}})_{\ol{B'}}"]\\
%(\mu T_{\rho[\ol{\gamma := C}]})\, \ol{B}  \ar[r, "{(\mu
%    T_{\rho[\ol{\gamma := C'}]})\, \ol{f} \; \circ \; \mu
%    \sigma_{\id_\rho[\ol{\gamma := g}]} \ol{B}}"] 
%    & (\mu T_{\rho[\ol{\gamma := C'}]})\, \ol{B'} 
\end{tikzcd}\]
Indeed, naturality of $\mathit{in}$ with respect to its functor
argument ensures that the top diagram commutes, and naturality of
$\mathit{in}_{T_{\rho[\ol{\gamma := C'}]}}$ ensures that the bottom
one commutes. 
\item To see that $\setsem{\Gamma;\emptyset \,|\, \emptyset \vdash
  \tin_H : Nat^{\ol{\beta},\ol{\gamma}} \, H[\phi := (\mu \phi.\lambda
    {\overline \alpha}.H){\overline \beta}][\ol{\alpha := \beta}]
  \;(\mu \phi.\lambda {\overline \alpha}.H){\overline
    \beta}}\,\rho\,d$ satisfies the additional property needed for
  it to be in \[\setsem{\Gamma;\emptyset \vdash
    Nat^{\ol{\beta},\ol{\gamma}} \, H[\phi := (\mu \phi.\lambda
      {\overline \alpha}.H){\overline \beta}][\ol{\alpha := \beta}]
    \;(\mu \phi.\lambda {\overline \alpha}.H){\overline
      \beta}}\,\rho\]
  let $\ol{R : \rel(B,B')}$ and $\ol{S : \rel(C,C')}$. Then
\[\begin{array}{ll}
 & (\,(\setsem{\Gamma;\emptyset \,|\, \emptyset \vdash
  \tin_H : Nat^{\ol{\beta},\ol{\gamma}} \, H[\phi := (\mu \phi.\lambda
    {\overline \alpha}.H){\overline \beta}][\ol{\alpha := \beta}]
  \;(\mu \phi.\lambda {\overline \alpha}.H){\overline
    \beta}}\,\rho\,d)_{\ol{B},\ol{C}},\,\\
 & \hspace*{0.5in}(\setsem{\Gamma;\emptyset \,|\, \emptyset \vdash
  \tin_H : Nat^{\ol{\beta},\ol{\gamma}} \, H[\phi := (\mu \phi.\lambda
    {\overline \alpha}.H){\overline \beta}][\ol{\alpha := \beta}]
  \;(\mu \phi.\lambda {\overline \alpha}.H){\overline
    \beta}}\,\rho\,d)_{\ol{B'},\ol{C'}}\,)\\
=& (\, (\mathit{in}_{T_{\rho[\ol{\gamma := C}]}})_{\ol{B}},
(\mathit{in}_{T_{\rho[\ol{\gamma := C'}]}})_{\ol{B'}}\,)
\end{array}\]
has type
\[\begin{array}{ll}
& (\, \setsem{\Gamma;\ol{\beta},\ol{\gamma} \vdash H[\phi := (\mu
    \phi.\lambda {\overline \alpha}.H){\overline \beta}][\ol{\alpha :=
      \beta}]}\rho[\ol{\beta := B}][\ol{\gamma := C}] \to\\ 
& \hspace*{0.5in}\setsem{\Gamma;\ol{\beta},\ol{\gamma} \vdash (\mu \phi.\lambda
  {\overline \alpha}.H){\overline \beta}} \rho[\ol{\beta :=
    B}][\ol{\gamma := C}],\\
& \hspace*{0.05in}\setsem{\Gamma;\ol{\beta},\ol{\gamma} \vdash H[\phi := (\mu
    \phi.\lambda {\overline \alpha}.H){\overline \beta}][\ol{\alpha :=
      \beta}]}\rho[\ol{\beta := B'}][\ol{\gamma := C'}] \to\\ 
& \hspace*{0.5in}\setsem{\Gamma;\ol{\beta},\ol{\gamma} \vdash (\mu \phi.\lambda
  {\overline \alpha}.H){\overline \beta}} \rho[\ol{\beta :=
    B'}][\ol{\gamma := C'}]\,)\\
= & (\, T_{\rho[\ol{\gamma := C}]}\, (\mu T_{\rho[\ol{\gamma := C}]}) \,
\ol{B} \to (\mu T_{\rho[\ol{\gamma := C}]}) \, \ol{B}, \, 
T_{\rho[\ol{\gamma := C'}]}\, (\mu T_{\rho[\ol{\gamma := C'}]}) \,
\ol{B'} \to (\mu T_{\rho[\ol{\gamma := C'}]}) \,\ol{B'} \, )\\
= & \relsem{\Gamma;\ol{\beta},\ol{\gamma} \vdash H[\phi := (\mu
    \phi.\lambda {\overline \alpha}.H){\overline \beta}][\ol{\alpha :=
      \beta}]}\Eq_\rho[\ol{\beta := R}][\ol{\gamma := S}] \to \\
& \hspace*{0.5in} \relsem{\Gamma;\ol{\beta},\ol{\gamma} \vdash (\mu
  \phi.\lambda \ol{\alpha}.H)\ol{\beta}} \Eq_\rho[\ol{\beta:=
    R}][\ol{\gamma :=S}] 
\end{array}\]
\end{comment}

\begin{comment}
\item The proofs that $\relsem{\Gamma;\emptyset \,|\,
    \emptyset \vdash \tin_H : Nat^{\ol{\beta},\ol{\gamma}} \, H[\phi
      := (\mu \phi.\lambda {\overline \alpha}.H){\overline
        \beta}][\ol{\alpha := \beta}] \;(\mu \phi.\lambda {\overline
      \alpha}.H){\overline \beta}}$ is a natural transformation from
  $\relsem{\Gamma;\emptyset \vdash \emptyset}$ to
  $\relsem{\Gamma;\emptyset \vdash Nat^{\ol{\beta},\ol{\gamma}} \,
    H[\phi := (\mu \phi.\lambda {\overline \alpha}.H){\overline
        \beta}][\ol{\alpha := \beta}] \;(\mu \phi.\lambda {\overline
      \alpha}.H){\overline \beta}}$ and that, for all $\rho : \relenv$
  and $d : \relsem{\Gamma;\emptyset \vdash \emptyset}$,
  \[\relsem{\Gamma;\emptyset \,|\, \emptyset \vdash \tin_H :
    Nat^{\ol{\beta},\ol{\gamma}} \, H[\phi := (\mu \phi.\lambda
      {\overline \alpha}.H){\overline \beta}][\ol{\alpha := \beta}]
    \;(\mu \phi.\lambda {\overline \alpha}.H){\overline
      \beta}}\,\rho\,d\] is a natural transformation from $\lambda
  \ol{R}\,\ol{S}.\,\relsem{\Gamma; \ol{\beta},\ol{\gamma} \vdash
    H[\phi := (\mu \phi.\lambda {\overline \alpha}.H){\overline
        \beta}][\ol{\alpha := \beta}]}\rho[\ol{\beta := R}][\ol{\gamma
      := S}]$ to $\lambda \ol{R}\,\ol{S}.\,\relsem{\Gamma;
    \ol{\beta},\ol{\gamma} \vdash (\mu \phi.\lambda {\overline
      \alpha}.H){\overline \beta}}\rho[\ol{\beta := R}][\ol{\gamma :=
      S}]$, are analogous.
\item Finally, to see that $\pi_i(\relsem{\Gamma;\emptyset \,|\,
  \emptyset \vdash \tin_H : Nat^{\ol{\beta},\ol{\gamma}} \, H[\phi :=
    (\mu \phi.\lambda {\overline \alpha}.H){\overline
      \beta}][\ol{\alpha := \beta}] \;(\mu \phi.\lambda {\overline
    \alpha}.H){\overline \beta}}\,\rho\,d) = \setsem{\Gamma;\emptyset
  \,|\, \emptyset \vdash \tin_H : Nat^{\ol{\beta},\ol{\gamma}} \,
  H[\phi := (\mu \phi.\lambda {\overline \alpha}.H){\overline
      \beta}][\ol{\alpha := \beta}] \;(\mu \phi.\lambda {\overline
    \alpha}.H){\overline \beta}}\,(\pi_i \rho)\,(\pi_i d)$ we first
  note that $d : \relsem{\Gamma; \emptyset \vdash \emptyset}\rho$ and
  $\pi_i d : \setsem{\Gamma; \emptyset \vdash \emptyset} (\pi_i \rho)$
  are uniquely determined. Further, the definition of natural
  transformations in $\rel$ ensures that, for any $\ol{R}$ and
  $\ol{S}$,
\[\begin{split}
 &~(\mathit{in}_{T_{\rho[\ol{\gamma := S}]}})_{\ol{R}}\\
=&~((\mathit{in}_{\pi_1 (T_{\rho[\ol{\gamma := S}]})})_{\ol{\pi_1 R}}, \,
(\mathit{in}_{\pi_2 (T_{\rho[\ol{\gamma := S}]})})_{\ol{\pi_2 R}})\\ 
=&~((\mathit{in}_{T^\set_{(\pi_1 \rho)[\ol{\gamma := \pi_1
        S}]}})_{\ol{\pi_1 R}}, \, (\mathit{in}_{T^\set_{(\pi_2
    \rho)[\ol{\gamma := \pi_2 S}]}})_{\ol{\pi_2 R}})\\
\end{split}\]
Observing that $\pi_1$ and $\pi_2$ are surjective, we therefore have
that
\[\begin{split}
 &~\pi_i(\relsem{\Gamma;\emptyset \,|\,
  \emptyset \vdash \tin_H : Nat^{\ol{\beta},\ol{\gamma}} \, H[\phi :=
    (\mu \phi.\lambda {\overline \alpha}.H){\overline
      \beta}][\ol{\alpha := \beta}] \;(\mu \phi.\lambda {\overline
    \alpha}.H){\overline \beta}}\,\rho\,d)\\
=&~\pi_i (\lambda \ol{R}\,\ol{S}.\, (\mathit{in}_{T_{\rho[\ol{\gamma :=
        S}]}})_{\ol{R}}) \\ 
=&~\lambda \ol{B}\,\ol{C}.\, (\mathit{in}_{T^\set_{(\pi_i
      \rho)[\ol{\gamma := C}]}})_{\ol{B}} \\
=&~\setsem{\Gamma;\emptyset \,|\,
  \emptyset \vdash \tin_H : Nat^{\ol{\beta},\ol{\gamma}} \, H[\phi :=
    (\mu \phi.\lambda {\overline \alpha}.H){\overline
      \beta}][\ol{\alpha := \beta}] \;(\mu \phi.\lambda {\overline
    \alpha}.H){\overline \beta}}\,(\pi_i \rho)\,(\pi_i d)
\end{split}\]
\end{itemize}
\end{comment}
\item
$\underline{\Gamma; \emptyset~|~\emptyset \vdash \fold^F_H :
  \Nat^\emptyset\;(\Nat^{\ol{\beta}, \ol{\gamma}}\,H[\phi
    :=_{\ol{\beta}} F][\ol{\alpha := \beta}]\,F)\;
  (\Nat^{{\ol{\beta},\ol{\gamma}} }\,(\mu \phi.\lambda \overline
  \alpha.H)\overline \beta \;F)}$ \; Since $\Phi$ is empty, to see
  that $\setsem{ \Gamma; \emptyset~|~\emptyset \vdash \fold^F_H :
    \Nat^\emptyset\;(\Nat^{\ol{\beta}, \ol{\gamma}}\,H[\phi
      :=_{\ol{\beta}} F][\ol{\alpha := \beta}]\,F)\;
    (\Nat^{{\ol{\beta},\ol{\gamma}} }\,(\mu \phi.\lambda \overline
    \alpha.H)\overline \beta \;F)}$ is a natural transformation
  $\setsem{\Gamma;\emptyset \vdash \emptyset}$ to \[\setsem{\Gamma;
    \emptyset \vdash \Nat^\emptyset\;(\Nat^{\ol{\beta},
      \ol{\gamma}}\,H[\phi :=_{\ol{\beta}} F][\ol{\alpha :=
        \beta}]\,F)\; (\Nat^{{\ol{\beta},\ol{\gamma}} }\,(\mu
    \phi.\lambda \overline \alpha.H)\overline \beta\,F}\] we need only
  show that, for all $\rho : \setenv$, the unique $d :
  \setsem{\Gamma;\emptyset \vdash \emptyset} \rho$, and all $\eta :
  \setsem{\Gamma; \emptyset \vdash \Nat^{\ol{\beta},
      \ol{\gamma}}\,H[\phi :=_{\ol{\beta}} F][\ol{\alpha :=
        \beta}]\,F} \rho$,
\[ \setsem{\Gamma; \emptyset~|~\emptyset \vdash \fold^F_H :
  \Nat^\emptyset\;(\Nat^{\ol{\beta}, \ol{\gamma}}\,H[\phi
    :=_{\ol{\beta}} F][\ol{\alpha := \beta}]\,F)\;
  (\Nat^{{\ol{\beta},\ol{\gamma}} }\,(\mu \phi.\lambda \overline
  \alpha.H)\overline \beta\,F}\,\rho\,d\,\eta\] has type
$\setsem{\Gamma; \emptyset \vdash \Nat^{{\ol{\beta},\ol{\gamma}}
  }\,(\mu \phi.\lambda \overline \alpha.H)\overline \beta\,F}\,\rho$
i.e., for any $\ol{B}$ and $\ol{C}$,
\[(\setsem{\Gamma; \emptyset~|~\emptyset \vdash \fold^F_H :
  \Nat^\emptyset\;(\Nat^{\ol{\beta}, \ol{\gamma}}\,H[\phi
    :=_{\ol{\beta}} F][\ol{\alpha := \beta}]\,F)\;
  (\Nat^{{\ol{\beta},\ol{\gamma}} }\,(\mu \phi.\lambda \overline
  \alpha.H)\overline \beta\,F}\,\rho\,d\,\eta)_{\ol{B}\,\ol{C}}\] is a
morphism from $\setsem{\Gamma; \ol{\beta},\ol{\gamma} \vdash (\mu
  \phi.\lambda \overline \alpha.H)\overline \beta}\rho[\ol{\beta :=
    B}][\ol{\gamma := C}] \,=\,(\mu T^\set_{H,\rho[\ol{\gamma := C}]})
\ol{B}$\\ to $\setsem{\Gamma; \ol{\beta},\ol{\gamma} \vdash
  F}\rho[\ol{\beta := B}][\ol{\gamma := C}]$.  To see this, note
%use Equations~\ref{eq:subs-var} and~\ref{eq:subs-const} to verify
that $\eta$ is a natural transformation from
\[\begin{array}{ll}
 & \lambda \ol{B}\,\ol{C}.\,\setsem{\Gamma; \ol{\beta},\ol{\gamma}
  \vdash H[\phi := F][\ol{\alpha := \beta}]}\rho[\ol{\beta :=
    B}][\ol{\gamma := C}]\\
%= & \lambda
%\ol{B}\,\ol{C}.\,\setsem{\Gamma;\ol{\beta},\ol{\gamma},\ol{\alpha}  
%  \vdash H[\phi := F]}\rho[\ol{\beta := B}][\ol{\gamma :=
%    C}][\ol{\alpha := B}]\\ 
%= & \lambda \ol{B}\,\ol{C}.\,\setsem{\Gamma;
%  \ol{\beta},\ol{\gamma},\ol{\alpha},\phi \vdash H}\rho
%[\ol{\beta := B}][\ol{\gamma := C}][\ol{\alpha := B}]\\
% & \hspace*{0.5in}[\phi := \lambda \ol{A}.\, \setsem{\Gamma;\ol{\beta},
%    \ol{\gamma},\ol{\alpha} \vdash F}\rho[\ol{\beta := B}][\ol{\gamma
%      := C}][\ol{\alpha := B}][\ol{\beta := A}]]\\ 
%= & \lambda \ol{B}\,\ol{C}.\, \setsem{\Gamma;
%  \ol{\gamma},\ol{\alpha},\phi \vdash H}\rho[\ol{\gamma :=
%    C}][\ol{\alpha := B}][\phi := \lambda \ol{A}.\,
%  \setsem{\Gamma;\ol{\beta}, 
%    \ol{\gamma}\vdash F}\rho[\ol{\gamma := C}][\ol{\beta := A}]]\\ 
= & \lambda \ol{B}\,\ol{C}.\,T^\set_{H,\rho[\ol{\gamma:=
     C}]}\,(\lambda \ol{A}. \, \setsem{\Gamma;\ol{\beta},\ol{\gamma} 
  \vdash F}\rho[\ol{\beta := A}][\ol{\gamma := C}]) \, \ol{B}
\end{array}\]
to
\[\begin{array}{ll}
 & \lambda \ol{B}\,\ol{C}.\,(\lambda
\ol{A}.\,\setsem{\Gamma;\ol{\beta},\ol{\gamma} \vdash F}\rho[\ol{\beta
    := A}][\ol{\gamma := C}]) \ol{B}\\
= & \lambda
\ol{B}\,\ol{C}.\,\setsem{\Gamma;\ol{\beta},\ol{\gamma} \vdash
  F}\rho[\ol{\beta := B}][\ol{\gamma := C}]
\end{array}\]
and thus
%if $x :
%\setsem{\Gamma; \ol{\beta},\ol{\gamma} \vdash (\mu \phi.\lambda
%  \overline \alpha.H)\overline \beta}\rho[\ol{\beta := B}][\ol{\gamma
%    := C}]\;=\;(\mu T^\set_{\rho[\ol{\gamma := C}]}) \ol{B}$, then
%\[\begin{array}{ll}
%  & (\setsem{\Gamma; \emptyset~|~\emptyset \vdash \fold^F_H :
%  \Nat^\emptyset\;(\Nat^{\ol{\beta}, \ol{\gamma}}\,H[\phi
%    :=_{\ol{\beta}} F][\ol{\alpha := \beta}]\,F)\;
%  (\Nat^{{\ol{\beta},\ol{\gamma}} }\,(\mu \phi.\lambda \overline
%  \alpha.H)\overline \beta\,F}\,\rho\,d\,\eta)_{\ol{B}\,\ol{C}}\,x\\
%= & (\mathit{fold}_{T^\set_{\rho[{\gamma := C}]}}\,(\lambda
%\ol{A}.\,\eta_{\ol{A}\,\ol{C}}))_{\ol{B}}\,x\\  
%:& (\lambda \ol{A}.\,\setsem{\Gamma;\ol{\beta},\ol{\gamma} \vdash
%  F}\rho[\ol{\beta := A}][\ol{\gamma := C}])\ol{B}
%\end{array}\]
%i.e.,
for each $\ol{B}$ and $\ol{C}$,
\[(\setsem{\Gamma; \emptyset~|~\emptyset \vdash \fold^F_H :
  \Nat^\emptyset\;(\Nat^{\ol{\beta}, \ol{\gamma}}\,H[\phi
    :=_{\ol{\beta}} F][\ol{\alpha := \beta}]\,F)\;
  (\Nat^{{\ol{\beta},\ol{\gamma}} }\,(\mu \phi.\lambda \overline
  \alpha.H)\overline \beta\,F}\,\rho\,d\,\eta)_{\ol{B}\,\ol{C}}\] is a
morphism from
$\setsem{\Gamma; \ol{\beta},\ol{\gamma} \vdash (\mu
  \phi.\lambda \overline \alpha.H)\overline \beta}\rho[\ol{\beta :=
    B}][\ol{\gamma := C}]\,=\, (\mu T^\set_{H,\rho[\ol{\gamma := C}]})
\ol{B}$ to\\ $\setsem{\Gamma;\ol{\beta},\ol{\gamma} \vdash
  F}\rho[\ol{\beta := B}][\ol{\gamma := C}]$.
\begin{comment}  
\begin{itemize}
\item
  To see that $\setsem{\Gamma; \emptyset~|~\emptyset \vdash
  \fold^F_H : \Nat^\emptyset\;(\Nat^{\ol{\beta}, \ol{\gamma}}\,H[\phi
    :=_{\ol{\beta}} F][\ol{\alpha := \beta}]\,F)\;
  (\Nat^{{\ol{\beta},\ol{\gamma}} }\,(\mu \phi.\lambda \overline
  \alpha.H)\overline \beta\,F}$ is a natural transformation from
  $\setsem{\Gamma;\emptyset \vdash \emptyset}$ to
  \[\setsem{\Gamma; \emptyset \vdash
    \Nat^\emptyset\;(\Nat^{\ol{\beta}, \ol{\gamma}}\,H[\phi 
    :=_{\ol{\beta}} F][\ol{\alpha := \beta}]\,F)\;
    (\Nat^{{\ol{\beta},\ol{\gamma}} }\,(\mu \phi.\lambda \overline
    \alpha.H)\overline \beta\,F}\] since the functorial part $\Phi$ of
  the context is empty, we need only show that, for every $\rho :
  \setenv$,
\[\setsem{\Gamma; \emptyset~|~\emptyset \vdash
  \fold^F_H : \Nat^\emptyset\;(\Nat^{\ol{\beta}, \ol{\gamma}}\,H[\phi
    :=_{\ol{\beta}} F][\ol{\alpha := \beta}]\,F)\;
  (\Nat^{{\ol{\beta},\ol{\gamma}} }\,(\mu \phi.\lambda \overline
  \alpha.H)\overline \beta\,F}\,\rho\]
is a morphism in $\set$
  from $\setsem{\Gamma;\emptyset \vdash \emptyset}\rho$ to
\[\setsem{\Gamma; \emptyset \vdash \Nat^\emptyset\;(\Nat^{\ol{\beta},
    \ol{\gamma}} \,H[\phi :=_{\ol{\beta}} F][\ol{\alpha :=
      \beta}]\,F)\; (\Nat^{{\ol{\beta},\ol{\gamma}} }\,(\mu
  \phi.\lambda \overline \alpha.H)\overline \beta\,F}\,\rho\] i.e.,
that, for the unique $d : \setsem{\Gamma;\emptyset \vdash \emptyset}
\rho$,
\[\setsem{\Gamma; \emptyset~|~\emptyset \vdash
  \fold^F_H : \Nat^\emptyset\;(\Nat^{\ol{\beta}, \ol{\gamma}}\,H[\phi
    :=_{\ol{\beta}} F][\ol{\alpha := \beta}]\,F)\;
  (\Nat^{{\ol{\beta},\ol{\gamma}} }\,(\mu \phi.\lambda \overline
  \alpha.H)\overline \beta\,F}\,\rho\,d\] is a morphism from
$\setsem{\Gamma; \emptyset \vdash \Nat^{\ol{\beta},
    \ol{\gamma}}\,H[\phi :=_{\ol{\beta}} F][\ol{\alpha := \beta}]\,F}
\rho$ to $\setsem{\Gamma; \emptyset \vdash
  \Nat^{{\ol{\beta},\ol{\gamma}} }\,(\mu \phi.\lambda \overline
  \alpha.H)\overline \beta\,F}\rho$. For this we show that for every
$\eta : \setsem{\Gamma; \emptyset \vdash \Nat^{\ol{\beta},
    \ol{\gamma}}\,H[\phi :=_{\ol{\beta}} F][\ol{\alpha := \beta}]\,F}
\rho$ we have
\[\begin{array}{ll}
  & \setsem{\Gamma; \emptyset~|~\emptyset \vdash \fold^F_H :
  \Nat^\emptyset\;(\Nat^{\ol{\beta}, \ol{\gamma}}\,H[\phi
    :=_{\ol{\beta}} F][\ol{\alpha := \beta}]\,F)\;
  (\Nat^{{\ol{\beta},\ol{\gamma}} }\,(\mu \phi.\lambda \overline
  \alpha.H)\overline \beta\,F}\,\rho\,d\,\eta\\ : & \setsem{\Gamma;
  \emptyset \vdash \Nat^{{\ol{\beta},\ol{\gamma}} }\,(\mu \phi.\lambda
  \overline \alpha.H)\overline \beta\,F}\,\rho
\end{array}\]



To this end we show that, for any $\ol{B}$ and $\ol{C}$,
\[(\setsem{\Gamma; \emptyset~|~\emptyset \vdash \fold^F_H :
  \Nat^\emptyset\;(\Nat^{\ol{\beta}, \ol{\gamma}}\,H[\phi
    :=_{\ol{\beta}} F][\ol{\alpha := \beta}]\,F)\;
  (\Nat^{{\ol{\beta},\ol{\gamma}} }\,(\mu \phi.\lambda \overline
  \alpha.H)\overline \beta\,F}\,\rho\,d\,\eta)_{\ol{B}\,\ol{C}}\]
is a morphism from 
\[\setsem{\Gamma; \ol{\beta},\ol{\gamma} \vdash (\mu
    \phi.\lambda \overline \alpha.H)\overline \beta}\rho[\ol{\beta :=
    B}][\ol{\gamma := C}]\,=\,(\mu T^\set_{\rho[\ol{\gamma := C}]})
\ol{B}\] to \[\setsem{\Gamma; \ol{\beta},\ol{\gamma} \vdash
  F}\rho[\ol{\beta := B}][\ol{\gamma := C}]
%= (\lambda \ol{A}.\,
%\setsem{\Gamma; \ol{\beta},\ol{\gamma} \vdash
%  F}\rho[\ol{\beta := A}][\ol{\gamma := C}])\,\ol{B}
\] To see this, we use
Equations~\ref{eq:subs-var} and~\ref{eq:subs-const} for the first and
second equalities below, together with weakening, to see that $\eta$
is itself a natural transformation from 
\[\begin{array}{ll}
 & \lambda \ol{B}\,\ol{C}.\,\setsem{\Gamma; \ol{\beta},\ol{\gamma}
  \vdash H[\phi := F][\ol{\alpha := \beta}]}\rho[\ol{\beta :=
    B}][\ol{\gamma := C}]\\ 
= & \lambda
\ol{B}\,\ol{C}.\,\setsem{\Gamma;\ol{\beta},\ol{\gamma},\ol{\alpha}  
  \vdash H[\phi := F]}\rho[\ol{\beta := B}][\ol{\gamma :=
    C}][\ol{\alpha := B}]\\ 
= & \lambda \ol{B}\,\ol{C}.\,\setsem{\Gamma;
  \ol{\beta},\ol{\gamma},\ol{\alpha},\phi \vdash H}\rho
[\ol{\beta := B}][\ol{\gamma := C}][\ol{\alpha := B}]\\
 & \hspace*{0.5in}[\phi := \lambda \ol{A}.\, \setsem{\Gamma;\ol{\beta},
    \ol{\gamma},\ol{\alpha} \vdash F}\rho[\ol{\beta := B}][\ol{\gamma
      := C}][\ol{\alpha := B}][\ol{\beta := A}]]\\ 
= & \lambda \ol{B}\,\ol{C}.\, \setsem{\Gamma;
  \ol{\gamma},\ol{\alpha},\phi \vdash H}\rho[\ol{\gamma :=
    C}][\ol{\alpha := B}][\phi := \lambda \ol{A}.\,
  \setsem{\Gamma;\ol{\beta}, 
    \ol{\gamma}\vdash F}\rho[\ol{\gamma := C}][\ol{\beta := A}]]\\ 
=& \lambda \ol{B}\,\ol{C}.\,T^\set_{\rho[\ol{\gamma:=
      C}]}\,(\lambda \ol{A}. \, \setsem{\Gamma;\ol{\beta},\ol{\gamma} 
  \vdash F}\rho[\ol{\beta := A}][\ol{\gamma := C}]) \, \ol{B}\\
\end{array}\]
to 
\[\lambda \ol{B}\,\ol{C}.\,(\lambda
\ol{A}.\,\setsem{\Gamma;\ol{\beta},\ol{\gamma} 
  \vdash F}\rho[\ol{\beta := A}][\ol{\gamma := C}]) \ol{B}\;\;= \;\;
\lambda \ol{B}\,\ol{C}.\,\setsem{\Gamma;\ol{\beta},\ol{\gamma}
  \vdash F}\rho[\ol{\beta := B}][\ol{\gamma := C}]\] Thus, if $x :
\setsem{\Gamma; \ol{\beta},\ol{\gamma} \vdash (\mu \phi.\lambda
  \overline \alpha.H)\overline \beta}\rho[\ol{\beta := B}][\ol{\gamma
    := C}]\;=\;(\mu T^\set_{\rho[\ol{\gamma := C}]}) \ol{B}$, then
\[\begin{array}{ll}
  & (\setsem{\Gamma; \emptyset~|~\emptyset \vdash \fold^F_H :
  \Nat^\emptyset\;(\Nat^{\ol{\beta}, \ol{\gamma}}\,H[\phi
    :=_{\ol{\beta}} F][\ol{\alpha := \beta}]\,F)\;
  (\Nat^{{\ol{\beta},\ol{\gamma}} }\,(\mu \phi.\lambda \overline
  \alpha.H)\overline \beta\,F}\,\rho\,d\,\eta)_{\ol{B}\,\ol{C}}\,x\\
= & (\mathit{fold}_{T^\set_{\rho[{\gamma := C}]}}\,(\lambda
\ol{A}.\,\eta_{\ol{A}\,\ol{C}}))_{\ol{B}}\,x\\  
:& (\lambda \ol{A}.\,\setsem{\Gamma;\ol{\beta},\ol{\gamma} \vdash
  F}\rho[\ol{\beta := A}][\ol{\gamma := C}])\ol{B}
\end{array}\]
i.e., for each $\ol{B}$ and $\ol{C}$
\[(\setsem{\Gamma; \emptyset~|~\emptyset \vdash \fold^F_H :
  \Nat^\emptyset\;(\Nat^{\ol{\beta}, \ol{\gamma}}\,H[\phi
    :=_{\ol{\beta}} F][\ol{\alpha := \beta}]\,F)\;
  (\Nat^{{\ol{\beta},\ol{\gamma}} }\,(\mu \phi.\lambda \overline
  \alpha.H)\overline \beta\,F}\,\rho\,d\,\eta)_{\ol{B}\,\ol{C}}\] is a
morphism from $(\mu T^\set_{\rho[\ol{\gamma := C}]}) \ol{B}$ to
$\setsem{\Gamma;\ol{\beta},\ol{\gamma} \vdash F}\rho[\ol{\beta :=
    B}][\ol{\gamma := C}]$.
\vspace*{0.1in}
\end{comment}
To see that this family of morphisms is natural in $\ol{B}$ and
$\ol{C}$, we observe that the following diagram commutes for all
$\ol{f : B \to B'}$ and $\ol{g : C \to C'}$:
{\tiny
\[\begin{tikzcd}[column sep=2.5in, row sep=0.75in]
(\mu T^\set_{H,\rho[\ol{\gamma := C}]})\, \ol{B}
\ar[d, "{(\mu T^\set_{H,\id_\rho[\ol{\gamma := g}]}) \, \ol{B}}"'] \ar[r, 
  "{(\mathit{fold}_{T^\set_{H,\rho[\ol{\gamma := C}]}}\,(\lambda
    \ol{A}.\,\eta_{\ol{A}\,\ol{C}}))_{\ol{B}}}"] 
& \setsem{\Gamma; \ol{\beta},\ol{\gamma} \vdash F}\rho[\ol{\gamma :=
    C}][\ol{\beta := B}]\ar[d, "{\setsem{\Gamma;
      \ol{\beta},\ol{\gamma} \vdash F}\id_\rho[\ol{\gamma := 
    g}][\ol{\beta := \id_B}]}"]\\
(\mu T^\set_{H,\rho[\ol{\gamma := C'}]})\, \ol{B} 
\ar[r,"{(\mathit{fold}_{T^\set_{H,\rho[\ol{\gamma :=
            C'}]}}\,(\lambda \ol{A}.\,\eta_{\ol{A}\,\ol{C'}}))_{\ol{B}}\,}" ] 
\ar[d, "{(\mu T^\set_{H,\rho[\ol{\gamma := C'}]}) \, \ol{f}}"'] & 
 \setsem{\Gamma; \ol{\beta},\ol{\gamma} \vdash F}\rho[\ol{\gamma :=
    C'}][\ol{\beta := B}]\ar[d, "{\setsem{\Gamma;
      \ol{\beta},\ol{\gamma} \vdash F}\id_\rho[\ol{\gamma := 
    \id_{C'}}][\ol{\beta := f}]}"]\\
 (\mu T^\set_{H,\rho[\ol{\gamma := C'}]})\, \ol{B'} 
\ar[r,"{(\mathit{fold}_{T^\set_{H,\rho[\ol{\gamma :=
            C'}]}}\,(\lambda \ol{A}.\,\eta_{\ol{A}\,\ol{C'}}))_{\ol{B'}}\,}"
] &  
 \setsem{\Gamma; \ol{\beta},\ol{\gamma} \vdash F}\rho[\ol{\gamma :=
    C'}][\ol{\beta := B'}]
\end{tikzcd}\]}
Indeed, naturality of $\mathit{fold}_{T^\set_{H,\rho[\ol{\gamma :=
        C'}]}}\,(\lambda \ol{A}.\,\eta_{\ol{A}\,\ol{C'}})$ ensures that
the bottom diagram commutes. To see that the top one commutes
%is considerably more delicate.
%\vspace*{0.1in}
% To see that the top diagram commutes
we first observe that, given a natural transformation $\Theta : H \to
K : [\set^k, \set] \to [\set^k, \set]$, the fixpoint natural
transformation $\mu \Theta : \mu H \to \mu K : \set^k \to \set$ is
defined to be $\textit{fold}_{H}(\Theta\,(\mu K) \circ
\textit{in}_{K})$, i.e., the unique morphism making the following
diagram commute:\label{page:dia1}
{\footnotesize
\[\begin{tikzcd}[column sep = large]
H(\mu H)
	\ar[dd, "{\textit{in}_H}"']
	\ar[r, "{H(\mu \Theta)}"]
& H(\mu K)
	\ar[d, "{\Theta (\mu K)}"] \\
& K(\mu K)
	\ar[d, "{\textit{in}_K}"] \\
\mu H
	\ar[r, "{\mu \Theta}"']
& \mu K
\end{tikzcd}\]}
Taking $\Theta = T^{\set}_{H,f}: T^{\set}_{H,\rho} \to
T^{\set}_{H,\rho'}$ thus gives that, for any $f : \rho \to \rho'$ in
$\setenv$,
\begin{equation}\label{eq:mu-sigma-def}
\mathit{in}_{T^{\set}_{H,\rho'}} \circ 
T^{\set}_{H,f} (\mu T^{\set}_{H,\rho'}) \circ 
T^{\set}_{H,\rho}(\mu T^{\set}_{H,f}) \,=\, 
\mu T^{\set}_{H,f} \circ \mathit{in}_{T^{\set}_{H,\rho}}
\end{equation}
%the following diagram commutes for any
%morphism of set environments $f : \rho \to \rho'$:
%\begin{equation}\label{eq:mu-sigma-def}
%\begin{tikzcd}[column sep = large]
%T^{\set}_{\rho}(\mu T^{\set}_{\rho})
%	\ar[dd, "{\mathit{in}_{T^{\set}_{\rho}}}"']
%	\ar[r, "{T^{\set}_{\rho}(\mu \sigma^{\set}_{f})}"]
%& T^{\set}_{\rho}(\mu T^{\set}_{\rho'})
%	\ar[d, "{\sigma^{\set}_{f} (\mu T^{\set}_{\rho'})}"] \\
%& T^{\set}_{\rho'}(\mu T^{\set}_{\rho'})
%	\ar[d, "{\mathit{in}_{T^{\set}_{\rho'}}}"] \\
%\mu T^{\set}_{\rho}
%	\ar[r, "{\mu \sigma^{\set}_{f}}"']
%& \mu T^{\set}_{\rho'}
%\end{tikzcd}
%\end{equation}
Next, note that the action of the functor
%\begin{equation*}
$\lambda \ol{B}. \lambda \ol{C}. \setsem{\Gamma; \ol{\beta},
  \ol{\gamma} \vdash H[\phi := F][\ol{\alpha := \beta}]}\rho
        [\ol{\beta := B}] [\ol{\gamma := C}]
$ %\end{equation*}
on the morphisms $\ol{f : B \to B'}, \ol{g : C \to C'}$ is given by
\[\begin{array}{ll}
 & \setsem{\Gamma; \ol{\beta}, \ol{\gamma} \vdash H[\phi :=
      F][\ol{\alpha := \beta}]} \id_{\rho} [\ol{\beta := f}]
           [\ol{\gamma := g}]\\
%= & \setsem{\Gamma; \ol{\alpha}, \ol{\gamma} \vdash H[\phi :=
%      F]} \id_{\rho} [\ol{\alpha := f}][\ol{\gamma := g}]\\
= & \setsem{\Gamma; \phi, \ol{\alpha}, \ol{\gamma} \vdash H}
\id_{\rho}[\ol{\alpha := f}] [\ol{\gamma := g}][\phi := \lambda
  \ol{A}. \setsem{\Gamma; \ol{\beta}, \ol{\gamma} \vdash F} \id_{\rho
    [\ol{\beta := A}]} [\ol{\gamma := g}]] \\ 
%=& \setsem{\Gamma; \phi, \ol{\alpha}, \ol{\gamma} \vdash H} (\id_{\rho
%  [\ol{\gamma := C'}] [\phi := \lambda \ol{A}. \setsem{\Gamma;
%      \ol{\beta}, \ol{\gamma} \vdash F} \rho [\ol{\beta := A}]
%    [\ol{\gamma := C'}]]} [\ol{\alpha := f}] \\
%&\hspace{3em} \circ \id_{\rho [\ol{\alpha := B}] [\phi := \lambda
%    \ol{A}. \setsem{\Gamma; \ol{\beta}, \ol{\gamma} \vdash F} \rho
%       [\ol{\beta := A}] [\ol{\gamma := C'}]]} [\ol{\gamma := g}] \\
%&\hspace{3em} \circ \id_{\rho [\ol{\alpha := B}] [\ol{\gamma := C}]}
%[\phi := \lambda \ol{A}. \setsem{\Gamma; \ol{\beta}, \ol{\gamma}
%    \vdash F} \id_{\rho [\ol{\beta := A}]}[\ol{\gamma := g}]] ) \\
=& \setsem{\Gamma; \phi, \ol{\alpha}, \ol{\gamma} \vdash H} \id_{\rho
  [\ol{\gamma := C'}][\phi := \lambda \ol{A}. \setsem{\Gamma;
      \ol{\beta}, \ol{\gamma} \vdash F} \rho [\ol{\beta := A}]
    [\ol{\gamma := C'}]]} [\ol{\alpha := f}] \\
&\hspace{3em} \circ \setsem{\Gamma; \phi, \ol{\alpha}, \ol{\gamma}
  \vdash H} \id_{\rho [\ol{\alpha := B}][\phi := \lambda
    \ol{A}. \setsem{\Gamma; \ol{\beta}, \ol{\gamma} \vdash F} \rho
       [\ol{\beta := A}] [\ol{\gamma := C'}]]} [\ol{\gamma := g}] \\
&\hspace{3em} \circ \setsem{\Gamma; \phi, \ol{\alpha}, \ol{\gamma}
  \vdash H} \id_{\rho [\ol{\alpha := B}] [\ol{\gamma := C}]} [\phi :=
  \lambda \ol{A}. \setsem{\Gamma; \ol{\beta}, \ol{\gamma} \vdash F}
  \id_{\rho [\ol{\beta := A}]} [\ol{\gamma := g}]] \\
= & T^{\set}_{H,\rho [\ol{\gamma := C'}]} (\lambda
\ol{A}. \setsem{\Gamma; \ol{\beta}, \ol{\gamma} \vdash F} \rho
   [\ol{\beta := A}] [\ol{\gamma := C'}]) \ol{f} \\
&\hspace{3em} \circ \big( T^{\set}_{H,\id_{\rho}[\ol{\gamma := g}]}
   (\lambda \ol{A}. \setsem{\Gamma; \ol{\beta}, \ol{\gamma} \vdash F}
   \rho [\ol{\beta := A}] [\ol{\gamma := C'}]) \big)_{\ol{B}} \\
&\hspace{3em} \circ \big( T^{\set}_{H,\rho [\ol{\gamma := C}]} (\lambda
   \ol{A}. \setsem{\Gamma; \ol{\beta}, \ol{\gamma} \vdash F} \id_{\rho
     [\ol{\beta := A}]} [\ol{\gamma := g}]) \big)_{\ol{B}}
\end{array}\]
So if $\eta$ is a natural transformation such that $\eta_{\ol B, \ol
  C}$ has type
\[
%\lambda \ol{B}\,\ol{C}.
\setsem{\Gamma; \ol{\alpha},
  \ol{\gamma} \vdash H[\phi := F][\ol{\alpha := \beta}]}\rho
[\ol{\beta := B}] [\ol{\gamma := C}]
%\]          to \[
\to
%\lambda \ol{B}\, \ol{C}.
\setsem{\Gamma; \ol{\beta},
   \ol{\gamma} \vdash F}\rho [\ol{\beta := B}] [\ol{\gamma := C}]\]
then, by naturality,
\[\begin{array}{ll}
 & \setsem{\Gamma; \ol{\beta}, \ol{\gamma} \vdash F} \id_{\rho}
       [\ol{\beta := f}] [\ol{\gamma := g}] \circ \eta_{\ol{B},
         \ol{C}} \\ 
%= & \eta_{\ol{B'}, \ol{C'}} \circ \setsem{\Gamma; \ol{\alpha},
%  \ol{\gamma} \vdash H[\phi := F][\ol{\alpha := \beta}]} \id_{\rho}
%[\ol{\beta := f}] [\ol{\gamma := g}] \\ 
= & \eta_{\ol{B'}, \ol{C'}} \circ T^{\set}_{\rho [\ol{\gamma := C'}]}
(\lambda \ol{A}. \setsem{\Gamma; \ol{\beta}, \ol{\gamma} \vdash F}
\rho [\ol{\beta := A}] [\ol{\gamma := C'}]) \ol{f} \\
& \circ \big( T^{\set}_{H,\id_{\rho}[\ol{\gamma := g}]} (\lambda
\ol{A}. \setsem{\Gamma; \ol{\beta}, \ol{\gamma} \vdash F} \rho
   [\ol{\beta := A}] [\ol{\gamma := C'}]) \big)_{\ol{B}} \\
& \circ \big( T^{\set}_{\rho [\ol{\gamma := C}]} (\lambda
   \ol{A}. \setsem{\Gamma; \ol{\beta}, \ol{\gamma} \vdash F} \id_{\rho
     [\ol{\beta := A}]} [\ol{\gamma := g}]) \big)_{\ol{B}}
\end{array}\]
As a special case when $\ol{f = \id_B}$ we have
%\[\begin{array}{ll}
% & \setsem{\Gamma; \ol{\beta}, \ol{\gamma} \vdash F} \id_{\rho [\ol{\beta
%      := B}]} [\ol{\gamma := g}] \circ \eta_{\ol{B}, \ol{C}}\\
%& = \eta_{\ol{B}, \ol{C'}} \circ \big(
%T^{\set}_{H,\id_{\rho}[\ol{\gamma := g}]} (\lambda
%\ol{A}. \setsem{\Gamma; \ol{\beta}, \ol{\gamma} \vdash F} \rho
%   [\ol{\beta := A}] [\ol{\gamma := C'}]) \big)_{\ol{B}} \\
%& \hspace*{0.32in} \circ \big(
%   T^{\set}_{\rho [\ol{\gamma := C}]} (\lambda \ol{A}. \setsem{\Gamma;
%     \ol{\beta}, \ol{\gamma} \vdash F} \id_{\rho [\ol{\beta := A}]}
%   [\ol{\gamma := g}]) \big)_{\ol{B}}
%\end{array}\]
%i.e., 
\begin{equation}\label{eq:T-sigma-functor}
\begin{split}
  & \lambda \ol{B}. \setsem{\Gamma; \ol{\beta}, \ol{\gamma} \vdash F}
\id_{\rho [\ol{\beta := B}]} [\ol{\gamma := g}] \circ \lambda
\ol{B}.\eta_{\ol{B}, \ol{C}} \\  
=\;\;\; & \lambda \ol{B}.\eta_{\ol{B}, \ol{C'}} \circ
\sigma^{\set}_{\id_{\rho}[\ol{\gamma := g}]} (\lambda
\ol{A}. \setsem{\Gamma; \ol{\beta}, \ol{\gamma} \vdash F} \rho
   [\ol{\beta := A}] [\ol{\gamma := C'}]) \\
 & \hspace*{0.5in} \circ
   T^{\set}_{\rho [\ol{\gamma := C}]} (\lambda \ol{A}. \setsem{\Gamma;
     \ol{\beta}, \ol{\gamma} \vdash F} \id_{\rho [\ol{\beta := A}]}
   [\ol{\gamma := g}]) 
\end{split}
\end{equation}
Finally, to see that the top diagram in the diagram on
page~\pageref{page:dia1} commutes we first note that functoriality of
$T^{\set}_{H,\rho [\ol{\gamma := C}]}$, naturality of
$T^{\set}_{H,\id_{\rho}[\ol{\gamma := g}]}$, the universal property of
$\fold_{T^{\set}_{H,\rho [\ol{\gamma := C'}]}} (\lambda
\ol{A}. \eta_{\ol{A}, \ol{C'}})$ and Equation~\ref{eq:mu-sigma-def}
ensure that the following diagram commutes: {\footnotesize
\begin{equation}\label{eq:one-side}
  \begin{tikzcd}[column sep = huge, row sep = huge]
T^{\set}_{H,\rho [\ol{\gamma := C}]} (\mu T^{\set}_{H,\rho [\ol{\gamma :=
      C}]}) \ar[rr, "{T^{\set}_{H,\rho [\ol{\gamma := C}]} (
    \fold_{T^{\set}_{H,\rho [\ol{\gamma := C'}]}} (\lambda
    \ol{A}. \eta_{\ol{A}, \ol{C'}}) \circ \mu
    T^{\set}_{H,\id_{\rho}[\ol{\gamma := g}]} )}"] \ar[dd,
  "{\tin_{T^{\set}_{H,\rho [\ol{\gamma := C}]}}}"']
&& T^{\set}_{H,\rho [\ol{\gamma := C}]} (\lambda \ol{B}. \setsem{\Gamma;
  \ol{\beta}, \ol{\gamma} \vdash F} \rho [\ol{\beta := B}] [\ol{\gamma
    := C'}]) \ar[d, "{ T^{\set}_{H,\id_{\rho}[\ol{\gamma := g}]}
    (\lambda \ol{B}. \setsem{\Gamma; \ol{\beta}, \ol{\gamma} \vdash F}
    \rho [\ol{\beta := B}] [\ol{\gamma := C'}]) }" description] \\
&& T^{\set}_{H,\rho [\ol{\gamma := C'}]} (\lambda
\ol{B}. \setsem{\Gamma; \ol{\beta}, \ol{\gamma} \vdash F} \rho
   [\ol{\beta := B}][\ol{\gamma := C'}]) \ar[d, "{ \lambda \ol{A}. \eta_{\ol{A}, \ol{C'}}
     }" description] \\
\mu T^{\set}_{H,\rho [\ol{\gamma := C}]}\ar[r, "{\mu
    T^{\set}_{H,\id_{\rho}[\ol{\gamma := g}]}}"'] 
&\mu T^{\set}_{H,\rho [\ol{\gamma := C'}]} \ar[r,
  "{\fold_{T^{\set}_{H,\rho [\ol{\gamma := C'}]}} (\lambda
    \ol{A}. \eta_{\ol{A}, \ol{C'}})}"']
& \lambda \ol{B}. \setsem{\Gamma; \ol{\beta}, \ol{\gamma} \vdash F}
\rho [\ol{\beta := B}] [\ol{\gamma := C'}]
  \end{tikzcd}
  \end{equation}}
%commutes
\begin{comment}
because
\[\begin{array}{ll}
& \lambda \ol{A}. \eta_{\ol{A}, \ol{C'}} \circ
\sigma^{\set}_{\id_{\rho}[\ol{\gamma := g}]} (\lambda
\ol{B}. \setsem{\Gamma; \ol{\beta}, \ol{\gamma} \vdash F} \rho
   [\ol{\beta := B}] [\ol{\gamma := C'}])\\
& \hspace{3.6em}   \circ T^{\set}_{\rho
     [\ol{\gamma := C}]} ( \fold_{T^{\set}_{\rho [\ol{\gamma := C'}]}}
   (\lambda \ol{A}. \eta_{\ol{A}, \ol{C'}}) \circ \mu
   \sigma^{\set}_{\id_{\rho}[\ol{\gamma := g}]} ) \\
= & \lambda \ol{A}. \eta_{\ol{A}, \ol{C'}} \circ
\sigma^{\set}_{\id_{\rho}[\ol{\gamma := g}]} (\lambda
\ol{B}. \setsem{\Gamma; \ol{\beta}, \ol{\gamma} \vdash F} \rho
   [\ol{\beta := B}] [\ol{\gamma := C'}]) \\
&\hspace{3.6em} \circ T^{\set}_{\rho [\ol{\gamma := C}]} (
   \fold_{T^{\set}_{\rho [\ol{\gamma := C'}]}} (\lambda
   \ol{A}. \eta_{\ol{A}, \ol{C'}}) ) \circ T^{\set}_{\rho [\ol{\gamma
         := C}]} ( \mu \sigma^{\set}_{\id_{\rho}[\ol{\gamma := g}]} )  \\
= & \lambda \ol{A}. \eta_{\ol{A}, \ol{C'}} \circ T^{\set}_{\rho
     [\ol{\gamma := C'}]} ( \fold_{T^{\set}_{\rho [\ol{\gamma :=
           C'}]}} (\lambda \ol{A}. \eta_{\ol{A}, \ol{C'}}) ) \circ
   \sigma^{\set}_{\id_{\rho}[\ol{\gamma := g}]} (\mu T^{\set}_{\rho
     [\ol{\gamma := C'}]}) \circ T^{\set}_{\rho [\ol{\gamma := C}]} (
   \mu \sigma^{\set}_{\id_{\rho}[\ol{\gamma := g}]} ) \\
= & \fold_{T^{\set}_{\rho [\ol{\gamma := C'}]}} (\lambda
\ol{A}. \eta_{\ol{A}, \ol{C'}}) \circ \tin_{T^{\set}_{\rho [\ol{\gamma
        := C'}]}} \circ \sigma^{\set}_{\id_{\rho}[\ol{\gamma := g}]}
(\mu T^{\set}_{\rho [\ol{\gamma := C'}]}) \circ T^{\set}_{\rho
  [\ol{\gamma := C}]} ( \mu \sigma^{\set}_{\id_{\rho}[\ol{\gamma :=
      g}]} ) \\
= &\fold_{T^{\set}_{\rho [\ol{\gamma := C'}]}} (\lambda
\ol{A}. \eta_{\ol{A}, \ol{C'}}) \circ \mu
\sigma^{\set}_{\id_{\rho}[\ol{\gamma := g}]} \circ
\tin_{T^{\set}_{\rho [\ol{\gamma := C}]}}
\end{array}\]
Here, the first equality is by functoriality of $T^{\set}_{\rho
  [\ol{\gamma := C}]}$, the second equality is by naturality of
$\sigma^{\set}_{\id_{\rho}[\ol{\gamma := g}]}$, the third equality by
the universal property of $\fold_{T^{\set}_{\rho [\ol{\gamma := C'}]}}
(\lambda \ol{A}. \eta_{\ol{A}, \ol{C'}})$ and the last equality by
Equation~\ref{eq:mu-sigma-def}.
\end{comment}
%That is, we have
%\begin{equation}\label{eq:one-side}
%\begin{split}
% & \fold_{T^{\set}_{\rho [\ol{\gamma := C'}]}} (\lambda
%\ol{A}. \eta_{\ol{A}, \ol{C'}}) \circ \mu
%\sigma^{\set}_{\id_{\rho}[\ol{\gamma := g}]}\\
%=\;\;\; & \fold_{T^{\set}_{\rho [\ol{\gamma := C}]}} ( \lambda
%\ol{A}. \eta_{\ol{A}, \ol{C'}} \circ
%\sigma^{\set}_{\id_{\rho}[\ol{\gamma := g}]} (\lambda
%\ol{B}. \setsem{\Gamma; \ol{\beta}, \ol{\gamma} \vdash F} \rho
%   [\ol{\beta := B}] [\ol{\gamma := C'}]) )
%\end{split}
%\end{equation}
Next, we note that functoriality of $T^{\set}_{H,\rho [\ol{\gamma :=
      C}]}$, Equation~\ref{eq:T-sigma-functor}, and the universal
property of $\fold_{T^{\set}_{H,\rho [\ol{\gamma := C}]}} (\lambda
\ol{A}. \eta_{\ol{A}, \ol{C}})$ ensure that the following diagram
commutes: {\footnotesize
\begin{equation}\label{eq:other-side}
\begin{tikzcd}[row sep = huge]
T^{\set}_{\rho [\ol{\gamma := C}]} (\mu T^{\set}_{\rho [\ol{\gamma :=
      C}]}) \ar[rr, "{T^{\set}_{\rho [\ol{\gamma := C}]} (
    \lambda \ol{B}.\setsem{\Gamma; \ol{\beta}, \ol{\gamma} \vdash F} \id_{\rho
      [\ol{\beta := B}]} [\ol{\gamma := g}] \circ
    \fold_{T^{\set}_{\rho [\ol{\gamma := C}]}} (\lambda
    \ol{A}. \eta_{\ol{A}, \ol{C}}) )}"] \ar[dd, "{\tin_{T^{\set}_{\rho
        [\ol{\gamma := C}]}}}"']
&& T^{\set}_{\rho [\ol{\gamma := C}]} (\lambda \ol{B}.\setsem{\Gamma; \ol{\beta},
  \ol{\gamma} \vdash F} \rho [\ol{\beta := B}] [\ol{\gamma := C'}])
\ar[d, "{ \sigma^{\set}_{\id_{\rho}[\ol{\gamma := g}]}
    (\lambda \ol{B}.\setsem{\Gamma; \ol{\beta}, \ol{\gamma} \vdash F} \rho [\ol{\beta
        := B}] [\ol{\gamma := C'}]) }" description]\\
&& T^{\set}_{\rho [\ol{\gamma := C'}]} (\lambda \ol{B}.\setsem{\Gamma; \ol{\beta},
  \ol{\gamma} \vdash F} \rho [\ol{\beta := B}][\ol{\gamma := C'}]) \ar[d, "{ \lambda
    \ol{A}. \eta_{\ol{A}, \ol{C'}} }" description] \\
\mu T^{\set}_{\rho [\ol{\gamma := C}]} \ar[r, bend right = 10,
  "{\fold_{T^{\set}_{\rho [\ol{\gamma := C}]}} (\lambda
    \ol{A}. \eta_{\ol{A}, \ol{C}})}"'] & \lambda \ol{B}.\setsem{\Gamma; \ol{\beta},
  \ol{\gamma} \vdash F} \rho [\ol{\beta := B}] [\ol{\gamma := C}]
\ar[r, bend right = 10, "{\lambda \ol{B}.\setsem{\Gamma; \ol{\beta}, \ol{\gamma}
      \vdash F} \id_{\rho [\ol{\beta := B}]} [\ol{\gamma := g}]}"'] &
\lambda \ol{B}.\setsem{\Gamma; \ol{\beta}, \ol{\gamma} \vdash F} \rho [\ol{\beta :=
    B}] [\ol{\gamma := C'}]
\end{tikzcd}
\end{equation}}
%commutes because
%\begin{align*}
%& \lambda \ol{A}. \eta_{\ol{A}, \ol{C'}} \circ
%  \sigma^{\set}_{\id_{\rho}[\ol{\gamma := g}]} (\lambda
%  \ol{B}.\setsem{\Gamma; \ol{\beta}, \ol{\gamma} \vdash F} \rho
%     [\ol{\beta := B}] [\ol{\gamma := C'}]) \\
%&\hspace{3em} \circ T^{\set}_{\rho [\ol{\gamma := C}]} ( \lambda
%     \ol{B}. \setsem{\Gamma; \ol{\beta}, \ol{\gamma} \vdash F}
%     \id_{\rho [\ol{\beta := B}]} [\ol{\gamma := g}] \circ
%     \fold_{T^{\set}_{\rho [\ol{\gamma := C}]}} (\lambda
%     \ol{A}. \eta_{\ol{A}, \ol{C}}) ) \\
%&= \lambda \ol{A}. \eta_{\ol{A}, \ol{C'}} \circ
%     \sigma^{\set}_{\id_{\rho}[\ol{\gamma := g}]} (\lambda
%     \ol{B}. \setsem{\Gamma; \ol{\beta}, \ol{\gamma} \vdash F} \rho
%        [\ol{\beta := B}] [\ol{\gamma := C'}]) \\
%&\hspace{3em} \circ T^{\set}_{\rho [\ol{\gamma := C}]} ( \lambda
%        \ol{B}. \setsem{\Gamma; \ol{\beta}, \ol{\gamma} \vdash F}
%        \id_{\rho [\ol{\beta := B}]} [\ol{\gamma := g}] ) \circ
%        T^{\set}_{\rho [\ol{\gamma := C}]} ( \fold_{T^{\set}_{\rho
%            [\ol{\gamma := C}]}} (\lambda \ol{A}. \eta_{\ol{A},
%          \ol{C}}) ) \\
% &= \lambda \ol{B}. \setsem{\Gamma; \ol{\beta}, \ol{\gamma} \vdash F}
%        \id_{\rho [\ol{\beta := B}]} [\ol{\gamma := g}] \circ \lambda
%        \ol{A}. \eta_{\ol{A}, \ol{C}} \circ T^{\set}_{\rho [\ol{\gamma
%              := C}]} ( \fold_{T^{\set}_{\rho [\ol{\gamma := C}]}}
%        (\lambda \ol{A}. \eta_{\ol{A}, \ol{C}}) ) \\
%&= \lambda \ol{B}. \setsem{\Gamma; \ol{\beta}, \ol{\gamma} \vdash F}
%        \id_{\rho [\ol{\beta := B}]} [\ol{\gamma := g}] \circ
%        \fold_{T^{\set}_{\rho [\ol{\gamma := C}]}} (\lambda
%        \ol{A}. \eta_{\ol{A}, \ol{C}}) \circ \tin_{T^{\set}_{\rho
%            [\ol{\gamma := C}]}}
%\end{align*}
%Here, the first equality is by functoriality of $T^{\set}_{\rho
%  [\ol{\gamma := C}]}$, the second equality is by
%Equation~\ref{eq:T-sigma-functor}, and the last equality is by the
%universal property of $\fold_{T^{\set}_{\rho [\ol{\gamma := C}]}}
%(\lambda \ol{A}. \eta_{\ol{A}, \ol{C}})$. That is, we have
%\begin{equation}\label{eq:other-side}
%\begin{split}
%  & \lambda \ol{B}. \setsem{\Gamma; \ol{\beta}, \ol{\gamma} \vdash F}
%\id_{\rho [\ol{\beta := B}]} [\ol{\gamma := g}] \circ
%\fold_{T^{\set}_{\rho [\ol{\gamma := C}]}} (\lambda
%\ol{A}. \eta_{\ol{A}, \ol{C}}) \\ 
%= \;\;\;& \fold_{T^{\set}_{\rho [\ol{\gamma := C}]}} (\lambda
%\ol{A}. \eta_{\ol{A}, \ol{C'}} \circ
%\sigma^{\set}_{\id_{\rho}[\ol{\gamma := g}]} (\lambda
%\ol{B}. \setsem{\Gamma; \ol{\beta}, \ol{\gamma} \vdash F} \rho
%   [\ol{\beta := B}] [\ol{\gamma := C'}]) ) 
%\end{split}
%\end{equation}
Combining the equations entailed by~\ref{eq:one-side}
and~\ref{eq:other-side}, we get that
%\[\fold_{T^{\set}_{\rho [\ol{\gamma := C'}]}} (\lambda \ol{A}. \eta_{\ol{A}, \ol{C'}})
%  \circ \mu \sigma^{\set}_{\id_{\rho}[\ol{\gamma := g}]} = \lambda
%  \ol{B}. \setsem{\Gamma; \ol{\beta}, \ol{\gamma} \vdash F} \id_{\rho
%    [\ol{\beta := B}]} [\ol{\gamma := g}] \circ \fold_{T^{\set}_{\rho
%      [\ol{\gamma := C}]}} (\lambda \ol{A}. \eta_{\ol{A}, \ol{C}})\]
%  i.e., that
the top diagram in the diagram on
  page~\pageref{page:dia1} commutes, as desired.
%We therefore have that
%\[(\setsem{\Gamma; \emptyset~|~\emptyset
%  \vdash \fold^F_H : \Nat^\emptyset\;(\Nat^{\ol{\beta},
%    \ol{\gamma}}\,H[\phi :=_{\ol{\beta}} F][\ol{\alpha :=
%      \beta}]\,F)\; (\Nat^{{\ol{\beta},\ol{\gamma}} }\,(\mu
%  \phi.\lambda \overline \alpha.H)\overline
%  \beta\,F}\,\rho\,d\,\eta)_{\ol{B}\,\ol{C}}\] is natural in $\ol{B}$
%and $\ol{C}$ as desired.
To see that, for all $\rho : \setenv$, $d \in
\setsem{\Gamma; \emptyset \vdash \emptyset}\rho$, and $\eta :
\setsem{\Gamma; \emptyset \vdash \Nat^{\ol{\beta},
    \ol{\gamma}}\,H[\phi :=_{\ol{\beta}} F][\ol{\alpha := \beta}]\,F}
\rho$,
\[\setsem{\Gamma; \emptyset~|~\emptyset
  \vdash \fold^F_H : \Nat^\emptyset\;(\Nat^{\ol{\beta},
    \ol{\gamma}}\,H[\phi :=_{\ol{\beta}} F][\ol{\alpha :=
      \beta}]\,F)\; (\Nat^{{\ol{\beta},\ol{\gamma}} }\,(\mu
  \phi.\lambda \overline \alpha.H)\overline \beta\,F}\,\rho\,d\,\eta\]
%\[\setsem{\Gamma;\emptyset \,|\, \Delta \vdash \fold^F_H\, t :
%    \Nat^{{\ol{\beta},\ol{\gamma}} }\,(\mu \phi.\lambda \overline
%    \alpha.H)\overline \beta\;F}\,\rho\,d\]
satisfies the additional condition needed for it to be in
$\setsem{\Gamma;\emptyset \vdash \Nat^{{\ol{\beta},\ol{\gamma}}
  }\,(\mu \phi.\lambda \overline \alpha.H)\overline \beta\;F}\,\rho$,
let $\ol{R : \rel(B,B')}$ and $\ol{S : \rel(C,C')}$.  Since $\eta$
satisfies the additional condition needed for it to be in
$\setsem{\Gamma; \emptyset \vdash \Nat^{\ol{\beta},
    \ol{\gamma}}\,(H[\phi := F][\ol{\alpha := \beta}])\,F} \rho$,
%--- i.e., since 
%\[\begin{array}{lll}
% (\eta_{\ol{B}\,\ol{C}}\,,
%\eta_{\ol{B'}\,\ol{C'}}) & \in & 
%\relsem{\Gamma;\ol{\gamma},\ol{\beta} \vdash H[\phi := F][\ol{\alpha
%      := \beta}]}\Eq_\rho[\ol{\gamma := S}][\ol{\beta := R}] \to\\
% &  & \hspace*{1in}
%\relsem{\Gamma;\ol{\gamma},\ol{\beta} \vdash F}\Eq_\rho[\ol{\gamma
%    := S}][\ol{\beta := R}]\\
%& = & T_{\Eq_\rho[\ol{\gamma := S}]}\;
%(\relsem{\Gamma;\ol{\gamma},\ol{\beta} \vdash F}\Eq_\rho[\ol{\gamma :=
%    S}][\ol{\beta := R}]) \to\\ 
%&  & \hspace*{1in}
%\relsem{\Gamma;\ol{\gamma},\ol{\beta} \vdash F}\Eq_\rho[\ol{\gamma
%    := S}][\ol{\beta := R}]
%\end{array}\]
%---
\[\begin{array}{ll}
 & (\,(\mathit{fold}_{T^\set_{H,\rho[\ol{\gamma :=
        C}]}}\,(\lambda \ol{A}.\,\eta_{\ol{A}\,\ol{C}}))_{\ol{B}},\,
(\mathit{fold}_{T^\set_{H,\rho[\ol{\gamma :=
        C'}]}}\,(\lambda \ol{A}.\eta_{\ol{A}\,\ol{C'}}))_{\ol{B'}}\,) 
\end{array}\]
has type
\[\begin{array}{ll}
  & (\mu T_{H,\Eq_\rho[\ol{\gamma := S}]}) \,\ol{R} \to
\relsem{\Gamma;\ol{\gamma},\ol{\beta} \vdash F}\Eq_\rho[\ol{\gamma := 
    S}][\ol{\beta:= R}]\\ 
= & (\mu T_{H,\Eq_\rho[\ol{\gamma :=
      S}]})\,\ol{\relsem{\Gamma;\ol{\gamma},\ol{\beta} 
  \vdash \beta}\Eq_\rho[\ol{\gamma := S}][\ol{\beta := R}]} \to
\relsem{\Gamma;\ol{\gamma},\ol{\beta} \vdash F}\Eq_\rho[\ol{\gamma := 
    S}][\ol{\beta:= R}]\\ 
= & \relsem{\Gamma; \ol{\gamma},\ol{\beta} \vdash (\mu \phi. \lambda
  \ol{\alpha}. H)\ol{\beta}}\Eq_\rho[\ol{\gamma := S}][\ol{\beta := R}] \to
\relsem{\Gamma;\ol{\gamma},\ol{\beta} \vdash F}\Eq_\rho[\ol{\gamma := 
    S}][\ol{\beta:= R}]
\end{array}\]
\begin{comment}
\item The proofs that
\[\relsem{\Gamma; \emptyset~|~\emptyset \vdash
  \fold^F_H : \Nat^\emptyset\;(\Nat^{\ol{\beta}, \ol{\gamma}}\,H[\phi
    :=_{\ol{\beta}} F][\ol{\alpha := \beta}]\,F)\;
  (\Nat^{{\ol{\beta},\ol{\gamma}} }\,(\mu \phi.\lambda \overline
  \alpha.H)\overline \beta\,F)}\] is a natural transformation from
  $\relsem{\Gamma;\emptyset \vdash \emptyset}$ to
  \[\relsem{\Gamma ; \emptyset \vdash
    \Nat^\emptyset\;(\Nat^{\ol{\beta}, \ol{\gamma}}\,H[\phi 
    :=_{\ol{\beta}} F][\ol{\alpha := \beta}]\,F)\;
    (\Nat^{{\ol{\beta},\ol{\gamma}} }\,(\mu \phi.\lambda \overline
    \alpha.H)\overline \beta\,F)}\] and that, for all $\rho : \relenv$
  and the unique $d : \relsem{\Gamma;\emptyset \vdash \emptyset}\rho$,
\[\relsem{\Gamma; \emptyset~|~\emptyset \vdash
  \fold^F_H : \Nat^\emptyset\;(\Nat^{\ol{\beta}, \ol{\gamma}}\,H[\phi
    :=_{\ol{\beta}} F][\ol{\alpha := \beta}]\,F)\;
  (\Nat^{{\ol{\beta},\ol{\gamma}} }\,(\mu \phi.\lambda \overline
  \alpha.H)\overline \beta\,F)}\,\rho\,d\] is a morphism from
$\relsem{\Gamma; \emptyset \vdash \Nat^{\ol{\beta},
    \ol{\gamma}}\,H[\phi :=_{\ol{\beta}} F][\ol{\alpha := \beta}]\,F)}
\rho$ to $\relsem{\Gamma; \emptyset \vdash
  \Nat^{{\ol{\beta},\ol{\gamma}} }\,(\mu \phi.\lambda \overline
  \alpha.H)\overline \beta\,F}\rho$, are analogous.
\item Finally, to see that
\[\begin{array}{ll}
& \pi_i(\relsem{\Gamma; \emptyset~|~\emptyset \vdash
  \fold^F_H : \Nat^\emptyset\;(\Nat^{\ol{\beta}, \ol{\gamma}}\,H[\phi
    :=_{\ol{\beta}} F][\ol{\alpha := \beta}]\,F)\;
  (\Nat^{{\ol{\beta},\ol{\gamma}} }\,(\mu \phi.\lambda \overline
  \alpha.H)\overline \beta\,F)} \rho)\\
= & \setsem{\Gamma; \emptyset~|~\emptyset \vdash
  \fold^F_H : \Nat^\emptyset\;(\Nat^{\ol{\beta}, \ol{\gamma}}\,H[\phi
    :=_{\ol{\beta}} F][\ol{\alpha := \beta}]\,F)\;
  (\Nat^{{\ol{\beta},\ol{\gamma}} }\,(\mu \phi.\lambda \overline
  \alpha.H)\overline \beta\,F)}(\pi_i\rho)
\end{array}\]
we compute
\[\begin{array}{ll}
 & \pi_i(\relsem{\Gamma; \emptyset~|~\emptyset \vdash
  \fold^F_H : \Nat^\emptyset\;(\Nat^{\ol{\beta}, \ol{\gamma}}\,H[\phi
    :=_{\ol{\beta}} F][\ol{\alpha := \beta}]\,F)\;
  (\Nat^{{\ol{\beta},\ol{\gamma}} }\,(\mu \phi.\lambda \overline
  \alpha.H)\overline \beta\,F)})\\
=& \pi_i (\lambda
e\,\eta\,\ol{R}\,\ol{S}.\,(\mathit{fold}_{T_{\rho[\ol{\gamma 
        := S}]}}\,(\lambda \ol{Z}.\,\eta_{\ol{Z}\,\ol{S}}))_{\ol{R}})\\
=& \lambda
e\,\eta\,\ol{R}\,\ol{S}.\,(\mathit{fold}_{T_{(\pi_i\rho)[\ol{\gamma 
        := \pi_i S}]}}\,(\lambda \ol{Z}.\, (\pi_i
\eta)_{\ol{\pi_iZ}\,\ol{\pi_iS}}))_{\ol{\pi_i R}}\\ 
=& \lambda
d\,\eta\,\ol{B}\,\ol{C}.\,(\mathit{fold}_{T_{(\pi_i\rho)[\ol{\gamma 
        := C}]}} \,(\lambda \ol{A}.\eta_{\ol{A}\,\ol{C}}))_{\ol{B}}\\
=& \relsem{\Gamma; \emptyset~|~\emptyset \vdash
  \fold^F_H : \Nat^\emptyset\;(\Nat^{\ol{\beta}, \ol{\gamma}}\,H[\phi
    :=_{\ol{\beta}} F][\ol{\alpha := \beta}]\,F)\;
  (\Nat^{{\ol{\beta},\ol{\gamma}} }\,(\mu \phi.\lambda \overline
  \alpha.H)\overline \beta\,F)}(\pi_i \rho)\\

\end{array}\]
Here, we are again using the fact that $\pi_1$ and $\pi_2$ are
surjective. 
\end{comment}
%\end{itemize}  
\end{itemize}
\end{proof}



\bibliography{references}

\end{document}


  
