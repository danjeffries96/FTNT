% For double-blind review submission, w/o CCS and ACM Reference (max
% submission space)
\documentclass[acmsmall,review,anonymous]{acmart}
\settopmatter{printfolios=true,printccs=false,printacmref=false}
%% For double-blind review submission, w/ CCS and ACM Reference
%\documentclass[acmsmall,review,anonymous]{acmart}\settopmatter{printfolios=true}
%% For single-blind review submission, w/o CCS and ACM Reference (max submission space)
%\documentclass[acmsmall,review]{acmart}\settopmatter{printfolios=true,printccs=false,printacmref=false}
%% For single-blind review submission, w/ CCS and ACM Reference
%\documentclass[acmsmall,review]{acmart}\settopmatter{printfolios=true}
%% For final camera-ready submission, w/ required CCS and ACM Reference
%\documentclass[acmsmall]{acmart}\settopmatter{}


%% Journal information
%% Supplied to authors by publisher for camera-ready submission;
%% use defaults for review submission.
\acmJournal{PACMPL}
\acmVolume{1}
\acmNumber{POPL} % CONF = POPL or ICFP or OOPSLA
\acmArticle{1}
\acmYear{2020}
\acmMonth{1}
\acmDOI{} % \acmDOI{10.1145/nnnnnnn.nnnnnnn}
\startPage{1}

%% Copyright information
%% Supplied to authors (based on authors' rights management selection;
%% see authors.acm.org) by publisher for camera-ready submission;
%% use 'none' for review submission.
\setcopyright{none}
%\setcopyright{acmcopyright}
%\setcopyright{acmlicensed}
%\setcopyright{rightsretained}
%\copyrightyear{2018}           %% If different from \acmYear

%% Bibliography style
\bibliographystyle{ACM-Reference-Format}
%% Citation style
%% Note: author/year citations are required for papers published as an
%% issue of PACMPL.
\citestyle{acmauthoryear}   %% For author/year citations
%\citestyle{acmnumeric}

%%%%%%%%%%%%%%%%%%%%%%%%%%%%%%%%%%%%%%%%%%%%%%%%%%%%%%%%%%%%%%%%%%%%%%
%% Note: Authors migrating a paper from PACMPL format to traditional
%% SIGPLAN proceedings format must update the '\documentclass' and
%% topmatter commands above; see 'acmart-sigplanproc-template.tex'.
%%%%%%%%%%%%%%%%%%%%%%%%%%%%%%%%%%%%%%%%%%%%%%%%%%%%%%%%%%%%%%%%%%%%%%



\usepackage[utf8]{inputenc}
\usepackage{ccicons}

\usepackage{amsmath}
\usepackage{amsthm}
\usepackage{amscd}
%\usepackage{MnSymbol}
\usepackage{xcolor}

\usepackage{bbold}
\usepackage{url}
\usepackage{upgreek}
%\usepackage{stmaryrd}

\usepackage{lipsum}
\usepackage{tikz-cd}
\usetikzlibrary{cd}
\usetikzlibrary{calc}
\usetikzlibrary{arrows}

\usepackage{bussproofs}
\EnableBpAbbreviations

\DeclareMathAlphabet{\mathpzc}{OT1}{pzc}{m}{it}

%\usepackage[amsmath]{ntheorem}

\newcommand{\lam}{\lambda}
\newcommand{\eps}{\varepsilon}
\newcommand{\ups}{\upsilon}
\newcommand{\mcB}{\mathcal{B}}
\newcommand{\mcD}{\mathcal{D}}
\newcommand{\mcE}{\mathcal{E}}
\newcommand{\mcF}{\mathcal{F}}
\newcommand{\mcP}{\mathcal{P}}
\newcommand{\mcI}{\mathcal{I}}
\newcommand{\mcJ}{\mathcal{J}}
\newcommand{\mcK}{\mathcal{K}}
\newcommand{\mcL}{\mathcal{L}}
\newcommand{\WW}{\mathcal{W}}

\newcommand{\ex}{\mcE_x}
\newcommand{\ey}{\mcE_y}
\newcommand{\bzero}{\boldsymbol{0}}
\newcommand{\bone}{{\boldsymbol{1}}}
\newcommand{\tB}{{\bone_\mcB}}
\newcommand{\tE}{{\bone_\mcE}}
\newcommand{\bt}{\mathbf{t}}
\newcommand{\bp}{\mathbf{p}}
\newcommand{\bsig}{\mathbf{\Sigma}}
\newcommand{\bpi}{\boldsymbol{\pi}}
\newcommand{\Empty}{\mathtt{Empty}}
\newcommand{\truthf}{\mathtt{t}}
\newcommand{\id}{id}
\newcommand{\coo}{\mathtt{coo\ }}
\newcommand{\mcC}{\mathcal{C}}
\newcommand{\Rec}{\mathpzc{Rec}}
\newcommand{\types}{\mathcal{T}}

%\newcommand{\Homrel}{\mathsf{Hom_{Rel}}}
\newcommand{\HomoCPOR}{\mathsf{Hom_{\oCPOR}}}

%\newcommand{\semof}[1]{\llbracket{#1}\rrbracket^\rel}
\newcommand{\sem}[1]{\llbracket{#1}\rrbracket}
\newcommand{\setsem}[1]{\llbracket{#1}\rrbracket^\set}
\newcommand{\relsem}[1]{\llbracket{#1}\rrbracket^\rel}
\newcommand{\dsem}[1]{\llbracket{#1}\rrbracket^{\mathsf D}}
\newcommand{\setenv}{\mathsf{SetEnv}}
\newcommand{\relenv}{\mathsf{RelEnv}}
\newcommand{\oCPOenv}{\mathsf{SetEnv}}
\newcommand{\oCPORenv}{\mathsf{RelEnv}}
\newcommand{\oCPOsem}[1]{\llbracket{#1}\rrbracket^{\set}}
\newcommand{\oCPORsem}[1]{\llbracket{#1}\rrbracket^{\rel}}
\newcommand{\denv}{\mathsf{DEnv}}

\newcommand{\rel}{\mathsf{Rel}}
\newcommand{\setof}[1]{\{#1\}}
\newcommand{\letin}[1]{\texttt{let }#1\texttt{ in }}
\newcommand{\comp}[1]{{\{#1\}}}
\newcommand{\bcomp}[1]{\{\![#1]\!\}}
\newcommand{\beps}{\boldsymbol{\eps}}
%\newcommand{\B}{\mcB}
%\newcommand{\Bo}{{|\mcB|}}

\newcommand{\lmt}{\longmapsto}
\newcommand{\RA}{\Rightarrow}
\newcommand{\LA}{\Leftarrow}
\newcommand{\rras}{\rightrightarrows}
\newcommand{\colim}[2]{{{\underrightarrow{\lim}}_{#1}{#2}}}
\newcommand{\lift}[1]{{#1}\,{\hat{} \; \hat{}}}
\newcommand{\graph}[1]{\langle {#1} \rangle}

\newcommand{\carAT}{\mathsf{car}({\mathcal A}^T)}
\newcommand{\isoAto}{\mathsf{Iso}({\mcA^\to})}
\newcommand{\falg}{\mathsf{Alg}_F}
\newcommand{\CC}{\mathsf{Pres}(\mathcal{A})}
\newcommand{\PP}{\mathcal{P}}
\newcommand{\DD}{D_{(A,B,f)}}
\newcommand{\from}{\leftarrow}
\newcommand{\upset}[1]{{#1}{\uparrow}}
\newcommand{\smupset}[1]{{#1}\!\uparrow}

\newcommand{\Coo}{\mathpzc{Coo}}
\newcommand{\code}{\#}
\newcommand{\nat}{\mathpzc{Nat}}

\newcommand{\eq}{\; = \;}
\newcommand{\of}{\; : \;}
\newcommand{\df}{\; := \;}
\newcommand{\bnf}{\; ::= \;}

\newcommand{\zmap}[1]{{\!{\between\!\!}_{#1}\!}}
\newcommand{\bSet}{\mathbf{Set}}

\newcommand{\dom}{\mathsf{dom}}
\newcommand{\cod}{\mathsf{cod}}
\newcommand{\adjoint}[2]{\mathrel{\mathop{\leftrightarrows}^{#1}_{#2}}}
\newcommand{\isofunc}{\mathpzc{Iso}}
\newcommand{\ebang}{{\eta_!}}
\newcommand{\lras}{\leftrightarrows}
\newcommand{\rlas}{\rightleftarrows}
\newcommand{\then}{\quad\Longrightarrow\quad}
\newcommand{\hookup}{\hookrightarrow}

\newcommand{\spanme}[5]{\begin{CD} #1 @<#2<< #3 @>#4>> #5 \end{CD}}
\newcommand{\spanm}[3]{\begin{CD} #1 @>#2>> #3\end{CD}}
\newcommand{\pushout}{\textsf{Pushout}}
\newcommand{\mospace}{\qquad\qquad\!\!\!\!}

\newcommand{\natur}[2]{#1 \propto #2}

\newcommand{\Tree}{\mathsf{Tree}\,}
\newcommand{\GRose}{\mathsf{GRose}\,}
\newcommand{\List}{\mathsf{List}\,}
\newcommand{\PTree}{\mathsf{PTree}\,}
\newcommand{\Bush}{\mathsf{Bush}\,}
\newcommand{\Forest}{\mathsf{Forest}\,}
\newcommand{\Lam}{\mathsf{Lam}\,}
\newcommand{\LamES}{\mathsf{Lam}^+}
\newcommand{\Expr}{\mathsf{Expr}\,}

\newcommand{\ListNil}{\mathsf{Nil}}
\newcommand{\ListCons}{\mathsf{Cons}}
\newcommand{\LamVar}{\mathsf{Var}}
\newcommand{\LamApp}{\mathsf{App}}
\newcommand{\LamAbs}{\mathsf{Abs}}
\newcommand{\LamSub}{\mathsf{Sub}}
\newcommand{\ExprConst}{\mathsf{Const}}
\newcommand{\ExprPair}{\mathsf{Pair}}
\newcommand{\ExprProj}{\mathsf{Proj}}
\newcommand{\ExprAbs}{\mathsf{Abs}}
\newcommand{\ExprApp}{\mathsf{App}}
\newcommand{\Ptree}{\mathsf{Ptree}}

\newcommand{\kinds}{\mathpzc{K}}
\newcommand{\tvars}{\mathbb{T}}
\newcommand{\fvars}{\mathbb{F}}
\newcommand{\consts}{\mathpzc{C}}
\newcommand{\Lan}{\mathsf{Lan}}
\newcommand{\zerot}{\mathbb{0}}
\newcommand{\onet}{\mathbb{1}}
\newcommand{\bool}{\mathbb{2}}
\renewcommand{\nat}{\mathbb{N}}
%\newcommand{\semof}[1]{[\![#1]\!]}
%\newcommand{\setsem}[1]{\llbracket{#1}\rrbracket^\set}
\newcommand{\predsem}[1]{\llbracket{#1}\rrbracket^\pred}
%\newcommand{\todot}{\stackrel{.}{\to}}
\newcommand{\todot}{\Rightarrow}
\newcommand{\bphi}{{\bm \phi}}

\newcommand{\bm}[1]{\boldsymbol{#1}}

\newcommand{\cL}{\mathcal{L}}
\newcommand{\T}{\mathcal{T}}
\newcommand{\Pos}{P\!}
%\newcommand{\Pos}{\mathcal{P}\!}
\newcommand{\Neg}{\mathcal{N}}
\newcommand{\Hf}{\mathcal{H}}
\newcommand{\V}{\mathbb{V}}
\newcommand{\I}{\mathcal{I}}
\newcommand{\Set}{\mathsf{Set}}
%\newcommand{\Nat}{\mathsf{Nat}}
\newcommand{\Homrel}{\mathsf{Hom_{Rel}}}
\newcommand{\CV}{\mathcal{CV}}
\newcommand{\lan}{\mathsf{Lan}}
\newcommand{\Id}{\mathit{Id}}
\newcommand{\mcA}{\mathcal{A}}
\newcommand{\inl}{\mathsf{inl}}
\newcommand{\inr}{\mathsf{inr}}
%\newcommand{\case}[3]{\mathsf{case}\,{#1}\,\mathsf{of}\,\{{#2};\,{#3}\}}
\newcommand{\tin}{\mathsf{in}}
\newcommand{\fold}{\mathsf{fold}}
\newcommand{\Eq}{\mathsf{Eq}}
\newcommand{\Hom}{\mathsf{Hom}}
\newcommand{\curry}{\mathsf{curry}}
\newcommand{\uncurry}{\mathsf{uncurry}}
\newcommand{\eval}{\mathsf{eval}}
\newcommand{\apply}{\mathsf{apply}}
\newcommand{\oCPO}{{\mathsf{Set}}}
\newcommand{\oCPOR}{{\mathsf{Rel}}}
\newcommand{\oCPORT}{{\mathsf{RT}}}

\newcommand{\ar}[1]{\##1}
\newcommand{\mcG}{\mathcal{G}}
\newcommand{\mcH}{\mathcal{H}}
\newcommand{\TV}{\mathpzc{V}}

\newcommand{\essim}[1]{\mathsf{EssIm}(#1)}
\newcommand{\hra}{\hookrightarrow}

\newcommand{\ol}[1]{\overline{#1}}
\newcommand{\ul}[1]{\underline{#1}}
\newcommand{\op}{\mathsf{op}}
\newcommand{\trige}{\trianglerighteq}
\newcommand{\trile}{\trianglelefteq}
\newcommand{\LFP}{\mathsf{LFP}}
\newcommand{\LAN}{\mathsf{Lan}}
%\newcommand{\Mu}{{\mu\hskip-4pt\mu}}
\newcommand{\Mu}{{\mu\hskip-5.5pt\mu}}
%\newcommand{\Mu}{\boldsymbol{\upmu}}
\newcommand{\Terms}{\mathpzc{Terms}}
\newcommand{\Ord}{\mathpzc{Ord}}
\newcommand{\Anote}[1]{{\color{blue} {#1}}}
\newcommand{\Pnote}[1]{{\color{red} {#1}}}

\newcommand{\greyout}[1]{{\color{gray} {#1}}}
\newcommand{\ora}[1]{\overrightarrow{#1}}

%\newcommand{\?}{{.\ }}
%\theoremheaderfont{\scshape}
%\theorembodyfont{\normalfont}
%\theoremseparator{.\ \ }
\newtheorem{thm}{Theorem}
\newtheorem{dfn}[thm]{Definition}
\newtheorem{prop}[thm]{Proposition}
\newtheorem{cor}[thm]{Corollary}
\newtheorem{lemma}[thm]{Lemma}
\newtheorem{rmk}[thm]{Remark}
\newtheorem{expl}[thm]{Example}
\newtheorem{notn}[thm]{Notation}
%\theoremstyle{nonumberplain}
%\theoremsymbol{\Box}


\theoremstyle{definition}
\newtheorem{exmpl}{Example}

\renewcommand{\greyout}[1]{} %{{\color{gray} {#1}}} -- toggle to remove greyed text

\newcommand{\emptyfun}{{[]}}
\newcommand{\cal}{\mathcal}
%\newcommand{\fold}{\mathit{fold}}
\newcommand{\F}{\mathcal{F}}
\renewcommand{\G}{\mathcal{G}}
\newcommand{\N}{\mathcal{N}}
\newcommand{\E}{\mathcal{E}}
\newcommand{\B}{\mathcal{B}}
\renewcommand{\P}{\mathcal{A}}
\newcommand{\pred}{\mathsf{Fam}}
\newcommand{\env}{\mathsf{Env}}
\newcommand{\set}{\mathsf{Set}}
\renewcommand{\S}{\mathcal S}
\renewcommand{\C}{\mathcal{C}}
\newcommand{\D}{\mathcal{D}}
\newcommand{\A}{\mathcal{A}}
\renewcommand{\id}{\mathit{id}}
\newcommand{\map}{\mathsf{map}}
\newcommand{\pid}{\underline{\mathit{id}}}
\newcommand{\pcirc}{\,\underline{\circ}\,}
\newcommand{\pzero}{\underline{0}}
\newcommand{\pone}{\underline{1}}
\newcommand{\psum}{\,\underline{+}\,}
%\newcommand{\inl}{\mathit{inL}\,}
%\newcommand{\inr}{\mathit{inR}\,}
\newcommand{\pinl}{\underline{\mathit{inL}}\,}
\newcommand{\pinr}{\underline{\mathit{inR}}\,}
\newcommand{\ptimes}{\,\underline{\times}\,}
\newcommand{\ppi}{\underline{\pi_1}}
\newcommand{\pppi}{\underline{\pi_2}}
\newcommand{\pmu}{\underline{\mu}}

\title[Free Theorems for Nested Types]{Free Theorems for 
Nested Types} %% [Short Title] is optional;
                                        %% when present, will be used in
                                        %% header instead of Full Title.
%\titlenote{with title note}             %% \titlenote is optional;
                                        %% can be repeated if necessary;
                                        %% contents suppressed with 'anonymous'
%\subtitle{Subtitle}                     %% \subtitle is optional
%\subtitlenote{with subtitle note}       %% \subtitlenote is optional;
                                        %% can be repeated if necessary;
                                        %% contents suppressed with 'anonymous'

\begin{document}
\maketitle
\section{Demotion Theorems}

\begin{thm}

If $\Gamma; \Phi, \phi^k \vdash \tau : \F$, then one can derive
$\Gamma, \psi^k; \Phi \vdash \tau[\phi^k :== \psi^k]$, 
where $\tau[\phi :== \psi]$ is the textual replacement of $\phi$ in $\tau$.
In other words, all occurences of $\phi\ol\sigma$ in $\tau$ become $\psi\ol\sigma$.
\end{thm}
\begin{proof}
By induction on the structure of $\tau$.
\begin{itemize}
  \item There is nothing to prove for types in $\T$ because their functorial contexts
        must be empty.
  \item Case $\Gamma; \Phi, \alpha \vdash \alpha : \F$. We must
    derive $\Gamma, \beta; \Phi \vdash \beta : \F$. 
    
    \[\begin{array}{c}
    \AXC{$\Gamma, \beta; \emptyset \vdash \beta : \T$}
    \UIC{$\Gamma, \beta; \emptyset \vdash \beta : \F$}
    \UIC{$\Gamma, \beta; \Phi   \vdash \beta : \F$}
    \DP
    \end{array}\]  \\

  \item Case $\Gamma; \Phi, \phi \vdash \onet :\F$,  $\Gamma; \Phi, \phi \vdash \zerot :\F$.
        We can immediately form the required judgments. \\

  \item Case $\Gamma; \Phi, \phi \vdash \phi \ol\tau : \F$. We must derive 
    $\Gamma, \psi; \Phi \vdash (\phi \ol\tau) [\phi :== \psi] : \F$. The induction hypothesis 
    gives that $\Gamma, \psi; \Phi \vdash \tau[\phi :== \psi] : \F$ for each $\tau$. 

    \[\begin{array}{c}
    \AXC{$\psi \in \{ \Gamma, \psi \} \cup \Phi : \F$}
      \AXC{$\ol{\Gamma, \psi; \Phi \vdash \tau[\phi :== \psi] : \F}$}
      \BIC{$\Gamma, \psi; \Phi \vdash \psi \ol{\tau[\phi :== \psi]} : \F$}
    \UIC{$\Gamma, \psi; \Phi \vdash (\phi \ol\tau)[\phi :== \psi] :\F$}
    \DP
    \end{array}\]  

    The case for $\Gamma; \Phi, \phi' \vdash \phi \ol\tau : \F$, i.e., the case in which
    the variable being ``demoted'' only appears in the arguments, 
    works by the same induction. \\

  \item Case $\Gamma; \Phi, \phi \vdash (\mu \phi'. \lambda \ol\alpha. H)\ol\tau : \F$. 
    We must derive $\Gamma, \psi; \Phi \vdash ((\mu \phi'. \lambda \ol\alpha. H)
     \ol\tau)[\phi :== \psi] : \F$. The induction hypothesis gives that 
     $\Gamma, \psi; \Phi \vdash \tau[\phi :== \psi] : \F$ for each $\tau$ and
     $\Gamma, \psi; \Phi, \ol\alpha, \phi' \vdash H[\phi :== \psi] : \F$.

    \[\begin{array}{c}
      \AXC{$\Gamma, \psi; \Phi, \ol\alpha, \phi' \vdash H[\phi :== \psi] : \F$}
      \AXC{$\ol{\Gamma, \psi; \Phi \vdash \tau[\phi :== \psi] : \F}$}
      \BIC{$\Gamma, \psi; \Phi \vdash (\mu \phi'. \lambda \ol\alpha. H[\phi :== \psi])
        \ol{\tau[\phi :== \psi]} : \F$}
      \UIC{$\Gamma, \psi; \Phi \vdash ((\mu \phi'. \lambda \ol\alpha. H)\ol\tau)[\phi :== \psi] : \F$}
    \DP
    \end{array}\]  \\

  \item Case $\Gamma; \Phi, \phi \vdash \sigma \times \tau : \F$. We must derive
    $\Gamma, \psi; \Phi \vdash (\sigma \times \tau) [\phi :== \psi] : \F$. 
    The induction hypothesis gives that
    $\Gamma, \psi; \Phi \vdash \sigma [\phi :== \psi]: \F$ and
    $\Gamma, \psi; \Phi \vdash \tau [\phi :== \psi]: \F$. 

    \[\begin{array}{c}
      \AXC{$\Gamma, \psi; \Phi \vdash \sigma [\phi :== \psi]: \F$}
      \AXC{$\Gamma, \psi; \Phi \vdash \tau   [\phi :== \psi]: \F$}
      \BIC{$\Gamma, \psi; \Phi \vdash \sigma [\phi :== \psi] \times \tau [\phi :== \psi]: \F$}
      \UIC{$\Gamma, \psi; \Phi \vdash (\sigma \times \tau) [\phi :== \psi]: \F$}
    \DP
    \end{array}\]  

  The case for $\sigma + \tau$ is analogous.

\end{itemize}

\end{proof}

      % $$
      % \setsem{\Gamma; \Phi, \phi \vdash ____ : \F} \rho = 
      % $$

Note that the next two theorems are proven by 
simultaneous induction. We are actually only interested in using Theorem~\ref{thm:demotion-objects}, 
but in order to prove the $\mu$ case for this theorem, we need Theorem~\ref{thm:demotion-morph} 
to show that two functors have equal actions on morphisms.


\begin{thm}\label{thm:demotion-objects}
If $\Gamma; \Phi, \phi \vdash  \tau : \F$, $\rho : \setenv$, and $\rho \phi = \rho \psi$, then
  $$\setsem{\Gamma; \Phi, \phi \vdash \tau} \rho = \setsem{\Gamma, \psi; \Phi \vdash \tau[\phi :== \psi] } \rho $$

Analogously, if $\rho :\relenv$, and $\rho \phi = \rho \psi$, then
$$\relsem{\Gamma; \Phi, \phi \vdash \tau} \rho = \relsem{\Gamma, \psi; \Phi \vdash \tau[\phi :== \psi] } \rho $$

\end{thm}
\begin{proof}
We prove the case for $\set$ by induction on the structure of $\tau$.
The case for $\rel$ proceeds analogously.
\begin{itemize}
  \item There is nothing to prove for types in $\T$ because their functorial contexts
        must be empty.
  \item Case $\Gamma; \Phi, \alpha \vdash \alpha : \F$. Given that $\rho \alpha = \rho \beta$, 
    \begin{align*}
         & \setsem{\Gamma; \Phi, \alpha \vdash \alpha } \rho \\
    = \; & \rho \alpha \\
    = \; & \rho \beta \\
    = \; & \setsem{\Gamma, \beta; \Phi \vdash \beta } \rho
    \end{align*}

  \item Case $\Gamma; \Phi, \phi \vdash \onet :\F$,  $\Gamma; \Phi, \phi \vdash \zerot :\F$.
    \begin{align*}
         & \setsem{\Gamma; \Phi, \phi \vdash \onet } \rho \\
    = \; & 1 \\
    = \; & \setsem{\Gamma, \psi; \Phi \vdash \onet } \rho
    \end{align*}

    Similarly for $\zerot$.

  \item Case $\Gamma; \Phi, \phi \vdash \phi \ol\tau : \F$. The induction hypothesis gives that 
    $$\setsem{\Gamma; \Phi, \phi \vdash \tau } \rho
      = \setsem{\Gamma, \psi; \Phi \vdash \tau[\phi :== \psi] } \rho$$ 
    for each $\tau$.
    \begin{align*}
         & \setsem{\Gamma; \Phi, \phi \vdash \phi \ol\tau } \rho \\
      = \; & (\rho \phi) \ol{\setsem{\Gamma; \Phi, \phi \vdash \tau } \rho} \\
      = \; & (\rho \phi) \ol{\setsem{\Gamma, \psi; \Phi \vdash \tau[\phi :== \psi] } \rho} \\
      = \; & (\rho \psi) \ol{\setsem{\Gamma, \psi; \Phi \vdash \tau[\phi :== \psi] } \rho} \\
      = \; & \setsem{\Gamma, \psi; \Phi \vdash \psi \ol{\tau[\phi :== \psi]} } \rho \\
      = \; & \setsem{\Gamma, \psi; \Phi \vdash (\phi \ol\tau)[\phi :== \psi] } \rho
    \end{align*}
  
    The first and fifth equalities above are by Definition~\ref{def:set-sem}. The fourth equality
    is by equality of the functors $\rho \phi$ and $\rho \psi$.  \\

  \item Case $\Gamma; \Phi, \phi \vdash (\mu \phi'. \lambda \ol\alpha. H)\ol\tau : \F$. 
      The induction hypothesis gives that 
      $$\setsem{\Gamma; \Phi, \phi', \ol{\alpha}, \phi \vdash H } \rho
        = \setsem{\Gamma, \psi; \Phi, \phi', \ol{\alpha} \vdash H[\phi :== \psi] } \rho$$ 
      and 
      $$\setsem{\Gamma; \Phi, \phi \vdash \tau } \rho
        = \setsem{\Gamma, \psi; \Phi \vdash \tau[\phi :== \psi] } \rho$$ 
      for each $\tau$. 

      \begin{align*}
        & \setsem{\Gamma; \Phi, \phi \vdash (\mu \phi'. \lambda \ol\alpha. H)\ol\tau } \rho \\
        %% = \; & (\mu T_{\rho}^\set) \ol{\setsem{ \Gamma; \Phi, \phi \vdash \tau} \rho} \text{\quad(T contains $\phi$)} \\ 
        % = \; & (\mu T_{\rho}^\set) \ol{\setsem{ \Gamma; \Phi, \phi \vdash \tau} \rho} \text{\quad(T contains $\phi$)} \\ 
        = \; & (\mu (
            \lambda F. \lambda \ol{A}. \setsem{\Gamma; \Phi, \phi', \ol{\alpha}, \phi \vdash H} \rho[\phi' := F][\ol{\alpha := A}]
          )
        ) \ol{\setsem{ \Gamma; \Phi, \phi \vdash \tau} \rho} \\
        = \; & (\mu (
            \lambda F. \lambda \ol{A}. \setsem{\Gamma, \psi; \Phi, \phi', \ol{\alpha} \vdash H[\phi :== \psi]} \rho[\phi' := F][\ol{\alpha := A}]
          )
        ) \ol{\setsem{ \Gamma; \Phi, \phi \vdash \tau} \rho} \\
        % = \; &  (\mu (
        %     \lambda F. \lambda \ol{A}. \setsem{\Gamma, \psi; \Phi, \phi', \ol{\alpha} \vdash H[\phi :== \psi]} \rho[\phi' := F][\ol{\alpha := A}]
        %   ) \ol{\setsem{ \Gamma; \Phi, \phi \vdash \tau} \rho} \\
        = \; &  (\mu (
            \lambda F. \lambda \ol{A}. \setsem{\Gamma, \psi; \Phi, \phi', \ol{\alpha} \vdash H[\phi :== \psi]} \rho[\phi' := F][\ol{\alpha := A}]
          )
        \ol{\setsem{ \Gamma, \psi; \Phi \vdash \tau[\phi :== \psi]} \rho} \\
        = \; & \setsem{\Gamma, \psi; \Phi \vdash (\mu \phi'. \lambda \ol\alpha. H[\phi :== \psi]) \ol{\tau[\phi :== \psi]} } \rho \\
        = \; & \setsem{\Gamma, \psi; \Phi \vdash ((\mu \phi'. \lambda \ol\alpha. H) \ol\tau)[\phi :== \psi] } \rho 
      \end{align*} 
      The first and fifth equalities are by Definition~\ref{def:set-sem}.
      The second equality follows from the following equality:
      \begin{align*}
        &\lambda F. \lambda \ol{A}. \setsem{\Gamma; \Phi, \phi', \ol{\alpha}, \phi \vdash H} \rho[\phi' := F][\ol{\alpha := A}] \\
        & = \lambda F. \lambda \ol{A}. \setsem{\Gamma, \psi; \Phi, \phi', \ol{\alpha} \vdash H[\phi :== \psi]} \rho[\phi' := F][\ol{\alpha := A}]
      \end{align*}
      These two functors have the same action on objects by the induction hypothesis on $H$, and the fact
      that the extended environment $\rho[\phi' := F][\ol{\alpha := A}]$ satisfies the required
      hypothesis. These two functors have the same action on morphisms by the induction hypothesis
      on $H$ from Theorem~\ref{thm:demotion-morph}. Thus they are the same functor with the same fixed point.

  \item Case $\Gamma; \Phi, \phi \vdash \sigma \times \tau : \F$. The induction hypothesis gives that 
      $$\setsem{\Gamma; \Phi, \phi \vdash \sigma } \rho
        = \setsem{\Gamma, \psi; \Phi \vdash \sigma[\phi :== \psi] } \rho$$ 
        and
      $$\setsem{\Gamma; \Phi, \phi \vdash \tau } \rho
        = \setsem{\Gamma, \psi; \Phi \vdash \tau[\phi :== \psi] } \rho$$ 

    \begin{align*}
           & \setsem{\Gamma; \Phi, \phi \vdash \sigma \times \tau} \rho \\
      = \; & \setsem{\Gamma; \Phi, \phi \vdash \sigma} \rho \times \setsem{\Gamma; \Phi, \phi \vdash \tau} \rho\\
      = \; & \setsem{\Gamma, \psi; \Phi \vdash \sigma[\phi :== \psi]} \rho \times \setsem{\Gamma, \psi; \Phi \vdash \tau[\phi :== \psi]} \rho\\
      = \; & \setsem{\Gamma, \psi; \Phi \vdash \sigma[\phi :==\psi] \times \tau[\phi :== \psi]} \rho \\
      = \; & \setsem{\Gamma, \psi; \Phi \vdash (\sigma \times \tau)[\phi :== \psi]} \rho
    \end{align*}

    The case for $\sigma + \tau$ is analogous.
\end{itemize}
\end{proof}

\begin{thm}\label{thm:demotion-morph}
If $\Gamma; \Phi, \phi \vdash \tau : \F$, and if 
$f : \rho \to \rho'$, is a morphism of set environments such that
  $\rho \phi = \rho \psi = \rho' \phi = \rho' \psi$, and
   $f \phi = f \psi = \id_{\rho \phi}$, then

  $$\setsem{\Gamma; \Phi, \phi \vdash \tau} f 
    = \setsem{\Gamma, \psi; \Phi \vdash \tau[\phi :== \psi]} f $$

Analogously, if
$f : \rho \to \rho'$, is a morphism of relation environments such that
  $\rho \phi = \rho \psi = \rho' \phi = \rho' \psi$, and
   $f \phi = f \psi = \id_{\rho \phi}$, then

  $$\relsem{\Gamma; \Phi, \phi \vdash \tau} f 
    = \relsem{\Gamma, \psi; \Phi \vdash \tau[\phi :== \psi]} f $$

\end{thm}

\begin{proof}
We prove the case for $\set$ by induction on the structure of $\tau$.
The case for $\rel$ proceeds analogously.

\begin{itemize}
  \item There is nothing to prove for types in $\T$ because their functorial contexts
        must be empty.
  \item Case $\Gamma; \Phi, \alpha \vdash \alpha : \F$. Given that $\rho \alpha = \rho \beta$, 
      
    \begin{align*}
         & \setsem{\Gamma; \Phi, \alpha \vdash \alpha } f \\
      = \; & \id_{\rho \alpha} \\
      = \; & \id_{\rho \beta}  \\
      = \; & \setsem{\Gamma, \beta; \Phi \vdash \beta } f
    \end{align*}

  \item Case $\Gamma; \Phi, \phi \vdash \onet :\F$,  $\Gamma; \Phi, \phi \vdash \zerot :\F$.

    \begin{align*}
         & \setsem{\Gamma; \Phi, \phi \vdash \onet } f \\
      = \; & \id_1 \\
      = \; & \setsem{\Gamma, \psi; \Phi \vdash \onet } f
    \end{align*}
    Similarly for $\zerot$.

  \item Case $\Gamma; \Phi, \phi \vdash \phi \ol\tau : \F$. The induction hypothesis gives that 
    $$\setsem{\Gamma; \Phi, \phi \vdash \tau} f = 
      \setsem{\Gamma, \psi; \Phi \vdash \tau[\phi :== \psi]} f$$
    for each $\tau$.

    \begin{align*}
         & \setsem{\Gamma; \Phi, \phi \vdash \phi \ol\tau} f \\
      = \; & (f \phi)_{\ol{\setsem{\Gamma; \Phi, \phi \vdash \tau} \rho'}} 
              \circ (\rho \phi) \ol{\setsem{\Gamma; \Phi, \phi \vdash \tau} f} \\
      = \; & (\id_{\rho \phi})_{\ol{\setsem{\Gamma; \Phi, \phi \vdash \tau} \rho'}} 
              \circ (\rho \phi) \ol{\setsem{\Gamma; \Phi, \phi \vdash \tau} f} \\
      = \; & (\rho \phi) \ol{\setsem{\Gamma; \Phi, \phi \vdash \tau} f} \\
      = \; & (\rho \psi) \ol{\setsem{\Gamma, \psi; \Phi \vdash \tau[\phi :== \psi]} f} \\
      = \; & (\id_{\rho \psi})_{\ol{\setsem{\Gamma, \psi; \Phi \vdash \tau[\phi :== \psi]} \rho'}}
              \circ (\rho \psi) \ol{\setsem{\Gamma, \psi; \Phi \vdash \tau[\phi :== \psi]} f} \\
      = \; & (f \psi)_{\ol{\setsem{\Gamma, \psi; \Phi \vdash \tau[\phi :== \psi]} \rho'}}
              \circ (\rho \psi) \ol{\setsem{\Gamma, \psi; \Phi \vdash \tau[\phi :== \psi]} f} \\
      = \; & \setsem{\Gamma, \psi; \Phi \vdash \psi \ol{\tau[\phi :== \psi]}} f \\
      = \; & \setsem{\Gamma, \psi; \Phi \vdash (\phi \ol\tau) [\phi :== \psi]} f
    \end{align*}

  \item Case $\Gamma; \Phi, \phi \vdash (\mu \phi'. \lambda \ol\alpha. H)\ol\tau : \F$. 
      The induction hypothesis gives that 
      $$\setsem{\Gamma; \Phi, \phi', \ol{\alpha}, \phi \vdash H } f
        = \setsem{\Gamma, \psi; \Phi, \phi', \ol{\alpha} \vdash H[\phi :== \psi] } f$$ 
      and 
      $$\setsem{\Gamma; \Phi, \phi \vdash \tau } f
        = \setsem{\Gamma, \psi; \Phi \vdash \tau[\phi :== \psi] } f$$ 
      for each $\tau$. 
      \begin{align*}
        & \setsem{\Gamma; \Phi, \phi \vdash (\mu \phi'. \lambda \ol\alpha. H)\ol\tau } f \\
        % = \; & (\mu \sigma^\set_f)_{\ol{\setsem{\Gamma; \Phi, \phi \vdash \tau} \rho'}}
        %         \circ (\mu T^\set_\rho) \ol{\setsem{\Gamma; \Phi, \phi \vdash \tau} f} \\
        = \; & 
           (\mu (
             \lambda F. \lambda \ol{A}. \setsem{\Gamma; \Phi, \phi', \ol{\alpha}, \phi \vdash H} f [\phi' := \id_F][\ol{\alpha := \id_A}]
           ))_{\ol{\setsem{\Gamma; \Phi, \phi \vdash \tau} \rho'}} \\
          \; & \circ
          (\mu (
            \lambda F. \lambda \ol{A}. \setsem{\Gamma; \Phi, \phi', \ol{\alpha}, \phi \vdash H} \rho[\phi' := F][\ol{\alpha := A}]
          )
        ) \ol{\setsem{ \Gamma; \Phi, \phi \vdash \tau} f} \\
        = \; & 
           (\mu (
             \lambda F. \lambda \ol{A}. \setsem{\Gamma; \Phi, \phi', \ol{\alpha}, \phi \vdash H} f [\phi' := \id_F][\ol{\alpha := \id_A}]
           ))_{\ol{\setsem{\Gamma, \psi; \Phi \vdash \tau[\phi :== \psi]} \rho'}} \\
          \; & \circ
          (\mu (
            \lambda F. \lambda \ol{A}. \setsem{\Gamma; \Phi, \phi', \ol{\alpha}, \phi \vdash H} \rho[\phi' := F][\ol{\alpha := A}]
          )
        ) \ol{\setsem{ \Gamma, \psi; \Phi \vdash \tau[\phi :== \psi]} f} \\
        = \; & 
           (\mu (
              \lambda F. \lambda \ol{A}. \setsem{\Gamma, \psi; \Phi, \phi', \ol{\alpha} \vdash H[\phi :== \psi]} f [\phi' := \id_F][\ol{\alpha := \id_A}]
           ))_{\ol{\setsem{\Gamma, \psi; \Phi \vdash \tau[\phi :== \psi]} \rho'}} \\
          \; & \circ
          (\mu (
            \lambda F. \lambda \ol{A}. \setsem{\Gamma; \Phi, \phi', \ol{\alpha}, \phi \vdash H} \rho[\phi' := F][\ol{\alpha := A}]
          )
        ) \ol{\setsem{ \Gamma, \psi; \Phi \vdash \tau[\phi :== \psi]} f} \\
        = \; & 
           (\mu (
              \lambda F. \lambda \ol{A}. \setsem{\Gamma, \psi; \Phi, \phi', \ol{\alpha} \vdash H[\phi :== \psi]} f [\phi' := \id_F][\ol{\alpha := \id_A}]
           ))_{\ol{\setsem{\Gamma, \psi; \Phi \vdash \tau[\phi :== \psi]} \rho'}} \\
          \; & \circ
          (\mu (
            \lambda F. \lambda \ol{A}. \setsem{\Gamma, \psi; \Phi, \phi', \ol{\alpha} \vdash H[\phi :== \psi]} \rho[\phi' := F][\ol{\alpha := A}]
          )
        ) \ol{\setsem{ \Gamma, \psi; \Phi \vdash \tau[\phi :== \psi]} f} \\
        % = \; & (\mu \sigma^\set_f)_{\ol{\setsem{\Gamma, \psi; \Phi \vdash \tau[\phi :== \psi]} \rho'}}
        %         \circ (\mu T^\set_\rho) \ol{\setsem{\Gamma, \psi; \Phi \vdash \tau[\phi :== \psi]} f} \\
        = \; & \setsem{\Gamma, \psi; \Phi \vdash (\mu \phi'. \lambda \ol\alpha. H[\phi :== \psi]) \ol{\tau[\phi :== \psi]} } f \\
        = \; & \setsem{\Gamma, \psi; \Phi \vdash ((\mu \phi'. \lambda \ol\alpha. H) \ol\tau)[\phi :== \psi] } f
      \end{align*} 

      \noindent
      The first and fifth equalities are by Definition~\ref{def:set-sem}.
      The third equality is by the equality of the arguments to the first $\mu$ operator:
      \begin{align*}
        & \lambda F. \lambda \ol{A}. \setsem{\Gamma; \Phi, \phi', \ol{\alpha}, \phi \vdash H} f [\phi' := \id_F][\ol{\alpha := \id_A}] \\
        & = \lambda F. \lambda \ol{A}. \setsem{\Gamma, \psi; \Phi, \phi', \ol{\alpha} \vdash H[\phi :== \psi]} f [\phi' := \id_F][\ol{\alpha := \id_A}]
      \end{align*}
      By the induction hypothesis on $H$ and the fact that the morphism $f[\phi' := \id_F][\ol{\alpha := \id_A}]
      : \rho[\phi' := F][\ol{\alpha := A}] \to \rho'[\phi' := F][\ol{\alpha := A}]$
      still satisfies the required hypotheses.
      The fourth equality is by the equality of the arguments to the second $\mu$ operator:
      \begin{align*}
        & \lambda F. \lambda \ol{A}. \setsem{\Gamma; \Phi, \phi', \ol{\alpha}, \phi \vdash H} \rho[\phi' := F][\ol{\alpha := A}] \\
        & = \lambda F. \lambda \ol{A}. \setsem{\Gamma, \psi; \Phi, \phi', \ol{\alpha} \vdash H[\phi :== \psi]} \rho[\phi' := F][\ol{\alpha := A}]
      \end{align*}

      These two functors have the same action on objects by the induction hypothesis on $H$ from Theorem~\ref{thm:demotion-objects},
      and they have the same action on morphisms by the induction hypothesis on $H$ from this theorem. 
      Thus they are the same functor with the same fixed point.

  \item Case $\Gamma; \Phi, \phi \vdash \sigma \times \tau : \F$. The induction hypothesis gives that 
      $$\setsem{\Gamma; \Phi, \phi \vdash \sigma } f
        = \setsem{\Gamma, \psi; \Phi \vdash \sigma[\phi :== \psi] } f$$ 
        and
      $$\setsem{\Gamma; \Phi, \phi \vdash \tau } f
        = \setsem{\Gamma, \psi; \Phi \vdash \tau[\phi :== \psi] } f$$ 

    \begin{align*}
           & \setsem{\Gamma; \Phi, \phi \vdash \sigma \times \tau} f \\
      = \; & \setsem{\Gamma; \Phi, \phi \vdash \sigma} f \times \setsem{\Gamma; \Phi, \phi \vdash \tau} f \\
      = \; & \setsem{\Gamma, \psi; \Phi \vdash \sigma[\phi :== \psi]} f \times \setsem{\Gamma, \psi; \Phi \vdash \tau[\phi :== \psi]} f \\
      = \; & \setsem{\Gamma, \psi; \Phi \vdash \sigma[\phi :==\psi] \times \tau[\phi :== \psi]} f \\
      = \; & \setsem{\Gamma, \psi; \Phi \vdash (\sigma \times \tau)[\phi :== \psi]} f
    \end{align*}


    The case for $\sigma + \tau$ is analogous.
\end{itemize}
\end{proof}
\end{document}

\end{itemize}


\end{proof}

\end{document}

