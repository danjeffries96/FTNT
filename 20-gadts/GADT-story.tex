% For double-blind review submission, w/o CCS and ACM Reference (max
% submission space)
\documentclass[acmsmall,screen,review,anonymous]{acmart}
\settopmatter{printfolios=true,printccs=false,printacmref=false}
%% For double-blind review submission, w/ CCS and ACM Reference
%\documentclass[acmsmall,review,anonymous]{acmart}\settopmatter{printfolios=true}
%% For single-blind review submission, w/o CCS and ACM Reference (max submission space)
%\documentclass[acmsmall,review]{acmart}\settopmatter{printfolios=true,printccs=false,printacmref=false}
%% For single-blind review submission, w/ CCS and ACM Reference
%\documentclass[acmsmall,review]{acmart}\settopmatter{printfolios=true}
%% For final camera-ready submission, w/ required CCS and ACM Reference
%\documentclass[acmsmall]{acmart}\settopmatter{}


%% Journal information
%% Supplied to authors by publisher for camera-ready submission;
%% use defaults for review submission.
\acmJournal{PACMPL}
\acmVolume{1}
\acmNumber{POPL} % CONF = POPL or ICFP or OOPSLA
\acmArticle{1}
\acmYear{2021}
\acmMonth{1}
\acmDOI{} % \acmDOI{10.1145/nnnnnnn.nnnnnnn}
\startPage{1}

%% Copyright information
%% Supplied to authors (based on authors' rights management selection;
%% see authors.acm.org) by publisher for camera-ready submission;
%% use 'none' for review submission.
\setcopyright{none}
%\setcopyright{acmcopyright}
%\setcopyright{acmlicensed}
%\setcopyright{rightsretained}
%\copyrightyear{2018}           %% If different from \acmYear

%% Bibliography style
\bibliographystyle{ACM-Reference-Format}
%% Citation style
%% Note: author/year citations are required for papers published as an
%% issue of PACMPL.
\citestyle{acmauthoryear}   %% For author/year citations
%\citestyle{acmnumeric}

%%%%%%%%%%%%%%%%%%%%%%%%%%%%%%%%%%%%%%%%%%%%%%%%%%%%%%%%%%%%%%%%%%%%%%
%% Note: Authors migrating a paper from PACMPL format to traditional
%% SIGPLAN proceedings format must update the '\documentclass' and
%% topmatter commands above; see 'acmart-sigplanproc-template.tex'.
%%%%%%%%%%%%%%%%%%%%%%%%%%%%%%%%%%%%%%%%%%%%%%%%%%%%%%%%%%%%%%%%%%%%%%



\usepackage[utf8]{inputenc}
\usepackage{ccicons}
\usepackage{verbatim}

\usepackage{amsmath}
\usepackage{amsthm}
\usepackage{amscd}
%\usepackage{MnSymbol}
\usepackage{xcolor}

\usepackage{bbold}
\usepackage{url}
\usepackage{upgreek}
%\usepackage{stmaryrd}

\usepackage{lipsum}
\usepackage{tikz-cd}
\usetikzlibrary{cd}
\usetikzlibrary{calc}
\usetikzlibrary{arrows}

\usepackage{bussproofs}
\EnableBpAbbreviations

\DeclareMathAlphabet{\mathpzc}{OT1}{pzc}{m}{it}

%\usepackage[amsmath]{ntheorem}

\newcommand{\lam}{\lambda}
\newcommand{\eps}{\varepsilon}
\newcommand{\ups}{\upsilon}
\newcommand{\mcB}{\mathcal{B}}
\newcommand{\mcD}{\mathcal{D}}
\newcommand{\mcE}{\mathcal{E}}
\newcommand{\mcF}{\mathcal{F}}
\newcommand{\mcP}{\mathcal{P}}
\newcommand{\mcI}{\mathcal{I}}
\newcommand{\mcJ}{\mathcal{J}}
\newcommand{\mcK}{\mathcal{K}}
\newcommand{\mcL}{\mathcal{L}}
\newcommand{\WW}{\mathcal{W}}

\newcommand{\ex}{\mcE_x}
\newcommand{\ey}{\mcE_y}
\newcommand{\bzero}{\boldsymbol{0}}
\newcommand{\bone}{{\boldsymbol{1}}}
\newcommand{\tB}{{\bone_\mcB}}
\newcommand{\tE}{{\bone_\mcE}}
\newcommand{\bt}{\mathbf{t}}
\newcommand{\bp}{\mathbf{p}}
\newcommand{\bsig}{\mathbf{\Sigma}}
\newcommand{\bpi}{\boldsymbol{\pi}}
\newcommand{\Empty}{\mathtt{Empty}}
\newcommand{\truthf}{\mathtt{t}}
\newcommand{\id}{id}
\newcommand{\coo}{\mathtt{coo\ }}
\newcommand{\mcC}{\mathcal{C}}
\newcommand{\Rec}{\mathpzc{Rec}}
\newcommand{\types}{\mathcal{T}}

%\newcommand{\Homrel}{\mathsf{Hom_{Rel}}}
\newcommand{\HomoCPOR}{\mathsf{Hom_{\oCPOR}}}

%\newcommand{\semof}[1]{\llbracket{#1}\rrbracket^\rel}
\newcommand{\sem}[1]{\llbracket{#1}\rrbracket}
\newcommand{\setsem}[1]{\llbracket{#1}\rrbracket^\set}
\newcommand{\relsem}[1]{\llbracket{#1}\rrbracket^\rel}
\newcommand{\dsem}[1]{\llbracket{#1}\rrbracket^{\mathsf D}}
\newcommand{\setenv}{\mathsf{SetEnv}}
\newcommand{\relenv}{\mathsf{RelEnv}}
\newcommand{\oCPOenv}{\mathsf{SetEnv}}
\newcommand{\oCPORenv}{\mathsf{RelEnv}}
\newcommand{\oCPOsem}[1]{\llbracket{#1}\rrbracket^{\set}}
\newcommand{\oCPORsem}[1]{\llbracket{#1}\rrbracket^{\rel}}
\newcommand{\denv}{\mathsf{DEnv}}

\newcommand{\rel}{\mathsf{Rel}}
\newcommand{\setof}[1]{\{#1\}}
\newcommand{\letin}[1]{\texttt{let }#1\texttt{ in }}
\newcommand{\comp}[1]{{\{#1\}}}
\newcommand{\bcomp}[1]{\{\![#1]\!\}}
\newcommand{\beps}{\boldsymbol{\eps}}
%\newcommand{\B}{\mcB}
%\newcommand{\Bo}{{|\mcB|}}

\newcommand{\lmt}{\longmapsto}
\newcommand{\RA}{\Rightarrow}
\newcommand{\LA}{\Leftarrow}
\newcommand{\rras}{\rightrightarrows}
\newcommand{\colim}[2]{{{\underrightarrow{\lim}}_{#1}{#2}}}
\newcommand{\lift}[1]{{#1}\,{\hat{} \; \hat{}}}
\newcommand{\graph}[1]{\langle {#1} \rangle}

\newcommand{\carAT}{\mathsf{car}({\mathcal A}^T)}
\newcommand{\isoAto}{\mathsf{Iso}({\mcA^\to})}
\newcommand{\falg}{\mathsf{Alg}_F}
\newcommand{\CC}{\mathsf{Pres}(\mathcal{A})}
\newcommand{\PP}{\mathcal{P}}
\newcommand{\DD}{D_{(A,B,f)}}
\newcommand{\from}{\leftarrow}
\newcommand{\upset}[1]{{#1}{\uparrow}}
\newcommand{\smupset}[1]{{#1}\!\uparrow}

\newcommand{\Coo}{\mathpzc{Coo}}
\newcommand{\code}{\#}
\newcommand{\nat}{\mathpzc{Nat}}

\newcommand{\eq}{\; = \;}
\newcommand{\of}{\; : \;}
\newcommand{\df}{\; := \;}
\newcommand{\bnf}{\; ::= \;}

\newcommand{\zmap}[1]{{\!{\between\!\!}_{#1}\!}}
\newcommand{\bSet}{\mathbf{Set}}

\newcommand{\dom}{\mathsf{dom}}
\newcommand{\cod}{\mathsf{cod}}
\newcommand{\adjoint}[2]{\mathrel{\mathop{\leftrightarrows}^{#1}_{#2}}}
\newcommand{\isofunc}{\mathpzc{Iso}}
\newcommand{\ebang}{{\eta_!}}
\newcommand{\lras}{\leftrightarrows}
\newcommand{\rlas}{\rightleftarrows}
\newcommand{\then}{\quad\Longrightarrow\quad}
\newcommand{\hookup}{\hookrightarrow}

\newcommand{\spanme}[5]{\begin{CD} #1 @<#2<< #3 @>#4>> #5 \end{CD}}
\newcommand{\spanm}[3]{\begin{CD} #1 @>#2>> #3\end{CD}}
\newcommand{\pushout}{\textsf{Pushout}}
\newcommand{\mospace}{\qquad\qquad\!\!\!\!}

\newcommand{\natur}[2]{#1 \propto #2}

\newcommand{\Tree}{\mathsf{Tree}\,}
\newcommand{\GRose}{\mathsf{GRose}\,}
\newcommand{\List}{\mathsf{List}\,}
\newcommand{\PTree}{\mathsf{PTree}\,}
\newcommand{\Bush}{\mathsf{Bush}\,}
\newcommand{\Forest}{\mathsf{Forest}\,}
\newcommand{\Lam}{\mathsf{Lam}\,}
\newcommand{\LamES}{\mathsf{Lam}^+}
\newcommand{\Expr}{\mathsf{Expr}\,}

\newcommand{\ListNil}{\mathsf{Nil}}
\newcommand{\ListCons}{\mathsf{Cons}}
\newcommand{\LamVar}{\mathsf{Var}}
\newcommand{\LamApp}{\mathsf{App}}
\newcommand{\LamAbs}{\mathsf{Abs}}
\newcommand{\LamSub}{\mathsf{Sub}}
\newcommand{\ExprConst}{\mathsf{Const}}
\newcommand{\ExprPair}{\mathsf{Pair}}
\newcommand{\ExprProj}{\mathsf{Proj}}
\newcommand{\ExprAbs}{\mathsf{Abs}}
\newcommand{\ExprApp}{\mathsf{App}}
\newcommand{\Ptree}{\mathsf{Ptree}}

\newcommand{\kinds}{\mathpzc{K}}
\newcommand{\tvars}{\mathbb{T}}
\newcommand{\fvars}{\mathbb{F}}
\newcommand{\consts}{\mathpzc{C}}
\newcommand{\Lan}{\mathsf{Lan}}
\newcommand{\zerot}{\mathbb{0}}
\newcommand{\onet}{\mathbb{1}}
\newcommand{\bool}{\mathbb{2}}
\renewcommand{\nat}{\mathbb{N}}
%\newcommand{\semof}[1]{[\![#1]\!]}
%\newcommand{\setsem}[1]{\llbracket{#1}\rrbracket^\set}
\newcommand{\predsem}[1]{\llbracket{#1}\rrbracket^\pred}
%\newcommand{\todot}{\stackrel{.}{\to}}
\newcommand{\todot}{\Rightarrow}
\newcommand{\bphi}{{\bm \phi}}

\newcommand{\bm}[1]{\boldsymbol{#1}}

\newcommand{\cL}{\mathcal{L}}
\newcommand{\T}{\mathcal{T}}
\newcommand{\Pos}{P\!}
%\newcommand{\Pos}{\mathcal{P}\!}
\newcommand{\Neg}{\mathcal{N}}
\newcommand{\Hf}{\mathcal{H}}
\newcommand{\V}{\mathbb{V}}
\newcommand{\I}{\mathcal{I}}
\newcommand{\Set}{\mathsf{Set}}
%\newcommand{\Nat}{\mathsf{Nat}}
\newcommand{\Homrel}{\mathsf{Hom_{Rel}}}
\newcommand{\CV}{\mathcal{CV}}
\newcommand{\lan}{\mathsf{Lan}}
\newcommand{\Id}{\mathit{Id}}
\newcommand{\mcA}{\mathcal{A}}
\newcommand{\inl}{\mathsf{inl}}
\newcommand{\inr}{\mathsf{inr}}
%\newcommand{\case}[3]{\mathsf{case}\,{#1}\,\mathsf{of}\,\{{#2};\,{#3}\}}
\newcommand{\tin}{\mathsf{in}}
\newcommand{\fold}{\mathsf{fold}}
\newcommand{\Eq}{\mathsf{Eq}}
\newcommand{\Hom}{\mathsf{Hom}}
\newcommand{\curry}{\mathsf{curry}}
\newcommand{\uncurry}{\mathsf{uncurry}}
\newcommand{\eval}{\mathsf{eval}}
\newcommand{\apply}{\mathsf{apply}}
\newcommand{\oCPO}{{\mathsf{Set}}}
\newcommand{\oCPOR}{{\mathsf{Rel}}}
\newcommand{\oCPORT}{{\mathsf{RT}}}

\newcommand{\ar}[1]{\##1}
\newcommand{\mcG}{\mathcal{G}}
\newcommand{\mcH}{\mathcal{H}}
\newcommand{\TV}{\mathpzc{V}}

\newcommand{\essim}[1]{\mathsf{EssIm}(#1)}
\newcommand{\hra}{\hookrightarrow}

\newcommand{\ol}[1]{\overline{#1}}
\newcommand{\ul}[1]{\underline{#1}}
\newcommand{\op}{\mathsf{op}}
\newcommand{\trige}{\trianglerighteq}
\newcommand{\trile}{\trianglelefteq}
\newcommand{\LFP}{\mathsf{LFP}}
\newcommand{\LAN}{\mathsf{Lan}}
%\newcommand{\Mu}{{\mu\hskip-4pt\mu}}
\newcommand{\Mu}{{\mu\hskip-5.5pt\mu}}
%\newcommand{\Mu}{\boldsymbol{\upmu}}
\newcommand{\Terms}{\mathpzc{Terms}}
\newcommand{\Ord}{\mathpzc{Ord}}
\newcommand{\Anote}[1]{{\color{blue} {#1}}}
\newcommand{\Pnote}[1]{{\color{red} {#1}}}

\newcommand{\greyout}[1]{{\color{gray} {#1}}}
\newcommand{\ora}[1]{\overrightarrow{#1}}

%\newcommand{\?}{{.\ }}
%\theoremheaderfont{\scshape}
%\theorembodyfont{\normalfont}
%\theoremseparator{.\ \ }
\newtheorem{thm}{Theorem}
\newtheorem{dfn}[thm]{Definition}
\newtheorem{prop}[thm]{Proposition}
\newtheorem{cor}[thm]{Corollary}
\newtheorem{lemma}[thm]{Lemma}
\newtheorem{rmk}[thm]{Remark}
\newtheorem{expl}[thm]{Example}
\newtheorem{notn}[thm]{Notation}
%\theoremstyle{nonumberplain}
%\theoremsymbol{\Box}


\theoremstyle{definition}
\newtheorem{exmpl}{Example}

\renewcommand{\greyout}[1]{}

\newcommand{\new}[1]{{\color{blue} {#1}}}


\newcommand{\emptyfun}{{[]}}
\newcommand{\cal}{\mathcal}
%\newcommand{\fold}{\mathit{fold}}
\newcommand{\F}{\mathcal{F}}
\renewcommand{\G}{\mathcal{G}}
\newcommand{\N}{\mathcal{N}}
\newcommand{\E}{\mathcal{E}}
\newcommand{\B}{\mathcal{B}}
\renewcommand{\P}{\mathcal{A}}
\newcommand{\pred}{\mathsf{Fam}}
\newcommand{\env}{\mathsf{Env}}
\newcommand{\set}{\mathsf{Set}}
\renewcommand{\S}{\mathcal S}
\renewcommand{\C}{\mathcal{C}}
\newcommand{\D}{\mathcal{D}}
\newcommand{\A}{\mathcal{A}}
\renewcommand{\id}{\mathit{id}}
\newcommand{\map}{\mathsf{map}}
\newcommand{\pid}{\underline{\mathit{id}}}
\newcommand{\pcirc}{\,\underline{\circ}\,}
\newcommand{\pzero}{\underline{0}}
\newcommand{\pone}{\underline{1}}
\newcommand{\psum}{\,\underline{+}\,}
\newcommand{\pinl}{\underline{\mathit{inL}}\,}
\newcommand{\pinr}{\underline{\mathit{inR}}\,}
\newcommand{\ptimes}{\,\underline{\times}\,}
\newcommand{\ppi}{\underline{\pi_1}}
\newcommand{\pppi}{\underline{\pi_2}}
\newcommand{\pmu}{\underline{\mu}}
\newcommand{\semmap}{\mathit{map}}
\newcommand{\subst}{\mathit{subst}}

\newcommand{\tb}[1]{~~ \mbox{#1} ~~}
\newcommand{\listt}[1]{(\mu \phi. \lambda \beta . \onet + \beta \times
  \phi \beta) #1} 
\newcommand{\filtype}{\Nat^\emptyset 
 (\Nat^\emptyset \, \alpha \, \mathit{Bool})\, (\Nat^\emptyset 
  (List \, \alpha) \, (List \, \alpha))} 
\newcommand{\filtypeGRose}{\Nat^\emptyset 
 (\Nat^\emptyset \, \alpha \, \mathit{Bool})\, (\Nat^\emptyset 
  (\mathit{GRose}\,\psi \, \alpha) \, (\mathit{GRose}\,\psi \, (\alpha
  + \onet)))} 
\newcommand{\maplist}{\mathit{map}_{\lambda A. \setsem{\emptyset; \alpha
      \vdash \mathit{List} \, \alpha} \rho[\alpha := A]}} 
\newcommand{\PLeaves}{\mathsf{PLeaves}}
\newcommand{\swap}{\mathsf{swap}}
\newcommand{\reverse}{\mathsf{reverse}}
\newcommand{\Bcons}{\mathit{Bcons}}
\newcommand{\Bnil}{\mathit{Bnil}}

\title[Parametricity Consequences of GADTs
  Representations]{Parametricity Consequences of GADTs
  Representations}

\author{Patricia Johann, Enrico Ghiorzi, and Daniel Jeffries}
\affiliation{ \institution{Appalachian State University}}
\email{johannp@appstate.edu, ghiorzie@appstate.edu,
  jeffriesd@appstate.edu}

\begin{document}

\begin{abstract}
FIX ME!!!! GADTs are misnamed. They don't really generalize ADTs. And
they don't behave the way we expect data types to behave. In
particular, they force us to choose between the naturality properties
we expect of their fmap functions and other parametricity results for
them, causing us to question in what sense GADTs are data types, and
what the value of the parametricity results for them as determined by
their syntax really is.
\end{abstract}

\maketitle

\section{Introduction}

There are two standard ways to represent data types when studying
modern functional languages: via their Church encodings in a (possibly
higher-kinded) extension of System F~\cite{gir72}, the calculus at the
core of all such languages, and by augmenting System F
%a Hindley-Milner style calculus
with primitives for constructing them directly as fixpoints.
%{\color{blue} Something about when each of these representations is
%  used. The former perhaps more closely represents how data types are
%  represented in implementations, but the latter are how they have
%  heretofore primarily been treated theoretically.}
In models that validate the program optimization known as {\em short
  cut fusion}~\cite{glp93} the two representations are, in fact,
equivalent, in the sense that they both have the same interpretation
in the model. Models in which the Church encodings and primitive
representations of all {\color{blue} (polynomial)} algebraic data
types (ADTs) are equivalent include the operationally-based model
of~\cite{pit98,pit00} and the categorical models
of~\cite{joh02,joh03}.
%\footnote{{\color{blue} Pitts proves the isomorphism only for
%    polynomial ADTs, whereas Johann {\em et al.}  prove it for any
%    data type that is the fixpoint of a first-order functor.}}
This equivalence was recently extended to the class of what might be
called ``hereditarily polynomial'' {\color{blue} to include truly
  nested such types} nested types via the categorical model
of~\cite{jgj21}. But, as shown in Section~\ref{sec:?} below, for some
{\color{blue} (even polynomial)} GADTs,
%that are not hereditarily polynomial nested types,
no such isomorphism holds.

GADTs are traditionally represented via Church encodings. In this
view, GADTs are completely determined by their syntax. For example,
the GADT $\mathsf{G}$ defined by
\[\begin{array}{l}
\mathsf{data\,G : * \to *\,where}\\
\mathsf{\;\;\;\;\;\;\;\;c : G\,1}
\end{array}\]
{\color{blue} Use $\top$ instead of $1$ throughout?}

\noindent
{\color{blue} data type vs data structure --- which should have the
functorial completion? The data type.}

\noindent
comprises exactly one element, namely $\mathsf{c}$. By contrast, the
primitive representation sees a GADT as the {\em functorial
  completion}~\cite{jp19} of its syntax. {\color{blue} Completing the
  syntax is necessary to have a {\em functor} to take the least fixed
  point of, so is essential to the view of data types as primitive.}
Viewed as a primitive data type, $\mathsf{G}$ comprises not just
$\mathsf{c}$, but also all data elements of the form
$\mathsf{map\;f\;c}$ for all types $\mathsf{t}$ and all functions
$\mathsf{f : 1 \to t}$ definable in the language, as well as all data
elements of the form $\mathsf{map\;g\;x}$ for each appropriately typed
function $\mathsf{g}$ and each element $\mathsf{x}$ already added to
the data type, and so on.  Although functorial completion adds no new
data elements to the syntax of a (polynomial) ADT or a (hereditarily
polynomial) nested type, so that its primitive Church encoding
representations coincide, these are rarely the same for GADTs.
Indeed, they are quite different for even the very simple GADT
$\mathsf{G}$ above.

But does it really matter which way we view GADTs?

A GADT programmer is likely to use GADTs precisely because they
exhibit different behaviors at different types, and thus to consider
the GADT to be completely specified by its syntax. When used this way,
the shape of a GADTs is actually {\em determined} by the data it
contains. As a result, it is not expected to support a $\mathsf{map}$
function --- i.e., to be an instance of Haskell's $\mathsf{Functor}$
class --- like other data types. For example, the definition for
$\mathsf{G}$ above specifies data only at instance $\mathsf{G\,1}$.
Since there is a function from $\mathsf{1}$ to any other type
$\mathsf{t}$, and yet $\mathsf{G\,t}$ is uninhabited at any instance
other than $\mathsf{1}$, $\mathsf{G}$ cannot support a $\mathsf{map}$
function. When it is intended to capture this kind of non-uniform
behavior, a GADT's behavior does not generalize that of ADTs as
advertised, and in some sense doesn't even behave like a traditional
data type.  When used this way, a GADT cannot be seen as the least
fixpoint of a functor, and must instead be represented via its Church
encoding.

A semanticist, on the other hand, is likely to follow~\cite{bfss90}
and expect GADTs to behave as their names suggests --- i.e., as
generalizations of ADTs. The semanticist is thus likely to expect a
GADT to be able to be used as a ADT is used --- i.e., as a shape
filled with data that can be changed without also changing the
shape. When used this way, it is more natural to view the GADT in
terms of its primitive representation. In fact, the functorial
completion of a GADT inherent in its primitive representation is
exactly what is needed to ensure that the GADT supports a
$\mathsf{map}$ function, i.e., is a functor both in the catgorical
sense and in the sense of being an instance of Haskell's
$\mathsf{Functor}$ class.  It is worth noting, however, that since the
functorial completion of a GADT includes only those data elements that
are derivable via repeated applications of $\mathsf{map}$ starting
from those specified by its syntax, primitive GADTs need not, in
general, be nested types. For example, the functorial completion of
our GADT $\mathsf{G}$ includes elements at any instance
$\mathsf{G\,t}$ because there is a function from $\mathsf{1}$ to any
type $\mathsf{t}$, but it does not include any terms at instance
$\mathsf{G\,0}$ because no calculus allows the definition of a
function into $\mathsf{0}$. {\color{blue} Use $\bot$ instead of $0$
  here?}

The key observation of this paper is that, while the viewpoints of
both the GADT programmer and the semanticist are both valid, the two
are irreconcilable. Unsurprisingly, which point of view is adopted has
deep consequences for the ways GADTs can be used and reasoned
about. In this paper we show that the choice of GADT representation
has deep implications for parametric reasoning about structures of
those types. Indeed, a GADT programmer who views GADTs as their Church
encodings cannot safely use program transformations or reasoning
principles for them that involve map functions for them. This may
surprise the GADT programmer by being contrary to expectation for
something supposedly generalizing a data type. The GADT programmer
may, however, be able to program with and reason about GADTs using
non-naturality consequences of parametricity, such as type
inhabitation results. On the other hand, a semanticist who views GADTs
as primitive data types will have all naturality-based program
transformations and reasoning principles for GADTs at their disposal,
but non-naturality consequences of parametric reasoning may not be
available to them.

In this paper we show how these observations can be made precise. That
is, we show that how we view GADTs has significant consequences, and
thus answer the question we posed above in the affirmative: Yes, it
really does matter.

\section{Representations of Data Types}

\subsection{Algebraic Data Types}

A (polynomial) {\em algebraic data type} (ADT) has the form
\[\mathsf{T\,a} = \mathsf{c_1 t_{11}}...\mathsf{t_{1k_1}}\, |\, ...\, |\,
\mathsf{c_n t_{n1}}...\mathsf{t_{nk_n}}\] where each $\mathsf{t_{ij}}$
is a type also depending only on $\mathsf{a}$. Such a data type can be
thought of as a ``container'' for data of type $\mathsf{a}$. The data
in an ADT are arranged at various {\em positions} in its underlying
{\em shape}, which is determined by the types of its {\em
  constructors} $\mathsf{c_1},...,\mathsf{c_n}$. An ADT's constructors
are used to build the data values of the data type, as well as to
analyze those values using {\em pattern matching}. ADTs are used
extensively in functional programming to structure computations, to
express invariants of the data over which computations are defined,
and to ensure the type safety of programs specifying those
computations.

List types are the quintessential examples of ADTs. The shape of the
container underlying the list type
\[\mathsf{List\,a} = \mathsf{Nil} \,|\, \mathsf{Cons\,a\,(List\,a)}\]
%of lists of elements of type $\mathsf{a}$
is determined by the types of its two constructors $\mathsf{Nil\, ::\,
  List\, a}$ and $\mathsf{Cons \,::\,a \to \List\,a\to
  List\,a}$. These constructors specify that the data in a list of
type $\mathsf{List\,a}$ are arranged linearly. The shape underlying
the type $\mathsf{List\,a}$ is therefore given by the set $\nat$ of
natural numbers, and the positions in a structure of this type are
given by specific natural numbers. Since the type argument to every
occurrence of the type constructor $\mathsf{List}$ in the right-hand
side of above definition is the same the type instance being defined
on its left-hand side, the type $\mathsf{List\,a}$ enforces the
invariant that all of the data in a structure of this type have the
same type $\mathsf{a}$. In a similar way, the tree type
\[\mathsf{Tree\, a} = \mathsf{Leaf\,a}
\,|\,\mathsf{Node\,(Tree\,a)\,a\,(Tree\,a)}\] of binary trees
has as its underlying shape the type of binary trees of units, and the
positions in a structure of this type are given by sequences of L (for
``left'') and R (for ``right'') navigating a path
%to either a node or a leaf in
through the structure. The type $\mathsf{Tree\, a}$ enforces the
invariant that all of the data at the nodes and leaves in a structure
of this type have the same type $\mathsf{a}$.

{\color{blue} polynomial ADTs?} Since the shape of an ADT structure
--- i.e., a structure whose type is an instance of an ADT --- is
independent of the type of data it contains, ADTs can be defined
polymorphically. As a result, an ADT structure containing data of type
$\mathsf{a}$ can be transformed into another ADT structure of the
exact same shape containing data of another type $\mathsf{b}$ simply
by applying a given function $\mathsf{f \,:\,a \to b}$ to each of its
elements. Indeed, every ADT can be made an instance of Haskell's
$\mathsf{Functor}$ class by defining a type-and-data-uniform
type-preserving, data-changing $\mathsf{fmap}$ {\color{blue} map
  vs. fmap} function for it.  Moreover, given a type-independent way
of rearranging an ADT structure's shape $\mathsf{T\,a}$ into the shape
for another ADT structure $\mathsf{T'\,a}$, we get the same structure
of type $\mathsf{T'\,b}$ regardless of whether we first rearrange the
original structure of type $\mathsf{T\,a}$ into one of type
$\mathsf{T'\,a}$ and then use $\mathsf{fmap_{T'}}$ to convert that
resulting structure to one of type $\mathsf{T'\,b}$, or we first use
$\mathsf{fmap_T}$ to convert the original structure of type
$\mathsf{T\,a}$ to one of type $\mathsf{T\,b}$ and then rearrange that
resulting structure into one of type $\mathsf{T'\,b}$. For example, if
$\mathsf{f : a \to b}$, $\mathsf{xs \,:\,List\,a}$, and $\mathsf{g :
  List\,a\to Tree\,a}$ arranges lists into trees in a type-independent
way, then
\[\mathsf{fmap_{Tree}\,f\,(g \, xs) \,=\, g\,(fmap_{List}\,f\,xs)}\]

\subsubsection{Church Encodings of ADTs}

One way to represent ADTs is via their Church encodings. A Church
encoding is a representation of a data type as a pure function in a
lambda calculus (System F in the case of functional languages). Church
encodings can be used to represent data types in languages that do not
support primitives for sum types, product types, or recursion.
{\color{blue} Church encodings (and the related Parigot encodings)
  have recently been popularized as {\em visitor patterns} in
  object-oriented programming.} Represention of ADTs as their Church
encodings is possible in any language that supports functions.

The Church encodings of the ADTs $\mathsf{List\,a}$ and
$\mathsf{Tree\,a}$, for example, are
\[\mathsf{List\,a} = \mathsf{\forall b.\,b \to (a \to b \to b) \to b}\]
and 
\[\mathsf{Tree\,a} = \mathsf{\forall b.\,(a \to b) \to (b \to a \to b
  \to b) \to b}\] respectively. The argument types are abstractions
of the types of ADTs' constructors. For instance, $\mathsf{b}$
abstracts the type of the constructor $\mathsf{Nil}$, and $\mathsf{a
  \to b \to b}$ abstracts the type of the constructor $\mathsf{Cons}$,
for lists. Because these types are each uniform in their argument type
$\mathsf{a}$, we can also encode the type constructors $\mathsf{List}$
and $\mathsf{Tree}$ themselves as
\[\mathsf{List} = \mathsf{\forall a. \forall b.\,b \to (a \to b \to b)
  \to b}\] 
and 
\[\mathsf{Tree} = \mathsf{\forall a. \forall b.\,(a \to b) \to (b \to
  a \to b \to b) \to b}\]
respectively. 

It is possible to make $\mathsf{List}$ and $\mathsf{Tree}$ instances
of Haskell's $\mathsf{Functor}$ class. Indeed, the Prelude's
$\mathsf{map}$ function does exactly that for $\mathsf{List}$. For
$\mathsf{Tree}$ we can define
\[\begin{array}{lll}
\mathsf{map_{Tree}} & \mathsf{::} & \mathsf{(a \to b) \to Tree\,a \to Tree \,b}\\
\mathsf{map_{Tree}\, f\, (Leaf\,x)} & \mathsf{=} & \mathsf{Leaf\,(f\,x)}\\
\mathsf{map_{Tree}\, f\, (Node\,t_1\,x\,t_2)} & \mathsf{=} &
\mathsf{Node\,(map_{Tree}\, f \,t_1)\,x\,(map_{Tree}\,f\,t_2)}\\
\end{array}\]
But the Church encoding of an ADT carries with it no expectation
whatsoever that such a type-and-data-uniform, type-preserving,
data-changing $\mathsf{map}$ function can be defined. Nevertheless,
for polynomial ADTs this can always be done. But even when the Church
encoding of an ADT supports a $\mathsf{map}$ function, there is no
guarantee that the rearrange-transform property will hold for it.

\subsubsection{ADTs as Primitive Data Types}

By contrast, the ability to define such a $\mathsf{map}$ function is
inherent in the view of ADTs as primitive data types. Such a view is
possible in any language that supports primitives for sum types,
product types, and recursion. In such a language, the primitive
representations of the ADTs $\mathsf{List\,a}$ and $\mathsf{Tree\,a}$
are
\[\mathsf{List\,a} = \mathsf{\mu b.\, 1 + (a \times b)}\] 
and 
\[\mathsf{Tree\,a} = \mathsf{\mu b.\, a + (b \times a \times b)}\]
respectively. Here, $\mathsf{\mu}$ is the primitive fixpoint
operator.

Primitive representations capture in syntax the fact that ADTs can be
considered as fixpoints. For example $\mathsf{List\,a}$ does indeed
satisfy the fixpoint equation indicated above, i.e., is indeed such
that
\[\mathsf{List\,a} = \mathsf{1 + (a \times List\,a)}\] 
since every element of $\mathsf{List\,a}$ is either empty or is
obtained by $\mathsf{Cons}$ing an element of type $\mathsf{a}$ onto an
already-existing structure of type $\mathsf{List\,a}$. (The fixpoint
equation is just a rewriting of the Haskell data type declaration for
$\mathsf{List\,a}$.) The ADT $\mathsf{Tree\,a}$ can similarly be seen
to be a solution to the fixpoint equation
\[\mathsf{Tree\,a} = \mathsf{a + Tree\,a \times a \times Tree\,a}\] 
determined by its Haskell data type declaration.

These fixpoint equations are sensible at the level of types. But to
ensure that the syntactic fixpoint actually denotes a semantic object
computed as a real-live fixpoint, the semantic fixpoint calculation
must converge. This means that the fixpoint being taken must be of
{\em functors} on the category interpreting types~\cite{tfca}, rather
than of mere functions between the sets interpreting them. Said
differently, functions between sets interpreting types must have
corresponding {\em functorial actions} on functions between
interpretations of types. Reflecting this requirement back into syntax
gives that the body of a primitive data type representation of an ADT
must support a $\mathsf{map}$ function, i.e., must be an instance of
Haskell's $\mathsf{Functor}$ class.

A concrete example is given by the ADT $\mathsf{List\,a}$. This type
is is modelled as the fixpoint $\mu F_a$ of the first-order functor
whose action on objects is given by $F_a\,X = 1 + a \times X$ and,
whose action on morphisms (which models its corresponding
$\mathsf{map}$ function) is given by $F_a\,f = 1 + \id_a \times
f$. The type constructor $\mathsf{List}$ is then modelled by the
higher-order functor $H$ whose action on objects is given by $H\,F\,X
= 1 + X \times F\,X$ and whose action on morphisms is given by
$H\,F\,f = \id_1 + f \times F\,f$. (A {\em higher-order functor} in
this context is a functor not directly on the category interpreting
types, but rather on the category of functors interpreting type
constructors, i.e., on the category of functors on the category of
functors interpreting types.) The ADT $\mathsf{List\,a}$ is then
modelled as the functor $(\mu H)a$. Since $H$ is a higher-order
functor, not only does it have a corresponding functorial action,
i.e., support a function $\mathsf{map_H}$, but $\mu H$ is also a
functor with its own functorial action. That is, the type constructor
$\mathsf{List}$ itself supports a function $\mathsf{fmap_{List}}$.

{\color{blue} At the level of objects/types, short cut fusion can be
  used to show that the Church encoding and the primitive
  representation of an ADT are isomorphic.} Note, however, that
whereas it is impossible even to {\em state} the rearrange-transform
property for ADTs whose Church encodings do not have corresponding
$\mathsf{map}$ functions, and although this property is guaranteed to
hold even when it can be stated for Church encodings, it is simply the
reflection back into syntax of the naturality property of the
functorial action of the interpretation of the type constructor
associated with an ADT's representation as a primitive data type.


%In the former case the ADT $\mathsf{List\,a}$ for each $\mathsf{a}$ is
%modelled as the fixpoint of its underlying functor $\mu F_a$; by
%contrast, in the latter case it is the type constructor
%$\mathsf{List}$ itself that is modelled as $\mu H$, and the ADT
%$\mathsf{List\,a}$ is actually modelled as $(\mu H)\,a$. That is, the
%ADTs $\mathsf{List\,a}$ for the various $\mathsf{a}$s are all modelled
%uniformly. (The extreme uniformity comes from the fact that the
%{\color{blue} type/second} argument $X$ is the same in each recursive
%call to $H$. This requirement will be lifted in
%Section~\ref{sec:gadts} below.)  This is precisely what makes it
%possible to define am $\mathsf{fmap}$ for list types.

%A crucial point is that while the $\mu F_a$ is not required to be a
%functor, and thus is not required to have a {\em functorial action} on
%morphisms --- i.e., to support an $\mathsf{fmap_{F_a}}$ function in
%programming terms --- the functor $\mu H$ modelling $\mathsf{List}$
%must do so. Indeed, since $H$ is a higher-order functor, $\mu H$ is
%itself a functor. Applying the functorial action $\mathsf{fmap_{\mu
%    H}}$ of this functor to a morphism $\mathsf{f}$ yields a natural
%transformation $\mathsf{fmap_{\mu H}\,f}$. This in turn guarantees the
%order-independence of type-independent rearrangement and conversion
%via $\mathsf{fmap_{\mu H}\,f}$ for lists {\color{blue} and trees}.

%{\color{blue} Extension of a container}


\begin{comment}
{\color{blue} Type-independent reasoning about ADTs in functional
  languages is usually justified by first representing ADTs by their
  Church encodings, and then reasoning type-independently about these
  encodings. This is typically justified by constructing a parametric
  model --- i.e, a model in which polymorphic functions preserve
  relations {\em \'a la} Reynolds --- for a suitable fragment of
  System F, demonstrating that an initial algebra exists for the
  positive type constructor corresponding to the functor underlying an
  ADT of interest, and showing that each such initial algebra is
  suitably isomorphic to its corresponding Church encoding. In fact,
  this isomorphism of initial algebras and their Church encodings is
  one of the ``litmus tests'' for the goodness of a parametric model.}

This approach works well for ADTs, which are always fixpoints of {\em
  first-order} functors, and whose Church encodings, which involve
quantification over only type variables, are always expressible in
System F. For example, $\mathtt{List\,A}$ is the fixpoint of the
first-order functor $F\,X = 1 + A \times X$ and has Church encoding
$\forall \alpha. \, \alpha \to (\mathtt{A} \to \alpha \to \alpha) \to
\alpha$. But despite Cardelli's~\cite{car97} claim that ``virtually
any basic type of interest can be encoded within F$_2$'' --- i.e.,
within System F --- non-ADT nested types cannot.  Not even our
prototypical nested type of perfect trees has a Church encoding
expressible in System F!  Indeed, $\mathtt{PTree\,A}$ cannot be
represented as the fixpoint of any {\em first-order} functor. However,
it can be seen as the instance at index $\mathtt{A}$ of the fixpoint
of the {\em higher-order} functor $H\,F\,A\,=\, (A \to F\,A) \to (F
\,(A \times A) \to F\,A) \to F\,A$. It thus has Church encoding
$\forall f.\, (\forall \alpha.\,\alpha \to f\alpha) \to (\forall
\alpha. \,f (\alpha \times \alpha) \to f\alpha) \to \forall \alpha.\,
f\alpha$, which requires quantification at the higher kind $* \to *$
for $f$. A similar situation obtains for any (non-ADT) nested
type. Unfortunately, higher-kinded quantification is not available in
System F, so if we want to reason type-independently about nested
types in a language based on it we have only two options: {\em
  i})\,move to an extension of System F, such as the higher-kinded
calculus F$_\omega$ or a dependent type theory, and reason via their
Church encodings in a known parametric model for that extension, or
{\em ii})\, add nested types to System F as primitives --- i.e., as
primitive type-level fixpoints --- and construct a parametric model
for the result.

Since the type systems of F$_\omega$ and dependent type theories are
designed to extend System F with far more than non-ADT data types, it
seems like serious overkill to pass to their parametric models to
reason about nested types in System F. Indeed, such calculi support
fundamentally new features that add complexity to their models that is
entirely unnecessary for reasoning about nested types. This paper
therefore pursues the second option above.  We first design a
Hindley-Milner-style calculus supporting primitive nested types,
together with primitive types of natural transformations representing
morphisms between them. Our calculus can express all nested types
appearing in the literature, including truly nested types.  At the
term-level, it supports primitive pattern matching, map functions, and
fold combinators for nested types.\footnote{We leave incorporating
  general term-level recursion to future work because, as
  Pitts~\cite{pit00} reminds us, ``it is hard to construct models of
  both impredicative polymorphism and fixpoint recursion''. In fact,
  as the development in this paper shows, constructing a parametric
  model even for our predicative calculus with primitive nested types
  --- and even without term-level fixpoints --- is already rather
  involved.  On the other hand, our calculus is strongly normalizing,
  so it perhaps edges us toward the kind of provably total practical
  programming language proposed in~\cite{wad89}.}  Our main
contribution is the construction of a parametric model for our
calculus. This is both delicate and challenging. To ensure the
existence of semantic fixpoints interpreting nested types, and thus to
establish a suitable Identity Extension Lemma, our type system must
explicitly track functoriality of types, and cocontinuity conditions
on the functors interpreting them must be appropriately threaded
throughout the model construction. Our model validates all standard
consequences of parametricity in the presence of primitive nested
types, including the isomorphism of primitive ADTs and their Church
encodings, and correctness of short cut fusion for nested types. The
relationship between naturality and parametricity has long been of
interest, and our inclusion of a primitive type of natural
transformations allows us to clearly delineate those consequences of
parametricity that follow from naturality, from those, such as short
cut fusion for nested types, that require the full power of
parametricity.
\end{comment}

\subsection{GADT Syntax}\label{sec:gadts}

{\color{blue} polynomial} GADTs relax the restriction on the type
instances appearing in a data type definition. The special form of
GADTs known as {\em nested types} allow the data constructors of a
GADT to take as arguments data whose types involve type instances of
the GADT other than the one being defined. For example, the 
definition 
\[\mathsf{PTree \, a = PLeaf \,a \,|\, PNode \,(PTree\,(a \times
  a))}\] of the nested type $\mathsf{PTree\,a}$ of perfect trees
introduces the data constructor(s $\mathsf{PLeaf :: a \to PTree\,a}$
and) $\mathsf{PNode :: PTree \,(a\times a) \to PTree\,a}$. It enforces
not only the invariant that all of the data in a structure of type
$\mathsf{PTree\,a}$ is of the same type $\mathsf{a}$, but also the
invariant that all perfect trees have lengths that are powers of
2. More general GADTs allow their constructors both to take as
arguments {\em and return as results} data whose types involve type
instances of the GADT other than the one being defined. An example is
the following GADT, given in Haskell syntax:
\[\begin{array}{l}
\mathsf{data\, Seq :: * \to *\,where}\\
\mathsf{\;\;\;\;\;\;\;\;Const ::\, a \to Seq\,a}\\
\mathsf{\;\;\;\;\;\;\;\;Pair\,\,\,\,\, ::\, Seq \,a \to Seq\,b \to
  Seq\,(a \times b)}\\ 
\end{array}\]
Since the return type of the data constructor $\mathsf{Pair}$ is not
simply a variable, $\mathsf{Seq}$ is not a nested type.

By contrast with the ADT $\mathsf{List\, a}$, where the type parameter
$\mathsf{a}$ is integral to the type being defined, the type parameter
$\mathsf{a}$ in $\mathsf{PTree\,a}$ and $\mathsf{Seq\,a}$ appears as a
``dummy'' parameter used only to give the kind \verb|* -> *| of the
type constructors $\mathsf{PTree}$ and $\mathsf{Seq}$. This is
explicitly captured by the above GADT syntax. A GADT --- even a nested
type --- thus does not define a {\em family of inductive types}, one
for each type argument, like an ADT does, but instead defines an
entire family of types that must be constructed simultaneously. That
is, a GADT defines an {\em inductive family of types}.

\subsubsection{Church Encodings of GADTs}

The syntax of GADTs is consonant with programmers' frequent use of
them in situations where type uniformity is specifically to be
eschewed.  {\color{blue} Typical applications of GADTs include generic
  programming, modeling programming languages via higher-order
  abstract syntax, maintaining invariants in data structures, and
  expressing constraints in embedded domain-specific languages. GADTs
  have also been used to implement tagless
  interpreters~\cite{pl04,pr06,pvww06} by trading the definition of a
  universal value domain for a direct specification of the property of
  being a value. Other applications are described in,
  e.g.,~\cite{js15,rou06,MORE,UPDATED}. OTHER SITUATIONS NEEDING
  TYPE-BASED DISPATCH.}

GADTs are useful in such situations precisely because generalizing the
type arguments in the instances of a GADT appearing as the return
types of its data constructors entails a strong connection between a
GADTs shape and the data it contains.  With ADTs, we first choose the
shape of the container and then fill that container with data of
whatever type we like. Critically, the choice of shape is independent
of the data to be stored. With GADTs, however, the shape of the
container may actually {\em depend} on (the type of) the data to be
contained. On other words, the shape and the data are not independent!
For example, while $\mathsf{Const}$ can create data of any shape
$\mathsf{Seq\,a}$, $\mathsf{Pair}$ can produce data of shape
$\mathsf{Seq\,a}$ where $\mathsf{a}$ is a pair type. As a result, it
may not always be possible to modify a GADT's data while retaining its
shape because changing (the type of) the data may actually change the
shape.

To determine the possible shapes of a GADT's container we have to
pattern-match on the type of the data to be contained. For this, our
GADT calculus must support an equality type $\mathsf{Eq}$ that is
inhabited when its two type arguments are the same and is uninhabited
otherwise, i.e., is the syntactic reflection of semantic equality
function $Eq$. Such a type can either be taken as primitive, as in
\[\begin{array}{l}
\mathsf{data\, Eq :: * \to * \to *\,where}\\
\mathsf{\;\;\;\;\;\;\;\;Refl ::\, Eq\,a\,a}\\
\end{array}\]
\noindent
or
represented by its own Church encoding
\[\mathsf{Eq\,t_1\,t_2} = \mathsf{\forall f.\, (\forall a.\,
  f\,a\,a) \to f\,t_1\,t_2}\] Using $\mathsf{Eq}$, the GADT
$\mathsf{Seq}$ can then be represented as its Church encoding
         {\color{red}
\[\mathsf{Seq\,a} = \mathsf{\forall f.\, (\forall a.\,a \to f\, a)
  \to (\forall a\,b\,c.\,Eq\,a\,(b\times c) \to f\,b \to f\,c \to
  f\,a) \to f\,a}\]}
in a higher-kinded calculus such as $F_\omega$.

Importantly, $Eq\,a$ cannot be made into a functor, or, equivalently,
$\mathsf{Eq\,a}$ cannot be made an instance of Haskell's
$\mathsf{Functor}$ class: if $\mathsf{Eq\,a}$ supported a function
$\mathsf{map_{Eq\,a} :: (b \to c) \to Eq\,a\,b \to Eq\,a\,c}$ then
defining
\[\begin{array}{l}
\mathsf{eqElim :: Eq\, a\, b \to b \to a}\,{\color{red} was Eq\,a\,b\to a\to b}\\
\mathsf{eqElim\, Refl\, x = x} 
\end{array}\]
would allow us to construct the element 
\[\begin{array}{l}
\mathsf{emptyFromMapEqa :: (\forall b\, c.\, (b \to c) \to Eq\, a\,
  b \to Eq\, a \,c) \to Void}\\ 
\mathsf{emptyFromMapEqa \,map_{Eq\,a} = eqElim\, (map_{\Eq\,a}\,
  absurd\, Refl)\, ()}  
\end{array}\]
of type $\mathsf{Void}$, which is impossible. It follows immediately
from this contradiction that $\mathsf{Seq}$ cannot support a map
function either. This is already intuitively evident from the Haskell
definition of $\mathsf{Seq}$, though. Indeed, the clause of
$\mathsf{map}$ for the $\mathsf{Pair}$ constructor should feed
$\mathsf{map}$ a function $\mathsf{f :: (b \times c) \to e}$ and a
term of the form $\mathsf{Pair \,t_1\,t_2}$ for $\mathsf{x :: b}$ and
$\mathsf{y :: c}$, and produce a term $\mathsf{Pair\,t_1'\,t_2'}$ for
some appropriately typed terms $\mathsf{t_1'}$ and
$\mathsf{t_2'}$. But it is not clear how to achieve this since
$\mathsf{e}$ need not necessarily be a product type. And even if
$\mathsf{e}$ {\em were} known to be of the form $\mathsf{w \times z}$,
then we still wouldn't necessarily have a way to produce data of type
$\mathsf{w \times z}$ from only $\mathsf{f :: b \times c \to w \times
  z}$ and $\mathsf{x}$ and $\mathsf{y}$ unless we knew, e.g., that
$\mathsf{f}$ was a pair of functions.

\subsubsection{GADTs as Primitive Data Types}

The primitive representations of {\color{blue} polynomial} GADTs
capture in syntax the fact that GADTs can be considered as fixpoints
of higher-order functors. Syntactically, the use of higher-order
funct{\em ions} is essential, since the type arguments to the GADT
being defined are not uniform across all of its instances in the types
of its data constructors. Semantically, the use of higher-order
funct{\em ors} is essential, as for ADTs, to the very existence of the
fixpoints being computed~\cite{tfca}.
%We will consider GADTs'
%semantics in more detail in Section~\ref{sec:GADT-semantics}.

Consider, again, the GADT $\mathsf{Seq}$ from the start of
Section~\ref{sec:gadts}. Because its type argument varies in the
instances of $\mathsf{Seq}$ appearing in the types of its data
constructor $\mathsf{Pair}$, $\mathsf{Seq}$ cannot be characterized by
any first-order fixpoint equation. It can, however, be viewed as a
solution to the higher-order fixpoint equation
\[H\,f\,a = a \,+\,Lan_{\lambda a b.\,a \times b}\, \lambda
a b.\,f\,a \times f\,b\] 
where $\Lan_K \,G$ is the 
%$Lan_k\, g$ is the syntactic reflection of the
{\em left Kan extension}~\cite{mac71}
%$\Lan_K \,G$
of the functor $G$
%interpreting $\mathsf{g}$
along the functor $K$.
%interpreting $\mathsf{k}$~\cite{mac71}.
The left Kan extension $\Lan_K \,G$ is the smallest functor
approximating/extending $K$ and coinciding with $G$ {\color{blue}
  better explanation}.



LKE gives the functorial completion of the GADT.

%written as
%%\[\mathsf{Seq\,a} = \mathsf{a + Eq\,a\,(b \times c) \times Seq\,b
%%  \times Seq\,c} \] 
%\[\begin{array}{l}
%\mathsf{data\, Seq :: * \to *\,where}\\
%\mathsf{\;\;\;\;\;\;\;\;Const ::\, a \to Seq\,a}\\
%\mathsf{\;\;\;\;\;\;\;\;Pair\,\,\,\,\, ::\, Eq\,a\,(b \times c) \to
%  Seq \,b \to Seq\,c \to Seq\,a}\\ 
%\end{array}\]
%
%The above rewriting of $\mathsf{Seq}$'s definition shows that it can
%be seen as a fixpoint $\mu H$ of the higher-order function $H$ defined
%on objects by $H\,F\,a = a + Eq\,a\,(b \times c) \times F\,b \times
%F\,c$. 

{\color{blue} Need syntax that reflects semantic Lan.}


%Other GADTs are similarly characterizable by higher-order functors.
%{\color{blue} GADTs make essential use of higher-order functors.}


Need syntax functors (Lan)
The representation choices are similar for GADTs. The difference is
that the two are not isomorphic.

Certain polynomial GADTs correspond to polynomial ADTs. So we expect
the same behaviors.  Polynomial nested types are actually the GADTs
that mimic polynomial ADT behavior.

But pattern matching fails for the extension (functorial completion)
of a GADT.


\section{Semantics of GADTs}\label{sec:GADT-semantics}

{\color{blue} OMIT THIS SECTION?}

But the data in a GADT shouldn't disappear just because you map a
function over it!  This is odd! So in what sense to GADTs generalize
ADTs? In what sense are they even data types?

The semantics of GADTs is not so clear.

Functorial completion doesn't give nested types

Functorial completion containerizes the GADT. Extension of a
container. The extension of a container is an endofunctor generalizing
map.


Eq semantics minimal (least commitment, most natural) for CEs; Lan
semantics minimal for primitives. Minimal way to add to any calculus
supporting them.


Bob requires map functions for fixpoint but then doesn't use them.

An analysis of the Eq A B type in Bob's systems shows that the set
interpretation is empty whenever the A and B have different set
interpretations.  Likewise, its relational interpretation is the empty
relation whenever the A and B have different relational
interpretations (even when the underlying set interpretations are the
same).  The consequences of parametricity are thus immediate: there
can be no term of type Eq A B if A and B are different types.
Moreover, this behavior is fundamentally non-functorial, so that there
is no hope of defining a map for Eq.  In particular, that means we
cannot expect to retrieve non-trivial naturality properties for the Eq
type, despite these are usually considered to be a special case of
parametricity properties.  Indeed, the common understanding that
parametricity generalizes naturality only makes sense for functorial
types.  We can specialize these considerations to the type Eq 1 B,
representing G, our example GADT defined by a unique constructor c ::
G 1.

Since the types encoding GADTs in Bob's system make essential use
of equality types, these considerations extend and possibly generalize
to all GADTs.  Still, the equality is just a shorthand for a type made
by higher-order forall and arrow types, and thus this considerations
are not limited to a specific type, but are intrinsic of Bob's
system.
%We can thus expect that not all types in Bob's system will be
%functorial and that, consequently, the consequences of parametricity
%will not include functoriality but will provide information about,
%e.g., inhabitatedness.  See
%<https://www.cis.upenn.edu/~sweirich/talks/param-gadt.pdf> and
%<https://www.seas.upenn.edu/~sweirich/papers/rtheorems.pdf> for more
%about parametricity and GADTs.

In contrast, $(Lan_1 1) B$, the type representing G in our system, is
functorial (as are all types in our systems) and inhabited for every
instance of B other than 0.  Indeed, we have a term $c : (Lan_1 1) 1$
given by the Lan introduction rule, and thus every term t : T yields a
term in $(Lan_1 1) T$ by use of the map of $(Lan_1 1) B$.  This shows
that, as expected by now, the same GADT really does have different
semantic interpretations in Bob's system and in ours.

\section{Parametricity}

{\em Relational parametricity} encodes a powerful notion of
type-uniformity, or representation independence, for data types in
functional languages. It formalizes the intuition that a polymorphic
program must act uniformly on all of its possible type instantiations
by requiring that every such program preserves all relations between
pairs of types at which it is instantiated. Parametricity was
originally put forth by Reynolds~\cite{rey83} for System F. It was
later popularized as Wadler's ``theorems for free''~\cite{wad89}, so
called because it can deduce properties of programs in such languages
solely from their types, i.e., with no knowledge whatsoever of the
text of the programs involved.  Most of Wadler's free theorems are
consequences of naturality for polymorphic list-processing
functions. However, parametricity can also derive results that go
beyond just naturality, such as establishing {\color{blue} proving}
the equivalence of the Church encodings and the primitive
representations of ADTs and {\color{blue} hereditarily polynomial}
nested types by validating the aforementioned short cut fusion
transformation for them.


One important consequence of representation choice is for parametric
reasoning about GADTs. In particular, we can use parametricity to get
our transform-rearrange property for ADTs. Do it for lists?



We can thus expect that not all types in Bob's system will
be functorial and that, consequently, the consequences of
parametricity will not include functoriality but will provide
information about, e.g., inhabitatedness.  See
<https://www.cis.upenn.edu/~sweirich/talks/param-gadt.pdf> and
<https://www.seas.upenn.edu/~sweirich/papers/rtheorems.pdf> for more
about parametricity and GADTs.


If we had a parametric model for GADTs, it could perhaps validate
short cut and this show that the two reps for them are equivalent.
(This is a sort of converse.)
The fact that GADTs don't validate any such iso has important
consequences for their parametricity properties. 
  
In this paper we show that there is not an iso between the two reps of
even the simplest GADT {\color{blue} that isn't a hereditarily
  polynomial nested type}, and investigate the consequences of this
discovery for parametric reasoning about GADTs.


Non-uniformity renders build, therefore short cut, for GADTs
impossible. Can't even formulate! Maybe there's another way to prove
equivalence, but I don't know what it would be. In any case, we have a
counterexample, so this settles the question.



Importantly, we get the same parametricity properties for ADTs whether
we regard them as fixpoints of higher-order functors --- i.e., as
primitive data types --- or as their Church encodings. This is not the
case for GADTs that are not nested types. (What about truly nested
types?)

We do minimal thing to fix functoriality, then investigate
parametricity properties for both interpretations. We are forced to
choose. 

Parametric polymorphism as a way to capture Strachey's notion of
uniform, type-independent behavior of functions. Parametric models as
formalizations of this notion of uniformity, as first given in
Reynolds' original paper.


Bob's system gives a parametric model that models GADTs. But given the
non-uniformity captured by GADTs by purposeful design, that should be
both completely unexpected and alarming! However, upon reflection it's
not really all that surprising once we realize what the parametricity
properties of GADTs in Bob's system look like.

%It turns out that such properties are essentially trivial (because of
%its essential use of Eq, which itself has a trivial parametricity
%property --- IS THIS TRUE??). So, in the end, parametricity is not
%really something I can use to reason non-trivially about GADTs in
%Bob's system. (MAY NEED TO REVISE AFTER INVESTIGATION INTO PAR
%PROPERTY OF EQ.)

Tellingly, Bob gives no parametricity properties for GADTs. If he had,
it would have illustrated the phenomenon we are describing, raised the
question: What good is a par property for a GADT or another type using
the Eq type?




GADTs Force a Representation Choice. It has consequences for how one
reasons about them.

It depends on primitive vs. Church encodings. Primitive forces
functoriality, as in LICS'19.

FOR ADTs and Nesteds, it does not depend. We get the same par property
whether they are seen as primitives or we use their CE.



\section{Conclusion}

{\color{blue} Some of this probably belongs in previous section.}

Moreover, the consequences of parametricity that one has in Bob's
system do not hold in our system, as we can construct terms of type
$(Lan_1 1)$ B for instances of B other than 1.  Thus, either our
system does not admit any parametric model, or, if it does, the
consequences of parametricity are different.  That the consequences of
parametricity depended on the model or the representation of GADTs
would come as a surprise, though!  (IT DOES. CONSIDER INTERPRETING
BASE TYPES AS ORDERED SETS. THIS GIVES DIFFERENT PARAMETRICITY
PROPERTIES THAN INTERPRETING THEM AS DIAGONAL RELATIONS.)

That parametricity properties for GADTs in Bob's system are as we
described before is interesting because, for ADTs and nested types,
which GADTs claim to generalize, we expect naturality to be a special
case of parametricity. However, we don't actually get non-trivial
naturality consequences of parametricity for GADTs in Bob's
system. Why? Because this would require GADTs to be interpreted as
functors, which, in Bob's system, requires the programmer to supply a
map function on the intepretation of every GADT defined. Moreover,
each such map function would have to behave functorially. Bob's system
does indeed deliver non-trivial naturality results for every GADT for
which the user can define a map function that behaves
functorially. The problem is that there may not be many such GADTs (in
fact it will not be possible for proper GADTs, i.e., GADTs that are
not actually nested types).








Perhaps expecting naturality as a special case of parametricity is
wrongheaded, though, since, with its arbitrary forall-types and arrow
types, System F is inherently non-functorial. Using Church encodings,
Bob's system handles arbitrary GADT syntax. However, GADTs are not
guaranteed to behave functorially in his system, and the user is
ultimately tasked with supplying the relevant map functions (and
proving their properties) for any GADTs they want to work with. That
the user must supply the map function for a GADT of interest makes
sense: Church encodings cannot say anything about functorial nature of
data types since they describe only the data (and not the action on
morphisms since they are not required to support map functions).  Our
approach is different. By contrast with Bob's system, we restrict to
GADTs generated by a specific grammar, and this grammar is designed
specifically to guarantee functoriality of all data types definable in
it. We also treat GADTs as primitive data types --- i.e., as fixpoints
of (higher-order) functors --- rather than via their Church encodings
since this more closely aligns with how ADTs are treated in
implemented programming languages.

Wrt Bob's system we can ask: what good is parametricity for GADTs in a
system that derives only trivial naturality consequences of
parametricity for them? It may be possible to get non-naturality
consequences of parametricity (such as short cut fusion or
VYTINIOTIS'?) for GADTs in Bob's system (IS IT?), but since many of
the most useful consequences of parametricity for ADTs and nested
types are actually natuality results (IS THIS TRUE? MAYBE NOT FOR
GADTs), we'd be missing out on a very large class of expected and
useful free theorems. Indeed, many of Wadler's free theorems are
precisely naturality results for functions over data types (mainly
lists, there). If GADTs really do generalize ADTs (and nested types)
then we would expect to be able to derive analogues for GADTs of the
parametricity results --- including naturality results --- that we can
derive for ADTs. Yet, by design we cannot. So are GADTs actually
misnamed? In what sense are they data types, and in what sense do they
generalize ADTs? Perhaps what are now known as nested types should
really be called GADTs since *they* are actually the (level-2)
higher-kinded generalizations of ADTs.

In our system we can show that we can recover the expected naturality
consequences of parametricity for (at least a large class of) GADTs
(that can be characterized by a grammar for which every type is
interpreted as a functor, including those characterized as least
fixpoints, i.e., as carriers of initial algebras of functors, i.e., as
primitive data types). The way our system achieves this is by
interpreting GADTs as their functorial completions. So in our system
the interpretations of GADTs are guaranteed to have fmap functions ---
i.e., to be functorial --- and to satisfy the expected naturality
properties. And our free theorems really are free for GADTs that fit
our syntax. But now the bump is elsewhere under the rug: First,
semantically we don't really have the same GADT anymore (functoriality
adds extra data to the data type), and second, our counterexample
shows that we don't get non-naturality consequences of parmetricity
even for the data types defined by our grammar.  So this isn't really
what we wanted either.


The upshot is that we can consider GADTs as data types (i.e., we can
compute the semantics of GADTs) in two ways: just as determined by
their syntax, as in Bob's system, or as the functorial completion of
their syntax, obtained by supplying a suitable map primitive, as in
ours. But, either way, GADTs fail to be well-behaved wrt
parametricity. In the first case, GADTs don't have fmaps so their
parametricity properties can't include naturality properties. In the
second case, we still can't get non-naturality consequences of
parametricity. That is, we can't get BOTH the naturality consequences
of parametricity AND the non-naturality consequences of parametricity
at the same time.

One can read our counterexample as showing that Bob's parametricity
results cannot be extended to one that gives a functorial semantics to
GADTs. So the user is forced to make a choice between naturality
consequences of parametricity and other consequences of
parametricity. If GADTs really did generalize ADTs, and really were
data types in a normal sense, then the user would not have to.







\bibliography{gadt-story-references}

\end{document}

Daniel:

I have tried reconstructing my example using our simpler GADT with a single
constructor c :: G 1. I don't think it is possible to modify my example for a
GADT with only a single constructor. In particular, I don't see a way to
exploit the map function unless we have constructors for G X and G Y where X
!= Y.
If we assume a map function exists, then given f :: 1 -> A we can produce
(map f c) :: G A. But I don't know what to do from there.

I was able to simplify the types in the example slightly though. The example
I gave before can be adapted to use either

data G3 b where
    G3unit :: G3 ()
    G3id    :: b -> G3 (Either b ())

or

data G5 b where
    G5unit   :: G5 ()
    G5void   :: Void -> G5 Bool

as the problematic GADT.

Enrico:

Just to make sure I understand it correctly: when we define the function

unGbool :: G c Bool -> c
unGbool (Gbool x) = x

we are doing so in Haskell, right?
In our system, it would not be possible to define unGbool (because we could
not define it only on the constructor Gbool of G). For this reason, this
example provides a contradiction in Haskell (+functoriality), but not in our
system. Is that correct?

Daniel:

The example is Haskell and relies on Haskell's rules for
pattern matching.








Enrico:

By the way, as a general remark not directly related to the introduction you
wrote, I think that the set semantics of our type G is

[[G \alpha]]^set [\alpha := A] = A

Indeed, (Lan_1 1) A = A for every set A. Intuitively this makes sense,
because for every a : A there is a fuction

k_a : 1 —> A

and thus an element

map k_a \ast : G A

In particular, [[G 0]]^set = \emptyset.

Still, it is NOT true that its relational interpretation is [[G \alpha]]^rel
[\alpha := R] = R (and indeed the relational interpretation is not fibered
over the set interpretation)!
