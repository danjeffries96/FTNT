% For double-blind review submission, w/o CCS and ACM Reference (max
% submission space)
\documentclass[acmsmall,review,anonymous]{acmart}
\settopmatter{printfolios=true,printccs=false,printacmref=false}
%% For double-blind review submission, w/ CCS and ACM Reference
%\documentclass[acmsmall,review,anonymous]{acmart}\settopmatter{printfolios=true}
%% For single-blind review submission, w/o CCS and ACM Reference (max submission space)
%\documentclass[acmsmall,review]{acmart}\settopmatter{printfolios=true,printccs=false,printacmref=false}
%% For single-blind review submission, w/ CCS and ACM Reference
%\documentclass[acmsmall,review]{acmart}\settopmatter{printfolios=true}
%% For final camera-ready submission, w/ required CCS and ACM Reference
%\documentclass[acmsmall]{acmart}\settopmatter{}


%% Journal information
%% Supplied to authors by publisher for camera-ready submission;
%% use defaults for review submission.
\acmJournal{PACMPL}
\acmVolume{1}
\acmNumber{POPL} % CONF = POPL or ICFP or OOPSLA
\acmArticle{1}
\acmYear{2020}
\acmMonth{1}
\acmDOI{} % \acmDOI{10.1145/nnnnnnn.nnnnnnn}
\startPage{1}

%% Copyright information
%% Supplied to authors (based on authors' rights management selection;
%% see authors.acm.org) by publisher for camera-ready submission;
%% use 'none' for review submission.
\setcopyright{none}
%\setcopyright{acmcopyright}
%\setcopyright{acmlicensed}
%\setcopyright{rightsretained}
%\copyrightyear{2018}           %% If different from \acmYear

%% Bibliography style
\bibliographystyle{ACM-Reference-Format}
%% Citation style
%% Note: author/year citations are required for papers published as an
%% issue of PACMPL.
\citestyle{acmauthoryear}   %% For author/year citations
%\citestyle{acmnumeric}

%%%%%%%%%%%%%%%%%%%%%%%%%%%%%%%%%%%%%%%%%%%%%%%%%%%%%%%%%%%%%%%%%%%%%%
%% Note: Authors migrating a paper from PACMPL format to traditional
%% SIGPLAN proceedings format must update the '\documentclass' and
%% topmatter commands above; see 'acmart-sigplanproc-template.tex'.
%%%%%%%%%%%%%%%%%%%%%%%%%%%%%%%%%%%%%%%%%%%%%%%%%%%%%%%%%%%%%%%%%%%%%%



\usepackage[utf8]{inputenc}
\usepackage{ccicons}
\usepackage{verbatim}

\usepackage{amsmath}
\usepackage{amsthm}
\usepackage{amscd}
%\usepackage{MnSymbol}
\usepackage{xcolor}

\usepackage{bbold}
\usepackage{url}
\usepackage{upgreek}
%\usepackage{stmaryrd}

\usepackage{lipsum}
\usepackage{tikz-cd}
\usetikzlibrary{cd}
\usetikzlibrary{calc}
\usetikzlibrary{arrows}

\usepackage{bussproofs}
\EnableBpAbbreviations

\DeclareMathAlphabet{\mathpzc}{OT1}{pzc}{m}{it}

%\usepackage[amsmath]{ntheorem}

\newcommand{\lam}{\lambda}
\newcommand{\eps}{\varepsilon}
\newcommand{\ups}{\upsilon}
\newcommand{\mcB}{\mathcal{B}}
\newcommand{\mcD}{\mathcal{D}}
\newcommand{\mcE}{\mathcal{E}}
\newcommand{\mcF}{\mathcal{F}}
\newcommand{\mcP}{\mathcal{P}}
\newcommand{\mcI}{\mathcal{I}}
\newcommand{\mcJ}{\mathcal{J}}
\newcommand{\mcK}{\mathcal{K}}
\newcommand{\mcL}{\mathcal{L}}
\newcommand{\WW}{\mathcal{W}}

\newcommand{\ex}{\mcE_x}
\newcommand{\ey}{\mcE_y}
\newcommand{\bzero}{\boldsymbol{0}}
\newcommand{\bone}{{\boldsymbol{1}}}
\newcommand{\tB}{{\bone_\mcB}}
\newcommand{\tE}{{\bone_\mcE}}
\newcommand{\bt}{\mathbf{t}}
\newcommand{\bp}{\mathbf{p}}
\newcommand{\bsig}{\mathbf{\Sigma}}
\newcommand{\bpi}{\boldsymbol{\pi}}
\newcommand{\Empty}{\mathtt{Empty}}
\newcommand{\truthf}{\mathtt{t}}
\newcommand{\id}{id}
\newcommand{\coo}{\mathtt{coo\ }}
\newcommand{\mcC}{\mathcal{C}}
\newcommand{\Rec}{\mathpzc{Rec}}
\newcommand{\types}{\mathcal{T}}

%\newcommand{\Homrel}{\mathsf{Hom_{Rel}}}
\newcommand{\HomoCPOR}{\mathsf{Hom_{\oCPOR}}}

%\newcommand{\semof}[1]{\llbracket{#1}\rrbracket^\rel}
\newcommand{\sem}[1]{\llbracket{#1}\rrbracket}
\newcommand{\setsem}[1]{\llbracket{#1}\rrbracket^\set}
\newcommand{\relsem}[1]{\llbracket{#1}\rrbracket^\rel}
\newcommand{\dsem}[1]{\llbracket{#1}\rrbracket^{\mathsf D}}
\newcommand{\setenv}{\mathsf{SetEnv}}
\newcommand{\relenv}{\mathsf{RelEnv}}
\newcommand{\oCPOenv}{\mathsf{SetEnv}}
\newcommand{\oCPORenv}{\mathsf{RelEnv}}
\newcommand{\oCPOsem}[1]{\llbracket{#1}\rrbracket^{\set}}
\newcommand{\oCPORsem}[1]{\llbracket{#1}\rrbracket^{\rel}}
\newcommand{\denv}{\mathsf{DEnv}}

\newcommand{\rel}{\mathsf{Rel}}
\newcommand{\setof}[1]{\{#1\}}
\newcommand{\letin}[1]{\texttt{let }#1\texttt{ in }}
\newcommand{\comp}[1]{{\{#1\}}}
\newcommand{\bcomp}[1]{\{\![#1]\!\}}
\newcommand{\beps}{\boldsymbol{\eps}}
%\newcommand{\B}{\mcB}
%\newcommand{\Bo}{{|\mcB|}}

\newcommand{\lmt}{\longmapsto}
\newcommand{\RA}{\Rightarrow}
\newcommand{\LA}{\Leftarrow}
\newcommand{\rras}{\rightrightarrows}
\newcommand{\colim}[2]{{{\underrightarrow{\lim}}_{#1}{#2}}}
\newcommand{\lift}[1]{{#1}\,{\hat{} \; \hat{}}}
\newcommand{\graph}[1]{\langle {#1} \rangle}

\newcommand{\carAT}{\mathsf{car}({\mathcal A}^T)}
\newcommand{\isoAto}{\mathsf{Iso}({\mcA^\to})}
\newcommand{\falg}{\mathsf{Alg}_F}
\newcommand{\CC}{\mathsf{Pres}(\mathcal{A})}
\newcommand{\PP}{\mathcal{P}}
\newcommand{\DD}{D_{(A,B,f)}}
\newcommand{\from}{\leftarrow}
\newcommand{\upset}[1]{{#1}{\uparrow}}
\newcommand{\smupset}[1]{{#1}\!\uparrow}

\newcommand{\Coo}{\mathpzc{Coo}}
\newcommand{\code}{\#}
\newcommand{\nat}{\mathpzc{Nat}}

\newcommand{\eq}{\; = \;}
\newcommand{\of}{\; : \;}
\newcommand{\df}{\; := \;}
\newcommand{\bnf}{\; ::= \;}

\newcommand{\zmap}[1]{{\!{\between\!\!}_{#1}\!}}
\newcommand{\bSet}{\mathbf{Set}}

\newcommand{\dom}{\mathsf{dom}}
\newcommand{\cod}{\mathsf{cod}}
\newcommand{\adjoint}[2]{\mathrel{\mathop{\leftrightarrows}^{#1}_{#2}}}
\newcommand{\isofunc}{\mathpzc{Iso}}
\newcommand{\ebang}{{\eta_!}}
\newcommand{\lras}{\leftrightarrows}
\newcommand{\rlas}{\rightleftarrows}
\newcommand{\then}{\quad\Longrightarrow\quad}
\newcommand{\hookup}{\hookrightarrow}

\newcommand{\spanme}[5]{\begin{CD} #1 @<#2<< #3 @>#4>> #5 \end{CD}}
\newcommand{\spanm}[3]{\begin{CD} #1 @>#2>> #3\end{CD}}
\newcommand{\pushout}{\textsf{Pushout}}
\newcommand{\mospace}{\qquad\qquad\!\!\!\!}

\newcommand{\natur}[2]{#1 \propto #2}

\newcommand{\Tree}{\mathsf{Tree}\,}
\newcommand{\GRose}{\mathsf{GRose}\,}
\newcommand{\List}{\mathsf{List}\,}
\newcommand{\PTree}{\mathsf{PTree}\,}
\newcommand{\Bush}{\mathsf{Bush}\,}
\newcommand{\Forest}{\mathsf{Forest}\,}
\newcommand{\Lam}{\mathsf{Lam}\,}
\newcommand{\LamES}{\mathsf{Lam}^+}
\newcommand{\Expr}{\mathsf{Expr}\,}

\newcommand{\ListNil}{\mathsf{Nil}}
\newcommand{\ListCons}{\mathsf{Cons}}
\newcommand{\LamVar}{\mathsf{Var}}
\newcommand{\LamApp}{\mathsf{App}}
\newcommand{\LamAbs}{\mathsf{Abs}}
\newcommand{\LamSub}{\mathsf{Sub}}
\newcommand{\ExprConst}{\mathsf{Const}}
\newcommand{\ExprPair}{\mathsf{Pair}}
\newcommand{\ExprProj}{\mathsf{Proj}}
\newcommand{\ExprAbs}{\mathsf{Abs}}
\newcommand{\ExprApp}{\mathsf{App}}
\newcommand{\Ptree}{\mathsf{Ptree}}

\newcommand{\kinds}{\mathpzc{K}}
\newcommand{\tvars}{\mathbb{T}}
\newcommand{\fvars}{\mathbb{F}}
\newcommand{\consts}{\mathpzc{C}}
\newcommand{\Lan}{\mathsf{Lan}}
\newcommand{\zerot}{\mathbb{0}}
\newcommand{\onet}{\mathbb{1}}
\newcommand{\bool}{\mathbb{2}}
\renewcommand{\nat}{\mathbb{N}}
%\newcommand{\semof}[1]{[\![#1]\!]}
%\newcommand{\setsem}[1]{\llbracket{#1}\rrbracket^\set}
\newcommand{\predsem}[1]{\llbracket{#1}\rrbracket^\pred}
%\newcommand{\todot}{\stackrel{.}{\to}}
\newcommand{\todot}{\Rightarrow}
\newcommand{\bphi}{{\bm \phi}}

\newcommand{\bm}[1]{\boldsymbol{#1}}

\newcommand{\cL}{\mathcal{L}}
\newcommand{\T}{\mathcal{T}}
\newcommand{\Pos}{P\!}
%\newcommand{\Pos}{\mathcal{P}\!}
\newcommand{\Neg}{\mathcal{N}}
\newcommand{\Hf}{\mathcal{H}}
\newcommand{\V}{\mathbb{V}}
\newcommand{\I}{\mathcal{I}}
\newcommand{\Set}{\mathsf{Set}}
%\newcommand{\Nat}{\mathsf{Nat}}
\newcommand{\Homrel}{\mathsf{Hom_{Rel}}}
\newcommand{\CV}{\mathcal{CV}}
\newcommand{\lan}{\mathsf{Lan}}
\newcommand{\Id}{\mathit{Id}}
\newcommand{\mcA}{\mathcal{A}}
\newcommand{\inl}{\mathsf{inl}}
\newcommand{\inr}{\mathsf{inr}}
%\newcommand{\case}[3]{\mathsf{case}\,{#1}\,\mathsf{of}\,\{{#2};\,{#3}\}}
\newcommand{\tin}{\mathsf{in}}
\newcommand{\fold}{\mathsf{fold}}
\newcommand{\Eq}{\mathsf{Eq}}
\newcommand{\Hom}{\mathsf{Hom}}
\newcommand{\curry}{\mathsf{curry}}
\newcommand{\uncurry}{\mathsf{uncurry}}
\newcommand{\eval}{\mathsf{eval}}
\newcommand{\apply}{\mathsf{apply}}
\newcommand{\oCPO}{{\mathsf{Set}}}
\newcommand{\oCPOR}{{\mathsf{Rel}}}
\newcommand{\oCPORT}{{\mathsf{RT}}}

\newcommand{\ar}[1]{\##1}
\newcommand{\mcG}{\mathcal{G}}
\newcommand{\mcH}{\mathcal{H}}
\newcommand{\TV}{\mathpzc{V}}

\newcommand{\essim}[1]{\mathsf{EssIm}(#1)}
\newcommand{\hra}{\hookrightarrow}

\newcommand{\ol}[1]{\overline{#1}}
\newcommand{\ul}[1]{\underline{#1}}
\newcommand{\op}{\mathsf{op}}
\newcommand{\trige}{\trianglerighteq}
\newcommand{\trile}{\trianglelefteq}
\newcommand{\LFP}{\mathsf{LFP}}
\newcommand{\LAN}{\mathsf{Lan}}
%\newcommand{\Mu}{{\mu\hskip-4pt\mu}}
\newcommand{\Mu}{{\mu\hskip-5.5pt\mu}}
%\newcommand{\Mu}{\boldsymbol{\upmu}}
\newcommand{\Terms}{\mathpzc{Terms}}
\newcommand{\Ord}{\mathpzc{Ord}}
\newcommand{\Anote}[1]{{\color{blue} {#1}}}
\newcommand{\Pnote}[1]{{\color{red} {#1}}}

\newcommand{\greyout}[1]{{\color{gray} {#1}}}
\newcommand{\ora}[1]{\overrightarrow{#1}}

%\newcommand{\?}{{.\ }}
%\theoremheaderfont{\scshape}
%\theorembodyfont{\normalfont}
%\theoremseparator{.\ \ }
\newtheorem{thm}{Theorem}
\newtheorem{dfn}[thm]{Definition}
\newtheorem{prop}[thm]{Proposition}
\newtheorem{cor}[thm]{Corollary}
\newtheorem{lemma}[thm]{Lemma}
\newtheorem{rmk}[thm]{Remark}
\newtheorem{expl}[thm]{Example}
\newtheorem{notn}[thm]{Notation}
%\theoremstyle{nonumberplain}
%\theoremsymbol{\Box}


\theoremstyle{definition}
\newtheorem{exmpl}{Example}
\newtheorem*{nono-theorem}{Theorem}


\renewcommand{\greyout}[1]{} %{{\color{gray} {#1}}} -- toggle to remove greyed text

\newcommand{\emptyfun}{{[]}}
\newcommand{\cal}{\mathcal}
%\newcommand{\fold}{\mathit{fold}}
\newcommand{\F}{\mathcal{F}}
\renewcommand{\G}{\mathcal{G}}
\newcommand{\N}{\mathcal{N}}
\newcommand{\E}{\mathcal{E}}
\newcommand{\B}{\mathcal{B}}
\renewcommand{\P}{\mathcal{A}}
\newcommand{\pred}{\mathsf{Fam}}
\newcommand{\env}{\mathsf{Env}}
\newcommand{\set}{\mathsf{Set}}
\renewcommand{\S}{\mathcal S}
\renewcommand{\C}{\mathcal{C}}
\newcommand{\D}{\mathcal{D}}
\newcommand{\A}{\mathcal{A}}
\renewcommand{\id}{\mathit{id}}
\newcommand{\map}{\mathsf{map}}
\newcommand{\pid}{\underline{\mathit{id}}}
\newcommand{\pcirc}{\,\underline{\circ}\,}
\newcommand{\pzero}{\underline{0}}
\newcommand{\pone}{\underline{1}}
\newcommand{\psum}{\,\underline{+}\,}
\newcommand{\pinl}{\underline{\mathit{inL}}\,}
\newcommand{\pinr}{\underline{\mathit{inR}}\,}
\newcommand{\ptimes}{\,\underline{\times}\,}
\newcommand{\ppi}{\underline{\pi_1}}
\newcommand{\pppi}{\underline{\pi_2}}
\newcommand{\pmu}{\underline{\mu}}
\newcommand{\semmap}{\mathit{map}}
\newcommand{\subst}{\mathit{subst}}

\newcommand{\tb}[1]{~~ \mbox{#1} ~~}
\newcommand{\listt}[1]{(\mu \phi. \lambda \beta . \onet + \beta \times
  \phi \beta) #1} 
\newcommand{\filtype}{\Nat^\emptyset 
 (\Nat^\emptyset \, \alpha \, \mathit{Bool})\, (\Nat^\emptyset 
  (List \, \alpha) \, (List \, \alpha))} 
\newcommand{\filtypeGRose}{\Nat^\emptyset 
 (\Nat^\emptyset \, \alpha \, \mathit{Bool})\, (\Nat^\emptyset 
  (\mathit{GRose}\,\psi \, \alpha) \, (\mathit{GRose}\,\psi \, (\alpha
  + \onet)))} 
\newcommand{\maplist}{\mathit{map}_{\lambda A. \setsem{\emptyset; \alpha
      \vdash \mathit{List} \, \alpha} \rho[\alpha := A]}} 
\newcommand{\PLeaves}{\mathsf{PLeaves}}
\newcommand{\swap}{\mathsf{swap}}
\newcommand{\reverse}{\mathsf{reverse}}
\newcommand{\Bcons}{\mathit{Bcons}}
\newcommand{\Bnil}{\mathit{Bnil}}

\title[Supplementary Material]{Supplementary Material} %% [Short Title] is optional; when present,
                         %% will be used in header instead of Full
                         %% Title.
%\titlenote{with title note}             %% \titlenote is optional;
                                        %% can be repeated if necessary;
                                        %% contents suppressed with 'anonymous'
%\subtitle{Subtitle}                     %% \subtitle is optional
%\subtitlenote{with subtitle note}       %% \subtitlenote is optional;
                                        %% can be repeated if necessary;
                                        %% contents suppressed with 'anonymous'


%% Author information
%% Contents and number of authors suppressed with 'anonymous'.
%% Each author should be introduced by \author, followed by
%% \authornote (optional), \orcid (optional), \affiliation, and
%% \email.
%% An author may have multiple affiliations and/or emails; repeat the
%% appropriate command.
%% Many elements are not rendered, but should be provided for metadata
%% extraction tools.

%% Author with single affiliation.
\author{Patricia Johann, Enrico Ghiorzi, and Daniel Jeffries}
%\authornote{with author1 note}          %% \authornote is optional;
%                                        %% can be repeated if necessary
%\orcid{nnnn-nnnn-nnnn-nnnn}             %% \orcid is optional
\affiliation{
%  \position{Position1}
%  \department{Department1}              %% \department is recommended
  \institution{Appalachian State University}            %% \institution is required
%  \streetaddress{Street1 Address1}
%  \city{City1}
%  \state{State1}
%  \postcode{Post-Code1}
%  \country{Country1}                    %% \country is recommended
}
\email{johannp@appstate.edu, ghiorzie@appstate.edu, jeffriesd@appstate.edu}          %% \email is recommended


\begin{document}

\maketitle

\begin{nono-theorem}[Identity Extension Lemma]\label{thm:iel}
If $\rho$ is any $\oCPO$ environment, and $\Gamma; \Phi \vdash F$, then
\[\oCPORsem{\Gamma; \Phi \vdash F} \Eq_\rho = \Eq_{\oCPOsem{\Gamma; \Phi
    \vdash F}\rho}\]
\end{nono-theorem}
\begin{proof}
By induction on $F$.
\begin{itemize}
%\item $\relsem{\Gamma; \emptyset \vdash v}\Eq_{\rho} = \Eq_{\rho} v =
%  \Eq_{\rho v} = \Eq_{\setsem{\Gamma; \emptyset \vdash v}\rho}$ where
%  $v \in \Gamma$.
\item $\oCPORsem{\Gamma; \Phi \vdash \zerot} \Eq_{\rho} = 0_\oCPOR =
  \Eq_{0_\oCPO} = \Eq_{\oCPOsem{\Gamma; \Phi \vdash \zerot}\rho}$
\item $\oCPORsem{\Gamma; \Phi \vdash \onet} \Eq_{\rho} = 1_\oCPOR =
  \Eq_{1_\oCPO} = \Eq_{\oCPOsem{\Gamma; \Phi \vdash \onet}\rho}$
\item
By definition, $\oCPORsem{\Gamma; \emptyset \vdash \Nat^\Phi \,F\,G}
\Eq_{\rho}$ is the relation on $\oCPOsem{\Gamma; \emptyset \vdash
  \Nat^\Phi \,F\,G} \rho$ relating $\eta$ and $\eta'$ if, for all $K_1
: \oCPORT_{k_1},...,K_n : \oCPORT_{k_n}$, $(\eta_{\overline{K^1}},
\eta'_{\overline{K^2}})$ is a morphism from $\oCPORsem{\Gamma; \Phi
  \vdash F} \Eq_{\rho}\overline{[\Phi := K]}$ to $\oCPORsem{\Gamma ;
  \Phi \vdash G} \Eq_{\rho}\overline{[\Phi := K]}$ in $\oCPOR$.  To
prove that this is equal to $\Eq_{\oCPOsem{\Gamma; \emptyset \vdash
    \Nat^\Phi F\,G} \rho}$ we need to show that
$(\eta_{\overline{K^1}}, \eta'_{\overline{K^2}})$ is a morphism from
$\oCPORsem{\Gamma; \Phi \vdash F} \Eq_{\rho}\overline{[\Phi := K]}$ to
$\oCPORsem{\Gamma ; \Phi \vdash G} \Eq_{\rho}\overline{[\Phi := K]}$
in $\oCPOR$ for all $K_1 : \oCPORT_{k_1},...,K_n : \oCPORT_{k_n}$ if
and only if $\eta = \eta'$ and $(\eta_{\overline{K^1}},
\eta'_{\overline{K^2}})$ is a morphism from $\oCPORsem{\Gamma; \Phi
  \vdash F} \Eq_{\rho}\overline{[\Phi := K]}$ to $\oCPORsem{\Gamma ;
  \Phi \vdash G} \Eq_{\rho}\overline{[\Phi := K]}$ in $\oCPOR$ for all
$K_1 : \oCPORT_{k_1},...,$ $ K_n : \oCPORT_{k_n}$.  The only
interesting part of this equivalence is to show that if
$(\eta_{\overline{K^1}}, \eta'_{\overline{K^2}})$ is a morphism from
$\oCPORsem{\Gamma; \Phi \vdash F} \Eq_{\rho}\overline{[\Phi := K]}$ to
$\oCPORsem{\Gamma ; \oCPO \vdash G} \Eq_{\rho}\overline{[\Phi := K]}$
in $\oCPOR$ for all $K_1 : \oCPORT_{k_1},...,K_n : \oCPORT_{k_n}$,
then $\eta = \eta'$. By hypothesis, $(\eta_{\overline{T}},
\eta'_{\overline{T}})$ is a morphism from $\oCPORsem{\Gamma; \Phi
  \vdash F} \Eq_{\rho}\overline{[\Phi := \Eq_{T}]}$ to
$\oCPORsem{\Gamma ; \Phi \vdash G} \Eq_{\rho}\overline{[\Phi :=
    \Eq_{T}]}$ in $\oCPOR$ for all $\ol{T :
  [\oCPO^k,\oCPO]_{\omega}}$.  By the induction hypothesis, it is
therefore a morphism from $\Eq_{\oCPOsem{\Gamma; \Phi \vdash F}
  \rho\overline{[\Phi := T]}}$ to $\Eq_{\oCPOsem{\Gamma; \Phi \vdash
    G} \rho\overline{[\Phi := T]}}$ in $\oCPOR$. This means that, for
every $x : \Eq_{\oCPOsem{\Gamma; \Phi \vdash F} \rho\overline{[\Phi :=
      T]}}$, $\eta_{\overline{T}}x = \eta'_{\overline{T}}x$.  Then, by
extensionality, $\eta = \eta'$.

\item The application case is proved by the following sequence of
  equalities, where the second equality is by the induction hypothesis
  and the definition of the $\oCPO$ relation environment $\Eq_\rho$,
  the third is by the definition of application of $\oCPO$ relation
  transformers from Definition~11, and the fourth is by
  Equation~10:
\[
\begin{split}
\oCPORsem{\Gamma; \Phi \vdash \phi\ol{F}}\Eq_{\rho} &=
(\Eq_{\rho}\phi)\ol{\oCPORsem{\Gamma; \Phi \vdash F}
\Eq_{\rho}}\\
&= \Eq_{\rho \phi}\, \ol{\Eq_{\oCPOsem{\Gamma; \Phi \vdash F}
  \rho}}\\
&= (\Eq_{\rho \phi})^* \,\ol{\Eq_{\oCPOsem{\Gamma; \Phi \vdash F}
  \rho}}\\
&= \Eq_{(\rho \phi) \,\ol{\oCPOsem{\Gamma; \Phi \vdash F} \rho}}\\
&= \Eq_{\oCPOsem{\Gamma; \Phi \vdash \phi\ol{F}}\rho}
\end{split}
\]
\item 
  The fixpoint case is proven by the sequence of equalities
\[
\begin{split}
\oCPORsem{\Gamma; \Phi, \ol\gamma \vdash (\mu \phi.\lambda
  \ol{\alpha}. H)\ol{F}}\Eq_{\rho} 
&=(\mu {T_{H, \Eq_{\rho}}}) \,\ol{\oCPORsem{\Gamma; \Phi, \ol\gamma
    \vdash F}\Eq_{\rho}}\\   
&= \colim{i < \omega}{T^i_{H, \Eq_{\rho}} K_0}\, \ol{\oCPORsem{\Gamma;
    \Phi, \ol\gamma \vdash F}\Eq_{\rho}}\\
&= \colim{i < \omega}{ T^i_{H, \Eq_{\rho}} K_0 \,\ol{\Eq_{\oCPOsem{\Gamma;
    \Phi, \ol\gamma  \vdash F}\rho}}}\\
&= \colim{i < \omega}{(\Eq_{(T^\oCPO_{H,\rho})^i K_0})^*
  \ol{\Eq_{\oCPOsem{\Gamma; \Phi, \ol\gamma  \vdash F}\rho}}}\\
&= \colim{i < \omega}{\Eq_{(T^\oCPO_{H,\rho})^i K_0 \,\ol{\oCPOsem{\Gamma;
        \Phi, \ol\gamma  \vdash F}\rho}}}\\ 
&= \Eq_{\colim{i < \omega}{ (T^\oCPO_{H,\rho})^i K_0\,
    \ol{\oCPOsem{\Gamma; \Phi, \ol\gamma  \vdash F}\rho}}}\\
&= \Eq_{\oCPOsem{\Gamma; \Phi, \ol\gamma  \vdash (\mu \phi.\lambda
      \ol{\alpha}. H)\ol{F}}\rho}
\end{split}
\]
Here, the third equality is by induction hypothesis, the fifth is by
Equation~10 and the fourth equality is because, for every $i < \omega$,
the following two statements can be proved by simultaneous induction:
and for any $H$, $\rho$, $\ol A$, and subformula $J$ of $H$,
\begin{equation}\label{eq:iel-fix-point-intermediate1}
T^i_{H, \Eq_{\rho}} K_0\, \ol{\Eq_A} = (\Eq_{(T^\oCPO_{H, \rho})^i K_0})^*
\ol{\Eq_A}
\end{equation}
and 
\begin{equation}\label{eq:iel-fix-point-intermediate2}
\begin{split}
  \oCPORsem{\Gamma; \Phi', \phi, \ol{\alpha} \vdash J}
\Eq_{\rho} [\phi := 
 & T^i_{H,\Eq_{\rho}} K_0] \overline{[\alpha :=
    \Eq_A]} \\
=\;\; & \oCPORsem{\Gamma; \Phi', \phi, \ol{\alpha} \vdash J} \Eq_{\rho} [\phi
  := \Eq_{(T^\oCPO_{H,\rho})^i K_0}] \overline{[\alpha :=
    \Eq_A]}
\end{split}
\end{equation}
(Notice that we don't know what's in the context $\Phi'$.)  We
prove~\eqref{eq:iel-fix-point-intermediate1} by induction on $i$.  The
case $i=0$ is trivial, because $T^0_{H,\Eq_{\rho}} K_0 = K_0$ and
$(T^\oCPO_{H,\rho})^0 K_0 = K_0$. The inductive step
%for successor ordinals
is proved by the following sequence of equalities:
\[
\begin{split}
T^{i+1}_{H,\Eq_{\rho}} K_0\, \overline{\Eq_A}
&= T^\oCPOR_{H,\Eq_{\rho}} (T^i_{H,\Eq_{\rho}} K_0)
\overline{\Eq_A} \\ 
&= \oCPORsem{\Gamma; \phi, \ol{\alpha}, \ol\gamma \vdash H} \Eq_{\rho} [\phi
  := T^i_{H, \Eq_{\rho}} K_0] \overline{[\alpha :=
    \Eq_A]} \\ 
&= \oCPORsem{\Gamma; \phi, \ol{\alpha}, \ol\gamma \vdash H} \Eq_{\rho} [\phi
  := \Eq_{(T^\oCPO_{H,\rho})^i K_0}] \overline{[\alpha :=
    \Eq_A]} \\ 
&= \oCPORsem{\Gamma; \phi, \ol{\alpha}, \ol\gamma \vdash H} \Eq_{\rho [\phi
    := (T^\oCPO_{H,\rho})^i K_0] \overline{[\alpha :=
      A]}} \\ 
&= \Eq_{\oCPOsem{\Gamma; \phi, \ol{\alpha}, \ol\gamma \vdash H} \rho [\phi
    := (T^\oCPO_{H,\rho})^i K_0] \overline{[\alpha :=
      A]}} \\ 
&= \Eq_{(T^\oCPO_{H,\rho})^{i+1} K_0 \overline{A}} \\ 
&= (\Eq_{(T^\oCPO_{H,\rho})^{i+1} K_0})^*\, \overline{\Eq_A} 
\end{split}
\]
Here, the third equality is by \eqref{eq:iel-fix-point-intermediate2}
for $J = H$, the fifth by the induction hypothesis of the IEL on $H$,
and the last is by Equation~10.
%{\color{blue} Finally, the inductive
%  step for limit ordinals is proved by the following sequence of
%  equalities:
%\[
%\begin{split}
%T^{\bigcup_{j<i}j}_{H,\Eq_\rho} K_0\, \overline{\Eq_A}
%&= (\colim{j < i}{T^j_{H,\Eq_\rho}}) K_0\, \overline{\Eq_A}\\
%&= (\colim{j < i}{(T^j_{H,\Eq_\rho} K_0)})\, \overline{\Eq_A}\\
%&= \colim{j < i}{T^j_{H,\Eq_\rho} K_0\, \overline{\Eq_A}}\\
%&= \colim{j < i}{(\Eq_{(T^\oCPO_{H,\rho})^j K_0})^*\,
%  \overline{\Eq_A}}\\
%&= (\colim{j < i}{(\Eq_{(T^\oCPO_{H,\rho})^j K_0})^*)\,
%  \overline{\Eq_A}}\\
%&= (\Eq_{\colim{j < i}{(T^\oCPO_{H,\rho})^j K_0}})^*\,
%\overline{\Eq_A}\\
%&= (\Eq_{(\colim{j < i}{(T^\oCPO_{H,\rho})^j}) K_0})^*\,
%\overline{\Eq_A}\\
%&= (\Eq_{(T^\oCPO_{H,\rho})^{\bigcup_{j < i}j} K_0})^*\, \overline{\Eq_A}\\
%\end{split}
%\]
%Here, the second, third, fifth, and seventh equalities are by Lemma~2
%of~{\color{red} [Johann and Polonsky 2019]}, the fourth is by the
%induction hypothesis, and the sixth is by cocontinuity of $\Eq$.}

\vspace*{0.1in}

We prove~\eqref{eq:iel-fix-point-intermediate2} by structural
induction on $J$. The only interesting cases, though, are when $J =
\phi \ol{G}$, when $J = (\mu \psi.\lambda \ol\beta. G)\ol K$, and when
$J = (\Lan_{\ol K}^{\ol \beta} F)\ol{A}$.
\begin{itemize}
\item The case $J = \phi \ol G$ is proved by the sequence of equalities:
\[
\begin{split}
& \oCPORsem{\Gamma; \Phi', \phi, \ol{\alpha} \vdash \phi
    \ol{G}} \Eq_{\rho} [\phi := T^i_{H,\Eq_{\rho}} K_0]
  \overline{[\alpha := \Eq_A]}
  \\
&= T^i_{H,\Eq_{\rho}} K_0\, \overline{\oCPORsem{\Gamma; \Phi',
      \phi, \ol{\alpha} \vdash G} \Eq_{\rho} [\phi :=
      T^i_{H,\Eq_{\rho}} K_0] \overline{[\alpha :=
        \Eq_A]}} \\ 
&= T^i_{H,\Eq_{\rho}} K_0\, \overline{\oCPORsem{\Gamma; \Phi',
      \phi, \ol{\alpha} \vdash G} \Eq_{\rho} [\phi :=
      \Eq_{(T^\oCPO_{H,\rho})^i K_0}] \overline{[\alpha :=
        \Eq_A]}} \\ 
&= T^i_{H,\Eq_{\rho}} K_0\, \overline{\oCPORsem{\Gamma; \Phi',
      \phi, \ol{\alpha} \vdash G} \Eq_{\rho [\phi := (T^\oCPO_{H,\rho})^i
        K_0] \overline{[\alpha := A]}}} \\ 
&= T^i_{H,\Eq_{\rho}} K_0\, \overline{\Eq_{\oCPOsem{\Gamma;
        \Phi', \phi, \ol{\alpha} \vdash G} \rho [\phi :=
        (T^\oCPO_{H,\rho})^i K_0] \overline{[\alpha :=
          A]}}} \\ 
&= (\Eq_{(T^\oCPO_{H,\rho})^i K_0})^* \,\overline{\Eq_{\oCPOsem{\Gamma;
        \Phi', \phi, \ol{\alpha} \vdash G} \rho [\phi :=
        (T^\oCPO_{H,\rho})^i K_0] \overline{[\alpha :=
          A]}}} \\ 
&= (\Eq_{(T^\oCPO_{H,\rho})^i K_0})^* \overline{\oCPORsem{\Gamma;
      \Phi', \phi, \ol{\alpha} \vdash G} \Eq_{\rho} [\phi :=
      \Eq_{(T^\oCPO_{H,\rho})^i K_0}] \overline{[\alpha :=
        \Eq_A]}} \\ 
&= \oCPORsem{\Gamma; \Phi', \phi, \ol{\alpha} \vdash \phi \ol{G}}
  \Eq_{\rho} [\phi := \Eq_{(T^\oCPO_{H,\rho})^i K_0}]
  \overline{[\alpha := \Eq_A]} 
\end{split}
\]
Here, the second equality is by the induction hypothesis for
~\eqref{eq:iel-fix-point-intermediate2} on the $G$s, the fourth is by
the induction hypothesis for the IEL on the $G$s, and the fifth is by
the induction hypothesis on $i$
for~\eqref{eq:iel-fix-point-intermediate1}.

\item 
The case $J = (\mu \psi.\, \lambda \ol{\beta}.\, G) \ol{K}$
is proved by the sequence of equalities (where $\Phi' = \Phi'', \ol\gamma$):
\[
\begin{split}
& \oCPORsem{\Gamma; \Phi'', \ol\gamma, \phi, \ol{\alpha} \vdash (\mu \psi.\, \lambda \ol{\beta}.\, G) \ol{K}}
  \Eq_{\rho} [\phi := T^i_{H,\Eq_{\rho}} K_0]
  [\overline{\alpha := \Eq_A}] \\
&= (\mu {T_{G, \Eq_{\rho}[\phi := T^i_{H,\Eq_{\rho}} K_0][\overline{\alpha := \Eq_A}]}})
  \,\ol{\oCPORsem{\Gamma; \Phi'', \ol\gamma, \phi, \ol{\alpha} \vdash K}\Eq_{\rho}
  [\phi := T^i_{H,\Eq_{\rho}} K_0][\overline{\alpha := \Eq_A}]} \\ 
&= \colim{j < \omega} {T^j_{G, \Eq_{\rho}[\phi := T^{i}_{H,\Eq_{\rho}} K_0][\overline{\alpha := \Eq_A}]}}\,K_0
  \,(\ol{\oCPORsem{\Gamma; \Phi'', \ol\gamma, \phi, \ol{\alpha} \vdash K}\Eq_{\rho}
  [\phi := T^{i}_{H,\Eq_{\rho}} K_0][\overline{\alpha := \Eq_A}]}) \\ 
&= \colim{j < \omega} {T^j_{G, \Eq_{\rho}[\phi := T^{i}_{H,\Eq_{\rho}} K_0][\overline{\alpha := \Eq_A}]}}\,K_0
  \,(\ol{\oCPORsem{\Gamma; \Phi'', \ol\gamma, \phi, \ol{\alpha} \vdash K}\Eq_{\rho}
  [\phi := \Eq_{(T^{\oCPO}_{H,\rho})^i K_0}][\overline{\alpha := \Eq_A}]}) \\  %%
&= \colim{j < \omega} {T^j_{G, \Eq_{\rho}[\phi := \Eq_{(T^{\oCPO}_{H,\rho})^i K_0}][\overline{\alpha := \Eq_A}]}}\,K_0
  \,(\ol{\oCPORsem{\Gamma; \Phi'', \ol\gamma, \phi, \ol{\alpha} \vdash K}\Eq_{\rho}
  [\phi := \Eq_{(T^{\oCPO}_{H,\rho})^i K_0}][\overline{\alpha := \Eq_A}]}) \\ 
&= (\mu {T_{G, \Eq_{\rho}[\phi := \Eq_{(T^{\oCPO}_{H,\rho})^i K_0}][\overline{\alpha := \Eq_A}]}})
  \,\ol{\oCPORsem{\Gamma; \Phi'', \ol\gamma, \phi, \ol{\alpha} \vdash K}\Eq_{\rho}
  [\phi := \Eq_{(T^{\oCPO}_{H,\rho})^i K_0}][\overline{\alpha := \Eq_A}]} \\ 
&= \oCPORsem{\Gamma; \Phi'', \ol\gamma, \phi, \ol{\alpha} \vdash (\mu \psi.\, \lambda \beta.\, G) \ol{K}}
  \Eq_{\rho}[\phi := \Eq_{(T^{\oCPO}_{H,\rho})^i K_0}][\overline{\alpha := \Eq_{A}}]
\end{split}
\]
Here, the third equality is by the induction hypothesis
for~\eqref{eq:iel-fix-point-intermediate2} on the $K$s, and the fourth
equality holds because we can prove that, for all $j < \omega$,
\begin{equation}\label{eq:helper}
T^j_{G,
\Eq_{\rho}[\phi := T^i_{H,\Eq_{\rho}} K_0][\overline{\alpha :=
    \Eq_A}]}\,K_0 \ = T^j_{G, \Eq_{\rho}[\phi :=
    \Eq_{(T^{\oCPO}_{H,\rho})^i K_0}][\overline{\alpha :=
      \Eq_A}]}\,K_0
\end{equation}
Indeed, the base case of~\eqref{eq:helper} is trivial because
\[T^0_{G,
  \Eq_{\rho}[\phi := T^{i}_{H,\Eq_{\rho}} K_0][\overline{\alpha :=
      \Eq_A}]}\,K_0 \ = K_0 = T^0_{G, \Eq_{\rho}[\phi :=
    \Eq_{(T^{\oCPO}_{H,\rho})^i K_0}][\overline{\alpha :=
      \Eq_A}]}\,K_0\] The inductive case of \eqref{eq:helper}
%for successor ordinals
is proved by: \\
\begin{align*}
&\;\;\;\; T^{j+1}_{G, \Eq_{\rho}[\phi := T^{i}_{H,\Eq_{\rho}} K_0][\overline{\alpha := \Eq_A}]}\,K_0 \\
&= T_{G, \Eq_{\rho}[\phi := T^{i}_{H,\Eq_{\rho}} K_0][\overline{\alpha
        := \Eq_A}]} (T^{j}_{G, \Eq_{\rho}[\phi := T^{i}_{H,\Eq_{\rho}}
      K_0][\overline{\alpha := \Eq_A}]}\,K_0) \\
&= T_{G, \Eq_{\rho}[\phi := T^{i}_{H,\Eq_{\rho}} K_0][\overline{\alpha
        := \Eq_A}]} (T^{j}_{G, \Eq_{\rho}[\phi :=
      \Eq_{(T^{\oCPO}_{H,\rho})^i K_0}][\overline{\alpha :=
        \Eq_A}]}\,K_0) \\
&= \lambda \ol{R}. \oCPORsem{\Gamma; \ol\gamma, \phi, \ol\alpha, \psi,
    \ol\beta \vdash G} \Eq_{\rho} [\phi := T^{i}_{H,\Eq_{\rho}}
    K_0][\overline{\alpha := \Eq_A}][\psi := T^{j}_{G, \Eq_{\rho}[\phi
        := \Eq_{(T^{\oCPO}_{H,\rho})^i K_0}][\overline{\alpha :=
          \Eq_A}]}\,K_0][\overline{\beta := R}] \\
&= \lambda \ol{R}. \oCPORsem{\Gamma; \ol\gamma, \phi, \ol\alpha, \psi,
    \ol\beta \vdash G} \Eq_{\rho} [\phi := \Eq_{(T^{\oCPO}_{H,\rho})^i
      K_0}][\overline{\alpha := \Eq_A}][\psi := T^{j}_{G,
      \Eq_{\rho}[\phi := \Eq_{(T^{\oCPO}_{H,\rho})^n
          K_0}][\overline{\alpha := \Eq_A}]}\,K_0][\overline{\beta :=
      R}] \\
&= T_{G, \Eq_{\rho}[\phi := \Eq_{(T^{\oCPO}_{H,\rho})^i
        K_0}][\overline{\alpha := \Eq_A}]} (T^{j}_{G, \Eq_{\rho}[\phi
      := \Eq_{(T^{\oCPO}_{H,\rho})^i K_0}][\overline{\alpha :=
        \Eq_A}]}\,K_0) \\
&= T^{j+1}_{G, \Eq_{\rho}[\phi := \Eq_{(T^{\oCPO}_{H,\rho})^i
        K_0}][\overline{\alpha := \Eq_A}]}\,K_0
\end{align*}
Here, the second equality holds by the induction hypothesis
for~\eqref{eq:helper} on $j$. The fourth equality holds because,
due to typing rule restrictions for the $\mu$ types,
$\phi$ either does not appear in $G$, or must have arity $0$, in which case
$\ol\alpha$ must be empty, if $\phi$ appears in $G$, and
uses~\eqref{eq:iel-fix-point-intermediate2} for $G$ when $\phi$ has
arity $0$.

%{\color{blue} The inductive case of \eqref{eq:helper} for limit
%  ordinals is proved by:
%
%\begin{align*}
%&\;\;\;\; T^{\bigcup_{j<i}j}_{G, \Eq_{\rho}[\phi :=
%      T^{i}_{H,\Eq_{\rho}} K_0][\overline{\alpha := \Eq_A}]}\,K_0\\
%&= (\colim{j < i}{T^{j}_{G, \Eq_{\rho}[\phi :=
%      T^{i}_{H,\Eq_{\rho}} K_0][\overline{\alpha := \Eq_A}]}})\,K_0\\
%&= \colim{j < i}{(T^{j}_{G, \Eq_{\rho}[\phi :=
%      T^{i}_{H,\Eq_{\rho}} K_0][\overline{\alpha := \Eq_A}]}\,K_0)}\\
%&= \colim{j<i}{(T^{j}_{G, \Eq_{\rho}[\phi := \Eq_{(T^{\oCPO}_{H,\rho})^i
%          K_0}][\overline{\alpha := \Eq_A}]}\,K_0)}\\
%&= (\colim{j<i}{T^{j}_{G, \Eq_{\rho}[\phi := \Eq_{(T^{\oCPO}_{H,\rho})^i
%          K_0}][\overline{\alpha := \Eq_A}]}})\,K_0\\
%&= T^{\bigcup_{j<i}j}_{G, \Eq_{\rho}[\phi := \Eq_{(T^{\oCPO}_{H,\rho})^i
%        K_0}][\overline{\alpha := \Eq_A}]}\,K_0
%\end{align*}
%} Here, the second and fourth equalities hold by Lemma~2
%of~{\color{red} [Johann and Polonsky 2019]}, and the third holds by
%the induction hypothesis.

\item {\color{blue} Finally, the case $J = (\Lan_{\ol K}^{\ol \beta}
  F)\ol{B}$ for (2) is proved by first observing that, for $\pi \in
  \{\pi_1,\pi_2\}$ and all $i < \omega$, induction on $i$ gives that
\begin{equation}\label{eq:helper-lan}
\pi\, (T^i_{H,\Eq_{\rho}} K_0) = (T^\oCPO_{H,\rho})^i K_0 
\end{equation}
Indeed, the inductive case is proved by:
\begin{align*}
\pi\, (T^{i+1}_{H,\Eq_{\rho}} K_0) &= \pi\,(\lambda R. \oCPORsem{\Gamma;\ol
  \gamma, \ol \alpha, \phi \vdash H}\Eq_\rho[\phi :=
  T^{i}_{H,\Eq_{\rho}} K_0][\ol{\alpha := R}])\\
&= \lambda A. \oCPOsem{\Gamma;\ol
  \gamma, \ol \alpha, \phi \vdash H}\rho[\phi :=
   (T^{\oCPO}_{H,\rho})^i K_0][\ol{\alpha := A}]\\
&= (T^\oCPO_{H,\rho})^i K_0 
\end{align*}
where the second is by surjectivity of projection and the induction
hypothesis.
%For the limit ordinal $i = \bigcup_{j<i}j$ we have
%\begin{align*}
%& \hspace*{0.18in} \pi\, (T^{\bigcup_{j<i}j}_{H,\Eq_{\rho}} K_0)\\
%&= \pi\, ((\colim{j < i}{T^j_{H,\Eq_{\rho}}})K_0)\\
%&= \pi\, (\colim{j < i}{(T^j_{H,\Eq_{\rho}} K_0)})\\
%&= \colim{j < i}{\pi\,(T^j_{H,\Eq_{\rho}} K_0)}\\
%&= \colim{j < i}{(T^\oCPO_{H,\rho})^j K_0 }\\
%&= (T^\oCPO_{H,\rho})^{\bigcup_{j<i}j} K_0\\
%\end{align*}
%Here, the third equality is by cocontinuity of projection and the
%fourth is by the induction hypothesis. 
%%\begin{equation}\label{eq:helper-lan2}
%%T^i_{H,\Eq_{\rho}} K_0 = \Eq_{(T^\oCPO_{H,\rho})^i K_0} \hspace{1in} \mbox{TOO FAST!!}
%%\end{equation}
We can now compute 
\begin{align*}
&\hspace*{0.2in}\oCPORsem{\Gamma;\Phi',\phi,\ol\alpha
    \vdash (\Lan_{\ol K}^{\ol \beta} F)\ol{B}}  \Eq_{\rho} [\phi :=
    T^i_{H,\Eq_{\rho}} K_0] [\overline{\alpha := \Eq_A}]\\
&= \{(t_1,t_2) \in \oCPOsem{\Gamma;\Phi',\phi,\ol\alpha
    \vdash (\Lan_{\ol K}^{\ol \beta} F)\ol{B}} \rho [\phi := \pi_1\,
    (T^i_{H,\Eq_{\rho}} K_0)] [\overline{\alpha := A}]\\
& \hspace*{0.57in}\times\;
  \oCPOsem{\Gamma;\Phi',\phi,\ol\alpha \vdash (\Lan_{\ol
      K}^{\ol \beta} F)\ol{B}} \rho [\phi := \pi_2\,
    (T^i_{H,\Eq_{\rho}} K_0)] [\overline{\alpha := A}]\\
  & \hspace*{0.63in}|~\exists\, \ol{Z : \oCPO_0}, t_1',\, t_2',\,
  f : \oCPOsem{\Gamma;\ol\beta \vdash K} \Eq_\rho
  [\phi := T^i_{H,\Eq_{\rho}} K_0] [\overline{\alpha := \Eq_A}]
  [\overline{\beta := \Eq_Z}]\\
  & \hspace*{1.75in}\to \oCPOsem{\Gamma;\Phi',
    \phi,\ol\alpha \vdash B} \Eq_\rho [\phi := T^i_{H,\Eq_{\rho}} K_0]
  [\overline{\alpha := \Eq_A}]\\
  &\hspace{0.9in}\mbox{such that } \iota_{Z,\pi_1 f}t_1' = t_1,\,
  \kappa_{Z,\pi_2 f}t_2' = t_2, \mbox{ and }\\
  &\hspace*{0.9in}(t_1',t_2') : \oCPOsem{\Gamma;\Phi',
    \phi,\ol\alpha,\ol\beta \vdash F} \Eq_\rho [\phi := T^i_{H,\Eq_{\rho}} K_0]
  [\overline{\alpha := \Eq_A}][\ol{\beta :=\Eq_Z}]\,\}\\
&= \{(t_1,t_2) \in (\oCPOsem{\Gamma;\Phi',\phi,\ol\alpha
    \vdash (\Lan_{\ol K}^{\ol \beta} F)\ol{B}} \rho [\phi := 
    (T^\set_{H,{\rho}})^i K_0] [\overline{\alpha := A}])^2\\
  & \hspace*{0.63in}|~\exists\, \ol{Z : \oCPO_0}, t_1',\, t_2',\,
  f : \oCPOsem{\Gamma;\ol\beta \vdash K} \Eq_\rho[\overline{\beta :=
      \Eq_Z}]\to \oCPOsem{\Gamma;\Phi',
    \phi,\ol\alpha \vdash B} \Eq_{\rho[\phi := (T^\set_{H,{\rho}})^i K_0]
  [\overline{\alpha := A}]}\\
  &\hspace{0.9in}\mbox{such that } \iota_{Z,\pi_1 f}t_1' = t_1,\,
  \kappa_{Z,\pi_2 f}t_2' = t_2, \mbox{ and }\\
  &\hspace*{0.9in}(t_1',t_2') : \oCPOsem{\Gamma;\Phi',
    \phi,\ol\alpha,\ol\beta \vdash F} \Eq_{\rho[\phi :=
      (T^\set_{H,{\rho}})^i K_0][\overline{\alpha := A}][\ol{\beta :=Z}]}\,\}\\
 &=\oCPORsem{\Gamma;\Phi',\phi,\ol\alpha
    \vdash (\Lan_{\ol K}^{\ol \beta} F)\ol{B}}  \Eq_{\rho[\phi :=
   (T^\oCPO_{H,\rho})^i K_0] [\overline{\alpha := A}]}\\
  &=\oCPORsem{\Gamma;\Phi',\phi,\ol\alpha
    \vdash (\Lan_{\ol K}^{\ol \beta} F)\ol{B}}  \Eq_{\rho} [\phi :=
   \Eq_{(T^\oCPO_{H,\rho})^i
      K_0}] [\overline{\alpha := \Eq_A}]
\end{align*}
The second equality uses~\eqref{eq:helper-lan} twice to rewrite the
types of $t_1$ and $t_2$, together with the induction hypothesis
for~\eqref{eq:iel-fix-point-intermediate2} twice --- once for $B$ and
$\rho$, and once for $F$ and $\rho' = \rho[\ol{\beta := Z}]$ after we
observe that $T^i_{H,\Eq_\rho} = T^i_{H,\Eq_{\rho'}} $ since, without
loss of generality, the variables in $\ol\beta$ do not appear free in
$H$.}
\end{itemize}
\item $\oCPORsem{\Gamma; \Phi \vdash F + G} \Eq_{\rho} =
  \oCPORsem{\Gamma; \Phi \vdash F} \Eq_{\rho} + \oCPORsem{\Gamma;
    \Phi \vdash G} \Eq_{\rho} = \Eq_{\oCPOsem{\Gamma; \Phi \vdash
      F}\rho} + \Eq_{\oCPOsem{\Gamma; \Phi \vdash G}\rho} =
  \Eq_{\oCPOsem{\Gamma; \Phi \vdash F}\rho + \oCPOsem{\Gamma; \Phi
      \vdash G}\rho} = \Eq_{\oCPOsem{\Gamma; \Phi \vdash F +
      G}\rho}$
\item $\oCPORsem{\Gamma; \Phi \vdash F \times G} \Eq_{\rho} =
  \oCPORsem{\Gamma; \Phi \vdash F}\Eq_{\rho} \times \oCPORsem{\Gamma;
    \Phi \vdash G}\Eq_{\rho} = \Eq_{\oCPOsem{\Gamma; \Phi \vdash
      F}\rho} \times \Eq_{\oCPOsem{\Gamma; \Phi \vdash G}\rho}
  = \Eq_{\oCPOsem{\Gamma; \Phi \vdash F}\rho \times
    \oCPOsem{\Gamma; \Phi \vdash G}\rho} = \Eq_{\oCPOsem{\Gamma;
      \Phi \vdash F \times G}\rho}$
\item The case of the IEL for $(\Lan_{\ol K}^{\ol \beta} F)\ol{A}$ is proved
  by first observing that \[\Eq_{\oCPOsem{\Gamma; \Phi \vdash
      (\Lan_{\ol K}^{\ol \beta} F)\ol{A}}\rho} \subseteq
  \oCPORsem{\Gamma; \Phi \vdash (\Lan_{\ol K}^{\ol \beta}
    F)\ol{A}}\Eq_\rho\] holds because
  \[\begin{array}{ll}
\Eq_{\oCPOsem{\Gamma; \Phi \vdash (\Lan_{\ol K}^{\ol \beta}
    F)\ol{A}}\rho} &= \{(t,t)~|~t \in \oCPOsem{\Gamma;\Phi \vdash
  (\Lan_{\ol K}^{\ol \beta} F)\ol{A}}\rho\}\\
&= \{(t,t)~|~t \in \Lan_{\oCPOsem{\Gamma;\ol\beta \vdash K}\rho[\ol{\beta
    := \_]}} \oCPOsem{\Gamma;\Phi,\ol\beta \vdash F}\rho[\ol{\beta :=
    \_}])\oCPOsem{\Gamma;\Phi \vdash A}\rho\\
&\hspace*{0.2in}\mbox{ and } (t,t) \in \oCPORsem{\Gamma;\Phi \vdash
  (\Lan_{\ol K}^{\ol \beta} F)\ol{A}}\Eq_\rho\}\\
&\subseteq \{ (t,t) \in \oCPORsem{\Gamma;\Phi \vdash
  (\Lan_{\ol K}^{\ol \beta} F)\ol{A}}\Eq_\rho\}\\
&\subseteq \oCPORsem{\Gamma;\Phi \vdash (\Lan_{\ol K}^{\ol \beta}
  F)\ol{A}}\Eq_\rho\\ 
\end{array}\]
To see that $\oCPORsem{\Gamma; \Phi \vdash (\Lan_{\ol K}^{\ol \beta}
  F)\ol{A}}\Eq_\rho \subseteq \Eq_{\oCPOsem{\Gamma; \Phi \vdash
    (\Lan_{\ol K}^{\ol \beta} F)\ol{A}}\rho}$, we note that
\[\begin{array}{ll}
&\oCPORsem{\Gamma;\Phi \vdash (\Lan_{\ol K}^{\ol \beta}
  F)\ol{A}}\Eq_\rho\\
&= \{ (t_1, t_2) : (\oCPOsem{\Gamma;\Phi \vdash
    (\Lan^{\ol{\alpha}}_{\ol{K}} F)\ol{A}}\rho)^2\\
  &\hspace{0.2in}|~\exists\, \ol{Z : \oCPO_0}, t_1',\, t_2',\,
 f : \oCPORsem{\Gamma; \ol{\alpha} \vdash K}\Eq_\rho[\ol{\alpha := \Eq_Z}] \to
  \oCPORsem{\Gamma; \Phi \vdash A}\Eq_\rho\\
  &\hspace{0.2in}\mbox{such that } \iota_{Z,\pi_1 f}t_1' = t_1,\,
  \kappa_{Z,\pi_2 f}t_2' = t_2,\\
  &  \hspace*{0.2in}\mbox{and } (t_1',t_2') :
  \oCPORsem{\Gamma; \Phi, \ol\alpha \vdash F}\Eq_\rho[\ol{\alpha :=
      \Eq_Z}]\,\}\\
&= \{ (t_1, t_2) : (\oCPOsem{\Gamma;\Phi \vdash
    (\Lan^{\ol{\alpha}}_{\ol{K}} F)\ol{A}}\rho)^2\\
  &\hspace{0.2in}|~\exists\, \ol{Z : \oCPO_0}, t_1',\, t_2',\,
 f : \Eq (\oCPORsem{\Gamma; \ol{\alpha} \vdash K}\rho[\ol{\alpha := Z}]) \to
  \Eq (\oCPORsem{\Gamma; \Phi \vdash A}\rho)\\
  &\hspace{0.2in}\mbox{such that } \iota_{Z,\pi_1 f}t_1' = t_1,\,
  \kappa_{Z,\pi_2 f}t_2' = t_2,\\
  &  \hspace*{0.2in}\mbox{and } (t_1',t_2') :
  \Eq (\oCPORsem{\Gamma; \Phi, \ol\alpha \vdash F}\rho[\ol{\alpha := Z}])\,\}
\end{array}\]
by the induction hypothesis for $K$, for $B$, and for $F$. The
condition on $(t_1', t_2')$ then implies that $t_1'=t_2'$.  Moreover,
since $\Eq$ is full, we have that $f = (g,g)$ for some $g :
\oCPOsem{\Gamma; \ol{\alpha} \vdash K}\rho[\ol{\alpha := Z}] \to
\oCPOsem{\Gamma; \Phi \vdash A}\rho$. But then the colimits into which
$\iota$ and $\kappa$ inject are the same, so that in fact $\iota_{Z,g}
= \kappa_{Z,g}$. We therefore have that $\iota_{Z,g} t_1' =
\kappa_{Z,g} t_2'$, and thus that $t_1 = t_2$. The final set displayed
above is therefore contained in $ \Eq_{\oCPOsem{\Gamma; \Phi \vdash
    (\Lan_{\ol K}^{\ol \beta} F)\ol{A}}\rho}$, as desired.

\end{itemize}
\end{proof}


\begin{nono-theorem}[Abstraction Theorem]\label{thm:at-gen}
Every well-formed term $\Gamma;\Phi~|~\Delta \vdash t : F$ induces
a natural transformation from $\sem{\Gamma; \Phi \vdash \Delta}$ to
$\sem{\Gamma; \Phi \vdash F}$, i.e., a triple of natural
transformations 
\[(\oCPOsem{\Gamma;\Phi~|~\Delta \vdash t : F},\,
\oCPOsem{\Gamma;\Phi~|~\Delta \vdash t : F},\,
\oCPORsem{\Gamma;\Phi~|~\Delta \vdash t : F})\]
where
\[\begin{array}{lll}
\oCPOsem{\Gamma;\Phi~|~\Delta \vdash t : F} & : & \oCPOsem{\Gamma;
  \Phi \vdash \Delta} \to \oCPOsem{\Gamma; \Phi \vdash F}
\end{array}\]
has as its component at $\rho : \oCPOenv$ a morphism
\[\begin{array}{lll}
\oCPOsem{\Gamma;\Phi~|~\Delta \vdash t : F}\rho & : & \oCPOsem{\Gamma;
  \Phi \vdash \Delta}\rho \to \oCPOsem{\Gamma; \Phi \vdash F}\rho
\end{array}\]
in $\oCPO$,
\[\begin{array}{lll}
\oCPORsem{\Gamma;\Phi~|~\Delta \vdash t : F} & : & \oCPORsem{\Gamma;
  \Phi \vdash \Delta} \to \oCPORsem{\Gamma; \Phi \vdash F}
\end{array}\]
has as its component at $\rho : \oCPORenv$ a morphism
\[\begin{array}{lll}
\oCPORsem{\Gamma;\Phi~|~\Delta \vdash t : F}\rho & : & \oCPORsem{\Gamma;
  \Phi \vdash \Delta}\rho \to \oCPORsem{\Gamma; \Phi \vdash F}\rho
\end{array}\]
in $\oCPOR$,
and, for all $\rho : \oCPORenv$,
\begin{equation}\label{eq:projs}
\oCPORsem{\Gamma;\Phi~|~\Delta \vdash t : F}\rho =
(\oCPOsem{\Gamma;\Phi~|~\Delta \vdash t : F}(\pi_1 \rho),\,
\oCPOsem{\Gamma;\Phi~|~\Delta \vdash t : F}(\pi_2 \rho))
\end{equation}
\end{nono-theorem}

\begin{proof}
By induction on $t$. The only interesting cases are the cases for
abstraction, application, $\map$, $\tin$, and $\fold$ so we omit the
others.   
\begin{itemize}


\item 
$\underline{\Gamma; \emptyset \,|\, \Delta \vdash L_{\overline{\alpha}}
  x.t : \Nat^{\overline{\alpha}} \,F \,G}$ \;
  To see that $\oCPOsem{\Gamma; \emptyset
    \,|\, \Delta \vdash L_{\overline{\alpha}} x.t :
    \Nat^{\overline{\alpha}} \,F \,G}$ is a natural transformation
  from $\oCPOsem{\Gamma; \emptyset \vdash \Delta}$ to
  $\oCPOsem{\Gamma; \emptyset \vdash \Nat^{\overline{\alpha}} \,F \,G}$
  we need show that, for every $\rho : \oCPOenv$,
  $\oCPOsem{\Gamma; \emptyset \,|\, \Delta \vdash L_{\overline{\alpha}}
    x.t : \Nat^{\overline{\alpha}} \,F \,G}\rho$ is a morphism in
  $\oCPO$ from $\oCPOsem{\Gamma; \emptyset \vdash \Delta}\rho$ to
  $\oCPOsem{\Gamma; \emptyset \vdash \Nat^{\overline{\alpha}} \,F
    \,G}\rho$, and that such family of morphisms is natural.
  First, we need to show that, for all $\ol{A : \oCPO}$ and all $d :
  \oCPOsem{\Gamma; \emptyset \vdash \Delta}\rho = \oCPOsem{\Gamma;
    \ol{\alpha} \vdash \Delta}\rho[\overline{\alpha := A}]$, we have
  $(\oCPOsem{\Gamma; \emptyset \,|\, \Delta \vdash L_{\overline{\alpha}}
    x.t : \Nat^{\overline{\alpha}} \,F \,G}\rho\,d)_{\ol{A}}$ $:
  \oCPOsem{\Gamma; \ol{\alpha} \vdash F}\rho[\overline{\alpha := A}]$
  $\to \oCPOsem{\Gamma; \ol{\alpha} \vdash G}\rho[\overline{\alpha :=
      A}]$, but this follows easily from the induction hypothesis.
That these maps comprise a natural transformation $\eta :
\oCPOsem{\Gamma; \ol{\alpha} \vdash F}\rho[\overline{\alpha := \_}] \to
\oCPOsem{\Gamma; \ol{\alpha} \vdash G}\rho[\overline{\alpha := \_}]$ is
clear since $\eta_{\ol{A}} \, = \, \curry\,
(\oCPOsem{\Gamma; \overline{\alpha} \,|\, \Delta, x : F \vdash t:
  G}\rho[\overline{\alpha := A}])\,d$ is the component at $\ol{A}$ of
the partial specialization to $d$ of the natural transformation
$$\oCPOsem{\Gamma; \overline{\alpha} \,|\, \Delta, x : F \vdash t:
  G}\rho[\overline{\alpha := \_}]$$  To see that the components of
$\eta$ also satisfy the additional condition needed for $\eta$ to
be in $\oCPOsem{\Gamma; \emptyset \vdash \Nat^{\overline{\alpha}} \,F
  \,G}\rho$, let $\overline{R : \oCPOR(A, B)}$ and suppose
\[\begin{array}{lll}
(u, v) &  \in & \oCPORsem{\Gamma;\overline{\alpha} \vdash F}
\Eq_{\rho}[\overline{\alpha := R}]\\
&  = & (\oCPOsem{\Gamma;\overline{\alpha} \vdash F}
\rho[\overline{\alpha := A}], \oCPOsem{\Gamma;\overline{\alpha} \vdash
  F} \rho[\overline{\alpha := B}],
(\oCPORsem{\Gamma;\overline{\alpha} \vdash F}
\Eq_{\rho}[\overline{\alpha := R}])^*)
\end{array}\]
Then the induction hypothesis and
$(d,d) \in \oCPORsem{\Gamma; \emptyset \vdash \Delta} \Eq_\rho =
\oCPORsem{\Gamma; \emptyset \vdash \Delta} \Eq_\rho[\ol{\alpha := R}]$
ensure that
\[\begin{array}{ll}
& (\eta_{\ol{A}}u,\eta_{\ol{B}}v)\\
= & (\curry\, (\oCPOsem{\Gamma; \overline{\alpha} \,|\, \Delta, x : F
  \vdash t: G}\rho[\overline{\alpha := A}])\,d\,u, \curry\,
(\oCPOsem{\Gamma; \overline{\alpha} \,|\, \Delta, x : F \vdash t:
  G}\rho[\overline{\alpha := B}])\,d\,v)\\
= & \curry\, (\oCPORsem{\Gamma; \overline{\alpha} \,|\, \Delta, x : F
  \vdash t: G}\Eq_\rho[\overline{\alpha := R}])\,(d,d)\,(u,v)\\
: & \oCPORsem{\Gamma; \overline{\alpha} \vdash G}
\Eq_{\rho}[\overline{\alpha := R}]  
\end{array}\]
Moreover,
$\oCPOsem{\Gamma; \emptyset \,|\, \Delta \vdash L_{\overline{\alpha}} x.t
: \Nat^{\overline{\alpha}} \,F \,G} \rho$
is trivially natural in $\rho$,
as the functorial action of
$\oCPOsem{\Gamma; \emptyset \vdash \Delta}$
and $\oCPOsem{\Gamma; \emptyset \vdash \Nat^{\overline{\alpha}} \,F \,G}$
on morphisms is the identity.
%let $f : \rho \to \rho'$ and consider the following computation
%  \[\begin{array}{ll}
%  &\oCPOsem{\Gamma;\ol\gamma \vdash \Nat^{\overline{\alpha}} \,F \,G}f
%  \circ
%  \oCPOsem{\Gamma;\ol\gamma \,|\, \Delta \vdash L_{\overline{\alpha}}
%    x.t : \Nat^{\overline{\alpha}} \,F \,G}\rho \\
%  = & \lambda d.\,
%  \oCPOsem{\Gamma;\ol\gamma \vdash \Nat^{\overline{\alpha}} \,F \,G}f
%  (\oCPOsem{\Gamma;\ol\gamma \,|\, \Delta \vdash L_{\overline{\alpha}}
%    x.t : \Nat^{\overline{\alpha}} \,F \,G}\rho\, d) \\
%  = & \lambda d.\,
%  (\lambda \ol{A}.\, \oCPOsem{\Gamma;\ol\gamma,\ol{\alpha} \vdash G}f[\ol{\alpha := \id_A}])
%  \circ
%  \oCPOsem{\Gamma;\ol\gamma \,|\, \Delta \vdash L_{\overline{\alpha}}
%    x.t : \Nat^{\overline{\alpha}} \,F \,G}\rho\, d \\
%  = & \lambda d\, \ol{A}.\,
%  \oCPOsem{\Gamma;\ol\gamma,\ol{\alpha} \vdash G}f[\ol{\alpha := \id_A}]
%  \circ
%  (\oCPOsem{\Gamma;\ol\gamma \,|\, \Delta \vdash L_{\overline{\alpha}}
%    x.t : \Nat^{\overline{\alpha}} \,F \,G}\rho\, d)_{\ol{A}} \\
%  = & \lambda d\, \ol{A}.\,
%  \oCPOsem{\Gamma;\ol\gamma,\ol{\alpha} \vdash G}f[\ol{\alpha := \id_A}]
%  \circ
%  \oCPOsem{\Gamma;\ol\gamma, \ol{\alpha} \,|\, \Delta, x : F \vdash t : G}\rho[\ol{\alpha := A}]\, d \\
%  = & \lambda d\, \ol{A}\, x.\,
%  \oCPOsem{\Gamma;\ol\gamma,\ol{\alpha} \vdash G}f[\ol{\alpha := \id_A}]
%  (\oCPOsem{\Gamma;\ol\gamma, \ol{\alpha} \,|\, \Delta, x : F \vdash t : G}\rho[\ol{\alpha := A}]\, d\, x) \\
%  = & \lambda d\, \ol{A}\, x.\,
%  \oCPOsem{\Gamma;\ol\gamma, \ol{\alpha} \,|\, \Delta, x : F \vdash t : G}\rho'[\ol{\alpha := A}]
%  (\oCPOsem{\Gamma;\ol\gamma,\ol{\alpha} \vdash \Delta, x : F}f[\ol{\alpha := \id_A}] \, d\, x) \\
%  = & \lambda d\, \ol{A}\, x.\,
%  \oCPOsem{\Gamma;\ol\gamma, \ol{\alpha} \,|\, \Delta, x : F \vdash t : G}\rho'[\ol{\alpha := A}]
%  (\oCPOsem{\Gamma;\ol\gamma \vdash \Delta}f\, d)
%  (\oCPOsem{\Gamma;\ol{\alpha} \vdash F}f[\ol{\alpha := \id_A}] x) \\
%  = & \lambda d\, \ol{A}\, x.\,
%  \oCPOsem{\Gamma;\ol\gamma, \ol{\alpha} \,|\, \Delta, x : F \vdash t : G}\rho'[\ol{\alpha := A}]
%  (\oCPOsem{\Gamma;\ol\gamma \vdash \Delta}f\, d)\, x \\
%  = & \lambda d\, \ol{A}.\,
%  (\oCPOsem{\Gamma;\ol\gamma \,|\, \Delta \vdash L_{\overline{\alpha}} x.t : \Nat^{\overline{\alpha}} \,F \,G}\rho'
%  (\oCPOsem{\Gamma;\ol\gamma \vdash \Delta}f\, d))_{\ol{A}} \\
%  = &
%  \oCPOsem{\Gamma;\ol\gamma \,|\, \Delta \vdash L_{\overline{\alpha}} x.t : \Nat^{\overline{\alpha}} \,F \,G}\rho'
%  \circ
%  \oCPOsem{\Gamma;\ol\gamma \vdash \Delta}f
%  \end{array}\]
%  where the sixth equality is by the naturality of the interpretation
%  of $\Gamma;\ol\gamma, \ol{\alpha} \,|\, \Delta, x : F \vdash t : G$,
%  which is given by the induction hypothesis, the seventh equality is
%  by currying, and the eighth equality uses the functoriality of
%  $\oCPOsem{\Gamma;\ol{\alpha} \vdash F}$ and the fact that the only
%  functorial variables in $F$ are in $\ol{\alpha}$.
%
  
\item 
$\underline{\Gamma;\Phi \,|\, \Delta \vdash t_{\overline K} s: G
  [\overline{\alpha := K}]}$\; 
  To see that $\oCPOsem{\Gamma;\Phi \,|\,
  \Delta \vdash t_{\overline K} s: G [\overline{\alpha := K}]}$ is a
  natural transformation from $\oCPOsem{\Gamma; \Phi \vdash \Delta}$ to
  $\oCPOsem{\Gamma;\Phi \vdash G [\overline{\alpha := K}]}$ we must
  show that, for every $\rho : \oCPOenv$, $\oCPOsem{\Gamma;\Phi \,|\,
    \Delta \vdash t_{\overline K} s: G [\overline{\alpha := K}]}\rho$
  is a morphism from $\oCPOsem{\Gamma; \Phi \vdash \Delta}\rho$ to
  $\oCPOsem{\Gamma;\Phi \vdash G [\overline{\alpha := K}]}\rho$, and
  that this family of morphisms is natural in $\rho$. Let $d :
  \oCPOsem{\Gamma; \Phi \vdash \Delta}\rho$. Then
  \[\begin{array}{ll}
  & \oCPOsem{\Gamma;\Phi \,|\, \Delta \vdash t_{\overline K} s: G
  [\overline{\alpha := K}]}\,\rho\,d\\
= & (\eval \circ \langle (\oCPOsem{\Gamma; \emptyset \,|\, \Delta \vdash
  t : \Nat^{\overline{\alpha}} \,F \,G}\rho\;
\_)_{\overline{\oCPOsem{\Gamma;\Phi \vdash K}\rho}},\,
\oCPOsem{\Gamma;\Phi \,|\, \Delta \vdash s: F [\overline{\alpha :=
      K}]}\rho \rangle)\,d\\
= & \eval ((\oCPOsem{\Gamma; \emptyset \,|\, \Delta \vdash t :
  \Nat^{\overline{\alpha}} \,F \,G}\rho\;
\_)_{\overline{\oCPOsem{\Gamma;\Phi \vdash K}\rho}} \,d,\,
\oCPOsem{\Gamma;\Phi \,|\, \Delta \vdash s: F [\overline{\alpha :=
      K}]}\rho\, d)\\
= & \eval ((\oCPOsem{\Gamma; \emptyset \,|\, \Delta \vdash t :
  \Nat^{\overline{\alpha}} \,F \,G}\rho\;
d)_{\overline{\oCPOsem{\Gamma;\Phi \vdash K}\rho}},\,
\oCPOsem{\Gamma;\Phi \,|\, \Delta \vdash s: F [\overline{\alpha :=
      K}]}\rho\, d)\\
\end{array}\]
The induction hypothesis ensures that $(\oCPOsem{\Gamma; \emptyset \,|\,
  \Delta \vdash t : \Nat^{\overline{\alpha}} \,F \,G}\rho\,
d)_{\overline{\oCPOsem{\Gamma;\Phi \vdash K}\rho}}$ has type
$\oCPOsem{\Gamma; \ol{\alpha} \vdash F}\rho[\ol{\alpha :=
    \oCPOsem{\Gamma;\Phi \vdash K}\rho}] \to \oCPOsem{\Gamma;
  \ol{\alpha} \vdash G}\rho[\ol{\alpha := \oCPOsem{\Gamma;\Phi \vdash
      K}\rho}]$.  Since, in addition, 
\[\begin{array}{ll}
  &\oCPOsem{\Gamma;\Phi \,|\,
  \Delta \vdash s: F [\overline{\alpha := K}]}\rho\, d :
\oCPOsem{\Gamma; \Phi \vdash F[\ol{\alpha := K}]}\rho \\ 
  =&\oCPOsem{\Gamma; \Phi, \ol{\alpha} \vdash F}\rho[\ol{\alpha :=
    \oCPOsem{\Gamma;\Phi \vdash K}\rho}] \\
    =& \oCPOsem{\Gamma;
  \ol{\alpha} \vdash F}\rho[\ol{\alpha := \oCPOsem{\Gamma;\Phi \vdash
      K}\rho}] 
\end{array}\]
by Equation~(6) from the paper, we have that 
\[\begin{array}{ll}
  &\oCPOsem{\Gamma;\Phi \,|\, \Delta
  \vdash t_{\overline K} s: G [\overline{\alpha :=
      K}]}\,\rho\,d : \oCPOsem{\Gamma; \Phi ,\ol{\alpha} \vdash
  G}\rho[\ol{\alpha := \oCPOsem{\Gamma;\Phi \vdash K}\rho}]  \\
  =&\oCPOsem{\Gamma; \Phi \vdash G[\ol{\alpha := K}]}\rho
\end{array}\]
as desired.

\vspace*{0.1in}

To see that the family of maps comprising $\oCPOsem{\Gamma;\Phi \,|\,
  \Delta \vdash t_{\overline K} s: G [\overline{\alpha := K}]}$
is natural in $\rho$
we need to show that, if $f : \rho \to \rho'$ in $\oCPOenv$, then the
following diagram commutes, where $g = \oCPOsem{\Gamma; \emptyset \,|\,
  \Delta \vdash t : \Nat^{\overline{\alpha}} \,F \,G}$ and $h =
\oCPOsem{\Gamma;\Phi \,|\, \Delta \vdash s: F [\overline{\alpha :=
      K}]}$:
{\footnotesize
\[\begin{tikzcd}
\oCPOsem{\Gamma;\Phi \vdash \Delta}\rho \ar[r, "{\oCPOsem{\Gamma;\Phi
  \vdash \Delta}f}"] \ar[d, "{\langle g \rho, h \rho\rangle}"']
& \oCPOsem{\Gamma;\Phi \vdash 
  \Delta}\rho' \ar[d, "{\langle g \rho', h \rho' \rangle}"]\\
\oCPOsem{\Gamma;\emptyset \vdash \Nat^{\overline{\alpha}} \,F \,G}\rho
\times \oCPOsem{\Gamma;\Phi \vdash F [\overline{\alpha := K}]}\rho
\ar[d, "{\eval \circ ((-)_{\overline{\sem{\Gamma;\Phi \vdash K}\rho}} \times
    \id)}"']
\ar[r, bend left = 5, "{\oCPOsem{\Gamma;\emptyset\vdash
      \Nat^{\overline{\alpha}} \,F \,G}f\, \times\, \oCPOsem{\Gamma;\Phi
      \vdash F [\overline{\alpha := K}]}f}"] &
\oCPOsem{\Gamma;\emptyset \vdash \Nat^{\overline{\alpha}} \,F \,G}\rho'
\times \oCPOsem{\Gamma;\Phi \vdash F [\overline{\alpha := K}]}\rho'
\ar[d, "{\eval \circ ((-)_{\overline{\sem{\Gamma;\Phi \vdash
          K}\rho'}} \times \id)}"] \\
\oCPOsem{\Gamma;\Phi \vdash G [\overline{\alpha := K}]}\rho
\ar[r, "{\oCPOsem{\Gamma;\Phi \vdash G [\overline{\alpha := K}]}f}"']
&
\oCPOsem{\Gamma;\Phi \vdash G [\overline{\alpha := K}]}\rho'
\end{tikzcd}\]}

\noindent
The top diagram commutes because $g$ and $h$ are natural in $\rho$ by
the induction hypothesis.
To see that the bottom diagram commutes,
we need to show that
$\oCPOsem{\Gamma;\Phi \vdash G [\overline{\alpha := K}]}f
(\eta_{\overline{\sem{\Gamma;\Phi \vdash K}\rho}} x) =
(\oCPOsem{\Gamma; \emptyset \vdash \Nat^{\overline{\alpha}} \,F \,G} f\, \eta
)_{\overline{\sem{\Gamma;\Phi \vdash K}\rho'}}
(\oCPOsem{\Gamma;\Phi \vdash F [\overline{\alpha := K}]}f x)$
holds for all $\eta \in \oCPOsem{\Gamma; \emptyset \vdash
  \Nat^{\overline{\alpha}} \,F \,G}\rho$ and $x \in
\oCPOsem{\Gamma;\Phi \vdash F [\overline{\alpha := K}]}\rho$,
i.e.,
by remembering the following facts,
\begin{align*}
\oCPOsem{\Gamma;\Phi \vdash F[\ol{\alpha := K}]}\rho
&= \oCPOsem{\Gamma; \ol{\alpha} \vdash F}\rho[\ol{\alpha :=
    \oCPOsem{\Gamma;\Phi \vdash K}\rho}] \\
\oCPOsem{\Gamma;\Phi \vdash F[\ol{\alpha := K}]}f
&= \oCPOsem{\Gamma; \ol{\alpha} \vdash F}
  f [\ol{\alpha := \oCPOsem{\Gamma;\Phi \vdash K}f}] \\
\oCPOsem{\Gamma;\Phi \vdash G[\ol{\alpha := K}]}\rho
&= \oCPOsem{\Gamma; \ol{\alpha} \vdash G}\rho[\ol{\alpha :=
    \oCPOsem{\Gamma;\Phi \vdash K}\rho}] \\
\oCPOsem{\Gamma;\Phi \vdash G[\ol{\alpha := K}]}f
&= \oCPOsem{\Gamma; \ol{\alpha} \vdash
  G} f [\ol{\alpha := \oCPOsem{\Gamma;\Phi \vdash K}f}]
\end{align*}
we need to show that
\begin{multline*}
\oCPOsem{\Gamma; \ol{\alpha} \vdash G} f[\overline{\alpha := \oCPOsem{\Gamma;\Phi \vdash K} f }]
  \circ
\eta_{\overline{\oCPOsem{\Gamma;\Phi \vdash K}\rho}} \\
=
%\oCPOsem{\Gamma; \ol{\alpha} \vdash G} f[\overline{\alpha := \id_{\oCPOsem{\Gamma;\Phi \vdash K}\rho'}}]
%\circ
\eta_{\overline{\oCPOsem{\Gamma;\Phi \vdash K}\rho'}}
\circ
\oCPOsem{\Gamma; \ol{\alpha} \vdash F} f [\ol{\alpha := \oCPOsem{\Gamma;\Phi \vdash K}f}]
\end{multline*}
for all $\eta \in \oCPOsem{\Gamma;\emptyset \vdash
  \Nat^{\overline{\alpha}} \,F \,G}\rho$.
But this follows from the naturality of $\eta$, which ensures the commutativity of
{\footnotesize
\[\begin{tikzcd}[column sep = large]
\oCPOsem{\Gamma; \ol{\alpha} \vdash F}\rho[\ol{\alpha :=
    \oCPOsem{\Gamma;\Phi \vdash K}\rho}] \ar[r,
  "{\;\;\;\eta_{\ol{\oCPOsem{\Gamma;\Phi \vdash K}\rho}}\;\;\; }"]
\ar[d, "{\oCPOsem{\Gamma; \ol{\alpha} \vdash F} f [\ol{\alpha := 
        \oCPOsem{\Gamma;\Phi \vdash K}f}]}"']
& \oCPOsem{\Gamma;
  \ol{\alpha} \vdash G}\rho[\ol{\alpha := \oCPOsem{\Gamma;\Phi \vdash
      K}\rho}]
\ar[d, "{\oCPOsem{\Gamma; \ol{\alpha} \vdash G} f [\ol{\alpha := 
        \oCPOsem{\Gamma;\Phi \vdash K}f}]}"]\\
\oCPOsem{\Gamma; \ol{\alpha} \vdash F}\rho'[\ol{\alpha :=
    \oCPOsem{\Gamma;\Phi \vdash K}\rho'}] \ar[r,
  "{\eta_{\ol{\oCPOsem{\Gamma;\Phi \vdash K}\rho'}} }"]
& \oCPOsem{\Gamma; \ol{\alpha} \vdash G}\rho'[\ol{\alpha :=
    \oCPOsem{\Gamma;\Phi \vdash K}\rho'}]
\end{tikzcd}\]}

%and the observation that
%$\oCPOsem{\Gamma;\ol\gamma, \ol{\alpha} \vdash G} f[\overline{\alpha := \oCPOsem{\Gamma;\Phi, \ol\gamma \vdash K} f }]$
%is equal to
%\[
%\oCPOsem{\Gamma;\ol\gamma, \ol{\alpha} \vdash G} f[\overline{\alpha := \id_{\oCPOsem{\Gamma;\Phi, \ol\gamma \vdash K}\rho'}}]
%\circ \oCPOsem{\Gamma;\ol\gamma, \ol{\alpha} \vdash G} \id_{\rho} [\overline{\alpha := \oCPOsem{\Gamma;\Phi, \ol\gamma \vdash K}f}]
%\]

\item 
  $\underline{\Gamma;\emptyset~|~\emptyset \vdash
  \map^{\ol{F},\ol{G}}_H :
  \Nat^\emptyset\;(\ol{\Nat^{\ol{\beta},\ol{\gamma}}\,F\,G})\;
  (\Nat^{\ol{\gamma}}\,H[\ol{\phi :=_{\ol{\beta}} F}]\,H[\ol{\phi
      :=_{\ol{\beta}} G}])}$\;
To see that
\[
\oCPOsem{\Gamma; \emptyset~|~\emptyset \vdash \map^{\ol{F},\ol{G}}_H
    : \Nat^\emptyset\;(\ol{\Nat^{\ol{\beta},\ol{\gamma}}\,F\,G})\;
    (\Nat^{\ol{\gamma}}\,H[\ol{\phi :=_{\ol{\beta}} F}]\,H[\ol{\phi
        :=_{\ol{\beta}} G}])}\,\rho\,d\, \ol{\eta}
\]
is in $\oCPOsem{\Gamma; \emptyset \vdash
    \Nat^{\ol{\gamma}}\,H[\ol{\phi :=_{\ol{\beta}} F}]\,H[\ol{\phi
        :=_{\ol{\beta}} G}]} \rho$
for all $\rho : \oCPOenv$, all $\ol{\eta : \oCPOsem{\Gamma; \emptyset
  \vdash\Nat^{\ol{\beta},\ol{\gamma}}\,F\,G} \rho}$,
  and $d$ the unique element of $\oCPOsem{\Gamma;\emptyset \vdash \emptyset} \rho$,
  we first note that
$\oCPOsem{\Gamma ;\ol{\phi}, \ol{\gamma} \vdash H}$ is a functor from
  $\oCPOenv$ to $\oCPO$ and, for any $\ol C$, $\id_{\rho[\ol{\gamma :=
        C}]}[\ol{\phi := \lambda \ol{B}. \eta_{\ol{B}\,\ol{C}}}]$ is a
  morphism in $\oCPOenv$ from \[\rho[\ol{\gamma := C}][\ol{\phi :=
      \lambda \ol{B}.\oCPOsem{\Gamma; \ol{\gamma},\ol{\beta} \vdash
        F}\rho[\ol{\gamma := C}][\ol{\beta := B}]}]\] to
\[\rho[\ol{\gamma := C}][\ol{\phi := \lambda \ol{B}.\oCPOsem{\Gamma;
\ol{\gamma},\ol{\beta} \vdash G}\rho[\ol{\gamma := C}][\ol{\beta := B}]}]\]
so that
$(\oCPOsem{\Gamma; \emptyset~|~\emptyset \vdash
\map^{\ol{F},\ol{G}}_H :
\Nat^\emptyset\;(\ol{\Nat^{\ol{\beta},\ol{\gamma}}\,F\,G})\;
(\Nat^{\ol{\gamma}}\,H[\ol{\phi :=_{\ol{\beta}} F}]\,H[\ol{\phi
    :=_{\ol{\beta}} G}])}\,\rho\,d\, \ol{\eta})_{\ol{C}}$
$=  \oCPOsem{\Gamma; \ol{\phi},\ol{\gamma} \vdash H}\id_{\rho[\ol{\gamma
      := C}]}[\ol{\phi := \lambda \ol{B}. \eta_{\ol{B}\,\ol{C}}}]$
is indeed a morphism from
$\oCPOsem{\Gamma ;\ol{\gamma} \vdash H[\ol{\phi := F}]}\rho[\ol{\gamma
      := C}]$
to 
$ \oCPOsem{\Gamma ;\ol{\gamma} \vdash H[\ol{\phi := G}]}\rho[\ol{\gamma
      := C}]$.
This family of morphisms is natural in $\ol C$: if $\ol{f : C \to C'}$
then, writing $\xi$ for
\[
\oCPOsem{\Gamma; \emptyset~|~\emptyset \vdash \map^{\ol{F},\ol{G}}_H
    : \Nat^\emptyset\;(\ol{\Nat^{\ol{\beta},\ol{\gamma}}\,F\,G})\;
    (\Nat^{\ol{\gamma}}\,H[\ol{\phi :=_{\ol{\beta}} F}]\,H[\ol{\phi
        :=_{\ol{\beta}} G}])}\,\rho\,d\, \ol{\eta}
\]
the naturality of $\eta$,
together with the fact that composition of environments is
computed componentwise, ensure that the following naturality diagram
for $\xi$ commutes:
{\footnotesize
\[\begin{tikzcd}
\oCPOsem{\Gamma ;\ol{\gamma} \vdash H[\ol{\phi := F}]}\rho[\ol{\gamma
      := C}] \ar[r, "{\xi_{\ol{C}}}"]
\ar[d, "{\oCPOsem{\Gamma ;\ol{\gamma} \vdash H[\ol{\phi :=
          F}]}\id_{\rho}[\ol{\gamma := f}]}"']
& \oCPOsem{\Gamma ; \ol{\gamma} \vdash H[\ol{\phi := G}]}\rho[\ol{\gamma
      := C}]
\ar[d, "{\oCPOsem{\Gamma ; \ol{\gamma} \vdash H[\ol{\phi :=
          G}]}\id_{\rho}[\ol{\gamma := f}]}"]\\
\oCPOsem{\Gamma ;\ol{\gamma} \vdash H[\ol{\phi := F}]}\rho[\ol{\gamma
      := C'}] \ar[r, "{\xi_{\ol{C'}}}"]
& \oCPOsem{\Gamma ; \ol{\gamma} \vdash H[\ol{\phi := G}]}\rho[\ol{\gamma
      := C'}] 
\end{tikzcd}\]}
That, for all $\rho : \oCPOenv$ and $d : \oCPOsem{\Gamma; \emptyset \vdash
  \emptyset}\rho$, $\xi$ satisfies the additional condition needed for
it to be in $\oCPOsem{\Gamma; \emptyset \vdash
  \Nat^{\ol{\gamma}}\,H[\ol{\phi :=_{\ol{\beta}} F}]\,H[\ol{\phi
      :=_{\ol{\beta}} G}]}\rho$ follows from the fact
that $\eta$ satisfies the extra
condition needed for it to be in its corresponding
$\oCPOsem{\Gamma; \emptyset \vdash \Nat^{\ol{\beta},\ol{\gamma}}\,F\,G} \rho$.

Finally, since $\Phi = \emptyset$, the naturality of
\[
\oCPOsem{\Gamma; \emptyset~|~\emptyset \vdash \map^{\ol{F},\ol{G}}_H
    : \Nat^\emptyset\;(\ol{\Nat^{\ol{\beta},\ol{\gamma}}\,F\,G})\;
    (\Nat^{\ol{\gamma}}\,H[\ol{\phi :=_{\ol{\beta}} F}]\,H[\ol{\phi
        :=_{\ol{\beta}} G}])}\rho
\]
in $\rho$ is trivial.

%For the naturality of
%$\oCPOsem{\Gamma; \ol\alpha~|~\Delta \vdash \map^{\ol{F},\ol{G}}_H\, \ol{t}
%  : \Nat^{\ol{\gamma}}\,H[\ol{\phi :=_{\ol{\beta}} F}]\,H[\ol{\phi
%      :=_{\ol{\beta}} G}]}$,
%consider $f : \rho \to \rho'$.
%We need to prove that
%\begin{multline*}
%\oCPOsem{\Gamma; \ol\alpha \vdash
% \Nat^{\ol{\gamma}}\,H[\ol{\phi :=_{\ol{\beta}} F}]\,H[\ol{\phi
%      :=_{\ol{\beta}} G}]} f
%\circ
%\oCPOsem{\Gamma; \ol\alpha~|~\Delta \vdash
%\map^{\ol{F},\ol{G}}_H\, \ol{t} :
%\Nat^{\ol{\gamma}}\,H[\ol{\phi :=_{\ol{\beta}} F}]\,H[\ol{\phi
%    :=_{\ol{\beta}} G}]}\,\rho \\
%= 
%\oCPOsem{\Gamma; \ol\alpha~|~\Delta \vdash
%\map^{\ol{F},\ol{G}}_H\, \ol{t} :
%\Nat^{\ol{\gamma}}\,H[\ol{\phi :=_{\ol{\beta}} F}]\,H[\ol{\phi
%    :=_{\ol{\beta}} G}]}\,\rho'
%\circ
%\oCPOsem{\Gamma; \ol\alpha \vdash \Delta}\, f
%\end{multline*}
%i.e., that
%\begin{multline*}
%\oCPOsem{\Gamma; \ol\alpha \vdash 
%\Nat^{\ol{\gamma}}\,H[\ol{\phi :=_{\ol{\beta}} F}]\,H[\ol{\phi
%      :=_{\ol{\beta}} G}]} f
%(\oCPOsem{\Gamma; \ol\alpha~|~\Delta \vdash
%\map^{\ol{F},\ol{G}}_H\, \ol{t} :
%\Nat^{\ol{\gamma}}\,H[\ol{\phi :=_{\ol{\beta}} F}]\,H[\ol{\phi
%    :=_{\ol{\beta}} G}]}\,\rho\,d) \\
%=
%\oCPOsem{\Gamma; \ol\alpha~|~\Delta \vdash
%\map^{\ol{F},\ol{G}}_H\, \ol{t} :
%\Nat^{\ol{\gamma}}\,H[\ol{\phi :=_{\ol{\beta}} F}]\,H[\ol{\phi
%    :=_{\ol{\beta}} G}]}\,\rho'
%(\oCPOsem{\Gamma; \ol\alpha \vdash \Delta}\, f\, d )
%\end{multline*}
%for any $d : \oCPOsem{\Gamma; \ol\alpha \vdash \Delta} \rho$.
%That is shown by the following calculations:
%\[
%\begin{array}{ll}
%&\oCPOsem{\Gamma; \ol\alpha \vdash 
%\Nat^{\ol{\gamma}}\,H[\ol{\phi :=_{\ol{\beta}} F}]\,H[\ol{\phi
%      :=_{\ol{\beta}} G}]} f
%(\oCPOsem{\Gamma; \ol\alpha~|~\Delta \vdash
%\map^{\ol{F},\ol{G}}_H\, \ol{t} :
%\Nat^{\ol{\gamma}}\,H[\ol{\phi :=_{\ol{\beta}} F}]\,H[\ol{\phi
%    :=_{\ol{\beta}} G}]}\,\rho\,d) \\
%=&
%\lambda \ol{B}.\,
%\oCPOsem{\Gamma; \ol\alpha, \ol{\gamma} \vdash 
%H[\ol{\phi :=_{\ol{\beta}} G}]} f[\ol{\gamma := \id_B}] \\
%&\hspace{2em}\circ
%(\oCPOsem{\Gamma; \ol\alpha~|~\Delta \vdash
%\map^{\ol{F},\ol{G}}_H\, \ol{t} :
%\Nat^{\ol{\gamma}}\,H[\ol{\phi :=_{\ol{\beta}} F}]\,H[\ol{\phi
%    :=_{\ol{\beta}} G}]}\,\rho\,d)_{\ol{B}} \\
%=&
%\lambda \ol{B}.\,
%\oCPOsem{\Gamma; \ol{\phi}, \ol{\gamma} \vdash 
%H} f[\ol{\gamma := \id_B}] [\ol{\phi := \lambda \ol{A}.\, \oCPOsem{\Gamma; \ol\alpha, \ol{\beta}, \ol{\gamma} \vdash G}f[\ol{\beta := \id_A}][\ol{\gamma := \id_B}]}] \\
%&\hspace{2em}\circ
%\oCPOsem{\Gamma; \ol{\phi},\ol{\gamma} \vdash H}\id_{\rho[\ol{\gamma := B}]}
%[\ol{\phi := \lambda \ol{A}. {(\oCPOsem{\Gamma; \ol\alpha~|~\Delta \vdash t : \Nat^{\ol{\beta},\ol{\gamma}}\,F\,G} \rho\, d)}_{\ol{A}\,\ol{B}}}] \\
%=&
%\lambda \ol{B}.\,
%\oCPOsem{\Gamma; \ol{\phi}, \ol{\gamma} \vdash H}
%f[\ol{\gamma := \id_B}] \\
%&\hspace{2em}[\ol{\phi := \lambda \ol{A}.\, \oCPOsem{\Gamma; \ol\alpha, \ol{\beta}, \ol{\gamma} \vdash G}f[\ol{\beta := \id_A}][\ol{\gamma := \id_B}]
%\circ {(\oCPOsem{\Gamma; \ol\alpha~|~\Delta \vdash t : \Nat^{\ol{\beta},\ol{\gamma}}\,F\,G} \rho\, d)}_{\ol{A}\,\ol{B}}}] \\
%=&
%\lambda \ol{B}.\, 
%\oCPOsem{\Gamma; \ol{\phi},\ol{\gamma} \vdash H}\id_{\rho'[\ol{\gamma := B}]}
%[\ol{\phi := \lambda \ol{A}. {(\oCPOsem{\Gamma; \ol\alpha \vdash \Nat^{\ol{\beta},\ol{\gamma}}\,F\,G} f (\oCPOsem{\Gamma; \ol\alpha~|~\Delta \vdash t : \Nat^{\ol{\beta},\ol{\gamma}}\,F\,G}\, \rho\, d ))}_{\ol{A}\,\ol{B}}}] \\
%=&
%\lambda \ol{B}.\, 
%\oCPOsem{\Gamma; \ol{\phi},\ol{\gamma} \vdash H}\id_{\rho'[\ol{\gamma := B}]}
%[\ol{\phi := \lambda \ol{A}. {(\oCPOsem{\Gamma; \ol\alpha~|~\Delta \vdash t : \Nat^{\ol{\beta},\ol{\gamma}}\,F\,G} \rho' (\oCPOsem{\Gamma; \ol\alpha \vdash \Delta}\, f\, d ))}_{\ol{A}\,\ol{B}}}] \\
%=&
%\oCPOsem{\Gamma; \ol\alpha~|~\Delta \vdash
%\map^{\ol{F},\ol{G}}_H\, \ol{t} :
%\Nat^{\ol{\gamma}}\,H[\ol{\phi :=_{\ol{\beta}} F}]\,H[\ol{\phi
%    :=_{\ol{\beta}} G}]}\,\rho'
%(\oCPOsem{\Gamma; \ol\alpha \vdash \Delta}\, f\, d )
%\end{array}
%\]
%where the third equality is given by composition of morphisms of environments
%and the fifth equality is given by the naturality of $\Gamma; \ol\alpha~|~\Delta \vdash t : \Nat^{\ol{\beta},\ol{\gamma}}\,F\,G$,
%which we have by the induction hypothesis.

\item
$\underline{\Gamma;\emptyset \,|\, \emptyset \vdash \tin_H :
  \Nat^{\ol{\beta},\ol{\gamma}} \, H[\phi := (\mu \phi.\lambda
    {\overline \alpha}.H){\overline \beta}][\ol{\alpha := \beta}]
  \;(\mu \phi.\lambda {\overline \alpha}.H){\overline \beta}}$\; To
  see that if $d : \oCPOsem{\Gamma;\emptyset \vdash \emptyset} \rho$
  then $\oCPOsem{\Gamma;\emptyset \,|\, \emptyset \vdash \tin_H :
    \Nat^{\ol{\beta},\ol{\gamma}} \, H[\phi := (\mu \phi.\lambda
      {\overline \alpha}.H){\overline \beta}][\ol{\alpha := \beta}]
    \;(\mu \phi.\lambda {\overline \alpha}.H){\overline \beta}}\,
  \rho\,d$ is in\\ $\oCPOsem{\Gamma;\emptyset \vdash
    \Nat^{\ol{\beta},\ol{\gamma}} \, H[\phi := (\mu \phi.\lambda
      {\overline \alpha}.H){\overline \beta}][\ol{\alpha := \beta}]
    \;(\mu \phi.\lambda {\overline \alpha}.H){\overline \beta}}\,
  \rho$, we first note that, for all $\ol{B}$ and $\ol{C}$,
  $(\oCPOsem{\Gamma;\emptyset \,|\, \emptyset \vdash \tin_H :
    \Nat^{\ol{\beta},\ol{\gamma}} \, H[\phi := (\mu \phi.\lambda
      {\overline \alpha}.H){\overline \beta}][\ol{\alpha := \beta}]
    \;(\mu \phi.\lambda {\overline \alpha}.H){\overline \beta}}\,
  \rho\,d)_{\ol{B}\,\ol{C}}\, =\,
  (\mathit{in}_{T^\oCPO_{H,\rho[\ol{\gamma := C}]}})_{\ol{B}}$ maps
  $\oCPOsem{\Gamma;\ol{\beta},\ol{\gamma} \vdash H[\phi := (\mu
      \phi.\lambda {\overline \alpha}.H){\overline \beta}][\ol{\alpha
        := \beta}]}\rho[\ol{\beta := B}][\ol{\gamma := C}] =
  T^\oCPO_{H,\rho[\ol{\gamma := C}]}\, (\mu T^\oCPO_{H,\rho[\ol{\gamma
        := C}]}) \, \ol{B}$ to $\oCPOsem{\Gamma;\ol{\beta},\ol{\gamma}
    \vdash (\mu \phi.\lambda {\overline \alpha}.H){\overline \beta}}
  \rho[\ol{\beta := B}][\ol{\gamma := C}] = (\mu
  T^\oCPO_{H,\rho[\ol{\gamma := C}]}) \, \ol{B}$. Secondly, we observe
  that\\ $\oCPOsem{\Gamma;\emptyset \,|\, \emptyset \vdash \tin_H :
    \Nat^{\ol{\beta},\ol{\gamma}} \, H[\phi := (\mu \phi.\lambda
      {\overline \alpha}.H){\overline \beta}][\ol{\alpha := \beta}]
    \;(\mu \phi.\lambda {\overline \alpha}.H){\overline
      \beta}}\,\rho\,d = \lambda
  \ol{B}\,\ol{C}.\,(\mathit{in}_{T^\oCPO_{H,\rho[\ol{\gamma :=
          C}]}})_{\ol{B}}$ is natural in $\ol{B}$ and $\ol{C}$, since
  naturality of $\mathit{in}$ with respect to its functor argument and
  naturality of $\mathit{in}_{T^\oCPO_{H,\rho[\ol{\gamma := C'}]}}$ ensure
  that the following diagram commutes for all $\ol{f : B \to B'}$ and
  $\ol{g : C \to C'}$:
{\tiny
  \[\begin{tikzcd}[column sep=2.5in, row sep=0.75in]
T^\oCPO_{H,\rho[\ol{\gamma := C}]}\, (\mu T^\oCPO_{H,\rho[\ol{\gamma := C}]})\, \ol{B}
\ar[d, "{T^\oCPO_{H,\id_\rho[\ol{\gamma := g}]}\,(\mu
    T^\oCPO_{H,\id_\rho[\ol{\gamma := g}]}) \, \ol{B}}"] \ar[r,
  "{(\mathit{in}_{T^\oCPO_{H,\rho[\ol{\gamma := C}]}})_{\ol{B}}}" ]
& (\mu T^\oCPO_{H,\rho[\ol{\gamma := C}]})\, \ol{B} \ar[d, "{(\mu
    T^\oCPO_{H,\id_\rho[\ol{\gamma := g}]}) \, \ol{B}}"]\\
T^\oCPO_{H,\rho[\ol{\gamma := C'}]}\, (\mu
T^\oCPO_{H,\rho[\ol{\gamma := C'}]})\, \ol{B} \ar[d, "{T^\oCPO_{H,\rho[\ol{\gamma :=
          C'}]}\, (\mu T^\oCPO_{H,\rho[\ol{\gamma := C'}]})\, \ol{f}}"]
\ar[r,"{(\mathit{in}_{T^\oCPO_{H,\rho[\ol{\gamma := C'}]}})_{\ol{B}}}" ] & 
  (\mu T^\oCPO_{H,\rho[\ol{\gamma := C'}]})\, \ol{B}
\ar[d,"{\mu {T^\oCPO_{H,\rho[\ol{\gamma := C'}]}}\, {\ol{f}}}" ] 
\\
T^\oCPO_{H,\rho[\ol{\gamma := C'}]}\, (\mu T^\oCPO_{H,\rho[\ol{\gamma := C'}]})\,
\ol{B'} \ar[r, "{(\mathit{in}_{T^\oCPO_{H,\rho[\ol{\gamma :=
            C'}]}})_{\ol{B'}}}"] & (\mu T^\oCPO_{H,\rho[\ol{\gamma := C'}]})\,
\ol{B'}
\end{tikzcd}\]
}
  That, for all $\rho : \oCPOenv$ and $d :
\oCPOsem{\Gamma;\emptyset \vdash \emptyset}\rho$,
\[\oCPOsem{\Gamma;\emptyset \,|\, \emptyset \vdash
  \tin_H : \Nat^{\ol{\beta},\ol{\gamma}} \, H[\phi := (\mu \phi.\lambda
    {\overline \alpha}.H){\overline \beta}][\ol{\alpha := \beta}]
  \;(\mu \phi.\lambda {\overline \alpha}.H){\overline
    \beta}}\,\rho\,d\] satisfies the additional property needed for
it to be in \[\oCPOsem{\Gamma;\emptyset \vdash
  \Nat^{\ol{\beta},\ol{\gamma}} \, H[\phi := (\mu \phi.\lambda
    {\overline \alpha}.H){\overline \beta}][\ol{\alpha := \beta}]
  \;(\mu \phi.\lambda {\overline \alpha}.H){\overline \beta}}\,\rho\]
  let $\ol{R : \oCPOR(B,B')}$ and $\ol{S : \oCPOR(C,C')}$ follows from the
  fact that 
\[\begin{array}{ll}
 & (\,(\oCPOsem{\Gamma;\emptyset \,|\, \emptyset \vdash
  \tin_H : \Nat^{\ol{\beta},\ol{\gamma}} \, H[\phi := (\mu \phi.\lambda
    {\overline \alpha}.H){\overline \beta}][\ol{\alpha := \beta}]
  \;(\mu \phi.\lambda {\overline \alpha}.H){\overline
    \beta}}\,\rho\,d)_{\ol{B},\ol{C}},\,\\
 & \hspace*{0.5in}(\oCPOsem{\Gamma;\emptyset \,|\, \emptyset \vdash
  \tin_H : \Nat^{\ol{\beta},\ol{\gamma}} \, H[\phi := (\mu \phi.\lambda
    {\overline \alpha}.H){\overline \beta}][\ol{\alpha := \beta}]
  \;(\mu \phi.\lambda {\overline \alpha}.H){\overline
    \beta}}\,\rho\,d)_{\ol{B'},\ol{C'}}\,)\\
=& (\, (\mathit{in}_{T^\oCPO_{H,\rho[\ol{\gamma := C}]}})_{\ol{B}},
(\mathit{in}_{T^\oCPO_{H,\rho[\ol{\gamma := C'}]}})_{\ol{B'}}\,)
\end{array}\]
has type
\[\begin{array}{ll}
%& (\, \oCPOsem{\Gamma;\ol{\beta},\ol{\gamma} \vdash H[\phi := (\mu
%    \phi.\lambda {\overline \alpha}.H){\overline \beta}][\ol{\alpha :=
%      \beta}]}\rho[\ol{\beta := B}][\ol{\gamma := C}] \to\\ 
%& \hspace*{0.5in}\oCPOsem{\Gamma;\ol{\beta},\ol{\gamma} \vdash (\mu \phi.\lambda
%  {\overline \alpha}.H){\overline \beta}} \rho[\ol{\beta :=
%    B}][\ol{\gamma := C}],\\
%& \hspace*{0.05in}\oCPOsem{\Gamma;\ol{\beta},\ol{\gamma} \vdash H[\phi := (\mu
%    \phi.\lambda {\overline \alpha}.H){\overline \beta}][\ol{\alpha :=
%      \beta}]}\rho[\ol{\beta := B'}][\ol{\gamma := C'}] \to\\ 
%& \hspace*{0.5in}\oCPOsem{\Gamma;\ol{\beta},\ol{\gamma} \vdash (\mu \phi.\lambda
%  {\overline \alpha}.H){\overline \beta}} \rho[\ol{\beta :=
%    B'}][\ol{\gamma := C'}]\,)\\
& (\, T^\oCPO_{H,\rho[\ol{\gamma := C}]}\, (\mu
T^\oCPO_{H,\rho[\ol{\gamma := C}]}) \, 
\ol{B} \to (\mu T^\oCPO_{H,\rho[\ol{\gamma := C}]}) \, \ol{B}, \, \\
& \hspace*{0.5in}T^\oCPO_{H,\rho[\ol{\gamma := C'}]}\, (\mu
T^\oCPO_{H,\rho[\ol{\gamma := C'}]}) \, 
\ol{B'} \to (\mu T^\oCPO_{H,\rho[\ol{\gamma := C'}]}) \,\ol{B'} \, )\\
= &
 \oCPORsem{\Gamma;\ol{\beta},\ol{\gamma} \vdash H[\phi := (\mu
    \phi.\lambda {\overline \alpha}.H){\overline \beta}][\ol{\alpha :=
      \beta}]}\Eq_\rho[\ol{\beta := R}][\ol{\gamma := S}] \to\\
 & \hspace*{0.5in} \oCPORsem{\Gamma;\ol{\beta},\ol{\gamma} \vdash (\mu
  \phi.\lambda \ol{\alpha}.H)\ol{\beta}} \Eq_\rho[\ol{\beta:=
    R}][\ol{\gamma :=S}]
\end{array}\]

Finally, since $\Phi = \emptyset$, the naturality of
\[
\oCPOsem{\Gamma;\emptyset \,|\, \emptyset \vdash \tin_H :
  \Nat^{\ol{\beta},\ol{\gamma}} \, H[\phi := (\mu \phi.\lambda
    {\overline \alpha}.H){\overline \beta}][\ol{\alpha := \beta}]
  \;(\mu \phi.\lambda {\overline \alpha}.H){\overline \beta}}
\]
in $\rho$ is trivial.



\item
$\underline{\Gamma; \emptyset~|~\emptyset \vdash \fold^F_H :
  \Nat^\emptyset\;(\Nat^{\ol{\beta}, \ol{\gamma}}\,H[\phi
    :=_{\ol{\beta}} F][\ol{\alpha := \beta}]\,F)\;
  (\Nat^{{\ol{\beta},\ol{\gamma}} }\,(\mu \phi.\lambda \overline
  \alpha.H)\overline \beta \;F)}$ \; Since $\Phi$ is empty, to see
  that $\oCPOsem{ \Gamma; \emptyset~|~\emptyset \vdash \fold^F_H :
    \Nat^\emptyset\;(\Nat^{\ol{\beta}, \ol{\gamma}}\,H[\phi
      :=_{\ol{\beta}} F][\ol{\alpha := \beta}]\,F)\;
    (\Nat^{{\ol{\beta},\ol{\gamma}} }\,(\mu \phi.\lambda \overline
    \alpha.H)\overline \beta \;F)}$ is a natural transformation
  $\oCPOsem{\Gamma;\emptyset \vdash \emptyset}$ to \[\oCPOsem{\Gamma;
    \emptyset \vdash \Nat^\emptyset\;(\Nat^{\ol{\beta},
      \ol{\gamma}}\,H[\phi :=_{\ol{\beta}} F][\ol{\alpha :=
        \beta}]\,F)\; (\Nat^{{\ol{\beta},\ol{\gamma}} }\,(\mu
    \phi.\lambda \overline \alpha.H)\overline \beta\,F)}\] we need only
  show that, for all $\rho : \oCPOenv$, the unique $d :
  \oCPOsem{\Gamma;\emptyset \vdash \emptyset} \rho$, and all $\eta :
  \oCPOsem{\Gamma; \emptyset \vdash \Nat^{\ol{\beta},
      \ol{\gamma}}\,H[\phi :=_{\ol{\beta}} F][\ol{\alpha :=
        \beta}]\,F} \rho$,
\[ \oCPOsem{\Gamma; \emptyset~|~\emptyset \vdash \fold^F_H :
  \Nat^\emptyset\;(\Nat^{\ol{\beta}, \ol{\gamma}}\,H[\phi
    :=_{\ol{\beta}} F][\ol{\alpha := \beta}]\,F)\;
  (\Nat^{{\ol{\beta},\ol{\gamma}} }\,(\mu \phi.\lambda \overline
  \alpha.H)\overline \beta\,F)}\,\rho\,d\,\eta\] has type
$\oCPOsem{\Gamma; \emptyset \vdash \Nat^{{\ol{\beta},\ol{\gamma}}
  }\,(\mu \phi.\lambda \overline \alpha.H)\overline \beta\,F}\,\rho$
i.e., for any $\ol{B}$ and $\ol{C}$,
\[(\oCPOsem{\Gamma; \emptyset~|~\emptyset \vdash \fold^F_H :
  \Nat^\emptyset\;(\Nat^{\ol{\beta}, \ol{\gamma}}\,H[\phi
    :=_{\ol{\beta}} F][\ol{\alpha := \beta}]\,F)\;
  (\Nat^{{\ol{\beta},\ol{\gamma}} }\,(\mu \phi.\lambda \overline
  \alpha.H)\overline \beta\,F)}\,\rho\,d\,\eta)_{\ol{B}\,\ol{C}}\] is a
morphism from $\oCPOsem{\Gamma; \ol{\beta},\ol{\gamma} \vdash (\mu
  \phi.\lambda \overline \alpha.H)\overline \beta}\rho[\ol{\beta :=
    B}][\ol{\gamma := C}] \,=\,(\mu T^\oCPO_{H,\rho[\ol{\gamma := C}]})
\ol{B}$\\ to $\oCPOsem{\Gamma; \ol{\beta},\ol{\gamma} \vdash
  F}\rho[\ol{\beta := B}][\ol{\gamma := C}]$.  To see this, note
%use Equations~\ref{eq:subs-var} and~\ref{eq:subs-const} to verify
that $\eta$ is a natural transformation from
\[\begin{array}{ll}
 & \lambda \ol{B}\,\ol{C}.\,\oCPOsem{\Gamma; \ol{\beta},\ol{\gamma}
  \vdash H[\phi := F][\ol{\alpha := \beta}]}\rho[\ol{\beta :=
    B}][\ol{\gamma := C}]\\
%= & \lambda
%\ol{B}\,\ol{C}.\,\oCPOsem{\Gamma;\ol{\beta},\ol{\gamma},\ol{\alpha}  
%  \vdash H[\phi := F]}\rho[\ol{\beta := B}][\ol{\gamma :=
%    C}][\ol{\alpha := B}]\\ 
%= & \lambda \ol{B}\,\ol{C}.\,\oCPOsem{\Gamma;
%  \ol{\beta},\ol{\gamma},\ol{\alpha},\phi \vdash H}\rho
%[\ol{\beta := B}][\ol{\gamma := C}][\ol{\alpha := B}]\\
% & \hspace*{0.5in}[\phi := \lambda \ol{A}.\, \oCPOsem{\Gamma;\ol{\beta},
%    \ol{\gamma},\ol{\alpha} \vdash F}\rho[\ol{\beta := B}][\ol{\gamma
%      := C}][\ol{\alpha := B}][\ol{\beta := A}]]\\ 
%= & \lambda \ol{B}\,\ol{C}.\, \oCPOsem{\Gamma;
%  \ol{\gamma},\ol{\alpha},\phi \vdash H}\rho[\ol{\gamma :=
%    C}][\ol{\alpha := B}][\phi := \lambda \ol{A}.\,
%  \oCPOsem{\Gamma;\ol{\beta}, 
%    \ol{\gamma}\vdash F}\rho[\ol{\gamma := C}][\ol{\beta := A}]]\\ 
= & \lambda \ol{B}\,\ol{C}.\,T^\oCPO_{H,\rho[\ol{\gamma:=
     C}]}\,(\lambda \ol{A}. \, \oCPOsem{\Gamma;\ol{\beta},\ol{\gamma} 
  \vdash F}\rho[\ol{\beta := A}][\ol{\gamma := C}]) \, \ol{B}
\end{array}\]
to
\[\begin{array}{ll}
 & \lambda \ol{B}\,\ol{C}.\,(\lambda
\ol{A}.\,\oCPOsem{\Gamma;\ol{\beta},\ol{\gamma} \vdash F}\rho[\ol{\beta
    := A}][\ol{\gamma := C}]) \ol{B}\\
= & \lambda
\ol{B}\,\ol{C}.\,\oCPOsem{\Gamma;\ol{\beta},\ol{\gamma} \vdash
  F}\rho[\ol{\beta := B}][\ol{\gamma := C}]
\end{array}\]
and thus
%if $x :
%\oCPOsem{\Gamma; \ol{\beta},\ol{\gamma} \vdash (\mu \phi.\lambda
%  \overline \alpha.H)\overline \beta}\rho[\ol{\beta := B}][\ol{\gamma
%    := C}]\;=\;(\mu T^\oCPO_{\rho[\ol{\gamma := C}]}) \ol{B}$, then
%\[\begin{array}{ll}
%  & (\oCPOsem{\Gamma; \emptyset~|~\emptyset \vdash \fold^F_H :
%  \Nat^\emptyset\;(\Nat^{\ol{\beta}, \ol{\gamma}}\,H[\phi
%    :=_{\ol{\beta}} F][\ol{\alpha := \beta}]\,F)\;
%  (\Nat^{{\ol{\beta},\ol{\gamma}} }\,(\mu \phi.\lambda \overline
%  \alpha.H)\overline \beta\,F}\,\rho\,d\,\eta)_{\ol{B}\,\ol{C}}\,x\\
%= & (\mathit{fold}_{T^\oCPO_{\rho[{\gamma := C}]}}\,(\lambda
%\ol{A}.\,\eta_{\ol{A}\,\ol{C}}))_{\ol{B}}\,x\\  
%:& (\lambda \ol{A}.\,\oCPOsem{\Gamma;\ol{\beta},\ol{\gamma} \vdash
%  F}\rho[\ol{\beta := A}][\ol{\gamma := C}])\ol{B}
%\end{array}\]
%i.e.,
for each $\ol{B}$ and $\ol{C}$,
\[(\oCPOsem{\Gamma; \emptyset~|~\emptyset \vdash \fold^F_H :
  \Nat^\emptyset\;(\Nat^{\ol{\beta}, \ol{\gamma}}\,H[\phi
    :=_{\ol{\beta}} F][\ol{\alpha := \beta}]\,F)\;
  (\Nat^{{\ol{\beta},\ol{\gamma}} }\,(\mu \phi.\lambda \overline
  \alpha.H)\overline \beta\,F)}\,\rho\,d\,\eta)_{\ol{B}\,\ol{C}}\] is a
morphism from
$\oCPOsem{\Gamma; \ol{\beta},\ol{\gamma} \vdash (\mu
  \phi.\lambda \overline \alpha.H)\overline \beta}\rho[\ol{\beta :=
    B}][\ol{\gamma := C}]\,=\, (\mu T^\oCPO_{H,\rho[\ol{\gamma := C}]})
\ol{B}$ to\\ $\oCPOsem{\Gamma;\ol{\beta},\ol{\gamma} \vdash
  F}\rho[\ol{\beta := B}][\ol{\gamma := C}]$.

To see that this family of morphisms is natural in $\ol{B}$ and
$\ol{C}$, we observe that the following diagram commutes for all
$\ol{f : B \to B'}$ and $\ol{g : C \to C'}$:
{\tiny
\[\begin{tikzcd}[column sep=2.5in, row sep=0.75in]
(\mu T^\oCPO_{H,\rho[\ol{\gamma := C}]})\, \ol{B}
\ar[d, "{(\mu T^\oCPO_{H,\id_\rho[\ol{\gamma := g}]}) \, \ol{B}}"'] \ar[r, 
  "{(\mathit{fold}_{T^\oCPO_{H,\rho[\ol{\gamma := C}]}}\,(\lambda
    \ol{A}.\,\eta_{\ol{A}\,\ol{C}}))_{\ol{B}}}"] 
& \oCPOsem{\Gamma; \ol{\beta},\ol{\gamma} \vdash F}\rho[\ol{\gamma :=
    C}][\ol{\beta := B}]\ar[d, "{\oCPOsem{\Gamma;
      \ol{\beta},\ol{\gamma} \vdash F}\id_\rho[\ol{\gamma := 
    g}][\ol{\beta := \id_B}]}"]\\
(\mu T^\oCPO_{H,\rho[\ol{\gamma := C'}]})\, \ol{B} 
\ar[r,"{(\mathit{fold}_{T^\oCPO_{H,\rho[\ol{\gamma :=
            C'}]}}\,(\lambda \ol{A}.\,\eta_{\ol{A}\,\ol{C'}}))_{\ol{B}}\,}" ] 
\ar[d, "{(\mu T^\oCPO_{H,\rho[\ol{\gamma := C'}]}) \, \ol{f}}"'] & 
 \oCPOsem{\Gamma; \ol{\beta},\ol{\gamma} \vdash F}\rho[\ol{\gamma :=
    C'}][\ol{\beta := B}]\ar[d, "{\oCPOsem{\Gamma;
      \ol{\beta},\ol{\gamma} \vdash F}\id_\rho[\ol{\gamma := 
    \id_{C'}}][\ol{\beta := f}]}"]\\
 (\mu T^\oCPO_{H,\rho[\ol{\gamma := C'}]})\, \ol{B'} 
\ar[r,"{(\mathit{fold}_{T^\oCPO_{H,\rho[\ol{\gamma :=
            C'}]}}\,(\lambda \ol{A}.\,\eta_{\ol{A}\,\ol{C'}}))_{\ol{B'}}\,}"
] &  
 \oCPOsem{\Gamma; \ol{\beta},\ol{\gamma} \vdash F}\rho[\ol{\gamma :=
    C'}][\ol{\beta := B'}]
\end{tikzcd}\]}
Indeed, naturality of $\mathit{fold}_{T^\oCPO_{H,\rho[\ol{\gamma :=
        C'}]}}\,(\lambda \ol{A}.\,\eta_{\ol{A}\,\ol{C'}})$ ensures that
the bottom diagram commutes. To see that the top one commutes
%is considerably more delicate.
%\vspace*{0.1in}
% To see that the top diagram commutes
we first observe that, given a natural transformation $\Theta : H \to
K : [\oCPO^k, \oCPO] \to [\oCPO^k, \oCPO]$, the fixpoint natural
transformation $\mu \Theta : \mu H \to \mu K : \oCPO^k \to \oCPO$ is
defined to be $\mathit{fold}_{H}(\Theta\,(\mu K) \circ
\mathit{in}_{K})$, i.e., the unique morphism making the following
diagram commute:\label{page:dia1}
{\footnotesize
\[\begin{tikzcd}[column sep = large]
H(\mu H)
	\ar[dd, "{\mathit{in}_H}"']
	\ar[r, "{H(\mu \Theta)}"]
& H(\mu K)
	\ar[d, "{\Theta (\mu K)}"] \\
& K(\mu K)
	\ar[d, "{\mathit{in}_K}"] \\
\mu H
	\ar[r, "{\mu \Theta}"']
& \mu K
\end{tikzcd}\]}
Taking $\Theta = T^{\oCPO}_{H,f}: T^{\oCPO}_{H,\rho} \to
T^{\oCPO}_{H,\rho'}$ thus gives that, for any $f : \rho \to \rho'$ in
$\oCPOenv$,
\begin{equation}\label{eq:mu-sigma-def}
\mathit{in}_{T^{\oCPO}_{H,\rho'}} \circ 
T^{\oCPO}_{H,f} (\mu T^{\oCPO}_{H,\rho'}) \circ 
T^{\oCPO}_{H,\rho}(\mu T^{\oCPO}_{H,f}) \,=\, 
\mu T^{\oCPO}_{H,f} \circ \mathit{in}_{T^{\oCPO}_{H,\rho}}
\end{equation}
Next, note that the action of the functor
$\lambda \ol{B}. \lambda \ol{C}. \oCPOsem{\Gamma; \ol{\beta},
  \ol{\gamma} \vdash H[\phi := F][\ol{\alpha := \beta}]}\rho
        [\ol{\beta := B}] [\ol{\gamma := C}]$
on the morphisms $\ol{f : B \to B'}, \ol{g : C \to C'}$ is given by
\[\begin{array}{ll}
 & \oCPOsem{\Gamma; \ol{\beta}, \ol{\gamma} \vdash H[\phi :=
      F][\ol{\alpha := \beta}]} \id_{\rho} [\ol{\beta := f}]
           [\ol{\gamma := g}]\\
= & \oCPOsem{\Gamma; \phi, \ol{\alpha}, \ol{\gamma} \vdash H}
\id_{\rho}[\ol{\alpha := f}] [\ol{\gamma := g}][\phi := \lambda
  \ol{A}. \oCPOsem{\Gamma; \ol{\beta}, \ol{\gamma} \vdash F} \id_{\rho
    [\ol{\beta := A}]} [\ol{\gamma := g}]] \\ 
=& \oCPOsem{\Gamma; \phi, \ol{\alpha}, \ol{\gamma} \vdash H} \id_{\rho
  [\ol{\gamma := C'}][\phi := \lambda \ol{A}. \oCPOsem{\Gamma;
      \ol{\beta}, \ol{\gamma} \vdash F} \rho [\ol{\beta := A}]
    [\ol{\gamma := C'}]]} [\ol{\alpha := f}] \\
&\hspace{3em} \circ \oCPOsem{\Gamma; \phi, \ol{\alpha}, \ol{\gamma}
  \vdash H} \id_{\rho [\ol{\alpha := B}][\phi := \lambda
    \ol{A}. \oCPOsem{\Gamma; \ol{\beta}, \ol{\gamma} \vdash F} \rho
       [\ol{\beta := A}] [\ol{\gamma := C'}]]} [\ol{\gamma := g}] \\
&\hspace{3em} \circ \oCPOsem{\Gamma; \phi, \ol{\alpha}, \ol{\gamma}
  \vdash H} \id_{\rho [\ol{\alpha := B}] [\ol{\gamma := C}]} [\phi :=
  \lambda \ol{A}. \oCPOsem{\Gamma; \ol{\beta}, \ol{\gamma} \vdash F}
  \id_{\rho [\ol{\beta := A}]} [\ol{\gamma := g}]] \\
= & T^{\oCPO}_{H,\rho [\ol{\gamma := C'}]} (\lambda
\ol{A}. \oCPOsem{\Gamma; \ol{\beta}, \ol{\gamma} \vdash F} \rho
   [\ol{\beta := A}] [\ol{\gamma := C'}]) \ol{f} \\
&\hspace{3em} \circ \big( T^{\oCPO}_{H,\id_{\rho}[\ol{\gamma := g}]}
   (\lambda \ol{A}. \oCPOsem{\Gamma; \ol{\beta}, \ol{\gamma} \vdash F}
   \rho [\ol{\beta := A}] [\ol{\gamma := C'}]) \big)_{\ol{B}} \\
&\hspace{3em} \circ \big( T^{\oCPO}_{H,\rho [\ol{\gamma := C}]} (\lambda
   \ol{A}. \oCPOsem{\Gamma; \ol{\beta}, \ol{\gamma} \vdash F} \id_{\rho
     [\ol{\beta := A}]} [\ol{\gamma := g}]) \big)_{\ol{B}}
\end{array}\]
So if $\eta$ is a natural transformation such that $\eta_{\ol B, \ol
  C}$ has type
\[
%\lambda \ol{B}\,\ol{C}.
\oCPOsem{\Gamma; \ol{\beta},
  \ol{\gamma} \vdash H[\phi := F][\ol{\alpha := \beta}]}\rho
[\ol{\beta := B}] [\ol{\gamma := C}]
%\]          to \[
\to
%\lambda \ol{B}\, \ol{C}.
\oCPOsem{\Gamma; \ol{\beta},
   \ol{\gamma} \vdash F}\rho [\ol{\beta := B}] [\ol{\gamma := C}]\]
then, by naturality,
\[\begin{array}{ll}
 & \oCPOsem{\Gamma; \ol{\beta}, \ol{\gamma} \vdash F} \id_{\rho}
       [\ol{\beta := f}] [\ol{\gamma := g}] \circ \eta_{\ol{B},
         \ol{C}} \\ 
%= & \eta_{\ol{B'}, \ol{C'}} \circ \oCPOsem{\Gamma; \ol{\alpha},
%  \ol{\gamma} \vdash H[\phi := F][\ol{\alpha := \beta}]} \id_{\rho}
%[\ol{\beta := f}] [\ol{\gamma := g}] \\ 
= & \eta_{\ol{B'}, \ol{C'}} \circ T^{\oCPO}_{H,\rho [\ol{\gamma := C'}]}
(\lambda \ol{A}. \oCPOsem{\Gamma; \ol{\beta}, \ol{\gamma} \vdash F}
\rho [\ol{\beta := A}] [\ol{\gamma := C'}]) \ol{f} \\
& \circ \big( T^{\oCPO}_{H,\id_{\rho}[\ol{\gamma := g}]} (\lambda
\ol{A}. \oCPOsem{\Gamma; \ol{\beta}, \ol{\gamma} \vdash F} \rho
   [\ol{\beta := A}] [\ol{\gamma := C'}]) \big)_{\ol{B}} \\
& \circ \big( T^{\oCPO}_{H,\rho [\ol{\gamma := C}]} (\lambda
   \ol{A}. \oCPOsem{\Gamma; \ol{\beta}, \ol{\gamma} \vdash F} \id_{\rho
     [\ol{\beta := A}]} [\ol{\gamma := g}]) \big)_{\ol{B}}
\end{array}\]
As a special case when $\ol{f = \id_B}$ we have
\begin{equation}\label{eq:T-sigma-functor}
\begin{split}
  & \lambda \ol{B}. \oCPOsem{\Gamma; \ol{\beta}, \ol{\gamma} \vdash F}
\id_{\rho [\ol{\beta := B}]} [\ol{\gamma := g}] \circ \lambda
\ol{B}.\eta_{\ol{B}, \ol{C}} \\  
=\;\;\; & \lambda \ol{B}.\eta_{\ol{B}, \ol{C'}} \circ
T^{\oCPO}_{H,\id_{\rho}[\ol{\gamma := g}]} (\lambda
\ol{A}. \oCPOsem{\Gamma; \ol{\beta}, \ol{\gamma} \vdash F} \rho
   [\ol{\beta := A}] [\ol{\gamma := C'}]) \\
 & \hspace*{0.5in} \circ
   T^{\oCPO}_{H,\rho [\ol{\gamma := C}]} (\lambda \ol{A}. \oCPOsem{\Gamma;
     \ol{\beta}, \ol{\gamma} \vdash F} \id_{\rho [\ol{\beta := A}]}
   [\ol{\gamma := g}]) 
\end{split}
\end{equation}
Finally, to see that the top diagram in the diagram on
page~\pageref{page:dia1} commutes we first note that functoriality of
$T^{\oCPO}_{H,\rho [\ol{\gamma := C}]}$, naturality of
$T^{\oCPO}_{H,\id_{\rho}[\ol{\gamma := g}]}$, the universal property of
$\mathit{fold}_{T^{\oCPO}_{H,\rho [\ol{\gamma := C'}]}} (\lambda
\ol{A}. \eta_{\ol{A}, \ol{C'}})$ and ~\eqref{eq:mu-sigma-def}
ensure that the following diagram commutes: {\footnotesize
\begin{equation}\label{eq:one-side}
  \begin{tikzcd}[column sep = huge, row sep = huge]
T^{\oCPO}_{H,\rho [\ol{\gamma := C}]} (\mu T^{\oCPO}_{H,\rho [\ol{\gamma :=
      C}]}) \ar[rr, "{T^{\oCPO}_{H,\rho [\ol{\gamma := C}]} (
    \mathit{fold}_{T^{\oCPO}_{H,\rho [\ol{\gamma := C'}]}} (\lambda
    \ol{A}. \eta_{\ol{A}, \ol{C'}}) \circ \mu
    T^{\oCPO}_{H,\id_{\rho}[\ol{\gamma := g}]} )}"] \ar[dd,
    "{\mathit{in}_{T^{\oCPO}_{H,\rho [\ol{\gamma := C}]}}}"']
&& T^{\oCPO}_{H,\rho [\ol{\gamma := C}]} (\lambda \ol{B}. \oCPOsem{\Gamma;
  \ol{\beta}, \ol{\gamma} \vdash F} \rho [\ol{\beta := B}] [\ol{\gamma
    := C'}]) \ar[d, "{ T^{\oCPO}_{H,\id_{\rho}[\ol{\gamma := g}]}
    (\lambda \ol{B}. \oCPOsem{\Gamma; \ol{\beta}, \ol{\gamma} \vdash F}
    \rho [\ol{\beta := B}] [\ol{\gamma := C'}]) }" description] \\
&& T^{\oCPO}_{H,\rho [\ol{\gamma := C'}]} (\lambda
\ol{B}. \oCPOsem{\Gamma; \ol{\beta}, \ol{\gamma} \vdash F} \rho
   [\ol{\beta := B}][\ol{\gamma := C'}]) \ar[d, "{ \lambda \ol{A}. \eta_{\ol{A}, \ol{C'}}
     }" description] \\
\mu T^{\oCPO}_{H,\rho [\ol{\gamma := C}]}\ar[r, "{\mu
    T^{\oCPO}_{H,\id_{\rho}[\ol{\gamma := g}]}}"'] 
&\mu T^{\oCPO}_{H,\rho [\ol{\gamma := C'}]} \ar[r, bend right = 10,
  "{\mathit{fold}_{T^{\oCPO}_{H,\rho [\ol{\gamma := C'}]}} (\lambda
    \ol{A}. \eta_{\ol{A}, \ol{C'}})}"']
& \lambda \ol{B}. \oCPOsem{\Gamma; \ol{\beta}, \ol{\gamma} \vdash F}
\rho [\ol{\beta := B}] [\ol{\gamma := C'}]
  \end{tikzcd}
  \end{equation}}
Next, we note that functoriality of $T^{\oCPO}_{H,\rho [\ol{\gamma :=
      C}]}$, ~\eqref{eq:T-sigma-functor}, and the universal
property of $\mathit{fold}_{T^{\oCPO}_{H,\rho [\ol{\gamma := C}]}} (\lambda
\ol{A}. \eta_{\ol{A}, \ol{C}})$ ensure that the following diagram
commutes: {\footnotesize
\begin{equation}\label{eq:other-side}
\begin{tikzcd}[row sep = huge]
T^{\oCPO}_{H,\rho [\ol{\gamma := C}]} (\mu T^{\oCPO}_{H,\rho [\ol{\gamma :=
      C}]}) \ar[rr, "{T^{\oCPO}_{H,\rho [\ol{\gamma := C}]} (
    \lambda \ol{B}.\oCPOsem{\Gamma; \ol{\beta}, \ol{\gamma} \vdash F} \id_{\rho
      [\ol{\beta := B}]} [\ol{\gamma := g}] \circ
    \mathit{fold}_{T^{\oCPO}_{H,\rho [\ol{\gamma := C}]}} (\lambda
    \ol{A}. \eta_{\ol{A}, \ol{C}}) )}"] \ar[dd, "{\mathit{in}_{T^{\oCPO}_{H,\rho
        [\ol{\gamma := C}]}}}"']
&& T^{\oCPO}_{H,\rho [\ol{\gamma := C}]} (\lambda \ol{B}.\oCPOsem{\Gamma; \ol{\beta},
  \ol{\gamma} \vdash F} \rho [\ol{\beta := B}] [\ol{\gamma := C'}])
\ar[d, "{ T^{\oCPO}_{H,\id_{\rho}[\ol{\gamma := g}]}
    (\lambda \ol{B}.\oCPOsem{\Gamma; \ol{\beta}, \ol{\gamma} \vdash F} \rho [\ol{\beta
        := B}] [\ol{\gamma := C'}]) }" description]\\
&& T^{\oCPO}_{H,\rho [\ol{\gamma := C'}]} (\lambda \ol{B}.\oCPOsem{\Gamma; \ol{\beta},
  \ol{\gamma} \vdash F} \rho [\ol{\beta := B}][\ol{\gamma := C'}]) \ar[d, "{ \lambda
    \ol{A}. \eta_{\ol{A}, \ol{C'}} }" description] \\
\mu T^{\oCPO}_{H,\rho [\ol{\gamma := C}]} \ar[r, bend right = 10,
  "{\mathit{fold}_{T^{\oCPO}_{H,\rho [\ol{\gamma := C}]}} (\lambda
    \ol{A}. \eta_{\ol{A}, \ol{C}})}"'] & \lambda \ol{B}.\oCPOsem{\Gamma; \ol{\beta},
  \ol{\gamma} \vdash F} \rho [\ol{\beta := B}] [\ol{\gamma := C}]
\ar[r, bend right = 10, "{\lambda \ol{B}.\oCPOsem{\Gamma; \ol{\beta}, \ol{\gamma}
      \vdash F} \id_{\rho [\ol{\beta := B}]} [\ol{\gamma := g}]}"'] &
\lambda \ol{B}.\oCPOsem{\Gamma; \ol{\beta}, \ol{\gamma} \vdash F} \rho [\ol{\beta :=
    B}] [\ol{\gamma := C'}]
\end{tikzcd}
\end{equation}}
Combining the equations entailed by ~\eqref{eq:one-side}
and ~\eqref{eq:other-side}, we get that
the top diagram in the diagram on
  page~\pageref{page:dia1} commutes, as desired.
To see that, for all $\rho : \oCPOenv$, $d \in
\oCPOsem{\Gamma; \emptyset \vdash \emptyset}\rho$, and $\eta :
\oCPOsem{\Gamma; \emptyset \vdash \Nat^{\ol{\beta},
    \ol{\gamma}}\,H[\phi :=_{\ol{\beta}} F][\ol{\alpha := \beta}]\,F}
\rho$,
\[\oCPOsem{\Gamma; \emptyset~|~\emptyset
  \vdash \fold^F_H : \Nat^\emptyset\;(\Nat^{\ol{\beta},
    \ol{\gamma}}\,H[\phi :=_{\ol{\beta}} F][\ol{\alpha :=
      \beta}]\,F)\; (\Nat^{{\ol{\beta},\ol{\gamma}} }\,(\mu
  \phi.\lambda \overline \alpha.H)\overline \beta\,F)}\,\rho\,d\,\eta\]
satisfies the additional condition needed for it to be in
$\oCPOsem{\Gamma;\emptyset \vdash \Nat^{{\ol{\beta},\ol{\gamma}}
  }\,(\mu \phi.\lambda \overline \alpha.H)\overline \beta\;F}\,\rho$,
let $\ol{R : \oCPOR(B,B')}$ and $\ol{S : \oCPOR(C,C')}$.  Since $\eta$
satisfies the additional condition needed for it to be in
$\oCPOsem{\Gamma; \emptyset \vdash \Nat^{\ol{\beta},
    \ol{\gamma}}\,(H[\phi := F][\ol{\alpha := \beta}])\,F} \rho$,
\[\begin{array}{ll}
 & (\,(\mathit{fold}_{T^\oCPO_{H,\rho[\ol{\gamma :=
        C}]}}\,(\lambda \ol{A}.\,\eta_{\ol{A}\,\ol{C}}))_{\ol{B}},\,
(\mathit{fold}_{T^\oCPO_{H,\rho[\ol{\gamma :=
        C'}]}}\,(\lambda \ol{A}.\eta_{\ol{A}\,\ol{C'}}))_{\ol{B'}}\,) 
\end{array}\]
has type
\[\begin{array}{ll}
  & (\mu T_{H,\Eq_\rho[\ol{\gamma := S}]}) \,\ol{R} \to
\oCPORsem{\Gamma;\ol{\gamma},\ol{\beta} \vdash F}\Eq_\rho[\ol{\gamma := 
    S}][\ol{\beta:= R}]\\ 
= & (\mu T_{H,\Eq_\rho[\ol{\gamma :=
      S}]})\,\ol{\oCPORsem{\Gamma;\ol{\gamma},\ol{\beta} 
  \vdash \beta}\Eq_\rho[\ol{\gamma := S}][\ol{\beta := R}]} \to
\oCPORsem{\Gamma;\ol{\gamma},\ol{\beta} \vdash F}\Eq_\rho[\ol{\gamma := 
    S}][\ol{\beta:= R}]\\ 
= & \oCPORsem{\Gamma; \ol{\gamma},\ol{\beta} \vdash (\mu \phi. \lambda
  \ol{\alpha}. H)\ol{\beta}}\Eq_\rho[\ol{\gamma := S}][\ol{\beta := R}] \to
\oCPORsem{\Gamma;\ol{\gamma},\ol{\beta} \vdash F}\Eq_\rho[\ol{\gamma := 
    S}][\ol{\beta:= R}]
\qedhere
\end{array}\]


\end{itemize}
\end{proof}


\end{document}


