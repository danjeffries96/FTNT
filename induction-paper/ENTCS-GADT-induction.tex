\documentclass[9pt]{entcs}
\usepackage{entcsmacro}
\usepackage{graphicx}
\usepackage{adjustbox}
\usepackage{amsmath}
\usepackage[utf8]{inputenc}
\usepackage{ccicons}
\usepackage{verbatim}
\usepackage{amscd}
\usepackage{MnSymbol}
\usepackage{xcolor}
\usepackage{bbold}
\usepackage{url}
\usepackage{upgreek}
\usepackage{stmaryrd}
\usepackage{lipsum}
\usepackage{tikz-cd}
\usetikzlibrary{cd}
\usetikzlibrary{calc}
\usetikzlibrary{arrows}
\usepackage{bussproofs}
\EnableBpAbbreviations

\sloppy
% The following is enclosed to allow easy detection of differences in
% ascii coding.
% Upper-case    A B C D E F G H I J K L M N O P Q R S T U V W X Y Z
% Lower-case    a b c d e f g h i j k l m n o p q r s t u v w x y z
% Digits        0 1 2 3 4 5 6 7 8 9
% Exclamation   !           Double quote "          Hash (number) #
% Dollar        $           Percent      %          Ampersand     &
% Acute accent  '           Left paren   (          Right paren   )
% Asterisk      *           Plus         +          Comma         ,
% Minus         -           Point        .          Solidus       /
% Colon         :           Semicolon    ;          Less than     <
% Equals        =3D           Greater than >          Question mark ?
% At            @           Left bracket [          Backslash     \
% Right bracket ]           Circumflex   ^          Underscore    _
% Grave accent  `           Left brace   {          Vertical bar  |
% Right brace   }           Tilde        ~

\bibliographystyle{entcs}

\def\lastname{Johann, Ghiorzi, and Jeffries}

\DeclareMathAlphabet{\mathpzc}{OT1}{pzc}{m}{it}

%\usepackage[amsmath]{ntheorem}

\newcommand{\lam}{\lambda}
\newcommand{\eps}{\varepsilon}
\newcommand{\ups}{\upsilon}
\newcommand{\mcB}{\mathcal{B}}
\newcommand{\mcD}{\mathcal{D}}
\newcommand{\mcE}{\mathcal{E}}
\newcommand{\mcF}{\mathcal{F}}
\newcommand{\mcP}{\mathcal{P}}
\newcommand{\mcI}{\mathcal{I}}
\newcommand{\mcJ}{\mathcal{J}}
\newcommand{\mcK}{\mathcal{K}}
\newcommand{\mcL}{\mathcal{L}}
\newcommand{\WW}{\mathcal{W}}

\newcommand{\ex}{\mcE_x}
\newcommand{\ey}{\mcE_y}
\newcommand{\bzero}{\boldsymbol{0}}
\newcommand{\bone}{{\boldsymbol{1}}}
\newcommand{\tB}{{\bone_\mcB}}
\newcommand{\tE}{{\bone_\mcE}}
\newcommand{\bt}{\mathbf{t}}
\newcommand{\bp}{\mathbf{p}}
\newcommand{\bsig}{\mathbf{\Sigma}}
\newcommand{\bpi}{\boldsymbol{\pi}}
\newcommand{\Empty}{\mathtt{Empty}}
\newcommand{\truthf}{\mathtt{t}}
\newcommand{\id}{id}
\newcommand{\coo}{\mathtt{coo\ }}
\newcommand{\mcC}{\mathcal{C}}
\newcommand{\Rec}{\mathpzc{Rec}}
\newcommand{\types}{\mathcal{T}}

%\newcommand{\Homrel}{\mathsf{Hom_{Rel}}}
\newcommand{\HomoCPOR}{\mathsf{Hom_{\oCPOR}}}

%\newcommand{\semof}[1]{\llbracket{#1}\rrbracket^\rel}
\newcommand{\sem}[1]{\llbracket{#1}\rrbracket}
\newcommand{\setsem}[1]{\llbracket{#1}\rrbracket^\set}
\newcommand{\relsem}[1]{\llbracket{#1}\rrbracket^\rel}
\newcommand{\dsem}[1]{\llbracket{#1}\rrbracket^{\mathsf D}}
\newcommand{\setenv}{\mathsf{SetEnv}}
\newcommand{\relenv}{\mathsf{RelEnv}}
\newcommand{\oCPOenv}{\mathsf{SetEnv}}
\newcommand{\oCPORenv}{\mathsf{RelEnv}}
\newcommand{\oCPOsem}[1]{\llbracket{#1}\rrbracket^{\set}}
\newcommand{\oCPORsem}[1]{\llbracket{#1}\rrbracket^{\rel}}
\newcommand{\denv}{\mathsf{DEnv}}

\newcommand{\rel}{\mathsf{Rel}}
\newcommand{\setof}[1]{\{#1\}}
\newcommand{\letin}[1]{\texttt{let }#1\texttt{ in }}
\newcommand{\comp}[1]{{\{#1\}}}
\newcommand{\bcomp}[1]{\{\![#1]\!\}}
\newcommand{\beps}{\boldsymbol{\eps}}
%\newcommand{\B}{\mcB}
%\newcommand{\Bo}{{|\mcB|}}

\newcommand{\lmt}{\longmapsto}
\newcommand{\RA}{\Rightarrow}
\newcommand{\LA}{\Leftarrow}
\newcommand{\rras}{\rightrightarrows}
\newcommand{\colim}[2]{{{\underrightarrow{\lim}}_{#1}{#2}}}
\newcommand{\lift}[1]{{#1}\,{\hat{} \; \hat{}}}
\newcommand{\graph}[1]{\langle {#1} \rangle}

\newcommand{\carAT}{\mathsf{car}({\mathcal A}^T)}
\newcommand{\isoAto}{\mathsf{Iso}({\mcA^\to})}
\newcommand{\falg}{\mathsf{Alg}_F}
\newcommand{\CC}{\mathsf{Pres}(\mathcal{A})}
\newcommand{\PP}{\mathcal{P}}
\newcommand{\DD}{D_{(A,B,f)}}
\newcommand{\from}{\leftarrow}
\newcommand{\upset}[1]{{#1}{\uparrow}}
\newcommand{\smupset}[1]{{#1}\!\uparrow}

\newcommand{\Coo}{\mathpzc{Coo}}
\newcommand{\code}{\#}
\newcommand{\nat}{\mathpzc{Nat}}

\newcommand{\eq}{\; = \;}
\newcommand{\of}{\; : \;}
\newcommand{\df}{\; := \;}
\newcommand{\bnf}{\; ::= \;}

\newcommand{\zmap}[1]{{\!{\between\!\!}_{#1}\!}}
\newcommand{\bSet}{\mathbf{Set}}

\newcommand{\dom}{\mathsf{dom}}
\newcommand{\cod}{\mathsf{cod}}
\newcommand{\adjoint}[2]{\mathrel{\mathop{\leftrightarrows}^{#1}_{#2}}}
\newcommand{\isofunc}{\mathpzc{Iso}}
\newcommand{\ebang}{{\eta_!}}
\newcommand{\lras}{\leftrightarrows}
\newcommand{\rlas}{\rightleftarrows}
\newcommand{\then}{\quad\Longrightarrow\quad}
\newcommand{\hookup}{\hookrightarrow}

\newcommand{\spanme}[5]{\begin{CD} #1 @<#2<< #3 @>#4>> #5 \end{CD}}
\newcommand{\spanm}[3]{\begin{CD} #1 @>#2>> #3\end{CD}}
\newcommand{\pushout}{\textsf{Pushout}}
\newcommand{\mospace}{\qquad\qquad\!\!\!\!}

\newcommand{\natur}[2]{#1 \propto #2}

\newcommand{\Tree}{\mathsf{Tree}\,}
\newcommand{\GRose}{\mathsf{GRose}\,}
\newcommand{\List}{\mathsf{List}\,}
\newcommand{\PTree}{\mathsf{PTree}\,}
\newcommand{\Bush}{\mathsf{Bush}\,}
\newcommand{\Forest}{\mathsf{Forest}\,}
\newcommand{\Lam}{\mathsf{Lam}\,}
\newcommand{\LamES}{\mathsf{Lam}^+}
\newcommand{\Expr}{\mathsf{Expr}\,}

\newcommand{\ListNil}{\mathsf{Nil}}
\newcommand{\ListCons}{\mathsf{Cons}}
\newcommand{\LamVar}{\mathsf{Var}}
\newcommand{\LamApp}{\mathsf{App}}
\newcommand{\LamAbs}{\mathsf{Abs}}
\newcommand{\LamSub}{\mathsf{Sub}}
\newcommand{\ExprConst}{\mathsf{Const}}
\newcommand{\ExprPair}{\mathsf{Pair}}
\newcommand{\ExprProj}{\mathsf{Proj}}
\newcommand{\ExprAbs}{\mathsf{Abs}}
\newcommand{\ExprApp}{\mathsf{App}}
\newcommand{\Ptree}{\mathsf{Ptree}}

\newcommand{\kinds}{\mathpzc{K}}
\newcommand{\tvars}{\mathbb{T}}
\newcommand{\fvars}{\mathbb{F}}
\newcommand{\consts}{\mathpzc{C}}
\newcommand{\Lan}{\mathsf{Lan}}
\newcommand{\zerot}{\mathbb{0}}
\newcommand{\onet}{\mathbb{1}}
\newcommand{\bool}{\mathbb{2}}
\renewcommand{\nat}{\mathbb{N}}
%\newcommand{\semof}[1]{[\![#1]\!]}
%\newcommand{\setsem}[1]{\llbracket{#1}\rrbracket^\set}
\newcommand{\predsem}[1]{\llbracket{#1}\rrbracket^\pred}
%\newcommand{\todot}{\stackrel{.}{\to}}
\newcommand{\todot}{\Rightarrow}
\newcommand{\bphi}{{\bm \phi}}

\newcommand{\bm}[1]{\boldsymbol{#1}}

\newcommand{\cL}{\mathcal{L}}
\newcommand{\T}{\mathcal{T}}
\newcommand{\Pos}{P\!}
%\newcommand{\Pos}{\mathcal{P}\!}
\newcommand{\Neg}{\mathcal{N}}
\newcommand{\Hf}{\mathcal{H}}
\newcommand{\V}{\mathbb{V}}
\newcommand{\I}{\mathcal{I}}
\newcommand{\Set}{\mathsf{Set}}
%\newcommand{\Nat}{\mathsf{Nat}}
\newcommand{\Homrel}{\mathsf{Hom_{Rel}}}
\newcommand{\CV}{\mathcal{CV}}
\newcommand{\lan}{\mathsf{Lan}}
\newcommand{\Id}{\mathit{Id}}
\newcommand{\mcA}{\mathcal{A}}
\newcommand{\inl}{\mathsf{inl}}
\newcommand{\inr}{\mathsf{inr}}
%\newcommand{\case}[3]{\mathsf{case}\,{#1}\,\mathsf{of}\,\{{#2};\,{#3}\}}
\newcommand{\tin}{\mathsf{in}}
\newcommand{\fold}{\mathsf{fold}}
\newcommand{\Eq}{\mathsf{Eq}}
\newcommand{\Hom}{\mathsf{Hom}}
\newcommand{\curry}{\mathsf{curry}}
\newcommand{\uncurry}{\mathsf{uncurry}}
\newcommand{\eval}{\mathsf{eval}}
\newcommand{\apply}{\mathsf{apply}}
\newcommand{\oCPO}{{\mathsf{Set}}}
\newcommand{\oCPOR}{{\mathsf{Rel}}}
\newcommand{\oCPORT}{{\mathsf{RT}}}

\newcommand{\ar}[1]{\##1}
\newcommand{\mcG}{\mathcal{G}}
\newcommand{\mcH}{\mathcal{H}}
\newcommand{\TV}{\mathpzc{V}}

\newcommand{\essim}[1]{\mathsf{EssIm}(#1)}
\newcommand{\hra}{\hookrightarrow}

\newcommand{\ol}[1]{\overline{#1}}
\newcommand{\ul}[1]{\underline{#1}}
\newcommand{\op}{\mathsf{op}}
\newcommand{\trige}{\trianglerighteq}
\newcommand{\trile}{\trianglelefteq}
\newcommand{\LFP}{\mathsf{LFP}}
\newcommand{\LAN}{\mathsf{Lan}}
%\newcommand{\Mu}{{\mu\hskip-4pt\mu}}
\newcommand{\Mu}{{\mu\hskip-5.5pt\mu}}
%\newcommand{\Mu}{\boldsymbol{\upmu}}
\newcommand{\Terms}{\mathpzc{Terms}}
\newcommand{\Ord}{\mathpzc{Ord}}
\newcommand{\Anote}[1]{{\color{blue} {#1}}}
\newcommand{\Pnote}[1]{{\color{red} {#1}}}

\newcommand{\greyout}[1]{{\color{gray} {#1}}}
\newcommand{\ora}[1]{\overrightarrow{#1}}


\begin{document}
\begin{frontmatter}
  \vspace*{-0.1in}
  \title{(Deep) Induction for GADTs\vspace*{-0.1in}}
  \author{Patricia Johann~~~~}%\thanksref{JohannEmail}}
  \author{Enrico Ghiorzi~~~~}%\thanksref{GhiorziEmail}}
  \author{Daniel Jeffries}%\thanksref{JeffriesEmail}}
%  \thanks[JohannEmail]{Email: \href{mailto:johannp@appstate.edu} {\texttt{\normalshape johannp@appstate.edu}}}
%  \thanks[GhiorziEmail]{Email: \href{mailto:ghiorzie@appstate.edu} {\texttt{\normalshape ghiorzie@appstate.edu}}}
%  \thanks[JeffriesEmail]{Email: \href{mailto:jeffriesd@appstate.edu} {\texttt{\normalshape jeffriesd@appstate.edu}}}
  \address{$\mathtt{\{johannp,ghiorzie,jeffriesd\}@appstate.edu}$\\
    Appalachian State University}

\vspace*{-0.1in}

  \begin{abstract} 
Deep data types are data types that are defined in terms of other such
data types, including, in the case of truly nested types, themselves.
Deep induction is an extension of structural induction that traverses
{\em all} of the structure present in a structure of such a type,
propagating suitable predicates to {\em all} of the data contained in
that structure. Deep induction has been shown to be the form of
induction most suitable for applications involving deep nested
types. In this paper we show how to extend deep induction to a robust
class of deep GADTs that are not truly nested. We also show that it
cannot be extended to truly nested GADTs.
\end{abstract}

%\begin{keyword}
%  Please list keywords from your paper here, separated by commas.
%\end{keyword}

\end{frontmatter}

\vspace*{-0.1in}

\section{Introduction}\label{sec:intro}

%\vspace*{-0.09in}

Induction is one of the most important techniques available for
working with advanced data types, so it is both inevitable and
unsurprising that it plays an essential role in modern proof
assistants. In the proof assistant Coq~\cite{coq20}, for example,
functions and predicates over advanced types are defined inductively,
and almost all non-trivial proofs of their properties are either
proved by induction outright or rely on lemmas that are. Every time a
new inductive data type is declared in Coq, an induction rule is
automatically generated for it.

The data types handled by Coq are (possibly mutually inductive)
polynomial ADTs, and the induction rules automatically generated for
them are the expected ones for standard structural induction. It has
long been understood, however, that these rules are too weak to be
genuinely useful for so-called {\em deep ADTs}~\cite{jp20}, i.e., ADTs
that are (possibly mutually inductively) defined in terms of (other)
such ADTs.\footnote{Such data types are called nested inductive types
  by Chlipala~\cite{chl}, reflecting the fact that ``inductive type''
  means ``ADT''~in~Coq.} Consider, for example, the following type of
rose trees, here coded in Agda and defined in terms of the standard
type of lists:

\vspace*{-0.075in}

\[\begin{array}{l}
\mathsf{data\, Rose\, : Set \to Set\,where}\\
\mathsf{\;\;\;\;\;\;\;\;empty\, :\, Rose\,A}\\
\mathsf{\;\;\;\;\;\;\;\;node\,\,\,\, :\, A \to List\,(Rose\,A) \to Rose\,A} 
\end{array}\]
\noindent
The induction rule Coq automatically generates for rose trees is

\vspace*{-0.075in}

\[\begin{array}{l}
\mathsf{\forall\, (a : Set)\,(P : Rose\,a \to Set) \to P\,empty \to}\\
\mathsf{\hspace*{0.15in}
 (\forall\, (x : a)\,(ts :
  List\,(Rose\,a)) \to P\,(node\,x\,ts)) \to 
  \forall \,(x : Rose\,a) \to P\,x}
\end{array}\]
\noindent
Unfortunately, this is neither the induction rule we intuitively
expect, nor is it expressive enough to prove even basic properties of
rose trees that ought to be amenable to inductive proof. What is
needed here is an enhanced notion of induction that, when specialized
to rose trees, will propagate the predicate $\mathsf{P}$ through the
outer list structure and to the rose trees sitting inside
$\mathsf{node}$'s list argument. More generally, this enhanced notion
of induction should traverse {\em all} of the structure present in a
data element, propagating suitable predicates to {\em all} of the data
contained in the structure. With data types becoming ever more
advanced, and with deeply structured such types becoming ever more
%increasingly
ubiquitous in formalizations, it is critically important that proof
assistants
%like Coq, Agda, and Lean 
be able to automatically generate genuinely useful induction
rules for data types that go well beyond traditional ADTs. Such
data types include (truly) nested types~\cite{bm98}\footnote{A truly
  nested type is a nested type that is defined over itself.},
generalized algebraic data types
(GADTs)~\cite{ch03,pvww06,sp04,xcc03}, more richly indexed
families~\cite{ch88}, and deep variants of all of these.

\pagebreak

{\em Deep induction}~\cite{jp20} is a generalization of structural
induction that fits this bill exactly. Whereas structural induction
rules induct over only the top-level structure of data, leaving any
data internal to the top-level structure untouched, deep induction
rules induct over {\em all} of the structured data present. The key
idea is to parameterize induction rules not just over a predicate over
the top-level data type being considered, but also over additional
custom predicates on the types of primitive data they contain. These
custom predicates are then lifted to predicates on any internal
structures containing these data, and the resulting predicates on
these internal structures are lifted to predicates on any internal
structures containing structures at the previous level, and so on,
until the internal structures at all levels of the data type
definition, including the top level, have been so
processed. Satisfaction of a predicate by the data at one level of a
structure is then conditioned upon satisfaction of the appropriate
predicates by {\em all} of the data at the preceding level.

Deep induction was shown in~\cite{jp20} to be the form of induction
most appropriate to nested types (including ADTs) that are defined
over, or mutually recursively with, other such types (including,
possibly, themselves). Deep induction delivers the following genuinely
useful induction rule for rose trees:

\vspace*{-0.075in}

\begin{equation}\label{eq:rose}
\begin{array}{l}
\mathsf{\forall \,(a : Set)\,(P : Rose\,a \to Set)\,(Q : a \to Set)
  \to P\,empty \to}\\ 
\mathsf{\hspace*{0.15in}(\forall \,(x : a)\, (ts :
  List\,(Rose\,a))\to Q\,x \to List^\land\,P\,ts \to P\,(node\,x\,ts)) \to}\\
\mathsf{\hspace*{0.15in}\forall \,(x :
  Rose\,a) \to Rose^\land\,Q\, x \to P\,x} 
\end{array}
\end{equation}

\noindent
%Here, $\mathsf{List^\land}$ lifts its predicate argument $\mathsf{P}$
%on data of type $\mathsf{Rose\,a}$ to a predicate on data of type
%$\mathsf{List\,(Rose\,a)}$ asserting that $\mathsf{P}$ holds for every
%element of its argument list, and similarly for the lifting
%$\mathsf{Rose^\land}$ and the predicate $\mathsf{Q}$.
Here, $\mathsf{List^\land}$ (resp., $\mathsf{Rose^\land}$) lifts its
predicate argument $\mathsf{P}$ (resp., $\mathsf{Q}$) on data of type
$\mathsf{Rose\,a}$ (resp., $\mathsf{a}$) to a predicate on data of
type $\mathsf{List\,(Rose\,a)}$ (resp., $\mathsf{Rose\,a}$) asserting
that $\mathsf{P}$ (resp., $\mathsf{Q}$) holds for every element of its
list (resp., rose tree) argument.\footnote{Predicate liftings such as
  $\mathsf{List^\land}$ and $\mathsf{Rose^\land}$ can either be
  supplied as primitives or generated automatically from their
  associated data type definitions as described in
  Section~\ref{sec:ADTs-and-nesteds} below. The predicate lifting for
  a container type like $\mathsf{List\,t}$ or $\mathsf{Rose\,t}$
  simply traverses containers of that type and applies its predicate
  argument pointwise to the constituent data of type $\mathsf{t}$.}
Deep induction was also shown in~\cite{jp20} to
% be the missing piece making it possible to
%be the theretofore missing component needed to
% be the essential ingredient making it possible to
deliver the first-ever induction rules --- structural or otherwise ---
for the $\mathsf{Bush}$ data type~\cite{bm98} and other truly nested
types. Deep induction for ADTs and nested types is reviewed in
Section~\ref{sec:ADTs-and-nesteds} below.

This paper shows how to extend deep induction to proper GADTs, i.e.,
to GADTs that are not simply nested types (and thus are not ADTs).  A
constructor for such a GADT $\mathsf{G}$ may, like a constructor for a
nested type, take as arguments data whose types involve instances of
$\mathsf{G}$ other than the one being defined --- including instances
that involve $\mathsf{G}$ itself. But if $\mathsf{G}$ is a proper GADT
then at least one of its constructors will also have such a structured
instance of $\mathsf{G}$ --- albeit one not involving $\mathsf{G}$
itself --- as its codomain.
%But while the types of the arguments to a GADT's constructor can
%include instances of $\mathsf{G}$ that themselves involve
%$\mathsf{G}$, its return type cannot.
For example, the constructor $\mathsf{pair}$ for the GADT {\color{red}
  Perhaps also show non-inhabitation?}

\vspace*{-0.1in}

\begin{equation}\label{eq:seq}
\begin{array}{l}
\mathsf{data\, Seq\,(a : Set) : Set\,where}\\
\mathsf{\;\;\;\;\;\;\;\;const :\, a \to Seq\,a}\\
\mathsf{\;\;\;\;\;\;\;\;\;\;\,pair :\, Seq \,a \to Seq\,b \to
  Seq\,(a \times b)}
\end{array}
\end{equation}
\noindent
of sequences only constructs sequences of pairs, rather than sequences
of arbitrary type, as does $\mathsf{const}$. If all of the
constructors for a GADT $\mathsf{G}$ return structured instances of
$\mathsf{G}$, then some of $\mathsf{G}$'s instances might not be
inhabited. GADTs therefore have two distinct, but equally natural,
semantics: a functorial semantics interpreting them as left Kan
extensions~\cite{mac71}, and a parametric semantics interpreting them
as their Church encodings~\cite{atk12,vw10}. As explained
in~\cite{jgj21}, a key difference in the two semantics is that the
former views GADTs as their {\em functorial completions}~\cite{jp19},
and thus as containing more data than just those expressible in
syntax. By contrast, the latter views them as what might be called
{\em syntax-only} GADTs. Happily, these two views of GADTs coincide
for those that are ADTs or other nested types. However, both they and
their attendant properties differ greatly for proper GADTs. In fact,
the views deriving from the functorial and parametric semantics for
proper GADTs are sufficiently distinct that, by contrast with the
situation for ADTs and other nested
types~\cite{bfss90,gjfor15,jgj21f}, it is not actually possible to
define a functorial parametric semantics for them~\cite{jgj21}.

This observation seems, at first, to be a death knell for the prospect
of extending deep induction to GADTs. Indeed, since induction can be
seen as unary parametricity, we quickly realize that GADTs viewed as
their functorial completions cannot possibly support induction rules.
This makes sense intuitively: induction is a syntactic proof
technique, so of course it cannot be used to prove properties of those
elements of a GADT's functorial completion that are not expressible in
syntax. All is not lost, however. As we show below, the Church
encoding interpretation's syntax-only view does support induction
rules --- including deep induction rules --- for GADTs. Perhaps
surprisingly, ours are the first-ever induction rules --- deep or
otherwise --- for a general class of proper GADTs. But this paper
actually delivers more: it gives a general framework for deriving deep
induction rules for a general class of deep GADTs directly from their
syntax. This framework can serve as a basis for extending modern proof
assistants' automatic generation of structural induction rules for
ADTs to automatic generation of deep induction rules for GADTs. As for
ADTs and other nested types, the structural induction rule for any
GADT can be recovered from its deep induction rule simply by taking
the custom predicates in its deep induction rule to be constantly
$\mathsf{True}$-valued predicates.

Deep induction rules for GADTs cannot, however, be derived by somehow
extending the techniques of~\cite{jp20} to syntax-only GADTs. Indeed,
the derivation of induction rules given there makes crucial use of the
functoriality of data types' interpretations from~\cite{jp19}, and
that is precisely what the interpretation of GADTs as their Church
encodings fails to deliver. Instead, we first give a predicate lifting
styled after those of~\cite{jp20}, together with a (deep) induction
rule, and for the simplest --- and arguably most important --- GADT,
namely the equality GADT. (See Section~\ref{sec:ind-equal}.) We can
then derive the deep induction rule for any other GADT $\mathsf{G}$ by
{\em i}) using the equality GADT to represent $\mathsf{G}$ as its
so-called {\em Henry Ford
  encoding}~\cite{ch03,hin03,mcb99,sjsv09,sp04}, and {\em ii}) using
the predicate liftings for the equality GADT and any other GADTs
appearing in the definition of $\mathsf{G}$ to appropriately thread
the custom predicates for the primitive types appearing in
$\mathsf{G}$ through its structure. This two-step process delivers
deep induction rules for a broad class of deep GADTs. In
Section~\ref{sec:GADTs} we introduce a series of increasingly complex
GADTs as running examples, and in Section~\ref{sec:deep-ind-GADTs} we
derive a deep induction rule for each of them. In particular, we
derive the deep induction rule for $\mathsf{Seq}$ in
Section~\ref{sec:ind-seq}. We present our general framework for
deriving (deep) induction rules for (deep) GADTs in
Section~\ref{sec:framework}, and observe that the derivations in
Section~\ref{sec:deep-ind-GADTs} are all instances of it.  In
Section~\ref{sec:GADT-nested} we show that, by contrast with truly
nested types, which do have a functorial semantics, syntax-only GADTs'
lack of functoriality means that it is not possible to extend
induction --- deep or otherwise --- to truly nested GADTs. This does
not appear to be much of a restriction, however, since GADTs defined
over themselves do not, to our knowledge, appear in applications or
the literature.

All of the deep induction rules appearing in this paper have been
derived using our general framework. Our Agda code implementing them
is available at~\cite{web-page}.

%\subsection{Related Work}

\vspace*{0.05in}

{\bf Related Work\/} Various techniques for deriving induction rules
for data types that go beyond ADTs have been studied. For example, Fu
and Selinger~\cite{fs18} show, via examples, how to derive induction
rules for arbitrary nested types. Unfortunately, however, their
technique is rather {\em ad hoc}, so is unclear how to generalize it
to nested types other than the specific ones in the
examples. Moreover, it actually derives induction rules for data types
{\em related} to the original nested types rather than for the
original nested types themselves, and it is unclear whether or not the
derived rules are sufficiently expressive to prove all results about
the original nested types that we would expect to be provable by
induction. This latter point echoes the issue with Coq-derived
induction rule for rose trees raised in Section~\ref{sec:intro}, which
has the unfortunate effect of forcing users to manually write
induction (and other) rules for such types for use in that
system. Tassi~\cite{tas19} has done exactly that, deriving induction
rules for data type definitions in Coq using unary
parametricity. Tassi's technique seems to be essentially equivalent to
that of~\cite{jp19} for nested types, although he does not permit true
nesting. More recently, Ulrich~\cite{ull20} has implemented a plugin
in MetaCoq to generate induction rules for nested types. This plugin
is also based on unary parametricity and, again, true nesting is not
permitted.  As far as we know, no attempt has yet been made to extend
either implementation to GADTs. In fact, we know of no work other than
that reported here that specifically addresses induction rules for
(deep) GADTs.

\section{Deep induction for ADTs and nested types}\label{sec:ADTs-and-nesteds}

A structural induction rule for a data type allows us to prove that if
a predicate holds for every element inductively produced by the data
type's constructors then it holds for every element of the data type.
In this paper, we are interested in induction rules for proof-relevant
predicates.  A proof-relevant predicate on a type $\mathsf{A : Set}$
is a function $\mathsf{P\,:\,A \to Set}$ mapping each $\mathsf{a : A}$
to the set of proofs that $\mathsf{P\,a}$ holds.  For example, the
induction rule for the standard list type

\vspace*{-0.1in}

\begin{equation*}\label{eq:list}
\begin{array}{l}
\mathsf{data\ List : Set \to Set\ where}\\
\mathsf{\;\;\;\;\;\;\;nil\,\,\,\,\,\, :\, List\,A}\\
\mathsf{\;\;\;\;\;\;\;cons\, :\, A \to List\,A \to List\,A} 
\end{array}
\end{equation*}

\vspace*{-0.1in}

\noindent
is

\vspace*{-0.2in}

\begin{equation*}
\mathsf{
\forall (A : Set) (P : List\,A \to Set)
\to P\,nil
\to \big( \forall (a : A) (as: List\,A)
\to P\,as
\to P\,(cons\,a\,as)\big)
\to \forall (as : List\,A)
\to P\, as
}
\end{equation*}

\vspace*{-0.1in}

\noindent
As in Coq's induction rule for rose trees, the data inside a
structure of type $\mathsf{List}$ is treated monolithically (i.e.,
ignored) by this structural induction rule.
By contrast, the deep induction rule for lists is parameterized over a
custom predicate $\mathsf{Q}$ on $\mathsf{A}$ as described in the
introduction. For $\mathsf{List^\wedge}$ as described in the
introduction it is

\vspace*{-0.2in}

\[\begin{array}{l}
\mathsf{\forall (A : Set) (P : List\, A \to Set) (Q : A \to Set)
\to P\,Nil \to \big( \forall (a : A) (as: List\,A) \to Q\,a \to P\,as
\to P\,(Cons\,a\,as)\big)} \\ 
\quad\mathsf{\to \forall (as : List\,A) \to List^{\wedge}\,A\,Q\,as
  \to P\,as } 
\end{array}\]

\vspace*{-0.1in}

Structural induction can be extended to nested types, such as the
following type of perfect trees~\cite{bm98}:

\pagebreak

\begin{equation*}\label{eq:ptree}
\begin{array}{l}
\mathsf{data\ PTree : Set \to Set\ where}\\
\mathsf{\;\;\;\;\;\;\;pleaf\,\,\, :\, A \to PTree\,A}\\
\mathsf{\;\;\;\;\;\;\;pnode\, :\, PTree\,(A \times A) \to PTree\,A} 
\end{array}
\end{equation*}
Perfect trees can be thought of as lists constrained to have lengths
that are powers of 2. In the above code, the constructor
$\mathsf{pnode}$ uses data of type $\mathsf{PTree\,(A \times A)}$ to
construct data of type $\mathsf{PTree\,A}$. Thus, it is clear that the
instances of $\mathsf{PTree}$ at various indices cannot be defined
independently, and that the entire inductive family of types must
therefore be defined at once. This intertwinedness of the instances of
nested types is reflected in their structural induction rules, which,
as explained in~\cite{jp20}, must necessarily involve polymorphic
predicates rather than the monomorphic predicates appearing in
structural induction rules for ADTs. The structural induction rule for
perfect trees, for example, is

\vspace*{-0.2in}

\[\begin{array}{l}
\mathsf{\forall (P : \forall (A : Set) \to PTree\, A \to Set)
\to \big( \forall (A : Set) (a : A) \to P\,A\,(pleaf\, a) \big)} \\
\quad\mathsf{\to \big( \forall (A : Set) (tt : PTree\,(A \times A))
  \to P\,(A \times A)\,tt \to P\,a\,(pnode\,tt)\big) \to \forall (A :
  Set) (t : PTree\,A) \to P\,A\,t } 
\end{array}\]

\vspace*{-0.1in}

\noindent
The deep induction rule for perfect trees similarly uses polymorphic
predicates but otherwise follows the now-familiar pattern:

\vspace*{-0.2in}

\[\begin{array}{l}
\mathsf{\forall (P : \forall (A : Set) \to (A \to Set) \to PTree\,A
  \to Set) \to \big( \forall (A : Set) (Q : A \to Set) (a : A) \to
  Q\,a \to P\,A\,Q\,(Pleaf\, a) \big)} \\ \quad \mathsf{\to \big(
  \forall (A : Set) (Q : A \to Set) (tt : PTree\,(A \times A)) \to
  P\,(A \times A)\,(Pair^{\wedge}\,A\,A\,Q\,Q)\,tt \to
  P\,A\,Q\,(Pnode\,tt)\big)} \\ \quad \mathsf{\to \forall (A : Set) (Q
  : A \to Set) (t : PTree\,A) \to PTree^{\wedge}\,A\,Q\,t \to
  P\,A\,Q\,t }
\end{array}\]

\vspace*{-0.1in}

\noindent
Here, $\mathsf{Pair^{\wedge} : \forall (A\; B: Set) \to (A \to Set)
  \to (B \to Set) \to A \times B \to Set}$ lifts predicates
$\mathsf{Q_A}$ on data of type $\mathsf{A}$ and $\mathsf{Q_B}$ on data
of type $\mathsf{B}$ to a predicate on pairs of type $\mathsf{A \times
  B}$ in such a way that $\mathsf{Pair^{\wedge}\,A\,B\,Q_A\,Q_B\,(a,b)
 = Q_A\,a \times Q_B\,b}$. Similarly, $\mathsf{PTree^{\wedge} :
  \forall (A : Set) \to (A \to Set) \to PTree\,A \to Set}$ lifts a
predicate $\mathsf{Q}$ on data of type $\mathsf{A}$ to a predicate on
data of type $\mathsf{PTree\,A}$ asserting that $\mathsf{Q}$ holds for
every element of type $\mathsf{A}$ contained in its perfect tree
argument.

It is not possible to extend structural induction to {\em truly}
nested types, i.e., to nested types whose recursive occurrences appear
below themselves. The quintessential example of such a type is that of
bushes~\cite{bm98}:
\begin{equation*}\label{eq:bush}
\begin{array}{l}
\mathsf{data\ Bush : Set \to Set\ where}\\
\mathsf{\;\;\;\;\;\;bnil\,\,\,\,\,\, :\, Bush\,A}\\
\mathsf{\;\;\;\;\;\;bcons\, :\, A \to Bush\,(Bush\,A) \to Bush\,A} 
\end{array}
\end{equation*}

\noindent
Even defining a structural induction rule for bushes requires that we
be able to lift the rule's polymorphic predicate argument to
$\mathsf{Bush}$ itself. The more general observation that an induction
rule for any truly nested type must therefore necessarily be a deep
induction rule was, in fact, the original motivation for the
development of deep induction in~\cite{jp20}. The deep induction rule
for bushes is
\[\begin{array}{l}
\mathsf{\forall (P : \forall (A : Set) \to (A \to Set) \to Bush\, A \to Set)
\to \big( \forall (A : Set) \to P\,A\,bnil \big)} \\ 
\quad\mathsf{\to \big( \forall (A : Set) (Q : A \to Set) (a : A) (bb :
  Bush\,(Bush\,A)) \to Q\,a \to
  P\,(Bush\,A)\,(Bush^{\wedge}\,A\,Q)\,bb \to P\,A\,Q\,(bcons\,a\,bb)
  \big)} \\ 
\quad\mathsf{\to \forall (A : Set) (Q : A \to Set) (b : Bush\,A) \to
  Bush^{\wedge}\,A\,Q\,b \to P\,A\,Q\,b } 
\end{array}\]
Here, $\mathsf{Bush^{\wedge} : \forall (A : Set) \to (A \to Set) \to
  Bush\,A \to Set}$ is the following lifting of a predicate
$\mathsf{Q}$ on data of type $\mathsf{A}$ to a predicate on data of
type $\mathsf{Bush\,A}$ asserting that $\mathsf{Q}$ holds for every
element of type $\mathsf{A}$ contained in its argument bush:
\begin{align*}
\mathsf{Bush^{\wedge}\,A\,Q\,bnil} &= \mathsf{\top} \\
\mathsf{Bush^{\wedge}\,A\,Q\,(bcons\,a\,bb)} &= \mathsf{Q\,a \times
  Bush^{\wedge}\,(Bush\,A)\,(Bush^{\wedge}\,A\,Q)\,bb}
\end{align*}

Although a truly nested type admits only a single induction rule, it
is worth noting that for those nested types that do admit distinct
structural induction and deep induction rules, the latter generalizes
the former. Indeed, the structural induction rule for such a nested
type is recoverable from its deep induction rule by taking the custom
predicates on its data of primitive types to be constantly
$\mathsf{True}$-valued predicates. This instantiation ensures that the
resulting induction rule only inspects the top-level structure of its
argument, rather than the contents of that structure, which exactly
coindices with what structural induction should do.
%A concrete example of such a derivation is given in
%Section~\ref{sec:ind-equal}.

\section{(Deep) GADTs}\label{sec:GADTs}

While a data constructor for a nested type can take {\em as arguments}
data whose types involve instances of that type at indices other than
the one being defined, its return type must still be at the (variable)
type instance being defined. For example, every data constructor for
$\mathsf{PTree\,A}$ must return an element of type
$\mathsf{PTree\,A}$, regardless of the instances of $\mathsf{PTree}$
appearing in the types of its arguments. GADTs relax this restriction,
allowing their data constructors both to take as arguments \emph{and
  return as results} data whose types involve instances of them other
than the one being defined. And as with the return type of
$\mathsf{pair}$ in~\eqref{eq:seq}, these instances can be structured.
%Applications of GADTs include generic programming, modeling
%programming languages via higher-order abstract syntax, maintaining
%invariants in data structures, and expressing constraints in embedded
%domain-specific languages. GADTs have also been used, e.g., to
%implement tagless interpreters~\cite{pl04,pr06,pvww06}, to improve
%memory performance~\cite{min15}, and to design APIs~\cite{pen20}.

GADTs are used in precisely those situations in which different
behaviors at different instances of data types are desired. This is
achieved by allowing the programmer to give the type signatures of the
GADT's data constructors independently, and then taking advantage of
pattern matching to force the desired type refinement. For example,
the {\em equality} GADT
\begin{equation}\label{eq:equal}
\begin{array}{l}
\mathsf{data\ Equal : Set \to Set \to Set\ where}\\
\mathsf{\;\;\;\;\;\;\;\;\;\;\;\,refl :\, Equal\,A\,A}
\end{array}
\end{equation}
is parametrized by two type indices, but it is only possible to
construct data elements of type $\mathsf{Equal\,a\,b}$ if $\mathsf{a}$
and $\mathsf{b}$ are instantiated at the same type. If the types
$\mathsf{a}$ and $\mathsf{b}$ are syntactically identical then the
type $\mathsf{Equal\,a\,b}$ contains the single data element
$\mathsf{refl}$. It contains no data elements otherwise.

The importance of the equality GADT lies in the fact that we can
understand other GADTs in terms of it. For example, the GADT
$\mathsf{Seq}$ from~\eqref{eq:seq} comprises constant sequences of
data of any type $\mathsf{A}$ and sequences obtained by pairing the
data in two already existing sequences. This GADT can be rewritten as
its Henry Ford encoding, which makes critical use of the equality
GADT, as follows:
\begin{equation}\label{eq:eq_seq}
\begin{array}{l}
\mathsf{data\ Seq : Set \to Set\ where}\\
\mathsf{\;\;\;\;\;\;\;const :\, A \to Seq\,A}\\ 
\mathsf{\;\;\;\;\;\;\;\;pair\,\, :\, \forall (B\,C : Set) \to
  Equal\,A\,(B \times C) \to Seq\,B \to Seq\,C \to Seq\,A}\\ 
\end{array}
\end{equation}
Here, the requirement that $\mathsf{pair}$ produce data at an instance
of $\mathsf{Seq}$ that is a product type is replaced with the
requirement that $\mathsf{pair}$ produce data at an instance of
$\mathsf{Seq}$ that is \emph{equal} to a product type. As we will see
in Section~\ref{sec:deep-ind-GADTs}, the presence of the equality GADT is
key to deriving deep induction rules for GADTs.

Neither $\mathsf{Equal}$ nor $\mathsf{Seq}$ is a deep GADT, but the
following GADT $\mathsf{LTerm}$, which encodes terms of a simply typed
lambda calculus, is. More robust variations on $\mathsf{LTerm}$ are,
of course, possible. But since this variation is rich enough to
illustrate all essential aspects of deep GADTs --- and later, in
Section~\ref{sec:ind-lam}, their deep induction rules --- while still
being small enough to ensure clarity of exposition, we keep it to a
minimum.

Types are either booleans, arrow types, or list types. They are
represented by the Henry Ford GADT
\begin{equation}\label{eq:eq_ltype}
\begin{array}{l}
\mathsf{data\ LType : Set \to Set\ where}\\
\mathsf{\;\;\;\;\;\;\;\;bool :\, \forall (B : Set) \to Equal\,A\,Bool
  \to LType\,A}\\ 
\mathsf{\;\;\;\;\;\;\;\;arr\,\,\, :\, \forall (B\,C : Set) \to
  Equal\,A\,(B \to C) \to LType\,B \to LType\,C \to LType\,A}\\ 
  \mathsf{\;\;\;\;\;\;\;\;list\,\, :\, \forall (B : Set) \to
    Equal\,A\,(\List\,B) \to LType\,B \to LType\,A} 
\end{array}
\end{equation}
Terms are either variables, abstractions, applications, or lists of
terms. They are represented by
\begin{equation}\label{eq:eq_lterm}
\begin{array}{l}
\mathsf{data\ LTerm : Set \to Set\ where}\\
\mathsf{\;\;\;\;\;\;\;\;var\,\,\,\,\,\,\,:\, String \to LType\,A \to
  LTerm\,A} \\  
\mathsf{\;\;\;\;\;\;\;\;abs\,\,\,\,\,\, :\, \forall (B\,C : Set) \to
  Equal\,A\,(B \to C) \to String \to LType\,B \to LTerm\,C \to
  LTerm\,A}\\ 
  \mathsf{\;\;\;\;\;\;\;\;app\,\,\,\,\, :\, \forall (B : Set) \to
    LTerm (B \to A) \to LTerm\,B \to LTerm\,A} \\ 
  \mathsf{\;\;\;\;\;\;\;\;list\,\,\, :\, \forall (B : Set) \to
    Equal\,A\,(List\,B) \to List\,(LTerm\,B) \to LTerm\,A} 
\end{array}
\end{equation}
The type parameter for $\mathsf{LTerm}$ tracks the types of simply
typed lambda calculus terms. For example, $\mathsf{LTerm\,A}$ is the
type of simply typed lambda terms of type $\mathsf{A}$. Variables are
tagged with their types by the data constructors $\mathsf{var}$ and
$\mathsf{abs}$, whose $\mathsf{LType}$ arguments ensure that their
type tags are legal types. This ensures that all lambda terms
produced by $\mathsf{var}$, $\mathsf{abs}$, $\mathsf{app}$, and
$\mathsf{list}$ are well-typed.  We will revisit these GADTs in
Sections~\ref{sec:deep-ind-GADTs} and~\ref{sec:app}.

\section{(Deep) induction for GADTs}\label{sec:deep-ind-GADTs}

The equality constraints engendered by GADTs' data constructors makes
deriving (deep) induction rules for then more involved than for ADTs
and other nested types. Nevertheless, we show in this section how to
do so. We first illustrate the key components of our approach by
deriving deep induction rules for the three specific GADTs introduced
in Section~\ref{sec:GADTs}. Then, in Section~\ref{sec:framework}, we
abstract these to a general framework that can be applied to any deep
GADT that is not truly nested. As hinted above, the predicate lifting
for the equality GADT plays a central role in deriving both s
tructural and deep induction rules for more general GADTs.
 
\subsection{(Deep) induction for $\mathsf{Equal}$}\label{sec:ind-equal}

To define the (deep) induction rule for any (deep) GADT $\mathsf{G}$
we first need to define a predicate lifting that maps a predicate on a
type $\mathsf{A}$ and to a predicate on $\mathsf{G\,A}$. Such a
predicate lifting $\mathsf{Equal^{\wedge} : \forall (A\,B : Set) \to
  (A \to Set) \to (B \to Set) \to Equal\,A\,B \to Set}$ for
$\mathsf{Equal}$ is defined by
$\mathsf{Equal^{\wedge}\,A\,A\,Q\,Q'\,refl = \forall (a : A) \to
  Equal\,(Q\,a)(Q'\,a)}$.  It takes two predicates on the same type as
input and tests them for extensional equality.
%
Next, we need to associate with each data constructor $\mathsf{c}$ of
$\mathsf{G}$ an {\em induction hypothesis} asserting that, if the
custom predicate arguments to a predicate $\mathsf{P}$ on $\mathsf{G}$
can be lifted to $\mathsf{G}$ itself, then $\mathsf{c}$ {\em respects}
$\mathsf{P}$, i.e., $\mathsf{c}$ constructs data satisfying the
instance of $\mathsf{P}$ at those custom predicates. The following
induction hypothesis $\mathsf{dIndRefl}$ is thus associated with the
$\mathsf{refl}$ constructor for $\mathsf{Equal}$:
\begin{equation*}\label{eq:ind-refl}
\begin{array}{l}
\mathsf{\lambda (P : \forall (A\,B : Set) \to (A \to Set) \to (B \to
  Set) \to Equal\,A\,B \to Set)} \\ 
\quad\mathsf{\to \forall (C : Set) (Q\, Q' : C \to Set) \to
  Equal^{\wedge}\,C\,C\,Q\,Q'\,refl \to P\,C\,C\,Q\,Q'\,refl} 
\end{array}
\end{equation*}

The deep induction rule for $\mathsf{G}$ now states that, if all of
$\mathsf{G}$'s data constructors respect a predicate $\mathsf{P}$,
then $\mathsf{P}$ is satisfied by every element of $\mathsf{G}$ to
which the custom predicate arguments to $\mathsf{P}$ can be
successfully lifted.  The deep induction rule for $\mathsf{Equal}$ is
thus
\begin{equation}\label{eq:ind-equal}
\begin{array}{l}
\mathsf{\forall (P : \forall (A\,B : Set) \to (A \to Set) \to (B \to
  Set) \to Equal\,A\,B \to Set) \to dIndRefl\,P \to}\\ \quad 
  \mathsf{\forall (A\,B : Set) (Q_A : A \to Set) (Q_B : B \to Set) (e:
  Equal\,A\,B) \to Equal^{\wedge}\,A\,B\,Q_A\,Q_B\,e \to
  P\,A\,B\,Q_A\,Q_B\,e}
\end{array}
\end{equation}
To prove that this rule is sound we must provide a witness
$\mathsf{dIndEqual}$ inhabiting the type in~\eqref{eq:ind-equal}.  By
pattern matching, we need only consider the case where $\mathsf{A} =
\mathsf{B}$ and $\mathsf{e} = \mathsf{refl}$, so we can define
$\mathsf{dIndEqual}$ by
$\mathsf{dIndEqual\;P\;crefl\;A\;A\;Q_A\;Q_A'\;refl\;liftE =
  crefl\;A\;Q_A\;Q_A'\,liftE}$. We can recover the structural
induction rule
\begin{equation}\label{eq:sind-equal}
\mathsf{\forall (Q : \forall (A\,B : Set)
  \to Equal\,A\,B \to Set) \to \big( \forall (C : Set) \to
  P\,C\,C\,refl \big) \to \forall (A\,B : Set) (e: Equal\,A\,B) \to
  P\,A\,B\,e}
\end{equation}
for $\mathsf{Equal}$ by defining a term $\mathsf{indEqual}$ of the
type in~\eqref{eq:sind-equal} by $\mathsf{indEqual\;Q\;srefl\;A\;B\;e
  = dIndEqual\;P\;srefl\;A\;B\;K^A_\top\;K^B_\top\;e\,liftE}$. Here,
$\mathsf{e : Equal\,A\,B}$, $\mathsf{P : \forall (A\,B : Set) \to (A
  \to Set) \to (B \to Set) \to Equal\,A\,B \to Set}$ is defined by
$\mathsf{P\;A\;B\;Q_A\;Q_B\;e =}$ $\mathsf{Q\;A\;B\;e}$,
$\mathsf{K^A_\top}$ and $\mathsf{K^B_\top}$ are the constantly
{\color{red} $\mathsf{\top}$}-valued predicates on $\mathsf{A}$ and
$\mathsf{B}$, respectively, and $\mathsf{liftE :}$
$\mathsf{Equal^{\wedge}\;A\;B\;K^A_\top\;K^B_\top\;e}$ is
%fully
defined
%(by pattern matching)
by $\mathsf{liftE\,a = refl : Equal\,A\,A}$ for every $\mathsf{a :
  A}$.  The structural induction rule for any GADT $\mathsf{G}$ that
is not truly nested can similarly be recovered from its deep induction
rule by instantiating every custom predicate by the appropriate
constantly {\color{red}$\mathsf{\top}$}-valued predicate.

\subsection{(Deep) induction for $\mathsf{Seq}$}\label{sec:ind-seq}

To derive the deep induction rule for the GADT $\mathsf{Seq}$ we use
its Henry Ford encoding from~\eqref{eq:eq_seq}.  We first define its
predicate lifting $\mathsf{Seq^\wedge : \forall (A : Set) \to (A \to
  Set) \to Seq\,A \to Set}$ by
\[\begin{array}{lll}
\mathsf{Seq^{\wedge}\,A\,Q_A\,(const\,a)} & = & \mathsf{Q_A\,a}\\
\mathsf{Seq^{\wedge}\,A\,Q_A\,(sPair\,B\,C\,e\,s_B\,s_C)}
&=&\mathsf{\exists [Q_B] \exists [Q_C]\, Equal^{\wedge}\,A\, (B
  \times C)\, Q_A\, (Q_B \times Q_C) \, e \times
  Seq^{\wedge}\,B\,Q_B\,s_B \times Seq^{\wedge}\,C\,Q_C\,s_C}
\end{array}\]
Here, $\mathsf{a : A}$, $\mathsf{Q_B : B \to Set}$, $\mathsf{Q_C : C
  \to Set}$, $\mathsf{e : Equal\,A\,(B \times C)}$, $\mathsf{s_B :
  Seq\,B}$, $\mathsf{s_C : Seq\,C}$, and $\mathsf{\exists [x]\, F
  \,x}$ is syntactic sugar for the type of dependent pairs
$\mathsf{(x,b)}$ where $\mathsf{x : A}$ and $\mathsf{b : F\, x}$ and
$\mathsf{F : A \to Set}$.

Next, let $\mathsf{dIndConst}$ be the induction hypothesis
\[\mathsf{
\lambda (P : \forall (A : Set) \to (A \to Set) \to Seq\,A \to Set) \to
\forall (A : Set) (Q_A : A \to Set) (a : A) \to Q_A\,x \to
P\,A\,Q_A\,(const\,a)}\] associated with the constructor
$\mathsf{const}$, and let $\mathsf{dIndPair}$ be the induction
hypothesis
\[\begin{array}{l}
\mathsf{\lambda (P : \forall (A : Set) \to (A \to Set) \to Seq\,A \to
  Set)} \to \\ 
\quad \mathsf{\forall (A\,B\,C : Set) (Q_A : A \to Set) (Q_B : B
  \to Set) (Q_C : C \to Set)(s_B : Seq\,B) (s_C : Seq\,C) (e :
  Equal\,A\,(B \times C)) \to} \\ 
\quad \mathsf{Equal^{\wedge} A\, (B \times C)\, Q_A\,
  (Pair^{\wedge}\,B\,C\,Q_B\,Q_C)\, e \to P\,B\,Q_B\,s_B \to
  P\,C\,Q_C\,s_C \to P A Q_A ( pair\,B\,C\,e\,s_B\,s_C )}
\end{array}\]
associated with the constructor $\mathsf{pair}$. Then the deep
induction rule for $\mathsf{Seq}$ is
\begin{equation}\label{eq:ind-seq}
\begin{array}{l}
\mathsf{\forall (P : \forall (A : Set) \to (A \to Set) \to Seq\,A \to
  Set)} \mathsf{\to dIndConst\,P \to dIndPair\,P \to} \\ \quad
\mathsf{\forall (A : Set)(Q_A : A \to Set)(s_A : Seq\,A) \to
  Seq^{\wedge}\,A\,Q_A\,s_A \to P\,A\,Q_A\,s_A}
\end{array}
\end{equation}

To prove that this rule is sound we provide a witness
$\mathsf{dIndSeq}$ inhabiting the type in~\eqref{eq:ind-seq} as
follows:
\[\begin{array}{lll}
\mathsf{dIndSeq\;P\;cconst\;cpair\;A\;Q_A\;(const\,a)\;liftA}  &
    = &  \mathsf{cconst\;A\;Q_A\;a\;liftA}\\
\mathsf{dIndSeq\,P\,cconst\,cpair\,A\,Q_A\,(sPair\,B\,C\,e\,s_B\,s_C)\,
    (Q_B,Q_C, liftE, liftB, liftC)}  & = &
\mathsf{cpair\,A\,B\,C\,Q_A\,Q_B\,Q_C\,s_B\,s_C\,e\,liftE\,p_B\,p_C}
\end{array}\]
In the first clause ahove, $\mathsf{a : A}$, $\mathsf{Q_A : A \to
  Set}$, $\mathsf{liftA : Seq^{\wedge}\,A\,Q_A\,(const\,a) =
  Q_A\,a}$. In the second, $\mathsf{Q_B : B \to Set}$, $\mathsf{Q_C :
  C \to Set}$, $\mathsf{e : Equal\,A\,(B \times C)}$, $\mathsf{s_B :
  Seq\,B}$, $\mathsf{s_C : Seq\,C}$, $\mathsf{liftE :
  Equal^{\wedge}\,A\, (B \times C)\, Q_A\, (Q_B \times Q_C) \, e}$,
$\mathsf{liftB : Seq^{\wedge}\,B\,Q_B\,s_B}$, and $\mathsf{liftC :}$
$\mathsf{Seq^{\wedge}\,C\,Q_C\,s_C}$ --- which together ensure that
$\mathsf{(Q_B, Q_C, liftE, liftB, liftC) :
  Seq^{\wedge}\,A\,Q\,(sPair\,B\,C\,e\,s_B\,s_C)}$ --- and
$\mathsf{p_B} =\mathsf{dIndSeq\,P\,cconst\,cpair\,B\,Q_B\,s_B\,liftB :
  P\,B\,Q_B\,s_B}$ and $\mathsf{p_C}
=\mathsf{dIndSeq\,P\,cconst\,cpair\,C\,Q_C\,s_C\,liftC :
  P\,C\,Q_C\,s_C}$.

\subsection{(Deep) induction for $\mathsf{LTerm}$}\label{sec:ind-lam} 

\begin{figure*}[t]

  \begin{adjustbox}{varwidth=6.8in, max width=6.6in, margin=-0.1in 0in
      -0.2in 0in, fbox, center} 
{\small
\[\begin{array}{lll}
\mathsf{LType^{\wedge}\,A\,Q_A\,(bool\,B\,e)} & = &\mathsf{\exists
  [Q_B]\, Equal^{\wedge}\, A\, B\, Q_A\, K^{Bool}_{\top} \,e}\\
\mathsf{LType^{\wedge}\,A\,Q_A\,(arr\, B\, C\, e\, T_B\, T_C)}
&=&\mathsf{\exists [Q_B] \,\exists [Q_c]\, Equal^{\wedge}\,A\,
  (B \to C)\, Q_A\, (Arr^{\wedge} \, B\, C\, Q_B \, Q_C) \, e \times
  \, LType^{\wedge}\,B\,Q_B\,T_B \times LType^{\wedge}\,C\,Q_C\,T_C}\\
\mathsf{LType^{\wedge}\,A\,Q_A\,(list\, B\, e\, T_B)} & = &
\mathsf{\exists [Q_B]\, Equal^{\wedge}\,A\, (List\, B)\, Q_A\,
  (List^{\wedge} \, B\, Q_B) \, e \times LType^{\wedge}\,B\,Q_B\,T_B}\\[1ex]
\mathsf{LTerm^{\wedge}\,A\,Q_A\,(var\,s\,T_A)} & = &
\mathsf{LType^{\wedge}\, A\, Q_A\, T_A}\\
\mathsf{LTerm^{\wedge}\,A\,Q_A\, (abs \,B \,C \,e \,s \,T_B \,t_C)} &
= & \mathsf{\exists [Q_B]\,\exists [Q_C]\, Equal^{\wedge} \, A\, (B \to
  C)\, Q_A\, (Arr^{\wedge} \, B\, C\, Q_B \, Q_C)\, e \times \,
  LType^{\wedge}\, B\, Q_B\, T_B \times \, LTerm^{\wedge}\, C\, Q_C\,
  t_C }\\
\mathsf{LTerm^{\wedge}\,A\,Q_A\, (app\, B\, t_{BA}\, t_B)} & = &
\mathsf{\exists [Q_B]\, LTerm^{\wedge}\, (B \to A)\, (Arr^{\wedge} \,
  B\, A\, Q_B \, Q_A)\, t_{BA} \times LTerm^{\wedge}\, B\, Q_B\,
  t_B}\\
\mathsf{LTerm^{\wedge}\,A\,Q_A\, (list\, B\, e\, ts)} & = &
\mathsf{\exists [Q_B]\, Equal^{\wedge} \, A\, (List\,B)\, Q_A\,
  (List^{\wedge} \, B\, Q_B) \, e \times List^{\wedge}\, (LTerm\,B) \,
  (LTerm^{\wedge} \, B\, Q_B) \, ts}
\end{array}   \]}

\vspace*{-0.1in}

\caption{Predicate liftings for $\mathsf{LType}$ and
  $\mathsf{LTerm}$}\label{fig:liftings} \vspace*{0.1in} 
\end{adjustbox}
\end{figure*}

To derive the deep induction rule for the GADT $\mathsf{LTerm}$ we use
its Henry Ford encoding from~\eqref{eq:eq_ltype}
and~\eqref{eq:eq_lterm}. We first define predicate lifting
$\mathsf{Arr^{\wedge} : \forall (A\, B : Set) \to (A \to Set) \to (B
  \to Set) \to (A \to B) \to Set}$ for arrow types, since arrow types
appear in $\mathsf{LType}$ and $\mathsf{LTerm}$. It is given by
$\mathsf{Arr^{\wedge}\, A\, B\, Q_A\, Q_B\, f = \forall (a : A) \to
  Q_A\,a \to Q_B\, (f\,a)}$. The predicate liftings
$\mathsf{LType^{\wedge} : \forall (A : Set) \to (A \to Set) \to
  LType\,A \to Set}$ for $\mathsf{LType}$ and $\mathsf{LTerm^{\wedge}
  :}$ $\mathsf{\forall (A : Set) \to (A \to Set) \to LTerm\,A \to
  Set}$ for $\mathsf{LTerm}$ are defined in Figure~\ref{fig:liftings}.
There, $\mathsf{s : String}$, $\mathsf{Q_A : A \to Set}$, $\mathsf{Q_B
  : B \to Set}$, $\mathsf{Q_C : C \to Set}$,
$\mathsf{K^{Bool}_{\top}}$ is the constantly {\color{red}
  $\mathsf{\top}$}-valued predicate on $\mathsf{Bool}$, $\mathsf{T_A :
  LType\, A}$, $\mathsf{T_B : LType \,B}$, $\mathsf{T_C : LType \,C}$,
$\mathsf{t_B : LTerm \, B}$, $\mathsf{t_C : LTerm \, C}$, and
$\mathsf{t_{BA} : LTerm \, (B \to A)}$.  Moreover, $\mathsf{e :
  Equal\,A\,Bool}$ in the first clause, $\mathsf{e : Equal\, A\, (B
  \to C)}$ in the second, $\mathsf{e : Equal\, A\, (List\, B)}$ in the
third, $\mathsf{e : Equal \, A \, (B \to C)}$ in the fifth, and
$\mathsf{e : Equal\, A\, (List \,B)}$, $\mathsf{ts : List\, (LTerm
  B)}$, and $\mathsf{List^\wedge}$ is the predicate lifting for lists
from~\eqref{eq:rose} in the seventh.

With these liftings in hand we can define the induction hypotheses
$\mathsf{dIndVar}$, $\mathsf{dIndAbs}$, $\mathsf{dIndApp}$, and
$\mathsf{dIndList}$ associated with $\mathsf{LTerms}$'s data
constructors. These are, respectively,
\[\begin{array}{l}
\mathsf{\lambda (P : \forall (A : Set) \to (A \to Set) \to
  LTerm\,A  \to Set) \to}\\
  \quad\mathsf{\forall (A : Set) (Q_A : A \to Set) (s : String) (T_A :
  LType\, A) \to LType^{\wedge} \, A\, Q_A\, T_A \to P \, A\, Q_A\,
  (var \; s\, T_A)}\\[1ex]
  \mathsf{\lambda (P : \forall (A : Set) \to (A \to Set) \to
  LTerm\,A \to Set)} \\ 
  \quad\mathsf{\to 
  \forall (A\,B\,C: Set) (Q_A : A \to Set)  (Q_B : B \to Set) (Q_C : C
  \to Set) (e : Equal\, A\, (B \to C)) (s : String) } \\ 
  \quad\mathsf{ \to (T_B : LType\, B) \to (t_C : LTerm\, C)
  \to Equal^{\wedge}\,A\,(B \to C)\, Q_A \, (Arr^{\wedge} \, B\, C\,
  Q_B \, Q_C) \, e } \\
  \quad\mathsf{
  \to LType^{\wedge}\, B\, Q_B\, T_B
  \to P\, C\, Q_C\, t_C\, 
  \to P \, A\, Q_A\, (abs \,B \,C \, e \,s \,T_B \, t_C)
  }\\[1ex]
\end{array}\]
\[\begin{array}{l}
  \mathsf{\lambda (P : \forall (A : Set) \to (A \to Set) \to LTerm\,A
    \to Set)} \\ 
  \quad \mathsf{\to 
  \forall (A \,B : Set) (Q_A : A \to Set)  (Q_B : B \to Set) 
   (t_{BA} : LTerm\, (B \to A)) (t_B : LTerm\, B)} \\
  \quad \mathsf{
  \to P\, (B \to A)\, (Arr^{\wedge} \, B\, A\, Q_B \, Q_A) \, t_{BA} \, 
  \to P\, B\, Q_B\, t_B\, 
  \to P \, A\, Q_A\, (app \,B \,t_{BA} \, t_B) }\\[1ex]
  \mathsf{\lambda (P : \forall (A : Set) \to (A \to Set) \to LTerm\,A
    \to Set)} \\ 
  \quad \mathsf{\to 
  \forall (A \,B : Set) (Q_A : A \to Set)  (Q_B : B \to Set) 
    (e : Equal\, A\, (List\, B)) (ts : List\, (LTerm\, B))} \\ 
  \quad \mathsf{
    \to Equal^{\wedge}\, A\, (List\,B)\, Q_A\, (List^{\wedge}\, B\, Q_B)\, e 
  \to List^{\wedge}\, (LTerm\,B) (P\, B\, Q_B)\, ts
  \to P \, A\, Q_A\, (list \,B \,e \, ts) }
\end{array}\]
The deep induction rule for $\mathsf{LTerm}$ is thus
\begin{equation}\label{eq:ind-lam}
\begin{array}{l}
\mathsf{\forall (P : \forall (A : Set) \to (A \to Set) \to LTerm\,A
  \to Set) \to dIndVar\,P \to dIndAbs\,P \to dIndApp\,P \to
  dIndList\,P \to} \\ \quad \mathsf{\forall (A : Set)(Q_A : A \to
  Set)(t_A : LTerm\,A) \to LTerm^{\wedge}\,A\,Q_A\,t_A \to
  P\,A\,Q_A\,t_A}
\end{array}
\end{equation}

\begin{figure*}[t]

  \begin{adjustbox}{varwidth=6.8in, max width=6.6in, margin=-0.1in 0in
      -0.2in 0in, fbox, center} 
{\small
\[\begin{array}{lll}
\mathsf{dIndLTerm \, P\, cvar \, cabs\, capp\, clist \, A\, Q_A\,
  (var\;s\,T_A) \, liftA} & = & \mathsf{cvar \, A\, Q_A\, s\, T_A\,
  liftA}\\ 
\mathsf{dIndLTerm \, P\, cvar \, cabs\, capp\, clist \, A\, Q_A\,
  (abs \,B \,C \,e \,s \,T_B \, t_C) \, (Q_B , Q_C , liftE,
  lift_{T_B}, lift_{t_C})} & = & \mathsf{cabs\,A\,B\,C\, Q_A\,
  Q_B\, Q_C\, e\, s\, T_B\, t_C\, liftE\, lift_{T_B}\, p_C}\\
\mathsf{dIndLTerm \, P\, cvar \, cabs\, capp\, clist \, A\, Q_A\,
    (app \,B \,\,t_{BA} \, t_B)\, (Q_B , list_{t_{BA}}, list_{t_B})} &
= & \mathsf{capp\,A\,B\,Q_A\, Q_B\, t_{BA}\, t_B\, p_{BA} \, p_B}\\
  \mathsf{dIndLTerm \, P\, cvar \, cabs\, capp\, clistc \, A\, Q_A\,
    (list \,B \,e \, ts) \, (Q_B , liftE', lift_{List})} & = & 
  \mathsf{clistc \,A\,B\,Q_A\, Q_B\, e\, ts\, liftE'\, p_{List} }
\end{array}\]}

\vspace*{-0.1in}

\caption{$\mathsf{dIndLTerm}$}\label{fig:dindlterm} \vspace*{0.1in} 
\end{adjustbox}
\end{figure*}

To prove that this rule is sound we define a witness
$\mathsf{dIndLTerm}$ inhabiting the type in~\eqref{eq:ind-lam} as in
Figure~\ref{fig:dindlterm}. There, $\mathsf{s : String}$, $\mathsf{Q_A
  : A \to Set}$, $\mathsf{Q_B : B \to Set}$, $\mathsf{Q_C : C \to
  Set}$, $\mathsf{T_A : LType\,A}$, $\mathsf{T_B : LType\,B}$,
$\mathsf{t_B : LTerm\,B}$, $\mathsf{t_C : LTerm\,C}$, $\mathsf{t_{BA}
  : LTerm\,{\color{red}??}}$, $\mathsf{liftA : LTerm^{\wedge}\, A\,
  Q_A\, (var\;s\,T_A) = LType^{\wedge}\,A\,Q_A\,T_A}$, $\mathsf{liftE
  : Equal^{\wedge} \, A\, (B \to C)\, Q_A\, (Arr^{\wedge} \, B\, C\,
  Q_B \, Q_C) \, e}$, $\mathsf{lift_{T_B}: LType^{\wedge} \, B\, Q_B\,
  T_B}$,~$\mathsf{lift_{t_C}: LTerm^{\wedge} \, C\, Q_C\,
  T_C}$,~$\mathsf{lift_{t_{BA}}: LTerm^{\wedge} \, (B \to A)\,
  (Arr^{\wedge} \, B\, A\, Q_B \, Q_A)\,
  t_{BA}}$,~$\mathsf{lift_{t_B}: LTerm^{\wedge} \, B\, Q_B\,
  t_B}$,\\ $\mathsf{liftE' : Equal^{\wedge}\, A\, (List\,B)\, Q_A\,
  (List^{\wedge}\, B\, Q_B)\, e}$, and $\mathsf{lift_{List}:
  List^{\wedge} \, (LTerm\, B) \, (LTerm^{\wedge}\, B\, Q_B) \, ts}$.
Moreover,
\[\begin{array}{lll}
\mathsf{p_C} & = & \mathsf{dIndLTerm\,P\,cvar\,cabs \,capp \,clist\,
  C\, Q_C\, t_C\, lift_{t_C} : P \, C\, Q_C \, t_C }\\
\mathsf{p_B} & = & \mathsf{dIndLTerm\,P\,cvar\,cabs \,capp \,clist\,
  B\, Q_B\, t_B\, lift_{t_B} : P \, B\, Q_B \, t_B }\\
\mathsf{p_{BA}} & = & \mathsf{dIndLTerm\,P\,cvar\,cabs \,capp
  \,clist\, (B \to A)\,(Arr^{\wedge} \, B\, A\, Q_B \, Q_A) \,
  t_{BA}\, lift_{t_{BA}} : P \, (B \to A)\, (Arr^{\wedge} \, B\, A\,
  Q_B \, Q_A) \, t_{BA}}\\ 
\mathsf{p_{List}} & = &\mathsf{liftListMap \, (LTerm\, B) \,
  (LTerm^{\wedge} \, B \, Q_B)\, (P\,B\,Q_B)\, p_{ts} \, ts\,
  lift_{List} : List^{\wedge}\, (LTerm\,B) \, (P\,B\,Q_B) \, ts}\\
\mathsf{p_{ts}} & = & \mathsf{dIndLTerm\, P\, cvar\, cabs\, capp\,
  clist\, B\, Q_B : PredMap\,(LTerm\,B) \,(LTerm^{\wedge}\, B\, Q_B)
  \, (P\,B\,Q_B)}
\end{array}\]
where, in the final clause, $\mathsf{PredMap : \forall\, (A : Set) \to
  (A \to Set) \to (A \to Set) \to Set }$ is the type constructor
producing the type of morphisms between predicates defined by
$\mathsf{PredMap \,A\, Q\,Q'\, = \forall\, (a : A) \to Q\,a \to
  Q'\,a}$ and $\mathsf{liftListMap : \forall\, (A : Set) \to (Q \, Q'
  : A \to Set) \to PredMap\,A\,Q\,Q' \to PredMap\,(List\,A)
  \,(List^{\wedge}\, A\, Q)\, (List^{\wedge}\, A\, Q')}$, which takes
a morphism $\mathsf{f}$ of predicates and produces a morphism of
lifted predicates, is defined by $\mathsf{liftListMap\, A\, Q\, Q'\,
  m\, nil\, tt = tt}$ (since $\mathsf{x : List^{\wedge}\, A\, Q\, nil
  = \top}$ must necessarily be $\mathsf{tt}$), and by
$\mathsf{liftListMap\, A\, Q\, Q'\, m\, (cons\, a\, l')\, (y, x') =
  (m\,a\,y, liftListMap\, A\, Q\, Q'\, m\, l'\, x')}$ (since $\mathsf{x
  : List^{\wedge}\, A\, Q\, (cons\, a\, l') = \top}$ must be of the
form $\mathsf{x = (y, x')}$ where $\mathsf{y : Q\,a}$ and $\mathsf{x'
  : List^{\wedge}\, A\, Q\, l'}$). {\color{red} Double-check!!}

\section{The general framework}\label{sec:framework}

We can generalize the approach taken in
Section~\ref{sec:deep-ind-GADTs} to a general framework for deriving
deep induction rules for deep GADTs. We will treat deep GADTs of the
form 

\vspace*{-0.05in}

\begin{equation}\label{eq:gadts}
\begin{array}{l}
  \mathsf{data\ G : Set^\alpha
    \to Set\ where}\\
\mathsf{\;\;\;\;\;\;\;\;c\, :\, F\,G\,\ol{B} \to G (\ol{K\,\ol{B}})}
\end{array}
\end{equation}

\vspace*{-0.05in}

\noindent
For brevity and clarity we indicate only one constructor $\mathsf{c}$
in~\eqref{eq:gadts}, even though a GADT can, in general, have any
finite number of them, each with a type the same form as
$\mathsf{c}$'s. In~\eqref{eq:gadts}, $\mathsf{F}$ and $\mathsf{K}$ are
type constructors with signatures $\mathsf{(Set^{\alpha} \to Set) \to
  Set^{\beta} \to Set}$ and $\mathsf{Set^{\beta} \to Set}$,
respectively. If $\mathsf{T}$ has type signature $\mathsf{Set^{\gamma}
  \to Set}$ then we say that $\mathsf{T}$ is a {\em type constructor
  of arity $\mathsf{\gamma}$}.  The overline notation denotes a finite
list whose length is exactly the arity of the type constructor being
applied to it. The number of type constructors $\mathsf{K}$ must
therefore be $\alpha$. The type constructor $\mathsf{F}$ must be
constructed inductively according to the following grammar:

\vspace*{-0.05in}

\[\mathsf{F\,G\,\ol{B}\; :=\;
F_1\,G\,\ol{B} \times F_2\,G\,\ol{B} \ \vert\ F_1\,G\,\ol{B} +
F_2\,G\,\ol{B} \ \vert\ F_1\,\ol{B} \to F_2\,G\,\ol{B}
\ \vert\ G\,(\ol{F_1\,\ol{B}}) \ \vert\ H\,\ol{B} \ \vert\ H\,
(\ol{F_1\,G\,\ol{B}}) }\]
This grammar is subject to the following
restrictions. In the third clause the type constructor $\mathsf{F_1}$
does not contain $\mathsf{G}$. In the fourth clause, none of the
$\mathsf{\alpha}$-many type constructors in $\mathsf{\ol{F_1}}$
contains $\mathsf{G}$.  This prevents nesting, which would make it
impossible to give an induction rule; see
Section~\ref{sec:GADT-nested} below. In the fifth clause, $\mathsf{H}$
does not contain $\mathsf{G}$. This clause therefore subsumes the
cases in which $\mathsf{F\,G\,\ol{B}}$ is a closed type or one of the
$\mathsf{B_i}$. In the sixth clause, $\mathsf{H : Set^\gamma \to Set}$
does not contain $\mathsf{G}$ (although $\ol{\mathsf{F_1\,G\,\ol{B}}}$
can). Moreover, although $\mathsf{H}$ can construct any (truly)
nested type, it cannot construct a GADT. This ensures that
$\mathsf{H}$ admits functorial semantics~\cite{jp20}, and thus has an
associated $\mathsf{map}$ function. From this we can also construct a
$\mathsf{map}$ function {\color{red} Say what PredMap is.}

\vspace*{-0.05in}

\[\mathsf{H^\wedge Map : \forall (\ol{A : Set}) (\ol{Q\;Q' : A \to Set}) 
\to \ol{PredMap\,A\,Q\,Q'} \to
PredMap\,(H\,\ol{A})\,(H^{\wedge}\,\ol{A}\,\ol{Q})\, 
(H^{\wedge}\,\ol{A}\,\ol{Q'})}\]
for $\mathsf{H^{\wedge}}$.  A
concrete way to define $\mathsf{H^\wedge Map}$ is by induction on the
structure of the type $\mathsf{H}$, but we omit such details since
they are not essential to the present discussion. A further
requirement that applies to all of the types appearing above,
including the $\mathsf{K}$s in~\eqref{eq:gadts}, is that they must all
admit predicate liftings. This is not an overly restrictive condition,
though: all types constructed from sums, products, arrow types and
type application admit predicate liftings, and so do GADTs constructed
from the above grammar; in fact, we have seen such liftings for
products and type application in Section~\ref{sec:deep-ind-GADTs}.  A
concrete way to define predicate liftings more generally is, again, by
induction on the the structure of the types. We do not give a general
definition of predicate liftings here, though, since that would
require us to first design a full type calculus, which is beyond the
scope of the present paper.

We assume in the development below that $\mathsf{G}$ is a unary type
constructor, i.e., that $\alpha = 1$ in~\eqref{eq:gadts}. Extending
the argument to GADTs of arbitrary arity presents no difficulty other
than heavier notation. The type of the single data constructor
$\mathsf{c}$ for $\mathsf{G}$ can be rewritten as $\mathsf{c : \forall
  (\ol{B : Set}) \to Equal\,A\,(K\,\ol{B}) \to F\,G\,\ol{B} \to
  G\,A}$. The predicate lifting $\mathsf{G^{\wedge} : \forall (A :
  Set) \to (A \to Set) \to G\,A \to Set}$ for $\mathsf{G}$ is
therefore
\[
\mathsf{G^{\wedge}\,A\,Q_A\,(c\,\ol{B}\,e\,x)
= \exists [\ol{Q_B}]\,
Equal^{\wedge}\,A\,(K\,\ol{B})\,Q_A\,(K^{\wedge}\,\ol{B}\,\ol{Q_B})\,e
\times F^{\wedge}\,G\,\ol{B}\,G^{\wedge}\,{\ol{Q_B}}\,x}
\]
where $\mathsf{Q_A : A \to Set}$, $\ol{\mathsf{Q_B : B \to Set}}$,
$\mathsf{e : Equal\,A\,(K\,\ol{B})}$, and $\mathsf{x : F\,G\,\ol{B}}$.
Assuming predicate liftings 
\[\begin{array}{lll}
\mathsf{F^{\wedge}} & : & \mathsf{\forall (G : Set^{\alpha} \to Set) (\ol{B : Set})
\to (\forall (A : Set) \to (A \to Set) \to G\,A \to Set)
\to (\ol{B \to Set}) \to F\,G\,\ol{B} \to Set}\\ 
\mathsf{K^{\wedge}} & : & \mathsf{\forall (\ol{B : Set}) \to (\ol{B \to Set}) \to
  K\,\ol{B} \to Set}
\end{array}\]
for $\mathsf{F}$ and for $\mathsf{K}$, respectively, the induction
hypothesis for $\mathsf{c}$ is
\[\begin{array}{l}
\mathsf{dIndC = \lambda (P : \forall (A : Set) \to (A \to Set) \to
  G\,A \to Set)}\\
\quad \mathsf{\to \forall (A : Set) (\ol{B : Set}) (Q_A : A \to Set)
  (\ol{Q_B : B \to Set}) (e : Equal\,A\,(K\,\ol{B})) (x :
  F\,G\,\ol{B})} \\ 
\quad \mathsf{\to
  Equal^{\wedge}\,A\,(K\,\ol{B})\,Q_A\,(K^{\wedge}\,\ol{B}\,\ol{Q_B})\,e
  \to F^{\wedge}\,G\,\ol{B}\,P\,\ol{Q_B}\,x \to
  P\,A\,Q_A\,(c\,\ol{B}\,e\,x)} 
\end{array}\]
and the induction rule for $\mathsf{G}$ is
\[
\mathsf{\forall (P : \forall (A : Set) \to (A \to Set) \to G\,A \to
  Set) \to dIndC\,P \to \forall (A : Set)(Q_A : A \to Set)(y : G\,A)
\to G^{\wedge}\,A\,Q_A\,y \to P\,A\,Q_A\,y} \]

To prove that this rule is sound we define the witness
$\mathsf{dIndG}$ inhabiting this type by
\[\mathsf{dIndG\,P\,cc\,A\,Q_A\,(c\,\ol{B}\,e\,x)\,(\ol{Q_B}, liftE, liftF)
= cc\,A\,\ol{B}\,Q_A\,\ol{Q_B}\,e\,x\,liftE\,(p\,x\,liftF)}\] Here,
$\mathsf{cc : dIndC\,P}$, $\mathsf{e : Equal\,A\,(K\,\ol{B})}$,
$\mathsf{x : F\,G\,\ol{B}}$, $\ol{\mathsf{Q_B : B \to Set}}$,
$\mathsf{liftE : Equal^{\wedge}\,A\,(K\,\ol{B})\,Q_A\,
  (K^{\wedge}\,\ol{B}\,\ol{Q_B})\,e}$, and $\mathsf{liftF :
  F^{\wedge}\,G\,\ol{B}\,G^{\wedge}\,{\ol{Q_B}}\,x}$. As a result,
$\mathsf{(\ol{Q_B}, liftE, liftF) : G^{\wedge}\,A\,Q_A
  (c\,\ol{B}\,e\,x)}$ as expected. 
Finally, the morphism of predicates
$\mathsf{p :
  PredMap\,(F\,G\,\ol{B})\,(F^{\wedge}\,G\,\ol{B}\,G^{\wedge}\,\ol{Q_B})
  (F^{\wedge}\,G\,\ol{B}\,P\,\ol{Q_B})}$ 
is defined by structural induction on $\mathsf{F}$ as follows:
\begin{itemize}
\item 
If $\mathsf{F\,G\,\ol{B} = F_1\,G\,\ol{B} \times F_2\,G\,\ol{B}}$ then
$\mathsf{F^{\wedge}\,G\,\ol{B}\,P\,\ol{Q_B} =
  Pair^{\wedge}\,(F_1\,G\,\ol{B})\, (F_2\,G\,\ol{B})\,
  (F_1^{\wedge}\,G\,\ol{B}\,P\,\ol{Q_B})\,
  (F_2^{\wedge}\,G\,\ol{B}\,P\,\ol{Q_B})}$. The induction hypothesis
ensures morphisms of predicates $\mathsf{p_1 :
  PredMap\,(F_1\,G\,\ol{B})\,(F_1^{\wedge}\,G\,\ol{B}\,G^{\wedge}\,\ol{Q_BQ})
  (F_1^{\wedge}\,G\,\ol{B}\,P\,\ol{Q_B})}$ and $\mathsf{p_2 :
  PredMap\,(F_2\,G\,\ol{B})\,(F_2^{\wedge}\,G\,\ol{B}\,G^{\wedge}\,\ol{Q_B})
  (F_2^{\wedge}\,G\,\ol{B}\,P\,\ol{Q_B})}$.  For $\mathsf{x_1 :
  F_1\,G\,\ol{B}}$, $\mathsf{liftF_1 :
  F_1^{\wedge}\,G\,\ol{B}\,G^{\wedge}\,\ol{Q_B}\,x_1}$, $\mathsf{x_2 :
  F_2\,G\,\ol{B}}$ and $\mathsf{liftF_2 :
  F_2^{\wedge}\,G\,\ol{B}\,G^{\wedge}\,\ol{Q_B}\,x_2}$ we therefore
define $\mathsf{p\, (x_1, x_2)\, (liftF_1, liftF_2) =
  (p_1\,x_1\,liftF_1,\, p_2\,x_2\,liftF_2)}$.
\item The case $\mathsf{F\,G\,\ol{B} = F_1\,G\,\ol{B} +
  F_2\,G\,\ol{B}}$ is analogous.
\item If $\mathsf{F\,G\,\ol{B} = F_1\,\ol{B} \to F_2\,G\,\ol{B}}$ then
  $\mathsf{F^{\wedge}\,G\,\ol{B}\,P\,\ol{Q_B}\,x = \forall (z :
  F_1\,\ol{B}) \to F_1^{\wedge}\,\ol{B}\,\ol{Q_B}\,z \to
  F_2^{\wedge}\,G\,\ol{B}\,P\,\ol{Q_B}\,(x\,z)}$, where $\mathsf{x :
  F\,G\,\ol{B}}$. The induction hypothesis ensures a morphism of
  predicates $\mathsf{p_2 : PredMap\,(F_2\,G\,\ol{B})\,
    (F_2^{\wedge}\,G\,\ol{B}\,G^{\wedge}\,\ol{Q_B})\,
    (F_2^{\wedge}\,G\,\ol{B}\,P\,\ol{Q_B})}$.  We define
  $\mathsf{p\,x\,liftF : F^{\wedge}\,G\,\ol{B}\,P\,\ol{Q_B}\,x}$,
  where $\mathsf{liftF :
    F^{\wedge}\,G\,\ol{B}\,G^{\wedge}\,\ol{Q_B}\,x}$ is defined by
  $\mathsf{p\,x\,liftF\,z\,liftF_1 = p_2 (x\,z) (liftF\,z\,liftF_1)}$
  for $\mathsf{z : F_1\,\ol{B}}$ and $\mathsf{liftF_1 :
    F_1^{\wedge}\,\ol{B}\,\ol{Q_B}\,z}$. Note that $\mathsf{F_1}$ not
  containing the recursive variable is a necessary restriction, since
  the proof relies on
  $\mathsf{F^{\wedge}\,G\,\ol{B}\,G^{\wedge}\,\ol{Q_B}\,x}$ and
  $\mathsf{F^{\wedge}\,G\,\ol{B}\,P\,\ol{Q_B}\,x}$ having the same
  domain $\mathsf{F_1^{\wedge}\,\ol{B}\,\ol{Q_B}\,z}$.
\item If $\mathsf{F\,G\,\ol{B} = G\,(F_1\,\ol{B})}$ and $\mathsf{F_1}$
  does not contain $\mathsf{G}$, then
  $\mathsf{F^{\wedge}\,G\,\ol{B}\,P\,\ol{Q_B} = P (F_1\,\ol{B})
    (F_1^{\wedge}\,\ol{B}\,\ol{Q_B})}$. We therefore define $\mathsf{p
    = dIndG\,P\, cc\,(F_1\,\ol{B})\,(F_1^{\wedge}\,\ol{B}\,\ol{Q_B})}$.
\item If $\mathsf{F\,G\,\ol{B} = H\,\ol{B}}$ and $\mathsf{H}$ does not
  contain $\mathsf{G}$, then $\mathsf{p : PredMap\, (H\,\ol{B})\,
    (H^{\wedge}\,\ol{B}\,\ol{Q_B})\, (H^{\wedge}\,\ol{B}\,\ol{Q_B})}$
  is defined to be the identity morphism on predicates.
\item If $\mathsf{F\,G\,\ol{B} = H\, (\ol{F_k\,G\,\ol{B}})}$, if
  $\mathsf{H}$ does not contain $\mathsf{G}$, if $\mathsf{H}$ has a
  predicate lifting $\mathsf{H^{\wedge} : \forall (\ol{C : Set}) \to
    (\ol{C \to Set})}$ $\mathsf{\to H\,\ol{C} \to Set}$, and if
  $\mathsf{H^\wedge}$ has a $\mathsf{map}$ function $\mathsf{H^\wedge
    Map = \forall (\ol{C : Set}) (\ol{Q_C\;Q_C' : C \to Set}) \to
    PredMap\,\ol{C}\,\ol{Q_C}\,\ol{Q_C'} \to}$
  $\mathsf{PredMap\,(H\,\ol{C})\,(H^{\wedge}\,\ol{C}\,\ol{Q_C})\,
    (H^{\wedge}\,\ol{C}\,\ol{Q_C'})}$, then since $\mathsf{H}$ cannot
  be a GADT, $\mathsf{F^{\wedge}\,G\,\ol{B}\,P\,\ol{Q_B} =
    H^{\wedge}\, (\ol{F_k\,G\,\ol{B}})\,
    (\ol{F_k^{\wedge}\,G\,\ol{B}\,P\,\ol{Q_B}})}$.  The induction
  hypothesis ensures morphisms of predicates $\ol{\mathsf{p_k :
      PredMap\,(F_k\,G\,\ol{B})
      (F_k^{\wedge}\,G\,\ol{B}\,G^{\wedge}\,\ol{Q_B})
      (F_k^{\wedge}\,G\,\ol{B}\,P\,\ol{Q_B})}}$. We therefore define
  $\mathsf{p = H^\wedge Map\,(\ol{F_k\,G\,\ol{B}})\,
    (\ol{F_k^{\wedge}\,G\,\ol{B}\,G^{\wedge}\,\ol{Q_B}})\,
    (\ol{F_k^{\wedge}\,G\,\ol{B}\,P\,\ol{Q_B}})\,\ol{p_k}}$.
\end{itemize}

\section{Truly Nested GADTs Do Not Admit Induction
    Rules}\label{sec:GADT-nested}

{\color{red} HERE!!}


In Sections~\ref{sec:deep-ind-GADTs} and~\ref{sec:framework} we
derived induction rules for GADTs that are not truly nesteed. Since
both nested types and GADTs without true nesting admit induction
rules, we might expect that truly nested GADTs would as
well. Surprisingly, however, they do not. {\color{red} That is, our
  results from the previous section are the strongest possible.}
%
{\color{red} In fact, the induction rule for a data type generally
  relies on (unary) parametricity of its semantic interpretation, and
  in the case of nested types and GADTs it also relies on the data
  types having a functorial semantics. But whereas nested types can
  admit a functorial parametric semantics GADTs cannot admit both
  functorial and parametric semantics at the same
  time~\cite{jgj21}. In this section we show how induction for truly
  nested GADTs nesting goes wrong by analyzing the following simple
  concrete example of such a type.}
\begin{equation*}\label{gadt-nested}
\begin{array}{l}
\mathsf{data\ G : Set \to Set\ where}\\
\mathsf{\;\;\;\;\;\;\;\;\;c :\, G\,(G\,A) \to G\,(A \times A)}
\end{array}
\end{equation*}
The constructor $\mathsf{c}$ can be rewritten as $\mathsf{c :
  \forall\, (B : Set) \to Equal\,A\,(B \times B) \to G\,(G\,B) \to
  G\,A}$, so the predicate lifting $\mathsf{G^{\wedge} : \forall\, (A
  : Set) \to (A \to Set) \to G\,A \to Set}$ for $\mathsf{G}$ is
\[
\mathsf{G^{\wedge}\,A\,Q\,(c\,B\,e\,x)
= \exists\, [Q']\,
Equal^{\wedge}\,A\,(B \times B)\,Q\,(Pair^{\wedge}\,B\,B\,Q'\,Q')\,e
\times G^{\wedge}\,(G\,B)\,(G^{\wedge}\,B\,Q')\,x}
\]
where $\mathsf{Q : A \to Set}$, $\mathsf{Q' : B \to Set}$, $\mathsf{e
  : Equal\,A\,(B \times B)}$, and $\mathsf{x : G\,(G\,B)}$.
The induction hypothesis $\mathsf{dIndC}$ for $\mathsf{c}$ is
\[\begin{array}{l}
\mathsf{\lambda\, (P : \forall\, (A : Set) \to (A \to Set) \to G\,A
  \to Set)} \\ 
\quad\mathsf{\to \forall\, (A\;B : Set)\, (Q : A \to Set)\, (Q' : B
  \to Set)\, (e : Equal\,A\,(B \times B))\, (x : G\,(G\,B))} \\ 
\quad\mathsf{\to Equal^{\wedge}\,A\,(B \times
  B)\,Q\,(Pair^{\wedge}\,B\,B\,Q'\,Q')\,e \to P\,(G\,B)\,(P\,B\,Q')\,x 
	\to P\,A\,Q\,(c\,B\,e\,x)}
\end{array}\]
so the induction rule for $\mathsf{G}$ is
\[\mathsf{\forall\, (P : \forall\, (A : Set) \to (A \to Set) \to G\,A \to Set)
\to dIndC\,P \to \forall\, (A : Set)\, (Q : A \to Set)\, (y : G\,a)
\to G^{\wedge}\,A\,Q\,y \to P\,A\,Q\,y}\] But if we now try to show
that this induction rule is sound by constructing a witness
$\mathsf{dIndG}$ inhabiting this type we run into problems. We can try
to define $\mathsf{dIndG\,P\,cc\,A\,Q\,(c\,B\,e\,x)\,(Q', liftE,
  liftG) = cc\,A\,B\,Q\,Q'\,e\,x\,liftE\,p}$, where $\mathsf{cc :
  dIndC\,P}$, $\mathsf{Q : A \to Set}$, $\mathsf{e : Equal\,A\,(B
  \times B)}$, $\mathsf{x : G\,(G\,B)}$, $\mathsf{liftG :
  G^{\wedge}\,(G\,B)\,(G^{\wedge}\,B\,Q')\,x}$, and $\mathsf{Q' : B
  \to Set}$, $\mathsf{liftE : Equal^{\wedge}\,A\,(B \times
  B)\,Q\,(Pair^{\wedge}\,B\,B\,Q'\,Q')\,e}$.  However, we still need
to define $\mathsf{p : P\,(G\,B)\,(P\,B\,Q')\,x}$.  If we try to do so
by using the induction rule and letting $\mathsf{p =
  dIndG\,P\,cc\,(G\,B)\,(P\,B\,Q')\,x\,q}$, then we'd still need to
provide $\mathsf{q : G^{\wedge}\,(G\,B)\,(P\,B\,Q')\,x}$.  If we had
the $\mathsf{map}$ function $\mathsf{G^\wedge Map : \forall\, (A :
  Set)\, (Q\;Q'' : A \to Set) \to PredMap\,A\,Q\,Q'' \to
  PredMap\,(G\,A)\,(G^{\wedge}\,A\,Q)\,(G^{\wedge}\,A\,Q'')}$ for
$\mathsf{G^{\wedge}}$, then we would be able to define $\mathsf{q =
  GLMap\,(G\,B)\,(G^{\wedge}\,B\,Q')\,(P\,B\,Q')\,
  (dIndG\,P\,cc\,B\,Q')\,x\,liftG}$.
Unfortunately, however, we cannot define such a $\mathsf{G^\wedge Map}$.
Indeed, its definition would have to be
\[
\mathsf{G^\wedge Map\,A\,Q\,Q_2\,m\,(c\,B\,e\,x)\,(Q_1, liftE, liftG) =
  (Q_3, liftE', liftG')}
\]
for some $\mathsf{Q_3 : B \to Set}$, $\mathsf{liftE' :
  Equal^{\wedge}\,A\,(B \times
  B)\,Q_2\,(Pair^{\wedge}\,B\,B\,Q_3\,Q_3)\,e}$, and $\mathsf{liftG' :
  G^{\wedge}\,(G\,B)\,(G^{\wedge}\,B\,Q_3)\,x}$, where $\mathsf{Q : A
  \to Set}$, $\mathsf{Q_2 : A \to Set}$, $\mathsf{m :
  PredMap\,A\,Q\,Q_2}$, $\mathsf{e : Equal\,A\,(B \times B)}$,
$\mathsf{x : G\,(G\,B)}$, $\mathsf{Q_1 : B \to Set}$, $\mathsf{liftE :
  Equal^{\wedge}\,A\,(B \times
  B)\,Q\,(Pair^{\wedge}\,B\,B\,Q_1\,Q_1)\,e}$, and $\mathsf{liftG :
  G^{\wedge}\,(G\,B)\,(G^{\wedge}\,B\,Q_1)\,x}$. In other words, we
have a proof $\mathsf{liftE}$ of the (extensional) equality of the
predicates $\mathsf{Q}$ and $\mathsf{Pair^{\wedge}\,B\,B\,Q_1\,Q_1}$
and a morphism of predicates $\mathsf{m}$ from $\mathsf{Q}$ to
$\mathsf{Q_2}$, and we need to use those data to deduce a proof of the
(extensional) equality of the predicates $\mathsf{Q_2}$ and
$\mathsf{Pair^{\wedge}\,B\,B\,Q_3\,Q_3}$, for some predicate
$\mathsf{Q_3}$ on $\mathsf{B}$.  But that is not generally possible:
the facts that $\mathsf{Q}$ is equal to
$\mathsf{Pair^{\wedge}\,B\,B\,Q_1\,Q_1}$ and that there is a morphism
of predicates $\mathsf{m}$ from $\mathsf{Q}$ to $\mathsf{Q_2}$ do not
guarantee that there exists a predicate $\mathsf{Q_3}$ such that
$\mathsf{Q_2}$ is equal to $\mathsf{Pair^{\wedge}\,B\,B\,Q_3\,Q_3}$.

At a deeper level, the fundamental issue is that the $\mathsf{Equal}$
type does not have functorial semantics~\cite{jgj21}, so that having
morphisms $\mathsf{A \to A'}$ and $\mathsf{B \to B'}$ (for any type
$\mathsf{A, A', B}$ and $\mathsf{B'}$) and a proof that $\mathsf{A}$
is equal to $\mathsf{A'}$ does not provide a proof that $\mathsf{B}$
is equal to $\mathsf{B'}$. {\color{red} This is because GADTs can
  either have a syntax-only semantics or a functorial semantics.
  Since we are interested in induction rules, we consider the
  syntax-only semantics, which is parametric but not functorial.  Had
  we considered the functorial-completion semantics, which is
  functorial, we would have forfeited parametricity instead.  In both
  cases, thus, we cannot derive an induction rule for GADTs featuring
  nesting.  Unlike nested types, indeed, GADTs do not admit a semantic
  interpretation that is both parametric and functorial~\cite{jgj21}.}

\section{Applications}\label{sec:app}

In this section we use deep induction for the $\mathsf{LTerm}$ GADT
from~\eqref{eq:eq_lterm} to extract the type from a lambda term.
Consider the predicate
\begin{align*}
  &\mathsf{GetType : \forall \, (A : Set) \to LTerm\,A \to Set} \\
  &\mathsf{GetType \,A \,t = Maybe \, (LType \, A)}
\end{align*}
that takes a lambda term and produces the type of its possible types.
This predicate uses $\mathsf{Maybe}$ to represent
potential failure, where $\mathsf{Maybe}$ is the type defined as follows:
\begin{equation}\label{eq:maybe}
\begin{array}{l}
\mathsf{data\ Maybe : Set \to Set\ where}\\
\mathsf{\;\;\;\;\;\;nothing :\, Maybe\,A}\\
\mathsf{\;\;\;\;\;\;\;\;\;\;\;\;just :\, A \to Maybe\,A}
\end{array}
\end{equation}
We want to show this predicate is satisfied for every element of $\mathsf{LTerm\,A}$,
i.e., we want to prove: 
\[
  \mathsf{getTypeProof : \forall \, (A : Set)\, (t : LTerm\,A) \to GetType \,A \,t} \\
\]
Because of the $\mathsf{listC}$ constructor of $\mathsf{LTerm}$, this cannot be achieved without 
deep induction. In particular, deep induction is required to apply the induction to the 
individual terms in a list of terms. 
So, using the deep induction rule for $\mathsf{LTerm}$ from Section~\ref{sec:dind-lterm} {\color{red} (add back-reference)},
we define $\mathsf{getTypeProof}$ as
\[
  \mathsf{getTypeProof \,A \,t = 
    dIndLTerm\, P \,
    cvar\,  cabs\,  capp\,  clistc\,  A\, K_\top\, t\, (LTermLKT\, A\, t)
    }
\]
where $\mathsf{t : LTerm\,A}$,
$\mathsf{P}$ is the polymorphic predicate $\mathsf{\lambda \, (A: Set)\, (Q : A \to Set)\, (t : LTerm\,A)\, \to Maybe \, (LType \, A)}$
%where $\mathsf{K : a \to Set}$ is the constantly true predicate:
%\[
%  \mathsf{K_\top \, x = \top}
%\]
and $\mathsf{LTermLKT : \forall\, (A : Set)\, (t : LTerm A) \to LTerm^{\wedge}\, A\, K_\top\,t}$
is a function that we will define later.
%Notice that there is no space in $\mathsf{LTermLKT}$, because
%\[
%  \mathsf{LTermLKT : \forall (a : Set) (t : LTerm A) \to LTerm^{\wedge}\, a\, K_\top\,t}
%\]
%is a function that we will define.
In addition to defining $\mathsf{LTermLKT}$,
we also have to give a term for each constructor of $\mathsf{LTerm}$,
as already discussed in Section~\ref{sec:dind-lterm} {\color{red} (add back-reference)}:
\begin{itemize}
% var
\item a term $\mathsf{cvar : dIndVar\,P}$ associated to the constructor $\mathsf{var}$, i.e.,
\[
  \mathsf{cvar : \forall\, (A : Set)\, (Q : A \to Set)\, (s : String)\, (T : LType\, A) 
      \to LType^{\wedge} \, A\, Q\, T\, \to Maybe\, (LType\, A)}
\]
% abs 
\item a term $\mathsf{cabs : dIndAbs\,P}$ associated to the constructor $\mathsf{abs}$, i.e.,
\[
\begin{array}{l}
    \mathsf{cabs : \forall\, (A \, B\, C: Set)\, (Q : A \to Set)\, (Q' : B \to Set)\, (Q'' : C \to Set)} \\
    \quad\mathsf{(e : Equal\, A\, (B \to C))\, (s : String)\, (T : LType\, B)\, (t : LTerm \, C)} \\ 
    \quad\mathsf{ 
      \to Equal^{\wedge}\, A\, (B \to C)\, Q\, (Arr^{\wedge}\, B\, C\, Q'\, Q'')\, e
      \to LType^{\wedge} \, B\, Q'\, T
      \to Maybe\, (LType\,C) \to Maybe\, (LType\, A)}
\end{array}
\]
% app
\item a term $\mathsf{capp : dIndApp\,P}$ associated to the constructor $\mathsf{app}$, i.e.,
\[
\begin{array}{l}
    \mathsf{capp : \forall\, (A \, B : Set)\, (Q : A \to Set)\, (Q' : B \to Set)\, 
      (t : LTerm\, (B \to A))\, (t' : LTerm\, B)} \\ 
    \quad\mathsf{
      \to Maybe\, (LType\, (B \to A))
      \to Maybe\, (LType\, B)
      \to Maybe\, (LType\, A)
    }
\end{array}
\]
% list 
\item a term $\mathsf{clistc : dIndListC\,P}$ associated to the constructor $\mathsf{listC}$, i.e.,
\[
\begin{array}{l}
    \mathsf{clistc : \forall\, (A \, B : Set)\, (Q : A \to Set)\, (Q' : B \to Set)\,
      (e : Equal\, A\, (List\, B))\, (ts : List\, (LTerm\, B))} \\ 
    \quad\mathsf{
      \to Equal^{\wedge}\, A\, (List\,B) \, Q\, (List^{\wedge}\, B\, Q')\, e
      \to List^{\wedge}\, (LTerm\, B)\, (GetType\, B)\, ts
      \to Maybe\, (LType\, A)
    }
\end{array}
\]
\end{itemize}
For variables we simply return the type $\mathsf{T : LType\, A}$ (wrapped in $\mathsf{just}$), 
and the cases for abstraction and application are similar.
The interesting case is $\mathsf{clistc}$, in which we have to use 
the results of $\mathsf{List^{\wedge}\, (LTerm\, B)\, (GetType\, B)\, ts}$
in order to extract the type of one of the terms in the list. 
To define $\mathsf{clistc}$ we pattern-match on the list of terms $\mathsf{ts}$.
If $\mathsf{ts}$ is the empty list $\mathsf{nil}$, we cannot extract a type, 
so we return $\mathsf{nothing}$:
\[
  \mathsf{clistc\, A\, B\, Q\, Q'\, e\, nil \, liftE\, lift_{ts} = nothing}
\]
where $\mathsf{liftE : Equal^{\wedge}\, A\, (List\,B) \, Q\, (List^{\wedge}\, B\, Q')\, e}$
and $\mathsf{lift_{ts} : List^{\wedge}\, (LTerm\, B)\, (GetType\, B)\, ts}$.
If $\mathsf{ts}$ is a non-empty list $\mathsf{cons\,t\,ts'}$,
we pattern match on $\mathsf{list_{ts}}$ and use the result to construct the type we need.
The type of $\mathsf{list_{ts}}$ is
\begin{align*}
\mathsf{List^{\wedge}\, (LTerm\, B)\, (GetType\, B)\, (cons\, t\, ts')}
&= \mathsf{GetType\, B\, t \times List^{\wedge}\, (LTerm\, B)\, (GetType\, B)\, ts'} \\
&= \mathsf{Maybe\, (LType\, B) \times List^{\wedge}\, (LTerm\, B)\, (GetType\, B)\, ts'}
\end{align*}
Thus, we define
\begin{align*}
  &\mathsf{clistc\, A\, B\, Q\, Q'\, e\, (cons\, t\, ts') \, liftE\, (nothing , lift_{ts'}) = nothing} \\
  &\mathsf{clistc\, A\, B\, Q\, Q'\, e\, (cons\, t\, ts') \, liftE\, (just\, T' , lift_{ts'}) = just \, (TList\, B\, e\, T')}
\end{align*}
where $\mathsf{e : Equal\, A\, (List\,B)}$, $\mathsf{T' : LType\, B}$
and $\mathsf{lift_{ts'} : List^{\wedge}\, (LTerm\, B)\, (GetType\, B)\, ts'}$. 



\subsection{Defining $\mathsf{LTermLKT}$}

{\color{red} Maybe this section can be deleted by assuming $\mathsf{LTerm^{\wedge}\, A\, K_\top\,t = K_\top}$.
Maybe we can say this derives from parametricity.
Currently we say that $\mathsf{K_\top \times K_\top = K_\top}$ based on the fact that products are a built-in type and so this seems to be obviously true.}

The last piece of infrastructure we need in order to define $\mathsf{getTypeProof}$ is a function 
\[
  \mathsf{LTermLKT : \forall\, (A : Set)\, (t : LTerm\, A) \to LTerm^{\wedge}\, A\, K_\top\,t}
\]
that provides a 
proof of $\mathsf{LTerm^{\wedge}\,A\,K_\top\,t}$ for any term $\mathsf{t : LTerm\, A}$. 
Because $\mathsf{LTerm^{\wedge}}$ is defined in terms of $\mathsf{LType^{\wedge}}$, $\mathsf{Arr^{\wedge}}$, 
and $\mathsf{List^{\wedge}}$, we need an analogous function for each of these liftings as well
(respectively, $\mathsf{LTypeLKT}$, $\mathsf{ArrLKT}$ and $\mathsf{ListLKT}$). 
We only give the definition of $\mathsf{LTermLKT}$, but the definitions for 
$\mathsf{LTypeLKT}$, $\mathsf{ArrLKT}$ and $\mathsf{ListLKT}$ are analogous. 
We define $\mathsf{LTermLKT}$ by pattern matching on the lambda term $\mathsf{t}$:
\begin{itemize}
\item For the $\mathsf{var}$ case, let $\mathsf{t = var\, s\, T}$ for $\mathsf{s : String}$ and $\mathsf{T : LType\,A}$,
and define
\[
  \mathsf{LTermLKT\,A\,(var\,s\,T) = LTypeLKT\,A\,T}
\]
\item For the $\mathsf{abs}$ case, let $\mathsf{t = abs \,B \,C \, e \,s \,T \, t'} $
for $\mathsf{e : Equal\,A\,(B \to C)}$, $\mathsf{s : String}$, $\mathsf{T : LType\,B}$ and $\mathsf{t' : LTerm\,C}$,
and recall the definition of $\mathsf{LTerm^{\wedge}}$ for the $\mathsf{abs}$ constructor, 
instantiating the predicate $\mathsf{Q : A \to Set}$ to $\mathsf{K_\top}$: 
% \,a\,Q_a\, (Abs \,b \,c \,e \,s \,T_b \,t_c)}$, :
\[
\begin{array}{l}
\mathsf{LTerm^{\wedge}\,A\,K_\top\,
  (abs \,B \,C \,e \,s \,T \,t')} \\ 
\quad\mathsf{
  = \sum\, [Q' : B \to Set]\, [Q'' : C \to Set]\,
      Equal^{\wedge} \, A\, (B \to C)\, K_\top\, (Arr^{\wedge} \, B\, C\, Q' \, Q'')\, e
      \times \, LType^{\wedge}\, B\, Q'\, T
      \times \, LTerm^{\wedge}\, C\, Q''\, t' }
\end{array}
\]
So, to define the $\mathsf{abs}$ case of $\mathsf{LTermLKT}$, we need a proof of 
\[
  \mathsf{Equal^{\wedge} \, A\, (B \to C)\, K_\top\, (Arr^{\wedge} \, B\, C\, Q' \, Q'')\, e}
\]
i.e., that $\mathsf{K_\top}$ is (extensionally) equal to the lifting $\mathsf{Arr^{\wedge} \, B\, C\, Q' \, Q''}$
for some predicates $\mathsf{Q' : B \to Set}$ and $\mathsf{Q'' : C \to Set}$.
The only reasonable choice for $\mathsf{Q'}$ and $\mathsf{Q''}$ 
is to let both be $\mathsf{K_\top}$, which means we need a proof of
\[
  \mathsf{Equal^{\wedge} \, A\, (B \to C)\, K_\top\, (Arr^{\wedge} \, B\, C\, K_\top \, K_\top)\, e}
\]
Since we are working with proof-relevant predicates
(i.e., functions into $\mathsf{Set}$ rather than functions into $\mathsf{Bool}$), 
the lifting $\mathsf{Arr^{\wedge} \, B\, C\, K_\top \, K_\top}$ of $\mathsf{K_\top}$ to arrow types
is not identical to $\mathsf{K_\top}$ on arrow types, but the predicates are (extensionally) isomorphic. 
We discuss this issue in more detail at the end of the section. 
For now, we assume a proof 
\[
  \mathsf{EqualLArrKT : Equal^{\wedge} \, A\, (B \to C)\, K_\top\, (Arr^{\wedge} \, B\, C\, K_\top \, K_\top)\, e}
\]
and define the $\mathsf{abs}$ case of $\mathsf{LTermLKT}$ as 
\[
  \mathsf{LTermLKT\,A\, (abs\, B \,C \, e \,s \,T \, t') = 
    (K_\top , K_\top , EqualLArrKT , LTypeLKT\, B\, T , LTermLKT\, C\, t')
  }
\]
\item For the $\mathsf{app}$ case, let $\mathsf{t = app\, B\, t' \, t''}$
for $\mathsf{t' : LTerm\,(B \to A)}$ and $\mathsf{t'' : LTerm\,B}$,
and, just as we did for the $\mathsf{abs}$ case, 
recall the definition of $\mathsf{LTerm^{\wedge}\, A\, K_\top\, (app\, B\, t'\, t'')}$
with all of the predicates instantiated with $\mathsf{K_\top}$: 
\[
\mathsf{
      LTerm^{\wedge}\, (B \to A)\, (Arr^{\wedge} \, B\, A\, K_\top \, K_\top)\, t'
      \times LTerm^{\wedge}\, B\, K_\top\, t''}
\]
The second component can be given using $\mathsf{LTermLKT}$, and we can 
define the first component using a proof of
\[
\mathsf{LTerm^{\wedge}\, (B \to A)\, K_\top\, t'}
\]
and a map-like function 
\[
  \mathsf{LTermLEqualMap : \forall\, (A : Set)\, (Q\,Q' : A \to Set) 
    \to Equal^{\wedge}\,A\,A\,Q\,Q'\,refl
    \to PredMap\,(LTerm\,A)\, (LTerm^{\wedge}\,A\,Q)\,(LTerm^{\wedge}\,A\,Q')}
\]
that takes two (extensionally) equal predicates with the same carrier 
and produces a morphism of predicates between their liftings. The definition is 
straightforward enough, so we omit the details. 
Using $\mathsf{LTermLEqualMap}$, we can define the $\mathsf{app}$ case of $\mathsf{LTermLKT}$ as 
\[
  \mathsf{LTermLKT\,A\, (app\, B\, t'\, t'') = 
    (K_\top , LTermLArr , LTermLKT\,B\,t'')}
\]
where $\mathsf{ LTermLArr : LTerm^{\wedge}\, (B \to A)\, (Arr^{\wedge} \, B\, A\, K_\top \, K_\top)\, t'}$ is defined as
\[
  \mathsf{LTermLArr =  LTermLEqualMap\,K_\top\,(Arr^{\wedge}\,B\,A\,K_\top\,K_\top)\,\, EqualLArrKT\, t'\, L_{K_\top}}
\]
where $\mathsf{L_{K_\top} = LTermLKT\, (B \to A)\, t' : LTerm^{\wedge}\, (B \to A) \, K_\top\, t'}$.
\item For the $\mathsf{listC}$ case, let $\mathsf{t = listC\, B\, e\, ts}$
for $\mathsf{e : Equal\,A\,(List\,B)}$ and $\mathsf{ts : List\,(LTerm\, B)}$,
and recall the definition of $\mathsf{LTerm^{\wedge}\,A\,(listC\, B\, e\, ts)}$
with all of the predicates instantiated to $\mathsf{K_\top}$: 
\[
\mathsf{ Equal^{\wedge} \, A\, (List\,B)\, K_\top\, (List^{\wedge} \, B\, K_\top) \, e
      \times List^{\wedge}\, (LTerm\,B) \, (LTerm^{\wedge} \, B\, K_\top) \, ts }
\]
We can give the first component by assuming a proof \,
$\mathsf{EqualLListKT : Equal^{\wedge} \, A\, (List\,B)\, K_\top\, (List^{\wedge} \, B\, K_\top) \, e}$, 
but for the second component we again have multiple liftings nested together.
In this case, we can get a proof of
\[
\mathsf{ List^{\wedge}\, (LTerm\,B) \, (LTerm^{\wedge} \, B\, K_\top) \, ts }
\]
using $\mathsf{liftListMap}$, as seen in Section~\ref{sec:dind-lterm} {(\color{red} add back-reference)},
to map a morphism of predicates 
\[
  \mathsf{PredMap\, (LTerm\,B)\, (K_\top)\, (LTerm^{\wedge}\,B\,K_\top)}
\]
to a morphism of lifted predicates
\[
  \mathsf{PredMap\,(List\,(LTerm\,B)) \,(List^{\wedge}\,(LTerm\,B)\,K_\top) \, (List^{\wedge}\,(LTerm\,B)\, (LTerm^{\wedge}\,B\,K_\top))}
\]
We then define the $\mathsf{listC}$ case of $\mathsf{LTermLKT}$ as 
\[
  \mathsf{LTermLKT\,A\, (listC\, B\, e\, ts) =
    (K_\top , \, EqualLListKT , \,  L_{ListLLTermLKT} ) }
\]
where
$\mathsf{L_{ListLLTermLKT} : List^{\wedge}\, (LTerm\,B) \, (LTerm^{\wedge} \, B\, K_\top) \, ts}$
is defined as
\[
\mathsf{ 
  L_{ListLLTermLKT} =
  liftListMap \, (LTerm\,B) \, K_\top \, (LTerm^{\wedge}\, B\, K_\top) \, m_{K_\top} \, ts \, (ListLKT\, (LTerm\, B)\, ts)
}
\]
and $\mathsf{m_{K_\top} : PredMap\, (LTerm\,B) \, (K_\top)\, (LTerm^{\wedge}\,B\,K_\top)}$
is defined as
\[
  \mathsf{m_{K_\top} \, t'\, tt\, = LTermLKT\, B\, t'}
\]
where $\mathsf{t' : LTerm\,B}$ and $\mathsf{tt}$ is the single element of $\mathsf{K_\top\, t'}$. 
The use of $\mathsf{liftListMap}$ is required in the $\mathsf{listC}$ case because 
$\mathsf{listC}$ takes an argument of type $\mathsf{List\, (LTerm\, B)}$.
\end{itemize}

{\color{red} These are general considerations, and are not required to conclude the above argument.}
The above techniques can be used
to define a function $\mathsf{GLKT : \forall\, (A : Set)\, (x : G\,A) \to G^{\wedge}\, A\, K_\top\, x}$
for an arbitrary GADT $\mathsf{G}$,
as defined in Section~\ref{sec:dind-generic} {\color{red} (add back-reference)}.
To provide a proof of $\mathsf{G^{\wedge}\, A\, K_\top \, x}$ for every term $\mathsf{x : G\, A}$, 
we need to know that the lifting of $\mathsf{K_\top}$ by any type constructor $\mathsf{F}$
is extensionally equal to $\mathsf{K_\top}$ on $\mathsf{F}$. 
For example, we might need a proof that $\mathsf{Pair^{\wedge}\,A\,B\,K_\top\,K_\top}$
is equal to the predicate $\mathsf{K_\top}$ on $\mathsf{A \times B}$. 
Given a pair $\mathsf{(a , b) : A \times B}$, we have
$\mathsf{Pair^{\wedge}\,A\,B\,K_\top\,K_\top (a, b) = K_\top \, a \times K_\top\, b = \top \times \top}$,
whereas $\mathsf{K_\top\, (a, b) = \top}$.
While these types are not equal, they are clearly isomorphic.
So, for simplicity of presentation, we assume $\mathsf{F^{\wedge}\,A\,K_\top}$
is equal to $\mathsf{K_\top}$ for every nested type and ADT $\mathsf{F}$. 
Moreover, remember that, whenever $\mathsf{G}$ has a constructor of the form $\mathsf{c : F \, (G\, A) \to G\, (K\,B)}$, 
$\mathsf{F}$ is only allowed to be a nested type or an ADT,
and we are guaranteed to have a $\mathsf{liftFMap}$ function. 



\section{Conclusion}\label{sec:conclusion}


\section{TODO}

\begin{itemize}
\item find correct entcsmacro file (current one is for 2018). Maybe ask Ana Sokolova (anas@cs.uni-salzburg.at).
\item reference (correctly) Haskell Symposium paper
\item reference inspiration for STLC GADT : https://www.seas.upenn.edu/~cis194/spring15/lectures/11-stlc.html
\item Data type vs. data structure
\item Weird to code in Agda if we're talking about induction rules for
  Coq? 
\item Agda style conventions
\item spacing in data type declarations
\item Mention Agda flags that need to be toggled to handle true nesting
\end{itemize}

\begin{thebibliography}{10}\label{bibliography}

\bibitem{atk12} Atkey, R.  {\em Relational parametricity for higher
  kinds}.  Computer Science Logic, pp.~46-61, 2012.

\bibitem{bfss90} Bainbridge, E. S., Freyd, P. Scedrov, A., and Scott,
  P. J. {\em Functorial polymorphism}. Theoretical Computer Science
  70(1), pp. 35-64, 1990.

\bibitem{bm98} Bird, R. and Meertens, L. {\em Nested
  datatypes}. Proceedings, Mathematics of Program Construction,
  pp. 52–67, 1998.

\bibitem{ch03} Cheney, J. and Hinze, R. {\em First-class phantom
 types}. CUCIS TR2003-1901, Cornell University, 2003.

\bibitem{ch88} Coquand, T. and Huet, G. {\em The calculus of
  constructions}. Information and Computation 76(2/3), 1988.

\bibitem{chl} Chlipala, A. {\em Library Inductive
  Types}. $\mathtt{http://adam.chlipala.net/cpdt/html/InductiveTypes.html}$

\bibitem{coq20} The Coq Development Team. {\em The Coq Proof
  Assistant}, version 8.11.0, January 2020.
  $\mathtt{https://doi.org/10.5281/zenodo.3744225}$

\bibitem{fs18} Fu, P. and Selinger, P.  Dependently typed folds for
  nested data types, 2018. $\mathtt{https://arxiv.org/abs/1806.05230}$ 
  
\bibitem{gjfor15} Ghani, N., Johann, P., Nordvall Forsberg, F.,
  Orsanigo, F., and Revell, T. {\em Bifibrational functorial semantics
    for parametric polymorphism}. Proceedings, Mathematical
  Foundations of Program Semantics, pp. 165-181, 2015.

\bibitem{hin03} Hinze, R. {\em Fun with phantom types}. Proceedings,
 The Fun of Programming, pp. 245–262, 2003.

\bibitem{web-page} Johann, P., Ghiorzi, E., and Jeffries,
  D. Accompanying Agda code for this paper.
    $\mathtt{https://cs.appstate.edu/~johannp/FoSSaCS21Code.html}$
 
\bibitem{jgj21f} Johann, P., Ghiorzi, E., and Jeffries, D. {\em
  Parametricity for primitive nested types}. Proceedings, Foundations
  of Software Science and Computation Structures, pp. 324-343, 2021.

\bibitem{jgj21} Johann, P., Ghiorzi, E., and Jeffries, D. {\em
  Parametricity in the presence of GADTs}. Submitted, 2021.

\bibitem{jp19} Johann, P. and Polonsky, A. {\em Higher-kinded data
  types: Syntax and semantics} Proceedings, Logic in Computer Science
  2019. {\color{red} PAGES?}

\bibitem{jp20} Johann, P. and Polonsky, A. {\em Deep induction:
  Induction rules for (truly) nested types}.  Proceedings, Foundations
  of Software Science and Computation Structures, pp. 339-358, 2020.

\bibitem{mac71} MacLane, S. {\em Catgories for the Working
  Mathematician}. Springer, 1971.

\bibitem{mcb99} McBride, C. {\em Dependently Typed Programs and their
  Proofs}. PhD thesis, University of Edinburgh, 1999.

\bibitem{min15} Minsky, Y.  {\em Why {GADT}s matter for performance}.
  $\mathtt{https://blog.janestreet.com/why-gadts-matter-for-performance/}$,
  2015.

\bibitem{pl04} Pasalic, E., and Linger, N.  {\em Meta-programming with
  typed object-language representations}.  Generic Programming and
  Component Engineering, pp.~136-167, 2004.

\bibitem{pen20} Penner, C.  {\em Simpler and safer {API} design using
  {GADT}s}.  $\mathtt{https://chrispenner.ca/posts/gadt-design}$,
  2020.

\bibitem{pvww06} Peyton Jones, S., Vytiniotis, D., Weirich, S., and
  Washburn, G. {\em Simple unification-based type inference for
    GADTs}. Proceedings, International Conference on Functional
  Programming, 2006. {\color{red} PAGES?}

\bibitem{pr06} Pottier, F., and R{\'e}gis-Gianas, Y.  {\em Stratified
  type inference for generalized algebraic data types}.  Principles of
  Programming Languages, pp.~232-244, 2006.

\bibitem{sjsv09} Schrijvers, T, Peyton Jones, S. L., Sulzmann, M., and
  Vytiniotis, D. {\em Complete and decidable type inference for
    GADTs}. Proceedings, International Conference on Functional
  Programming, pp. 341– 352, 2009.

\bibitem{sp04} Sheard, T., and Pasalic, E. {\em Meta-programming with
  built-in type equality}. Proceedings, Workshop on Logical Frameworks
  and Meta-languaegs, 2004. {\color{red} PAGES?}

\bibitem{tas19} Tassi, E.: Deriving proved equality tests in Coq-elpi:
  Stronger induction principles for containers in Coq. Proceedings,
  Interactive Theorem Proving, pp. 1-18, 2019.
  
\bibitem{vw10} Vytiniotis, D., and Weirich, S.  {\em Parametricity,
  type equality, and higher-order polymorphism}.  Journal of
  Functional Programming 20(2), pp.~175--210, 2010.

\bibitem{xcc03} Xi, H., Chen, C. and Chen, G. {\em Guarded recursive
  datatype constructors}. Proceedings, Principles of Programming
  Languages, pp. 224–235, 2003.

\bibitem{ull20} Ullrich, M. {\em Generating Induction Principles for
Nested Induction Types in MetaCoq}. PhD thesis, Saarland University,
  2020.  
  
\end{thebibliography}

\end{document}



